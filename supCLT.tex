%%%----Versions-----------------------------
% supclt1209.tex 2018/12/07 by Zhenyao
% supclt1207.tex 2018/12/07 by Renming
% supCLT1204.tex 2018/12/04 by Zhenyao
% supCLT1130.tex 2018/11/30 by Jianjie
% supCLT1124.tex 2018/11/24 by Zhenyao
% supCLT1115.tex 2018/11/15 by Jianjie
% supCLT9.tex ......... by Zhenyao
% supCLT8.tex ......... by Jianjie
% supCLT7.tex ......... by Zhenyao
% supCLT6.tex ......... by Jianjie
% supCLT5.tex ......... by Zhenyao
% supCLT4.tex ......... by Jianjie
% supCLT3.tex 2018/8/25 by Zhenyao
% supCLT2.tex 2018/8/13 by Jianjie
% supCLT1.tex 2018/7/24 by Zhenyao
% supCLT.tex 2018/6/29 by Jianjie
%---The preamble----------------------
\documentclass[12pt]{amsart}
\usepackage[utf8]{inputenc}
\usepackage{geometry}
\geometry{tmargin=1in,bmargin=1in,lmargin=1in,rmargin=1in}
\usepackage{mathtools}
\mathtoolsset{showonlyrefs}
\usepackage{amsthm}
\usepackage{stackrel}
\usepackage[backref=section]{hyperref}
\def\MR#1{\href{http://www.ams.org/mathscinet-getitem?mr=#1}{MR-#1}}
\def\ARXIV#1{\href{https://arxiv.org/abs/#1}{arXiv:#1}}
\usepackage{mathrsfs}
\usepackage{color}
\usepackage{comment}
\theoremstyle{plain}
\newtheorem{thm}{Theorem}[section]
\newtheorem{lem}[thm]{Lemma}
\newtheorem{prop}[thm]{Proposition}
\newtheorem{cor}[thm]{Corollary}
\newtheorem{conj}[thm]{Conjecture}
\theoremstyle{definition}
\newtheorem{defi}[thm]{Definition}
\newtheorem{rem}[thm]{Remark}
\newtheorem{exa}[thm]{Example}
\newtheorem{asp}{Assumption}
\numberwithin{equation}{section}
\allowdisplaybreaks
\newcommand{\added}[1]{{\color{blue}#1}}\newcommand{\deleted}[1]{{\color{red}#1}}
%\newcommand{\added}[1]{#1}\newcommand{\deleted}[1]{}

%---Top matter------------------------------
\begin{document}
\title
    [CLT for Super-OU processes]
    {CLT for Super-OU Processes with stable Branching}
\author
    [Y.-X. Ren, R. Song, Z. Sun and J. Zhao]
    {Yan-Xia Ren, Renming Song, Zhenyao Sun and Jianjie Zhao}
\address
    {Yan-Xia Ren\\
    School of Mathematical Sciences\\
    Peking University\\
    Beijing, P. R. China, 100871}
\email{yxren@math.pku.edu.cn}
\thanks{The research of Yan-Xia Ren is supported in part by NSFC (Grant Nos. 11671017  and 11731009).}
\address
    {Zhenyao Sun\\
    School of Mathematical Sciences\\
    Peking University\\
    Beijing, P. R. China, 100871}
\email{zhenyao.sun@pku.edu.cn}
\address
    {Jianjie Zhao\\
    School of Mathematical Sciences\\
    Peking University\\
    Beijing, P. R. China, 100871}
\email{zhaojianjie@pku.edu.cn}
%\begin{abstract}
%\end{abstract}
%\subjclass[2010]{60J80, 60F05}
%\keywords{}p
%\date{}
%\maketitle
%---Contents-----------------------
\begin{abstract}
    Consider a super-OU process $\{X_t, t\geq 0\}$ with branching mechanism $\psi(z)=-\alpha z +z^{1+\beta}$, where $\alpha >0$ and $\beta\in (0,1)$. For each testing function $f$ of at most polynomial growth,
    denote by $\kappa_f$ the order of $f$ in term of the spectral decomposition of $f$ in terms of the spectrum of the the mean semigroup of $X$. 
    Conditioned on no-extinction, we establish some spatial central limit theorems for $\langle f, X_t \rangle$ in three different regimes: In the small branching rate case, $\alpha\beta< \kappa_f b(1+\beta)$, $\langle f,X_t\rangle/\|X_t\|^{\frac{1}{+\beta}}$ converges weakly.
    In the critical branching rate case, $\alpha\beta= \kappa_f b(1+\beta)$,  $\langle f,X_t\rangle/(t\|X_t\|)^{\frac{1}{+\beta}}$ converges weakly.
    In the large branching rate case,  $\alpha\beta> \kappa_f b(1+\beta)$, $e^{-(\alpha-\kappa(f)b)t}\langle f,X_t\rangle$ converges almost surely and in $L^{1+\gamma}$, for any $\gamma \in (0,\beta)$.
\end{abstract}
\maketitle
\section{Introduction}
\subsection{Model}
    Let $d \in \mathbb N:= \{1,2,\dots\}$ and
    $\xi=\{(\xi_t)_{t\geq 0}; (\Pi_x)_{x\in \mathbb R^d}\}$ be an $\mathbb R^d$-valued Ornstein-Uhlenbeck prcess (OU process) with infinitesimal generator 
\begin{equation}
\label{eq: OU generator}
    Lf(x)
        = \frac{1}{2}\sigma^2\Delta f(x)-b x \cdot \nabla f(x),
        \quad  x\in \mathbb R^d,
        f \in C^2(\mathbb{R}^d),
\end{equation}
    where $\sigma>0$ and $b>0$ are constants.
    Denote by $(T_t)_{t\geq 0}$ the transition semigroup of $\xi$.
    Let $\psi$ be the function on $\mathbb R_+:= [0,\infty)$ defined by
\begin{equation}\label{mechanism}
    \psi(z)
    := - \alpha z + z^{1+\beta},
    \quad z \in \mathbb R_+,
\end{equation}
	where $\alpha > 0$ and $\beta \in (0,1) $ are constants.
    Denote by $\mathcal{M}(\mathbb{R}^d)$ the space of all finite Borel measures on $\mathbb{R}^d$.
    For all Borel function $f$ and Borel measure $\mu$, write $\langle f,\mu\rangle = \int f(x)\mu(dx)$ whenever the integral makes sense.
    Put $\|\mu\|=\langle 1,\mu\rangle$.
    We say an $\mathcal{M}(\mathbb{R}^d)$-valued Markov process $X = \{(X_t)_{t\geq 0}; (\mathbb{P}_{\mu})_{\mu \in \mathcal M(\mathbb R^d)}\}$ is a super Ornstein-Uhlenbeck process (super-OU process) with branching mechanism $\psi$, if for each non-negative bounded Borel function $f$ on $\mathbb{R}^d$, we have
\begin{equation} \label{super}
    \mathbb{P}_{\mu}[e^{-\langle f,X_t \rangle}]
    = e^{-\langle V_tf, \mu \rangle},
    \quad t\geq 0, \mu \in \mathcal M(\mathbb R^d),
\end{equation}
	where $(t,x) \mapsto V_tf(x)$ is the unique locally bounded positive solution to the equation
\begin{equation}\label{eq1}
	V_tf(x) + \Pi_x \Big[ \int_0^t\psi\big(\xi_s,V_{t-s}f(\xi_s)\big)~ds\Big]
	= \Pi_x [f(\xi_t)],
    \quad x\in \mathbb R^d, t\geq 0.
\end{equation}	

	The existence of superprocesses is well known, see \cite{Dynkin1993Superprocesses} for instance.
    According to \cite[Theorem 12.5 \& Theorem 12.7]{Kyprianou2014Fluctuations}, we have
\begin{equation}
    \mathbb{P}_{\mu} (\exists t\geq 0,~\text{s.t.}~\|X_t\|=0)
    = e^{-\alpha^* \|\mu\|},
\end{equation}
    where $\alpha^* := \sup\{\lambda \geq 0: \psi(\lambda) = 0\}$.
    Define 
\begin{equation}\label{meansemigroup}
    T^{\alpha}_t f(x)
    :=
    e^{\alpha t} T_t f(x) =
    \Pi_x [e^{\alpha t}f(\xi_t)],
    \quad x\in \mathbb{R}^d,t\geq 0, f\in \mathcal B(\mathbb R^d, \mathbb R_+).
\end{equation}
    It is konwn that, see \cite[Proposition 2.27]{Li2011Measure-valued} for example, $(T^\alpha_t)_{t\geq 0}$ is the \emph{mean semigroup} of $X$, in the sense that
\begin{equation}\label{eq:meanformula}
    \mathbb{P}_{\mu}[\langle f, X_t \rangle]
    = \langle T_t^\alpha f, \mu \rangle,
    \quad t\geq 0, f\in \mathcal B(\mathbb R^d, \mathbb R_+), \mu \in \mathcal M(\mathbb R^d).
\end{equation}

\subsection{OU semigroup}
    In this subsection, we recall some results on the spectral properties of the OU operator $L$ from \cite{GD}.
    Let $\xi$ be an OU process with generator $L$ given by \eqref{eq: OU generator}.
    It is known that $\xi$ has an invariant measure
\begin{equation}
\label{invariantdensity}
    \varphi(x)dx
    =\Big (\frac{b}{\pi \sigma^2}\Big )^{d/2}\exp \Big(-\frac{b}{\sigma^2}|x|^2 \Big)dx,
    \quad x\in \mathbb{R}^d.
\end{equation}
    Let $L^2(\varphi):= \left\{ h  \in \mathcal B(\mathbb R^d, \mathbb R): \int_{\mathbb{R}^d} |h(x)|^2 \varphi(x) dx < \infty \right\}$. $L^2(\varphi)$ is a Hilbert space with inner product
\begin{equation}
    \langle f_1, f_2 \rangle_{\varphi}
    := \int_{\mathbb{R}^d}f_1(x)f_2(x)\varphi(x) dx, \quad f_1,f_2 \in L^2(\varphi).
\end{equation}
     Let $\mathbb N = \{1,2,\dots\}$ and $\mathbb Z_+ := \mathbb N\cup\{0\}$.
    For each $p = (p_k)_{k = 1}^d \in \mathbb{N}_0^{d}$, let $|p| \added{ :}=\sum_{k=1}^d p_k$, $p!:= \prod_{k= 1}^d p_k !$ and $\partial x^p:= \prod_{k = 1}^d\partial x_k^{p_k}$.
    The Hermite polynomials are defined by
\begin{equation}
    H_p(x)
    :=(-1)^{|p|}\exp(|x|^2) \frac{\partial ^{|p|}}{\partial x^p} \exp(-|x|^2) ,
    \quad x\in \mathbb R^d, p \in \mathbb{N}_0^{d}.
\end{equation}
    It is known that $(T_t)$ is a strongly continuous semigroup on $L^2(\varphi)$ and $L$ has
    discrete spectrum $\sigma(L)= \{-bk: k \in \mathbb Z_+\}$.
    For each $k \in \mathbb Z_+$, denote by $A_k$ the eigenspace corresponding to the eigenvalue $-bk$, then
\begin{equation}
    A_k
    = \operatorname{Span} \{\phi_p : p\in \mathbb Z_+^d, |p|=k\},
\end{equation}
where
\begin{equation}\label{eigenfunction}
    \phi_p(x)
    := \frac{1}{\sqrt{ p! 2^{|p|} }} H_p \Big(\frac{ \sqrt{b} }{\sigma}x \Big),
    \quad x\in \mathbb{R}^d, p\in \mathbb Z_+^d.
\end{equation}
    It is known that
\begin{equation}\label{semigroupformula}
    T_t\phi_p(x)
    =e^{-b|p|t}\phi_p(x),
    \quad t\geq 0, x\in \mathbb{R}^d, p\in \mathbb Z_+^d.
\end{equation}
    Moreover, $\{\phi_p: p \in \mathbb Z_+^d\}$ forms a complete orthogonal basis for $L^2(\varphi)$.
    Thus, for each $f\in L^2(\varphi)$,
\begin{equation}\label{semicomp1}
    f
    =\sum_{k=0}^{\infty}\sum_{|p|=k}\langle f, \phi_p \rangle_{\varphi} \phi_p,
    \quad \text{in~} L^2(\varphi).
\end{equation}
    Denote by
\begin{equation}\label{order}
    \kappa_f
    :=\inf \left \{k\geq 0: \exists ~ p\in \mathbb Z_+^d ,{\rm ~s.t.~}|p|=k {\rm ~and~}  \langle f, \phi_p \rangle_{\varphi}\neq 0\right \}
\end{equation}
    the order of the function $f$ in $L^2(\varphi)$.
    Note that $ \kappa_f\geq 0$.
    In particular, the order of any constant function is zero.
    Denote by $\mathcal P$ the class of functions of polynomial growth on $\mathbb R^d$, i.e.,
\begin{equation}
    \mathcal{P}
    :=\big \{f\in \mathscr B(\mathbb{R}^d):\exists~ C>0, n \in \mathbb Z_+, {\rm ~s.t.~} \forall x\in \mathbb{R}^d,~ |f(x)|\leq C(1+|x|)^n\big\}.
\end{equation}
    It is clear that $\mathcal{P} \subset L^2(\varphi)$.
    For all $\kappa \in \mathbb Z_+$ and $f\in \mathcal P$, define
$
	Q_\kappa f := \sup_{t\geq 0} e^{\kappa b t}|T_t f|,
$
and
$
	Q f:= Q_{\kappa_f}f.
$
    Then according to \cite[Fact 1.2]{MM}, $Q$ is an operator from $\mathcal P$ to $\mathcal P$.
\subsection{}
    Let $X$ be the super OU-process discussed in the first subsection.
    Its branching mechanism $\psi$ can be extended to $\mathbb C_+:=\{x+iy: x\in [0, \infty), y\in \mathbb R\}$:
 $$\psi(z) := -\alpha z + z^{1+\beta}, \quad z\in \mathbb C_+.$$ 
    According to the analysis in Subsection A.1, 
    $\psi$ is continuous on $\mathbb C_+$ and is holomorphic on $\mathbb C_+^0:=
    \{x+iy: x\in (0, \infty), y\in \mathbb R\}$ with derivative
\begin{equation}
\label{eq: deriavetive of the Poission part}
    \psi'(z) := -\alpha + (1+\beta)z^{\beta},
    \quad z\in \mathbb C_+.
\end{equation}
    Write $\psi_0(z) := \psi(z) + \alpha z$. Then$\psi'_0(z) = \psi'(z) + \alpha=(1+\beta)z^{\beta}$.
    For all $f\in \mathcal B(\mathbb R^d, \mathbb C_+)$ and $x\in \mathbb R^d$, define $\Psi_0f(x) := \psi_0(f(x))$ and $\Psi'_0 f(x) := \psi'_0(f(x))$.
    For all $t\in [0,\infty), x\in \mathbb R^d $ and $f \in \mathcal{P}$, define $U_tf(x) := \operatorname{Log} \mathbb P_{\delta_x}[e^{i\theta \langle f, X_t\rangle}]|_{\theta = 1}$, the value of the L\'evy exponent of the infinite divisible random variable $\langle f, X_t\rangle$ at $1$.
    From \eqref{eq: -v has positive real part}, we know that $-U_tf(x)$ takes values in $\mathbb C_+$. Furthermore, we know from Proposition \ref{prop: complex FKPP-equation} that
\begin{align}
\label{eq:chareq2}
    U_tf(x)-\int_0^t T^{\alpha}_{t-s} \Psi_0(-U_sf)(x)ds
    =i T^{\alpha}_t f(x),
    \quad t\in [0,\infty), x\in E, f\in \mathcal P.
\end{align}

\subsection{Main Result}

In this subsection, we will give the main results of this paper. Suppose $\mu\in \mathcal M_c(\mathbb R_d)$ has compact support.
\subsubsection{Large Branching Rate}

For each $p\in \mathbb{N}_+^d$, we define
$$H_t^p:= e^{-(\alpha-|p|b)t}\langle\phi_p,X_t\rangle,\quad t\geq 0.$$
  For any $\gamma\in (0, \beta)$, we will prove (see Lemma \ref{lemma26}) that, if $\alpha\beta>|p|b(1+\beta)$, $H_t^p$ is a $\mathbb{P}_{\mu}$-martingale bounded in $L^{1+\gamma}(\mathbb{P}_{\mu})$, thus the limit $H^p_{\infty}:=\lim_{t\rightarrow \infty}H_t^p$ exists $\mathbb{P}_{\mu}$-almost surely and in $L^{1+\gamma}(\mathbb{P}_{\mu})$.
 \begin{thm}\label{Theorem11}
     If $f \in \mathcal{P}$ satisfies $\alpha\beta>\kappa(f)b(1+\beta)$, then for any $\gamma\in (0, \beta)$,
     as $t\rightarrow \infty$,
     $$e^{-(\alpha-\kappa_fb)t}\langle f, X_t\rangle \rightarrow\sum_{|p|=\kappa_f}\langle f, \phi_p\rangle_{\varphi} H_{\infty}^p \quad in~ L^{1+\gamma}(\mathbb{P}_{\mu}).$$

     Moreover, if $f$ is twice differentiable and all its second order partial derivatives are in $\mathcal{P}$, then we also have almost sure convergence. 
 \end{thm}
For any $t>0$, let $H_t:=e^{-\alpha t}\|X_t\|$, then $H_t$ is equal to $H_t^0$ and is a non-negative martingale with limit $H_{\infty}:=\lim_{t\rightarrow\infty}H_t$,  $\mathbb{P}_{\mu}$-a.s. and in $L^{1+\gamma}(\mathbb{P}_{\mu})$.
 \begin{rem}
    If $\kappa_f=0$, $\langle f, \phi_{\kappa_f}\rangle_{\varphi}$ reduces to $\langle f,\varphi\rangle$. Hence by Theorem \ref{Theorem11},  for any $\gamma\in (0, \beta)$, as $t\rightarrow \infty$
     $$e^{-\alpha t}\langle f, X_t\rangle \rightarrow \langle f, \varphi\rangle H_{\infty} \quad in~ L^{1+\gamma}(\mathbb{P}_{\mu})$$
    Moreover, if $f$ is twice differentiable and all its second order partial derivatives are in $\mathcal{P}$, then we also have almost sure convergence.
 \end{rem}

\subsubsection{Critical Branching Rate}
    For $f\in \mathcal{P}$, define
\[
    \tilde{m}[f]
    := \langle(-i\phi)^{1+\beta},\varphi\rangle
\]
    where
\[
    \phi(x)
    =\sum_{|p|=\kappa(f)}\langle f,\phi_p\rangle\phi_p(x).
\]
    Denote by $D$ the extinction event of the super OU-process.
\begin{thm}
\label{Theorem12}
    If $f\in\mathcal{P}$ satisfies  $\alpha\beta=\kappa_fb(1+\beta)$, then, under $\tilde{\mathbb P}_\mu := \mathbb{P}_{\mu}(\cdot|D^c)$, it holds that
\[
    \frac{\langle f,X_t\rangle}{\left(t\|X_t\|\right)^{\frac{1}{1+\beta}}}
    \xrightarrow{d} \eta_1, 
    \quad t\rightarrow \infty,
\]
    where $\eta_1$ is a $(1+\beta)$-stable random variable with
\[
    \mathbb{E} [e^{i\theta \eta_1}]
    =\exp(\tilde{m}[\theta f]), 
    \quad \theta\in \mathbb R.
\]
\end{thm}

\subsubsection{Small Branching Rate}

For $f\in \mathcal{P}$, define
\[
    m[f]
    :=\int_{\mathbb{R}^d}\int_0^{\infty} e^{-\alpha s}(-iT_{s}^{\alpha}f)^{1+\beta}(x)~ds~\varphi(x)~dx
\]
\begin{thm}
\label{Theorem13}
    If $f\in\mathcal{P}$ satisfies  $\alpha\beta<\kappa_f b(1+\beta)$, then, under $\mathbb{P}_{\mu}(\cdot|D^c)$, it holds that
    $$\frac{\langle f,X_t\rangle}{\|X_t\|^{\frac{1}{1+\beta}}}\xrightarrow{d} \eta_2, \quad t\rightarrow \infty,$$
    where $\eta_2$ is a $(1+\beta)$-stable random variable with 
    $$\mathbb{E} [e^{i\theta \eta_2}]=\exp(m[\theta f]), \quad \theta\in \mathbb R.$$
\end{thm}

\section{Preliminaries}

    Define $\mathcal P^+:= \mathcal P \cap \mathcal B(\mathbb R^d, \mathbb R_+)$ and $\mathcal P^*:= \{f\in \mathcal B(\mathbb R^d, \mathbb C): |f|\in \mathcal P\}$.
    We say $S$ is a $\gamma$-scalable operator for some $\gamma\in \mathbb R$ if $S: \mathcal P^+ \to \mathcal P^+$ and $S(\theta f) = \theta^\gamma Sf$ for all $\theta \geq 0$ and $f \in \mathcal P^+$.
    We say $R$ is a monotone operator if $R:\mathcal P^+ \to \mathcal P^+$ and $Rf \leq Rg$ for all $f, g \in \mathcal P^+$ with $f\leq g$.
    We say $(R,S)$ is a $\gamma$-control-pair for some $\gamma \in \mathbb R$ if $R$ is a monotone operator and $S$ is a $\gamma$-scalable operator and $Rf\leq Sf$ for all $f\in \mathcal P^+$.
    We say an operator $A$ is $\gamma$-controllable on $\mathcal D \subset \mathcal P^*$ for some $\gamma \in \mathbb R$ if $A: \mathcal D \to \mathcal P^*$ and there  is a $\gamma$-control pair $(R,S)$ such that $|Af|\leq R|f| (\le S|f|)$ for all $f\in \mathcal D$.
    In this case we say $A$ is $\gamma$-controlled by the $\gamma$-control-pair $(R,S)$ on $\mathcal D$.
    We say a family of operator $(A_s)_{s\in \Lambda}$ is uniformly $\gamma$-controllable on $\mathcal D\subset \mathcal P^*$ for some $\gamma \in \mathbb R$ if there is a $\gamma$-control pair $(R,S)$ such that, for each $s\in \Lambda$, $A_s$ is $\gamma$-controlled by $(R, S)$ on $\mathcal D$.
 
	For two operators $A: \mathcal D_A \subset \mathcal P^*\to \mathcal P^*$ and $B: \mathcal D_B \subset \mathcal P^*\to \mathcal P^*$, define $(A\times B)f (x):= Af(x) \times Bf(x)$ for all $f\in \mathcal D_A \cap \mathcal D_B$ and $x\in \mathbb R^d$.
    Let $a > 0$, define $A^{\times a}f(x):= (Af(x))^a$ for all $f\in \mathcal D_A$ and $x\in \mathbb R^d$.
    One of the reason for considering $\gamma$-controllable operators is that they have good algebraic properties:
\begin{lem}
\label{lem: property of controllable operators}
    Let $\mathcal D \subset \mathcal P^*$, $\Lambda$ be an index set, and $(A_\lambda)_{\lambda\in \Lambda}$ be a family of operators from $\mathcal D$ to $ \mathcal P^*$. Assume that $(A_\lambda)_{\lambda\in \Lambda}$ is uniformly $\gamma$-controllable on $\mathcal D$.
\begin{itemize}
\item[(1)]
    Suppose that $(\Lambda, \mathscr F)$ is a measurable space
    and that $(\lambda,x)\mapsto A_\lambda f(x)$ is $\mathscr F \otimes \mathscr B(\mathbb R^d)$-measurable for each $f\in \mathcal D$.
    For any probability measure $\mu$ on $(\Lambda, \mathscr F)$, write
\[
    A_\mu f(x):= \int_{\Lambda} A_\lambda f (x)~\mu(d\lambda), \quad f\in \mathcal D, x\in \mathbb R^d.
\]
    Then  $\{A_\mu: \mu \text{ is  a probability measure on } (\Lambda, \mathscr F)\}$ is uniformly $\gamma$-controllable on $\mathcal D$.
\item[(2)]
    Suppose that $\Delta$ is another index set and $(B_\delta)_{\delta\in \Delta}$ is a family of operators from $\mathcal D_0\subset \mathcal P^*$ to $ \mathcal P^*$.
    Assume that $(B_\delta)_{\delta\in \Delta}$ is uniformly $\beta$-controllable on $\mathcal D_0$ for some $\beta \in \mathbb R$ and that, for each $\lambda$, $A_\lambda:\mathcal D \to \mathcal D_0$.
    Then  $(B_\delta A_\lambda)_{\delta\in \Delta, \lambda \in \Lambda}$ is uniformly $(\gamma\beta)$-controllable on $\mathcal D$.
\item[(3)]
    Suppose that $\Delta$ is another index set and $(B_\delta)_{\delta\in \Delta}$ is a family of operators from $\mathcal D$ to $ \mathcal P^*$.
    Assume that $(B_\delta)_{\delta\in \Delta}$ is uniformly $\beta$-controllable on $\mathcal D$ for some $\beta \in \mathbb R$.
    Then  $(B_\delta\times A_\lambda)_{\delta \in \Delta, \lambda \in \Lambda}$ is uniformly $(\gamma+\beta)$-controllable.
\item[(4)]
    Let $a>0$. Suppose that, for each $\lambda \in \Lambda$, $A_\lambda : \mathcal D \to \mathcal P^+$.
    Then $(A^{\times a}_\lambda)_{\lambda \in \Lambda}$ is uniformly $(a\gamma)$-controllable.
\end{itemize}
\end{lem}
\begin{proof}
    (1). Let $(R,S)$ be the $\gamma$-control-pair of $(A_\lambda)_{\lambda\in\Lambda}$ on $\mathcal{D}$. For all $f \in \mathcal{D}$, $x\in \mathbb R^d$ and  probability measure $\mu$ on $(\Lambda, \mathscr F)$,
\[
   |A_{\mu}f(x)|\leq \int_{\Lambda}|A_{\lambda}f(x)|\mu(d\lambda) \leq \int_{\Lambda}R|f|(x)\mu(d\lambda) \leq R|f|(x).
\]

   	(2). Let $(R_A, S_A)$ be the $\gamma$-control-pair of $(A_\lambda)_{\lambda\in\Lambda}$ on $\mathcal{D}$ and $(R_B, S_B)$ be the $\beta$-control-pair of $(B_{\delta})_{\delta\in\Delta}$ on $\mathcal{D}_0$.
	Note that $(R_BR_A, S_BS_A)$ is a $\beta \gamma$-control-pair.
	In fact:
\begin{itemize}
\item
	For each $f,g \in \mathcal P^+$ with $f\leq g$, since $R_Af \leq R_A g$ we have $R_B(R_A f)\leq R_B(R_A g)$.
\item
	For each $f\in \mathcal{P}^+$ and $\theta \geq 0$, we have $S_BS_A(\theta f)=S_B(\theta^{\gamma}S_Af)=\theta^{\beta\gamma}S_BS_Af$.
\item
	For each $f,g \in \mathcal P^+$ with $f\leq g$, we have $R_B R_A f \leq R_B S_A f \leq S_BS_A f$.
\end{itemize}
	Finally, note that for each $\delta\in \Delta, \lambda\in\Lambda$ and $f\in \mathcal D$, $|B_{\delta}A_{\lambda}f|\leq R_B|A_{\lambda}f|\leq R_BR_A|f|$ which says that $(B_\delta\times A_\lambda)_{\delta \in \Delta, \lambda \in \Lambda}$ is uniformly $(\beta\gamma)$-controlled by $(R_BR_A,S_BS_A)$.

   The proofs of (3) and (4) are similar to that of (2). 
\end{proof}
\subsection{}
    For all $f \in \mathcal{P}$, $x\in \mathbb{R}^d$ and $t\geq 0$, define
\begin{align}
\label{eq: def of Zf}
    \tilde U_t f(x)
    &:= i T^\alpha_t f(x) + \int_0^t T^\alpha_{t-s} \Psi_0(-i T_s^{\alpha}f)(x)ds,
    \\Z_t f (x)
    &:= \int_0^t T^\alpha_{t-s} \Psi_0(-i T_s^{\alpha}f)(x)ds.
\end{align}

\begin{lem}
\label{lem: upper bound for usgx}
The following statements are true for our super-OU process:
\begin{itemize}
\item[(1)]
   $(-U_t)_{0\leq t\leq 1}$ is uniformly $1$-controllable from $\mathcal P$ to $\mathcal P^*\cap \mathcal B(\mathbb R^d, \mathbb C_+)$.
\item[(2)]
    $(T^\alpha_t)_{0\leq t\leq 1}$ is uniformly $1$-controllable on $\mathcal P$.
\item[(3)]
    $\Psi_0$ is $(1+\beta)$-controllable on $\mathcal P^* \cap \mathcal B(\mathbb R^d, \mathbb C_+)$.
\item[(4)]
    $(U_t- iT_t^{\alpha})_{0\leq t\leq 1}$ is uniformly $(1+\beta)$-controllable on $\mathcal P$.
\item[(5)]
    $\{\Psi_0(-U_t) - \Psi_0(-iT_t^\alpha): 0\leq t\leq 1\}$ is uniformly $(1+2\beta)$-controllable on $\mathcal P$.
\item[(6)]
    $(U_t-\tilde U_t)_{0\leq t\leq 1}$ is uniformly $(1+2\beta)$-controllable on $\mathcal P$.
\item[(7)]
    $(Z_t)_{0\leq t\leq 1}$ is uniformly $(1+\beta)$-controallable on $\mathcal P$.
\end{itemize}
\end{lem}

\begin{proof}
    (1). It follows from \eqref{eq: upper bound for vf}, that for all $g\in \mathcal P$, $0\leq t\leq 1$ and $x\in \mathbb R^d$,
\[
    |U_t g(x)|
    \leq \sup_{0\leq u\leq 1}T_u^\alpha |g| (x).
\]
    Now note that $f\mapsto\sup_{0\leq u\leq 1}T^{\alpha}_u|f|$ is monotone and $1$-scalable.

    (2). Similar to the proof of (1).

    (3). According to Lemma \ref{lem: Lip of power function}, for each $f\in \mathcal P^* \cap \mathcal B(\mathbb R^d, \mathbb C_+)$,
\[
    |\Psi_0 f(x)| = |f(x)^{1+\beta}| = |f(x)|^{1+\beta}.
\]
    Now note that $f\mapsto |f|^{1+\beta}$ is monotone and $(1+\beta)$-scalable.

    (4). From (1), (2), (3) and Lemma \ref{lem: property of controllable operators}.(2), we know that the operators
\[
    f
    \mapsto T^{\alpha}_{t-s}\Psi_0(-U_sf)(x),
    \quad 0\leq s\leq t\leq 1
\]
    is uniformly $(1+\beta)$-controllable.
    Combining this with \eqref{eq:chareq2} and Lemma \ref{lem: property of controllable operators}.(1), we get the desired result.

    (5).  Notice that from Lemma \ref{lem: Lip of power function},
\[
    |\Psi_0(-U_t f) - \Psi_0(-iT_t^\alpha f) |
    \leq  (1+\beta) |U_u f-iT_u^{\alpha}f|(|U_u f|^{\beta}+|i T_u^{\alpha}f|^{\beta}).
\]
    Now using (1), (2), (4) and Lemma \ref{lem: property of controllable operators}.(3)-(4), we get the desired result.

    (6). Note that
\[
    U_sf - \tilde U_sf
    = \int_0^s T_{s-u}^{\alpha}\big(\Psi_0(-U_t f)-\Psi_0(-i T_t^{\alpha}f)\big)~du.
\]
    Now using (2), (5) and Lemma \ref{lem: property of controllable operators}.(1)--(2), we get the desired result.

    (7). The proof is similar to that of (4).
\end{proof}

\subsection{}

 In this subsection,  we want to bound the $(1+\gamma)$-th moment of $\langle g ,X_t \rangle$ for $\gamma \in (0,\beta)$. Denote by $\mathcal{M}_c(\mathbb{R}^d)$ by the space of all finite Borel measures on $\mathbb{R}^d$ with compact support.
	For all $0 \leq a_1 \leq a_2 <\infty$ and  random variable $Y$ with finite mean, define
$
   	\mathcal I_{a_1}^{a_2} Y
    := \mathbb P[Y|\mathscr F_{a_2}] - \mathbb P[Y|\mathscr F_{a_1}].
$
\begin{lem}
\label{lem: control pair for P(M>lambda)}
    There is a $(1+\beta)$-control-pair $(R,S)$ such that for all $0\leq t\leq 1$, $g\in \mathcal P$, $\lambda >0$ and $\mu\in \mathcal M_c(\mathbb R^d)$, we have
\[
    \mathbb P_\mu ( |\mathcal{I}_0^t\langle g,X_t\rangle| > \lambda)
    \leq \frac{\lambda}{2}\int_{-2/\lambda}^{2/\lambda}\langle R|\theta g|,\mu\rangle d\theta.
\]
\end{lem}

\begin{proof}
    Denote by $(R,S)$ the $(1+\beta)$-control pair for Lemma \ref{lem: upper bound for usgx}.(4).
    Using Lemma \ref{lem: estimate of exponential remaining} and the argument in the proof \cite[Theorem 3.3.6]{Durrett2010Probability}, we get
\begin{align}
    &\big|\mathbb P_\mu (|\mathcal{I}_0^t\langle g,X_t\rangle| > \lambda)\big|
    \leq \Big|\frac{\lambda}{2}\int_{-2/\lambda}^{2/\lambda}(1 - \mathbb P_\mu[e^{i\theta \mathcal{I}_0^t\langle g,X_t\rangle]})d\theta\Big|
    \\&\leq \frac{\lambda}{2}\int_{-2/\lambda}^{2/\lambda}|1-e^{\langle U_t(\theta g)-iT_t^{\alpha}(\theta g),\mu \rangle}|d\theta
    \leq \frac{\lambda}{2}\int_{-2/\lambda}^{2/\lambda}\langle |U_t(\theta g) - iT_t^\alpha(\theta g)|,\mu\rangle d\theta
    \\&\leq \frac{\lambda}{2}\int_{-2/\lambda}^{2/\lambda}\langle R|\theta g|,\mu\rangle d\theta.
      \qedhere
\end{align}
\end{proof}

\begin{lem}\label{lem: temp}
	For all $h \in \mathcal P^+$ and $\mu \in \mathcal M_c(\mathbb{R}^d)$, there exist a constant $C > 0$ such that for all $\kappa \in \mathbb Z_+ $ and $0\leq r\leq s\leq t$ with $s-r \leq 1$, we have
\[
    \sup_{g \in \mathcal P: Q_\kappa g\leq h}\mathbb P_{\mu}(|\mathcal I_r^s\langle g, X_t\rangle|>\lambda)
    \leq \frac{ C}{\lambda^{1+\beta}} e^{(1+\beta)(t-s)(\alpha- \kappa b)+ \alpha r}.
\]
\end{lem}
\begin{proof}
    Denote by $(R,S)$ the $(1+\beta)$-control-pair in Lemma \ref{lem: control pair for P(M>lambda)}.
    Let $h \in \mathcal P^+$, $\mu \in \mathcal M_c(\mathbb{R}^d)$, $\kappa \in \mathbb Z_+ $ and $0\leq r\leq s\leq t$ with $s-r \leq 1$.
    Suppose that $g\in \mathcal P$ satisfies $Q_\kappa g \leq h$.
    Using the Markov property of $(X_t)_{t\geq 0}$, we get
\begin{align}
    &\mathbb P_{\mu}(|\mathcal I_r^s\langle g, X_t\rangle|>\lambda)
    = \mathbb P_\mu \big[\mathbb P_\mu[|\langle T_{t-s}^\alpha g, X_{s}\rangle - \langle T_{t-r}^\alpha g, X_{r}\rangle|> \lambda\big| \mathscr F_r]\big]
    \\&= \mathbb P_\mu \big[\mathbb P_{X_r}(|\langle T_{t-s}^\alpha g, X_{s-r}\rangle - \langle T_{t-r}^\alpha g, X_{0}\rangle|> \lambda)\big]
    = \mathbb P_\mu \big[\mathbb P_{X_r}(|\mathcal I_0^{s-r}\langle T_{t-s}^\alpha g, X_{s-r}\rangle |> \lambda)\big]
    \\&\leq \mathbb P_\mu \Big[ \frac{\lambda}{2}\int_{-2/\lambda}^{2/\lambda}\langle R|\theta T^\alpha_{t-s}g|,X_r\rangle d\theta \Big]
    \leq \mathbb P_\mu \Big[ \frac{\lambda}{2}\int_{-2/\lambda}^{2/\lambda}\langle R|\theta e^{(t-s)(\alpha- \kappa b)}h|,X_r\rangle d\theta \Big]
	\\&\leq \mathbb P_\mu \Big[ \frac{\lambda}{2}\int_{-2/\lambda}^{2/\lambda}\langle S|\theta e^{(t-s)(\alpha- \kappa b)}h|,X_r\rangle d\theta \Big]
    = e^{(1+\beta)(t-s)(\alpha- \kappa b)} \mathbb P_\mu [ \langle Sh,X_r\rangle ] \frac{\lambda}{2}\int_{-2/\lambda}^{2/\lambda}|\theta|^{1+\beta}d\theta
    \\& = e^{(1+\beta)(t-s)(\alpha- \kappa b)} \langle T_r^\alpha Sh,\mu\rangle  \frac{2^{2+\beta}}{2+\beta}\frac{1}{\lambda^{1+\beta}}
    \\&\leq e^{(1+\beta)(t-s)(\alpha- \kappa b)+ \alpha r} \langle Q_0 Sh,\mu\rangle  \frac{2^{2+\beta}}{2+\beta}\frac{1}{\lambda^{1+\beta}}.
\end{align}
    Therefore the desired result is true with
\[
    C := \frac{2^{2+\beta}}{2+\beta}\langle Q_0Sh, \mu\rangle.
    \qedhere
\]
\end{proof}


\begin{lem}
\label{lem: control of mgtrs}
	For all $h \in \mathcal P$, $\mu \in \mathcal M_c(\mathbb{R}^d)$ and $\gamma\in (0, \beta)$, there exist a constant $C > 0$, such that for each $\kappa \in \mathbb Z_+$ and $0\leq r\leq s\leq t$ with $s-r \leq 1$, we have
\[
    \sup_{g \in \mathcal P: Q_\kappa g\leq h} \mathbb P_\mu\big[|\mathcal I_r^s\langle g, X_t\rangle|^{1+\gamma}\big]
    \leq C e^{t\alpha+(t-s) (\gamma\alpha- (1+\gamma)\kappa b)}.
\]
\end{lem}

\begin{proof}
	Fix $h \in \mathcal P$ and $\mu \in \mathcal M_c(\mathbb R^d)$. Let $C_0$ be the constant in the Lemma \ref{lem: temp}.
    For all $\kappa \in \mathbb Z_+$,  $0\leq r\leq s\leq t$ with $s-r \leq 1$,  $g\in \mathcal P$ with $Q_{\kappa} g \leq h$, and $c>0$, we have
\begin{align}
    &\mathbb P_\mu\big[|\mathcal I_r^s\langle g, X_t\rangle|^{1+\gamma}\big]
    = (1+\gamma)\int_0^\infty \lambda^{\gamma} \mathbb P_{\mu}(|\mathcal I_r^s\langle g, X_t\rangle|>\lambda) d\lambda
    \\&\leq (1+\gamma)\int_0^c \lambda^{\gamma} d\lambda +(1+\gamma)\int_c^\infty \lambda^{\gamma}\mathbb P_\mu(|\mathcal I_r^s\langle g, X_t\rangle|> \lambda) d\lambda
    \\& \leq c^{1+\gamma} + e^{(1+\beta)(t-s)(\alpha- \kappa b) + \alpha r} C_0  (1+\gamma)\int_c^\infty \frac{1}{\lambda^{1+(\beta-\gamma)}}d\lambda
    \\&\leq c^{1+\gamma} e^{\alpha r} + e^{(1+\beta)(t-s)(\alpha- \kappa b) + \alpha r} C_0   \frac{1+\gamma}{\beta - \gamma} \frac{1}{c^{\beta - \gamma}}.
\end{align}
    Taking $c = e^{(t-s)(\alpha- \kappa b)}$, we get
\begin{align}
    &\mathbb P_\mu\big[|\mathcal I_r^s\langle g, X_t\rangle|^{1+\gamma}\big]
    \leq e^{(1+\gamma)(t-s)(\alpha- \kappa b)} e^{\alpha r}\Big(1+ C_0 \frac{1+\gamma}{\beta - \gamma}\Big).
\end{align}
    Note that
\[
    (1+\gamma)(t-s)(\alpha- \kappa b) + \alpha r
    \leq t\alpha+(t-s) (\gamma\alpha- (1+\gamma)\kappa b).
\]
	So the desired result is true with
\[
	C := \Big(1+ C_0 \frac{1+\gamma}{\beta - \gamma}\Big).
	\qedhere
\]
\end{proof}

\begin{lem}
\label{lem: control moment}
	For all $h \in \mathcal P$, $\mu \in \mathcal M_c(\mathbb{R}^d)$, $\gamma\in (0, \beta)$ and $\kappa \in \mathbb Z_+$, there exist a constant $C > 0$ such that for each $t\geq 0$, we have
\begin{itemize}
\item[(1)]
    $\sup_{g\in \mathcal P: Q_\kappa g \leq h}\|\langle g,X_t\rangle\|_{\mathbb{P}_{\mu};1+\gamma}\leq C e^{(\alpha-\kappa b)t}$ provided $\alpha\gamma > \kappa (1+\gamma)b$;
\item[(2)]
    $\sup_{g\in \mathcal P: Q_\kappa g \leq h}\|\langle g,X_t\rangle\|_{\mathbb{P}_{\mu};1+\gamma}\leq C te^{\frac{\alpha}{1+\gamma}t}$ provided $\alpha\gamma = \kappa (1+\gamma)b$;
\item[(3)]
    $\sup_{g\in \mathcal P: Q_\kappa g \leq h} \|\langle g,X_t\rangle\|_{\mathbb{P}_{\mu};1+\gamma}\leq C e^{\frac{\alpha}{1+\gamma}t}$ provided $\alpha\gamma < \kappa (1+\gamma)b$.
\end{itemize}
\end{lem}
\begin{proof}
    Fix $\gamma \in (0,\beta)$ and $\mu \in \mathcal M_c(\mathbb R^d)$.
    Let $C$ be the constant in Lemma \ref{lem: control of mgtrs}.
    Using the triangle inequality, for all $\kappa\in \mathbb Z_+$, $g \in \mathcal P$ with $Q_\kappa g \leq h$ and $t\geq 0$, we have
\begin{align}
    &\|\langle g,X_t\rangle\|_{\mathbb P_\mu;1+\gamma}
    \\ &\leq \sum_{k=0}^{\lfloor t\rfloor - 1}\big\| \mathcal{I}_{t-k-1}^{t-k}\langle g,X_t\rangle \big\|_{\mathbb P_\mu;1+\gamma}+\big\| \mathcal{I}_{0}^{t-\lfloor t \rfloor}\langle g,X_t\rangle  \big\|_{\mathbb P_\mu;1+\gamma}
    + |\langle T^\alpha_t g,\mu\rangle|
    \\ &\leq C^{\frac{1}{1+\gamma}} e^{\frac{\alpha}{1+\gamma}t} \sum_{k=0}^{\lfloor t\rfloor} e^{\frac{\gamma\alpha-\kappa (1+\gamma)b}{1+\gamma} k} + e^{(\alpha - \kappa b)t} \langle h,\mu\rangle. 
\end{align}
    By calculating the sum on the right, we get the desired result.
\end{proof}

\subsection{Martingales}
    In this subsection, we will prove the almost sure and $L^{1+\gamma}(\mathbb{P}_{\mu})$ convergence of some martingales for $\gamma\in (0, \beta)$. Recall that $L$ is the infinitesimal generator of the OU-process. Let $f\in \mathcal{P}\cap C^2(\mathbb{R}^d)$ be such that $Lf \in \mathcal{P}$, and let $a\in \mathbb{R}$, we define that
\begin{align}
\label{defmartingale}
    M_t^{f,a}:=e^{-(\alpha-ab)t}\langle f,X_t\rangle-\int_0^t e^{-(\alpha-ab)s}\langle (L+ab)f, X_s\rangle~ ds.
\end{align}
    The following lemma says that $\{M_t^{f,a}: t\geq 0\}$ is a martingale with respect to $\{\mathscr{F}_t\}_{t\geq 0}$.
\begin{lem}
\label{lemma25}
    Let $f\in \mathcal{P}\cap C^2(\mathbb{R}^d)$ be such that $Lf \in \mathcal{P}$ and let $a\in \mathbb R$. Then the process $(M_t^{f,a})_{t\geq 0}$ is an $(\mathscr F_t)$-martingale.
\end{lem}
\begin{proof}
    Write $\bar{f}=(L+ab)f$. According to \cite[Theorem A.55]{Li2011Measure-valued}, we have
\begin{align}\label{Theorem55}
    T_t^{ab}f(x)= f(x)+\int_0^t T_s^{ab}\bar{f}(x)~ds,\quad t\geq 0,x\in \mathbb{R}^d.
\end{align}
Let $0\leq s\leq t$,
\begin{align}
\label{martingale1}
    &\mathbb{P}_{\mu}[M_t^{f,a}|\mathscr{F}_s]
    =e^{-(\alpha-ab)t}\mathbb{P}_{\mu}\left[\langle f,X_t\rangle|\mathscr{F}_s\right]-\mathbb{P}_{\mu}\Big[\int_0^t e^{-(\alpha-ab)u}\langle \bar{f}, X_u\rangle~ du\Big|\mathscr{F}_s\big]
    \\&=e^{-(\alpha-ab)t}\langle T_{t-s}^{\alpha}f, X_s\rangle-\int_0^s e^{-(\alpha-ab)u}\langle \bar{f}, X_u\rangle~ du -\int_s^t e^{-(\alpha-ab)u}\langle T_{u-s}^{\alpha} \bar{f},X_s\rangle~ du.
\end{align}
    Using \eqref{Theorem55} and Fubini's theorem, we have
\begin{align}
    &\int_s^t e^{-(\alpha-ab)u}\langle T_{u-s}^{\alpha} \bar{f},X_s\rangle~ du=e^{-(\alpha-ab)s}\int_s^t\langle T_{u-s}^{ab}\bar{f},X_s\rangle~du\\
    &=e^{-(\alpha-ab)s}\Big\langle\int_0^{t-s}T_{u}^{ab}\bar{f}~du,X_s\Big\rangle=e^{-(\alpha-ab)s}\left(\langle T_{t-s}^{ab}f,X_s\rangle-\langle
    f,X_s\rangle\right)\\
    &=e^{-(\alpha-ab)t}\langle T_{t-s}^{\alpha}f, X_s\rangle-e^{-(\alpha-ab)s}\langle
    f,X_s\rangle.
\end{align}
   Using this and \eqref{martingale1}, we get the desired result.
\end{proof}

    Let $p=(p_1,...,p_d)\in \mathbb Z_+^d$, recall that $\phi_p$ is an eigenfunctions of $L$ corresponding to the eigenvalue $-|p|b$. Define
\[
    H_t^p
    :=e^{-(\alpha-|p|b)t}\langle\phi_p,X_t\rangle, \quad t\geq 0.
\]

\begin{lem}\label{lemma26}
    $(H^p_t)_{t\geq 0}$ is a martingale with respect to $(\mathscr F_t)$.
    Moreover, if $\alpha\beta>|p|b(1+\beta)$, then for any $\gamma\in (0, \beta)$ and $\mu \in \mathcal M_c(\mathbb R^d)$, we have $\sup_{t\geq 0}\|H_t^p\|_{\mathbb P_\mu;1+\gamma}< \infty$ and
\[
    H_{\infty}^p
    :=\lim_{t\rightarrow \infty}H_t^p
\]
exists $\mathbb{P}_{\mu}$-a.s and in $L^{1+\gamma}(\mathbb{P}_{\mu}).$
\end{lem}
\begin{proof}
   It follows from Lemma \ref{lemma25} that $(H_t^p)_{t\geq 0}$ is a martingale.

    There exists $\gamma_0 \in (0,\beta)$ close enough to $\beta$ such that for $\gamma\in [\gamma_0, \beta)$, $\alpha\gamma>|p|(1+\gamma)b$.
    Using  Lemma \ref{lem: control moment} and the fact $\kappa_{\phi_p}=|p|$, we get that there exists a constant $C_{\gamma, \mu, p}>0$ (depending on $\gamma$, $\mu$ and $p$)  such that
\[
 	\|H_t^p\|_{\mathbb P_\mu;1+\gamma}
    \leq C_{\gamma, \mu, p} e^{-(\alpha-|p|b)t}e^{(\alpha-|p|b)t}
    =C_{\gamma, \mu, p}, \quad t\geq 0.
\]
    For any $\gamma\in (0, \gamma_0)$, note that
\[
	\|H_t^p\|_{\mathbb P_\mu;1+\gamma}
	\leq\|H_t^p\|_{\mathbb P_\mu;1+\gamma_0}
	<C_{\gamma_0, \mu, p},
	\quad t\geq 0.
\]

    Hence, for each $\gamma \in (0,\beta)$, the martingale is bounded in $L^{1+\gamma}(\mathbb{P}_{\mu})$ and hence converges in $L^{1+\gamma}(\mathbb{P}_{\mu}) $ and almost surely, by \cite[Theorem 5.4.5]{Durrett2010Probability}.
\end{proof}

In particular, when $p=0$, $H_t^0$ reduces to $H_t:=e^{-\alpha t}\|X_t\|$, thus, as $t\rightarrow \infty$, $H_t$ converges to $H_{\infty}$, $\mathbb{P}_{\mu}$ almost surely and in $L^{1+\gamma}(\mathbb{P}_{\mu})$.

\begin{lem}\label{lem: control of Wt}
 For all $\gamma\in (0,\beta)$ and $\mu\in \mathcal M_c(\mathbb R^d)$, there exists a $C> 0$ such that for all $0\leq s<t$,
\[
    \|H_t-H_s\|_{\mathbb{P}_{\mu};1+\gamma}
    \leq C e^{-\frac{\alpha \gamma}{1+\gamma}s}.
\]
\end{lem}

\begin{proof}
     Fix $\gamma \in (0,\beta)$ and $\mu\in \mathcal M_c(\mathbb R^d)$.
 According to Lemma \ref{lem: control of mgtrs}, there exists a constant $C_0>0$ such that for all $0\leq r\leq s $ with $s-r\leq1$, we have
    \begin{align}
        \mathbb{P}_{\mu}\big[\big|e^{\alpha(t-s)}\|X_s\|-e^{\alpha(t-r)}\|X_r\|\big|^{1+\gamma}\big]
        \leq C_0e^{\alpha t+(t-s)\alpha\gamma}.
    \end{align}
    Dividing both sides by $e^{\alpha t(1+\gamma)}$, we get
    \begin{align}
        \mathbb{P}_{\mu}\big[|H_s-H_r|^{1+\gamma}\big]\leq  C_0 e^{-\alpha \gamma s}.
    \end{align}
    Then there is a $C>0$, for any $0\leq s<t$,
\begin{align}
	& \|H_t-H_s\|_{\mathbb{P}_{\mu};1+\gamma}
	\\&\leq \|H_{\lfloor s \rfloor+1}-H_s\|_{\mathbb{P}_{\mu};1+\gamma}+\sum_{k=\lfloor s \rfloor+1}^{\lfloor t \rfloor}\|H_{k+1}-H_{k}\|_{\mathbb{P}_{\mu};1+\gamma}+\|H_t-H_{\lfloor t \rfloor+1}\|_{\mathbb{P}_{\mu};1+\gamma}
	\\& \leq C_0^{\frac{1}{1+\gamma}} \Big(e^{-\frac{\alpha \gamma s}{1+\gamma}}+\sum_{k=\lfloor s \rfloor+1}^{\lfloor t \rfloor}e^{-\frac{\alpha \gamma k}{1+\gamma}}+ e^{-\frac{\alpha \gamma t}{1+\gamma}}\Big)
%	\leq Ce^{-\frac{\alpha \gamma}{1+\gamma}s}\leq Ce^{-\frac{\alpha}{2}s}.
	\leq Ce^{-\frac{\alpha \gamma}{1+\gamma}s}.
    \qedhere
\end{align}	
\end{proof}

Write $\mathbb{\tilde{P}}_{\mu}=\mathbb{P}_{\mu}(\cdot|D^c)$.
\begin{lem}
\label{lem: control of XT}
	For any $\mu\in \mathcal M_c(\mathbb R^d)$, there is a constant $C>0$ such that for all $f:[0,\infty)\to [1,\infty)$ and $t\geq 0$, we have
\begin{align}
    \mathbb{\tilde{P}}_{\mu}\big(\|X_t\|\leq f(t)\big)\leq C \Big(\frac{f(t)}{e^{\alpha t}}\Big)^{\beta}.
\end{align}
\end{lem}

\begin{proof}
    Note that $(\|X_t\|)_{t\geq 0}$ is a continuous-state branching process with branching mechanism $\psi$. According to \cite[Example 3.1]{Li2011Measure-valued}, we have
\begin{align}\label{CSBP}
    \mathbb{P}_{\mu} [e^{-\lambda\|X_t\|}]
    =e^{-\|\mu\|v_t(\lambda)},
    \quad \lambda \geq 0, t\geq 0,
\end{align}
    where
\[
    v_t(\lambda)=\frac{\alpha^{1/\beta}}{(\alpha\lambda^{-\beta}e^{-\alpha \beta t}+1-e^{-\alpha \beta t})^{1/\beta}},\quad t\geq 0,\lambda\geq 0.
\]
    Recall that the extinction probability of the superprocess is $\mathbb{P}_{\mu}(D)=e^{-\alpha^*\|\mu\|}$, where $\alpha^*=\alpha^{1/\beta}$ is the larger roots of $\psi(z)=0$.
    By \eqref{CSBP} and the Markov property we have for all $\mu \in \mathcal M_c(\mathbb R^d), \lambda \geq 0$ and
    $t\geq 0$,

\begin{align}
    \label{laplaceexpress}
    \mathbb{\tilde{P}}_{\mu}[e^{-\lambda \|X_t\|}]
    &=\frac{ \mathbb P_\mu[e^{-\lambda \|X_t\| }\mathbb P_\mu(D^c| \mathscr F_t)] }{\mathbb P_\mu(D^c)}
    =\frac{ \mathbb P_\mu[e^{-\lambda \|X_t\| }(1-e^{-\alpha^*\|X_t\|})] }{\mathbb P_\mu(D^c)}
    \\&=\frac{e^{-\|\mu\|v_t(\lambda)}-e^{-\|\mu\|v_t(\lambda+\alpha^*)}}{1-e^{-\|\mu\|\alpha^*}}
    \leq \frac{\|\mu\|}{1-e^{-\|\mu\|\alpha^*}}\left|v_t(\lambda)-v_t(\lambda+\alpha^*)\right|.
\end{align}

By Chebyshev's inequality and \eqref{laplaceexpress}, we get
\begin{align*}
    \mathbb{\tilde{P}}_{\mu}[\|X_t\|\leq f(t)]&=\mathbb{\tilde{P}}_{\mu}[e^{-\lambda\|X_t\|}\geq e^{-\lambda f(t)}]\leq e^{\lambda f(t)}\mathbb{\tilde{P}}_{\mu}[e^{-\lambda \|X_t\|}]\\
    &\leq \frac{\|\mu\|}{1-e^{-\|\mu\|\alpha^*}}e^{\lambda f(t)}\left|v_t(\lambda)-v_t(\lambda+\alpha^*)\right|.
\end{align*}
	Let $w_t(\lambda)=\alpha \lambda^{-\beta}e^{-\alpha \beta t}+1-e^{-\alpha\beta t}$, then $v_t(\lambda)=\alpha^{1/\beta}w_t(\lambda)^{-1/\beta}$.
Note that
\begin{align*}
    \left|v_t(\lambda)-v_t(\lambda+\alpha^*)\right|&\leq\frac {\alpha^{1/\beta}}{\beta} \left|w_t(\lambda)-w_t(\lambda+\alpha^*)\right| \sup_{s\in [\lambda,\lambda+\alpha^*]}w_t(s)^{-1/\beta-1}\\
    &\leq  \frac{\alpha^{1/\beta+1}}{\beta}\left|\frac{1}{\lambda^{\beta}}-\frac{1}{(\lambda+\alpha^*)^{\beta}}\right|e^{-\alpha\beta t} \sup_{s\in [\lambda,\lambda+\alpha^*]}w_t(s)^{-1/\beta-1}\\
    &\leq  \frac{\alpha^{1/\beta+1}}{\beta}\frac{2}{\lambda^{\beta}}e^{-\alpha\beta t} \sup_{s\in [\lambda,\lambda+\alpha^*]}w_t(s)^{-1/\beta-1}.
\end{align*}
Therefore, for each $\mu\in \mathcal M_c(\mathbb R^d)$, there is a constant $C_0>0$ such that for all non-negative function $f$ on $[0,\infty)$, $\lambda \geq 0$ and $t\geq 0$, we have
\begin{align}
\label{eq: inequality before take lambda}
    \mathbb{\tilde{P}}_{\mu}\big(\|X_t\|\leq f(t)\big)
    \leq C_0e^{\lambda f(t)}\frac{1}{\lambda^{\beta}}e^{-\alpha\beta t} \sup_{s\in [\lambda,\lambda+\alpha^*]}w_t(s)^{-1/\beta-1}.
\end{align}

Notice that
\begin{align*}
    w_t(s)^{-1/\beta-1}&\leq \frac{1}{(\alpha s^{-\beta}e^{-\alpha \beta t})^{1/
    \beta+1}}\mathbf{1}_{\{t\leq 1\}}+\frac{1}{(1-e^{-\alpha\beta t})^{1/\beta+1}}\mathbf{1}_{t\geq 1}\\
    &\leq \frac{s^{1+\beta}e^{\alpha(1+\beta)}}{\alpha^{1/\beta +1}}+\frac{1}{(1-e^{-\alpha\beta })^{1/\beta+1}}.
\end{align*}
Therefore, for each $\lambda \in [0,1]$,
\begin{align*}
    \sup_{s\in [\lambda,\lambda+\alpha^*]}w_t(s)^{-1/\beta-1}
    &\leq\sup_{s\in [\lambda,\lambda+\alpha^*]}\frac{s^{1+\beta}e^{\alpha(1+\beta)}}{\alpha^{1/\beta +1}}+\frac{1}{(1-e^{-\alpha\beta })^{1/\beta+1}}\\
    &\leq \frac{(1+\alpha^*)^{1+\beta}e^{\alpha(1+\beta)}}{\alpha^{1/\beta+1}}+\frac{1}{(1-e^{-\alpha\beta })^{1/\beta+1}} =: C_1.
\end{align*}
Finally, plugging $\lambda = f(t)^{-1} \in [0,1]$ back to \eqref{eq: inequality before take lambda} and using the above inequality, we get the desired result.
\end{proof}

\subsection{Parameters}
    Recall the definition of operator $Z_1$ in \eqref{eq: def of Zf} and the invariant density $\varphi$ in \eqref{invariantdensity}.
    Define
 \begin{align}
 \label{parameter_mk}
      m_t[f]
      :=e^{-\alpha (t+1)}\int_{\mathbb R^d} (Z_1T_t^{\alpha}f)(x)\varphi(x)dx,\quad t\geq 0, f\in \mathcal P.
 \end{align}
    According to \cite[Lemma 2.7]{MM}, for each $f\in\mathcal{P}$, there is a constant $C>0$ such that
\[
    |m_t[f]|
    \leq C e^{(\alpha\beta-\kappa_fb(1+\beta))t},
    \quad t\geq 0.
\]
    Accodting to \cite[Lemma 4.2]{MM}, if $f \in \mathcal{P}$ satisfies $\alpha\beta=\kappa_f b(1+\beta)$, then the following limit exists:
\begin{align}
\label{para: critical case}
    \tilde{m}[f]:=\lim_{t\rightarrow \infty}\frac{1}{t}\sum_{k=0}^{\lfloor t \rfloor}m_k[f]=\langle\Psi_0(-i\phi),\varphi\rangle,
\end{align}
    where
\[
    \phi
    :=\sum_{p\in \mathbb Z_+^d:|p|=\kappa_f}\langle f, \phi_p\rangle\phi_p.
\]
    According to \cite[Lemma 5.1]{MM}, if $f\in \mathcal{P}$ satisfies that $\alpha\beta<\kappa_fb(1+\beta)$, then 
\begin{align}
    m[f]
    :=\sum_{k=0}^\infty m_k[f]
    =\alpha e^{\alpha}\int_{\mathbb{R}^d}\int_0^{\infty} e^{-\alpha s}\Psi_0(-iT_{s}^{\alpha}f)(x)~ds~\varphi(x)~dx, \label{msmallcase}
\end{align}
and both the series and the integral converge absolutely.
\begin{lem}
\label{lem: charactreisticfunction}
    Let $q$ be a measure on $\mathbb R^d\setminus\{0\}$ with $\int |x|^{1+\beta} q(dx) \in (0,\infty)$.
    Then $$\theta \mapsto  \exp\Big\{\int_{\mathbb R^d\setminus\{0\}} (i\theta \cdot x)^{1+\beta} q(dx)\Big\}$$
    is the characteristic function of a $(1+\beta)$-stable random variable.
\end{lem}
\begin{proof}
    From measure theory, there exist a measure $\lambda$ on $S:= \{\xi\in \mathbb R^d:|\xi| = 1\}$ and a kernel $k(\xi,dt)$ from $S$ to $\mathbb R_+$ such that
\[
    \int_{\mathbb R^d\setminus \{0\}} f(x)q(dx) = \int_S \lambda(d\xi) \int_{\mathbb R^+} f(\xi t)k(\xi,dt),\quad
    f\in \mathcal B(\mathbb R^d\setminus \{0\}, \mathbb R_+).
\]
    Define another measure $\lambda_0$ on $S$ by
\[
    \lambda_0(d\xi) := \frac{\int_0^\infty t^{1+\beta}k(\xi,dt)}{\Gamma(-1-\beta)} \lambda (d\xi)
\]
    Then $\lambda_0$ is a non-zero finite measure, since
\[
    \lambda_0(S) = \frac{1}{\Gamma(-1-\beta)} \int_S \lambda (d\xi) \int_0^\infty |t\xi|^{1+\beta}k(\xi,dt)
    = \frac{1}{\Gamma(-1-\beta)} \int_{\mathbb R^d\setminus\{0\}} |x|^{1+\beta} q(dx) \in (0,\infty).
\]
    Define a measure $\nu$ on $\mathbb R^d\setminus\{0\}$ by
\[
    \int_{\mathbb R^d\setminus\{0\}}f(x)\nu(dx)= \int_{S} \lambda_0(d\xi) \int_0^\infty f(r\xi) \frac{dr}{r^{2+\beta}} .
\]
    Then, according to \cite[Remark 14.4]{Sato1999Levy}, $\nu$ is the L\`evy measure of a $(1+\beta)$-stable distribution on $\mathbb R^d$, say $\mu$, whose characteristic function is \[\hat \mu(\theta)=\exp\Big\{\int_{\mathbb R^d\setminus\{0\}} (e^{-i\theta \cdot y}-1+i\theta \cdot y) \nu(dy)\Big\}.\]
Finally, 
\begin{align}
    &\int_{\mathbb R^d\setminus\{0\}} (e^{-i\theta \cdot y}-1+i\theta \cdot y) \nu(dy)
    = \int_S \lambda_0(d\xi) \int_0^\infty (e^{-ir\theta \cdot \xi}-1+ir\theta \cdot \xi) \frac{dr}{r^{2+\beta}}
\\&\quad = \int_S \lambda (d\xi) \int_0^\infty (e^{-ir\theta \cdot \xi}-1+ir\theta \cdot \xi) \frac{dr}{\Gamma(-1-\beta)r^{2+\beta}}\int_0^\infty t^{1+\beta} k(\xi,dt)
\\&\quad = \int_S \lambda (d\xi) \int_0^\infty (i\theta\cdot \xi)^{1+\beta} t^{1+\beta} k(\xi,dt)
= \int_S \lambda(d\xi) \int_0^\infty (i\theta \cdot t\xi)^{1+\beta} k(\xi,dt)
\\&\quad = \int_{\mathbb R^d} (i\theta \cdot x)^{1+\beta} q(dx).
\qedhere
\end{align}
\end{proof}

\begin{cor}
\label{cor: alpha stable rv}
	For all $t\geq 0$ and $f\in \mathcal P$, 
\begin{enumerate}
\item 
\label{it: first stable}
    $\theta \mapsto \exp(m_t[\theta f])$ is the characteristic function of a $(1+\beta)$-stable random variable.
\item 
\label{it: second stable}
    $\theta \mapsto \exp(\tilde m[\theta f])$ is the characteristic function of a $(1+\beta)$-stable random variable, provided $\alpha\beta=\kappa_f b(1+\beta)$.
\item 
\label{it: third stable}
    $\theta \mapsto \exp(m[\theta f])$ is the characteristic function of a $(1+\beta)$-stable random variable, provided $\alpha\beta < \kappa_f b(1+\beta)$.
\end{enumerate}
\end{cor}
\begin{proof}
    \eqref{it: first stable}: 
	Fix $t\geq 0$ and $f\in \mathcal P$.
	Note that $m_t[\theta f]$ can be rewritten as 
\[
    m_t[\theta f]= e^{-\alpha (t+1)}\int_{\mathbb R^d} dx~\varphi(x)\int_0^1 T_{1-s}^\alpha (-i\theta T_{s+t}^\alpha f)^{1+\beta}(x)~ds,
    \quad \theta \in \mathbb R.
\]
	Therefore, according to Lemma \ref{lem: charactreisticfunction}, we only need to show that
\begin{equation}
\label{eq: what I want to proof}
	\int_{\mathbb R^d} dx~\varphi(x)\int_0^1 T_{1-s}^\alpha (|T_{s+t}^\alpha f|^{1+\beta})(x)~ds < \infty.
\end{equation}
	According to Lemma \ref{lem: upper bound for usgx}.(2), parts (1), (2) and (4) of Lemma \ref{lem: property of controllable operators},  we know that
$
	g \mapsto \int_0^1 T_{1-s}^\alpha (|T_{s}^\alpha g|^{1+\beta})~ds
$
	is a $(1+\beta)$-controallable operator on $\mathcal P$.
	This implies that $x \mapsto \int_0^1 T_{1-s}^\alpha (|T_{s+t}^\alpha f|^{1+\beta})(x) ds$ is an element of $\mathcal P$.
	Therefore \eqref{eq: what I want to proof} is true.

    The proofs of \eqref{it: second stable} and \eqref{it: third stable} are similar to that of \eqref{it: first stable}.
\end{proof}


\section{Proof of main results}

In this section, we will prove the main results of this paper. Recall that $\mathbb{\tilde{P}}_{\mu}=\mathbb{P}_{\mu}(\cdot|D^c)$.

\begin{lem}\label{lem: mainlemma}
Let $f\in \mathcal{P}$.
Assume that $\alpha\beta\leq \kappa_fb(1+\beta)$.
Then for all $k\geq 0$ and $\mu \in \mathcal{M}_c(\mathbb{R}^d)$, under $\mathbb{P}_{\mu}(\cdot | D ^c)$, we have
 \begin{align}
      \gamma_{t,k}:=\frac{\mathcal I_{t-k-1}^{t-k}\langle f ,X_t\rangle}{(e^{\alpha k}\|X_{t-k-1}\|)^{\frac{1}{1+\beta}}}\xrightarrow{d}\zeta_k, \quad t\rightarrow \infty, \label{limitdistribution1}
 \end{align}
 where $\zeta_k$ is a $(1+\beta)$-stable random variable with characteristic function
 $$\mathbb{E}(e^{i\theta\zeta_k})=\exp(e^{\alpha}m_k[\theta f]),\quad \theta \in \mathbb R.$$
 \end{lem}
 \begin{proof}
	We only need to show that
\begin{align}
    \mathbb{P}_{\mu}[\exp(i\gamma_{t,k}); D^c]
    \xrightarrow[t\rightarrow \infty]{}\mathbb{P}_{\mu}(D^c)\exp(e^{\alpha}m_k[f]),
    \quad \mu \in \mathcal M_c(\mathbb R^d), f\in \mathcal P, k \geq 0.
\end{align}
	In fact, once we prove this, we can replace $f$ by $\theta f$, with $\theta \in \mathbb R$ being  arbitrary,  to get the desired result.
	In the rest of the proof we fix a $\mu \in \mathcal M_c(\mathbb R^d)$ and an $f\in \mathcal P$.
	Define 
\[
	A_t(\epsilon):=\{ \|X_t\| \geq \|\mu\|e^{(\alpha - \epsilon)t} \},\quad t\geq 0, \epsilon > 0.
\]
	
	Step 1. We will show that for all $\epsilon > 0, k\geq 0$ and $t>k+1$, we have
\begin{align}
    \big|\mathbb{P}_{\mu}\big[e^{i\gamma_{t,k}}-e^{e^{\alpha}m_k[f]}; D^c\big]\big|
    \leq J_1(t,k,\epsilon)+J_2(t,k,\epsilon)+J_3(t,k,\epsilon)
\end{align}
	where
\begin{align}
\label{eq: Def of Ji}
	J_1(t,k,\epsilon)
	&:= \mathbb{P}_{\mu}\big[|\langle U_1(\theta_{t,k}T_k^{\alpha}f)-\tilde U_1(\theta_{t,k}T_k^{\alpha}f), X_{t-k-1}\rangle|; A_{t-k-1}(\epsilon) \big],
	\\ J_2(t,k,\epsilon)
	&:= \mathbb{P}_{\mu}\big[|\langle Z_1(\theta_{t,k}T_k^{\alpha}f),X_{t-k-1}\rangle-e^{\alpha}m_k[f]|; A_{t-k-1}(\epsilon)\big],
	\\ J_3(t,k, \epsilon)
	&:=2\mathbb{P}_{\mu}(A_{t-k-1}(\epsilon)\Delta D^c),
	\\\theta_{t,k}
	&:= (e^{\alpha k}\|X_{t-k-1}\|)^{-\frac{1}{1+\beta}}.
\end{align}
In fact, from the definitions of $U_1, \tilde U_1$ and $Z_1$, we have
\begin{align}
\label{eq: need1}
    &\mathbb{P}_{\mu}[e^{i\gamma_{t,k}}|\mathscr{F}_{t-k-1}]
    =\mathbb{P}_{\mu}[e^{i\theta_{t,k}\langle T_k^{\alpha}f,X_{t-k}\rangle-i\theta_{t,k}\langle T_{k+1}^{\alpha}f, X_{t-k-1}\rangle}|\mathscr{F}_{t-k-1}]
    \\&=e^{\langle (U_1 - iT_{1}^{\alpha}) (\theta_{t,k}T_k^{\alpha}f),X_{t-k-1}\rangle}
    \\&=e^{\langle (U_1 - \tilde U_1+Z_1) (\theta_{t,k}T_k^{\alpha}f),X_{t-k-1}\rangle},
    \quad k\geq 0, t\geq k+1.
\end{align}
	From Corollary \ref{cor: alpha stable rv}, we have
\begin{equation}
\label{eq: need2}
	\operatorname {Re} m_t[f] < 0.
\end{equation}
    According to the definitions of $\tilde U_1$ and $Z_1$, \eqref{eq: need1}, \eqref{eq: -v has positive real part}, \eqref{eq: need2} and the fact that
\[
	|e^{-x} - e^{-y}| \leq |x-y|,\quad x,y \in \mathbb C_+,
\]
	we get that for all $k\geq 0$, $t\geq k+1$ and $\epsilon> 0$,
\begin{align}
\label{eq: inequality that will used later}
    &\big|\mathbb{P}_{\mu}\big[e^{i\gamma_{t,k}}-e^{e^{\alpha}m_k[f]}; D^c\big]\big|
    \\& \leq \mathbb{P}_{\mu}\Big[\big| \mathbb{P}_{\mu}[e^{i\gamma_{t,k}}-e^{e^{\alpha}m_k[f]}; D^c | \mathscr F_{t-k-1}]\big|\Big]
    \\& \leq \mathbb{P}_{\mu}\Big[\big| \mathbb{P}_{\mu}[e^{i\gamma_{t,k}}-e^{e^{\alpha}m_k[f]}; A_{t-k-1}(\epsilon)| \mathscr F_{t-k-1}]\big| + 2\mathbb P_\mu(A_{t-k-1}(\epsilon) \Delta D^c| \mathscr F_{t-k-1})\Big]
    \\& = \mathbb{P}_{\mu}\Big[ \big|\mathbb{P}_{\mu}[e^{i\gamma_{t,k}}| \mathscr F_{t-k-1}]-e^{e^{\alpha}m_k[f]}\big|;A_{t-k-1}(\epsilon)\Big] + J_3(t,k,\epsilon)
    \\& \leq \mathbb{P}_{\mu}\big[|e^{\langle (U_1 - iT_{1}^{\alpha}) (\theta_{t,k}T_k^{\alpha}f),X_{t-k-1}\rangle}-e^{e^{\alpha}m_k[f]}|;A_{t-k-1}(\epsilon)\big]+  J_3(t,k,\epsilon)
    \\& \leq \mathbb{P}_{\mu}\big[|\langle (U_1 - i T_1^\alpha)(\theta_{t,k}T_k^{\alpha}f),X_{t-k-1}\rangle-e^{\alpha}m_k[f]|;A_{t-k-1}(\epsilon)\big]+  J_3(t,k,\epsilon)
    \\&\leq J_1(t,k,\epsilon) + J_2(t,k,\epsilon)+J_3(t,k,\epsilon).
\end{align}

Step 2.  We will show that there exists a $C>0$ such that for all $k\geq 0$, $t\geq k+1$ and $\epsilon> 0$,
    \begin{align}
    \label{lemma31q}
      J_1(k,t,\epsilon)\leq C\exp\Big\{-\Big(\alpha\frac{\beta}{1+\beta}-\epsilon\frac{1+2\beta}{1+\beta}\Big)(t-k-1)\Big\}.
    \end{align}
	In fact, let $(R,S)$ be the $(1+2\beta)$-control-pair in Lemma \ref{lem: upper bound for usgx}. (6).
	Then, we have for all $k\geq 0$, $t\geq k+1$ and $\epsilon> 0$,
\begin{align}
   & |(U_1 - \tilde U_1)(\theta_{t,k}T_k^{\alpha}f)|\mathbf{1}_{A_{t-k-1}(\epsilon)}
   \leq R(|\theta_{t,k}T_k^{\alpha}f|)\mathbf{1}_{A_{t-k-1}(\epsilon)}
   \\&\leq R \Big(\frac{e^{(\alpha-\kappa_fb)k} Qf}{(e^{\alpha k}\|\mu\|e^{(\alpha-\epsilon)(t-k-1)})^\frac{1}{1+\beta}}\Big)
    \leq S \Big(\frac{e^{(\alpha-\kappa_fb)k}}{(e^{\alpha k}\|\mu\|e^{(\alpha-\epsilon)(t-k-1)})^\frac{1}{1+\beta}}Qf\Big)
   \\&\leq \Big(\frac{e^{(\alpha-\kappa_fb)k}}{(e^{\alpha k}\|\mu\|e^{(\alpha-\epsilon)(t-k-1)})^\frac{1}{1+\beta}}\Big)^{1+2\beta}SQf
   \\&=\|\mu\|^{-\frac{(1+2\beta)}{(1+\beta)}}e^{\frac{1+2\beta}{1+\beta}(\alpha\beta-\kappa_fb(1+\beta))k}e^{-\frac{(\alpha-\epsilon)(1+2\beta)(t-k-1)}{1+\beta}}SQf
   \\&\leq\|\mu\|^{-\frac{(1+2\beta)}{(1+\beta)}} e^{-\frac{(\alpha-\epsilon)(1+2\beta)(t-k-1)}{1+\beta}}SQf.
\end{align}
Thus for all $k\geq 0$, $t\geq k+1$ and $\epsilon> 0$,
\begin{align}
\label{eq: estimate of J1}
     J_1(k,t,\epsilon)&
     \leq \|\mu\|^{-\frac{(1+2\beta)}{(1+\beta)}} e^{-\frac{(\alpha-\epsilon)(1+2\beta)(t-k-1)}{1+\beta}}\mathbb{P}_{\mu}[\langle SQf,X_{t-k-1}\rangle]\\
     & \leq \|\mu\|^{-\frac{1+2\beta}{1+\beta}} \langle Q_0 SQf, \mu \rangle \exp\Big\{-\Big(\alpha\frac{\beta}{1+\beta}-\epsilon\frac{1+2\beta}{1+\beta}\Big)(t-k-1)\Big\}.
\end{align}
	Step 3.
	Fix a $\gamma\in(0,\beta)$ small enough so that $\alpha \gamma < b < (1+\gamma)b$.
	Then there exists a constant $C > 0$ such that for all $k\geq 0$, $t\geq k+1$ and $\epsilon> 0$,
\begin{align}
\label{eq:31step3}
    J_2(t,k,\epsilon)
    \leq C \exp\Big(-(\frac{\alpha\gamma}{1+\gamma}-\epsilon)(t-k-1)\Big).
\end{align}
    In fact, according to the definitions of $Z_1$ and $m_t$, we have for all $k\geq 0$, $t\geq k+1$ and $\epsilon> 0$,
    \begin{align}
          &\langle Z_1(\theta_{t,k}T_k^{\alpha}f),X_{t-k-1}\rangle-e^{\alpha}m_k[f]
          \\&= \theta_{t,k}^{1+\beta} \langle Z_1T_k^{\alpha}f,X_{t-k-1}\rangle - e^{-\alpha k}\langle  Z_1T_k^{\alpha}f,\varphi\rangle
          \\&=e^{-\alpha k}\Big(\frac{\langle Z_1T_k^{\alpha}f ,X_{t-k-1}\rangle}{\|X_{t-k-1}\|}-\langle  Z_1T_k^{\alpha}f ,\varphi\rangle\Big).
    \end{align}
Therefore, for all $k\geq 0$, $t\geq k+1$ and $\epsilon> 0$,
\begin{align}
\label{eq: prevJ2}
&J_2(k,t,\epsilon)
	\\&= \mathbb P_\mu\big[|\langle Z_1(\theta_{t,k}T_k^{\alpha}f),X_{t-k-1}\rangle-e^{\alpha}m_k[f]|;A_{t-k-1}(\epsilon)\big]
    \\&=e^{-\alpha k}\mathbb{P}_{\mu}\bigg[\Big|\frac{\langle Z_1T^{\alpha}_k f,X_{t-k-1}\rangle}{\|X_{t-k-1}\|}-\langle  Z_1T^{\alpha}_k f,\varphi\rangle\Big|;A_{t-k-1}(\epsilon)\bigg]\nonumber\\
    &\leq \|\mu\|^{-1} e^{-\alpha k}e^{-(\alpha-\epsilon)(t-k-1)}e^{(\alpha-\kappa_f b)(1+\beta)k} \mathbb{P}_{\mu}\left[\left|\langle g_k,X_{t-k-1}\rangle\right|\right],
\end{align}
where
\[
g_k
	:= \frac{Z_1 T^{\alpha}_k f-\langle  Z_1T^{\alpha}_k f,\varphi\rangle}{e^{(\alpha-\kappa_f b)(1+\beta)k}},
	\quad k \in \mathbb Z_+.
\]
According to \cite[Lemma 2.8]{MM}, there exists $h \in \mathcal P$ such that
\[
     |T_t g_k|\leq e^{-bt}h,
     \quad k\geq 0,t\geq 0.
\]
 	This implies that
 \[
 	Q_1 (\operatorname{Re} g_k) \leq h
 	\text{ and } Q_1 (\operatorname{Im} g_k)\leq h,
 	\quad k \geq 0.
 \]
	Recall that $\gamma\in(0,\beta)$ is chosen small enough so that $\alpha \gamma < b < (1+\gamma)b$.
	According to Lemma \ref{lem: control moment}.(3) (with $\kappa=1$), there exists $C>0$ such that for all $k\geq 0$, $t\geq k+1$ and $\epsilon> 0$,
\begin{align}
    &\mathbb{P}_{\mu}\left[\left|\langle g_k,X_{t}\rangle\right|\right]
    \leq \|\langle \operatorname{Re} g_k, X_{t}\rangle\|_{\mathbb{P}_{\mu,1+\gamma}} + \|\langle \operatorname{Im} g_k, X_{t}\rangle\|_{\mathbb{P}_{\mu,1+\gamma}}
    \\& \leq 2\sup_{g\in \mathcal P: Q_1 g\leq h} \|\langle g, X_t\rangle\|_{\mathbb P_\mu; 1+\gamma} \leq C e^{\frac{\alpha t}{1+\gamma}}.
\end{align}
Therefore, according to \eqref{eq: prevJ2}, we have for all $k\geq 0$, $t\geq k+1$ and $\epsilon> 0$,
\begin{align}
	&J_2(k,t,\epsilon)
	\leq  \|\mu\|^{-1}e^{-\alpha k}e^{-(\alpha-\epsilon)(t-k-1)}e^{(\alpha-\kappa_f b)(1+\beta)k} \mathbb{P}_{\mu}\left[\left|\langle g_k,X_{t-k-1}\rangle\right|\right]
	\\&\leq \|\mu\|^{-1}C e^{-\alpha k}e^{-(\alpha-\epsilon)(t-k-1)}e^{(\alpha-\kappa_f b)(1+\beta)k} e^{\frac{\alpha}{1+\gamma}(t-k-1)}
	\\&= \|\mu\|^{-1}C e^{(\alpha \beta - \kappa_f b(1+\beta))k}e^{-(\frac{\alpha\gamma}{1+\gamma}-\epsilon)(t-k-1)}
	\\&\leq \|\mu\|^{-1}C e^{-(\frac{\alpha\gamma}{1+\gamma}-\epsilon)(t-k-1)},
\end{align}
	as desired.

	Step 4. We will show that for each $\epsilon\in (0,  \alpha]$ there exist $C,\delta>0$ such that
\begin{align}\label{ineq: control of J3}
    J_3(k,t,\epsilon)\leq Ce^{-\delta (t-k-1)}\quad k\geq0, t\geq k+1.
\end{align}
	In fact, we have
\begin{align}
	&\mathbb P_{\mu}(A_{t}(\epsilon), D) = \mathbb P_{\mu}[\mathbb P_{\mu}(D|\mathscr F_t);A_t(\epsilon)]
	\\&= \mathbb P_\mu[e^{-\alpha^*\|X_t\|};A_t(\epsilon)]
	\leq \exp({-\alpha^* e^{(\alpha - \epsilon)t}}),\quad t\geq 0, \epsilon >0.
\end{align}
According to Lemma \ref{lem: control of XT}, there is a constant $C>0$ such that
\begin{align}
	\mathbb P_\mu(A_t(\epsilon)^c,D^c) = \mathbb P_\mu(D^c) \tilde{\mathbb P}_\mu(\|X_t\|\leq e^{(\alpha - \epsilon) t}) \leq \mathbb P_\mu(D^c)C e^{-\epsilon \beta t},\quad t\geq 0, \epsilon >0.
\end{align}
	Combining these, we get the desired result in step 4.

	Step 5.  Combining the results in Step 1-4, noticing that, if $\epsilon>0$ is chosen small enough then $J_{i}, i = 1,2,3$ converge to $0$ exponentially fast as $t\rightarrow\infty$, 
we immediately get the desired result.
\end{proof}

\begin{cor}\label{cor: used in next corollary}
Let $f\in \mathcal{P}$ and $\Theta >0$.
Assume that $\alpha\beta\leq \kappa_fb(1+\beta)$.
Then there exist $C>0$ and $\delta>0$ such that for all $k \geq 0, t\geq k+1$ and $|\theta|\leq \Theta$,
\begin{align}
    \mathbb{P}_{\mu}\Big[\big|\mathbb{P}_{\mu}[e^{i\theta\gamma_{t,k}}-e^{e^{\alpha}m_k[\theta f]}; D^c | \mathscr F_{t-k-1}]\big|\Big]\leq Ce^{-\delta(t-k-1)}.
\end{align}
\end{cor}
\begin{proof}
	Define $J_1^f(t,k,\epsilon), J_2^f(t,k,\epsilon)$ and $J_3(t,k,\epsilon)$ as $J_1, J_2$ and $J_3$ in \eqref{eq: Def of Ji}.
	Fix $\mu \in \mathcal M_c(\mathbb R^d)$ and $f\in \mathcal P$.
    According to \eqref{eq: inequality that will used later},  we have for all $\theta\in \mathbb R$, $k\geq 0$, $t\geq k+1$ and $\epsilon> 0$,
\begin{align}
    &\mathbb{P}_{\mu}\Big[\big| \mathbb{P}_{\mu}[e^{i\theta \gamma_{t,k}}-e^{e^{\alpha}m_k[\theta f]}; D^c | \mathscr F_{t-k-1}]\big|\Big]
    \\&\leq J^{\theta f}_1(t,k,\epsilon) + J^{\theta f}_2(t,k,\epsilon)+J_3(t,k,\epsilon).
\end{align}
	According to \eqref{eq: estimate of J1}, we have for all $\theta\in \mathbb R$, $k\geq 0$, $t\geq k+1$ and $\epsilon> 0$,
\begin{align}
	J^{\theta f}_1(k,t,\epsilon)
     & \leq \|\mu\|^{-\frac{1+2\beta}{1+\beta}} \langle Q_0 SQ (\theta f), \mu \rangle \exp\Big\{-\Big(\alpha\frac{\beta}{1+\beta}-\epsilon\frac{1+2\beta}{1+\beta}\Big)(t-k-1)\Big\}
     \\& = |\theta|^{1+2\beta} \|\mu\|^{-\frac{1+2\beta}{1+\beta}} \langle Q_0 SQf, \mu \rangle \exp\Big\{-\Big(\alpha\frac{\beta}{1+\beta}-\epsilon\frac{1+2\beta}{1+\beta}\Big)(t-k-1)\Big\},
\end{align}
	where $(R,S)$ is the $(1+2\beta)$-control-pair in Lemma \ref{lem: upper bound for usgx}.(6).
	From the definitions of $Z_1$ and $m_t$ we get that
\[
	Z_1( \pm \theta g) = \theta^{1+\beta} Z_1(\pm g), \quad m_t[\pm \theta g] = \theta^{1+\beta} m_t[\pm g],\quad g\in \mathcal P, \theta \geq 0, t\geq 0.
\]
	Therefore, we have
\[
J^{\pm \theta f}_2(t,k,\epsilon)
	:= \theta^{1+\beta} J_2^{\pm f}(t,k,\epsilon), \quad \theta >0, k \geq 0, t\geq k+1, \epsilon > 0.
\]
	According to this and Step 3 in the proof of the previous lemma, we have that there exists a constant $C > 0$ such that for all $\theta\in \mathbb R$, $k\geq 0$, $t\geq k+1$ and $\epsilon> 0$,
\begin{align}
\label{eq:31step3b}
    J^{\theta f}_2(t,k,\epsilon)
    \leq C |\theta|^{1+\beta}\exp\Big(-(\frac{\alpha\gamma}{1+\gamma}-\epsilon)(t-k-1)\Big).
\end{align}
	Finally, using the estimates of $J^{\theta f}_{i}, i = 1,2$ and the estimate of $J_3$ in Step 4 of the proof of the previous lemma, we get the desired result by choosing $\epsilon$ small enough.
\end{proof}

\begin{prop}\label{corollary31}
Let $f\in \mathcal{P}$ and $\Theta>0$. Assume that $\alpha\beta\leq\kappa_fb(1+\beta)$. Then there exist $C,\delta>0$ such that for all $T\geq 0$, $n \in \{0,...,\lfloor t \rfloor\}$ and $(\theta_0,...,\theta_n)\in \mathbb{R}^{n+1}$ satisfying $|\theta_i|\leq \Theta$, we have
\begin{align}
\label{32corollary}
    \Big|\mathbb{\tilde{P}}_{\mu}\Big[\prod_{k=0}^n\exp\Big(i\theta_k \frac {\mathcal I_{t-k-1}^{t-k}\langle f ,X_t\rangle}{(e^{\alpha k}\|X_{t-k-1}\|)^\frac{1}{1+\beta}}\Big)-\prod_{k=0}^n\exp(e^{\alpha}m_k[\theta_k f])\Big]\Big|\leq C e^{-\delta(t-n)}.
\end{align}
\end{prop}
\begin{proof}
	Let $C,\delta > 0$ be the constants in Corollary \ref{cor: used in next corollary}.
    Recall that \[\gamma_{t,k}:=\frac {\mathcal I_{t-k-1}^{t-k}\langle f ,X_t\rangle}{(e^{\alpha k}\|X_{t-k-1}\|)^\frac{1}{1+\beta}},\quad k \geq 0, t\geq k+1. \]
    Fix $t\geq 0$, $n \in \{0,...,\lfloor t \rfloor\}$ and $(\theta_0,...,\theta_n)\in \mathbb{R}^{n+1}$ satisfying $|\theta_i|\leq \Theta$.
    For each $k\in\{-1,...,n\}$, we define
    \[a_k:=\prod_{l=0}^{k}\exp(e^{\alpha}m_l[\theta_lf])\mathbb{\tilde{P}}_{\mu}\Big(\prod_{l=k+1}^{n}\exp\left(i\theta_l\gamma_{t,l}\right)\Big)\],
     where by convention the product is $1$ for $k=-1$. Then we get for each $k > -1$,
    \begin{align*}
        &a_{k-1} - a_k
        \\&=\mathbb{P}_{\mu}(D^c)^{-1}\prod_{l=0}^{k-1}e^{e^{\alpha}m_l[\theta_l f]}\mathbb{P}_{\mu}\Big[(e^{i\theta_{k}\gamma_{t,k}}-e^{e^{\alpha}m_k[\theta_k f]})\prod_{l=k+1}^ne^{i\theta_{l}\gamma_{t,l}};D^c\Big]
        \\&=\mathbb{P}_{\mu}(D^c)^{-1}\prod_{l=0}^{k-1}e^{e^{\alpha}m_l[\theta_l f]}\mathbb{P}_{\mu}\Big[\mathbb P_\mu[e^{i\theta_{k}\gamma_{t,k}}-e^{e^{\alpha}m_k[\theta_k f]}; D^c|\mathscr F_{t-k-1}]\prod_{l=k+1}^ne^{i\theta_{l}\gamma_{t,l}}\Big].
    \end{align*}
    According to Corollary \ref{cor: alpha stable rv} and Corollary \ref{cor: used in next corollary}, there exist $C,\delta>0$ such that for any $k\in\{0,1,...,n\}$ we have
    \begin{align*}
        &|a_{k-1}- a_k|
        \\&\leq \frac{1}{\mathbb{P}_{\mu}(D^c)}\mathbb{P}_{\mu}\Big[\big|\mathbb P_\mu[e^{i\theta_{k}\gamma_{t,k}}-e^{e^{\alpha}m_k[\theta_k f]}; D^c\big|\mathscr{F}_{t-k-1}]\big|\Big]
        \\& \leq C e^{-\delta(t-k-1)}.
    \end{align*}
Therefore,
\begin{align}
    \text{LHS of \eqref{32corollary}}&= \left|a_{-1}-a_n\right|
    \leq\sum_{k=0}^n\left|a_{k-1}-a_k\right|
    \leq \sum_{k=0}^n C e^{-\delta(t-k-1)}.
\end{align}
	Recall that $C, \delta>0$ are independent of the choice of $t\geq 0$, $n \in \{0,...,\lfloor t \rfloor\}$ and $(\theta_0,...,\theta_n)\in \mathbb{R}^{n+1}$ with $|\theta_i|\leq \Theta$.
\end{proof}
\subsection{Proof of Theorem \ref{Theorem12}}
\label{sec: proof of thm 1.3}
    Fix $f\in\mathcal{P}$. 
    Chose $t_0 > 0$ large enough so that $\lceil t_0-\ln t_0\rceil \leq \lfloor t_0 \rfloor - 1.$
    Write
    \begin{align*}
        &(t\|X_t\|)^{-\frac{1}{1+\beta}}\langle f,X_t\rangle
        \\ &=\sum_{k=0}^{\lfloor t-\ln t \rfloor} \frac{\mathcal I_{t-k-1}^{t-k}\langle f ,X_t\rangle}{(t\|X_t\|)^{\frac{1}{1+\beta}}}+\Big(\sum_{k=\lceil t-\ln t \rceil}^{\lfloor t \rfloor-1} \frac{\mathcal I_{t-k-1}^{t-k}\langle f ,X_t\rangle}{(t\|X_t\|)^{\frac{1}{1+\beta}}}+\frac{\mathcal I_0^{t-\lfloor t \rfloor}\langle f ,X_t\rangle}{(t\|X_t\|)^{\frac{1}{1+\beta}}}\Big) + \frac{\langle P^\alpha_tf,X_0\rangle}{(t\|X_t\|)^{\frac{1}{1+\beta}}}.
        \\&=:I_1(t)+I_2(t) + I_3(t), 
        \quad t\geq t_0.
    \end{align*}
    Define
\[
    \tilde I_1(t)
    :=\sum_{k=0}^{\lfloor t-\ln t \rfloor}\frac{\mathcal I_{t-k-1}^{t-k}\langle f ,X_t\rangle}{(t e^{\alpha(k+1)}\|X_{t-k-1}\|)^{\frac{1}{1+\beta}}}, 
    \quad t\geq t_0.
\]
    Fix $\mu \in \mathcal M_c(\mathbb R^d)$ and $\theta\in \mathbb{R}$. Taking $\theta_k=(t e^{\alpha})^{-\frac{1}{1+\beta}} \theta $ and $n={\lfloor t-\ln t \rfloor}$ in Corollary \ref{corollary31}, we get that there exist $C,\delta>0$ such that,
\begin{align*}
        \Big|\mathbb{\tilde{P}}_{\mu} [e^{i\theta\tilde{I}_1(t)}]-\exp\Big(\frac{1}{t}\sum_{k=0}^{\lfloor t-\ln t \rfloor}m_k[\theta f]\Big)\Big|\leq C \frac{1}{t^{\delta}},
        \quad t\geq t_0.
\end{align*}
    According to \eqref{para: critical case},  $\tilde{I}_1(t)\xrightarrow[t\to \infty]{d}\eta_1$ under $\tilde {\mathbb P}_\mu$.
    Therefore, we only need to prove $|\mathbb{\tilde{P}}_{\mu}[e^{i\theta I_1(t)}]-\mathbb{\tilde{P}}_{\mu}[e^{i\theta\tilde{I}_1(t)}]|\xrightarrow[t\to \infty]{} 0$ and $I_i(t)\xrightarrow[t\to \infty]{d} 0$ for $i = 2,3$ under $\tilde {\mathbb P}_\mu$.

By \cite[Lemma 3.4.3.]{Durrett2010Probability},
\begin{align}\label{ineq: control of I1t}
    |\mathbb{\tilde{P}}_{\mu}[e^{i\theta I_1(t)}] - \mathbb{\tilde{P}}_{\mu} [e^{i\theta\tilde{I}_1(t)}]|
    \leq \sum_{k=0}^{\lfloor t-\ln t \rfloor}\mathbb{\tilde{P}}_{\mu}\big[|Y_{t,k}|\big],
    \quad t\geq t_0,
\end{align}
    where for all $k \geq 0$ and $t\geq k+1$,
\begin{align*}
    Y_{t,k}
    :=\exp\Big(i\theta\frac{\mathcal I_{t-k-1}^{t-k}\langle f ,X_t\rangle}{(t e^{\alpha(k+1)}\|X_{t-k-1}\|)^{\frac{1}{1+\beta}}}\Big)-\exp\Big(i\theta\frac{\mathcal I_{t-k-1}^{t-k}\langle f ,X_t\rangle}{(t\|X_t\|)^{\frac{1}{1+\beta}}}\Big).
\end{align*}
    Let $\gamma \in (0,\beta)$ be close enough to $\beta$ such that
\[
    \frac{\alpha \gamma}{1+\gamma} > \frac{\alpha}{1+\gamma} - \frac{\alpha}{1+\beta} > 0.
\]
    Fix this $\gamma$, then chose $\eta_0,\eta_1>0$ such that
\[
    \frac{\alpha \gamma}{1+\gamma} >\eta_0 > \eta_0 - 3\eta_1 > \frac{\alpha}{1+\gamma} - \frac{\alpha}{1+\beta} > 0.
\]
    Define for all $k \geq 0$ and $t\geq k+1$,
\begin{align}
\label{def: Dtk}
    %\mathcal{D}_{t,k}&:=\left\{|H_t-H_{t-k-1}|\leq \|\mu\| e^{-\frac{\alpha}{4}(t-k-1)}, H_{t-k-1}>\|\mu\|e^{-\frac{\alpha}{32}(t-k-1)}\right\}.
    \mathcal{D}_{t,k}&:=\left\{|H_t-H_{t-k-1}|\leq  e^{-\eta_0 (t-k-1)}, H_{t-k-1}> 2e^{-\eta_1(t-k-1)}\right\}.
\end{align}

    Step 1. We will show that there exists a $C,\delta >0$ such that for all $k \geq 0$ and $t\geq k+1$,
\begin{align}
\label{thm121}
    \mathbb{\tilde{P}}_{\mu}\big[|Y_{t,k}|\mathbf{1}_{\mathcal{D}^c_{t,k}}\big]
    \leq C e^{-\delta (t-k)}.
\end{align}
    In fact, let $C_1$ be the constant in  Lemma \ref{lem: control of Wt} corresponding to the fixed $\gamma$ and $C_2$ be the constant in Lemma \ref{lem: control of XT}. 
    By Chebyshev's inequality, there exists $C>0$ such that for all $k \geq 0$ and $t\geq k+1$,
\begin{align}
\label{eq: prob of Dtkc}
    &\mathbb{\tilde{P}}_{\mu}(\mathcal{D}_{t,k}^c)
    \\&\leq \mathbb{\tilde{P}}_{\mu}(|H_t-H_{t-k-1}| > e^{-\eta_0 (t-k-1)})+\mathbb{\tilde{P}}_{\mu}(H_{t-k-1}\leq 2e^{-\eta_1(t-k-1)}),
    \\&\leq \mathbb{P}_{\mu}(D^c)^{-1}e^{\eta_0(t-k-1)}\mathbb{P}_{\mu}[|H_t-H_{t-k-1}|]+\mathbb{\tilde{P}}(H_{t-k-1}\leq 2e^{-\eta_1(t-k-1)})
    \\&\leq \mathbb{P}_{\mu}(D^c)^{-1}  e^{\eta_0(t-k-1)}\|H_t - H_{t-k-1}\|_{\mathbb P_\mu; 1+\gamma}+\mathbb{\tilde{P}}(H_{t-k-1}\leq 2e^{-\eta_1(t-k-1)})
    \\&\leq C_1  \mathbb{P}_{\mu}(D^c)^{-1}  e^{-(\frac{\alpha \gamma}{1+\gamma} - \eta_0)(t-k-1)}+C_2 e^{-\eta_1\beta(t-k-1)}.
\end{align}
    This implies the desired result in Step 1, since $|Y_{t,k}| \leq 2$ a.s..

    Step 2. We will show that there exists constant $C,\delta > 0$ such that for all $k\geq 0$ and $t\geq k+1$,
\begin{align}
\label{thm122}
     \mathbb{\tilde{P}}_{\mu}\big[|Y_{t,k}|\mathbf{1}_{\mathcal{D}_{t,k}}\big]
     \leq  C e^{-\delta (t-k)}.
\end{align}
    In fact, since $|e^{ix}-e^{iy}|\leq|x-y|$ for all $x,y\in \mathbb R$, we have for all $k \geq 0$ and $t\geq k+1$,
\begin{align}
\label{eq: control of Ykt}
        &\mathbb{\tilde{P}}_{\mu}\big[|Y_{t,k}|\mathbf{1}_{\mathcal{D}_{t,k}}\big]
        \\&\leq|\theta|t^{-\frac{1}{1+\beta}} \mathbb{\tilde{P}}_{\mu}\bigg[|\mathcal I_{t-k-1}^{t-k}\langle f ,X_t\rangle|\cdot\Big|\frac{1}{(e^{\alpha(k+1)}\|X_{t-k-1}\|)^{\frac{1}{1+\beta}}}-\frac{1}{\|X_t\|^{\frac{1}{1+\beta}}}\Big|\mathbf{1}_{\mathcal{D}_{t,k}}\bigg]
        \\&\leq |\theta| e^{-\frac{\alpha}{1+\beta}t}\mathbb{\tilde{P}}_{\mu}\big[|\mathcal I_{t-k-1}^{t-k}\langle f ,X_t\rangle|\cdot K_{t,k}\big],
\end{align}
    where
\begin{equation}
\label{def: Ktk}
    K_{t,k}
    :=\Big|\frac{H_t^{\frac{1}{1+\beta}}-H_{t-k-1}^{\frac{1}{1+\beta}}}{H_t^{\frac{1}{1+\beta}}H_{t-k-1}^{\frac{1}{1+\beta}}}\Big|\mathbf{1}_{\mathcal{D}_{t,k}}.
\end{equation}
    Note that, since $\eta_1 < \eta_0$, we have
\begin{align*}
    H_t
    &\geq H_{t-k-1}- e^{-\eta_0(t-k-1)}
    \geq 2e^{-\eta_1(t-k-1)}-e^{-\eta_0(t-k-1)}
    \\&\geq e^{-\eta_1(t-k-1)},
    \quad \text{ on } \mathcal D_{t,k}.
\end{align*}
    Therefore, for each $k \geq 0$ and $t\geq k+1$,
\begin{align*}
     \Big|H_t^{\frac{1}{1+\beta}}-H_{t-k-1}^{\frac{1}{1+\beta}}\Big|
     &\leq \frac{1}{1+\beta}\max \Big\{H_t^{-\frac{\beta}{1+\beta}},H_{t-k-1}^{-\frac{\beta}{1+\beta}}\Big\}\left|H_t-H_{t-k-1}\right|,
    \\&\leq \frac{1}{1+\beta} \max\{e^{\eta_1 (t-k-1)}, \frac{1}{2}e^{\eta_1(t-k-1)}\}^{\frac{\beta}{1+\beta}}e^{-\eta_0(t-k-1)}
    \\&\leq \frac{1}{1+\beta} e^{\eta_1 (t-k-1)} e^{-\eta_0(t-k-1)}
    =\frac{1}{1+\beta}  e^{-(\eta_0 - \eta_1)(t-k-1)},
    \quad \text{ on } \mathcal D_{t,k}, 
\end{align*}
    and
\begin{align*}
    |H_t^{\frac{1}{1+\beta}}H_{t-k-1}^{\frac{1}{1+\beta}}|
    \geq 2^{\frac{1}{1+\beta}} e^{-2\eta_1(t-k-1)},
    \quad \text{ on } \mathcal D_{t,k}.
\end{align*}
    Thus, there is a constant $C\geq 0$ such that,
\begin{align}
\label{ineq: control of Kkt}
     K_{t,k}
     \leq C e^{-(\eta_0 - 3\eta_1)(t-k-1)},
     \quad k \geq 0, t\geq k+1.
\end{align}
    According to Lemma \ref{lem: control of mgtrs}, \eqref{eq: control of Ykt}, \eqref{ineq: control of Kkt}, there exist constants $C,C'>0$ such that  for all $k\geq 0$ and $t\geq k+1$,
\begin{align}
\label{eq: Y in D}
    &\mathbb{\tilde{P}}_{\mu}\big[|Y_{t,k}|\mathbf{1}_{\mathcal{D}_{t,k}}\big]
    \leq C|\theta|e^{-\frac{\alpha}{1+\beta}t}\mathbb{\tilde{P}}_{\mu}\big[|\mathcal{I}_{t-k-1}^{t-k}\langle f,X_t\rangle|\big]e^{-(\eta_0 - 3\eta_1)(t-k-1)}
    \\&\leq \frac{C}{\mathbb{P}_{\mu}(D^c)}|\theta|e^{-\frac{\alpha}{1+\beta}t}\mathbb{P}_{\mu}\big[|\mathcal{I}_{t-k-1}^{t-k}\langle f,X_t\rangle|\big]e^{-(\eta_0 - 3\eta_1)(t-k-1)}
    \\&\leq \frac{C}{\mathbb{P}_{\mu}(D^c)}|\theta|e^{-\frac{\alpha}{1+\beta}t}\|\mathcal{I}_{t-k-1}^{t-k}\langle f,X_t\rangle\|_{\mathbb P_\mu; 1+\gamma} e^{-(\eta_0 - 3\eta_1)(t-k - 1)}
    \\&\leq C' e^{-\frac{\alpha}{1+\beta}t}e^{\frac{\alpha}{1+\gamma}t}e^{\frac{\gamma \alpha-\kappa_f(1+\gamma)b}{1+\gamma}k}e^{-(\eta_0 - 3\eta_1)(t-k)}\\&= C' e^{(\frac{\alpha}{1+\gamma}-\frac{\alpha}{1+\beta})(t-k)}e^{-(\eta_0 - 3\eta_1)(t-k-1)},
\end{align}
    as required in step 2. 
    In the last equality, we used the fact that
\[
    -(\frac{\alpha}{1+\gamma}-\frac{\alpha}{1+\beta})
    = \alpha(1-\frac{1}{1+\gamma}) - \alpha(1-\frac{1}{1+\beta})
    = \frac{\gamma \alpha}{1+\gamma} - k_f b
    =\frac{\alpha \gamma-\kappa_f(1+\gamma)b}{1+\gamma}.
\]

    Step 3. We will show that there exist $C, \delta> 0$ such that for $t$ large enough,
\[
    |\mathbb{\tilde{P}}_{\mu}[e^{i\theta I_1(t)}] - \mathbb{\tilde{P}}_{\mu}[e^{i\theta\tilde{I}_1(t)}]|
    \leq C t^{-\delta}.
\]
    In fact, according to \eqref{thm121} and \eqref{thm122}, there exist $C,\delta > 0$ such that
    for all $k \geq 0$ and $t\geq k+1$ we have,
\[
    \tilde{\mathbb P}_\mu\big[|Y_{t,k}|\big] 
    \leq C e^{-\delta(t-k)}.
\]
    Therefore, there exist $C, \delta > 0$ and $C'$ such that for all $t \geq 1$,
\begin{align}
    &\sum_{k=0}^{\lfloor t-\ln t \rfloor} \tilde {\mathbb P}_\mu\big[|Y_{t,k}|\big]
    \leq C\sum_{k=0}^{\lfloor t-\ln t \rfloor} e^{-\delta(t-k)}
    = C e^{-\delta t}\sum_{k=0}^{\lfloor t-\ln t \rfloor} e^{\delta k} 
    \\&\leq C' e^{-\delta t}e^{\delta (t-\ln t)} 
    = C'\frac{1}{t^{\delta}},
\end{align}
    as required in this step.

    Step 4. We will show that $I_2(t) \xrightarrow[t\to \infty]{d} 0$ with respect to $\mathbb{\tilde{P}}_{\mu}$.
    In fact, let $\mathcal{E}_t:=\{\|X_t\|>t^{-1/2}e^{\alpha t}\}$. According to Lemma \ref{lem: control of XT}, there exists $C>0$ such that
\begin{align}
    \mathbb{\tilde{P}}_{\mu}(\mathcal{E}^c_t)\leq C t^{-\frac{\beta}{2}}, \quad t\geq0.
\end{align}
    Therefore, there exists $C>0$ such that
\begin{align}\label{Theorem123}
    |\mathbb{\tilde{P}}_{\mu}[(e^{i\theta I_2(t)}-1)\mathbf{1}_{\mathcal{E}^c_t}]|
    \leq 2\mathbb{\tilde{P}}_{\mu}(\mathcal{E}^c_t)\leq Ct^{-\frac{\beta}{2}},
    \quad t\geq t_0.
\end{align}
    Chose a $\gamma\in (0,\beta)$ close enough to $\beta$ such that $\alpha(\frac{1}{1+\gamma}-\frac{1}{1+\beta})\leq \frac{1}{2(1+\beta)}$.
	According to Lemma \ref{lem: control of mgtrs}, there exist $C,C',C''>0$ such that
\begin{align*}
    &|\mathbb{\tilde{P}}_{\mu} [ (e^{i\theta I_2(t)}-1)\mathbf{1}_{\mathcal{E}_t}]|
    \leq |\theta| \mathbb{\tilde{P}}_{\mu} \big[ |I_2(t)|\mathbf{1}_{\mathcal{E}_t}\big]
    \\&\leq|\theta| t^{-\frac{1}{2(1+\beta)}}e^{-\frac{\alpha}{1+\beta}t}\Big(\sum_{k=\lceil t-\ln t \rceil}^{\lfloor t \rfloor - 1}\mathbb{\tilde{P}}_{\mu}\big[| \mathcal{I}_{t-k-1}^{t-k}\langle f,X_t\rangle|\big] + \mathbb{\tilde{P}}_{\mu}\big[| \mathcal{I}_{0}^{t-\lfloor t\rfloor}\langle f,X_t\rangle|\big]\Big)
    \\& \leq C |\theta| t^{-\frac{1}{2(1+\beta)}}e^{-\frac{\alpha}{1+\beta}t}\Big(\sum_{k=\lceil t-\ln t \rceil}^{\lfloor t \rfloor - 1}\|\mathcal{I}_{t-k-1}^{t-k}\langle f,X_t\rangle\|_{\mathbb P_\mu; 1+\gamma} + \|\mathcal I_0^{t-\lfloor t \rfloor} \langle f, X_t\rangle\|_{\mathbb P_\mu;1+\gamma}\Big)
    \\ &\leq C' |\theta| t^{-\frac{1}{2(1+\beta)}}e^{-\frac{\alpha}{1+\beta}t}\sum_{k=\lceil t-\ln t \rceil}^{\lfloor t \rfloor}e^{\frac{\alpha}{1+\gamma}t}e^{\frac{\alpha\gamma-\kappa_f(1+\gamma)b}{1+\gamma}k}\\
    &= C' |\theta| t^{-\frac{1}{2(1+\beta)}}e^{(\frac{\alpha }{1+\gamma}-\frac{\alpha }{1+\beta})t} \sum_{k=\lceil t-\ln t \rceil}^{\lfloor t \rfloor}e^{-(\frac{\alpha}{1+\gamma}-\frac{\alpha}{1+\beta})k}\\
    &\leq C' |\theta| t^{-\frac{1}{2(1+\beta)}}e^{(\frac{\alpha }{1+\gamma}-\frac{\alpha }{1+\beta})(t - \lceil t - \ln t\rceil)} \sum_{j=0}^{\infty}e^{-(\frac{\alpha}{1+\gamma}-\frac{\alpha}{1+\beta})j}\\
    &\leq C''|\theta| t^{-\frac{1}{2(1+\beta)}}t^{(\frac{\alpha}{1+\gamma}- \frac{\alpha}{1+\beta})},
    \quad t\geq t_0.
\end{align*}
	From this and \eqref{Theorem123}, we get the required result in this step.

	Step 5. We will show that $I_3(t) \xrightarrow[t\to \infty]{\tilde {\mathbb P}_\mu \text{-} a.s.} 0$. 
	In fact, we can write
\begin{align}
	&|I_3(t)| 
	\leq \frac{\langle |P^\alpha_tf|,X_0\rangle}{(t\|X_t\|)^{\frac{1}{1+\beta}}}
	\leq \frac{\langle e^{\alpha t - \kappa_f b t}Qf,X_0\rangle}{(te^{\alpha t} H_t)^{\frac{1}{1+\beta}}}
	\\& = t^{-\frac{1}{1+\beta}} e^{\frac{\beta \alpha t}{1+\beta} - k_fbt} H_t^{-\frac{1}{1+\beta}} \langle Qf,X_0\rangle
	\\& = t^{-\frac{1}{1+\beta}} H_t^{-\frac{1}{1+\beta}} \langle Qf,X_0\rangle
	\xrightarrow[t\to \infty]{\tilde {\mathbb P}_\mu \text{-} a.s.} 0.
\end{align}

	Finally, combining Steps 3--5, we complete the proof of Theorem \ref{Theorem12}.

\subsection{Proof of Theorem \ref{Theorem11}}
	Fix  $f \in \mathcal P$ such that $\alpha \beta > \kappa_f b (1+\beta)$.
	Fix $\mu \in \mathcal M_c(\mathbb R^d)$.
	Write
\begin{align}
    f
    =\sum_{p\in \mathbb Z_+^d:|p|\geq \kappa_f}\langle f,\phi_p\rangle_\varphi \phi_p
    %=
    =:
    \sum_{p\in \mathbb Z_+^d:|p|= \kappa_f}\langle f,\phi_p\rangle_\varphi \phi_p+\tilde{f}.
\end{align}
	Then
\begin{align*}
    &e^{-(\alpha-\kappa_fb)t}\langle f,X_t\rangle=
    \sum_{p\in \mathbb Z_+^d:|p|= \kappa_f}\langle f,\phi_p\rangle_\varphi H_t^p+e^{-(\alpha-\kappa_fb)t} \langle \tilde{f},X_t\rangle,
    \quad t\geq 0.
\end{align*}
	According to Lemma \ref{lemma26}, 
	we have
\begin{align}
\label{as convergence}
     \sum_{|p|= \kappa_f}\langle f,\phi_p\rangle_\varphi H_t^p
     \xrightarrow[t\to \infty]{} \sum_{|p|=\kappa_f}\langle f, \phi_p\rangle_{\varphi} H_{\infty}^p,
\end{align}
$\mathbb{P}_{\mu}$-a.s. and in $L^{1+\gamma}(\mathbb{P}_{\mu})$ for any $\gamma\in(0,\beta)$.
	Therefore, it is suffice to show that
\begin{align}
    J_t
    :=e^{-(\alpha-\kappa_fb)t}\langle \tilde{f},X_t\rangle,
    \quad t\geq 0,
\end{align}
	converges in $L^{1+\gamma}(\mathbb{P}_{\mu})$ for any $\gamma\in(0,\beta)$, and converges a.s. provided $f\in C^2(\mathbb R^d)$.

	Step 1. Let $g\in \mathcal P$. 
	Let $a > 0$ be such that $a< \kappa_g$ and $a < \frac{\alpha \beta}{b(1+\beta)}$. 
	We will show that for each $\gamma\in (0,\beta)$ there exist constants $C,\delta > 0$ such that
\[
	\|e^{-(\alpha - ab)t} \langle g, X_t\rangle\|_{\mathbb P_\mu;1+\gamma} 
	\leq C e^{-\delta t},
	\quad t\geq 0.
\] 
	In order to do this, we choice a $\gamma_0 \in (0,\beta)$ close enough to $\beta$ such that, for each $\gamma \in [\gamma_0, \beta)$, we have $a < \frac{\alpha\gamma}{b(1+\gamma)}$.
	Then, according to Lemma \ref{lem: control moment}, we have, for each $\gamma \in (0,\beta)$,
\begin{enumerate}
\item 
	if $\gamma \in [\gamma_0, \beta)$ and $\alpha\gamma> \kappa_g (1+\gamma)b$, then there exists $C>0$ such that
\[
    \|e^{-(\alpha - ab)t} \langle g, X_t\rangle\|_{\mathbb P_\mu;1+\gamma}
    \leq C e^{-(\alpha-ab)t}e^{(\alpha-\kappa_g b)t}
    \leq C  e^{-(\kappa_g - a)bt},
    \quad t\geq 0;
\]
\item 
	if $\gamma \in [\gamma_0, \beta)$ and $\alpha\gamma=\kappa_g(1+\gamma)b$, then there exists $C>0$ such that
\[
     \|e^{-(\alpha - ab)t} \langle g, X_t\rangle\|_{\mathbb P_\mu;1+\gamma}
     \leq C t e^{-(\alpha - a b)t}e^{\frac{\alpha}{1+\gamma}t}
     = C t e^{-(\frac{\alpha \gamma}{1+\gamma} - ab)t},
     \quad t\geq 0;
\]
\item 
	if $\gamma \in [\gamma_0, \beta)$ and $\alpha\gamma < \kappa_g (1+\gamma)b$, then there exists $C>0$ such that
\[
    \|J_t\|_{\mathbb{P}_{\mu};1+\gamma}
    \leq C  e^{-(\alpha - a b)t}e^{\frac{\alpha}{1+\gamma}t}
     = C  e^{-(\frac{\alpha \gamma}{1+\gamma} - ab)t},
     \quad t\geq 0;
\]
\item
	if $\gamma \in (0,\gamma_0)$ then, thanks to (1)--(3) above, and the fact that $\|J_t\|_{\mathbb{P}_{\mu};1+\gamma} 
	\leq \|J_t\|_{\mathbb{P}_{\mu};1+\gamma_0}$, there exist $C, \delta >0$ such that
\[
	\|J_t\|_{\mathbb{P}_{\mu};1+\gamma} 
	\leq Ce^{-\delta t},
	\quad t\geq 0.
\]
\end{enumerate}
	Thus, the desired conclusion in this step is valid. 
	In particular, by taking $g = \tilde f$ and $a = \kappa_f$, we get that $J_t$ converges to $0$ in $L^{1+\gamma}(\mathbb{P}_{\mu})$ for any $\gamma\in(0,\beta)$.

	Step 2.
	Further assume that $f\in C^2(\mathbb R^d)$, we will show that $J_t$ converges to $0$ almost surely.
	Define
\begin{align}
	L_t^{g,a}:=\int_0^t e^{-(\alpha-ab)s}\langle (L+ab)g,X_s\rangle ds,
	\quad g\in \mathcal P\cap C^2(\mathbb R^d), a \geq 0, t\geq 0,
\end{align}
and
\begin{align}
    Y_t^{g,a}
    :=\int_0^t e^{-(\alpha-ab)s}|\langle (L+ab)g,X_s\rangle|ds, 
    \quad g\in \mathcal P\cap C^2(\mathbb R^d), a \geq 0, t\geq 0.
\end{align}
	Now choose $a_0 \in (\kappa_{f}, \kappa_f + 1)$ close enough to $\kappa_f$ so that $a_0 < \frac{\alpha \beta}{b(1+\beta)}$. 
	According to \eqref{defmartingale}, we have
\begin{align*}
    J_t
    =e^{-(a_0-\kappa_f)bt} (M_t^{\tilde{f}, a_0}+L_t^{\tilde{f}, a_0}),
    \quad t\geq 0.
\end{align*}
	So we only need to show that
\begin{align*}
    e^{-(a_0-\kappa_f)b t}M_t^{\tilde{f},a_0}
    \xrightarrow[t\to \infty]{} 0, 
    \quad e^{-(a_0-\kappa_f)b t}L_t^{\tilde{f},a_0}
    \xrightarrow[t\to \infty]{} 0 
    \quad \mathbb{P}_{\mu}\text{-a.s.}
\end{align*}
	Notice that $\kappa_{(L+a_0 b)\tilde{f}}=\kappa_{\tilde{f}}\geq \kappa_f+1 > a_0$.
	So according to Step 1, for an arbitrary fixed $\gamma\in (0,\beta)$, there exist $C, \delta>0$ such that for each $t\geq 0$,
\begin{align}
    \|e^{-(\alpha-a_0 b)t}\langle \tilde{f},X_t\rangle)\|_{\mathbb{P}_{\mu};1+\gamma}
    \leq C e^{-\delta t},
    \quad \|e^{-(\alpha-a_0 b)t}\langle L\tilde{f}+a_0 b\tilde{f},X_t\rangle\|_{\mathbb{P}_{\mu};1+\gamma}
    \leq C e^{-\delta t}.
\end{align}
	Now, by the triangle inequality, for each $t\geq 0$,
\begin{align*}
    &\|L_t^{\tilde{f},a_0}\|_{\mathbb{P}_{\mu};1+\gamma}
    \leq\|Y_t^{\tilde{f},a_0}\|_{\mathbb{P}_{\mu};1+\gamma}
    \\&\leq \int_0^t \|e^{-(\alpha-a_0 b)s}\langle L\tilde{f}+a_0 b\tilde{f},X_s\rangle\|_{\mathbb{P}_{\mu};1+\gamma}ds\leq C \int_0^t e^{-\delta s}ds\leq\frac{C}{\delta}.
\end{align*}
Since $Y_t^{\tilde{f},a_0}$ is increasing in $t$, so it converges to some finite random variable $Y_{\infty}^{\tilde{f},a_0}$ almost surely and in $L^{1+\gamma}(\mathbb{P}_{\mu})$.
	As a consequence, almost surely we have
\begin{align*}
    \lim_{t\rightarrow \infty}e^{-(a_0 - \kappa_f)bt}|L_t^{\tilde{f},a_0}|
    \leq  \lim_{t\rightarrow \infty}e^{-(a_0 - \kappa_f)bt}|Y_t^{\tilde{f},a_0}|=0.
\end{align*}
	On the other hand, the martingale $M_t^{\tilde{f},a}$ satisfies
\begin{align*}
    \|M_t^{\tilde{f},a_0}\|_{\mathbb{P}_{\mu};1+\gamma}\leq  \|e^{-(\alpha-a_0 b)t}\langle \tilde{f},X_t\rangle)\|_{\mathbb{P}_{\mu};1+\gamma}+\|L_t^{\tilde{f},a_0}\|_{\mathbb{P}_{\mu};1+\gamma}\leq C(e^{-\delta t}+\frac{1}{\delta}),\quad t\geq 0.
\end{align*}
	This implies that the martingale converges almost surely. 
	As a consequence, 
\[
	\lim_{t\rightarrow\infty} e^{-(a_0-\kappa_f)bt}M_t^{\tilde{f},a}
	=0,
	\quad \mathbb P_\mu\text{-a.s.}
\] 
	
	The proof is now complete.

\subsection{Proof of Theorem \ref{Theorem13}}
	Let $f\in \mathcal P$ be such that $\alpha \beta < \kappa_f b(1+\beta)$. 
	Let $t_0 > 1$ be large enough so that 
\[
	\lceil t - \ln t\rceil 
	\leq \lfloor t \rfloor - 1,
	\quad t\geq t_0.
\]
	Write
\begin{align*}
	&\frac{\langle f,X_t\rangle}{\|X_t\|^{\frac{1}{1+\beta}}}
	\\&=\sum_{k=0}^{\lfloor t-\ln t \rfloor} \frac{\mathcal I_{t-k-1}^{t-k}\langle f ,X_t\rangle}{\|X_t\|^{\frac{1}{1+\beta}}}+ \Big(\sum_{k=\lceil t-\ln t \rceil}^{\lfloor t \rfloor-1} \frac{\mathcal I_{t-k-1}^{t-k}\langle f ,X_t\rangle}{\|X_t\|^{\frac{1}{1+\beta}}}+\frac{\mathcal I_0^{t-\lfloor t \rfloor}\langle f ,X_t\rangle}{\|X_t\|^{\frac{1}{1+\beta}}}\Big) + \frac{\langle P_t^\alpha f, X_0\rangle}{\|X_t\|^{\frac{1}{1+\beta}}}
	\\&=:I'_1(t)+I'_2(t)+I'_3(t),
	\quad t\geq t_0.
\end{align*}
	Define
 \[
 	\tilde I'_1(t)
 	:=\sum_{k=0}^{\lfloor t-\ln t \rfloor}\frac{\mathcal I_{t-k-1}^{t-k}\langle f ,X_t\rangle}{( e^{\alpha(k+1)}\|X_{t-k-1}\|)^{\frac{1}{1+\beta}}},
 	\quad t > t_0,
 \]
    Let $\theta\in \mathbb{R}$. 
    Taking $\theta_k=(e^{\alpha})^{-\frac{1}{1+\beta}} \theta $ and $n={\lfloor t-\ln t \rfloor}$ in Corollary \ref{corollary31}, we get that there exist $C,\delta > 0$ such that
\begin{align*}
    \Big|\mathbb{\tilde{P}}_{\mu} [e^{i\theta\tilde I'_1(t)} ]-\exp\Big(\sum_{k=0}^{\lfloor t-\ln t \rfloor}m_k[\theta f]\Big)\Big|
    \leq C e^{-\delta(t - \lfloor t - \ln t\rfloor)}
    \leq C \frac{1}{t^{\delta}},
    \quad t\geq 0.
\end{align*}
    Hence, according to \eqref{msmallcase}, we have $\tilde I'_1(t)\xrightarrow[t\to \infty]{d} \eta_2$ under $\tilde {\mathbb P}_\mu$.
    So we only need to prove that $|\mathbb{\tilde{P}}_{\mu}[e^{i\theta I'_1(t)}]-\mathbb{\tilde{P}}_{\mu}[e^{i\theta\tilde I'_1(t)}]|\xrightarrow[t\to \infty]{} 0$ and $I'_i(t)\xrightarrow[t\to \infty]{d} 0,~i=2,3,$ under $\tilde {\mathbb P}_\mu$.

    Step 1. We will show that $|\mathbb{\tilde{P}}_{\mu}[e^{i\theta I'_1(t)}]-\mathbb{\tilde{P}}_{\mu}[e^{i\theta\tilde I'_1(t)}]|\xrightarrow[t\to \infty]{} 0$.
    Define for $k\geq 0$ and $t\geq k+1$,
\begin{align*}
    Y'_{t,k}
    :=\exp\Big(i\theta\frac{\mathcal I_{t-k-1}^{t-k}\langle f ,X_t\rangle}{( e^{\alpha(k+1)}\|X_{t-k-1}\|)^{\frac{1}{1+\beta}}}\Big)-\exp\Big(i\theta\frac{\mathcal I_{t-k-1}^{t-k}\langle f ,X_t\rangle}{\|X_t\|^{\frac{1}{1+\beta}}}\Big).
\end{align*}
	Then we have 
\begin{align}
\label{ineq: control of I1tb}
    |\mathbb{\tilde{P}}_{\mu}[e^{i\theta I'_1(t)}] - \mathbb{\tilde{P}}_{\mu} [e^{i\theta\tilde{I}'_1(t)}]|
    \leq \sum_{k=0}^{\lfloor t-\ln t \rfloor}\mathbb{\tilde{P}}_{\mu}\big[|Y'_{t,k}|\big],
    \quad t\geq t_0,
\end{align}
    Let $\gamma \in (0,\beta)$ be close enough to $\beta$ such that
\[
    \frac{\alpha \gamma}{1+\gamma} > \frac{\alpha}{1+\gamma} - \frac{\alpha}{1+\beta} > 0.
\]
    Fix this $\gamma$, then chose $\eta_0,\eta_1>0$ such that
\[
    \frac{\alpha \gamma}{1+\gamma} >\eta_0 > \eta_0 - 3\eta_1 > \frac{\alpha}{1+\gamma} - \frac{\alpha}{1+\beta} > 0.
\]
	Define $\mathcal D_{t,k}$ and $K_{t,k}$ as in \eqref{def: Dtk} and \eqref{def: Ktk} respectively.
	Using Lemma \ref{lem: control of mgtrs}, \eqref{eq: prob of Dtkc}, \eqref{ineq: control of Kkt} and an argument similar to that used in proving \eqref{eq: control of Ykt}, we get that there exist $C,C',\delta>0$ such that for $k\geq 0$ and $t\geq k+1$,
\begin{align*}
    &\mathbb{\tilde{P}}_{\mu}\big[|Y'_{t,k}|\big]
    = \mathbb{\tilde{P}}_{\mu}\big[|Y'_{t,k}|; \mathcal D_{t,k}\big] + \mathbb{\tilde{P}}_{\mu}\big[|Y'_{t,k}|; \mathcal D_{t,k}^c \big]
    \\& \leq |\theta| e^{-\frac{\alpha}{1+\beta} t}\mathbb{\tilde{P}}_{\mu}\big[|\mathcal I_{t-k-1}^{t-k}\langle f ,X_t\rangle|\cdot K_{t,k}\big] + 2\mathbb{\tilde{P}}_{\mu}( \mathcal D_{t,k}^c )
    \\& \leq C e^{-\frac{\alpha}{1+\beta} t} \|\mathcal I_{t-k-1}^{t-k}\langle f, X_t\rangle \|_{\mathbb P_\mu; 1+\gamma}e^{-(\eta_0 - 3\eta_1)(t-k)} + Ce^{-\delta(t-k)}
    \\& \leq C'( e^{-\frac{\alpha}{1+\beta}t}e^{\frac{\alpha}{1+\gamma}t}e^{\frac{\gamma \alpha-\kappa_f(1+\gamma)b}{1+\gamma}k}e^{-(\eta_0 - 3\eta_1)(t-k)}+ e^{-\delta(t-k)}).
\end{align*}
Since $\alpha\beta<\kappa_f(1+\beta)b$,  we have  
\begin{align}
\label{eq: condition for supercritical}
	&-(\frac{\alpha}{1+\gamma}-\frac{\alpha}{1+\beta})
    = \alpha(1-\frac{1}{1+\gamma}) - \alpha(1-\frac{1}{1+\beta})
    \\&> \frac{\gamma \alpha}{1+\gamma} - k_f b
    =\frac{\alpha \gamma-\kappa_f(1+\gamma)b}{1+\gamma}.
\end{align}
	Using this, we have that there exists $C,C',
    \delta > 0$ such that for $k\geq 0$ and $t\geq k+1$,
\begin{align*}
    &\mathbb{\tilde{P}}_{\mu}\big[|Y'_{t,k}|\big]
    \\& \leq C( e^{(\frac{\alpha}{1+\gamma} - \frac{\alpha}{1+\beta})(t-k)}e^{-(\eta_0 - 3\eta_1)(t-k)}+ e^{-\delta(t-k)})
    \\& \leq C'e^{-\delta (t-k)}.
\end{align*}
Now we can use the same argument as in Step 3 in Section \ref{sec: proof of thm 1.3} to prove $|\mathbb{\tilde{P}}_{\mu}[e^{i\theta I'_1(t)}]-\mathbb{\tilde{P}}_{\mu}[e^{i\theta\tilde I'_1(t)}]|\xrightarrow[t\to \infty]{} 0$.

	Step 2. 
	We will show that $I'_2(t)\xrightarrow[t\to \infty]{d} 0$. 
	Let $\gamma \in (0,\beta)$.
	According to \eqref{eq: condition for supercritical}, we can choose $\epsilon > 0$ small enough so that
\[
	q:= - \frac{\alpha \gamma-\kappa_f(1+\gamma)b}{1+\gamma} 
	> \frac{\alpha}{1+\gamma}-\frac{\alpha}{1+\beta} + \frac{2\epsilon}{1+\beta} > 0.
\]
	Let 
\[
	\mathcal E_t
	:=\{\|X_t\|>e^{(\alpha-\epsilon )t}\},
	\quad t\geq 0.
\]
	According to Lemma \ref{lem: control of XT}, there exists
	$C>0$ such that  $\mathbb{\tilde{P}}_{\mu}(\mathcal E_t^c)\leq C e^{-\epsilon \beta t}$
	for all $t\geq 0$.
	Therefore, there exists $C>0$ such that
\begin{align}
    |\mathbb{\tilde{P}}_{\mu}[(e^{i\theta I'_2(t)}-1)\mathbf{1}_{\mathcal{E}^{c}_t}]|
    \leq 2\mathbb{\tilde{P}}_{\mu}(\mathcal{E}^c_t)\leq Ce^{-\epsilon \beta t},
    \quad t\geq t_0.
\end{align}
	According to Lemma \ref{lem: control of mgtrs}, there exist $C,C',C''>0$ such that for all
	$t\ge t_0$, 
\begin{align*}
    &|\mathbb{\tilde{P}}_{\mu} [ (e^{i\theta I'_2(t)}-1)\mathbf{1}_{\mathcal{E}_t}]|
    \leq |\theta| \mathbb{\tilde{P}}_{\mu} \big[ |I'_2(t)|\mathbf{1}_{\mathcal{E}_t}\big]
    \\&\leq|\theta| e^{-\frac{(\alpha - \epsilon )t}{1+\beta}} \Big(\sum_{k=\lceil t-\ln t \rceil}^{\lfloor t \rfloor - 1}\mathbb{\tilde{P}}_{\mu}\big[| \mathcal{I}_{t-k-1}^{t-k}\langle f,X_t\rangle|\big] + \mathbb{\tilde{P}}_{\mu}\big[| \mathcal{I}_{0}^{t-\lfloor t\rfloor}\langle f,X_t\rangle|\big]\Big)
    \\& \leq C  e^{-\frac{(\alpha - \epsilon )t}{1+\beta}} \Big(\sum_{k=\lceil t-\ln t \rceil}^{\lfloor t \rfloor - 1}\|\mathcal{I}_{t-k-1}^{t-k}\langle f,X_t\rangle\|_{\mathbb P_\mu; 1+\gamma} + \|\mathcal I_0^{t-\lfloor t \rfloor} \langle f, X_t\rangle\|_{\mathbb P_\mu;1+\gamma}\Big)
    \\ &\leq C'  e^{-\frac{(\alpha - \epsilon )t}{1+\beta}} \sum_{k=\lceil t-\ln t \rceil}^{\lfloor t \rfloor}e^{\frac{\alpha}{1+\gamma}t}e^{\frac{\alpha\gamma-\kappa_f(1+\gamma)b}{1+\gamma}k}\\
    &\leq C'  e^{-\frac{\epsilon}{1+\beta} t}e^{(\frac{\alpha }{1+\gamma}-\frac{\alpha }{1+\beta} + \frac{2\epsilon}{1+\beta})t} \sum_{k=\lceil t-\ln t \rceil}^{\lfloor t \rfloor}e^{\frac{\alpha\gamma-\kappa_f(1+\gamma)b}{1+\gamma}k}\\
    &\leq C'  e^{-\frac{\epsilon}{1+\beta} t} e^{qt} \sum_{k=\lceil t-\ln t \rceil}^{\lfloor t \rfloor}e^{-qk}\\
    &\leq C'  e^{-\frac{\epsilon}{1+\beta} t} e^{q(t - \lceil t - \ln t\rceil)} \sum_{j=0}^{\infty}e^{-qj}\leq C'' e^{-\frac{\epsilon}{1+\beta} t} t^q.
\end{align*}
	Therefore, we get the required result in this step.

	Step 3. We will show that $I'_3(t) \xrightarrow[t\to \infty]{\tilde {\mathbb P}_\mu \text{-} a.s.} 0$. 
	In fact, we can write
\begin{align}
	&|I'_3(t)| 
	\leq \frac{\langle |P^\alpha_tf|,X_0\rangle}{\|X_t\|^{\frac{1}{1+\beta}}}
	\leq \frac{\langle e^{\alpha t - \kappa_f b t}Qf,X_0\rangle}{(e^{\alpha t} H_t)^{\frac{1}{1+\beta}}}
	\\& = e^{(\frac{\beta \alpha }{1+\beta} - k_fb)t} H_t^{-\frac{1}{1+\beta}} \langle Qf,X_0\rangle
	\xrightarrow[t\to \infty]{\tilde {\mathbb P}_\mu \text{-} a.s.} 0.
\end{align}
	The proof is completed.

\appendix
\section{}

\subsection{Analytic facts}
    In this subsection, we present some analytic facts that will be useful.
\begin{lem}
\label{lem: estimate of exponential remaining}
    Suppose that $z\in \mathbb C_+$. Then
\begin{equation}
\label{eq: estimate of exponential remaining}
    \Big|e^{-z} - \sum_{k=0}^n \frac{(-z)^k}{k!} \Big|
    \leq \frac{|z|^{n+1}}{(n+1)!} \wedge \frac{2|z|^{n}}{n!}, \quad n\in \mathbb Z_+.
\end{equation}
\end{lem}
\begin{proof}
    Notice that $|e^{-z}| = e^{- \operatorname{Re} z} \leq 1$.
    %Therefore, according to \cite[Theorem 7.20]{Rudin1987Real},
    Therefore,
\begin{equation}
    |e^{-z} - 1| = \Big| \int_0^1 e^{-\theta z} z d\theta\Big|
    \leq |z|.
\end{equation}
    Also, notice that $|e^{-z} - 1| \leq |e^{-z}|+1 \leq 2$.
    Thus \eqref{eq: estimate of exponential remaining} is true when $n = 0$.
    Now, suppose that \eqref{eq: estimate of exponential remaining} is true when $n = m$ for some $m \in \mathbb Z_+$.
    %According to \cite[Theorem 7.20]{Rudin1987Real},
    We have
\begin{align}
    &\Big|e^{-z} - \sum_{k=0}^{m+1} \frac{(-z)^k}{k!}\Big|
    = \Big| \int_0^1\Big(e^{-\theta z} - \sum_{k=0}^m \frac{(-\theta z)^k}{k!} \Big) z d\theta \Big|
    \\&\quad \leq  \Big(\int_0^1 \frac{|\theta z|^{m+1}}{(m+1)!} |z| d\theta\Big) \wedge \Big(\int_0^1 \frac{2|\theta z|^{m}}{m!} |z| d\theta\Big)
    = \frac{|z|^{m+2}}{(m+2)!} \wedge \frac{2|z|^{m+1}}{(m+1)!},
\end{align}
    which says that \eqref{eq: estimate of exponential remaining} is true for $n = m + 1$.
    The proof is complete.
\end{proof}

\begin{lem}
\label{lem: extension lemma for branching mechanism}
%%%
%Suppose that  $\nu$ is a measure on $(0,\infty)$ such that $\int_{(0,\infty)} (u \wedge u^2) \nu(du)< \infty$. Then 
%\begin{equation}
%    h (z) = \int_{(0,\infty)} (e^{-zu} - 1 + zu) \nu(du), \quad z \in \mathbb C_+,
%\end{equation}
%is well defined.
%    Moreover, $h$ is continuous on $\mathbb C_+$ and is holomorphic on $\mathbb C_+^0$ with 
%\begin{equation}
%\label{eq: deriavetive of the Poission partb}
%    h'(z) = \int_{(0,\infty)}(1- e^{-uz})u \nu(du).
%\end{equation}
    Suppose that  $\nu$ is a measure on $(0,\infty)$ with $\int_{(0,\infty)} (u \wedge u^2) \nu(du)< \infty$. 
    Then functions
\begin{equation}
    h (z) = \int_{(0,\infty)} (e^{-zu} - 1 + zu) \nu(du), \quad z \in \mathbb C_+
\end{equation}
and
\begin{equation}
\label{eq: deriavetive of the Poission partb}
    h'(z) = \int_{(0,\infty)}(1- e^{-uz})u \nu(du), \quad z \in \mathbb C_+
\end{equation}
    are well defined, continuous on $\mathbb C_+$ and holomorphic on $\mathbb C_+^0$.
    Moreover, 
\[
    \frac{h(z)-h(z_0)}{z-z_0} \xrightarrow[\mathbb C_+\ni z \to z_0]{} h'(z_0),\quad z_0 \in \mathbb C_+.
\] %%
\end{lem}
\begin{proof}
    %From Lemma \ref{lem: estimate of exponential remaining}, we know that $h$ is well defined on $\mathbb C_+$.
    From Lemma \ref{lem: estimate of exponential remaining}, we know that $h$ and $h'$ are well defined on $\mathbb C_+$. %%
    %According to \cite[Theorem 3.2. \& Theorem 3.5]{SchillingSongVondracek2010Bernstein}, \eqref{eq: deriavetive of the Poission partb} defines a continuous function $h'$ on $\mathbb C_+$ which is holomorphic on $\mathbb C_+^0$.
    According to \cite[Theorem 3.2. \& Theorem 3.5]{SchillingSongVondracek2010Bernstein}, $h'$ is continuous on $\mathbb C_+$ and holomorphic on $\mathbb C_+^0$.
    
%    Let $z_0, z \in \mathbb C_+$.
%    Let $\gamma: [0,1]\mapsto \mathbb C_+$ be a $C^1$ path with $\gamma(0) = z_0$ and $\gamma(1) = z$.
%    Notice that, according to Lemma \ref{lem: estimate of exponential remaining},
%\begin{align}
%    &\int_0^1 \int_0^\infty |1-e^{-u\gamma(\theta)}|u~\nu(du)~d\theta
%    \\ &= \int_0^1~d\theta~ \Big( \int_0^1 |1-e^{-u\gamma(\theta)}|u~\nu(du) + \int_1^\infty |1-e^{-u\gamma(\theta)}|u~\nu(du) \Big)
%    \\ &\leq \int_0^1~d\theta~ \Big( |\gamma(\theta)|\int_0^1 u^2~\nu(du) + 2\int_1^\infty u~\nu(du) \Big)
%    < \infty.
%\end{align}
%    Therefore, according to Fubini's theorem and \cite[Theorem 7.20]{Rudin1987Real},
    From Lemma \ref{lem: estimate of exponential remaining}, for each $z_0 \in \mathbb C_+$, we have that there exists $C>0$ such that
\begin{align}
    &\Big| \frac{e^{-zu} - e^{-z_0u}+(z-z_0) u}{z-z_0} \Big|
    = \frac{1}{|z-z_0|}\Big| \int_0^1 \big(-ue^{-(\theta z+(1-\theta)z_0)u}+u\big)(z-z_0)d\theta\Big|
    \\ &\leq u\int_0^1 |1-e^{-(\theta z +(1-\theta)z_0)u}| d\theta 
    \leq (2u) \wedge\Big( u^2\int_0^1|\theta z+(1-\theta)z_0|d\theta\Big)
    \\ &\leq C(u\wedge u^2),
    \quad z \in \mathbb C_+ \text{ close enough to } z_0, u>0.
\end{align}
    Using this and dominated convergence theorem, we have
\begin{align}
    &\frac{h(z)-h(z_0)}{z-z_0} = \int_{(0,\infty)} \frac{e^{-zu}+zu -(e^{-z_0u}+z_0u)}{z-z_0}  \nu(du) 
    \\&\xrightarrow[\mathbb C_+\ni z\to z_0]{} \int_{(0,\infty)}(1 - e^{-z_0u} )u\nu(du) = h'(z_0),
\end{align}
    which says that $h$ is continuous on $\mathbb C_+$ and holomorphic on $\mathbb C_+^0$.
\end{proof}

	For each $z\in \mathbb C\setminus (-\infty,0]$, define
$
	\log z := \log |z| + i \arg z
$
	where $\arg z \in (-\pi,\pi)$ is uniquely determined by
$
	z = |z|e^{i \arg z}.
$ 	
	For all $z\in \mathbb C\setminus (-\infty,0]$ and $\gamma \in \mathbb C$, define
$
	z^\gamma := e^{\gamma \log z}.
$
	Then it is known, see \cite[Theorem 6.1]{SteinShakarchi2003Complex} for example, $z\mapsto \log z$ is holomorphic on $\mathbb C\setminus (-\infty,0]$.
	Therefore, for each $\gamma \in \mathbb C$, $z\mapsto z^\gamma$ is holomorphic on $\mathbb C\setminus (-\infty,0]$.
(We use the convention tha  $0^\gamma := \mathbf 1_{\gamma = 0}$.)
    From this definition we can show that $(z_1z_0)^\gamma = z_1^\gamma z_0^\gamma$ provided $\arg (z_1z_0)=\arg z_1 + \arg(z_0)$.


    Recall that $\Gamma$ the Gamma function defined by
\begin{equation}
    \Gamma (x) := \int_0^\infty t^{x-1} e^{-t}dt,
    \quad x>0.
\end{equation}
	It is known, see \cite[Theorem 6.1.3]{SteinShakarchi2003Complex} and the remark following it for example, the function $\Gamma$ has an unique analytic extension on $\mathbb C\setminus\{0, -1,-2,\dots\}$ and that
\[
	\Gamma(z+1) = z \Gamma(z),\quad z\in \mathbb C\setminus\{0, -1,-2,\dots\}.
\]
	Using this recursively, one get that
\begin{align}
\label{eq: definition of Gamma function}
    \Gamma(x)
    := \int_0^\infty t^{x-1} \Big(e^{-t} - \sum_{k=0}^{n-1} \frac{(-t)^k}{k!}\Big) dt,
    \quad -n< x< -n+1, n\in \mathbb N.
\end{align}

    Fix a $\beta \in (0,1)$.
    %Using the uniqueness of holomorphic extension and \cite[Theorem 3.2 \& Theorem 3.5]{SchillingSongVondracek2010Bernstein}, we get that
    Using the uniqueness of holomorphic extension and Lemma \ref{lem: extension lemma for branching mechanism}, we get that
\begin{equation}
    z^{\beta}
	= \int_0^\infty (e^{-zy}-1) \frac{dy}{\Gamma(-\beta)y^{1+\beta}},
    \quad z\in \mathbb C_+,
\end{equation}
	by showing that the both sides
\begin{itemize}
\item
    %are extension of the real function $x\mapsto x^{\beta}$ on $[0,\infty)$;
    are extension of the real function $x\mapsto x^{\beta}$ defined on $[0,\infty)$;
\item
    %are holomorphic on $\mathbb C_+^0:= \{x+iy:x\in (0,\infty), y\in \mathbb R\}$;
    are holomorphic on $\mathbb C_+^0$;
\item
    %are continuous on $\mathbb C_+ :=\{x+iy: x\in [0,\infty), y\in \mathbb R\}$.
    are continuous on $\mathbb C_+$.
\end{itemize}
    %Similarly, using Lemma \ref{lem: extension lemma for branching mechanism}, we get that
    Similarly, we get that
\begin{equation}
\label{eq: stable branching on C+}
    z^{1+\beta}
    = \int_0^\infty (e^{-zy}-1+zy)\frac{dy}{\Gamma(-1-\beta)y^{2+\beta}},
    \quad z\in \mathbb C_+.
\end{equation}
%    Moreover, according to \eqref{eq: path integration representation of h}, for any $C^1$ path $\gamma:[0,1]\to \mathbb C_+$ ,
%\begin{align}
%\label{eq: integration formula for 1+beta-th power of z}
%    &\gamma(1)^{1+\beta} - \gamma(0)^{1+\beta}
%    = \int_0^1 \gamma'(\theta)d\theta \int_{(0,\infty)}(1-e^{-\gamma(\theta)y})\frac{ydy}{\Gamma(-1-\beta)y^{2+\beta}}
%    \\&=\int_0^1 \gamma'(\theta)d\theta \int_{(0,\infty)}(1-e^{-\gamma(\theta)y})\frac{(-1-\beta)dy}{\Gamma(-\beta)y^{1+\beta}}
%    = \int_0^1 (1+\beta) \gamma(\theta)^{\beta} \gamma'(\theta)d\theta.
%\end{align}
%    This also shows that the derivative of $z\mapsto z^{1+\beta}$ is $z\mapsto (1+\beta)z^{\beta}$ on $\mathbb C^0_+$.
    Lemma \ref{lem: extension lemma for branching mechanism} also says that the derivative of $z^{1+\beta}$ is $(1+\beta)z^{\beta}$ on $\mathbb C^0_+$.
\begin{lem}
\label{lem: Lip of power function}
    %There is a constant $(1+\beta)$ such that for all $z_0,z_1 \in \mathbb C_+$,
    For each $z_0,z_1 \in \mathbb C_+$, we have
\begin{equation}
\label{eq: Lip of power function}
    |z_0^{1+\beta} - z_1^{1+\beta}|
    \leq (1+\beta)(|z_0|^{\beta}+|z_1|^{\beta})|z_0 - z_1|.
\end{equation}

\end{lem}
\begin{proof}
%added
    Since $z^{1+\beta}$ is continuous on $\mathbb C_+$, we only have to proof the Lemma assuming $z_0,z_1 \in \mathbb C^0_+$.
%end added
    Notice that
\begin{align}
\label{eq: upper bound for beta power of z}
	|z^\beta| 
	= |e^{\beta \log |z| +i\beta \operatorname {arg}z}| = e^{\beta \log |z|} = |z|^\beta,
	\quad z \in \mathbb C\setminus (-\infty, 0].
\end{align}
    %Define a path $\gamma: [0,1] \to \mathbb C_+$ such that
    Define a path $\gamma: [0,1] \to \mathbb C^0_+$ such that
\[
    \gamma(\theta)
    = z_0 (1-\theta) + \theta z_1,
    \quad \theta \in [0,1].
\]
    %Then, according to \eqref{eq: integration formula for 1+beta-th power of z}, we have
    Then, we have
\begin{align}
    |z_0^{1+\beta} - z_1^{1+\beta}|
    &\leq (1+\beta) \int_0^1 |\gamma(\theta)^{\beta}|\cdot |\gamma'(\theta)|d\theta
    \leq (1+\beta)  \sup_{\theta \in [0,1]} |\gamma(\theta)|^{\beta} \cdot |z_1-z_0|
    \\&\leq (1+\beta)  ( |z_1|^{\beta}+|z_0|^{\beta} ) |z_1-z_0|.
    \qedhere
\end{align}
\end{proof}

	Suppose that $\varphi(\theta)$ is a continuous function from $\mathbb R$ into $\mathbb C$ such that $\varphi(0) = 1$ and $\varphi(\theta) \neq 0$ for all $\theta \in \mathbb R$. 
	Then according to \cite[Lemma 7.6]{Sato1999Levy}, there is a unique continuous function $f(\theta)$ from $\mathbb R$ into $\mathbb C$ such that $f(0) = 0$ and $e^{f(\theta)} = \varphi(\theta)$. 
	Such a function $f$ is called the distinguished logarithm of the function $\varphi$ and is denoted as $\operatorname{Log} \varphi(\theta)$.
	In particular, let $\varphi$ be the characteristic function of an infinitely divisible random variable $Y$, then $h$ is called the L\'evy exponent of $Y$. 
	This distinguished logarithm should not be confused with the $\log$ function defined on $\mathbb C\setminus (-\infty, 0]$.

\subsection{Feynman-Kac formula with complex values}
\label{seq: complex Feynman-Kac transform}
    In this subsection we give a version of the Feynman-Kac formula with complex values.
    Suppose that $\{(\xi_t)_{t \in [r,\infty)}; (\Pi_{r,x})_{r\in [0,\infty), x\in E}\}$ is a (possibly non-homogeneous) Hunt process in a locally compact seperable metric space $E$.
%    Fix a time $t >0$.
%    For all $s\in (0,  t)$ and  bounded complex valued Borel function $\rho$ on $[0,t) \times E$, we write
%\begin{equation}
%    H^{(\rho)}_{(s,t)}:= \exp\Big\{\int_s^t \rho(u,\xi_u) du\Big\}.
%\end{equation}
    Write
\begin{equation}
    H^{(h)}_{(s,t)}
    := \exp\Big\{\int_s^t h(u,\xi_u) du\Big\},
    \quad 0 \leq s \leq t, h \in \mathcal B_b([0,t] \times E,\mathbb C).
\end{equation}
%deleted
%    Notice that
%\begin{equation}
%\label{eq: crucial for Feynman-Kac}
%    \frac{\partial}{\partial s} H^{(\rho)}_{(s,t)}= -H^{(\rho)}_{(s,t)}\rho(s,\xi_s),
%    \quad s\in (0,t).
%\end{equation}
%end deleted
%    Suppose that $\beta$, $\rho$ are complex valued bounded measurable functions on $[0,t) \times E$ and that $F$ is a complex valued bounded measurable function on $E$.
%    Define
%\begin{equation}
%    g(r,x) := \Pi_{r,x}[ H_{(r,t)}^{(\beta+\rho)} F(\xi_t)],\quad r \in [0,t), x\in E.
%\end{equation}
\begin{lem}
    Let $t \geq 0$. Suppose that $\beta,\rho\in \mathcal B_b([0,t] \times E, \mathbb C)$ and $f\in \mathcal B_b(E, \mathbb C)$.
    Then 
\begin{equation}
    g(r,x) := \Pi_{r,x}[ H_{(r,t)}^{(\beta+\rho)} f(\xi_t)],\quad r \in [0,t], x\in E,
\end{equation}
    is the unique solution to the equation
\[
    g(r,x)= \Pi_{r,x} [ H_{(r,t)}^{(\beta)} f(\xi_t)]+\Pi_{r,x} \Big[ \int_r^tH_{(r,s)}^{(\beta)}\rho(s,\xi_s) g(s,\xi_s)~ds \Big],\quad r \in [0,t], x\in E.
\]
\end{lem}
%    Notice that
%\begin{align}
%    \Pi_{r,x} \Big[ \int_r^t | H_{(r,t)}^{(\beta)}\rho(s,\xi_s) H_{(s,t)}^{(\rho)} F(\xi_t)| ~ds \Big]
%    \leq  \int_r^t e^{(t-r)\|\beta\|_\infty}e^{(t-s)\|\rho\|_\infty}\|\rho\|_\infty\|F\|_\infty ~ds
%    < \infty.
%\end{align}
%    Therefore, from the Markov property of $\xi$ and Fubini's theorem we get that
%\begin{align}
%    &\Pi_{r,x} \Big[ \int_r^tH_{(r,s)}^{(\beta)}~(\rho g)(s,\xi_s)~ds \Big]
%    =\Pi_{r,x} \Big[ \int_r^t H_{(r,s)}^{(\beta)}\rho(s,\xi_s) \Pi_{s,\xi_s}[ H_{(s,t)}^{(\beta+\rho)} F(\xi_t)]~ds \Big]
%    \\&= \Pi_{r,x} \Big[ \int_r^t H_{(r,t)}^{(\beta)}\rho(s,\xi_s) H_{(s,t)}^{(\rho)} F(\xi_t) ~ds \Big]
%    = \Pi_{r,x} [ H_{(r,t)}^{(\beta)}F(\xi_t)(H_{(r,t)}^{(\rho)} - 1)]
%    \\&= g(r,x) - \Pi_{r,x} [ H_{(r,t)}^{(\beta)} F(\xi_t)].
%\end{align}
\begin{proof}
    The proof is similar to the proof of \cite[Lemma A.1.5]{Dynkin1993Superprocesses}. We included it for the sake of completeness.
    Notice that
\begin{align}
    \Pi_{r,x} \Big[ \int_r^t | H_{(r,t)}^{(\beta)}\rho(s,\xi_s) H_{(s,t)}^{(\rho)} f(\xi_t)| ~ds \Big]
    \leq  \int_r^t e^{(t-r)\|\beta\|_\infty}e^{(t-s)\|\rho\|_\infty}\|\rho\|_\infty\|f\|_\infty ~ds
    < \infty.
\end{align}
    Also notice that
\begin{equation}
\label{eq: crucial for Feynman-Kac}
    \frac{\partial}{\partial s} H^{(\rho)}_{(s,t)}= -H^{(\rho)}_{(s,t)}\rho(s,\xi_s),
    \quad s\in (0,t).
\end{equation}
    Therefore, from the Markov property of $\xi$ and Fubini's theorem we get that
\begin{align}
    &\Pi_{r,x} \Big[ \int_r^tH_{(r,s)}^{(\beta)}~(\rho g)(s,\xi_s)~ds \Big]
    =\Pi_{r,x} \Big[ \int_r^t H_{(r,s)}^{(\beta)}\rho(s,\xi_s) \Pi_{s,\xi_s}[ H_{(s,t)}^{(\beta+\rho)} f(\xi_t)]~ds \Big]
    \\&= \Pi_{r,x} \Big[ \int_r^t H_{(r,t)}^{(\beta)}\rho(s,\xi_s) H_{(s,t)}^{(\rho)} f(\xi_t) ~ds \Big]
    = \Pi_{r,x} [ H_{(r,t)}^{(\beta)}f(\xi_t)(H_{(r,t)}^{(\rho)} - 1)]
    \\&= g(r,x) - \Pi_{r,x} [ H_{(r,t)}^{(\beta)} f(\xi_t)].
\end{align}
%added
    For the uniqueness, suppose that $\tilde g$ is another solution. Writting $h(r) = \sup_{x\in E}|g(r,x) - \tilde g(r,x)|$, we have
\[
    h(r) \leq e^{t\|\beta\|_\infty}\|\rho\|_\infty \int_r^t h(s)ds,
    \quad r\le t.
\]
    According to \cite[Lemma A.1.5]{Dynkin1993Superprocesses}, this says that $h(r) =  0$ for $r\in [0,t]$.
%end added
\end{proof}

\subsection{Superprocesses}
\label{sec: definition of superprocess}
%added
    In this subsection, we will give the definition of the superprocesses that are considered in this Appendix.
%end added
Let $E$ be locally compact separable metric space. Denote by $\mathcal M_E^1$ the collection of all the finite measures on $E$ equipped with weak topology.
    Recall that $X=\{(X_t)_{t\geq 0}; (\mathbf P_\mu)_{\mu \in \mathcal M^1_E}\}$ is said to be a $(\xi,\psi)$-superprocess if
\begin{itemize}
\item
    The spatial motion $\xi=\{(\xi_t)_{t\geq 0};(\Pi_x)_{x\in E}\}$ is an $E$-valued Hunt process with its lifetime denoted by $\zeta$.
\item
    The branching mechanism $\psi: E\times[0,\infty) \to \mathbb R$ is given by
\begin{equation}
\label{eq: branching mechanism}
    \psi(x,z)=
    - \beta(x) z + \alpha (x) z^2 + \int_{(0,\infty)} (e^{-zy} - 1 + zy) \pi(x,dy).
\end{equation}
    where $\beta \in \mathcal B_b(E)$, $\alpha \in \mathcal B_b(E, \mathbb R_+)$ and $\pi(x,dy)$ is a kernel from $E$ to $(0,\infty)$ such that $\sup_{x\in E} \int_{(0,\infty)} (y\wedge y^2) \pi(x,dy) < \infty$.
\item
    $X=\{(X_t)_{t\geq 0}; (\mathbf P_\mu)_{\mu \in \mathcal M^1_E}\}$ is an $\mathcal M^1_E$-valued Hunt process with transition probability determined by
\begin{align}
    \mathbf P_\mu [e^{-X_t(f)}] = e^{-\mu(V_tf)},
    \quad t\geq 0, \mu \in \mathcal M_E^1, f\in \mathscr B^+_b(E),
\end{align}
    where for each $f\in \mathcal B_b(E)$, the function $(t,x)\mapsto V_tf(x)$ on $[0,\infty) \times E$ is the unique locally bounded positive solution to the equation
\begin{align}\label{eq:FKPP_in_definition}
    V_tf(x) + \Pi_x \Big[  \int_0^{t\wedge \zeta} \psi(\xi_s,V_{t-s}f)ds \Big]
    = \Pi_x [ f(\xi_t)\mathbf 1_{t<\zeta} ],
    \quad t \geq 0, x \in E.
\end{align}
\end{itemize}
    We refer our reader to \cite{Li2011Measure-valued} for more discussion about the definition and the existence of superprocesses.
    To avoid triviality, we assume that $\psi(x,z)\neq -\beta(x)z$ for some $x \in E$ and $z \geq 0$.

    Notice that, the branching mechanism $\psi$ can be extended into a map from $E \times \mathbb C_+$ to $\mathbb C$ using \eqref{eq: branching mechanism}.
    Define
\begin{equation}
    \psi'(x,z):= - \beta(x) + 2\alpha(x) z + \int_{(0,\infty)} (1-e^{-zy})y\pi(x,dy),
    \quad x\in E, z\in \mathbb C_+.
\end{equation}
    Then according to Lemma \ref{lem: extension lemma for branching mechanism}, for each $x \in E$, $z \mapsto \psi(x,z)$ is a holomorphic function on $\mathbb C_+^0$ with deriavetive $z \mapsto \psi'(x,z)$.
    Define $\psi_0(x,z) := \psi(x,z)+ \beta(x)z $ and $\psi'_0(x,z) := \psi'(x,z) + \beta(x)$.

    Denote by $\mathbb W$ the space of $\mathcal M_E^1$-valued c\`{a}dl\`{a}g paths with its conanical path denoted by $(W_t)_{t\geq 0}$.
    We say $X$ is \emph{non-persistent} if $\mathbf P_{\delta_x}(\|X_t\|= 0) > 0$ for all $x\in E$ and $t> 0$.
    Suppose that $(X_t)_{t\geq 0}$ is non-persistent, then according to \cite[Section 8.4]{Li2011Measure-valued},
    there is a unique family of measures $(\mathbb N_x)_{x\in E}$ on $\mathbb W$ such that
\begin{itemize}
\item
    $\mathbb N_x (\forall t \geq 0, \|W_t\|=0) =0$;
\item
    $\mathbb N_x(\|W_0 \|\neq 0) = 0$;
\item
    For any $\mu \in \mathcal M_E^1$, if $\mathcal N$ is a Poisson random measure defined on some probability space
    with intensity $\mathbb N_\mu(\cdot):= \int_E \mathbb N_x(\cdot )\mu(dx)$,
    then the superprocess $\{X;\mathbf P_\mu\}$ can be realized by $\widetilde X_0 := \mu$ and $\widetilde X_t(\cdot) := \mathcal N[W_t(\cdot)]$ for each $t>0$.
\end{itemize}
    We refer to $(\mathbb N_x)_{x\in E}$ as the \emph{Kuznetsov measures} of $X$.
\subsection{{Semigroups for superprocesses}}
\label{sec: definition of vf}
    %Let $X$ be the superprocess defined in Subsection \ref{sec: definition of superprocess}.
    %Assume that the superprocess $X$ is non-presistent with its Kuznestov measure denoted by $(\mathbb N_x)_{x\in E}$.
    Let $X$ be the non-presistent superprocess defined in Subsection \ref{sec: definition of superprocess} with its Kuznestov measure denoted by $(\mathbb N_x)_{x\in E}$.
    Define the mean semigroup
\begin{equation}
    P_t^\beta f(x)
    := \Pi_{x}[e^{\int_0^t \beta(\xi_s)ds}f(\xi_t) \mathbf 1_{t< \zeta}],
    \quad t\geq 0, x\in E, f\in \mathcal B_b(E,\mathbb R_+).
\end{equation}
    It is known from \cite[Proposition 2.27]{Li2011Measure-valued} and \cite[Theorem 2.7]{Kyprianou2014Fluctuations} that for all $t > 0$, $\mu \in \mathcal M_E^1$ and $f\in \mathcal B_b(E,\mathbb R_+)$,
\begin{equation}
\label{eq: mean formula for superprocesses}
    \mathbb N_{\mu}[W_t(f)]
    =\mathbf P_{\mu}[X_t(f)]=\mu(P^\beta_t f).
\end{equation}
    
    Define
\begin{align}
    L_1(\xi)
    &:= \{f\in \mathcal B(E): \forall x\in E, t\geq 0, \quad \Pi_x[|f(\xi_t)|]< \infty\},
    \\L_2(\xi)
    &:= \{f \in \mathcal B(E): |f|^2 \in L_1(\xi)\}.
\end{align}
    %Let $f\in L_1(\xi), t >0$ and $x\in E$.
    Using monotonicity and linearity, we get from \eqref{eq: mean formula for superprocesses}  that
\begin{equation}
    \mathbb N_x[W_t(f)]
    =\mathbf P_{\delta_x}[X_t(f)]=P^\beta_t f(x) \in \mathbb R,
%added
    \quad f\in L_1(\xi), t > 0,x\in E.
%end added
\end{equation}
%added
    This says that random variables $X_t(f)$ are well defined under probability $\mathbf P_{\delta_x}$ provided $f\in L_1(\xi)$.
%end added
    %Notice that, from the branching property of the superprocess $X$, $\{X_t(f); \mathbf P_{\delta_x}\}$ is an infinitely divisible random variable with finite moment.  %({\bf Do you want to say mean or higher moment?})
    According to the branching property of the superprocess, they are infinitely divisible random variables.
    %Denote by $U_t(\theta f)(x) := \operatorname{Log} \mathbf P_{\delta_x}[e^{i \theta X_t(f)}]$  the charateristic exponent of the random variable $\{X_t(f); \mathbf P_{\delta_x}\}$.
    Therefore, we can write 
\[
    U_t(\theta f)(x) := \operatorname{Log} \mathbf P_{\delta_x}[e^{i \theta X_t(f)}],
%added    
    \quad t\geq 0, f\in L_1(\xi), \theta \in \mathbb R, x\in E,
%end added
\] 
    %the charateristic exponent of the random variable $\{X_t(f); \mathbf P_{\delta_x}\}$.
    as their charateristic exponent.
    %According to Campbell's formula, see \cite[Theorem 2.7]{Kyprianou2014Fluctuations} for example, for each $\theta \in \mathbb R$,
%$   \mathbf P_{\delta_x} [e^{i\theta X_t(f)}]
%    = \exp(\mathbb N_x[ e^{i\theta W_t(f)} - 1]).
%$
    According to Campbell's formula, see \cite[Theorem 2.7]{Kyprianou2014Fluctuations} for example, we have
\[   
    \mathbf P_{\delta_x} [e^{i\theta X_t(f)}]
    = \exp(\mathbb N_x[ e^{i\theta W_t(f)} - 1]),
    \quad t>0, f\in L_1(\xi), \theta \in \mathbb R, x\in E.
\]
    %Notice that $\theta \mapsto \mathbb N_x[e^{i\theta W_t(f)} - 1]$ is a continuous function.
    %Also notice that $\mathbb N_x[e^{i\theta W_t(f)} - 1] = 0$ if $\theta = 0$.
    %Therefore, according to \cite[Lemma 7.6]{Sato1999Levy}, we have
    Noticing that $\theta \mapsto \mathbb N_x[e^{i\theta W_t(f)} - 1]$ is a continuous function on $\mathbb R$ and that $\mathbb N_x[e^{i\theta W_t(f)} - 1] = 0$ if $\theta = 0$, according to \cite[Lemma 7.6]{Sato1999Levy}, we have
\begin{equation}
\label{eq: N and characteristic exponent}
    %U_tf(x) = \mathbb N_x[e^{i W_t(f)} - 1],
    U_t(\theta f)(x) = \mathbb N_x[e^{i W_t(\theta f)} - 1],
    %\quad t \geq 0.
    \quad t>0, f\in L_1(\xi), \theta \in \mathbb R, x\in E.
\end{equation}

%added
\begin{lem}
    There exists constants $C_1, C_2\geq 0$ such that for each $f \in L_1(\xi),x\in E$ and $t\geq 0$, we have
\begin{align}
\label{eq: upper bound of psi(v)}
    \big|\psi\big(x,-U_tf\big)\big|
    \leq C_1 P^\beta_t |f|(x)+C_2 P^\beta_t |f| (x)^2.
\end{align}
\end{lem}
\begin{proof}
     Noticing that
\[
     e^{\operatorname{Re} U_tf(x)}
    = |e^{U_tf(x)}|
    = |\mathbf P_{\delta_x}[e^{i X_t(f)}]|
    \leq 1,
\]
    we have
\begin{equation}
\label{eq: -v has positive real part}
 \operatorname{Re} U_tf(x)
    \leq 0.
\end{equation}
    Therefore, we can talk about $\psi(x,-U_tf)$ since $z\mapsto \psi(x,z)$ is well defined on $\mathbb C_+$.
    According to Lemma \ref{lem: estimate of exponential remaining}, we have that
\begin{equation}
\label{eq: upper bound for vf}
    |U_tf(x)| \leq \mathbb N_x[|e^{i W_t(f)} - 1|]
    \leq \mathbb N_x[|i W_t(f)|]
    \leq (P^\beta_t |f|)(x).
\end{equation}
    Notice that, for any compact $K \subset \mathbb R$,
\begin{equation}
\label{eq: estimate of deriavetive of v(theta)}
    \mathbb N_x\Big[\sup_{\theta \in K} \Big|\frac{\partial}{\partial \theta} (e^{i\theta W_t(f)} - 1) \Big|\Big]
    \leq \mathbb N_x[|W_t(f)|] \sup_{\theta \in K}|\theta| \leq (P^\beta_t |f|)(x) \sup_{\theta \in K}|\theta| < \infty.
\end{equation}
    Therefore, according to \cite[Theorem A.5.2.]{Durrett2010Probability},
    $U_t(\theta f)(x)$ is differentiable in $\theta \in \mathbb R$ with
\[
    \frac{\partial}{\partial \theta} U_t(\theta f)(x)
    = i\mathbb N_x[W_t(f)e^{i\theta W_t(f)}],
    \quad \theta \in \mathbb R.
\]
    Moreover, from the above, it is clear that
\begin{equation}
\label{eq: upper bounded for derivative of v(theta)}
    \sup_{\theta \in \mathbb R}\Big| \frac{\partial}{\partial \theta}U_t(\theta f)(x)\Big|
    \leq ( P^\beta_t |f|)(x).
\end{equation}
    From the dominate convergence theorem, we can verify that $(\partial/\partial \theta)U_tf(x)$ is continuous in $\theta$.
    In other words, $\theta \mapsto -U_t(\theta f)(x)$ is a $C^1$ map from $\mathbb R$ to $\mathbb C_+$.
    According to this, we have
\begin{equation}
\label{eq: path integration representation of psi(v)}
    \psi(x,-U_tf) = -\int_0^1 \psi'\big(x,-U_t(\theta f)\big) \frac{\partial}{\partial \theta} U_t(\theta f)(x)~d\theta.
\end{equation}
    Notice that
\begin{equation}
\label{eq: upper bound of psi'(v)}
\begin{split}
    &|\psi'(x, -U_tf)|
    \\&= \Big| -\beta(x)- 2\alpha(x) U_tf(x)+ \int_{(0,\infty)} y (1- e^{y U_tf(x)} ) \pi(x,dy)\Big|
    \\&= \Big| - \beta(x)- 2\alpha(x)\mathbb N_x[e^{i W_{t}(f)} - 1]  + \int_{(0,\infty)} y \mathbf P_{y \delta_x}[1-e^{i X_{t}(f)}] \pi(x,dy) \Big|
\\ &\leq \|\beta\|_\infty + 2\alpha(x)\mathbb N_x[W_t(|f|)]+ \int_{(0,\infty)} y\mathbf P_{y\delta_x}[2\wedge X_t(|f|)] \pi(x,dy)
\\ &\leq \|\beta\|_\infty + 2\|\alpha\|_\infty  P^\beta_t |f|(x) + \Big(\sup_{x\in E}\int_{(0,1]}y^2 \pi(x,dy)\Big)~P^\beta_t |f|(x) + 2\sup_{x\in E}\int_{(1,\infty)} y \pi(x,dy)
\\ &=: C_1 + C_2(P^\beta_t |f|)(x),
\end{split}
\end{equation}
    where $C_1, C_2$ are constants independent on $f,x$ and $t$.
    Now, using \eqref{eq: path integration representation of psi(v)}, \eqref{eq: upper bounded for derivative of v(theta)} and \eqref{eq: upper bound of psi'(v)}, we have get the desired result.
\end{proof}
%end added


    %We claim that if $f\in L^2(\xi)$ then the following expectations
    This Lemma also says that if $f\in L^2(\xi)$ then the following expectations
\[
    \Pi_x\Big[\int_0^t \psi(\xi_s,- U_{t-s}f)ds\Big]
    \in \mathbb C,
%added
    \quad x\in E, t\geq 0.
%end added
\]
    are well defined. 
%    First, we only assume that $f\in L^1(\xi)$.
%    Noticing that
%\[
%     e^{\operatorname{Re} U_tf(x)}
%    = |e^{U_tf(x)}|
%    = |\mathbf P_{\delta_x}[e^{i X_t(f)}]|
%    \leq 1,
%\]
%    we have
%\begin{equation}
%\label{eq: -v has positive real part}
% \operatorname{Re} U_tf(x)
%    \leq 0.
%\end{equation}
%    Therefore, we can talk about $\psi(x,-U_tf)$ since $z\mapsto \psi(x,z)$ is well defined on $\mathbb C_+$.
%    According to Lemma \ref{lem: estimate of exponential remaining}, we have that
%\begin{equation}
%\label{eq: upper bound for vf}
%    |U_tf(x)| \leq \mathbb N_x[|e^{i W_t(f)} - 1|]
%    \leq \mathbb N_x[|i W_t(f)|]
%    \leq (P^\beta_t |f|)(x).
%\end{equation}
%    Notice that, for any compact $K \subset \mathbb R$,
%\begin{equation}
%\label{eq: estimate of deriavetive of v(theta)}
%    \mathbb N_x\Big[\sup_{\theta \in K} \Big|\frac{\partial}{\partial \theta} (e^{i\theta W_t(f)} - 1) \Big|\Big]
%    \leq \mathbb N_x[|W_t(f)|] \leq (P^\beta_t |f|)(x) < \infty.
%\end{equation}
%    Therefore, according to \cite[Theorem A.5.2.]{Durrett2010Probability},
%    $U_t(\theta f)(x)$ is differentiable in $\theta \in \mathbb R$ with
%\[
%    \frac{\partial}{\partial \theta} U_t(\theta f)(x)
%    = i\mathbb N_x[W_t(f)e^{i\theta W_t(f)}],
%    \quad \theta \in \mathbb R.
%\]
%    Moreover, from the above, it is clear that
%\begin{equation}
%\label{eq: upper bounded for derivative of v(theta)}
%    \sup_{\theta \in \mathbb R}\Big| \frac{\partial}{\partial \theta}U_t(\theta f)(x)\Big|
%    \leq ( P^\beta_t |f|)(x).
%\end{equation}
%    From the dominate convergence theorem, we can verify that $(\partial/\partial \theta)U_tf(x)$ is continuous in $\theta$.
%    In other words, $\theta \mapsto -U_t(\theta f)(x)$ is a $C^1$ map from $\mathbb R$ to $\mathbb C_+$.
%    According to this and \eqref{eq: path integration representation of h}, we can write
%\begin{equation}
%\label{eq: path integration representation of psi(v)}
%    \psi(x,-U_tf) = -\int_0^1 \psi'\big(x,-U_t(\theta f)\big) \frac{\partial}{\partial \theta} U_t(\theta f)(x)~d\theta.
%\end{equation}
%    Notice that
%\begin{equation}
%\label{eq: upper bound of psi'(v)}
%\begin{split}
%    &|\psi'(x, -U_tf)|
%    \\&= \Big| -\beta(x)- 2\alpha(x) U_tf(x)+ \int_{(0,\infty)} y (1- e^{y U_tf(x)} ) \pi(x,dy)\Big|
%    \\&= \Big| - \beta(x)- 2\alpha(x)\mathbb N_x[e^{i W_{t}(f)} - 1]  + \int_{(0,\infty)} y \mathbf P_{y \delta_x}[1-e^{i X_{t}(f)}] \pi%(x,dy) \Big|
%\\ &\leq \|\beta\|_\infty + 2\alpha(x)\mathbb N_x[W_t(|f|)]+ \int_{(0,\infty)} y\mathbf P_{y\delta_x}[2\wedge X_t(|f|)] \pi(x,dy)
%\\ &\leq \|\beta\|_\infty + 2\|\alpha\|_\infty  P^\beta_t |f|(x) + \Big(\sup_{x\in E}\int_{(0,1]}y^2 \pi(x,dy)\Big)~P^\beta_t |f|(x) + 2%\sup_{x\in E}\int_{(1,\infty)} y \pi(x,dy)
%\\ &=: C_1 + C_2(P^\beta_t |f|)(x),
%\end{split}
%\end{equation}
%    where $C_1, C_2$ are constants independent on $f,x$ and $t$.
%    Now, using \eqref{eq: path integration representation of psi(v)}, \eqref{eq: upper bounded for derivative of v(theta)} and \eqref{eq: upper bound of psi'(v)}, we have
%\begin{align}
%\label{eq: upper bound of psi(v)}
%    \big|\psi\big(x,-U_tf\big)\big|
%    \leq C_1 P^\beta_t |f|(x)+C_2 P^\beta_t |f| (x)^2.
%\end{align}
%    Now assume that $f \in L_2(\xi) \subset L_1(\xi)$.
    In fact, using Jensen's inequality and the markovian property, we have 
\begin{align}
\label{eq: domination of psi(v)}
    &\Pi_x\Big[\int_0^t \big|\psi \big(\xi_s,-U_{t-s}f\big)\big|ds\Big]
    \\&\leq \Pi_x\Big[\int_0^t \big(C_1 P_{t-s}^\beta|f|(\xi_s)+C_2 P_{t-s}^\beta|f|(\xi_s)^2\big)ds\Big]
    \\ &\leq \int_0^t \big(C_1 e^{t\|\beta\|}\Pi_x \big[ \Pi_{\xi_s}[|f(\xi_{t-s})|] \big]+C_2 e^{2t\|\beta\|}\Pi_x \big[ \Pi_{\xi_s}[|f (\xi_{t-s})|]^2 \big]\big)~ds
    \\ &\leq \int_0^t (C_1 e^{t\|\beta\|}\Pi_x [ |f(\xi_{t})|]+C_2e^{2t\|\beta\|}\Pi_x [ |f (\xi_{t})|^2 ])~ds < \infty.
\end{align}
%deleted
%	As a consequence, the expectation
%\[
%     \Pi_x\Big[\int_0^t \psi(\xi_s,-U_{t-s}f)ds\Big]
%    \in \mathbb C
%\]
%    is well defined for all $f\in L_2(\xi)$.
%end deleted

\subsection{}
	Let $X$ be the non-persistent superprocess discussed in Subsection \ref{sec: definition of vf}. 
	In this subsection, we will prove the following:
	\begin{prop}
\label{prop: complex FKPP-equation}
    If $f\in L_2(\xi)$,  then for all $t\geq 0$ and $x\in E$,
\begin{equation}
\label{eq: complex FKPP-equation}
    U_tf(x) - \Pi_x \Big[\int_0^t \psi\big(\xi_s, - U_{t-s}f\big) ds \Big]
    \\= i \Pi_x [f(\xi_t)]
\end{equation}
and
\begin{equation}
\label{eq: complex FKPP-equation with FK-transform}
    U_tf(x) -  \int_0^t P_{t-s}^{\beta} \psi_0\big(\cdot,-U_sf\big) (x)~ds
    \\= iP_t^\beta f(x).
\end{equation}
\end{prop} 

    To prove this, we will need the generalized spine decomposition theorem from \cite{RenSongSun2017Spine} which we now recall.
    Let $X$ be the non-persistent superprocess discussed in Subsection \ref{sec: definition of vf}.
    Let $f\in \mathcal B_b(E,\mathbb R_+)$, $T >0$ and $x\in E$.
    Suppose that $\mathbf P_{\delta_x}[X_T(f)] = \mathbb N_x[W_T(f)] = P^\beta_T f(x) \in (0,\infty)$, then we can define the following probability transforms:
\begin{equation}
    d\mathbf P_{\delta_x}^{X_T(f)}
    := \frac{X_T(f)}{P_T^\beta f(x)} d\mathbf P_{\delta_x};
    \quad d\mathbb N_x^{W_T(f)}
    :=  \frac{W_T(f)}{P_T^\beta f(x)} d\mathbb N_x.
\end{equation}
    Following the definition in \cite{RenSongSun2017Spine}, we say that $\{\xi, \mathbf n;\mathbf Q_{x}^{(f,T)}\}$ is a spine representation of $\mathbb N_x^{W_T(f)}$ if
\begin{itemize}
\item
    The spine process $\{(\xi_t)_{0\leq t\leq T}; \mathbf Q^{(f,T)}_x\}$ is a copy of $\{(\xi_t)_{0\leq t\leq T}; \Pi^{(f,T)}_{x}\}$,
    where
\begin{equation}
    d\Pi_x^{(f,T)} := \frac{f(\xi_T)e^{\int_0^T \beta(\xi_s)ds}}{P^\beta_T f(x)} d \Pi_x;
\end{equation}
\item
    Given $\{(\xi_t)_{0\leq t\leq T}; \mathbf Q^{(f,T)}_x\}$, the immigration measure $\{\mathbf n(\xi,ds,dw); \mathbf Q^{(f,T)}_x[\cdot |(\xi_t)_{0\leq t\leq T}]\}$ is a Poisson random measure on $[0,T] \times \mathbb W$ with intensity
\begin{align}
\label{eq: conditional intensity}
    \mathbf m(\xi,ds,dw)
    := 2 \alpha(\xi_s) ds \cdot \mathbb N_{\xi_s}(dw) + ds \cdot \int_{y\in (0,\infty)} y \mathbf P_{y\delta_{\xi_s}}(X\in dw) \pi(\xi_s,dy);
\end{align}
\item
    $\{(Y_t)_{0\leq t\leq T}; \mathbf Q^{(f,T)}_x\}$ is an $\mathcal M^1_E$-valued process defined by
\begin{align}
    Y_t
    := \int_{(0,t] \times \mathbb W} w_{t-s} \mathbf n(\xi,ds,dw),
    \quad 0 \leq t\leq T.
\end{align}
\end{itemize}
    According to the spine decomposition theorem in \cite{RenSongSun2017Spine}, we have that
\begin{align}
\label{eq: Spine decomposition 1}
    \{(X_s)_{s \geq 0};\mathbf P_{\delta_x}^{X_T(f)}\}
    \overset{f.d.d.}{=} \{(X_s + W_s)_{s \geq 0};\mathbf P_{\delta_x} \otimes \mathbb N_x^{W_T(f)} \}
\end{align}
    and
\begin{align}
\label{eq: Spine decomposition 2}
    \{(W_s)_{0\leq s\leq T};\mathbb N_x^{W_T(f)}\}
    \overset{f.d.d.}{=} \{(Y_s)_{s \geq 0};\mathbf Q_x^{(f,T)}\}.
\end{align}

\begin{proof}[Proof of Proposition \ref{prop: complex FKPP-equation}]
    Assume that $f\in \mathcal B_b(E)$.
    Fix $t>0, r\in [0,t), x\in E$ and $g\in \mathcal B_b(E)^{++}$.
    Denote by $\{\xi, \mathbf n; \mathbf Q_x^{(g,t)}\}$ the spine representation of $\mathbb N_x^{W_t(g)}$.
    Conditioned on $\{\xi; \mathbf Q_x^{(g,t)}\}$, denote by $\mathbf m(\xi, ds,dw)$ the conditional intensity of $\mathbf n$ in \eqref{eq: conditional intensity}.
    Denote by $\Pi_{r,x}$ the probability of Hunt process $\{\xi; \Pi\}$ initiated at time $r$ and position $x$.
    From Lemma \ref{lem: estimate of exponential remaining}, we have $\mathbf Q^{(g,t)}_{x}$-almost surely
\begin{align}
&\int_{[0,t]\times \mathbb W}|e^{i w_{t-s}(f)} - 1| \mathbf m(\xi, ds,dw)
    \leq \int_{[0,t]\times \mathbb W}\big(| w_{t-s}(f)| \wedge 2\big) \mathbf m(\xi, ds,dw)
    \\&\leq \int_0^t \Big(2\alpha(\xi_s)\mathbb N_{\xi_s}\big( W_{t-s}(|f|)\big)  + \int_{(0,1]} y \mathbf P_{y \delta_{\xi_s}}[ X_{t-s}(|f|)] \pi(\xi_s,dy)
    \\&\qquad\qquad+ 2\int_{(1,\infty)}y\pi(\xi_s,dy)\Big) ds
     \\&\leq \int_0^t (P_{t-s}^\beta |f|)(\xi_s)\Big(2\alpha(\xi_s)  + \int_{(0,1]} y^2 \pi(\xi_s,dy)\Big) ds + 2t \sup_{x\in E}\int_{(1,\infty)}y\pi(x,dy)
    \\&\leq \Big(2\|\alpha\|_\infty +\sup_{x\in E}\int_{(0,1]} y^2 \pi(x,dy)\Big) t e^{t\|\beta\|_\infty}\|f\|_\infty + 2t \sup_{x\in E}\int_{(1,\infty)}y\pi(x,dy)
    < \infty.
\end{align}
    Using this, Fubini's theorem, \eqref{eq: N and characteristic exponent} and \eqref{eq: -v has positive real part} we have $\mathbf Q^{(g,t)}_{x}$-almost surely,
\begin{align}
    &\int_{[0,t]\times \mathbb N}(e^{i w_{t-s}(f)} - 1) \mathbf m(\xi, ds,dw)
    \\&=\int_0^t \Big(2\alpha(\xi_s)\mathbb N_{\xi_s}(e^{i W_{t-s}(f)} - 1)  + \int_{(0,\infty)} y \mathbf P_{y \delta_{\xi_s}}[e^{i X_{t-s}(f)} - 1] \pi(\xi_s,dy)\Big) ds
    \\&=\int_0^t \Big( 2\alpha(\xi_s) U_{t-s} f(\xi_s) + \int_{(0,\infty)} y (e^{y U_{t-s}f(\xi_s)} - 1) \pi(\xi_s,dy) \Big) ds
    \\&= -\int_0^t \psi'_0 \big(\xi_s, -U_{t-s}f\big)ds.
\end{align}
    Therefore, according to \eqref{eq: Spine decomposition 2}, Campbell's formula and above, we have that
\begin{align}
\label{eq: N to Pi}
    \mathbb N_x^{W_t(g)}[e^{i W_t(f)}]
    &=\mathbf Q_x^{(g,t)} \Big[\exp\Big\{\int_{[0,t]\times \mathbb N}(e^{i w_{t-s}(f)} - 1) \mathbf m(\xi, ds,dw)\Big\}\Big]
    \\&= \Pi_x^{(g,t)} [e^{-\int_0^t \psi'_0(\xi_s, -U_{t-s}f)ds}]
    \\&= \frac{1}{T_t^\beta g (x)} \Pi_x[ g(\xi_t) e^{-\int_0^t \psi'(\xi_s, -U_{t-s}f)ds} ].
\end{align}
    Let $\epsilon >0$.
    Define $f^+ = (f \vee 0) + \epsilon$ and $f^- = (-f) \vee 0 + \epsilon$, then $f^\pm \in b\mathscr B^{++}_E$ and $f = f^+ - f^-$.
    According to \eqref{eq: Spine decomposition 1}, we have that
\begin{equation}
    \frac{\mathbf P_{\delta_x}[X_t(f^{\pm})e^{i X_t(f)}]}{\mathbf P_{\delta_x}[X_t(f^{\pm})]}
    = \mathbf P_{\delta_x}[e^{i X_t(f)}] \mathbb N_x^{W_t(f^{\pm})}[e^{i X_t(f)}].
\end{equation}
    Using \eqref{eq: N to Pi} and the above, we have
\begin{align}
    \frac{\mathbf P_{\delta_x}[X_t(f)e^{i X_t(f)}] }{\mathbf P_{\delta_x}[e^{i X_t(f)}]}
    &= \mathbf P_{\delta_x}[X_t(f^+)] \mathbb N_x^{W_t(f^+)} [e^{i X_t(f)}] - \mathbf P_{\delta_x}[X_t(f^-)]\mathbb N_x^{W_t(f^-)}[e^{i X_t(f)}]
    \\& = \Pi_x[ f(\xi_t) e^{- \int_0^t \psi'(\xi_s, -U_{t-s}f) ds}  ].
\end{align}
    Therefore, we have
\begin{align}
    \frac{\partial}{\partial \theta} {U_t(\theta f)(x)}
    = \frac{\mathbf P_{\delta_x}[iX_t(f)e^{i X_t(f)}] }{\mathbf P_{\delta_x}[e^{i X_t(f)}]}
    =  \Pi_x[ if(\xi_t) e^{ - \int_0^t \psi'(\xi_s, -U_{t-s}(\theta f)) ds} ].
\end{align}
    Since $\{(\xi_{r+t})_{t \geq 0}; \Pi_{r,x}\} \overset{d}{=} \{(\xi_{t})_{t\geq 0}; \Pi_{x}\} $, we have

\begin{align}
    &\frac{\partial}{\partial \theta} U_{t-r}(\theta f)( x)
    = \Pi_x[ i f(\xi_{t-r}) e^{-\int_0^{t-r} \psi'(\xi_s, -U_{t-r-s}(\theta f)) ds} ]
    \\&= \Pi_{r,x}[i f(\xi_t)e^{-\int_0^{t-r} \psi'(\xi_{r+s}, -U_{t-r-s}(\theta f)) ds} ]
    = \Pi_{r,x}[if(\xi_t)e^{-\int_r^t \psi'(\xi_{s}, -U_{t-s}(\theta f)) ds} ].
\end{align}

    From \eqref{eq: upper bound of psi'(v)}, we know that for each $\theta\in \mathbb R$, $(t,x) \mapsto |\psi'(x,-U_tf(x))|$ is locally bounded (i.e. bounded on $[0,T]\times E$ for each $T \geq 0$).
    Therefore, we can apply the argument in Subsection \ref{seq: complex Feynman-Kac transform} and get that
\[
    \frac{\partial}{\partial \theta} U_{t-r}(\theta f)(x) + \Pi_{r,x} \Big[\int_r^t \psi'\big(\xi_s,- U_{t-s}(\theta f)\big)\frac{\partial}{\partial \theta} U_{t-s}(\theta f)(\xi_s)~ds\Big]
    = \Pi_{r,x} [i f(\xi_t)]
\]
    and
\begin{align}
    &\frac{\partial}{\partial \theta} U_{t-r}(\theta f)(x) + \Pi_{r,x} \Big[\int_r^t e^{\int_r^s \beta(\xi_u)du}\psi_0'\big(\xi_s,- U_{t-s}(\theta f)\big)\frac{\partial}{\partial \theta} U_{t-s}(\theta f)(\xi_s)~ds\Big]\\
    &= \Pi_{r,x} [i e^{\int_r^t \beta(\xi_s)ds}f(\xi_t)].
\end{align}
    Integrating the two displays above with respect to $\theta$  on [0,1], using \eqref{eq: path integration representation of psi(v)}, \eqref{eq: upper bound of psi'(v)}, \eqref{eq: upper bounded for derivative of v(theta)} and Fubini's theorem, we get
\begin{equation}
    U_{t-r}f(x) - \Pi_{r,x} \Big[\int_r^t \psi\big(\xi_s,-U_{t-s}f\big) ~ds\Big]
    = i\theta \Pi_{r,x} [f(\xi_t)]
\end{equation}
    and 
\begin{equation}
    U_{t-r}f(x) - \Pi_{r,x} \Big[\int_r^t e^{\int_r^s \beta(\xi_u)du} \psi_0\big(\xi_s,- U_{t-s}f\big) ~ds\Big]
    = i\Pi_{r,x} [e^{\int_r^t\beta(\xi_u)du}f(\xi_t)].
\end{equation}
    Taking $r = 0$, we get that \eqref{eq: complex FKPP-equation} and \eqref{eq: complex FKPP-equation with FK-transform} are true if $f\in \mathcal B_b(E)$.

    The rest of the proof is to evaluate \eqref{eq: complex FKPP-equation} and \eqref{eq: complex FKPP-equation with FK-transform} for all $f\in L_2(\xi)$. We only do this for \eqref{eq: complex FKPP-equation} since the argument for \eqref{eq: complex FKPP-equation with FK-transform} is similar.
    Let $n \in \mathbb N$.
    Writing $f_n := (f^+ \wedge n) - (f^- \wedge n)$, then $f_n \xrightarrow[n\to \infty]{} f$ pointwise.
    From what we have proved, we have
\begin{equation}
\label{eq: complex FKPP-equation for fn}
    U_tf_n(x) - \Pi_{x} \Big[\int_0^t \psi\big(\xi_s, - U_{t-s}f_n\big) ~ds\Big]
    = i \Pi_{x} [f_n(\xi_t)].
\end{equation}
    Notice the following:
\begin{itemize}
\item
    It is clear that $\Pi_{x}[f_n(\xi_t)] \xrightarrow[n\to \infty]{} \Pi_{x}[f(\xi_t)]$.
\item
     $U_tf_n(x) \xrightarrow[n\to \infty]{} U_tf(x)$ due to \eqref{eq: N and characteristic exponent}, the dominated convergence theorem and the fact that
\[
    |e^{i W_t(f_n)} - 1| \leq W_t(|f|);
    \quad \mathbb N_x[W_t(|f|)] = (P_t^\beta |f|)(x) < \infty.
\]
\item
     $\Pi_{x} [\int_0^t \psi(\xi_s,- U_{t-s}f_n)ds] \xrightarrow[n\to \infty]{} \Pi_{x} [\int_0^t \psi(\xi_s,- U_{t-s}f)ds]$ due to the dominated convergence theorem, \eqref{eq: domination of psi(v)} and the fact (see \eqref{eq: upper bound of psi(v)}) that
\begin{align}
    \big|\psi(\xi_s,- U_{t-s}f_n)\big|
    \leq C_1 P_{t-s}^\beta|f|(\xi_s)+C_2 P_{t-s}^\beta|f|(\xi_s)^2.
\end{align}
\end{itemize}
    Using the above arguments, letting $n \to \infty$ in \eqref{eq: complex FKPP-equation for fn}, we get the desired result.
\end{proof}




\begin{thebibliography} {10}

\bibitem{BAM}
Berestycki, J., Kyrianou, A.E., Murillo-Salas, A
\newblock{\em The prolific backbone for supercritical superprocesses}. Stoch. Proc. Appl. 121, 1315-1331(2011)

\bibitem{Cuppens1975Decomposition}
Cuppens, R.:
\emph{Decomposition of multivariate probabilities.}
Probability and Mathematical Statistics, Vol. 29. Academic Press [Harcourt Brace Jovanovich, Publishers], New York-London, 1975.
\MR{0517412}

\bibitem{Durrett2010Probability}
Durrett, R.:
\emph{Probability: theory and examples.}
Fourth edition. Cambridge Series in Statistical and Probabilistic Mathematics, 31. Cambridge University Press, Cambridge, 2010.
\MR{2722836}

\bibitem{Dynkin1993Superprocesses}
Dynkin,~E.B.
\newblock {\em Superprocesses and partial differential equations}, Ann. Probab (1993): 1185-1262.

\bibitem{DK}
Dynkin, E. B., Kuznetsov, S. E.:
\newblock {\em $\mathbb{N}$-measure  for branching exit Markov system and their applications to differential equations}, Probab. Theory Related Fields 130(2004) 135-150

\bibitem{Kyprianou2014Fluctuations}
Kyprianou, A. E.:
\emph{Fluctuations of L\'{e}vy processes with applications.}
Introductory lectures. Second edition. Universitext. Springer, Heidelberg, 2014.
\MR{3155252}

\bibitem{Li2011Measure-valued}
Li, Z.:
\emph{Measure-valued branching Markov processes.}
Probability and its Applications (New York), Springer, Heidelberg, 2011.
\MR{2060602}

\bibitem{Linde1986Probability}
Linde, W.:
\emph{Probability in Banach spaces—stable and infinitely divisible distributions.}
Second edition. A Wiley-Interscience Publication. John Wiley \& Sons, Ltd., Chichester, 1986.

\bibitem{MM}
Marks, R., Mil\'{o}s, P.:
\newblock {\em CLT for supercritical branching processes with heavy-tailed branching law}.
\ARXIV{1803.05491}

\bibitem{GD}
Metafune,~G., Pallara,D.
\newblock {\em Specturm of Ornstein-Uhlenbeck operators in $\mathcal{L}^p$ space with respect to invariant measures}. J. Funct. Anal. 196, 40-60(2002)

\bibitem{RSZ}
Ren, Y.-X., Song, R., Zhang, R.:
\newblock {\em Central limit theorems for super Ornstein-Uhlenbeck processes}, Acta Appl. Math. 130(2014)9-49.

\bibitem{RenSongSun2017Spine}
Ren, Y.-X., Song, R., Sun, Z.:
\newblock{\em Spine decompositions and limit theorems for a class of critical superpeocesses,} arXiv preprint arXiv:1711.09188(2017).

\bibitem{Rudin1987Real}
Rudin, W.:
\emph{Real and complex analysis.}
Third edition. McGraw-Hill Book Co., New York, 1987.
\MR{0924157}

\bibitem{SchillingSongVondracek2010Bernstein}
Schilling, R., Song, R., Vondra\v{c}ek, Z.:
\emph{Bernstein functions. Theory and applications.}
De Gruyter Studies in Mathematics, 37. Walter de Gruyter \& Co., Berlin, 2010.
\MR{2598208}


\bibitem{Sato1999Levy}
Sato, K.:
\emph{Lévy processes and infinitely divisible distributions.}
Cambridge Studies in Advanced Mathematics, 68. Cambridge University Press, Cambridge, 1999.

\bibitem{SteinShakarchi2003Complex}
Stein, E. M. and Shakarchi, Rami:
\emph{Complex analysis.}
Princeton Lectures in Analysis, 2. Princeton University Press, Princeton, NJ, 2003.

\bibitem{LSR}
Liu, R.,  Ren, Y.-X., Song, R.:
\emph{ L Log L Criterion for a Class of Superdiffusions.}
Journal of Applied Probability, 46(2), 479-496. 2009.

\end{thebibliography}




\end{document}
