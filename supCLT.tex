% * The preamble
\documentclass[12pt,a4paper]{amsart}
\setlength{\textwidth}{\paperwidth}
\addtolength{\textwidth}{-2in}
\calclayout
\usepackage[utf8]{inputenc}
\usepackage[T1]{fontenc}
\usepackage{mathtools}
\mathtoolsset{showonlyrefs}
\usepackage{stackrel}
\usepackage{mathrsfs}
\usepackage{hyperref}
\usepackage{comment}
\usepackage{amsthm}
\theoremstyle{plain}
\newtheorem{thm}{Theorem}[section]
\newtheorem{lem}[thm]{Lemma}
\newtheorem{prop}[thm]{Proposition}
\newtheorem{cor}[thm]{Corollary}
\newtheorem{conj}[thm]{Conjecture}
\theoremstyle{definition}
\newtheorem{defi}[thm]{Definition}
\newtheorem{rem}[thm]{Remark}
\newtheorem{exa}[thm]{Example}
\newtheorem{asp}{Assumption}
\numberwithin{equation}{section}
\allowdisplaybreaks
% * Top matter
\begin{document}
\title
[stable CLT for super-OU processes]
{Stable Central Limit Theorems for Super Ornstein-Uhlenbeck Processes}
\author
[Y.-X. Ren, R. Song, Z. Sun and J. Zhao]
{Yan-Xia Ren, Renming Song, Zhenyao Sun and Jianjie Zhao}
\address{
  Yan-Xia Ren \\
  LMAM School of Mathematical Sciences \& Center for Statistical Science \\
  Peking University \\
  Beijing, P. R. China, 100871}
\email{yxren@math.pku.edu.cn}
\thanks{The research of Yan-Xia Ren is supported in part by NSFC (Grant Nos. 11671017  and 11731009) and LMEQF.}
\address{
  Renming Song \\
  Department of Mathematics \\
  University of Illinois at Urbana-Champaign \\
  Urbana, IL, USA, 61801}
\email{rsong@illinois.edu}
\thanks{The Research of Renming Song is support in part by a grant from the Simons Foundation (\#429343, Renming Song)}
\address{
  Zhenyao Sun \\
  School of Mathematics and Statistics\\
  Wuhan University \\
  Hubei, P. R. China, 100871}
\email{zhenyao.sun@gmail.com}
\address{
  Jianjie Zhao \\
  School of Mathematical Sciences \\
  Peking University \\
  Beijing, P. R. China, 100871}
\email{zhaojianjie@pku.edu.cn}
\thanks{Jianjie Zhao is the corresponding author}
\begin{abstract}
  % ADDED
  In this paper, we study the asymptotic behaviors of a supercritical $(\xi,\psi)$-superprocess $(X_t)_{t\geq 0}$
  % Let $\xi$ be an Ornstein-Uhlenbeck process on $\mathbb R^d$ with generator $L = \frac{1}{2}\sigma^2\Delta - b x \cdot \nabla$, where $\sigma, b >0$.
  whose underlying spatial motion $\xi$ is an Ornstein-Uhlenbeck process on $\mathbb R^d$ with generator $L = \frac{1}{2}\sigma^2\Delta - b x \cdot \nabla$ where $\sigma, b >0$;
  % Let $\psi$ be a branching mechanism which is close to a function of the form $\widetilde{\psi}(z)=-\alpha z +\rho z^2+ \eta z^{1+\beta}$ with $\alpha>0$, $\rho\ge 0$, $\eta>0$ and $\beta\in (0, 1)$, in some sense.
  and whose branching mechanism $\psi$ satisfies Grey's condition and some perturbation condition which guarantees that, while $z\to 0$, $\psi(z)=-\alpha z + \eta z^{1+\beta} (1+o(1))$ with $\alpha > 0$, $\eta>0$ and $\beta\in (0, 1)$.
  % In this paper, we study asymptotic behaviors of $(\xi, \psi)$-superprocesses $(X_t)_{t\geq 0}$.
  Some law of large numbers and $(1+\beta)$-stable central limit theorems are established for $(\langle f, X_t\rangle )_{t\geq 0}$ where the testing function $f$ is assumed to be of polynomial growth.
  % Conditioned on non-extinction, we establish some stable central limit theorems for $\langle f, X_t \rangle$ in three different regimes:
  % the small branching rate regime $\alpha\beta< \kappa_f b(1+\beta)$;
  % the critical branching rate regime $\alpha\beta = \kappa_f b(1+\beta)$;
  % and the large branching rate regime $\alpha\beta > \kappa_f b(1+\beta)$.
  A phase transition arises for the central limit theorems in the sense that three different type of results can be obtained which correspond to whether the branching rate $\alpha$ is relatively small, large or at a critical balanced value.
\end{abstract}
\subjclass[2010]{60J68, 60F05}
\keywords{Superprocesses, Ornstein-Uhlenbeck processes, Stable distribution, Central limit theorem, Law of large numbers, Branching rate regime}
\maketitle
% * Contents
% ** Introduction
\section{Introduction}
% *** Motivation
\subsection{Motivation}
\label{subsec:M}
Let $d \in \mathbb N:= \{1,2,\dots\}$ and $\mathbb R_+:= [0,\infty)$.
Let $\xi=\{(\xi_t)_{t\geq 0}; (\Pi_x)_{x\in \mathbb R^d}\}$ be an $\mathbb R^d$-valued Ornstein-Uhlenbeck process (OU process) with generator
\begin{align}
  Lf(x)
  = \frac{1}{2}\sigma^2\Delta f(x)-b x \cdot \nabla f(x)
  , \quad  x\in \mathbb R^d, f \in C^2(\mathbb R^d),
\end{align}
where $\sigma > 0$ and $b > 0$ are constants.
Let $\psi$ be a function on $\mathbb R_+$ of the form
\begin{align}
  \label{eq: honogeneou branching mechanism}
  \psi(z)
  =- \alpha z + \rho z^2 + \int_{(0,\infty)} (e^{-zy} - 1 + zy)~\pi(dy)
  , \quad  z \in \mathbb R_+,
\end{align}
where $\alpha > 0 $, $\rho \geq0$ and $\pi$ is a measure on $(0,\infty)$ with $\int_{(0,\infty)}(y\wedge y^2)~\pi(dy)< \infty$.
$\psi$ is referred to as the branching mechanism and $\pi$ is referred to as the L\'evy measure of $\psi$.
Denote by $\mathcal M(\mathbb R^d)$ the space of all finite Borel measures on $\mathbb R^d$.
For each $f,g\in \mathcal B(\mathbb R^d, \mathbb R)$ and $\mu \in \mathcal M(\mathbb R^d)$, write $\mu(f)=\langle f,\mu\rangle = \int f(x)\mu(dx)$ and $\langle f, g\rangle = \int f(x)g(x) dx$ whenever the integrals make sense.
We say a real-valued Borel function $f:(t,x)\mapsto f(t,x)$ on $\mathbb R_+\times \mathbb R^d$ is \emph{locally bounded} if, for each $t\in \mathbb R_+$, we have $ \sup_{s\in [0,t],x\in \mathbb R^d} |f(s,x)|<\infty. $
We say that an $\mathcal M(\mathbb R^d)$-valued Hunt process $X = \{(X_t)_{t\geq 0}; (\mathbb{P}_{\mu})_{\mu \in \mathcal M(\mathbb R^d)}\}$ 
% on a measurable space $(\Omega, \mathcal{F})$ 
on a measurable space $(\Omega, \mathscr{F})$ 
is a \emph{super Ornstein-Uhlenbeck process (super-OU process)} with branching mechanism $\psi$, or a $(\xi, \psi)$-superprocess, if for each non-negative bounded Borel function $f$ on $\mathbb R^d$, we have
\begin{align}
  \label{eq: def of V_t}
  \mathbb{P}_{\mu}[e^{-\langle f,X_t \rangle}]
  = e^{-\langle V_tf, \mu \rangle}
  , \quad t\geq 0, \mu \in \mathcal M(\mathbb R^d),
\end{align}
where $(t,x) \mapsto V_tf(x)$ is the unique locally bounded non-negative solution to the equation
\begin{align}
  V_tf(x) + \Pi_x \Big[ \int_0^t\psi\big(V_{t-s}f(\xi_s)\big)~ds\Big]
	= \Pi_x [f(\xi_t)]
  , \quad x\in \mathbb R^d, t\geq 0.
\end{align}	
The existence of such super-OU process $X$ is well known, see \cite{Dynkin1993Superprocesses} for instance.

Recently, there have been quite a few papers on laws of large numbers for superdiffusions.
In \cite{Englander2009Law, EnglanderWinter2006Law, EnglanderTuraev2002A-scaling}, some weak laws of large numbers (convergence in law or in probability) were established.
The strong law of large numbers for superprocesses was first studied in \cite{ChenRenWang2008An-almost}, followed by \cite{ChenRenSongZhang2015Strong-law, ChenRenYang2019Skeleton, EckhoffKyprianouWinkel2015Spines, KouritzinRen2014A-strong, LiuRenSong2013Strong, Wang2010An-almost} under different settings.
For a good survey on recent developments in laws of large numbers for branching Markov processes and superprocesses, see \cite{EckhoffKyprianouWinkel2015Spines}.

The strong law of large numbers for the super-OU process $X$ introduced at the beginning can be stated as follows:
Under some conditions on the branching mechanism $\psi$ (these conditions are satisfied under our Assumptions 1 and 2 below), there exists an $\Omega_0$ of $\mathbb{P}_\mu$-full probability for every $\mu\in\mathcal M(\mathbb R^d)$ such that on $\Omega_0$, for every Lebesgue-a.\/e. continuous bounded non-negative function $f$ on $\mathbb R^d$, we have $\lim_{t\to\infty} e^{-\alpha t} \langle f, X_t\rangle =H_\infty\langle f, \varphi\rangle $, where $H_\infty$ is the limit of the martingale $e^{-\alpha t}\langle 1,X_t\rangle$ and $\varphi$ is the invariant density of the OU process $\xi$ defined in \eqref{invariantdensity} below.
See \cite[Theorem 2.13 \& Example 8.1]{ChenRenYang2019Skeleton} and \cite[Theorem 1.2 \& Example 4.1]{EckhoffKyprianouWinkel2015Spines}.

In this paper, we will establish some spatial central limit theorems (CLTs) for the super-OU process $X$ above.
% Our key assumption is that the branching mechanism $\psi(z)$ takes the form of $-\alpha z +\rho z^2+ \psi_1(z)$, where $\psi_1$ is close to $\eta z^{1+\beta}$ with $\eta>0$ and $\beta\in (0, 1)$, in some sense.
Our key assumption is that the branching mechanism $\psi$ satisfies Grey's condition and some perturbation condition which guarantees that, while $z\to 0$, $\psi(z)=-\alpha z + \eta z^{1+\beta} (1+o(1))$ with $\alpha > 0$, $\eta>0$ and $\beta\in (0, 1)$.
The general goal is to find $(F_t)_{t\geq 0}$ and $(G_t)_{t\geq 0}$ so that $ (\langle f, X_t \rangle -G_t)/F_t $ converges weakly to some non-degenerate random variable as $t\rightarrow\infty$, for a large class of functions $f$.
Note that, in the setting of this paper, $\langle f,X_t\rangle$ typically have infinite second moment.

There are many papers studying CLTs for branching processes, branching diffusions and superprocesses under the second moment condition.
See \cite{Heyde1970A-rate, HeydeBrown1871An-invariance, HeydeLeslie1971Improved} for supercritical Galton-Watson processes (GW processes), \cite{KestenStigum1966Additional,KestenStigum1966A-limit} for supercritical multi-type GW processes, \cite{Athreya1969Limit,Athreya1969LimitB,Athreya1971Some} for supercritical multi-type continuous time branching processes and \cite{AsmussenHering1983Branching} for general supercritical branching Markov processes under certain conditions.
Some spatial CLTs for supercritical branching OU processes with binary branching mechanism were proved in \cite{AdamczakMilos2015CLT} and some spatial CLTs for supercritical super-OU processes with branching mechanisms satisfying a fourth moment condition were proved in \cite{Milos2012Spatial}.
These two papers made connections between the CLTs and the branching rate regimes.
Some spatial central limit theorems for supercritical super-OU  processes with branching mechanisms satisfying only a second moment condition were established in \cite{RenSongZhang2014Central}.
Moreover, compared with the results of \cite{AdamczakMilos2015CLT,Milos2012Spatial}, the limit distributions in \cite{RenSongZhang2014Central} are non-degenerate.
Since then, a series of spatial CLTs for a large class of general supercritical branching Markov processes and superprocesses with spatially dependent branching mechanisms were proved in \cite{RenSongZhang2014CentralB,RenSongZhang2015Central,RenSongZhang2017Central}.
The functional version of the CLTs was established in \cite{Janson2004Functional} for supercritical multitype branching processes, and in \cite{RenSongZhang2017Functional} for supercritical superprocesses.

There are also many limiting theorem type results for supercritical branching processes and branching Markov processes with branching mechanisms of infinite second moments.
Heyde \cite{Heyde1971Some} established some  CLTs for supercritical GW processes when the offspring distribution belongs to the domain of attraction of a stable law of index $\alpha\in (1, 2]$, and proved that the limit laws are stable laws.
Similar results  for supercritical multi-type GW processes and supercritical continuous time branching processes,
under some $p$-th ($p\in(1,2]$) moment condition on the offspring distribution, were given in Asmussen \cite{Asmussen76Convergence}.
Recently, Marks and Milo\'s \cite{MarksMilos2018CLT} considered the limit behavior of supercritical branching OU processes with a special stable branching law.
They established some spatial CLTs in the small and critical branching rate regimes, but they did not prove any central limit theorem type result in the large branching rate regime.
We also mention here that very recently \cite{IksanovKoleskoMeiners2018Stable-like} considered stable fluctuations of Biggins' martingales in the context of branching random walks and \cite{RenSongSun2018Limit} considered the asymptotic behavior of a class of critical superprocesses with spatially dependent stable branching.

As far as we know, this paper is the first to study spatial central limit theorems for supercritical superprocesses without the second moment condition.

% *** Main results
\subsection{Main results}
\label{sec:I:R}
% **** Assumption 1
We will always assume that the following assumption holds.
\begin{asp}
  \label{asp: Greys condition}
  The branching mechanism satisfies Grey's condition, i.e., there exists $z' > 0$ such that $\psi(z) > 0$ for all $z>z'$ and  $\int_{z'}^\infty \psi(z)^{-1}dz < \infty$.
\end{asp}
For each $\mu \in \mathcal M(\mathbb R^d)$, write $\|\mu\| = \langle 1, \mu\rangle$.
It is known (see \cite[Theorems 12.5 \& 12.7]{Kyprianou2014Fluctuations} for example) that, under Assumption \ref{asp: Greys condition}, the \emph{extinction event} $D :=\{\exists t\geq 0,~\text{s.t.}~ \|X_t\| =0 \}$ has positive probability with respect to $\mathbb P_\mu$ for each  $\mu \in \mathcal M(\mathbb R^d)$.
% In fact, $ \mathbb{P}_{\mu} (D) = e^{-\bar v \|\mu\|}$ for each $\mu\in \mathcal M(\mathbb R^d)$ where $ \bar v := \sup\{\lambda \geq 0: \psi(\lambda) = 0\} \in (0,\infty) $ is the largest root of $\psi$.
In fact, $ \mathbb{P}_{\mu} (D) = e^{-\bar v \|\mu\|}$ where $ \bar v := \sup\{\lambda \geq 0: \psi(\lambda) = 0\} \in (0,\infty) $ is the largest root of $\psi$.

% **** Assumption 2
Denote by $\Gamma$ the gamma function.
For any $\sigma$-finite signed measure $\mu$, we use $|\mu|$ to denote the total variation measure of $\mu$.
In this paper, we will also assume the following:
\begin{asp}
  \label{asp: branching mechanism}
  There exist constants $\eta > 0$ and $\beta \in (0,1)$ such that
  \begin{align}
    \label{eq: asp of branching mechanism}
    \int_{(1,\infty)}y^{1+\beta +\delta}~\Big|\pi(dy)-\frac{\eta~dy}{\Gamma(-1-\beta)y^{2+\beta}}\Big| <\infty,
  \end{align}
	for some $\delta > 0$.
\end{asp}
We will show in Subsection \ref{sec: branching mechanism} that if Assumption \ref{asp: branching mechanism} holds, then $\eta$ and $\beta$ are uniquely determined by the L\'evy measure $\pi$.
In the reminder of the paper, we will always use $\eta$ and $\beta$ to denote the constants in Assumption  \ref{asp: branching mechanism}.
	Note that $\delta$ is not uniquely determined by $\pi$.
	In fact, if $\delta>0$ is a constant such that \eqref{eq: asp of branching mechanism} holds, then replacing $\delta$ by any smaller positive number, \eqref{eq: asp of branching mechanism} still holds.
	Therefore, Assumption \ref{asp: branching mechanism} is equivalent to the following statement:
	There exist constants $\eta > 0$ and $\beta \in (0,1)$ such that, for each small enough $\delta>0$, \eqref{eq: asp of branching mechanism} holds.

\begin{rem}
  \label{rem:SP}
% Assumption \ref{asp: branching mechanism} says that there exist constants $\eta>0$ and $\beta > 0$ such that the L\'evy measure $\pi(dy)$ is ``not too far away'' from the measure $\eta \Gamma(-1-\beta)^{-1}y^{-2-\beta} dy$.
Roughly speaking, Assumption \ref{asp: branching mechanism} says that the branching mechanism $\psi$ is ``not too far away from $\widetilde \psi(z) := - \alpha z + \eta z^{1+\beta}$ near $0$''.
 % In particular, if $\pi(dy)$ is equal to $\eta \Gamma(-1-\beta)^{-1}y^{-2-\beta} dy$, then the branching mechanism $\psi(z)$ takes the form of $-\alpha z + \rho z^2 + \eta z^{1+\beta}$.
In fact, if we consider their differences
\begin{align}
  \label{eq:PB}
  & \psi_1(z)
  := \psi(z) - \widetilde \psi(z)
  \\ &= \rho z^2+ \int_{(0,\infty)}(e^{-yz}-1+yz) \Big(\pi(dy) - \frac{\eta~dy}{\Gamma(-1-\beta) y^{2+\beta}}\Big),
  \quad z\geq 0,
\end{align}
then it can be verified that (see Lemma \ref{lem:CEP} below) $\psi_1(z)/z^{1+\beta} \xrightarrow[z\to 0]{} 0$.
Therefore, we can write $ \psi(z)  = - \alpha z + z^{1+\beta}(\eta + o(1))$ as $z\to 0$.
One can further write that $\psi(z) = - \alpha z + z^{1+\beta} l(z)$ where $l$ is a function on $[0,\infty)$ which is slowly varying at $0$.
% END ADDED
\end{rem}

\begin{rem}
It will be proved in Lemma \ref{lem: LlogL criterion} that, under Assumption \ref{asp: branching mechanism}, the branching mechanism $\psi$ satisfies the $L \log L$ condition, i.e., $ \int_{(1,\infty)} y\log y~\pi(dy) < \infty. $
This guarantees that $H_\infty$, the limit of the non-negative martingale $(e^{-\alpha t} \|X_t\|)_{t\geq 0}$, is non-degenerate.
\end{rem}
% **** Mean semigroup
% ADDED
Let us introduce some notation in order to give a precise formulation of our main result.
% END ADDED
Denote by $\mathcal B(\mathbb R^d, \mathbb R)$ the space of all $\mathbb R$-valued Borel functions on $\mathbb R^d$.
Denote by $\mathcal B(\mathbb R^d, \mathbb R_+)$ the space of all $\mathbb R_+$-valued Borel functions on $\mathbb R^d$.
We use  $(P_t)_{t\geq 0}$ to denote the transition semigroup of $\xi$.	
Define
\(  
P^{\alpha}_t f(x)
  := e^{\alpha t} P_t f(x)
  = \Pi_x [e^{\alpha t}f(\xi_t)]
\)
for each $x\in \mathbb R^d$, $t\geq 0$ and $f\in \mathcal B(\mathbb R^d, \mathbb R_+)$.
It is known that, see \cite[Proposition 2.27]{Li2011Measure-valued} for example, $(P^\alpha_t)_{t\geq 0}$ is the \emph{mean semigroup} of $X$ in the sense that
\( 
  \mathbb{P}_{\mu}[\langle f, X_t \rangle]
  = \langle P^\alpha_t f, \mu \rangle,
\)
for each $\mu\in \mathcal M(\mathbb R^d)$, $t\geq 0$ and $f\in \mathcal B(\mathbb R^d, \mathbb R_+)$.
% The limiting behavior of the super-OU process is closely related to the asymptotic property of this mean semigroup $(P^\alpha_t)_{t\geq 0}$, and therefore, to the property of the OU semigroup $(P_t)_{t\geq 0}$.

% **** Spectral Property
The limiting behavior of the super-OU process is closely related to the spectral property of the OU semigroup $(P_t)_{t\geq 0}$ which we now recall (See \cite{MetafunePallaraPriola2002Spectrum} for more details).
% It is known that the OU process $\xi$ has an invariant density
It is known that the OU process $\xi$ has an invariant probability on $\mathbb R^d$
\begin{align}
  \label{invariantdensity}
  \varphi(x)dx
  :=\Big (\frac{b}{\pi \sigma^2}\Big )^{d/2}\exp \Big(-\frac{b}{\sigma^2}|x|^2 \Big)dx
  % \quad x\in \mathbb R^d.
\end{align}
which is a   symmetric multivariate Gaussian distribution.
% Let \[L^2(\varphi):= \Big\{ h  \in \mathcal B(\mathbb R^d, \mathbb R): \int_{\mathbb R^d} |h(x)|^2 \varphi(x) dx < \infty \Big\}.\]
% Then $L^2(\varphi)$ is a Hilbert space with inner product
Let $L^2(\varphi)$ be the Hilbert space with inner product
\begin{align}
  \langle f_1, f_2 \rangle_{\varphi}
  := \int_{\mathbb R^d}f_1(x)f_2(x)\varphi(x) dx, \quad f_1,f_2 \in L^2(\varphi).
\end{align}
Let $\mathbb Z_+ := \mathbb N\cup\{0\}$.
For each $p = (p_k)_{k = 1}^d \in \mathbb{Z}_+^{d}$, write $|p|:=\sum_{k=1}^d p_k$, $p!:= \prod_{k= 1}^d p_k!$ and $\partial_p:= \prod_{k = 1}^d(\partial/\partial x_k^{p_k})$.
The \emph{Hermite polynomials} are defined by
\begin{align}
  H_p(x)
  :=(-1)^{|p|}\exp(|x|^2) \partial_p \exp(-|x|^2)
  , \quad x\in \mathbb R^d, p \in \mathbb{Z}_+^{d}.
\end{align}
It is known that $(P_t)_{t\geq 0}$ is a strongly continuous semigroup in $L^2(\varphi)$ and its generator $L$ has discrete spectrum $\sigma(L)= \{-bk: k \in \mathbb Z_+\}$.
For each $k \in \mathbb Z_+$, denote by $\mathcal{A}_k$ the eigenspace corresponding to the eigenvalue $-bk$, then $ \mathcal{A}_k = \operatorname{Span} \{\phi_p : p\in \mathbb Z_+^d, |p|=k\}$, where
\begin{align}
  \label{eigenfunction}
  \phi_p(x)
  := \frac{1}{\sqrt{ p! 2^{|p|} }} H_p \Big(\frac{ \sqrt{b} }{\sigma}x \Big)
  , \quad x\in \mathbb R^d, p\in \mathbb Z_+^d.
\end{align}
In other words,
\(  
  P_t\phi_p(x)
  = e^{-b|p|t}\phi_p(x)
\)
for each $t\geq 0$, $x\in \mathbb R^d$ and $p\in \mathbb Z_+^d$.
Moreover, $\{\phi_p: p \in \mathbb Z_+^d\}$ forms a complete orthonormal basis of $L^2(\varphi)$.
Thus for each $f\in L^2(\varphi)$, we have
\begin{align}
  \label{semicomp1}
  f
  = \sum_{k=0}^{\infty}\sum_{p\in \mathbb Z_+^d:|p|=k}\langle f, \phi_p \rangle_{\varphi} \phi_p
  , \quad \text{in~} L^2(\varphi).
\end{align}
For each function $f\in L^2(\varphi)$, define the order of $f$ as
\[
  \kappa_f
  := \inf \left \{k\geq 0: \exists ~ p\in \mathbb Z_+^d , {\rm ~s.t.~} |p|=k {\rm ~and~}  \langle f, \phi_p \rangle_{\varphi}\neq 0\right \}
\]
which is the lowest non-trivial frequency in the eigen-expansion \eqref{semicomp1}.
Note that $ \kappa_f\geq 0$ and that, if $f\in L^2(\varphi)$ is non-trivial, then $\kappa_f<\infty$.
In particular, the order of any constant non-zero function is zero.

% **** Martingale limits
Denote by $\mathcal M_c(\mathbb R^d)$ the space of all finite Borel measures of compact support on $\mathbb R^d$.
For each $p\in \mathbb{Z}_+^d$, define
\(
  H_t^p
  := e^{-(\alpha-|p|b)t}\langle\phi_p,X_t\rangle
\)
with $t\geq 0$.
If $\alpha\beta>|p|b(1+\beta)$, then for all $\gamma\in (0, \beta)$ and $\mu\in \mathcal M_c(\mathbb R^d)$, we will prove in Lemma \ref{lem:M:L:ML} that $(H_t^p)_{t\geq 0}$ is a $\mathbb{P}_{\mu}$-martingale bounded in $L^{1+\gamma}(\mathbb{P}_{\mu})$.
Thus the limit $H^p_{\infty}:=\lim_{t\rightarrow \infty}H_t^p$ exists $\mathbb{P}_{\mu}$-almost surely and in $L^{1+\gamma}(\mathbb{P}_{\mu})$.

We first present a law of large numbers for our model which extends the strong laws of large numbers of \cite{ChenRenYang2019Skeleton, EckhoffKyprianouWinkel2015Spines} in which the first order asymptotic ($\kappa_f=0$) was identified.
Denote by $\mathcal P$ the class of functions of polynomial growth on $\mathbb R^d$, i.e.,
\begin{align}
  \label{eq: polynomial growth function}
  \mathcal{P}
  := \{f\in \mathcal B(\mathbb R^d, \mathbb R):\exists C>0, n \in \mathbb Z_+ \text{~s.t.~} \forall x\in \mathbb R^d, |f(x)|\leq C(1+|x|)^n \}.
\end{align}
It is clear that $\mathcal{P} \subset L^2(\varphi)$.
\begin{thm}
  \label{thm: law of large number}
  If $f \in \mathcal{P}$ satisfies $\alpha\beta>\kappa_fb(1+\beta)$, then for all $\gamma\in (0, \beta)$ and  $\mu\in \mathcal M_c(\mathbb R^d)$,
  \[
    e^{-(\alpha-\kappa_fb)t}\langle f, X_t\rangle
    \xrightarrow[t\to \infty]{}\sum_{p\in \mathbb Z_+^d:|p|=\kappa_f}\langle f, \phi_p\rangle_{\varphi} H_{\infty}^p
    \quad in~ L^{1+\gamma}(\mathbb{P}_{\mu}).
  \]
  Moreover, if $f$ is twice differentiable and all its second order partial derivatives are in $\mathcal{P}$, then we also have almost sure convergence.
\end{thm}
If $f\in \mathcal B(\mathbb R^d, \mathbb R_+)$ is non-trivial and  bounded, then $\kappa_f=0$.
Hence, Theorem \ref{thm: law of large number} says that for any $\gamma\in (0, \beta)$ and  $\mu\in \mathcal M_c(\mathbb R^d)$, as $t\rightarrow \infty$,
\(
  e^{-\alpha t}\langle f, X_t\rangle
  \rightarrow \langle f, \varphi\rangle H_{\infty}
\)
in $L^{1+\gamma}(\mathbb{P}_{\mu})$.
Moreover, if $f$ is twice differentiable and all its second order partial derivatives are in $\mathcal{P}$, then we also have a.s.\ convergence.
However, to get a.s.\ convergence for bounded non-negative
Lebesgue-a.e.\ continuous functions $f$, we do not need $f$ to be twice differentiable. 
See \cite[Theorem 2.13 \& Example 8.1]{ChenRenYang2019Skeleton} and \cite[Theorem 1.2 \& Example 4.1]{EckhoffKyprianouWinkel2015Spines}.

% **** Stable CLT
% For the rest of this subsection, we focus on the central limit theorems of $\langle f, X_t\rangle$ for $f\in \mathcal P\setminus \{0\}$.
For the rest of this subsection, we focus on the central limit theorems of $\langle f, X_t\rangle$ for some $f\in \mathcal P\setminus \{0\}$.
Write $\tilde u = \frac{u}{ 1+ u}$ for each $u \neq -1$.
It turns out that there is a phase transition in the sense that the results are different in the following three cases:
\begin{enumerate}
\item
  the small branching rate case where $f \in \mathcal C_s := \mathcal P \cap \operatorname{Span} \{ \phi_p: \alpha \tilde \beta < |p| b \}$;
\item
  the critical branching rate case where $f \in \mathcal C_c   := \mathcal P \cap \operatorname{Span} \{ \phi_p : \alpha \tilde \beta = |p| b \} $; and
\item
  the large branching rate case  where $f \in \mathcal C_l   := \mathcal P \cap \overline{\operatorname{Span}} \{ \phi_p: \alpha \tilde \beta > |p| b \}$.
\end{enumerate}
Here, the small (rspt. large) branching means that the branching rate $\alpha$ is relatively small (rspt. relatively large); and the critical branching means that the branching rate $\alpha$ is at a critical balanced value.
To present our result, let us define a family of operators $(T_t)_{t\geq 0}$ on $\mathcal P$ by
\begin{align}
  \label{eq:I:R:1}
  T_t f
  := \sum_{p \in \mathbb Z_+^d} e^{-| |p|b - \alpha \tilde \beta |t} \langle f, \phi_p \rangle_{\varphi} \phi_p,
  ,\quad t\geq 0, f\in \mathcal P,
\end{align}
and a family of $\mathbb C$-valued functionals $(m_t)_{0 \leq s \leq t < \infty}$ on $\mathcal P$ by
\begin{align}
  \label{eq:I:R:2}
  m_t[f]
  := \eta \int_0^t ~du \int_{\mathbb R^d} (-iT_u f(x))^{1+\beta} \varphi(x) ~dx
  , \quad 0 \leq t< \infty, f\in \mathcal P.
\end{align}
For each $f\in \mathcal P\setminus \{0\}$, in Lemma \ref{lem:m} and Proposition \ref{prop:PL:S} below we will show that
\begin{align}
  \label{eq:I:R:3}
  m[f]
  := \begin{cases}
    \lim_{t\to \infty} m_t[f], &
    f \in \mathcal C_s \oplus \mathcal C_l, \\
    \lim_{t\to \infty} \frac{1}{t} m_t[f], & f\in \mathcal P \setminus \mathcal C_s \oplus \mathcal C_l,
  \end{cases}
\end{align}
is well defined, and moreover, there exists an $(1+\beta)$-stable random variable $\zeta^f$ with characteristic function $\theta \mapsto e^{m[\theta f]}$.
Let us now give the main results of this paper.

\begin{thm}
  \label{thm:M}
  Let $\mu\in \mathcal M_c(\mathbb R^d)\setminus \{0\}$.
  Then under $\mathbb{P}_{\mu}(\cdot|D^c)$, the follows hold:
\begin{enumerate}
\item
  \label{thm:M:1}
  if $f\in \mathcal C_s\setminus\{0\}$, then $\|X_t\|^{- \frac{1}{1+\beta}} X_t(f)  \xrightarrow[t\to \infty]{d} \zeta^f$;
\item
  \label{thm:M:2}
  if $f\in \mathcal C_c\setminus\{0\}$, then $ \|t X_t\|^{-\frac{1}{1+\beta}} X_t(f) \xrightarrow[t\to \infty]{d} \zeta^f$; 
\item
  \label{thm:M:3}
  if $f\in \mathcal C_l\setminus\{0\}$, then
  \[
    \|X_t\|^{-\frac{1}{1+\beta}} \Big( X_t(f) - \sum_{p\in \mathbb Z^d_+:\alpha \tilde \beta>|p|b}\langle f,\phi_p\rangle_\varphi e^{(\alpha-|p|b)t}H^p_{\infty}\Big)
    \xrightarrow[t\to \infty]{d}
    \zeta^{-f}.
  \]
\end{enumerate}
\end{thm}

% ADDED
At this point, we should mention that the CLTs above did not cover for all polynomial growth testing functions $f\in \mathcal P$.
Indeed, considering a general $f \in \mathcal P$ which can be decomposed as $f_s + f_c + f_l$ with $f_s \in \mathcal C_s$, $f_c \in \mathcal C_c\setminus\{0\}$ and $f_l \in \mathcal C_l$, as a simple corollary of Theorem \ref{thm:M}, one can verify that 
\[
  \|tX_t\|^{-\frac{1}{1+\beta}} \Big( X_t(f) - \sum_{p\in \mathbb Z^d_+:\alpha \tilde \beta>|p|b}\langle f,\phi_p\rangle_\varphi e^{(\alpha-|p|b)t}H^p_{\infty}\Big)
  \xrightarrow[t\to \infty]{d}
  \zeta^{f_c}
\]
for each $\mathbb P_\mu(\cdot | D^c)$ with $\mu \in \mathcal M_c(\mathbb R^d)\setminus \{0\}$.
However, in the case that $f_c\equiv 0$, $f_s\not \equiv 0$ and $f_l\not \equiv 0$, it seems hard to establish any non-trivial CLTs for $X_t(f)$ using the methodology of this paper.  
In general, we conjecture that under the condition of Theorem \ref{thm:M} the random vector
\begin{align}
  \|X_t\|^{- \frac{1}{1+\beta}} \Big( X_t(f_s) ,t X_t(f_c),  X_t(f_l) - \sum_{p\in \mathbb Z^d_+:\alpha \tilde \beta>|p|b}\langle f_l,\phi_p\rangle_\varphi e^{(\alpha-|p|b)t}H^p_{\infty} \Big)
\end{align}
convergences in distribution to $(\zeta^{f_s},\zeta^{f_c}, \zeta^{-f_l})$ where $\zeta^{f_s}$, $\zeta^{f_c}$ and $\zeta^{-f_l}$ are independent.
Similar convergence has already been proved under the second moment condition, see \cite{RenSongZhang2015Central}.

Let us now give some intuitive explanation of this phase transition. 
First we mention that the super-OU process arises as the ``high density'' limit of the superposition of a sequence of branching-OU processes, see \cite{Li2011Measure-valued} for example.
And similar phase transition have already been discussed in the context of branching-OU processes, see \cite{MarksMilos2018CLT}. 
Though it is hard to describe rigorously, one can consider superprocess as a branching particle system with infinitely many infinitesimal particles, and we employ this particle picture only for the sake of this interpretation. 
It is believed that this phase transition is due to an interplay of two competing effects in the system: coarsening and smoothing. 
The coarsening effect corresponds to the increasing of the spatial inequality and is a consequence of the branching: simply an area with more particles will produce more offspring.
The smoothing effect corresponds to the decreasing of the spatial inequality and is a consequence of the mixing property of the OU processes: each OU particles will ``forget'' its initial position exponentially fast.

Let us consider $X_t(\phi_p)$ as an example, and discuss how parameters $\alpha, \beta, b$ and $|p|$
influences those two effects:
\begin{itemize}
\item
  The branching rate $\alpha$ captures the mean intensity of the branching in the system. 
  Therefore, lager the branching rate $\alpha$, stronger the coarsening effect.
\item
  The tail index $\beta$ describes the heaviness of the tail of the offspring numbers which belongs to the attraction of some $(1+\beta)$-stable random variable. 
When $\beta$ is smaller i.e. the tail is heavier, then it is more likely that an area with few particles can suddenly have a large amount of offspring. 
In other words, larger the tail index $\beta$, stronger the coarsening effect. 
\item
  The drift parameter $b$ captures the level of the mixing property of the OU particles.
  Larger the drift parameter $b$, faster the OU-particles forgetting their initial position, and therefore stronger the smoothing effect.
\item
  The order $|p|$ describes the ability of the testing function $\phi_p$ while capturing the mixing property of the OU particles. 
  In particular, in the case that $|p| = 0$, no mixing property can be captured by $\phi_p \equiv 1$ since we are only considering the total mass $\|X_t\|$. 
  In general, higher the order $|p|$, more mixing property can be captured by the testing function $\phi_p$, and therefore stronger the smoothing effect.
\end{itemize} 
We also discuss the role of other parameters $\rho, \eta$ and $\sigma$ in our model:
\begin{itemize}
\item
  The coefficient $\rho$ dose not influences the results since the second term $\rho z^2$ in the branching mechanism $\psi$ is a part of the small perturbation $\psi_1$ (See Remark \ref{rem:SP}). 
\item
  The coefficients $\eta$ and $\sigma$ are hidden in the definition of the functional $m[f]$, and therefore influence the actual distribution of the limiting $(1+\beta)$-stable random variable $\xi^f$.
  Their role in the coarsening and smoothing effects are negligible compared to the four parameters $\alpha, \beta, b$ and $|p|$ mentioned above.
\end{itemize}
% END ADDED


% *** An outline of the methodology
\subsection{An outline of the methodology}
% Here we give some intuitive explanation of the central limit theorems above.
Let us give some intuitive explanation of the methodology used in this paper.
% Deleted
% The main ideas are similar to those given in \cite{MarksMilos2018CLT} for branching OU processes.
% End delted
For any $\mu\in \mathcal M_c(\mathbb R^d)$ and any random variable $Y$ with finite mean, we define
$
  \mathcal I_s^t Y
  := \mathcal I_s^t [Y, \mu]
  := \mathbb P_\mu[Y|\mathscr F_t] - \mathbb P_\mu[Y|\mathscr F_s]
$
  where $0 \leq s \leq t <\infty.$
We will use the shorter notation $\mathcal I_s^t Y$ when there is no danger of confusion.
For each $f\in \mathcal{P}$, consider the following decomposition over the time interval $[0,t]$:
\begin{align}
  % \langle f,X_t\rangle
  X_t(f)
  := \sum_{k=0}^{\lfloor t \rfloor-1} \mathcal I_{t-k-1}^{t-k} X_t (f)+\mathcal I_0^{t-\lfloor t \rfloor} X_t(f) + X_0( P^\alpha_tf),
  \quad t\geq 0.
\end{align}
To find the fluctuation of $\langle f,X_t\rangle$, we will investigate the fluctuation of each term on the right hand side above.
% DELETED
% Recall $\|X_t\|\sim e^{\alpha t}$ as $t\to\infty$.
% END DELETED
The second term and third term are negligible after the rescaling, and for the first term we will establish some multi-variant unite-interval CLTs which says that
\[
  \Big( \|X_t\|^{-\frac{1}{1+\beta}}\mathcal I^{t-k}_{t-k-1} X_t(f) \Big)_{k=0}^n
  \xrightarrow [t\to \infty]{d} (\zeta^f_k)_{k=0}^n,
\]
where $(\zeta_k)_{k \in \mathbb N}$ are some independent $(1+\beta)$-stable random variables.
% So if $f\in \mathcal C_s$, i.e., $\alpha \tilde \beta < \kappa_f b$, we have roughly that
% \[
%  \frac{\langle f,X_t\rangle}{\|X_t\|^{\frac{1}{1+\beta}}}
%  \xrightarrow[t\to \infty]{d} \zeta\overset{d}{=}\sum_{k=0}^\infty \zeta_k,
% \]
If $f \in \mathcal C_s\setminus\{0\}$, then it can be argued that $\sum_{k=0}^{\lfloor t \rfloor} \xi^f_k \xrightarrow[t\to \infty]{d} \zeta^f$ and then intuitively we have 
\(
  \|X_t\|^{-\frac{1}{1+\beta}}  X_t(f) 
  \xrightarrow[t\to \infty]{d} \zeta^f.
  \)
If $f \in \mathcal C_c \setminus \{0\}$, then it can be argued that 
\(
t^{-\frac{1}{1+\beta}} \sum_{k=0}^{\lfloor t\rfloor} \zeta_k \xrightarrow[t\to \infty]{ d} \zeta^f
\)
and then intuitively we have 
\(
\|tX_t\|^{-\frac{1}{1+\beta}}  X_t(f) 
\xrightarrow[t\to \infty]{d} \zeta^f.
\)
% DELETED
\begin{comment}
In the explanation above, we have used the facts that $(\zeta_k)_{k\in \mathbb N}$, $\zeta$ are all well defined, and
\begin{align}
  \label{eq: equatlity for mf for small rate}
  m[f]
  = \sum_{k=0}^\infty \langle Z_1 T_k \tilde f,\varphi \rangle,\quad f\in\mathcal C_s.
\end{align}
These facts will be made clear in Lemma \ref{lem:m} and formula \eqref{eq:PM:CLTS:2}.

If $f\in \mathcal C_c$, i.e., $\alpha \tilde \beta = \kappa_f b$,
then
\[
  m[f]
  = \lim_{t\rightarrow \infty}\frac{1}{t}\sum_{k=0}^{\lfloor t \rfloor}\langle Z_1 T_k \tilde f,\varphi \rangle, \quad f\in \mathcal C_c.
\]
In this case, we roughly have that
\[
 	\frac{\langle f,X_t\rangle}{(t\|X_t\|)^{\frac{1}{1+\beta}}}
  \overset{d}{\approx} \frac{1}{t^{\frac{1}{1+\beta}}} \sum_{k=0}^{\lfloor t\rfloor} \zeta_k
  \xrightarrow[t\to \infty]{d} \zeta,
\]
where $\zeta$ is a $(1+\beta)$-stable random variable with characteristic function $\theta\mapsto \exp( m[\theta f])$.
\end{comment}
% END DELETED
% For $f\in \mathcal C_l$, the general idea is almost the same, except that we need to consider the decomposition over the time interval $[t,\infty)$.
If $f\in \mathcal C_l$, the general idea is almost the same, except that we need to consider the decomposition over the time interval $[t,\infty)$.
\begin{comment}
Taking $f = \phi_p$ with $\alpha \tilde \beta > |p|b$ as an example, we do the following decomposition:
\[
  H^p_t-H^p_\infty
  =\sum^{\infty}_{n=1}(H^p_{t+n-1}-H^p_{t+n}).
\]
The fluctuation behaviors of each of the terms $H_{t+n-1}^p - H_{t+n}^p$ and their asymptotic independence will also be established in Theorem \ref{lem:PR:LC} and Corollary \ref{cor:MI} below.
These lemmas will eventually lead us to the fluctuation result in the large branching rate regime.
\end{comment}

\begin{comment}
For a general $f$, we have a unique decomposition: $f=f_l+f_c+f_s$ with $f_l\in\mathcal C_l$, $f_c\in\mathcal C_c$ and $f_s\in\mathcal C_c$.
Note that there may be no $p$ such that $|p|=\frac{\alpha \tilde\beta}{b}$. In this case $f_c=0$.
Our main results above give central limit type results for $\langle f_l, X_t\rangle$, $\langle f_c, X_t\rangle$ and $\langle f_s, X_t\rangle$, respectively.
We conjecture that the limits of these three terms, normalized properly, are independent, because intuitively these limits come from small time intervals, intermediate time intervals and large time intervals, respectively.
If this is valid, we can get a central limit type result for $\langle f, X_t\rangle$ for general $f\in\mathcal{P}$.
This independence was proved under the second moment condition, see \cite{RenSongZhang2015Central}.
We leave the question of independence for stable branching mechanism to a future project.
\end{comment}

This paper is our first attempt on the stable CLT for superprocesses.
There are still many open questions.
Ren, Song and Zhang have established some spatial  central limit theorems in \cite{RenSongZhang2015Central} for a class of superprocesses with general spatial motions under the assumption that the branching mechanisms satisfy a second moment condition.
We hope to prove spatial CLT's for superprocesses with general motions without the second moment assumption on the branching mechanism in a future project.

Recall that our Assumption \ref{asp: branching mechanism} says that the branching mechanism $\psi$ is $-\alpha z +\eta z^{1+\beta}$ plus a small perturbation
%\begin{align}
%  \label{eq:PB}
%	\psi_1(z)
%	& := \rho z^2+ \int_{(0,\infty)}(e^{-yz}-1+yz) \Big(\pi(dy) - \frac{\eta~dy}{\Gamma(-1-\beta) y^{2+\beta}}\Big)
%\end{align}
$\psi_1(z)$
% where $\psi_1$ satisfies \eqref{eq: asp of branching mechanism} with some $\delta>0$.
which satisfies \eqref{eq: asp of branching mechanism} with some $\delta>0$.
It would be interesting to consider more general branching mechanisms.

Here are some examples of branching mechanisms satisfying Assumptions \ref{asp: Greys condition} and \ref{asp: branching mechanism}:
If $h$ is a complete Bernstein function which is regularly varying at 0 with index $\beta_1\in (\beta, 1)$, then
\[
  \psi(z)
  := -\alpha z + \rho z^2+\eta z^{1+\beta}+zh(z)
  , \qquad z>0
\]
satisfies Assumptions \ref{asp: Greys condition} and \ref{asp: branching mechanism}.
If $\beta_1\in (\beta, 1)$, $c_1\in (0, \eta/\Gamma(-1-\beta))$ and $c_2\ge 1$, then
\[
  \psi(z)
  :=-\alpha z + \rho z^2+\eta z^{1+\beta}-\int^\infty_{c_2} (e^{-yz}-1+yz)\frac{c_1dy}{y^{1+\beta_1}}
  , \qquad z\in \mathbb R_+
\]
satisfies Assumptions \ref{asp: Greys condition} and \ref{asp: branching mechanism}.

The rest of the paper is organized as follows:
In Subsection \ref{sec: branching mechanism} we will give some preliminary results for our branching mechanism $\psi$.
In Subsections \ref{sec: controller} and \ref{sec: h-controller} we will give some estimates for some operators related to the super-OU process $X$ that will be used in our proofs.
In Subsection \ref{sec: stable distributions} we will give the definitions of all the $(1+\beta)$-stable random variables involved in this paper.
In Subsection \ref{sec: Small value probability} we will give some estimates for the small value probability of continuous state branching processes.
In Subsection \ref{sec: Moments for super-OU processes} we will give upper bounds for the $(1+\gamma)$-moments for our superprocesses.
These estimates and upper bounds will be crucial in the proofs of our main results.
In Subsection \ref{sec: large rate lln}, we will give the proof of Theorem \ref{thm: law of large number}.
In Subsections \ref{sec:critical}--\ref{sec: large rate clt}, we will give the proof of Theorem \ref{thm:M}.
In the Appendix, we consider a general superprocess $(X_t)_{t\geq 0}$, and we prove there that the characteristic exponent of $\langle f,X_t\rangle$ satisfies a complex-valued non-linear integral equation.
This fact will be used at several places in this paper, and we think it is of independent interest.

% ** Preliminaries
\section{Preliminaries}
% *** Branching mechanisms
\subsection{Branching mechanism}
\label{sec: branching mechanism}
Let $\psi$ be the branching mechanism given in \eqref{eq: honogeneou branching mechanism}.
Suppose that Assumptions \ref{asp: Greys condition} and \ref{asp: branching mechanism} hold.
In this subsection, we give some preliminary results about the branching mechanism $\psi$.
Recall that $\eta$ and $\beta$ are the constants in Assumption \ref{asp: branching mechanism}.
Let $\mathbb C_+:= \{x+iy: x\in \mathbb R_+, y \in \mathbb R\}$ and $\mathbb C^0_+:= \{x+iy: x\in (0,\infty), y \in \mathbb R\}$.
\begin{lem}
  \label{lem:CEP}
	The function $\psi_1$ given by \eqref{eq:PB} can be uniquely extended as a complex-valued continuous function on $\mathbb C_+$ which is holomorphic on $\mathbb C^0_+$.
  Moreover, for each $\delta > 0$ small enough, there exists $C>0$ such that for all $z\in \mathbb C_+$, we have $|\psi_1(z)| \leq C |z|^{1+\beta+\delta} + C|z|^2.$
\end{lem}
\begin{proof}
  According to Lemma \ref{lem: extension lemma for branching mechanism} below and the uniqueness of holomorphic extensions, we know that $\psi_1$ can be uniquely extended as a complex-valued continuous function on $\mathbb C_+$ which is holomorphic on $\mathbb C^0_+$.
	The extended $\psi_1$ has the following form:
  \[
    \psi_1(z)
    = \rho z^2 + \int_{(0,\infty)}(e^{-yz}-1+yz) \Big(\pi(dy) - \frac {\eta~dy} {\Gamma(-1-\beta)y^{2+\beta}} \Big)
    , \quad z\in \mathbb C_+.
  \]
	% Now, according to  Assumption \ref{asp: branching mechanism} and Remark \ref{rem: small enough delta}, for each small enough $\delta > 0$, we have
	Now, according to  Assumption \ref{asp: branching mechanism}, for each small enough $\delta > 0$, we have
  \begin{align}
    |\psi_1(z)|
    & \leq \rho |z|^2 + \int_{(0,\infty)} (|yz|\wedge |yz|^2) \Big|\pi(dy) - \frac{\eta~dy}{\Gamma(-1-\beta)y^{2+\beta}}\Big| \\
    & \leq  |z|^2 \Big(\rho + \int_{(0,1)} y^2 \Big|\pi(dy) - \frac{\eta~dy}{\Gamma(-1-\beta)y^{2+\beta}}\Big|\Big) \\
    & \quad + |z|^{1+\beta +\delta}\int_{(1,\infty)} y^{1+\beta + \delta} \Big|\pi(dy) - \frac{\eta~dy}{\Gamma(-1-\beta)y^{2+\beta}}\Big|,
      \quad z \in \mathbb C_+,
  \end{align}
	as desired.
\end{proof}
The following lemma says that the constants $\eta, \beta$ in Assumption \ref{asp: branching mechanism} are uniquely determined by the L\'evy measure $\pi$.
\begin{lem}
  \label{lem: unique of beta and eta}
  Suppose Assumption  \ref{asp: branching mechanism} holds. Suppose that there are $\eta', \delta'>0$ and $\beta'\in (0,1)$ such that
  \[
    \int_{(1,\infty)} y^{ 1 + \beta'  + \delta' }~ \Big| \pi(dy) - \frac {\eta' ~dy} {\Gamma (- 1 - \beta ) y^{2 + \beta'}} \Big|
    < \infty.
  \]
	Then $\eta'= \eta$ and $\beta ' = \beta$.
\end{lem}
\begin{proof}
	Without loss of generality, we assume that $\beta+\delta \leq \beta'+ \delta'$.
	Using  the fact that $y^{1+\beta+ \delta} \leq y^{1+\beta'+\delta'}$ with $y \geq 1$, we get
  \[
    \int_{(1, \infty)} y^{1 + \beta + \delta}   \Big| \pi(dy) - \frac {\eta' ~dy} {\Gamma( - 1 - \beta)y^{2 + \beta'}} \Big|
    < \infty .
  \]
	Comparing this with Assumption \ref{asp: branching mechanism}, we get
  \[
    \int_{(1,\infty)} y^{ 1 + \beta + \delta} \Big| \frac { \eta ~dy} {\Gamma (- 1 - \beta) y^{2 + \beta}} - \frac {\eta' ~dy} {\Gamma (- 1 - \beta) y^{2 + \beta'}} \Big| < \infty.
  \]
	In other words, if we denote by $\widetilde \pi(dy)$ the measure $\eta' \Gamma(-1-\beta)^{-1} y^{-2-\beta'} dy$, then $\widetilde \pi$ is a L\'evy measure which satisfies Assumption \ref{asp: branching mechanism}.
	Applying Lemma \ref{lem:CEP} to this measure $\widetilde \pi$, we have that there exists $c>0$ such that
  \[
    | \eta z^{ 1 + \beta } - \eta' z^{ 1 + \beta' } |
    \leq c z^{ 1 + \beta + \delta } + c z^2
    , \quad z \in \mathbb R_+.
  \]
  Dividing both sides by $z^{1+\beta}$ we have
  \[
    | \eta - \eta' z^{ \beta' - \beta } |
    \leq cz^{\delta}+cz^{1-\beta}
    ,	\quad z \in \mathbb R_+.
  \]
	This implies that $ \eta' z^{\beta' - \beta} \xrightarrow[\mathbb R^+\ni z\to 0]{} \eta >0. $
	So we must have $\beta'= \beta$ and $\eta'= \eta$.
\end{proof}
\begin{lem}
  \label{lem: LlogL criterion}
  If $\psi$ satisfies Assumption  \ref{asp: branching mechanism}, then $\psi$ satisfies the $L \log L$ condition, i.e.,
  \[
    \int_{(1,\infty)} y \log y~\pi(dy)
    < \infty.
  \]
\end{lem}
\begin{proof}
	Using  Assumption \ref{asp: branching mechanism} and the fact that $y\log y \leq y^{1+\beta+\delta}$ for $y$ large enough, we get
  \[
    \int_{(1,\infty)} y \log y ~\Big| \pi(dy) - \frac { \eta ~dy } { \Gamma ( - 1 - \beta ) y^{ 2 + \beta } } \Big|
    < \infty.
  \]
	Therefore we have
  \[
    \int_{ ( 1, \infty ) } y \log y ~\Big( \pi(dy) - \frac { \eta ~dy } { \Gamma ( - 1 - \beta ) y^{ 2 + \beta } } \Big)
    < \infty.
  \]
  Combining this with
  \[
    \int_{ ( 1, \infty ) } \frac { \eta \log y ~dy } { \Gamma ( - 1 - \beta ) y^{ 1 + \beta } }
    < \infty,
  \]
  we immediately get the desired result.
\end{proof}

% *** Definition of controller 
\subsection{Definition of controller}
\label{sec: controller}
Denote by $\mathcal B(\mathbb R^d, \mathbb C)$ the space of all $\mathbb C$-valued Borel functions on $\mathbb R^d$.
Recall that $\mathcal P$ is given in \eqref{eq: polynomial growth function}.
Define $\mathcal P^+:= \mathcal P \cap \mathcal B(\mathbb R^d, \mathbb R_+)$ and $\mathcal P^*:= \{f\in \mathcal B(\mathbb R^d, \mathbb C): |f|\in \mathcal P\}$.

In this paper, we say $R$ is a \emph{monotone} operator on $\mathcal P^+$ if $R:\mathcal P^+ \to \mathcal P^+$ satisfies that $Rf\leq Rg$ for each $f\leq g$ in $\mathcal P^+$.
For a function $h: [0,\infty) \to [0,\infty)$, we say $R$ is an \emph{$h$-controller} if $R$ is a monotone operator on $\mathcal P^+$ and that $R(\theta f)\leq h(\theta) Rf$ for each $f\in \mathcal P^+$ and $\theta \in [0,\infty)$.
For subsets $\mathcal D, \mathcal I\subset \mathcal P^*$ and an operator $R$ on $\mathcal P^+$, we say an operator $A$ is \emph{controlled by $R$ from $\mathcal D$ to $\mathcal I$} if $A:\mathcal D \to \mathcal I$ and that $|Af| \leq R|f|$ for each $f\in \mathcal D$;
we say a family of operators $\mathscr O$ is \emph{uniformly controlled by $R$ from $\mathcal D$ to $\mathcal I$} if for each operator $A\in \mathscr O$, $A$ is controlled by $R$ from $\mathcal D$ to $\mathcal I$.
For subsets $\mathcal D, \mathcal I\subset \mathcal P^*$ and a function $h:[0,\infty) \to [0,\infty)$, we say an operator $A$ (resp. a family of operators $\mathscr O$) is \emph{$h$-controllable} (resp. \emph{uniformly $h$-controllable}) from $\mathcal D$ to $\mathcal I$ if there exists an $h$-controller $R$ such that $A$ (resp. $\mathscr O$) is controlled (resp. uniformly controlled) by $R$ from $\mathcal D$ to $\mathcal I$.

For two operators $A: \mathcal D_A \subset \mathcal P^*\to \mathcal P^*$ and $B: \mathcal D_B \subset \mathcal P^*\to \mathcal P^*$, define $(A \times B)f (x):= Af(x) \times Bf(x)$ for all $f\in \mathcal D_A \cap \mathcal D_B$ and $x\in \mathbb{R}^d$.
For any $a \in \mathbb R$ and any operator $A :\mathcal D_A \to \mathcal B(\mathbb R^d, \mathbb C\setminus (-\infty, 0])$, define $A^{\times a}f(x):= (Af(x))^a$ for all $f\in \mathcal D_A$ and $x\in \mathbb R^d$.

The following lemma is easy to verify.
\begin{lem}
  \label{lem: property of controllable operators}
  For each $i \in \{0,1\}$, let $\mathscr O_i$ be a family of operators which is  uniformly controlled by an $h_i$-controller $R_i$ from $\mathcal D_i \subset \mathcal P^*$ to $ \mathcal I_i \subset \mathcal P^*$.
  Then the followings hold:
  \begin{enumerate}
  \item
    If $\mathcal I_0 \subset \mathcal D_1$, then $\{A_1A_0: A_i \in \mathscr O_i, i = 0,1\}$ is uniformly controlled by the $(h_1 \circ h_0)$-controller $R_1R_0$ from $\mathcal D_0$ to $\mathcal I_1$.
  \item
    $\{ A_1 \times A_0: A_i \in \mathscr O_i, i = 0,1\}$ is uniformly controlled by the $(h_1\times h_0)$-controller $R_1 \times R_0$ from $\mathcal D_0 \cap \mathcal D_1$ to $\mathcal P^*$.
  \item
    $\{ A_1 + A_0: A_i \in \mathscr O_i, i = 0,1\}$ is uniformly controlled by the $(h_1 \vee h_0)$-controller $R_1 + R_0$ from $\mathcal D_0 \cap \mathcal D_1$ to $\mathcal P^*$.
  \item
    If $\mathcal I_0 \subset \mathcal B(\mathbb R^d, \mathbb C \setminus (\infty, 0])$ and $a>0$, then $\{A^{\times a} : A \in \mathscr O_0\}$ is uniformly controlled by the $(h_0^a)$-controller $R_0^{\times a}$ from $\mathcal D_0$ to $\mathcal P^*$.
  \item
    Suppose that $\mathscr O_0 = \{A_\theta: \theta \in \Theta \}$ where $\Theta$ is an index set.
    Further suppose that $(\Theta, \mathcal J )$ is a measurable space and that $(\theta,x) \mapsto A_\theta f(x)$ is $\mathcal J \otimes \mathcal B(\mathbb R^d)$-measurable for each $f\in \mathcal D$.
    Then the following space of operators
    \[
      \Big\{ f \mapsto \int_{\Theta} A_\theta f~\nu(d\theta) : \nu \text{ is a probability measure on } (\Theta, \mathcal J) \Big\}
    \]
    is uniformly controlled by $R_0$ from $\mathcal D_0$ to $\mathcal P^*$.
  \end{enumerate}
\end{lem}

% *** Controllers for the super-OU processes
\subsection{Controllers for the super-OU processes}
\label{sec: h-controller}
Let $X$ be the super-OU process introduced in Subsection \ref{subsec:M} with branching mechanism $\psi$ satisfying
Assumptions \ref{asp: Greys condition} and \ref{asp: branching mechanism}.
In this subsection, we will define several operators and study their properties that will be used in this paper.

Define $\psi_0(z) = \psi(z) + \alpha z$ for each $z\in \mathbb{R}_+$.
According to Lemma \ref{lem:CEP}, $\psi, \psi_1$ and $\psi_0$ can all be uniquely extended as complex-valued continuous functions on $\mathbb C_+$ which are also holomorphic on $\mathbb C^0_+$.
For each $f\in \mathcal B(\mathbb R^d, \mathbb C_+)$ and $x\in \mathbb R^d$, define $\Psi f (x) = \psi\circ f(x)$, $\Psi_0 f(x)= \psi_0 \circ f(x)$ and $\Psi_1 f(x)= \psi_1 \circ f(x)$.

For all $t\in [0,\infty), x\in \mathbb R^d $ and $f \in \mathcal{P}$, let $ U_tf(x) := \operatorname{Log} \mathbb P_{\delta_x}[e^{i\theta \langle f, X_t\rangle}]|_{\theta = 1} $ be the value of the characteristic exponent of the infinitely divisible random variable $\langle f, X_t\rangle$ at $1$.
(See the paragraph after Lemma \ref{lem: Lip of power function}.)
It follows from \eqref{eq: -v has positive real part} that $-U_tf(x)$ takes values in $\mathbb C_+$. Furthermore, we know from Proposition \ref{prop: complex FKPP-equation} that
\begin{align}
  \label{eq:chareq2}
  U_tf(x) - \int_0^t P^\alpha_{t-s} \Psi_0(-U_sf)(x)ds
  = i P^{\alpha}_t f(x)
  , \quad t\in [0,\infty), x\in \mathbb{R}^d, f\in \mathcal P.
\end{align}

For all $t\geq 0$ and $f\in \mathcal P$, we define
\begin{align}
  \label{eq: def of Zf}
  Z_t f
  := \int_0^t P^\alpha_{t-s}\big( \eta (-i P^\alpha_sf)^{1+\beta}\big)ds,
  & \qquad Z'_t f
    := \int_0^t P^\alpha_{t-s}\big( \eta (-U_s f)^{1+\beta}\big)ds,
  \\ Z''_t f
  := \int_0^t P^\alpha_{t-s}\Psi_1(-U_s f)ds,
  & \qquad\  Z'''_t f
    := (Z'_t - Z_t+ Z''_t)f.
\end{align}
Then we have that
\begin{align}
  \label{eq: key equality}
  U_t - i P^\alpha_t
  = Z'_t + Z''_t
  = Z_t + Z'''_t
  , \quad t\geq 0.
\end{align}
For all $\kappa \in \mathbb Z_+$ and $f\in \mathcal P$, define
\begin{align}
  \label{eq:Q}
  Q_\kappa f
  := \sup_{t\geq 0} e^{\kappa b t}|P_t f|,
  \qquad  Q f
  := Q_{\kappa_f}f.
\end{align}
Then according to \cite[Fact 1.2]{MarksMilos2018CLT}, $Q$ is an operator from $\mathcal P$ to $\mathcal P$.

\begin{lem}
  \label{lem: upper bound for usgx}
  Under Assumptions \ref{asp: Greys condition} and \ref{asp: branching mechanism}, the following statements are true:
  \begin{enumerate}
  \item
    $(-U_t)_{0\leq t\leq 1}$ is uniformly $\theta$-controllable from $\mathcal P$ to $\mathcal P^*\cap \mathcal B(\mathbb R^d, \mathbb C_+)$.
  \item
    $(P^\alpha_t)_{0\leq t\leq 1}$ is uniformly $\theta$-controllable on $\mathcal P^*$.
  \item
    $\Psi_0$ is $(\theta^2\vee \theta^{1+\beta})$-controllable from $\mathcal P^* \cap \mathcal B(\mathbb R^d, \mathbb C_+)$ to $\mathcal P^*$.
  \item
    $(U_t- iP_t^{\alpha})_{0\leq t\leq 1}$ is uniformly $(\theta^2\vee \theta^{1+\beta})$-controllable from $\mathcal P$ to $\mathcal P^*$.
  \item
    $(Z'_t-Z_t)_{0\leq t\leq 1}$ is uniformly $(\theta^{2+\beta}\vee \theta^{1+2\beta})$-controllable from $\mathcal P$ to $\mathcal P^*$.
  \item
    For any $\delta > 0$ small enough, we have that $(Z''_t)_{0\leq t\leq 1}$ is uniformly $(\theta^2\vee \theta^{1+\beta+\delta})$-controllable from $\mathcal P$ to $\mathcal P^*$.
  \item
    For any $\delta > 0$ small enough, we have that $(Z'''_t)_{0\leq t\leq 1}$ is uniformly $(\theta^{2+\beta}\vee \theta^{1+\beta+\delta})$-controllable from $\mathcal P$ to $\mathcal P^*$.
  \end{enumerate}
\end{lem}

\begin{proof}
  (1). According to \eqref{eq: -v has positive real part}, $U_t$ is an operator from $\mathcal P$ to $\mathcal B(\mathbb R^d, \mathbb C_+)$.
  It follows from \eqref{eq: upper bound for vf} that for all $g\in \mathcal P$, $0\leq t\leq 1$ and $x\in \mathbb R^d$, we have $ |U_t g(x)| \leq \sup_{0\leq u\leq 1}P_u^\alpha |g| (x). $
  We claim that $f\mapsto\sup_{0\leq u\leq 1}P^{\alpha}_u f$ is a map from $\mathcal P^+$ to $\mathcal P^+$. In fact, if $f\in \mathcal P^+$, there exists constant $c>0$ such that
  \[
    0
    \leq \sup_{0\leq u\leq 1}P^{\alpha}_u f
    \leq \sup_{0\leq u\leq 1} P_u (e^{\alpha u} e^{-\kappa_f u} e^{\kappa_f u} f )
    \leq c \sup_{0\leq u\leq 1} (e^{\kappa_fu}P_u f) \leq c Qf \in \mathcal P.
  \]
	It is clear that $f\mapsto\sup_{0\leq u\leq 1}P^{\alpha}_u f$ is a $\theta$-controller.

  (2). Similar to the proof of (1).

  (3). By Lemma \ref{lem:CEP}, there exist $C, \delta >0$ satisfying $\beta+\delta< 1$ such that for all $ f \in \mathcal P^* \cap \mathcal B( \mathbb R^d, \mathbb C_+ )$, it holds that $ |\Psi_0 f| \leq \eta |f|^{1+\beta} + |\Psi_1 f| \leq \eta |f|^{1+\beta} + C|f|^2+ C |f|^{1+\beta + \delta}.$
  Note that operator
  \[
    f \mapsto \eta f^{1+\beta} + Cf^2+ Cf^{1+\beta + \delta}
    , \quad f\in \mathcal P^+,
  \]
  is a $(\theta^2 \vee \theta^{1+\beta})$-controller.

  (4). From (1)--(3) above and Lemma \ref{lem: property of controllable operators}.(1), we know that the operators
  \[
    f \mapsto P^\alpha_{t-s}\Psi_0(-U_sf)
    , \quad 0\leq s\leq t\leq 1,
  \]
  are uniformly $(\theta^2\vee \theta^{1+\beta})$-controllable.
  Combining this with \eqref{eq:chareq2} and
  % YX Lemma \ref{lem: property of controllable operators}.(4), we get the desired result.
  % ZY Lemma \ref{lem: property of controllable operators}.(3), we get the desired result.
  Lemma \ref{lem: property of controllable operators}.(5), we get the desired result.

  (5). Notice that from Lemma \ref{lem: Lip of power function},
  \[
    |(-U_t f)^{1+\beta} - (-iP^\alpha_t f)^{1+\beta} |
    \leq  (1+\beta) |U_t f-iP^\alpha_t f|(|U_t f|^{\beta}+|i P^\alpha_t f|^{\beta}).
  \]
  Now using (1), (2) and (4) above, and Lemma \ref{lem: property of controllable operators}.(2)--(3), we get that the operators
  \[
    f \mapsto (-U_t f)^{1+\beta} - (-iP^\alpha_t f)^{1+\beta},\quad 0\leq t\leq 1,
  \]
  are uniformly $(\theta^{2+\beta}\vee \theta^{1+2\beta})$-controllable.
  Combining with Lemma \ref{lem: property of controllable operators}.(1) and (4), and
  \[
    (Z'_t - Z_t)f
    = \int_0^t P^\alpha_{t-s}\Big( \eta ((-U_s f)^{1+\beta} - (-iT_s^\alpha f)^{1+\beta} )\Big)ds
    , \quad 0\leq t\leq 1, f\in \mathcal P,
  \]
  we get the desired result.

  (6). By  Lemma \ref{lem:CEP}, for any $\delta > 0$ small enough, there exists  $C>0$ such that
  \[
    |\Psi_1(f)|
    \le C(|f|^2+|f|^{1+\beta+ \delta})
    , \quad f\in \mathcal P^*\cap\mathcal B(\mathbb R^d, \mathbb C_+).
  \]
  Note that, for all $\delta, C>0$,
  \[
    f \mapsto C(f^2+f^{1+\beta+\delta})
    , \quad f\in \mathcal P^+
  \]
  is a $(\theta^2 \vee \theta^{1+\beta+\delta})$-controller.
  Therefore, for any $\delta > 0$ small enough, we have that $\Psi_1$ is a $(\theta^2 \vee \theta^{1+\beta+\delta})$-controllable operator from $\mathcal P^*\cap\mathcal B(\mathbb R^d, \mathbb C_+)$ to $\mathcal P^*$.
  Combining  this with  (1)--(2) above,
  and Lemma \ref{lem: property of controllable operators}.(1) and (4), we get that, for any $\delta > 0$ small enough, the operators
  \[
    f
    \mapsto Z_t'' f
    = \int_0^t P_{t-s}^\alpha \Psi_1(-U_sf)ds
    , \quad 0\leq t\leq 1,
  \]
  are uniformly $(\theta^2 \vee \theta^{1+\beta+\delta})$-controllable from $\mathcal P$ to $\mathcal P^*$.

  (7). Since $Z'''_t = (Z'_t-Z_t)+Z''_t$, the desired result follows from (5)-- (6) above and Lemma \ref{lem: property of controllable operators}.(3).
\end{proof}

% *** Stable distributions
\subsection{Stable distributions}
\label{sec: stable distributions}
Recall that the operators $(T_t)_{t\geq 0}$ are defined by \eqref{eq:I:R:1}, and functionals $(m_{t})_{0\leq t< \infty}$ and $m$ are given by \eqref{eq:I:R:2} and \eqref{eq:I:R:3} respectively.
\begin{lem}
  \label{lem:m}
  Operators $(T_t)_{t\geq 0}$, functionals $(m_{t})_{0\leq t< \infty}$ and functional $m$ are well defined.
\end{lem}
\begin{proof}
  \emph{Step 1.} We will show that for each $f \in \mathcal P$, there exists $h \in \mathcal P$ such that $ |T_tf| \leq  e^{- \delta t} h$ for each $t\geq 0$, where
  \begin{align}
    \label{eq:m:1}
    \delta
    := \inf \big\{ |\tilde \beta \alpha - |p|b| : p \in \mathbb Z_+^d, \langle f, \phi_p\rangle_\varphi \neq 0 \big\}
    \geq 0.
  \end{align}
  From this upper bound, it can be verified that $(T_t)_{t\geq 0}$ and $(m_{t})_{0 \leq t < \infty}$ are well defined.
  In fact, we can write $f = f_0 + f_1$ with $f_0\in \mathcal C_s \oplus \mathcal C_c$ and $f_1 \in \mathcal C_l$.
  According to \cite[Lemma 2.7]{MarksMilos2018CLT}, there exists $h_0 \in \mathcal P$ such that for each $t\geq 0$,
  \[
    |T_t f_0|
    = \Big| \sum_{p \in  \mathbb Z_+^d: \tilde \beta \alpha \leq |p|b } e^{- ( |p| b - \tilde \beta \alpha ) t} \langle f, \phi_p \rangle \phi_p \Big|
    = e^{\tilde \beta \alpha t} | P_t f_0 |
    \leq e^{- ( \kappa_{(f_0)} b - \tilde \beta \alpha) t} h_0
    \leq e^{- \delta t} h_0.
  \]
  On the other hand
  \[
    |T_t f_1|
    \leq e^{- \delta t}\sum_{p \in \mathbb Z_+^d : \tilde \beta \alpha > |p|b}
    |\langle f, \phi_p \rangle \phi_p|
    =: e^{- \delta t} h_1,
    \quad t\geq 0.
  \]
  So the desired result in this step follows with $h := h_0 + h_1$.

  \emph{Step 2.} We will show that if $f \in \mathcal C_s \oplus \mathcal C_l$, then $m[f]$ is well defined.
  In fact, let $\delta$ be given by \eqref{eq:m:1}, then in this case $\delta > 0$.
  Now, according to Step 1 there exists $h \in \mathcal P$ such that $|T_tf| \leq e^{- \delta t} h$ for each $t\geq 0$.
  This exponential decay implies the desired result in this step.

  \emph{Step 3.} We will show that if $f\in \mathcal P \setminus (\mathcal C_s \oplus \mathcal C_l)$, then $m[f]$ is also well defined.
  In fact, $f$ can be decomposed as $f = f_c + f_{sl}$ where $f \in \mathcal C_c\setminus \{0\}$ and $f_{sl}\in \mathcal C_s \oplus \mathcal C_l$.
  Note that $T_t f_c = f_c$ for each $t\geq 0$.
  Also note that in Step 2, we already have shown that there exist $\delta > 0$ and $h \in \mathcal P^+$ such that for each $t\geq 0$, we have $|T_t f_{sl}| \leq e^{- \delta t}h$.
  Therefore, using Lemma \ref{lem: Lip of power function} we have
  \begin{align}
    &|(-iT_t f)^{1+\beta} - (-i f_c)^{1+\beta}|
      \leq (1+\beta) ( |T_tf|^\beta + |f_c|^\beta) |T_tf_{sl}|
    \\&\leq (1+\beta) ( |f_c + T_t f_{sl}|^\beta + |f_c|^\beta) e^{- \delta t} h
    \\&\leq (1+\beta) ( (|f_c| + |h|)^\beta + |f_c|^\beta) e^{- \delta t} h
    =: e^{- \delta t} g
  \end{align}
  where $g\in \mathcal P^+$.
  Therefore
  \begin{align}
    \label{eq:P:S:1}
    &\Big| \frac{1}{t} m_t[f] - \langle (-if_c)^{1+\beta}, \varphi\rangle\Big|
      = \Big| \frac{1}{t} \cdot t \int_0^1  \big\langle (-iT_{rt}f)^{1+\beta} - (-if_c)^{1+\beta}, \varphi\big \rangle ~dr \Big|\\
    &\leq \int_0^1  \big\langle |(-iT_{rt}f)^{1+\beta} - (-if_c)^{1+\beta}|, \varphi\big \rangle ~dr = \langle g,\varphi\rangle\int_0^1 e^{-\delta rt} ~dr
      \xrightarrow[t\to \infty]{} 0.
      \qedhere
  \end{align}
\end{proof}

\begin{prop}
  \label{prop:PL:S}
  For each $f \in \mathcal P\setminus \{0\}$, there exists a non-degenerate $(1+\beta)$-stable random variable $\zeta^f$ such that $ E[e^{i\theta\zeta^f}] = e^{m[\theta f]}$ for each $\theta \in \mathbb R$.
\end{prop}

The proof of the above proposition relies on the following lemma:
\begin{lem}
  \label{lem: charactreisticfunction}
  Let $q$ be a measure on $\mathbb R^d\setminus\{0\}$ with
  $\int_{\mathbb R^d\setminus\{0\}} |x|^{1+\beta} q(dx) \in (0,\infty)$.
  Then
  \[
    \theta
    \mapsto  \exp\Big\{\int_{\mathbb R^d\setminus\{0\}} (i\theta \cdot x)^{1+\beta} q(dx)\Big\},
    \quad \theta \in \mathbb R^d
  \]
  is the characteristic function of an $\mathbb R^d$-valued $(1+\beta)$-stable random variable.
\end{lem}
\begin{proof}
  It follows from disintegration that there exist a measure $\lambda$ on $S:= \{\xi\in \mathbb R^d:|\xi| = 1\}$ and a kernel $k(\xi,dt)$ from $S$ to $\mathbb R_+$ such that
  \[
    \int_{\mathbb R^d\setminus \{0\}} f(x)q(dx)
    = \int_S \lambda(d\xi) \int_{\mathbb R_+} f(\xi t)k(\xi,dt)
    , \quad f\in \mathcal B(\mathbb R^d\setminus \{0\}, \mathbb R_+).
  \]
  We define another measure $\lambda_0$ on $S$ by
  \[
    \lambda_0(d\xi)
    := \frac1{\Gamma(-1-\beta)}\int_0^\infty t^{1+\beta}k(\xi,dt) \lambda (d\xi),
  \]
  where $\Gamma$ is the Gamma function.
  Then $\lambda_0$ is a non-zero finite measure, since
  \begin{align}
    &\lambda_0(S)
      = \frac{1}{\Gamma(-1-\beta)} \int_S \lambda (d\xi) \int_0^\infty |t\xi|^{1+\beta}k(\xi,dt) \\
    & = \frac{1}{\Gamma(-1-\beta)} \int_{\mathbb R^d\setminus\{0\}} |x|^{1+\beta} q(dx) \in (0,\infty).
  \end{align}
  Define a measure $\nu$ on $\mathbb R^d\setminus\{0\}$ by
  \[
    \int_{\mathbb R^d\setminus\{0\}}f(x)\nu(dx)
    = \int_{S} \lambda_0(d\xi) \int_0^\infty f(r\xi) \frac{dr}{r^{2+\beta}}
    ,\quad f\in \mathcal B(\mathbb R^d\setminus \{0\}, \mathbb R_+).
  \]
  Then, according to \cite[Remark 14.4]{Sato2013Levy}, $\nu$ is the L\'evy measure of a $(1+\beta)$-stable distribution on $\mathbb R^d$, say $\mu$, whose characteristic function is
  \[
    \hat \mu(\theta)
    = \exp \Big \{ \int_{\mathbb R^d\setminus\{0\}} (e^{-i\theta \cdot y}-1+i\theta \cdot y) \nu(dy) \Big \}
    , \quad \theta \in \mathbb R.
  \]
  Finally, according to \eqref{eq: stable branching on C+}, we have
  \begin{align}
    & \int_{\mathbb R^d\setminus\{0\}} (e^{-i\theta \cdot y}-1+i\theta \cdot y) \nu(dy)
      = \int_S \lambda_0(d\xi) \int_0^\infty (e^{-ir\theta \cdot \xi}-1+ir\theta \cdot \xi) \frac{dr}{r^{2+\beta}} \\
    & = \int_S \lambda (d\xi) \int_0^\infty (e^{-ir\theta \cdot \xi}-1+ir\theta \cdot \xi) \frac{dr}{\Gamma(-1-\beta)r^{2+\beta}}\int_0^\infty t^{1+\beta} k(\xi,dt) \\
    & = \int_S \lambda (d\xi) \int_0^\infty (i\theta\cdot \xi)^{1+\beta} t^{1+\beta} k(\xi,dt)
      = \int_S \lambda(d\xi) \int_0^\infty (i\theta \cdot t\xi)^{1+\beta} k(\xi,dt) \\
    & = \int_{\mathbb R^d} (i\theta \cdot x)^{1+\beta} q(dx).
      \qedhere
  \end{align}
\end{proof}

\begin{proof}[Proof of Proposition \ref{prop:PL:S}]
	Suppose that $f\in \mathcal C_s \oplus \mathcal C_l$.
Note that $m[\theta f]$ can be written as
  \begin{align}
    \label{eq:PL:S:1}
    m[\theta f]
    = \eta \int_0^{\infty}~ds\int_{\mathbb R^d} \big(-i\theta T_s f(x)\big)^{1+\beta} \varphi(x)~dx,
    \quad \theta \in \mathbb R.
  \end{align}
	Therefore, according to Lemma \ref{lem: charactreisticfunction}, in order to show that $\zeta^f$ is an $(1+\beta)$-stable random variable we only need to show that
  \[
    \int_0^{\infty}~ds\int_{\mathbb R^d} | T_{s} f(x)|^{1+\beta} \varphi(x)~dx
    \in (0, \infty).
  \]
  According to the Step 1 of the proof of Lemma \ref{lem:m}, we know that there exist $\delta> 0$ and $h \in \mathcal P$ such that $|T_sf| \leq e^{- \delta s} h$ for each $s\geq 0$.
  The above then follows.

  If $f \in \mathcal P \setminus (\mathcal C_s \oplus \mathcal C_l)$, then $f$ can be written by $f = f_c +(f - f_c)$ where $f_c \in \mathcal C_c\setminus\{0\}$ and $f - f_c \in \mathcal C_s \oplus \mathcal C_l$.
  In this case, according to \eqref{eq:P:S:1}, $m[\theta f]$ has an integral representation:
  \begin{align}
    \label{eq:PL:S:2}
    m[\theta f]
    = \int_{\mathbb R^d} (-i\theta f_c(x))^{1+\beta} \varphi(x) ~dx,
    \quad \theta \in \mathbb R.
  \end{align}
  Finally, according to Lemma \ref{lem: charactreisticfunction} and the fact that
  \begin{align}
    \int_{\mathbb R^d} | f_c(x)|^{1+\beta} \varphi(x)~dx
    \in (0, \infty),
  \end{align}
  We have that $\zeta^f$ is a non-degenerate $(1+\beta)$-stable random variable.
\end{proof}


\subsection{A refined estimation for the OU semigroup}
% YX Do we need this subsection?
It turns out that our proof of the CLT for the super OU processes relies on the following refined estimation for the OU semigroup.

\begin{lem}
  \label{lem:P:R}
  Suppose that $g \in \mathcal P$, then there exists $h \in \mathcal P^+$ such that for each $ f \in \mathcal P_g := \{\theta T_n g: n \in \mathbb Z_+, \theta \in [-1,1]\} $ and $t\geq 0$, we have $ | P_t (Z_1 f - \langle Z_1 f, \varphi \rangle )| \leq e^{-bt} h$.
\end{lem}
\begin{proof}
  In the following $g \in \mathcal P$ is fixed. We write  $g = g_0 + g_1$ with $g_0 \in \mathcal C_s \oplus \mathcal C_c$ and $g_1 \in \mathcal C_l$,  and $q_f:=Z_1f - \langle Z_1f, \varphi \rangle\in \mathcal P^*$ for each $f\in \mathcal P$. 
  % ZY We need  to prove that for each $f\in \mathcal P_g$, there exists  $h \in \mathcal P^+$ such that for each $|P_tq_f| \leq e^{-bt} h.$
  We need to prove that there exists $h \in \mathcal P^+$ such that for each $f\in \mathcal P_g$, $|P_tq_f| \leq e^{-bt} h$.

  \emph{Step 1.} We claim  that we only need to prove the result for all
  $f \in \widetilde{\mathcal P}_g:= \{T_{n+1} g : n \in \mathbb Z_+\}$.
  In fact, both $\operatorname{Re} q_g$ and $\operatorname{Im} q_g$ are functions in $\mathcal P$ of order $\geq 1$.
  The result is valid for $f = T_0 g = g$ according to \cite[Fact 1.2]{MarksMilos2018CLT}.
  Also, note that if the result is valid for some $f \in \mathcal P$, it is also valid for any $\theta f$ with $\theta \in [-1,1]$.


  \emph{Step 2.} We show that $\{T_s g: s> 0\} \subset C_\infty (\mathbb R^d) \cap \mathcal P$.
  In fact, for each $s > 0$,
  \[
    T_s g
    = T_s (g_0 + g_1)
    = e^{\alpha \tilde \beta s}P_s g_0 + \sum_{p \in \mathbb Z_+^d: \alpha \tilde \beta > |p|b} \langle g_1, \phi_p \rangle e^{-(\alpha \tilde \beta - |p|b)s} \phi_p.
  \]
  Notice that the second term is in $C_\infty(\mathbb R^d)\cap \mathcal P$ since it is a finite summation of polynomials, and the first term is also in $C_\infty (\mathbb R^d) \cap \mathcal P$ according to \cite[Fact 1.1]{MarksMilos2018CLT}.

  \emph{Step 3.} We show that there exists $h_3 \in \mathcal P^+$ such that for each $j \in \{1,\dots, d\}$ and $f \in \widetilde {\mathcal P}_g$, it holds that $|\partial_j f| \leq h_3$.
  In fact, it is known that for $t\geq 0$ and $x\in \mathbb R^d$, the position of the OU process $\xi_t$ under $\mathbb P_x$ is a Gaussian random variable with mean $xe^{-bt}$ and variance $\frac{\sigma^2}{2b} (1- e^{-2bt})$. From this it can be verified that (see \cite{MetafunePallaraPriola2002Spectrum} for example)
  \begin{align}
    \label{eq:P:R:3:-1}
    P_t f(x) 
    = \int_{\mathbb R^d} f\big(x e^{-bt} + y \sqrt{1-e^{-2bt}}\big) \varphi(y)~dy,
    \quad t\geq 0, x\in \mathbb R^d, f\in \mathcal P.
  \end{align}
  For $f \in C_\infty(\mathbb R^d)\cap \mathcal P$ it can be verified from above that
  \begin{align}
    \label{eq:P:R:3:1}
    \partial_j P_t f 
    = e^{-bt} P_t \partial_j f,
    \quad t \geq 0, j \in \{1,\dots, d\}.
  \end{align}
  Thanks to Step 2, $T_1 g_0 \in C_\infty(\mathbb R^d)\cap \mathcal P$.
  According to \cite[Fact 1.3]{MarksMilos2018CLT} and the fact that $\alpha \tilde \beta \leq \kappa _{g_0} b$, we have for each $j \in \{1,\dots, d\}$,
  \[
    \kappa_{(\frac{\partial T_1 g_0}{\partial x_j})}
    \geq \kappa_{(T_1 g_0)} - 1
    = \kappa_{g_0} - 1
    \geq \frac{\alpha \tilde \beta}{b} - 1.
  \]
  Therefore, there exists  $h'_3\in \mathcal P^+$n such that for each $n \in \mathbb Z_+$ and $j\in \{1,\dots,d\}$,
  \begin{align}
    & | \partial_j T_{n+1}g_0 |
      = | \partial_j e^{\alpha \tilde \beta n}P_n T_1g_0 |
      = e^{\alpha \tilde \beta n-bn} |P_n \partial_j T_1 g_0| \\
    & \leq e^{\alpha \tilde \beta n-bn} \exp\{-\kappa_{(\partial_j T_1 g_0)}bn\}Q \partial_j T_1g_0
      \leq Q\partial_j T_1g_0
      \leq h'_3.
  \end{align}
  On the other hand, there exists $h_3''\in \mathcal P^+$ such that for each $n \in \mathbb Z_+$ and $j\in \{1,\dots,d\}$,
  \begin{align}
    & |\partial_j T_{n+1}g_1 |
      = \Big| \sum_{p\in \mathbb Z_+^d: \alpha \tilde \beta > |p|b} e^{- (\alpha \tilde \beta - |p|b)(n+1)} \langle g_1, \phi_p \rangle_\varphi \partial_j \phi_p \Big| \\
    & \leq \sum_{p\in \mathbb Z_+^d: \alpha \tilde \beta > |p|b} |\langle g_1, \phi_p \rangle_\varphi \partial_j \phi_p |
      \leq h_3''.
  \end{align}
  Then the desired result in this step follows.

  \emph{Step 4.} We show that there exists $h_{4} \in \mathcal P^+$ such that for each $j \in \{1,\dots, d\}, u \in [0, 1]$ and $f \in \widetilde {\mathcal P}_g$, it holds that $ | \partial_j  P_{1-u}^\alpha (- i P_u^\alpha f)^{1+\beta} | \leq h_4$.
  In fact, thanks to Step 2 and \eqref{eq:P:R:3:1}, for each $j \in \{1,\dots, d\}, u \in [0, 1]$ and $f \in \widetilde{\mathcal P}_g$, we have
  \begin{align}
    & \partial_j  P_{1-u}^\alpha (- i P_u^\alpha f)^{1+\beta}
      = e^{-(1-u)b} P_{1-u}^\alpha \partial_j (- i P_u^\alpha f)^{1+\beta}
    \\ & = (1+\beta) e^{-(1-u)b} P_{1-u}^\alpha [ (- i P_u^\alpha f)^\beta \partial_j (- i P_u^\alpha f) ]
    \\ & = -i(1+\beta) e^{-(1-u)b} P_{1-u}^\alpha[ (- i P_u^\alpha f)^\beta e^{-ub} P_u^\alpha \partial_j f]
    \\ & = -i(1+\beta) e^{-b} e^{(1-u)\alpha} e^{u\alpha (1+\beta)} P_{1-u} [ (- i P_u f)^\beta P_u \partial_j f ].
  \end{align}
  Recall from Step 1 of the proof of Lemma \ref{lem:m} there exists $h'_4\in \mathcal P^+$ such that for each $f \in \{T_sg:s\geq 0\}$ it holds that $|f| \leq h'_4$.
  Therefore, using Step 3, we have for each $j \in \{1,\dots, d\}, u \in [0, 1]$ and $f \in \widetilde {\mathcal P}_g$,
  \begin{align}
    & |\partial_j  P_{1-u}^\alpha (- i P_u^\alpha f)^{1+\beta}|
    \leq (1+\beta) e^{\alpha (1+\beta)} P_{1-u} [  (P_u |f|)^\beta P_u |\partial_j f| ]
    \\ & \leq (1+\beta) e^{\alpha (1+\beta)} P_{1-u} [  (P_u h'_4)^\beta P_u h_3 ]
    \leq (1+\beta) e^{\alpha (1+\beta)} Q_0 [  (Q_0 h_4')^\beta Q_0 h_3 ],
  \end{align}
  where $Q_0$ is defined by \eqref{eq:Q}.
  This implies the desired result in this step.
  
  \emph{Step 5.} We show that there exits $h_5 \in \mathcal P^+$ such that for each $f \in \widetilde {\mathcal P}_g$, we have $ |\nabla (Z_1f)| \leq h_5$.
  In fact, according to Step 4, for each $j \in \{1,\dots, d\}$, $f \in \widetilde{\mathcal P}_g$ and compact $A \subset \mathbb R^d$, we have
  \[
    \int_0^1 \sup_{x\in A} | (\partial_j  P_{1-u}^\alpha (-i P_u^\alpha f)^{1+\beta}) (x) |~du
    \leq \sup_{x\in A} h_4(x) < \infty.
  \]
  Using this and \cite[Theorem A.5.2]{Durrett2010Probability}, for each $j \in \{1,\dots, d\}$, $f\in \widetilde {\mathcal P}_g$ and $x\in \mathbb R^d$, it holds that
  \[
    | \partial_j Z_1 f(x)|
    = \Big| \int_0^1  ( \partial_jP_{1-u}^\alpha (-iP_u^\alpha f)^{1+\beta} ) (x) ~du  \Big|
    \leq h_4(x).
  \]
  Now, the desired result for this step is valid.

  \emph{Final step.}
  Let $h_5$ be the function in Step 5. 
  There are $c_0, n_0> 0$ such that for each $x\in \mathbb R^d$, $h_5(x) \leq c_0(1+|x|)^{n_0}$.
  Note that for each $x, y \in \mathbb R^d$, $1+|x|+|y|\leq (1+|x|) (1+|y|)$; and that for each $\theta \in [0,1]$, $|\sqrt {1 - \theta} - 1| \leq \theta$.
  Write $D_{x,y} = \{ax+by: a, b \in [0,1]\}$ for each $x, y \in \mathbb R^d$.
  Using \eqref{eq:P:R:3:-1} and Step 5, there exists  $h_6 \in \mathcal P^+$ such that for each $t \geq 0$, $f \in \widetilde {\mathcal P}_g$ and $x \in \mathbb R^d$, we have
  \begin{align}
    & |P_t q_f(x)|
      = \Big| \int_{\mathbb R^d} ( (Z_1f)(x e^{-bt}+ y \sqrt{1 - e^{-2bt}}) - Z_1 f (y) ) \varphi(y) ~dy \Big| \\
    & \leq \int_{\mathbb R^d} \Big(\sup_{z\in D_{x,y}} |\nabla  (Z_1f) (z) |\Big) | x e^{-bt} + y \sqrt{1 - e^{-2bt}} - y | \varphi(y) ~dy \\
    & \leq e^{-bt} \int_{\mathbb R^d} c_0(1+|x|+|y|)^{n_0} (|x|+|y|) \varphi(y) ~dy \\
    & \leq c_0 e^{-bt}(1+|x|)^{n_0}\Big(|x|\int_{\mathbb R^d} (1+|y|)^{n_0}\varphi(y) ~dy + \int_{\mathbb R^d} (1+ |y|)^{n_0} |y| \varphi(y) ~dy \Big) \\
    & \leq e^{-bt} h_6(x).
      \qedhere
  \end{align}
\end{proof}

% *** Small value probability
\subsection{Small value probability}
\label{sec: Small value probability}
In this subsection, we digress briefly from our super-OU process and consider a (supercritical) \emph{continuous-state branching process (CSBP)} $\{(Y_t)_{t\geq 0}; \mathbf P_x\}$ with branching mechanism $\psi$ given by \eqref{eq: honogeneou branching mechanism}.
Such a process $\{(Y_t)_{t\geq 0}; \mathbf P_x\}$ is defined as an $\mathbb R^+$-valued Hunt process satisfying
\[
  \mathbf P_x[e^{-\lambda Y_t}] = e^{- x v_t(\lambda)},
  \quad x\in \mathbb R^+, t\geq 0, \lambda \in \mathbb R^+,
\]
where for each $\lambda\geq 0$, $t\mapsto v_t(\lambda)$ is the unique positive solution to the equation
\begin{align}
  \label{eq: fkpp equation for CSBP}
  v_t(\lambda) - \int_0^t \psi(v_s(\lambda))~ds = \lambda,
  \quad t\geq 0.
\end{align}
It can be verified that for each $\mu \in \mathcal M(\mathbb R^d)$ with $x = \| \mu \|$, we have $ \{(\|X_t\|)_{t\geq 0}; \mathbb P_\mu\} \overset{\text{law}}{=} \{(Y_t)_{t\geq 0}; \mathbf P_x\}$.

Our goal in this subsection is to determine how fast the probability $\mathbf P_x(0<e^{-\alpha t}Y_t \leq k_t)$ converges to $0$ when $t\mapsto k_t$ is a strictly positive function on $[0,\infty)$ such that $k_t \to 0$ and $k_t e^{\alpha t} \to \infty$ as $t\to \infty$.
Suppose that Grey's condition is satisfied, i.e., there is some constant $z' > 0$ such that $\psi(z) > 0$ for all $z>z'$, and that $\int_{z'}^\infty \psi(z)^{-1}dz < \infty$.
Also suppose that the $L \log L$ condition is satisfied, i.e.,
\[
  \int_1^\infty y \log y~\pi(dr)
  < \infty.
\]
We write $W_t = e^{-\alpha t}Y_t$ for each $t\geq 0$.

\begin{prop}
  \label{lem: control of XT}
  Suppose that $t\mapsto k_t$ is a strictly positive function on $[0,\infty)$ such that $k_t \to 0$ and $k_t e^{\alpha t} \to \infty$ as $t\to \infty$.
  Then, for each $x\geq 0$, there exist $C,\delta>0$ such that
  \[
    \mathbf P_x(0<W_t\leq k_t)
    \leq C(k_t^\delta + e^{-\delta t}), \quad t\geq 0.
  \]
\end{prop}

\begin{proof}
  \emph{Step 1.} 
  We recall some known facts about the CSBP $(Y_t)$.
  For each $\lambda \geq 0$, we denote by $t\mapsto v_t(\lambda)$ the unique positive solution of \eqref{eq: fkpp equation for CSBP}.
  Letting $\lambda \to \infty$ in \eqref{eq: fkpp equation for CSBP}, we have by monotonicity that $\bar v_t:= \lim_{\lambda \to \infty}v_t(\lambda)$ exists in $(0,\infty]$ for all $t\geq 0$, and that
  \begin{align}
    \label{eq: svp1}
    \mathbf P_x(Y_t = 0)=e^{-x\bar v_t}, \quad t\geq 0, x\ge 0.
  \end{align}
  % It is known, see \cite[Theorems 3.5--3.8]{Li2011Measure-valued} for example, that under Grey's condition, we have (i) $0\leq \bar v_t < \infty$ for all $t>0$; (ii) $t\mapsto \bar v_t$ is decreasing; and (iii) $\bar v:= \lim_{t\to \infty} \bar v_t \in [0,\infty)$ is the largest root of $\psi(z) = 0$.
  % It is known, see \cite[Theorems 3.5--3.8]{Li2011Measure-valued} for example, that under Grey's condition, we have (i) $0\leq \bar v_t < \infty$ for all $t>0$; (ii) $t\mapsto \bar v_t$ is decreasing; and (iii) $\bar v:= \lim_{t\to \infty} \bar v_t \in [0,\infty)$ is the largest root of $\psi$ on $[0,\infty)$.
  It is known, see \cite[Theorems 3.5--3.8]{Li2011Measure-valued} for example, that under Grey's condition $\bar v:= \lim_{t\to \infty} \bar v_t \in [0,\infty)$ exists and is the largest root of $\psi$ on $[0,\infty)$.
  Letting $t \to \infty$ in \eqref{eq: svp1}, we have by monotonicity that
  \[
    \mathbf P_x(\exists t \geq 0, Y_t = 0)
    = e^{-x\bar v}, \quad x\geq 0.
  \]
  Note the derivative of $\psi$
  \[
    \psi'(z)
    = -\alpha + 2\rho z + \int_{(0,\infty)}(1-e^{-zy})y\pi(dy),\quad z\geq 0,
  \]
  is non-decreasing.
  This says that $\psi$ is a convex function.
  Also notice that $\psi'(0+)=-\alpha <0$ and that there exists $z>0$ such that $\psi(z)>0$.
  Therefore we have (i) $\bar v > 0$; (ii) $\psi(z) < 0$ on $z\in (0,\bar v)$; and (iii) $\psi(z) > 0 $ on $z\in (\bar v, \infty)$.
  It is also known, see \cite[Proposition 3.3]{Li2011Measure-valued} for example, that (i) if $\lambda \in (0,\bar v)$, then $0<\lambda \leq v_t(\lambda)<\bar v $; (ii) if $\lambda \in (\bar v, \infty)$, then $\bar v < v_t(\lambda)\leq \lambda< \infty$; and (iii) for each $\lambda \in (0,\infty)\setminus \{\bar v\}$ and $t\geq 0$, we always have
    \(
      \int_{v_t(\lambda)}^\lambda  \psi(z)^{-1}dz = t.
    \)
  % By monotonicity, we have that
  Taking $\lambda \to \infty$ and using monotone convergence theorem, we have that
  \begin{align}
    \label{eq:svp2}
    \int_{\bar v_t}^\infty \frac{dz}{\psi(z)} = t, \quad t\geq 0.
  \end{align}

  \emph{Step 2.} We will show that, for each $x \geq 0$ there exists a constant $c_1>0$ such that
  \[
    \mathbf P_{x}(0< W_t\leq k_t)
    \leq c_1\big(|\bar v- v_t(k_t^{-1}e^{-\alpha t})|+|\bar v_t - \bar v|\big),
    \quad t\geq 0.
  \]
  In fact, for all $x\geq 0$ and $t\geq 0$, we have
  \begin{align}
    & \mathbf P_{x}(0<W_t \leq k_t)
      = \mathbf P_{x}( e^{-k_t^{-1}W_t}\geq e^{-1},W_t > 0) \\
    & \leq e \mathbf P_{x}[e^{-k_t^{-1} W_t};W_t > 0]
      =  e\big(\mathbf P_x[e^{-k_t^{-1} W_t}]-\mathbf P_x(W_t = 0)\big) \\
    & = e\big(e^{-xv_t(k_t^{-1} e^{-\alpha t})}-e^{-x\bar v_t}\big)
      \leq ex \big(|\bar v-v_t(k_t^{-1} e^{-\alpha t})|+ |\bar v_t- \bar v|\big),
  \end{align}
  as desired in this step.

  \emph{Step 3.} We will show that there exist $c_2, \delta_1, t_0 > 0$ such that
  \[
    |\bar v_t-\bar v|
    \leq c_2e^{-\delta_1 t}
    , \quad t\geq t_0.
  \]
  In fact, since $\psi$ is a convex function, we must have $\tau:=\psi'(\bar v)>0$ and that  $\psi(z) \geq (z-\bar v)\tau$ for each $z\geq \bar v$.
  According to Grey's condition, we can find a constant $z_0 >\bar v $ such that $t_0 := \int^\infty_{z_0}\psi(z)^{-1}dz<\infty$.
  For each $t > t_0$, according to \eqref{eq:svp2}, we have
  \begin{align}
    & t - t_0 =
      \int^\infty_{\bar v_t} \frac{dz}{\psi(z)} - \int_{z_0}^\infty \frac{dz}{\psi(z)}
      = \int_{\bar v_t}^{z_0} \frac{dz}{\psi(z)} \\
    & \leq \int_{\bar v_t}^{z_0} \frac{dz}{(z-\bar v)\tau}
      = \frac{1}{\tau} \big(\log (z_0-\bar v) - \log(\bar v_t-\bar v)\big).
  \end{align}
  Rearranging, we get $ \bar v_t - \bar v \leq (z_0 - \bar v)e^{-\tau(t-t_0)}, $ for all $t\geq t_0$.
  This implies the desired result in this step.

  \emph{Step 4.}
  We will show that there exist $c_3, \delta_2, t_1>0$ such that
  \[
    |\bar v - v_t(k_t^{-1} e^{-\alpha t})|\leq
    c_3k_t^{\delta_2}, \quad t\geq t_1.
  \]
  Define $\rho_t := 1+(\log k_t)/(t\alpha)$ for all $t\geq 0$.
  By the fact that $k_t^{-1}e^{-\alpha t} = e^{-\alpha \rho_t t}$ for  all $t\geq 0$ and the condition that $k_t e^{\alpha t} \xrightarrow[t\to \infty]{} \infty$, we have $\rho_t t \xrightarrow[t\to \infty]{} \infty $.
  Since the $L\log L$ condition is satisfied, we have (see \cite{LiuRenSong2009Llog} for example), $W_t \xrightarrow[t\to \infty]{a.s.} W_\infty$, where the martingale limit $W_\infty$ is a non-degenerate positive random variable. 
  This implies that
  \[
    v_t(e^{-\alpha t})
    = -\log \mathbf P_1[e^{-W_t}]\xrightarrow[t\to \infty]{} - \log \mathbf P_{1}[e^{-W_\infty}]
    =: z^* \in (0,\infty).
  \]
  The $L \log L$ condition also guarantees that (see again \cite{LiuRenSong2009Llog} for example) $\{W_\infty = 0\} = \{\exists t \geq 0, X_t= 0\}$  a.s. in $\mathbf P_1$. This and the non-degeneracy of $W_\infty$ imply that
  \[
    z^*
    = -\log \mathbf P_1[e^{-W_\infty}]
    < -\log \mathbf P_1(W_\infty = 0) = \bar v.
  \]

  Fix an arbitrary $\epsilon \in (0,\tau)$.
  According to the fact that $\tau=\psi'(\bar v)>0$, there exists $z_0 \in (0,\bar v)$ such that for all $z\in (z_0, \bar v)$, we have $-\psi(z)\geq (\bar v - z)(\tau- \epsilon)$.
  Fix this $z_0$.
  For $t$ large enough, we have $0<k_t^{-1}e^{-\alpha t} < v_t(k_t^{-1}e^{-\alpha t})< \bar v$.
  % Then using \eqref{eq:svp2} with $\lambda=k_t^{-1} e^{-\alpha t}$, we have for $t>0$ large enough,
  Then we have for $t>0$ large enough,
  \begin{align}
    t
    & =\int^{v_t(k_t^{-1} e^{-\alpha t})}_{k_t^{-1} e^{-\alpha t}}\frac{dz}{-\psi(z)}
      = \Big(\int^{v_{\rho_t t}(e^{-\alpha \rho_t t})}_{e^{-\alpha \rho_t t}}  + \int^{z_0}_{v_{\rho_t t}(e^{-\alpha \rho_t t})} +\int^{v_t(k_t^{-1}e^{-\alpha  t})}_{z_0}\Big)\frac{dz}{-\psi(z)} \\
    & = \rho_t t + O(1) +\int^{v_t(k_t^{-1}e^{-\alpha t})}_{z_0} \frac{dz}{-\psi(z)},
  \end{align}
  where we used the fact that
  \[
    \int_{v_{\rho_t t}(e^{-\alpha \rho_tt})}^{z_0} \frac{dz}{-\psi(z)}
    \xrightarrow[t\to \infty] {} \int_{z^*}^{z_0} \frac{dz}{-\psi(z)}.
  \]
  Now we have, for $t$ large enough,
  \begin{align}
    & t
      \leq  \rho_t t + O(1) + \int_{z_0}^{v_t(k_t^{-1}e^{-\alpha t})} \frac{dz}{(\bar v-z)(\tau - \epsilon)} \\
    & =  \rho_t t +O(1)- \frac{1}{\tau-\epsilon}\Big( \log \big(\bar v-v_t(e^{-\alpha \rho_t t})\big) - \log(\bar v-z_0)\Big).
  \end{align}
  Rearranging, we get, for $t$ large enough,
  \[
    e^{-t(\tau - \epsilon)}
    \geq e^{-\rho_t t(\tau - \epsilon)+O(1)}(\bar v - v_t(e^{-\alpha \rho_t t})).
  \]
  Therefore, there exist $c_3>0$ and $t_1>0$ such that for all $t\geq t_1$,
  \[
    \bar v - v_t(k_t^{-1} e^{-\alpha t})
    \leq e^{-t(\tau -\epsilon)+ (1+\frac{\log k_t}{t\alpha})t(\tau - \epsilon)+O(1)}
    \leq c_3k_t^{\frac{\tau - \epsilon}{\alpha}}.
  \]
  This implies the desired result in this step.
  
  Finally, by Steps 2-4, we have for each $x\geq 0$, there exist $c_4, \delta_3, t_2 > 0$ such that
  \[
    \mathbf P_{x}(0< W_t\leq k_t)
    \leq c_4(k_t^{\delta_3}+e^{-\delta_3 t})
    , \quad t\geq t_2.
  \]
  Note that the left side is always bounded by $1$, so we can take $t_2 =0$ in the above statement.
\end{proof}

% *** Moments for super-OU processes
\subsection{Moments for super-OU processes}
\label{sec: Moments for super-OU processes}
In this subsection,  we want to find some upper bound for the $(1+\gamma)$-th moment of $\langle g ,X_t \rangle$, where $\gamma \in (0,\beta)$, $g\in \mathcal P$ and $\{(X_t)_{t\geq 0}; (\mathbb P_\mu)_{\mu \in \mathcal M(\mathbb R^d)}\}$ is the super OU process considered in Subsection \ref{sec:I:R} satisfying Assumption \ref{asp: Greys condition} and \ref{asp: branching mechanism}.
% Recall that the operator $\mathcal{I}^t_s$ is defined in \eqref{Ist}.

\begin{lem}
  \label{lem: control pair for P(M>lambda)}
  There is a $(\theta^2\vee\theta^{1+\beta})$-controller $R$ such that for all $0\leq t\leq 1$, $g\in \mathcal P$, $\lambda >0$ and $\mu\in \mathcal M_c(\mathbb R^d)$, we have
  \[
    \mathbb P_\mu ( |\mathcal{I}_0^t\langle g,X_t\rangle| > \lambda)
    \leq \frac{\lambda}{2}\int_{-2/\lambda}^{2/\lambda}\langle R|\theta g|,\mu\rangle d\theta.
  \]
\end{lem}

\begin{proof}
  It is elementary calculus (see the proof of \cite[Theorem 3.3.6]{Durrett2010Probability} for example) that for each $ux \neq 0$ in $\mathbb R$,
  \[\frac{1}{u}\int_{-u}^u (1- e^{i\theta x})~d\theta = 2 - \frac{2\sin ux}{ux} \geq \mathbf 1_{ux>2}.\]
  Denote by $R$ the $(\theta^2\vee\theta^{1+\beta})$-controller in Lemma \ref{lem: upper bound for usgx}.(4).
  Therefore, using Lemma \ref{lem: estimate of exponential remaining} we get
  \begin{align}
    & |\mathbb P_\mu (|\mathcal{I}_0^t\langle g,X_t\rangle| > \lambda)|
      \leq \Big|\frac{\lambda}{2}\int_{-2/\lambda}^{2/\lambda}(1 - \mathbb P_\mu[e^{i\theta \mathcal{I}_0^t\langle g,X_t\rangle}])d\theta\Big| \\
    & \leq \frac{\lambda}{2}\int_{-2/\lambda}^{2/\lambda}|1-e^{\langle U_t(\theta g)-iP^\alpha_t (\theta g),\mu \rangle}|d\theta
    \leq \frac{\lambda}{2}\int_{-2/\lambda}^{2/\lambda}\langle |U_t(\theta g) - iP^\alpha_t(\theta g)|,\mu\rangle d\theta \\
    & \leq \frac{\lambda}{2}\int_{-2/\lambda}^{2/\lambda}\langle R|\theta g|,\mu\rangle d\theta.
      \qedhere
  \end{align}
\end{proof}

\begin{lem}
  \label{lem: temp}
  For all $h \in \mathcal P^+$ and $\mu \in \mathcal M_c(\mathbb R^d)$, there exists $C > 0$ such that for all $\kappa \in \mathbb Z_+ $, $\lambda > 0$ and $0\leq r\leq s\leq t<\infty$ with $s-r \leq 1$, we have
  \[
    \sup_{g \in \mathcal P: Q_\kappa g\leq h}\mathbb P_{\mu}(|\mathcal I_r^s\langle g, X_t\rangle|>\lambda)
    \leq C e^{\alpha r} \Big(\Big( \frac{e^{(t-s)(\alpha - \kappa b)}}{\lambda}\Big)^{1+\beta} + \Big( \frac{e^{(t-s)(\alpha - \kappa b)}}{\lambda}\Big)^{2} \Big).
  \]
\end{lem}

\begin{proof}
  Denote by $R$ the $(\theta^2\vee\theta^{1+\beta})$-controller in Lemma \ref{lem: control pair for P(M>lambda)}.
  Fix $h \in \mathcal P^+$, $\mu \in \mathcal M_c(\mathbb R^d)$ $\kappa \in \mathbb Z_+ $ and $0\leq r\leq s\leq t < \infty$ with $s-r \leq 1$.
  Suppose that $g\in \mathcal P$ satisfies $Q_\kappa g \leq h$.
  Using the Markov property of $X$, we get
  \begin{align}
    & \mathbb P_{\mu}(|\mathcal I_r^s\langle g, X_t\rangle|>\lambda)
      = \mathbb P_\mu \Big[\mathbb P_\mu [|\langle P_{t-s}^\alpha g, X_{s}\rangle - \langle P_{t-r}^\alpha g, X_{r}\rangle|> \lambda | \mathscr F_r ]\Big] \\
    & = \mathbb P_\mu \big[\mathbb P_{X_r}(|\langle P_{t-s}^\alpha g, X_{s-r}\rangle - \langle P_{t-r}^\alpha g, X_{0}\rangle|> \lambda)\big] \\
    & = \mathbb P_\mu \big[\mathbb P_{X_r}(|\mathcal I_0^{s-r}\langle P_{t-s}^\alpha g, X_{s-r}\rangle |> \lambda)\big]
      \leq \mathbb P_\mu \Big[ \frac{\lambda}{2}\int_{-2/\lambda}^{2/\lambda}\langle R|\theta P^\alpha_{t-s}g|,X_r\rangle d\theta \Big] \\
    & \leq \mathbb P_\mu \Big[ \frac{\lambda}{2}\int_{-2/\lambda}^{2/\lambda}\langle R|\theta e^{(t-s)(\alpha- \kappa b)}h|,X_r\rangle d\theta \Big] \\
    & \leq \mathbb P_\mu [ \langle Rh,X_r\rangle ] \frac{\lambda}{2}\int_{-2/\lambda}^{2/\lambda}(|\theta e^{(t-s)(\alpha- \kappa b)}|^{1+\beta} + |\theta e^{(t-s)(\alpha- \kappa b)}|^{2})d\theta
    \\ & =  \langle P_r^\alpha Rh,\mu\rangle \Big(  \frac{2^{2+\beta}}{2+\beta}\Big(\frac{e^{(t-s)(\alpha- \kappa b)}}{\lambda}\Big)^{1+\beta} + \frac{2^{3}}{3}\Big(\frac{e^{(t-s)(\alpha- \kappa b)}}{\lambda}\Big)^2\Big)
    \\ & \leq C e^{\alpha r} \Big(\Big( \frac{e^{(t-s)(\alpha - \kappa b)}}{\lambda}\Big)^{1+\beta} + \Big( \frac{e^{(t-s)(\alpha - \kappa b)}}{\lambda}\Big)^{2} \Big),
  \end{align}
  where the $C>0$ above is chosen as
  \[
    C 
    := \Big(\frac{2^{2+\beta}}{2+\beta} + \frac{2^{3}}{3} \Big)\langle Q_0Rh, \mu\rangle.
    \qedhere
  \]
\end{proof}

For each random variable $\{Y; \mathbb P\}$ and $p \in [1,\infty)$, we write $ \|Y\|_{\mathbb P;p} := \mathbb P[|Y|^p]^{1/p}$.
Recall that we write $\tilde u = \frac{u}{1+u}$ for each $u\neq -1$.
\begin{lem}
  \label{lem: control of mgtrs}
  For all $h \in \mathcal P$, $\mu \in \mathcal M_c(\mathbb R^d)$ and $\gamma\in (0, \beta)$, there exists $C > 0$ such that for all $\kappa \in \mathbb Z_+$ and $0\leq r \leq s\leq t<\infty$ with $s-r \leq 1$, we have
  \[
    \sup_{g \in \mathcal P: Q_\kappa g \leq h} \|\mathcal I_r^s\langle g, X_t\rangle \|_{\mathbb P_\mu;1+\gamma}
    \leq C e^{t\alpha (1- \tilde \gamma)+(t-s) (\alpha \tilde \gamma - \kappa b)}.
  \]
\end{lem}

\begin{proof}
  Fix $h \in \mathcal P$ and $\mu \in \mathcal M_c(\mathbb R^d)$. Let $C_0$ be the constant in the Lemma \ref{lem: temp}.
  For all $\kappa \in \mathbb Z_+$,  $0\leq r\leq s\leq t$ with $s-r \leq 1$,  $g\in \mathcal P$ with $Q_{\kappa} g \leq h$, and $c>0$, we have
  \begin{align}
    & \mathbb P_\mu\big[|\mathcal I_r^s\langle g, X_t\rangle|^{1+\gamma}\big]
      = (1+\gamma)\int_0^\infty \lambda^{\gamma} \mathbb P_{\mu}(|\mathcal I_r^s\langle g, X_t\rangle|>\lambda) d\lambda \\
    & \leq (1+\gamma)\int_0^c \lambda^{\gamma} d\lambda +(1+\gamma)\int_c^\infty \lambda^{\gamma}\mathbb P_\mu(|\mathcal I_r^s\langle g, X_t\rangle|> \lambda) d\lambda \\
    & \leq c^{1+\gamma} + C_0  e^{\alpha r}(1+\gamma)\int_c^\infty \bigg(\Big(\frac{e^{(t-s)(\alpha - \kappa b)}}{\lambda}\Big)^{1+\beta}+\Big(\frac{e^{(t-s)(\alpha - \kappa b)}}{\lambda}\Big)^{2}\bigg)\lambda^{\gamma}d\lambda \\
    & \leq c^{1+\gamma} e^{\alpha r} + C_0e^{\alpha r}(1+\gamma)\Big(  \frac{e^{(1+\beta)(t-s)(\alpha- \kappa b)}}{(\beta - \gamma)c^{\beta - \gamma}}  + \frac{e^{2(t-s)(\alpha- \kappa b)}}{(1 - \gamma)c^{1 - \gamma}} \Big).
  \end{align}
  Taking $c = e^{(t-s)(\alpha- \kappa b)}$, we get
  \begin{align}
    & \mathbb P_\mu\big[|\mathcal I_r^s\langle g, X_t\rangle|^{1+\gamma}\big]
      \leq e^{(1+\gamma)(t-s)(\alpha- \kappa b)} e^{\alpha r}\Big(1+ C_0 \frac{1+\gamma}{\beta - \gamma}+ C_0 \frac{1+\gamma}{1 - \gamma}\Big).
  \end{align}
  Note that
  \begin{align}
    & (1+\gamma) (t-s) (\alpha- \kappa b) + \alpha r
      = (t-s)\alpha+(t-s) (\gamma \alpha- (1+\gamma )\kappa b) \\
    & \leq t\alpha+(t-s) (\gamma \alpha- (1+\gamma)\kappa b).
  \end{align}
  So the desired result is true.
\end{proof}

\begin{lem}
  \label{lem:P:M:uc}
  For all $h \in \mathcal P$, $\mu \in \mathcal M_c(\mathbb R^d)$, $\gamma\in (0, \beta)$ and $\kappa \in \mathbb Z_+$, there exists a constant $C > 0$ such that for each $t\geq 0$, we have
  \begin{enumerate}
  \item
    \label{item:P:M:uc:1}
    $\sup_{g\in \mathcal P: Q_\kappa g \leq h}\|\langle g,X_t\rangle\|_{\mathbb{P}_{\mu};1+\gamma}\leq C e^{(\alpha-\kappa b)t}$ provided $\alpha \tilde \gamma > \kappa b$;
  \item
    \label{item:P:M:uc:2}
    $\sup_{g\in \mathcal P: Q_\kappa g \leq h}\|\langle g,X_t\rangle\|_{\mathbb{P}_{\mu};1+\gamma}\leq C te^{\frac{\alpha}{1+\gamma}t}$ provided $\alpha \tilde \gamma = \kappa b$;
  \item
    \label{item:P:M:uc:3}
    $\sup_{g\in \mathcal P: Q_\kappa g \leq h} \|\langle g,X_t\rangle\|_{\mathbb{P}_{\mu};1+\gamma}\leq C e^{\frac{\alpha}{1+\gamma}t}$ provided $\alpha \tilde \gamma < \kappa b$.
  \end{enumerate}
\end{lem}

\begin{proof}
  Fix $\gamma \in (0,\beta)$ and $\mu \in \mathcal M_c(\mathbb R^d)$.
  Let $C$ be the constant in Lemma \ref{lem: control of mgtrs}.
  Using the triangle inequality, for all $\kappa\in \mathbb Z_+$, $g \in \mathcal P$ with $Q_\kappa g \leq h$ and $t\geq 0$, we have
  \begin{align}
    & \|\langle g,X_t\rangle\|_{\mathbb P_\mu;1+\gamma}
      \leq \sum_{l=0}^{\lfloor t\rfloor - 1}\big\| \mathcal{I}_{t-l-1}^{t-l}\langle g,X_t\rangle \big\|_{\mathbb P_\mu;1+\gamma}+\big\| \mathcal{I}_{0}^{t-\lfloor t \rfloor}\langle g,X_t\rangle  \big\|_{\mathbb P_\mu;1+\gamma} + |\langle P^\alpha_t g,\mu\rangle| \\
    & \leq C^{\frac{1}{1+\gamma}} e^{\frac{\alpha}{1+\gamma}t} \sum_{l=0}^{\lfloor t\rfloor} e^{\frac{\gamma\alpha-\kappa (1+\gamma)b}{1+\gamma} l} + e^{(\alpha - \kappa b)t} \langle h,\mu\rangle.
  \end{align}
  By calculating the sum on the right, we get the desired result.
\end{proof}

% ** Proofs of main results
\section{Proofs of main results}
\label{proofs of main results}
In this section, we will prove the main results of this paper. Recall that $\mathcal{M}_c(\mathbb{R}^d)$ is the space of all finite Borel measures of compact support on $\mathbb{R}^d$.
For simplicity, we will write $\mathbb{\widetilde{P}}_{\mu}=\mathbb{P}_{\mu}(\cdot|D^c)$ in this section.

% *** law of large numbers
\subsection{Law of large numbers}
\label{sec: large rate lln}

In this subsection, we prove Theorem \ref{thm: law of large number}.
For this purpose, we first prove the almost sure and $L^{1+\gamma}(\mathbb{P}_{\mu})$ convergence of a family of martingales for $\gamma\in (0, \beta)$. Recall that $L$ is the infinitesimal generator of the OU-process.  For $f\in \mathcal{P}\cap C^2(\mathbb R^d)$ and  $a\in \mathbb R$, we define
\begin{align}
  \label{defmartingale}
  M_t^{f,a}
  :=e^{-(\alpha-ab)t}\langle f,X_t\rangle-\int_0^t e^{-(\alpha-ab)s}\langle (L+ab)f, X_s\rangle~ ds.
\end{align}
Let $(\mathscr{F}_t)_{t\geq 0}$ be the natural filtration of $X$.  The following lemma says that $\{M_t^{f,a}: t\geq 0\}$ is a martingale with respect to $(\mathscr{F}_t)_{t\geq 0}$.

\begin{lem}
  \label{lemma25}
  For all $f\in \mathcal{P}\cap C^2(\mathbb R^d)$, $a\in \mathbb R$ and $\mu\in \mathcal M_c(\mathbb R^d)$, the process $(M_t^{f,a})_{t\geq 0}$ is a $\mathbb P_\mu$-martingale with respect to $(\mathscr F_t)_{t\geq 0}$.
\end{lem}

\begin{proof}
  Put$\bar{f} :=(L+ab)f$.
  It follows easily from Ito's formula that
  \begin{align}
    \label{Theorem55}
    P_t^{ab}f(x)
    = f(x)+\int_0^t P_s^{ab}\bar{f}(x)~ds,\quad t\geq 0,x\in \mathbb R^d,
  \end{align}
  where $P_t^{ab} := e^{abt}P_t$.
  For $0\leq s\leq t$, we have
  \begin{align}
    \label{martingale1}
    & \quad\mathbb{P}_{\mu}[M_t^{f,a}|\mathscr{F}_s]
    =e^{-(\alpha-ab)t}\mathbb{P}_{\mu}\left[\langle f,X_t\rangle|\mathscr{F}_s\right]-\mathbb{P}_{\mu}\Big[\int_0^t e^{-(\alpha-ab)u}\langle \bar{f}, X_u\rangle~ du\Big|\mathscr{F}_s\big] \\
    & =e^{-(\alpha-ab)t}\langle P_{t-s}^{\alpha}f, X_s\rangle-\int_0^s e^{-(\alpha-ab)u}\langle \bar{f}, X_u\rangle~ du - \int_s^t e^{-(\alpha-ab)u}\langle P_{u-s}^{\alpha} \bar{f},X_s\rangle~ du.
  \end{align}
  Using \eqref{Theorem55} and Fubini's theorem, we have
  \begin{align}
    & \int_s^t e^{-(\alpha-ab)u}\langle P_{u-s}^{\alpha} \bar{f},X_s\rangle~ du=e^{-(\alpha-ab)s}\int_s^t\langle P_{u-s}^{ab}\bar{f},X_s\rangle~du\\
    & = e^{ - ( \alpha - ab ) s } \Big \langle \int_0^{t-s} P_{u}^{ab} \bar{f}~ du, X_s \Big \rangle
      = e^{-(\alpha-ab)s}\left(\langle P_{t-s}^{ab}f,X_s\rangle - \langle f, X_s \rangle \right) \\
    & = e^{-(\alpha-ab)t}\langle P_{t-s}^{\alpha}f, X_s\rangle - e^{ - ( \alpha - ab ) s} \langle f,X_s\rangle.
  \end{align}
  Using this and \eqref{martingale1}, we get the desired result.
\end{proof}

Recall that, for $p\in \mathbb Z_+^d$,  $\phi_p$ is an eigenfunction of $L$ corresponding to the eigenvalue $-|p|b$. 
Define $ H_t^p :=e^{-(\alpha-|p|b)t}\langle\phi_p,X_t\rangle$ for each $t\geq 0$.

\begin{lem}
  \label{lem:M:L:ML}
   For each $\mu\in \mathcal M_c(\mathbb R^d)$ and $p \in \mathbb Z_+^d$, $(H^p_t)_{t\geq 0}$ is a $\mathbb P_{\mu}$-martingale with respect to $(\mathscr F_t)_{t\geq 0}$.
    Moreover if $\alpha\tilde \beta>|p|b$, the martingale is bounded in $L^{1+\gamma}(\mathbb P_\mu)$ for each $\gamma\in (0, \beta)$.
  Thus the limit $ H_{\infty}^p := \lim_{t\rightarrow \infty}H_t^p $  exists $\mathbb{P}_{\mu}$-a.s. and in $L^{1+\gamma}(\mathbb{P}_{\mu})$ for each $\gamma \in (0,\beta)$.
\end{lem}

\begin{proof}
  Fix the measure $\mu \in \mathcal M_c(\mathbb R^d)$ and the multi-index $p \in \mathbb Z_+^d$.
  It follows from Lemma \ref{lemma25} that $(H_t^p)_{t\geq 0}$ is a $\mathbb P_\mu$-martingale.
  Further suppose that $\alpha \tilde \beta > |p| b$.
  Then there exists a $\gamma_0 \in (0,\beta)$ which is close enough to $\beta$ so that $\alpha\tilde \gamma>|p|b$ for each $\gamma\in [\gamma_0, \beta)$.
  Using  Lemma \ref{lem:P:M:uc} and the fact $\kappa_{\phi_p}=|p|$, we get that, for each $\gamma\in [\gamma_0, \beta)$, there exists a constant $C>0$ such that
  \[
    \|H_t^p\|_{\mathbb P_\mu;1+\gamma}
    \leq c e^{-(\alpha-|p|b)t}e^{(\alpha-|p|b)t}
    = C
    , \quad t\geq 0.
  \]
  % For each $\gamma\in (0, \gamma_0)$ there exists a constant $C>0$, such that
  For each $\gamma\in (0, \gamma_0)$ there exists a constant $C'>0$, such that
  \[
    \| H_t^p \|_{\mathbb P_\mu;1+\gamma}
    \leq \| H_t^p \|_{\mathbb P_\mu;1+\gamma_0}
    % \leq C,
    \leq C',
    \quad t\geq 0.
  \]
  Therefore, for each $\gamma \in (0,\beta)$, the martingale $(H_t^p)_{t\geq 0}$ is bounded in $L^{1+\gamma}(\mathbb{P}_{\mu})$.
\end{proof}

\begin{lem}
  \label{lem: control of wt}
  Suppose that $\mu\in \mathcal M_c(\mathbb R^d)$ and that $p \in \mathbb Z_+^d$ satisfies $\alpha \tilde \beta > |p|b$.
  Then for each $\gamma \in (0,\beta)$ satisfying $\alpha \tilde \gamma > |p|b$, there exists a constant $C> 0$ such that,
  \[
    \|H^p_t-H^p_s\|_{\mathbb{P}_{\mu};1+\gamma}
    \leq C e^{-(\alpha \tilde \gamma-|p|b)s},
    \quad 0 \leq s < t \leq \infty.
  \]
\end{lem}

\begin{proof}
  Thanks to Lemma \ref{lem:M:L:ML}, we only  need to prove the inequality  when $0\leq s < t<\infty$.
  Suppose $p\in \mathbb{Z}_+^d$, $\mu\in \mathcal M_c(\mathbb R^d)$ and  $\gamma \in (0,\beta)$ with $\alpha \tilde \gamma > |p|b$ are fixed.
  Using Lemma \ref{lem: control of mgtrs} with $g=\phi_p$ and $k=|p|$,  we know that there exists a constant $C_1>0$ such that for all $0\leq r\leq s $ with $s-r\leq1$,
  \begin{align}
    \|H^p_s-H^p_r\|_{\mathbb P_\mu; 1+\gamma}
    \leq  C_1 e^{-(\alpha\tilde \gamma-|p|b)s}.
  \end{align}
  Thus there exists $C_2>0$ such that for any $0\leq s<t$,
  \begin{align}
    & \|H^p_t-H^p_s\|_{\mathbb{P}_{\mu};1+\gamma} \\
    & \leq \|H^p_{\lfloor s \rfloor+1}-H^p_s\|_{\mathbb{P}_{\mu};1+\gamma}+\sum_{k=\lfloor s \rfloor+1}^{\lfloor t \rfloor}\|H^p_{k+1}-H^p_{k}\|_{\mathbb{P}_{\mu};1+\gamma}+\|H^p_t-H^p_{\lfloor t \rfloor+1}\|_{\mathbb{P}_{\mu};1+\gamma} \\
    & \leq C_1 \Big(e^{-(\alpha \tilde \gamma- |p|b) s}+\sum_{k=\lfloor s \rfloor+1}^{\lfloor t \rfloor} e^{-(\alpha \tilde \gamma- |p|b) k} + e^{-(\alpha \tilde \gamma-|p|b) t}\Big)
      \leq C_2e^{-(\alpha \tilde \gamma-|p|b)s}.
      \qedhere
  \end{align}	
\end{proof}

\begin{proof}[Proof of Theorem \ref{thm: law of large number}]
	Fix $f \in \mathcal P$ such that $\alpha \beta > \kappa_f b (1+\beta)$ and $\mu \in \mathcal M_c(\mathbb R^d)$.
	Write
  \[  
    f
    = \sum_{p\in \mathbb Z_+^d:|p|\geq \kappa_f}\langle f,\phi_p\rangle_\varphi \phi_p
    =: \sum_{p\in \mathbb Z_+^d:|p|= \kappa_f}\langle f,\phi_p\rangle_\varphi \phi_p+\widetilde{f}.
  \]
	Then
  \begin{align}
    & e^{-(\alpha-\kappa_fb)t}\langle f,X_t\rangle=
      \sum_{p\in \mathbb Z_+^d:|p|= \kappa_f}\langle f,\phi_p\rangle_\varphi H_t^p+e^{-(\alpha-\kappa_fb)t} \langle \widetilde{f},X_t\rangle,
      \quad t\geq 0.
  \end{align}
	According to Lemma \ref{lem:M:L:ML}, we have
  \begin{align}
    \label{as convergence}
    \sum_{p\in \mathbb{Z}_+^d:|p|= \kappa_f}\langle f,\phi_p\rangle_\varphi H_t^p
    \xrightarrow[t\to \infty]{} \sum_{p\in \mathbb{Z}_+^d:|p|=\kappa_f}\langle f, \phi_p\rangle_{\varphi} H_{\infty}^p,
  \end{align}
  $\mathbb{P}_{\mu}$-a.s. and in $L^{1+\gamma}(\mathbb{P}_{\mu})$ for each $\gamma\in(0,\beta)$.
  Therefore, it suffices to show that
  \begin{align}
    J_t
    :=e^{-(\alpha-\kappa_fb)t}\langle \widetilde{f},X_t\rangle,
    \quad t\geq 0,
  \end{align}
  converges in $L^{1+\gamma}(\mathbb{P}_{\mu})$ for all $\gamma\in(0,\beta)$, and converges almost surely provided $f$ is twice differentiable and all its second order partial derivatives are in $\mathcal{P}$.

  \emph{Step 1.} Let $g\in \mathcal P$.
  Let $\kappa > 0$ be such that $\kappa < \kappa_g$ and $\kappa b < \alpha \tilde \beta$.
  We will show that for each $\gamma \in (0,\beta)$ there exist $C_1,\delta_1 > 0$ such that
  \[
    \|e^{-(\alpha - \kappa b)t} \langle g, X_t\rangle\|_{\mathbb P_\mu;1+\gamma}
    \leq C_1 e^{-\delta_1 t},
    \quad t\geq 0.
  \]
  In order to do this, we choose a $\gamma_0 \in (0,\beta)$ close enough to $\beta$ such that, $\kappa b< \alpha \tilde \gamma$ for each $\gamma \in [\gamma_0, \beta)$.
  According to Lemma \ref{lem:P:M:uc}, we have for each $\gamma \in (0,\beta)$,
  \begin{enumerate}
  \item
    if $\gamma \in [\gamma_0, \beta)$ and $\alpha\tilde \gamma> \kappa_g b$, then there exists $C_2>0$ such that
    \[
      \|e^{-(\alpha - \kappa b)t} \langle g, X_t\rangle\|_{\mathbb P_\mu;1+\gamma}
      \leq C_2 e^{-(\alpha-\kappa b)t}e^{(\alpha-\kappa_g b)t}
      \leq C_2  e^{-(\kappa_g - \kappa )bt},
      \quad t\geq 0;
    \]
  \item
    if $\gamma \in [\gamma_0, \beta)$ and $\alpha\tilde \gamma=\kappa_g b$, then there exists $C_3>0$ such that
    \[
      \|e^{-(\alpha - \kappa b)t} \langle g, X_t\rangle\|_{\mathbb P_\mu;1+\gamma}
      \leq C_3 t e^{-(\alpha - \kappa b)t}e^{\frac{\alpha}{1+\gamma}t}
      = C_3 t e^{-(\alpha \tilde \gamma - \kappa b)t},
      \quad t\geq 0;
    \]
  \item
    if $\gamma \in [\gamma_0, \beta)$ and $\alpha\tilde \gamma < \kappa_g b$, then there exists $C_4>0$ such that
    \[
      \|e^{-(\alpha - \kappa b)t} \langle g, X_t\rangle\|_{\mathbb{P}_{\mu};1+\gamma}
      \leq C_4  e^{-(\alpha - \kappa b)t}e^{\frac{\alpha}{1+\gamma}t}
      = C_4  e^{-(\alpha \tilde \gamma - \kappa b)t},
      \quad t\geq 0;
    \]
  \item
    if $\gamma \in (0,\gamma_0)$ then, thanks to (1)--(3) above and the fact that \[\|e^{-(\alpha - \kappa b)t} \langle g, X_t\rangle\|_{\mathbb{P}_{\mu};1+\gamma}
      \leq \|e^{-(\alpha - \kappa b)t} \langle g, X_t\rangle\|_{\mathbb{P}_{\mu};1+\gamma_0},\] there exist $C_5, \delta_2 >0$ such that
    \[
      \|e^{-(\alpha - \kappa b)t} \langle g, X_t\rangle\|_{\mathbb{P}_{\mu};1+\gamma}
      \leq C_5e^{-\delta_2 t},
      \quad t\geq 0.
    \]
  \end{enumerate}
  Thus, the desired conclusion in this step is valid.
  In particular, by taking $g = \widetilde f$ and $\kappa = \kappa_f$, we get that $J_t$ converges to $0$ in $L^{1+\gamma}(\mathbb{P}_{\mu})$ for any $\gamma\in(0,\beta)$.
  
  \emph{Step 2.}
  We further assume that $f\in C^2(\mathbb R^d)$ and $D^2f \in \mathcal{P}$.
  We will show that $J_t$ converges to $0$ almost surely.
  For $a \geq 0$, $ t\geq 0$, and $g\in \mathcal{P}\cap C^2(\mathbb{R}^d)$ satisfying $D^2g\in \mathcal{P}$, we define
  \begin{align}
    L_t^{g,a}
    & :=\int_0^t e^{-(\alpha-ab)s}\langle (L+ab)g,X_s\rangle ds,\\
    Y_t^{g,a}
    & :=\int_0^t e^{-(\alpha-ab)s}|\langle (L+ab)g,X_s\rangle|ds.
  \end{align}
  Now choose $a_0 \in (\kappa_{f}, \kappa_f + 1)$ close enough to $\kappa_f$ so that $a_0 b < \alpha \tilde \beta$.
  According to \eqref{defmartingale},
  \begin{align}
    J_t
    = e^{-(a_0-\kappa_f)bt} (M_t^{\widetilde{f}, a_0}+L_t^{\widetilde{f}, a_0}),
    \quad t\geq 0.
  \end{align}
  So we only need to show that
  \begin{align}
    e^{-(a_0-\kappa_f)b t}M_t^{\widetilde{f},a_0}
    \xrightarrow[t\to \infty]{} 0,
    \quad e^{-(a_0-\kappa_f)b t}L_t^{\widetilde{f},a_0}
    \xrightarrow[t\to \infty]{} 0
    \quad \mathbb{P}_{\mu}\text{-a.s.}
  \end{align}
  Notice that $\kappa_{(L+a_0 b)\widetilde{f}}\geq \kappa_{\widetilde{f}}\geq \kappa_f+1 > a_0$.
  By Step 1, for any fixed $\gamma\in (0,\beta)$, there exist $C_6, \delta_3>0$ such that for each $t\geq 0$,
  \[
    \| e^{-(\alpha-a_0 b)t}\langle \widetilde{f},X_t\rangle)\|_{\mathbb{P}_{\mu};1+\gamma}
    \leq C_6 e^{-\delta_3 t},
    \quad \|e^{-(\alpha-a_0 b)t}\langle L\widetilde{f}+a_0 b\widetilde{f},X_t\rangle\|_{\mathbb{P}_{\mu};1+\gamma}
    \leq C_6 e^{-\delta_3 t}.
  \]
  Now, by the triangle inequality, for each $t\geq 0$,
  \begin{align}
    & \|L_t^{\widetilde{f},a_0}\|_{\mathbb{P}_{\mu};1+\gamma}
      \leq\|Y_t^{\widetilde{f},a_0}\|_{\mathbb{P}_{\mu};1+\gamma} \\
    & \leq \int_0^t \|e^{-(\alpha-a_0 b)s}\langle L\widetilde{f}+a_0 b\widetilde{f},X_s\rangle\|_{\mathbb{P}_{\mu};1+\gamma}ds\leq C_6 \int_0^t e^{-\delta_3 s}ds\leq\frac{C_6}{\delta_3}.
  \end{align}
  Since $Y_t^{\widetilde{f},a_0}$ is increasing in $t$, it converges to some finite random variable $Y_{\infty}^{\widetilde{f},a_0}$ almost surely and in $L^{1+\gamma}(\mathbb{P}_{\mu})$.
  Consequently,  we have
  \begin{align}
    \lim_{t\rightarrow \infty}e^{-(a_0 - \kappa_f)bt}|L_t^{\widetilde{f},a_0}|
    \leq  \lim_{t\rightarrow \infty}e^{-(a_0 - \kappa_f)bt}|Y_t^{\widetilde{f},a_0}|=0, 
    \quad \mathbb P_\mu\text{-a.s.}
  \end{align}
  On the other hand, the martingale $M_t^{\widetilde{f},a_0}$ satisfies
  \begin{align}
    \|M_t^{\widetilde{f},a_0}\|_{\mathbb{P}_{\mu};1+\gamma}
    \leq \|e^{-(\alpha-a_0 b)t}\langle \widetilde{f},X_t\rangle)\|_{\mathbb{P}_{\mu};1+\gamma}+\|L_t^{\widetilde{f},a_0}\|_{\mathbb{P}_{\mu};1+\gamma}
    \leq C_6(e^{-\delta_3 t}+\frac{1}{\delta_3}),
    \quad t\geq 0.
  \end{align}
  This implies that the martingale converges almost surely.
  Consequently,
  \[
    \lim_{t\rightarrow\infty} e^{-(a_0-\kappa_f)bt}M_t^{\widetilde{f},a_0}
    = 0,
    \quad \mathbb P_\mu\text{-a.s.}.
    \qedhere
  \]
\end{proof}

% **** Central limit theorems for one unite time interval
\subsection{Central limit theorems for one unite time intervals}
\label{sec:critical}
% ***** Local CLT for one interval 
In this subsection, we will establish the following  central limit theorem for our super-OU processes.
\begin{thm}
  \label{lem:PR:LC}
  Let $\mu \in \mathcal M_c(\mathbb R^d)$ and $f\in \mathcal{P}\setminus \{0\}$.
  Then under $\mathbb{P}_{\mu}(\cdot | D ^c)$, we have
  \begin{align}
    \label{eq:PR:LC:1}
    \Upsilon^f_t
    := \frac{X_{t+1} (f) - X_t(P_1^\alpha f)}{\| X_t\|^{1-\tilde \beta}}
    \xrightarrow[t\to \infty]{d}\zeta^f_0,
  \end{align}
  where $\zeta^f_0$ is an $(1+\beta)$-stable random variable with characteristic function $\theta\mapsto e^{\langle Z_1(\theta f), \varphi\rangle}$.
\end{thm}

In fact, we prove a stronger result:

\begin{prop}
  \label{thm:Key}
  For each $\mu \in \mathcal M_c(\mathbb R^d)$ and $g \in \mathcal P \setminus \{0\}$, there exist $C,\delta>0$ such that
  for each $t\geq 1$ and $f \in \mathcal P_g:= \{\theta T_ng:n \in \mathbb Z_+, \theta \in [-1,1]\}$, we have
  \[
    \mathbb P_\mu
    \Big[  |\mathbb P_\mu [e^{i\Upsilon^f_t} - e^{\langle Z_1f, \varphi\rangle}; D^c | \mathscr F_t ]  |\Big]
    \leq C e^{- \delta t}.
  \]
\end{prop}

\begin{proof}
  Fix the $\mu \in \mathcal M_c(\mathbb R^d)$ and the $g \in \mathcal P\setminus \{0\}$.

  \emph{Step 1.} Write $ A_t(\epsilon) :=\{ \|X_t\| \geq e^{(\alpha - \epsilon)t} \} $ with $t\geq 0$ and $\epsilon > 0$.
  We will show that for each $f\in \mathcal P \setminus \{0\}$, $\epsilon > 0$ and $t\geq 0$, it holds that
  \[
    \mathbb P_\mu \Big[ | \mathbb P_\mu [e^{i\Upsilon^f_t} - e^{\langle Z_1(\theta f), \varphi\rangle}; D^c | \mathscr F_t ]| \Big]
    \leq J^f_1(t,\epsilon)+J^f_2(t,\epsilon)+J^f_3(t,\epsilon),
  \]
where
\begin{align}
\label{eq: Def of Ji}
  J^f_1(t,\epsilon)
  & := \mathbb{P}_{\mu} [ |\langle Z'''_1(\theta_t f), X_t\rangle |; A_t(\epsilon) ],
  \\ J^f_2(t,\epsilon)
  & := \mathbb{P}_{\mu}[|\langle Z_1(\theta_t f),X_t \rangle-\langle Z_1f, \varphi\rangle |; A_t(\epsilon)],
  \\ J_3(t,\epsilon)
  & :=2\mathbb{P}_{\mu}(A_t (\epsilon)\Delta D^c),
  \\ \theta_t
  & := \|X_t\|^{-(1 - \tilde \beta)}.
\end{align}
In fact, it follows from \eqref{eq: key equality}, the definitions of $U_1$, $Z'''_1$ and $Z_1$, that for all $t\geq 0$,
\begin{align}
  \label{eq: need1}
  & \mathbb{P}_{\mu}[e^{i\Upsilon^f_t}|\mathscr{F}_t]
    = \mathbb{P}_{\mu}[\exp\{i\theta_t X_{t+1} (f) - i \theta_t X_t(P_1^\alpha f)\} |\mathscr{F}_{t}] \\
  & = \exp\{\langle (U_1 - iP^\alpha_1 ) (\theta_t f),X_t\rangle\}
    = \exp\{\langle (Z_1 + Z'''_1) (\theta_t f),X_t\rangle\}.
\end{align}
From Lemma \ref{lem: charactreisticfunction}, we  get that $\theta\mapsto \langle Z_1(\theta f),\varphi\rangle$ is the characteristic function of some $(1+\beta)$-stable random variable, and then  $\operatorname{Re} \langle Z_1f, \varphi\rangle \leq 0$.
Using this, \eqref{eq: need1}, \eqref{eq: -v has positive real part} and the fact $|e^{-x} - e^{-y}| \leq |x-y|$ for all $x,y \in \mathbb C_+$, we get for each $t\geq 0$ and $\epsilon> 0$,
\begin{align}
  \label{eq: inequality that will used later}
  & \mathbb{P}_\mu \Big[ |  \mathbb{P}_\mu [ e^{i\Upsilon^f_t} - e^{\langle Z_1f,\varphi \rangle} ; D^c | \mathscr F_{t}]   |\Big]  \\
  &  \leq \mathbb{P}_\mu   \Big[ |    \mathbb{P}_\mu [ e^{i \Upsilon^f_t }-e^{\langle Z_1f, \varphi\rangle}; A_{t}(\epsilon) | \mathscr F_{t}] |  + 2\mathbb P_\mu ( A_{t}(\epsilon) \Delta D^c | \mathscr F_{t}) \Big] \\
  & = \mathbb{P}_{\mu}\Big[ |\mathbb{P}_\mu [e^{i\Upsilon^f_t}| \mathscr F_{t}]-e^{\langle Z_1f, \varphi\rangle}| ; A_{t}(\epsilon) \Big] + J_3(t,\epsilon) \\
  & \leq \mathbb{P}_\mu \Big[ |e^{\langle (Z_1+Z'''_1) (\theta_t f),X_{t} \rangle}-e^{\langle Z_1f, \varphi\rangle} | ; A_{t}(\epsilon) \Big]+  J_3(t,\epsilon) \\
  & \leq \mathbb{P}_\mu \Big[ | \langle (Z_1+Z'''_1)(\theta_t f),X_{t} \rangle - \langle Z_1f, \varphi\rangle | ;A_{t}(\epsilon)\Big]+  J_3(t,\epsilon) \\
  & \leq J^f_1(t,\epsilon) + J^f_2(t,\epsilon)+ J_3(t,\epsilon).
\end{align}

\emph{Step 2.} We will show that for $\epsilon>0$ small enough, there exist  $C_2, \delta_2>0$ such that for each $t\geq 1$ and
$f \in \mathcal P_g$, we have $ J^f_1(t,\epsilon) \leq C_2e^{-\delta_2 t}$.

In fact, let $\delta_0 >0$ be the constant in Lemma \ref{lem: upper bound for usgx}.(7) and let $R$ be the corresponding $(\theta^{2+\beta}\vee \theta^{1+\beta+\delta_0})$-controller.
Acording to Step 1 of the proof of Lemma \ref{lem:m}, there exist $h_{2} \in \mathcal P^+$ such that for each $f \in \mathcal P_g$ it holds that $|f| \leq h_{2}$.
Then, we have for all $t\geq 0$, $\epsilon> 0$ and $f\in \mathcal P_g$,
\begin{align}
  & |Z'''_1(\theta_t f)|\mathbf{1}_{A_{t}(\epsilon)}
    \leq R(|\theta_{t} f|)\mathbf{1}_{A_{t}(\epsilon)}
    \leq R \Big(\frac{h_{2}}{e^{(\alpha-\epsilon)t(1-\tilde \beta)}}\Big) \\
  & \leq \sum_{\rho \in \{\delta_0,1\}}e^{-\frac{1+\beta+\rho}{1+\beta}(\alpha-\epsilon)t}Rh_{2}.
\end{align}
Thus for all $t\geq 0$, $\epsilon> 0$ and $f\in \mathcal P_g$,
\begin{align}
  \label{eq: estimate of J1}
  J^f_1(t,\epsilon)
  & \leq \sum_{\rho \in \{\delta_0,1\}}e^{-\frac{1+\beta+\rho}{1+\beta}(\alpha-\epsilon)t}\mathbb{P}_{\mu}[\langle Rh_2,X_{t-1}\rangle]\\
  & \leq \sum_{\rho \in \{\delta_0,1\}} \langle Q_0 R h_{2}, \mu \rangle e^{-(\alpha\frac{\rho}{1+\beta}-\epsilon\frac{1+\beta+\rho}{1+\beta})t},
\end{align}
where $Q_0$ is defined by \eqref{eq:Q}.
By taking $\epsilon>0$ small enough, we get the desired result in this step.

\emph{Step 3.}
We will show that for $\epsilon>0$ small enough there exist $C_3, \delta_3 > 0$ such that for each $t \geq 0$ and $f\in \mathcal P_g$,  we have $ J^f_2(t,\epsilon) \leq C_3 e^{-\delta_3 t}$.
In fact, for all $t\geq 0$, and $f\in \mathcal P_g$,
\begin{align}
  & \langle Z_1(\theta_t f),X_{t}\rangle- \langle Z_1f, \varphi\rangle
     = \theta_t^{1+\beta} \langle Z_1 f,X_t\rangle - \langle  Z_1 f,\varphi \rangle
     =\Big\langle Z_1f - \langle  Z_1 f ,\varphi \rangle, \frac{X_{t}}{\|X_{t}\|}\Big\rangle,
\end{align}
and therefore,
\begin{align}
  \label{eq: prevJ2}
  J^f_2(t,\epsilon)
  & = \mathbb P_\mu\Big[\Big| \Big\langle Z_1f - \langle  Z_1 f ,\varphi \rangle, \frac{X_t}{\|X_t\|}\Big\rangle \Big|;A_t(\epsilon)\Big]
    \leq e^{-(\alpha-\epsilon)t} \mathbb{P}_{\mu}[| \langle q_f,X_t \rangle |].
\end{align}
where $ q_f = Z_1 f-\langle  Z_1 f,\varphi\rangle \in \mathcal P^*$.
It follows from Lemma \ref{lem:P:R} that there exists $h_{3}\in \mathcal{P}$ such that for each $ f\in \mathcal P_g$, we have $Q_1 (\operatorname{Re} q_f) \leq h_{3} \text{ and } Q_1 (\operatorname{Im} q_f)\leq h_3$, where $Q_1$ is given by \eqref{eq:Q} with $\kappa=1$.
In the rest of this step, we  fix a $\gamma\in(0,\beta)$ small enough such that $\alpha \gamma < b < (1+\gamma)b$.
According to Lemma \ref{lem:P:M:uc}.(3) (with $\kappa=1$), there exists $C_{3}>0$ such that for all $t\geq 0$ and $f\in \mathcal P_g$,
\begin{align}
  & \mathbb{P}_{\mu}\left[\left|\langle q_f,X_{t}\rangle\right|\right]
    \leq \|\langle \operatorname{Re} q_f, X_{t}\rangle\|_{\mathbb{P}_{\mu,1+\gamma}} + \|\langle \operatorname{Im} q_f, X_{t}\rangle\|_{\mathbb{P}_{\mu,1+\gamma}} \\
 & \leq 2\sup_{q\in \mathcal P: Q_1 q\leq h_{3}} \|\langle q, X_t\rangle\|_{\mathbb P_\mu; 1+\gamma} \leq C_{3} e^{\frac{\alpha t}{1+\gamma}}.
\end{align}
Therefore, for all $t\geq 0, \epsilon > 0$ and $f \in \mathcal P_g$, we have
\begin{align}
  \label{eq: right bound for J2}
   J^f_2(t, \epsilon)
    \leq  C_3 e^{-(\alpha-\epsilon)t}e^{\frac{\alpha t}{1+\gamma}}
   \leq C_{3} e^{-(\alpha\tilde \gamma -\epsilon)t}.
\end{align}
By taking $\epsilon >0$ small enough, we get the required result in this step.

\emph{Step 4.}
We will show that, for each $\epsilon\in (0,  \alpha)$, there exist $C_4,\delta_4>0$ such that for all $t\geq 1$, $J_3(t,\epsilon)\leq C_4e^{-\delta_4 t}.$
In fact, we have  for all $t\geq 0, \epsilon >0$,
\[
  \mathbb P_{\mu}(A_{t}(\epsilon), D)
  = \mathbb P_{\mu}[\mathbb P_{\mu}(D|\mathscr F_t);A_t(\epsilon)]
  = \mathbb P_\mu[e^{-\bar v\|X_t\|};A_t(\epsilon)]
  \leq \exp({-\bar v \|\mu\|e^{(\alpha - \epsilon)t}}).
\]
On the other hand, by Proposition \ref{lem: control of XT}, for each $\epsilon \in (0, \alpha)$, there exists  $C_{4}, \delta_{4}>0$ such that for all $t\geq 0$,
\begin{align}
  \mathbb P_\mu(A_t(\epsilon)^c,D^c)
  \leq \mathbb P_\mu(0 < e^{-\alpha t}\|X_t\|
  \leq e^{ - \epsilon t}) \leq C_{4} (e^{-\epsilon \delta_{4} t}+e^{-\delta_{4} t}).
\end{align}
 Combining these results, we get the desired result in this step.

 \emph{Final step}. Combining the results in Steps 1--4, we immediately get the desired result.
\end{proof}


% ***** LCLT for Multiple interval
The following corollary of Proposition \ref{thm:Key} will be used later in the proof of our main result Theorem \ref{thm:M}.
\begin{cor}
  \label{cor:MI}
  Suppose that $g\in \mathcal{P}\setminus\{0\}$ and $\mu\in \mathcal M_c(\mathbb R^d)$.
  Then there exist $C,\delta>0$ such that for each $l\leq n$ in $\mathbb Z_+$ and each $(f_j)_{j=l}^n\subset \mathcal P_g$,
\begin{align}
  \label{32corollary}
  \Big|\mathbb{\widetilde{P}}_{\mu}\Big[\prod_{k=l}^ne^{i \Upsilon^{f_k}_{k} }-\prod_{k=l}^n e^{\langle Z_1f_k, \varphi\rangle}\Big]\Big|\leq C e^{-\delta l}.
\end{align}
\end{cor}
\begin{proof}
  For each $l\leq n$ in $\mathbb Z_+$, $k \in \{l,\dots,n\}$ and each $(f_j)_{j=l}^n\subset \mathcal P_g$, define
  \[
    a_k
    :=  \mathbb{\widetilde{P}}_{\mu}\Big[\prod_{j=l}^{k} e^{i\Upsilon_j^{f_j}}\Big] \times \Big(\prod_{j=k+1}^{n} e^{ \langle Z_1f_j,\varphi \rangle} \Big).
  \]
  Then for each $l\leq n$ in $\mathbb Z_+$, $k \in \{l,\dots,n\}$ and each $(f_j)_{j=l}^n\subset \mathcal P_g$, we have
  \begin{align}
    & a_{k} - a_{k-1}
      =\mathbb{P}_{\mu}(D^c)^{-1} \mathbb{P}_{\mu}\Big[(e^{i\Upsilon^{f_k}_k}-e^{\langle Z_1f_k, \varphi\rangle})\prod_{j=l}^{k-1} e^{i\Upsilon_j^{f_j}};D^c\Big] \Big(\prod_{j=k+1}^n e^{\langle Z_1f_j, \varphi\rangle}\Big)\\
    & =\mathbb{P}_{\mu}(D^c)^{-1} \mathbb{P}_{\mu}\Big[\mathbb P_\mu[e^{i\Upsilon_k^{f_k}}-e^{\langle Z_1f_k, \varphi \rangle}; D^c|\mathscr F_k] \prod_{j=l}^{k-1} e^{i\Upsilon_j^{f_j}}\Big] \Big(\prod_{j=k+1}^{n}e^{\langle Z_1f_j, \varphi\rangle}\Big).
  \end{align}
  By Lemma \ref{thm:Key}, there exist $C_0,\delta_0 >0$ such that for each $l\leq n$ in $\mathbb Z_+$,  $k \in \{l,\dots , n\}$, and $(f_j)_{j=l}^n\subset \mathcal P_g$, we have
  \begin{align}
    | a_{k} - a_{k-1}|
    & \leq \mathbb{P}_{\mu}(D^c)^{-1}\mathbb{P}_{\mu}\Big[\big|\mathbb P_\mu[e^{i\Upsilon_k^{f_k}}-e^{\langle Z_1f_k, \varphi \rangle}; D^c | \mathscr{F}_k]\big|\Big]
    \leq C_0 e^{-\delta_0 k}.
  \end{align}
  Therefore, there exist $C,\delta >0$ such that for each $l\leq n$ in $\mathbb Z_+$ and each $(f_j)_{j=l}^n\subset \mathcal P_g$, we have

  \begin{align}
    \text{LHS of \eqref{32corollary}}
    & = \left|a_n-a_{l-1}\right|
      \leq \sum_{k=l}^n\left|a_{k}-a_{k-1}\right|
      \leq \sum_{k=l}^n C_0 e^{-\delta_0 k}
      \leq C e^{- \delta l}.
      \qedhere
  \end{align}
\end{proof}

% *** The small branching rate regime
\subsection{Central limit theorem for $f\in \mathcal C_s$}
\label{sec: small rate}

\begin{proof}[Proof of Theorem \ref{thm:M}.(\ref{thm:M:1})]
	Fix $\mu\in \mathcal M_c(\mathbb R^d)$, $f\in \mathcal C_s$ and $t_0 > 1$  large enough so that $ \lceil t - \ln t\rceil \leq \lfloor t \rfloor - 1$ for each $t\geq t_0$.
  For each $t\geq t_0$, in this proof we write $\theta_t = \|X_t\|^{\tilde \beta - 1}$,
  \begin{multline}
    \label{eq:PM:CLTS:1}
    \theta_t  X_t(f)
    = I^f_1(t) + I^f_2(t) + I^f_3(t)
    := \Big(\sum_{k=0}^{\lfloor t-\ln t \rfloor} \theta_t \mathcal I_{t-k-1}^{t-k} X_t(f) \Big)\\
    + \Big( \theta_t \mathcal I_0^{t-\lfloor t \rfloor} X_t(f)   + \sum_{k=\lceil t-\ln t \rceil}^{\lfloor t \rfloor-1} \theta_t \mathcal I_{t-k-1}^{t-k} X_t(f) \Big) + (\theta_t X_0(P_t^\alpha f) ),
  \end{multline}
  and $ I^f_0(t) := \sum_{k=0}^{\lfloor t-\ln t \rfloor} \Upsilon_{t-k-1}^{T_k \tilde f},$ where $\tilde f:= e^{\alpha(\tilde \beta - 1)} f$.

% ****** Step 1.
  \emph{Step 1.} We show that $I^f_0(t) \xrightarrow[t\to \infty]{d} \zeta^f$.
  In fact, for each $k \in \mathbb Z_+$, we have  $T_{k} \tilde f \in \mathcal P_{\tilde f}:=\{\theta T_n \tilde f: n \in \mathbb Z_+, \theta \in [-1,1]\}$.
  Therefore from Corollary \ref{cor:MI} we get that there exist $C_1,\delta_1 > 0$ such that
  \begin{align}
    \Big|\mathbb{\widetilde{P}}_{\mu} [e^{i I^f_0(t)} ]-\exp\Big(\sum_{k=0}^{\lfloor t-\ln t \rfloor} \langle Z_1T_{k}\tilde f, \varphi\rangle \Big)\Big|
    \leq C_1 e^{-\delta_1(t - \lfloor t - \ln t\rfloor)},
    \quad t\geq t_0.
  \end{align}
  On the other hand, using \eqref{eq:PL:S:1} and the fact that $\varphi(x)dx$ is the invariant probability of the semigroup $(P_t)_{t\geq 0}$, we have
  \begin{align}
    \label{eq:PM:CLTS:2}
    & \sum_{k=0}^\infty \langle Z_1 T_{k} \tilde f, \varphi \rangle
    = \sum_{k=0}^\infty \int_0^1 \langle P_u^\alpha ((-iP_{1 - u}^\alpha T_k \tilde f)^{1+\beta}), \varphi\rangle ~du
    \\& = \sum_{k=0}^\infty \int_0^1 e^{\alpha u} \langle  (-iP_{1 - u}^\alpha T_{k}\tilde f)^{1+\beta}, \varphi \rangle ~du
    \\& = \sum_{k=0}^\infty \int_0^1 \langle  (-iT_{k+1 - u} f)^{1+\beta}, \varphi\rangle~du
    = \int_0^\infty \langle  (-iT_{u} f)^{1+\beta}, \varphi\rangle~du = m[f].
  \end{align}
  Therefore, we have $\mathbb{\widetilde{P}}_{\mu} [e^{i I^f_0(t)} ] \xrightarrow[t\to \infty]{} e^{m[f]}$. Since $I_0^f(t)$ is linear in $f$, we can replace $f$ with $\theta f$, $\theta \in \mathbb R$, and then the desired result in this step follows.

% ****** Step 2.
  \emph{Step 2.} We show that $I^f_1(t) - I^f_0(t) \xrightarrow[t\to \infty]{d} 0$.
  In fact,  by \cite[Lemma 3.4.3]{Durrett2010Probability} we have that for each $t\geq t_0$,
  \begin{align}
    \label{eq:PM:S:1}
    |\mathbb{\widetilde{P}}_{\mu}[e^{i (I^f_1(t) - I^f_0(t) ) }] - 1|
    \leq \sum_{k=0}^{\lfloor t-\ln t \rfloor}\mathbb{\widetilde{P}}_{\mu}\big[|Y_{t,k}|\big],
  \end{align}
  where $ Y_{t,k} := \exp(i \Upsilon_{t-k-1}^{T_{k} \tilde f} - i\theta_t \mathcal I_{t-k-1}^{t-k} X_t(f)) - 1. $
  We claim that there exists $C_2, \delta_2>0$ such that \(\widetilde {\mathbb P}_\mu [|Y_{t,k}|] \leq C_2 e^{-\delta_2 (t-k-1)}\) for each $k\in \mathbb Z_+$ and $t\geq k+1$.
  And therefore there exists $C_2'>0$ such that for each $t \geq t_0$, $|\mathbb{\widetilde{P}}_{\mu}[e^{i (I^f_1(t)- I^f_0(t))}]-1| \leq C_2't^{-\delta_1}$  which, combining with the fact that $I^f_1(t) - I^f_0(t)$ is linear in $f$, completes this step.

  We will show this claim in the following substeps 2.1 and 2.2.
  First we chose $\gamma \in (0,\beta)$ close enough to $\beta$ such that there exists $\eta,\eta'>0$ with $ \alpha \tilde \gamma > \eta > \eta - 3\eta' > \alpha \tilde \beta - \alpha \tilde \gamma > 0;$ and define for each $k \in \mathbb Z_+$ and $t\geq k+1$,
  \[
    \mathcal{D}_{t,k}
    :=\{|H_t-H_{t-k-1}|\leq  e^{-\eta (t-k-1)}, H_{t-k-1}> 2e^{-\eta' (t-k-1)}\},
  \]
  where $H_t := e^{-\alpha t}\|X_t\|$.

% ******* Step 2.1
  \emph{Substep 2.1.} We show that there exist $C_{2.1},\delta_{2.1} >0$ such that for all $k \in \mathbb Z_+$ and $t\geq k+1$, $ \mathbb{\widetilde{P}}_{\mu} \big[ |Y_{t,k}| ;\mathcal{D}^c_{t,k} \big] \leq C_{2.1} e^{-\delta_{2.1} (t-k)}.$
  In fact, it follows from Proposition \ref{lem: control of XT}, Lemma \ref{lem: control of wt} with $|p|=0$ and Chebyshev's inequality that there exist $C_{2.1}', \delta_{2.1}'>0$ such that for all $k \geq 0$ and $t\geq k+1$,
  \begin{align}
    \label{eq: prob of Dtkc}
    & \mathbb{\widetilde{P}}_{\mu}(\mathcal{D}_{t,k}^c)
    \leq \mathbb{\widetilde{P}}_{\mu}(|H_t-H_{t-k-1}| > e^{-\eta (t-k-1)})+\mathbb{\widetilde{P}}_{\mu}(H_{t-k-1}\leq 2e^{-\eta'(t-k-1)}) \\
    & \leq \mathbb{P}_{\mu}(D^c)^{-1}e^{\eta(t-k-1)}\mathbb{P}_{\mu}[|H_t-H_{t-k-1}|] +  \mathbb{P}_{\mu}(D^c)^{-1} \mathbb P_\mu(H_{t-k-1}\leq 2e^{-\eta'(t-k-1)}; D^c) \\
    & \leq \mathbb{P}_{\mu}(D^c)^{-1}  e^{\eta(t-k-1)}\|H_t - H_{t-k-1}\|_{\mathbb P_\mu; 1+\gamma} + \mathbb{P}_{\mu}(D^c)^{-1} \mathbb P_\mu(0<H_{t-k-1}\leq 2e^{-\eta'(t-k-1)}) \\
    & \leq C'_{2.1} e^{-(\alpha \tilde \gamma - \eta)(t-k-1)}+C'_{2.1} e^{-\delta'_{2.1}(t-k-1)}.
  \end{align}
  This implies the desired result in this substep, since $|Y_{t,k}| \leq 2$ a.s..

% ******* Step 2.2
  \emph{Substep 2.2.} We will show that there exist $C_{2.2},\delta_{2.2} > 0$ such that for each $k\in \mathbb Z_+$ and $t\geq k+1$, it holds that $ \mathbb{\widetilde{P}}_{\mu} [|Y_{t,k}|;\mathcal{D}_{t,k}] \leq  C_{2.2} e^{-\delta_{2.2} (t-k)}$.
  In fact, noticing that for $f\in \mathcal C_s$ and $k\in \mathbb Z_+$, we have $T_kf = e^{\alpha (\tilde \beta - 1 )k}P_k^\alpha f $; and therefore for each $k\in \mathbb Z_+$ and $t \geq k + 1$,
  \begin{align}
    \label{eq:gammafunction11}
    \Upsilon_{t-k-1}^{T_{k} \tilde f}
    = \frac{X_{t-k}(T_{k} \tilde  f) - X_{t -k-1}(P_1^\alpha T_{k} \tilde f)}{\|X_{t-k-1}\|^{1-\tilde \beta}}
    = \frac{\mathcal I_{t - k - 1}^{t - k} X_t(f)}{\|e^{\alpha (k+1)}X_{t-k-1} \|^{1 -\tilde \beta}}.
  \end{align}
  Since $|e^{ix}-e^{iy}|\leq|x-y|$ for all $x,y\in \mathbb R$, we have for all $k \in \mathbb Z_+$ and $t\geq k+1$,
  \begin{align}
    \label{eq: control of Ykt}
    & \mathbb{\widetilde{P}}_{\mu}[|Y_{t,k}|;\mathcal{D}_{t,k}] \\
    & \leq \mathbb{\widetilde{P}}_{\mu}\Big[|\mathcal I_{t-k-1}^{t-k}\langle f , X_t\rangle | \cdot \Big| \| e^{\alpha(k+1)}X_{t-k-1}\| ^{ \tilde \beta - 1} - \|X_t\|^{ \tilde \beta - 1}\Big|; \mathcal D_{t,k}\Big] \\
    & \leq  e^{\alpha(\tilde \beta - 1)t}\mathbb{\widetilde{P}}_{\mu}\big[|\mathcal I_{t-k-1}^{t-k}X_t(f)|\cdot K_{t,k}\big],
  \end{align}
  where
  \[
    K_{t,k}
    := \Big| \frac {H_t^{1- \tilde \beta} - H_{t-k-1}^{1 - \tilde \beta}} {H_t^{1 - \tilde \beta} H_{t-k-1}^{ 1- \tilde \beta }} \Big| \mathbf{1}_{\mathcal{D}_{t,k}}.
  \]
  Note that, since $\eta' < \eta$, we have almost surely on $\mathcal D_{t,k}$,
  \begin{align}
    H_t
    & \geq H_{t-k-1}- e^{-\eta (t-k-1)}
      \geq 2e^{-\eta'(t-k-1)}-e^{-\eta(t-k-1)}
      \geq e^{-\eta'(t-k-1)}.
  \end{align}
  Therefore, for each $k \in \mathbb Z_+$ and $t\geq k+1$, almost surely  on $\mathcal D_{t,k}$,
  \begin{align}
    & \Big|H_t^{1- \tilde \beta}-H_{t-k-1}^{1- \tilde \beta}\Big|
      \leq (1- \tilde \beta) \max \{ H_t^{-\tilde \beta }, H_{t-k-1}^{ -\tilde \beta} \} | H_t - H_{t-k-1} | \\
    & \leq (1- \tilde \beta ) \max\{e^{\eta' (t-k-1)}, \frac{1}{2}e^{\eta'(t-k-1)}\}^{\tilde \beta} e^{-\eta(t-k-1)}  \leq (1- \tilde \beta) e^{-(\eta - \eta') (t-k-1)}
  \end{align}
  and $ |H_t^{1 - \tilde \beta} H_{t-k-1}^{ 1 - \tilde \beta}| \geq 2^{\frac{1}{1+\beta}} e^{-2\eta'(t-k-1)}$.
  Thus, there exists $C_{2.2}'> 0$ such that for all $k \geq 0, t\geq k+1$, almost surely
  \begin{align}
    K_{t,k}
    \leq C_{2.2}' e^{-(\eta - 3\eta')(t-k-1)}.
  \end{align}
  Now, by Lemma \ref{lem: control of mgtrs}, there exist $C''_{2.2}>0$ such that for all $k\geq 0$ and $t\geq k+1$,
  \begin{align}
    \label{eq: Y in D}
    & \mathbb{\widetilde{P}}_{\mu} [|Y_{t,k}| ; \mathcal{D}_{t,k} ]
    \leq C_{2.2}' e^{\alpha (\tilde \beta - 1)t} \mathbb{\widetilde{P}}_{\mu} [ | \mathcal{I}_{t-k-1}^{t-k}X_t(f)| ] e^{-(\eta - 3\eta')(t-k-1)} \\
    & \leq \frac{C_{2.2}' } {\mathbb{P}_{\mu}(D^c)} e^{ \alpha (\tilde \beta - 1)t} \|\mathcal{I}_{t-k-1}^{t-k} X_t(f)\|_{\mathbb P_\mu; 1+\gamma} e^{-(\eta - 3\eta')(t-k - 1)} \\
    & \leq C_{2.2}'' e^{\alpha(\tilde \beta - \tilde \gamma)t} e^{ (\alpha \tilde \gamma - \kappa_f b)k} e^{-(\eta - 3\eta')(t-k)}
     \leq C_{2.2}'' e^{\alpha(\tilde \beta - \tilde \gamma)(t-k)} e^{-(\eta - 3\eta')(t-k)},
  \end{align}
  as desired in this step.
  In the last inequality, we used the fact that $f\in \mathcal C_s$ and therefore $\alpha \tilde \beta < \kappa_f b$.

% ****** Step 3
  \emph{Step 3.}
  We show that $I^f_2(t)\xrightarrow[t\to \infty]{d} 0$.
  First fix a $\gamma \in (0,\beta)$ in this step.
  From the fact that $\kappa_f b -\alpha \tilde \gamma > \alpha (\tilde \beta - \tilde \gamma)$, we can chose $\epsilon >0$ small enough such that $q:=\kappa_fb- \alpha \tilde \gamma  > \alpha (\tilde \beta - \tilde \gamma) + 2\epsilon (1 - \tilde \beta)$.
  Now writing $\mathcal{E}_t:=\{\|X_t\|>e^{(\alpha-\epsilon) t}\}$, according to Proposition \ref{lem: control of XT}, there exist $C_3, \delta_3>0$ such that
  \begin{align}
    \mathbb{\widetilde{P}}_{\mu}(\mathcal{E}^c_t)
    \leq \frac{1}{\mathbb{P}_{\mu}(D^c)}\mathbb{P}_{\mu}(0<e^{-\alpha t}\|X_t\|\leq e^{-\epsilon t})\leq C_3e^{-\delta_3 t}
    , \quad t\geq0.
  \end{align}
  Therefore,
  \begin{align}
    \label{Theorem123}
    |\mathbb{\widetilde{P}}_{\mu}[e^{i I^f_2(t)}-1;\mathcal{E}^c_t]|
    \leq 2\mathbb{\widetilde{P}}_{\mu}(\mathcal{E}^c_t)
    \leq 2C_3e^{-\delta_3 t},
    \quad t\geq t_0.
  \end{align}
	According to Lemma \ref{lem: control of mgtrs}, there exist $C_3',C_3'',C_3'''>0$ such that for each $t\geq t_0 >1$,
  \begin{align}
    & |\mathbb{\widetilde{P}}_{\mu} [ (e^{i I^f_2(t)}-1);\mathcal{E}_t]|
      \leq  \mathbb{\widetilde{P}}_{\mu} [ |I^f_2(t)|;\mathcal{E}_t] \\
    & \leq  ( e^{(\alpha-\epsilon) t} )^{\tilde \beta - 1}\Big(\sum_{k=\lceil t-\ln t \rceil}^{\lfloor t \rfloor - 1}\mathbb{\widetilde{P}}_{\mu} [| \mathcal{I}_{t-k-1}^{t-k} X_t(f) |] + \mathbb{\widetilde{P}}_{\mu}[| \mathcal{I}_{0}^{t-\lfloor t\rfloor} X_t(f)|]\Big) \\
    & \leq ( e^{(\alpha-\epsilon) t} )^{\tilde \beta - 1}\Big(\sum_{k=\lceil t-\ln t \rceil}^{\lfloor t \rfloor - 1}\|\mathcal{I}_{t-k-1}^{t-k} X_t(f) \|_{\mathbb P_\mu; 1+\gamma} + \|\mathcal I_0^{t-\lfloor t \rfloor} X_t(f)\|_{\mathbb P_\mu;1+\gamma}\Big) \\
    & \leq C_3' e^{\alpha (\tilde \beta - \tilde \gamma)t} e ^{\epsilon (1-\tilde \beta) t}\sum_{k=\lceil t-\ln t \rceil}^{\lfloor t \rfloor}  e^{(\alpha\tilde \gamma-\kappa_f b)k}
      \leq C_3' e^{q t}e^{-\epsilon ( 1 - \tilde \beta)t}\sum_{k=\lceil t-\ln t \rceil}^{\lfloor t \rfloor}  e^{-q k}
    \\ & \leq C_3'' e^{q(t - \lceil t - \ln t\rceil)}e^{-\epsilon(1 - \tilde \beta) t}
         \leq C_3'' t^q e^{- \epsilon(1 - \tilde \beta) t}.
  \end{align}
  From this and \eqref{Theorem123}, we get that $\widetilde {\mathbb P}_\mu[e^{i I^f_2(t)}] \xrightarrow[t\to \infty]{} 1$.
  Note that $I^f_2(t)$ is linear in $f$ so we can replace $f$ with $\theta f$ for $\theta \in \mathbb R$ and get the desired result in this step.

% ****** Step 4
	\emph{Step 4.} We will show that $I^f_3(t) \xrightarrow[t\to \infty]{\widetilde {\mathbb P}_\mu \text{-} a.s.} 0$.
  In fact, we have
  \begin{align}
    & |I^f_3(t)|
      \leq \frac{\langle |P^\alpha_tf|,X_0\rangle}{\|X_t\|^{1 - \tilde \beta }}
      \leq \frac{\langle e^{\alpha t - \kappa_f b t}Qf,X_0\rangle}{(e^{\alpha t} H_t)^{1 - \tilde \beta}}
      = e^{(\alpha \tilde \beta - k_fb)t} H_t^{\tilde \beta - 1} \langle Qf,X_0\rangle
      \xrightarrow[t\to \infty]{\widetilde {\mathbb P}_\mu \text{-} a.s.} 0.
  \end{align}

% ****** Final Step
  \emph{Final step.} Combining Steps 1--4, we complete the proof of \ref{thm:M}.\eqref{thm:M:1}.
\end{proof}

% *** Critical branching rate regime
\subsection{Central limit theorem for $f \in \mathcal C_c$}
% ***** Proof of thm:M:2
\begin{proof}[Proof of Theorem \ref{thm:M}.(\ref{thm:M:2})]
  Fix $\mu\in \mathcal M_c(\mathbb R^d)$, $f\in \mathcal C_c$ and $t_0 > 1$ large enough so that $ \lceil t - \ln t\rceil \leq \lfloor t \rfloor - 1$ for each $t\geq t_0$.
  For each $t\geq t_0$, in this proof we write $\theta_t = \|t X_t\|^{\tilde \beta - 1}$,  define $I_i^f(t)$ using \eqref{eq:PM:CLTS:1}  for $i = 1,2,3$, and set $ I^f_0(t) := t^{\tilde \beta - 1}\sum_{k=0}^{\lfloor t-\ln t \rfloor} \Upsilon_{t-k-1}^{T_{k} \tilde f}$, where $\tilde f = e^{\alpha(\tilde \beta - 1)} f$.

% ****** Step 1.
  \emph{Step 1.} We show that $I^f_0(t) \xrightarrow[t\to \infty]{d} \zeta^f$.
  In fact, for each $t \geq  t_0 > 1$ we have $t^{\tilde \beta - 1} < 1$; and therefore, for each $k \in \mathbb Z_+$, we have $t^{\tilde \beta - 1} T_{k+1} f \in \mathcal P_f:=\{\theta T_n f: n \in \mathbb Z_+, \theta \in [-1,1]\}$.
  Therefore from Proposition \ref{cor:MI} and that $\tilde \beta - 1 = -\frac{1}{1+\beta}$ we get that there exist $C_1,\delta_1 > 0$ such that
  \begin{align}
    \Big|\mathbb{\widetilde{P}}_{\mu} [e^{i I^f_0(t)} ]-\exp\Big(\frac{1}{t}\sum_{k=0}^{\lfloor t-\ln t \rfloor} \langle Z_1T_{k}\tilde f, \varphi\rangle \Big)\Big|
    \leq C_1 e^{-\delta_1(t - \lfloor t - \ln t\rfloor)}
    \leq \frac{C_1}{t^{\delta_1}},
    \quad t\geq t_0.
  \end{align}
  Since $f \in \mathcal C_c\setminus \{0\}$, we have $T_k \tilde f = \tilde f$ for each $k \in \mathbb Z_+$.
  Similar to the argument in \eqref{eq:PM:CLTS:2} we have
  \begin{align}
    \label{CLT:C:eq:m}
    \lim_{t\to \infty} \frac{1}{t}\sum_{k=0}^{\lfloor t-\ln t \rfloor} \langle Z_1 T_{k}\tilde f, \varphi\rangle
    = \langle Z_1 \tilde f,\varphi \rangle
    = \langle (-if)^{1+\beta}, \varphi \rangle
    = m[f].
  \end{align}
  Therefore $\mathbb {\widetilde P}_\mu[e^{i I^f_0(t)}] \xrightarrow[t\to \infty]{} e^{m[f]}$.
  The desired result in this step follows.

% ****** Step 2.
  \emph{Step 2.} We show that $ I^f_1(t) - I^f_0 (t) \xrightarrow[t\to \infty]{d} 0$.
  In fact, similar to Step 2 of the proof of Theorem \ref{thm:M}.(\ref{thm:M:1}), we  have \eqref{eq:PM:S:1} is valid with $ Y_{t,k} := \exp(i t^{\tilde \beta - 1} \Upsilon_{t-k-1}^{T_{k}\tilde f} - i\theta_t \mathcal I_{t-k-1}^{t-k} X_t(f)) - 1$.
  Similarly, we claim that there exists $C_2, \delta_2>0$ such that $\widetilde {\mathbb P}_\mu [|Y_{t,k}|] \leq C_2 e^{-\delta_2 (t-k-1)}$ for each $k\in \mathbb N$ and $t\geq k+1$, and  then the desired result in this step follows.

  We will show this claim in the following substeps 2.1 and 2.2.
  First we chose $\gamma \in (0,\beta)$ close enough to $\beta$ such that there exists $\eta,\eta'>0$ with $ \alpha \tilde \gamma > \eta > \eta - 3\eta' > \alpha \tilde \beta - \alpha \tilde \gamma > 0$; and define for each $k \in \mathbb N$ and $t\geq k+1$ that $ \mathcal{D}_{t,k} := \{|H_t-H_{t-k-1}| \leq  e^{-\eta (t-k-1)}, H_{t-k-1}> 2e^{-\eta' (t-k-1)}\}$.

% ******* subStep 2.1
  \emph{Substep 2.1.}
  Similar to Substep 2.1 in the proof of Theorem \ref{thm:M}.(\ref{thm:M:1}), there exist $C_{2.1},\delta_{2.1} >0$ such that for all $k \in \mathbb N$ and $t\geq k+1$, $\mathbb{\widetilde{P}}_{\mu}[|Y_{t,k}|;\mathcal{D}^c_{t,k}] \leq C_{2.1} e^{-\delta_{2.1} (t-k)}$.
  We omit the details.

% ******* Step 2.2
  \emph{Substep 2.2.} We will show that there exist $C_{2.2},\delta_{2.2} > 0$ such that for each $k\in \mathbb N$ and $t\geq k+1$, $ \mathbb{\widetilde{P}}_{\mu}[|Y_{t,k}|;\mathcal{D}_{t,k}] \leq  C_{2.2} e^{-\delta_{2.2} (t-k)}.$
  In fact, noticing that for $f\in \mathcal C_c$ and $k\in \mathbb Z_+$, we have $T_kf = e^{\alpha (\tilde \beta - 1 )k}P_k^\alpha $; and therefore for each $k\in \mathbb Z_+$ and $t \geq k + 1$,
  \[
    t^{\tilde \beta - 1} \Upsilon_{t-k-1}^{T_{k} \tilde f}
    = \frac{X_{t-k}(T_{k} \tilde f) - X_{t -k-1}(P_1^\alpha T_{k} \tilde f)}{\|t X_{t-k-1}\|^{1-\tilde \beta}}
    = \frac{\mathcal I_{t - k - 1}^{t - k} X_t(f)}{\|te^{\alpha (k+1)}X_{t-k-1} \|^{1 -\tilde \beta}}.
  \]
  The rest is similar to Substep 2.2 in the proof of Theorem \ref{thm:M}.(\ref{thm:M:2}).
  We omit the details.

% ****** Step 3
  \emph{Step 3.}
  We show that $ I^f_2(t)\xrightarrow[t\to \infty]{d} 0$.
  In fact, writing $\mathcal{E}_t:=\{\|X_t\|>t^{-1/2}e^{\alpha t}\}$, according to Proposition \ref{lem: control of XT}, there exist $C_3, \delta_3>0$ such that
  \[
    \mathbb{\widetilde{P}}_{\mu}(\mathcal{E}^c_t)
    \leq \frac{1}{\mathbb{P}_{\mu}(D^c)}\mathbb{P}_{\mu}(0<e^{-\alpha t}\|X_t\|\leq t^{-1/2})\leq C_3( t^{-\delta_3}+e^{-\delta_3 t})
    , \quad t\geq0.
  \]
  Therefore,
  \begin{align}
    \label{Theorem123}
    |\mathbb{\widetilde{P}}_{\mu}[e^{i I^f_2(t)}-1;\mathcal{E}^c_t]|
    \leq 2\mathbb{\widetilde{P}}_{\mu}(\mathcal{E}^c_t)
    \leq C_3(t^{-\delta_3}+e^{-\delta_3 t}),
    \quad t\geq t_0.
  \end{align}
  Choose a $\gamma\in (0,\beta)$ close enough to $\beta$ such that $\alpha(\tilde \beta - \tilde \gamma) \leq \frac{1}{2}(1- \tilde \beta)$.
	According to Lemma \ref{lem: control of mgtrs}, there exist $C_3',C_3'',C_3'''>0$ such that for each $t\geq t_0 >1$,
  \begin{align}
    & |\mathbb{\widetilde{P}}_{\mu} [ (e^{i I^f_2(t)}-1)\mathbf{1}_{\mathcal{E}_t}]|
      \leq  \mathbb{\widetilde{P}}_{\mu} [ |I^f_2(t)|\mathbf{1}_{\mathcal{E}_t}] \\
    & \leq  (t^{\frac{1}{2}} e^{\alpha t} )^{\tilde \beta - 1}\Big(\sum_{k=\lceil t-\ln t \rceil}^{\lfloor t \rfloor - 1}\mathbb{\widetilde{P}}_{\mu} [| \mathcal{I}_{t-k-1}^{t-k} X_t(f) |] + \mathbb{\widetilde{P}}_{\mu}[| \mathcal{I}_{0}^{t-\lfloor t\rfloor} X_t(f)|]\Big) \\
    & \leq C_3' t^{\frac{1}{2}(\tilde \beta - 1)} e^{\alpha(\tilde \beta - 1)t}\Big(\sum_{k=\lceil t-\ln t \rceil}^{\lfloor t \rfloor - 1}\|\mathcal{I}_{t-k-1}^{t-k} X_t(f) \|_{\mathbb P_\mu; 1+\gamma} + \|\mathcal I_0^{t-\lfloor t \rfloor} X_t(f)\|_{\mathbb P_\mu;1+\gamma}\Big) \\
    & \leq C_3' t^{\frac{1}{2}(\tilde \beta - 1)} e^{\alpha (\tilde \beta - \tilde \gamma)t}\sum_{k=\lceil t-\ln t \rceil}^{\lfloor t \rfloor}  e^{(\alpha\tilde \gamma-\kappa_f b)k}
      = C_3' t^{\frac{1}{2}(\tilde \beta - 1)} e^{\alpha(\tilde \beta - \tilde \gamma) t}\sum_{k=\lceil t-\ln t \rceil}^{\lfloor t \rfloor}  e^{-\alpha (\tilde \beta -\tilde \gamma) k}
    \\ & \leq C_3'' t^{\frac{1}{2}(\tilde \beta - 1)} e^{\alpha (\tilde \beta - \tilde \gamma)(t - \lceil t - \ln t\rceil)}
         \leq C_3'' t^{\frac{1}{2}(\tilde \beta - 1)} t^{\alpha (\tilde \beta - \tilde \gamma)}.
  \end{align}
  From this and \eqref{Theorem123}, we get the desired result in this step.

% ****** Step 4
  \emph{Step 4.} Similar to Step 4 of the Proof of Theorem \ref{thm:M}.(\ref{thm:M:1}), we can verify that $I_3(t) \xrightarrow[t\to \infty]{\widetilde {\mathbb P}_\mu \text{-} a.s.} 0$.
  We omit the details.

% ****** Final step
 	\emph{Final step.} Combining Steps 1--4, we complete the proof of \ref{thm:M}.\eqref{thm:M:2}.
\end{proof}

% *** The large branching rate regime: CLT
\subsection{Central limit theorem for $f\in \mathcal C_l$}
\label{sec: large rate clt}
% ***** Proof or the main result
\begin{proof}[Proof of Theorem \ref{thm:M}.(\ref{thm:M:3})]
  Fix $\mu \in \mathcal M_c(\mathbb R^d)$ and $f \in \mathcal C_l$.
  Define $\mathcal N:= \{p\in \mathbb Z_+^d: \alpha \tilde \beta > |p|b\}$.
  In this proof we write for each $t\geq 0$ that
  \begin{align}
    & \frac{X_t(f) - \sum_{p\in \mathbb Z_+^d: \alpha \tilde \beta \geq |p|b} \langle f,\phi_p\rangle_\varphi e^{(\alpha - |p|b)t}H_\infty^p}{\|X_t\|^{1- \tilde \beta}}
      = \sum_{p\in \mathcal N}\frac{ \langle f,\phi_p\rangle_\varphi [X_t(\phi_p) - e^{(\alpha - |p|b)t}H_\infty^p]}{\|X_t\|^{1- \tilde \beta}}
    \\& = \sum_{p \in \mathcal N} \frac{\langle f,\phi_p\rangle_\varphi e^{(\alpha - |p|b)t}(H_t^p - H_\infty^p)}{\|X_t\|^{1- \tilde \beta}}
    = \sum_{k=0}^\infty \sum_{p \in \mathcal N}  \langle f,\phi_p\rangle_\varphi e^{(\alpha - |p|b)t}\frac{ H_{t+k}^p - H_{t+k+1}^p}{\|X_t\|^{1- \tilde \beta}}
    \\ & =: \sum_{k=0}^\infty \widetilde \Upsilon_{t,k}
         = \Big(\sum_{k = 0}^{\lfloor t^2 \rfloor}  \widetilde \Upsilon_{t,k} \Big) + \Big(\sum_{k = \lceil t^2 \rceil}^\infty  \widetilde \Upsilon_{t,k}\Big)
         = : I^f_1(t) + I^f_2(t),
  \end{align}
  and $I^f_0(t):= \sum_{k = 0}^{\lfloor t^2 \rfloor} \Upsilon_{t+k}^{- T_k \tilde f}$ where $\tilde f := \sum_{p\in \mathcal N} e^{-(\alpha - |p|b)}\langle f, \phi_p \rangle_\varphi \phi_p$.

  \emph{Step 1.} We show that $I^f_0(t) \xrightarrow [t\to \infty]{d} \zeta^{-f}$.
  In fact, since $T_k\tilde f \in \mathcal P_{\tilde f}$ for each $k\in \mathbb Z_+$, from Corollary \ref{cor:MI} we have $\widetilde{ \mathbb P}_\mu[e^{i I_0^f(t)}]\xrightarrow[t\to \infty]{}\exp\{\sum_{k=0}^\infty \langle Z_1T_k(-\tilde f),\varphi\rangle\}$.
  Using \eqref{eq:PL:S:1} and the fact that $\varphi(x)dx$ is the invariant probability of the semigroup $(P_t)_{t\geq 0}$ we have
  \begin{align}
    \label{eq:PM:CLTS:2}
    & \sum_{k=0}^\infty \langle Z_1 T_{k} (-\tilde f), \varphi \rangle
      = \sum_{k=0}^\infty \int_0^1 \langle P_u^\alpha ((iP_{1 - u}^\alpha T_k \tilde f)^{1+\beta}), \varphi\rangle ~du
    \\& = \sum_{k=0}^\infty \int_0^1 e^{\alpha u} \langle  (iP_{1 - u}^\alpha T_{k}\tilde f)^{1+\beta}, \varphi \rangle ~du
    \\& = \sum_{k=0}^\infty \int_0^1 \langle  (iT_{k+ u} f)^{1+\beta}, \varphi\rangle~du
    = \int_0^\infty \langle  (iT_{u} f)^{1+\beta}, \varphi\rangle~du = m[-f].
  \end{align}
  The result in this step follows.

  \emph{Step 2.} We show that $I^f_1(t) - I^f_0(t) \xrightarrow[t\to \infty]{d} 0$.
  In fact, by \cite[Lemma 3.4.3]{Durrett2010Probability} we have, for each $t\geq 0$, that $|\widetilde {\mathbb P}_{\mu}[e^{i(I_{1}^{f}(t) - I_0^f(t))} - 1]| \leq \sum_{k=0}^{\lfloor t^2 \rfloor} \widetilde {\mathbb {P}}_\mu[|Y_{t,k}|]$ where $Y_{t,k} := e^{i(\widetilde {\Upsilon}_{t,k} - \Upsilon_{t+k}^{-T_{k}\widetilde {f}})} - 1. $
  We claim that there exist $C_2, \delta_2>0$ such that $\widetilde {\mathbb {P}}_\mu[|Y_{t,k}|] \leq C_2e^{-\delta_2 t}$ for each $t\geq 0$ and $k \in \mathbb Z_+$.
  And therefore $|\widetilde {\mathbb P}_{\mu}[e^{i(I_{1}^{f}(t) - I_0^f(t))} - 1]| \leq (t^2+1)C_2e^{-\delta_2 t}$ which completes this step.

  We will show this claim in the following Substeps 2.1 and 2.2.
  First we chose $\gamma\in(0, \beta)$ close enough to $\beta$ so that $\alpha \tilde \gamma > |p|b$ for each $p\in \mathcal N$; 
  and even closer so that there exists $\eta,\eta'>0$ satisfying $\alpha \tilde \gamma > \eta>\eta - 3\eta'> \alpha (\tilde \beta - \tilde \gamma)>0$. We also define $\mathcal{D}_{t,k} :=\{|H_t-H_{t+k}|\leq  e^{-\eta t}, H_{t}> 2e^{-\eta' t}\}$.

% ******* subStep 2.1
  \emph{Substep 2.1.} Similar to the substep 2.1 in the proof of Theorem \ref{thm:M}.(\ref{thm:M:1}), we have that there exist $C_{2.1},\delta_{2.1} >0$ such that for all $k \in \mathbb Z_+$ and $t\geq 0$, $ \mathbb{\widetilde{P}}_{\mu}[|Y_{t,k}|;\mathcal{D}^c_{t,k}] \leq C_{2.1} e^{-\delta_{2.1} t}$.
  We omit the details.

  \emph{Substep 2.2.} We show that there exist $C_{2.2}, \delta_{2.2}>0$ such that for each $k \in \mathbb Z_+$ and $t\geq 0$, we have $\widetilde {\mathbb {P}}_\mu[|Y_{t,k}|; \mathcal D_{t,k}]\leq C_{2.2}e^{- \delta_{2,2} t}$.
  In fact, it can be verified that for each $k \in \mathbb Z_+$ and $t\geq 0$,
  \begin{align}
    & \Upsilon_{t+k}^{-T_k\tilde f}
      = \frac{X_{t+k}(P^\alpha_1T_k\tilde f) - X_{t+k+1}(T_k \tilde f)}{\|X_{t+k}\|^{1 - \tilde \beta}}
    \\& = \sum_{p\in \mathcal N}
    \langle\tilde f,\phi_p\rangle_\varphi e^{-(\alpha \tilde \beta - |pb|)k}\frac{ X_{t+k}(P_1^\alpha \phi_p) - X_{t+k+1}(\phi_p)}{\|X_{t+k}\|^{1 - \tilde \beta}}
    \\& = \sum_{p\in \mathcal N}
    \langle f,\phi_p\rangle_\varphi  e^{(\alpha  -|p|b)t}\frac{H_{t+k}^p-H_{t+k+1}^p }{\|e^{-\alpha k}X_{t+k}\|^{1 - \tilde \beta}}.
  \end{align}
  Therefore for all $k\in \mathbb Z_+$ and $t\geq 0$,
  \begin{align}
    &|Y_{t,k}| \mathbf 1_{\mathcal D_{t,k}}
      \leq \Big( \sum_{p\in \mathcal N}|\langle f,\phi_p\rangle_\varphi|  e^{(\alpha  -|p|b)t} | H_{t+k}^p-H_{t+k+1}^p |\Big) \Big( \frac{1}{\|X_t\|^{1 - \tilde \beta}} - \frac{1}{\|e^{-\alpha k}X_{t+k}\|^{1 - \tilde \beta}} \Big)\mathbf 1_{\mathcal D_{t,k}}.
    \\ &= \Big( \sum_{p\in \mathcal N}|\langle f,\phi_p\rangle_\varphi|  e^{(\alpha  -|p|b)t} | H_{t+k}^p-H_{t+k+1}^p |\Big)e^{\alpha (\tilde \beta - 1)t} K_{t,k}
    \\ &= \Big( \sum_{p\in \mathcal N}|\langle f,\phi_p\rangle_\varphi|  e^{(\alpha \tilde \beta  -|p|b)t} | H_{t+k}^p-H_{t+k+1}^p |\Big) K_{t,k},
  \end{align}
  where
  \[
    K_{t,k}
    := \Big| \frac {H_t^{1- \tilde \beta} - H_{t+k}^{1 - \tilde \beta}} {H_t^{1 - \tilde \beta} H_{t+k}^{ 1- \tilde \beta }} \Big| \mathbf{1}_{\mathcal{D}_{t,k}}.
  \]
  Similar to Substep 2.2 of the proof of Theorem \ref{thm:M}.(\ref{thm:M:1}), we can verify that for all $k\in \mathbb Z_+$ and $t\geq 0$, almost surely $K_{t,k} \leq C_{2.2}'' e^{- (\eta - 3\eta')t}$.
  From this and Lemma \ref{lem: control of wt} we know that there exists $C'''_{2.2}$ such that for each $k\in \mathbb Z_+$ and $t\geq 0$,
  \begin{align}
    & \widetilde{\mathbb P}_\mu[|Y_{t,k}|; \mathcal D_{t,k}]
      \leq \mathbb P_\mu(D)^{-1}\mathbb P_\mu[ |Y_{t,k}| ;\mathcal D_{t,k} ]
    \\ & \leq \mathbb P_{\mu}(D)^{-1} C_{2.2}'' e^{- (\eta - 3\eta') t}\sum_{p\in \mathcal {N}} |\langle f,\phi_p\rangle_\varphi|  e^{(\alpha \tilde \beta  -|p|b)t} \mathbb P_\mu[| H_{t+k}^p-H_{t+k+1}^p |]
    \\ & \leq \mathbb P_{\mu}(D)^{-1} C_{2.2}'' e^{- (\eta - 3\eta') t}\sum_{p\in \mathcal {N}} |\langle f,\phi_p\rangle_\varphi|  e^{(\alpha \tilde \beta  -|p|b)t} \| H_{t+k}^p-H_{t+k+1}^p \|_{\mathbb P_\mu; 1+\gamma}
    \\&\leq  \mathbb P_{\mu}(D)^{-1} C_{2.2}'' e^{- (\eta - 3\eta') t}\sum_{p\in \mathcal N} |\langle f,\phi_p\rangle_\varphi|  e^{(\alpha \tilde \beta  -|p|b)t} e^{-(\alpha \tilde \gamma - |p|b)(t+k)} \\
    &  \leq  C_{2.2}''' e^{- (\eta - 3\eta') t} e^{(\alpha \tilde \beta - \alpha \tilde \gamma)t},
  \end{align}
  as desired in this substep.

  \emph{Step 3.} We show that $I^f_2(t) \xrightarrow[t\to \infty]{d} 0$.
  In order to do this, chose an $\epsilon \in (0,\alpha)$ and a $\gamma \in (0,\beta)$ close enough to $\beta$ such that for each $p\in \mathcal N$, it holds that $\alpha \tilde \gamma > |p|b$.
  Define $\mathcal E_t:= \{\|X_t\| > e^{(\alpha - \epsilon)t}\}$.
  According to Proposition \ref{lem: control of XT}, there exist $C_3, \delta_3 > 0$ such that for each $t\geq 0$, $|\widetilde {\mathbb {P}}_\mu[e^{i I_2^f(t)} - 1; \mathcal E_t^c]|\leq 2\widetilde {\mathbb {P}}_\mu(\mathcal E_t^c) \leq C_3 e^{- \delta_3 t}$.
  On the other hand, according to Lemma \ref{lem: control of wt}, we know that there exist $C_3',C_3''>0$ and $\delta_3'>0$ such that
  \begin{align}
    & |\widetilde {\mathbb {P}}_\mu[e^{i I_{2}^{f}(t)} - 1; \mathcal {E}_t]|
      \leq  \widetilde {\mathbb {P}}_\mu[ | I_{2}^{f}(t)|; \mathcal {E}_t]
      \leq \sum_{k = \lceil t^2\rceil}^\infty \widetilde {\mathbb {P}}_\mu[ |\widetilde {\Upsilon}_{t,k}|; \mathcal {E}_t]
    \\ & \leq \mathbb P_\mu(D^c)^{-1} \sum_{k = \lceil t^2\rceil}^\infty \sum_{p \in \mathcal N} |\langle f,\phi_p\rangle_\varphi| e^{(\alpha - |p|b)t}\mathbb {P}_\mu\Big[\frac{ |H_{t+k}^p - H_{t+k+1}^p|}{\|X_t\|^{1- \tilde \beta}}; \mathcal E_t\Big]
    \\ & \leq \mathbb P_\mu(D^c)^{-1} e^{(\alpha - \epsilon) (\tilde \beta - 1) t} \sum_{k = \lceil t^2\rceil}^\infty \sum_{p \in \mathcal N} |\langle f,\phi_p\rangle_\varphi| e^{(\alpha - |p|b)t}\|H_{t+k}^p - H_{t+k+1}^p\|_{\mathbb P_\mu; 1+\gamma}
    \\ & \leq C_3' e^{(\alpha - \epsilon) (\tilde \beta - 1) t} \sum_{k = \lceil t^2\rceil}^\infty \sum_{p \in \mathcal N} |\langle f,\phi_p\rangle_\varphi| e^{(\alpha - |p|b)t} e^{- (\alpha \tilde \gamma - |p|b)(t+k)}
    \\ & = C_3'' e^{ \alpha (\tilde \beta - \tilde \gamma) t } e^{ \epsilon (1 - \tilde \beta) t}e^{- \delta'_3 t^2}.
  \end{align}
  To sum up we have that $\widetilde {\mathbb P}_\mu[e^{iI_2^f(t)}] \xrightarrow[t\to \infty]{} 1$, which completes this step.

  \emph{Final step.} Combining Steps 1--3, we complete the proof of \ref{thm:M}.\eqref{thm:M:3}.
\end{proof}

% ** Appendix
\appendix
\section{ }
\subsection{Analytic facts}
In this subsection, we collect some useful analytic facts.
\begin{lem}
  \label{lem: estimate of exponential remaining}
  For $z\in \mathbb C_+$,  we have
  \begin{align}
    \label{eq: estimate of exponential remaining}
    \Big|e^{-z} - \sum_{k=0}^n \frac{(-z)^k}{k!} \Big|
    \leq \frac{|z|^{n+1}}{(n+1)!} \wedge \frac{2|z|^{n}}{n!}, \quad n\in \mathbb Z_+.
  \end{align}
\end{lem}
\begin{proof}
  Notice that $|e^{-z}| = e^{- \operatorname{Re} z} \leq 1$.
  Therefore, $ |e^{-z} - 1| = \Big| \int_0^1 e^{-\theta z} z d\theta\Big| \leq |z|. $
  Also, notice that $|e^{-z} - 1| \leq |e^{-z}|+1 \leq 2$.
  Thus \eqref{eq: estimate of exponential remaining} is true when $n = 0$.
  Now, suppose that \eqref{eq: estimate of exponential remaining} is true when $n = m$ for some $m \in \mathbb Z_+$.
  Then
  \begin{align}
    &\Big|e^{-z} - \sum_{k=0}^{m+1} \frac{(-z)^k}{k!}\Big|
      = \Big| \int_0^1\Big(e^{-\theta z} - \sum_{k=0}^m \frac{(-\theta z)^k}{k!} \Big) z d\theta \Big| \\
    & \quad \leq  \Big(\int_0^1 \frac{|\theta z|^{m+1}}{(m+1)!} |z| d\theta\Big) \wedge \Big(\int_0^1 \frac{2|\theta z|^{m}}{m!} |z| d\theta\Big)
      = \frac{|z|^{m+2}}{(m+2)!} \wedge \frac{2|z|^{m+1}}{(m+1)!},
  \end{align}
  which says that \eqref{eq: estimate of exponential remaining} is true for $n = m + 1$.
\end{proof}

\begin{lem}
  \label{lem: extension lemma for branching mechanism}
  Suppose that  $\pi$ is a measure on $(0,\infty)$ with $\int_{(0,\infty)} (y \wedge y^2) \pi(dy)< \infty$.
  Then the functions
  \begin{align}
    & h (z)
      = \int_{(0,\infty)} (e^{-zy} - 1 + zy) \pi(dy), \quad z \in \mathbb C_+ \\
    & h'(z)
      = \int_{(0,\infty)}(1- e^{-zy})y \pi(dy), \quad z \in \mathbb C_+
  \end{align}
  are well defined, continuous on $\mathbb C_+$ and holomorphic on $\mathbb C_+^0$.
  Moreover,
  \[
    \frac{h(z)-h(z_0)}{z-z_0}
    \xrightarrow[\mathbb C_+\ni z \to z_0]{} h'(z_0),\quad z_0 \in \mathbb C_+.
  \]
\end{lem}
\begin{proof}
  It follows from Lemma \ref{lem: estimate of exponential remaining} that $h$ and $h'$ are well defined on $\mathbb C_+$.
  According to \cite[Theorems 3.2. \& Proposition 3.6]{SchillingSongVondravcek2010Bernstein}, $h'$ is continuous on $\mathbb C_+$ and holomorphic on $\mathbb C_+^0$.

  It follows from Lemma \ref{lem: estimate of exponential remaining} that, for each $z_0 \in \mathbb C_+$,  there exists $C>0$ such that for $z \in \mathbb C_+$ close enough to $z_0$ and any $y>0$,
  \begin{align}
    & \Big| \frac{e^{-zy} - e^{-z_0 y}+(z-z_0) y}{z-z_0} \Big|
      = \frac{1}{|z-z_0|}\Big| \int_0^1 \big(-y e^{-(\theta z+(1-\theta)z_0)y}+y\big)(z-z_0)d\theta\Big| \\
    & \leq y\int_0^1 |1-e^{-(\theta z +(1-\theta)z_0)y}| d\theta
      \leq (2y) \wedge\Big( y^2\int_0^1|\theta z+(1-\theta)z_0|d\theta\Big)
      \leq C(y\wedge y^2).
  \end{align}
  Using this and the dominated convergence theorem, we have
  \begin{align}
    & \frac{h(z)-h(z_0)}{z-z_0} = \int_{(0,\infty)} \frac{e^{-zy}+zy -(e^{-z_0 y}+z_0 y)}{z-z_0}  \pi(dy) \\
    & \xrightarrow[\mathbb C_+\ni z\to z_0]{} \int_{(0,\infty)}(1 - e^{-z_0 y} )y\pi(dy) = h'(z_0),
  \end{align}
  which says that $h$ is continuous on $\mathbb C_+$ and holomorphic on $\mathbb C_+^0$.
\end{proof}

For each $z\in \mathbb C\setminus (-\infty,0]$, we define $ \log z := \log |z| + i \arg z$ where $\arg z \in (-\pi,\pi)$ is uniquely determined by $ z = |z|e^{i \arg z}$. 	
For all $z\in \mathbb C\setminus (-\infty,0]$ and $\gamma \in \mathbb C$, we define $ z^\gamma := e^{\gamma \log z}. $
Then it is known, see \cite[Theorem 6.1]{SteinShakarchi2003Complex} for example, that $z\mapsto \log z$ is holomorphic in $\mathbb C\setminus (-\infty,0]$.
Therefore, for each $\gamma \in \mathbb C$, $z\mapsto z^\gamma$ is holomorphic in $\mathbb C\setminus (-\infty,0]$. (We use the convention that  $0^\gamma := \mathbf 1_{\gamma = 0}$.)
Using the definition above we can easily show that $(z_1z_0)^\gamma = z_1^\gamma z_0^\gamma$ provided $\arg (z_1z_0)=\arg (z_1) + \arg(z_0)$.

It is known, see, for instance, \cite[Theorem 6.1.3]{SteinShakarchi2003Complex} and the remark following it, that the Gamma function $\Gamma$ has an unique analytic extension in $\mathbb C\setminus\{0, -1,-2,\dots\}$ and that
\[
	\Gamma(z+1) 
  = z \Gamma(z),\quad z\in \mathbb C\setminus\{0, -1,-2,\dots\}.
\]
Using this recursively, one gets that
\begin{align}
  \label{eq: definition of Gamma function}
  \Gamma(x)
  := \int_0^\infty t^{x-1} \Big(e^{-t} - \sum_{k=0}^{n-1} \frac{(-t)^k}{k!}\Big) dt,
  \quad -n< x< -n+1, n\in \mathbb N.
\end{align}

Fix a $\beta \in (0,1)$.
Using the uniqueness of holomorphic extension and Lemma \ref{lem: extension lemma for branching mechanism}, we get that
\begin{align}
  z^{\beta}
	= \int_0^\infty (e^{-zy}-1) \frac{dy}{\Gamma(-\beta)y^{1+\beta}},
  \quad z\in \mathbb C_+,
\end{align}
and similarly that
\begin{align}
  \label{eq: stable branching on C+}
  z^{1+\beta}
  = \int_0^\infty (e^{-zy}-1+zy)\frac{dy}{\Gamma(-1-\beta)y^{2+\beta}},
  \quad z\in \mathbb C_+.
\end{align}
Lemma \ref{lem: extension lemma for branching mechanism} also says that the derivative of $z^{1+\beta}$ is $(1+\beta)z^{\beta}$ on $\mathbb C^0_+$.
\begin{lem}
  \label{lem: Lip of power function}
  For all $z_0,z_1 \in \mathbb C_+$, we have
\begin{align}
  \label{eq: Lip of power function}
  |z_0^{1+\beta} - z_1^{1+\beta}|
  \leq (1+\beta)(|z_0|^{\beta}+|z_1|^{\beta})|z_0 - z_1|.
\end{align}
\end{lem}
\begin{proof}
  Since $z^{1+\beta}$ is continuous on $\mathbb C_+$, we only need to prove the lemma assuming $z_0,z_1 \in \mathbb C^0_+$.
  Notice that
  \begin{align}
    \label{eq: upper bound for beta power of z}
    |z^\beta|
    = |e^{\beta \log |z| +i\beta \operatorname {arg}z}| = e^{\beta \log |z|} = |z|^\beta,
    \quad z \in \mathbb C\setminus (-\infty, 0].
  \end{align}
  Define a path $\gamma: [0,1] \to \mathbb C^0_+$ such that
  \[
    \gamma(\theta)
    = z_0 (1-\theta) + \theta z_1,
    \quad \theta \in [0,1].
  \]
  Then, we have
  \begin{align}
    |z_0^{1+\beta} - z_1^{1+\beta}|
    & \leq (1+\beta) \int_0^1 |\gamma(\theta)^{\beta}|\cdot |\gamma'(\theta)|d\theta
      \leq (1+\beta)  \sup_{\theta \in [0,1]} |\gamma(\theta)|^{\beta} \cdot |z_1-z_0| \\
    & \leq (1+\beta)  ( |z_1|^{\beta}+|z_0|^{\beta} ) |z_1-z_0|.
      \qedhere
  \end{align}
\end{proof}

Suppose that $\varphi(\theta)$ is a continuous function from $\mathbb R$ into $\mathbb C$ such that $\varphi(0) = 1$ and $\varphi(\theta) \neq 0$ for all $\theta \in \mathbb R$.
Then according to \cite[Lemma 7.6]{Sato2013Levy}, there is a unique continuous function $f(\theta)$ from $\mathbb R$ into $\mathbb C$ such that $f(0) = 0$ and $e^{f(\theta)} = \varphi(\theta)$.
Such a function $f$ is called the distinguished logarithm of the function $\varphi$ and is denoted as $\operatorname{Log} \varphi(\theta)$.
In particular, when $\varphi$ is the characteristic function of an infinitely divisible random variable $Y$,  $\operatorname{Log} \varphi(\theta)$ is called the L\'evy exponent of $Y$.
This distinguished logarithm should not be confused with the $\log$ function defined on $\mathbb C\setminus (-\infty, 0]$.
See the paragraph immediately after \cite[Lemma 7.6]{Sato2013Levy}.

% *** Feynman-Kac formula with complex valued functions
\subsection{Feynman-Kac formula with complex valued functions}
\label{seq: complex Feynman-Kac transform}
In this subsection we give a version of the Feynman-Kac formula with complex valued functions.
Suppose that $\{(\xi_t)_{t \in [r,\infty)}; (\Pi_{r,x})_{r\in [0,\infty), x\in E}\}$ is a (possibly non-homogeneous) Hunt process in a locally compact separable metric space $E$.
We write
\begin{align}
  H^{(h)}_{(s,t)}
  := \exp\Big\{\int_s^t h(u,\xi_u) du\Big\},
  \quad 0 \leq s \leq t, h \in \mathcal B_b([0,t] \times E,\mathbb C).
\end{align}

\begin{lem}
  \label{eq: complex FK}
  Let $t \geq 0$. Suppose that $\rho_1, \rho_2\in \mathcal B_b([0,t] \times E, \mathbb C)$ and $f\in \mathcal B_b(E, \mathbb C)$.
  Then
  \begin{align}
    \label{eq: expresion of g}
    g(r,x)
    := \Pi_{r,x}[ H_{(r,t)}^{(\rho_1+\rho_2)} f(\xi_t)],\quad r \in [0,t], x\in E,
  \end{align}
  is the unique locally bounded solution to the equation
  \[
    g(r,x)
    = \Pi_{r,x} [ H_{(r,t)}^{(\rho_1)} f(\xi_t)]+\Pi_{r,x} \Big[ \int_r^tH_{(r,s)}^{(\rho_1)}\rho_2(s,\xi_s) g(s,\xi_s)~ds \Big],\quad r \in [0,t], x\in E.
  \]
\end{lem}

\begin{proof}
  The proof is similar to that of \cite[Lemma A.1.5]{Dynkin1993Superprocesses}. We include it here for the sake of completeness.
  We first verify that \eqref{eq: expresion of g} is a solution.
  Notice that
  \begin{align}
    \Pi_{r,x} \Big[ \int_r^t | H_{(r,t)}^{(\rho_1)}\rho_2(s,\xi_s) H_{(s,t)}^{(\rho_2)} f(\xi_t)| ~ds \Big]
    \leq  \int_r^t e^{(t-r)\|\rho_1\|_\infty}e^{(t-s)\|\rho_2\|_\infty}\|\rho_2\|_\infty\|f\|_\infty ~ds
    < \infty.
  \end{align}
  Also notice that
  \begin{align}
    \label{eq: crucial for Feynman-Kac}
    \frac{\partial}{\partial s} H^{(\rho_2)}_{(s,t)}= -H^{(\rho_2)}_{(s,t)}\rho_2(s,\xi_s),
    \quad s\in (0,t).
  \end{align}
  Therefore, from the Markov property of $\xi$ and Fubini's theorem we get that
  \begin{align}
    & \Pi_{r,x} \Big[ \int_r^tH_{(r,s)}^{(\rho_1)}~(\rho_2 g)(s,\xi_s)~ds \Big]
      =\Pi_{r,x} \Big[ \int_r^t H_{(r,s)}^{(\rho_1)}\rho_2(s,\xi_s) \Pi_{s,\xi_s}[ H_{(s,t)}^{(\rho_1+\rho_2)} f(\xi_t)]~ds \Big] \\
    & = \Pi_{r,x} \Big[ \int_r^t H_{(r,t)}^{(\rho_1)}\rho_2(s,\xi_s) H_{(s,t)}^{(\rho_2)} f(\xi_t) ~ds \Big]
      = \Pi_{r,x} [ H_{(r,t)}^{(\rho_1)}f(\xi_t)(H_{(r,t)}^{(\rho_2)} - 1)] \\
    & = g(r,x) - \Pi_{r,x} [ H_{(r,t)}^{(\rho_2)} f(\xi_t)].
  \end{align}
  For uniqueness, suppose  $\widetilde g$ is another solution. Put $h(r) = \sup_{x\in E}|g(r,x) - \widetilde g(r,x)|$.
  Then
  \[
    h(r)
    \leq e^{t\|\rho_1\|_\infty}\|\rho_2\|_\infty \int_r^t h(s)ds,
    \quad r\le t.
  \]
  Applying Gronwall's inequality, we get that $h(r) =  0$ for $r\in [0,t]$.
\end{proof}

% ** Superprocesses
\subsection{Superprocesses}
\label{sec: definition of superprocess}
In this subsection, we will give the definition of a general superprocess.
Let $E$ be locally compact separable metric space. Denote by $\mathcal M(E)$ the collection of all the finite measures on $E$ equipped with the topology of weak convergence.
For each function $F(x,z)$ on $E\times \mathbb R_+$ and each $\mathbb R_+$-valued function $f$ on $E$, we use the following convention in this subsection:
\[
  F(x,f)
  := F(x,f(x)),\quad x\in E.
\]
A process $X=\{(X_t)_{t\geq 0}; (\mathbf P_\mu)_{\mu \in \mathcal M(E)}\}$ is said to be a $(\xi,\psi)$-superprocess if
\begin{itemize}
\item
  the spatial motion $\xi=\{(\xi_t)_{t\geq 0};(\Pi_x)_{x\in E}\}$ is an $E$-valued Hunt process with its lifetime denoted by $\zeta$;
\item
  the branching mechanism $\psi: E\times[0,\infty) \to \mathbb R$ is given by
\begin{align}
  \label{eq: branching mechanism}
  \psi(x,z)=
  -\rho_1(x) z + \rho_2 (x) z^2 + \int_{(0,\infty)} (e^{-zy} - 1 + zy) \pi(x,dy).
\end{align}
where $\rho_1 \in \mathcal B_b(E)$, $\rho_2 \in \mathcal B_b(E, \mathbb R_+)$ and $\pi(x,dy)$ is a kernel from $E$ to $(0,\infty)$ such that $\sup_{x\in E} \int_{(0,\infty)} (y\wedge y^2) \pi(x,dy) < \infty$;
\item
  $X=\{(X_t)_{t\geq 0}; (\mathbf P_\mu)_{\mu \in \mathcal M(E)}\}$ is an $\mathcal M(E)$-valued Hunt process with transition probability determined by
  \begin{align}
    \mathbf P_\mu [e^{-X_t(f)}] = e^{-\mu(V_tf)},
    \quad t\geq 0, \mu \in \mathcal M(E), f\in \mathcal B^+_b(E),
  \end{align}
  where for each $f\in \mathcal B_b(E)$, the function $(t,x)\mapsto V_tf(x)$ on $[0,\infty) \times E$ is the unique locally bounded positive solution to the equation
  \begin{align}
    \label{eq:FKPP_in_definition}
    V_tf(x) + \Pi_x \Big[  \int_0^{t\wedge \zeta} \psi(\xi_s,V_{t-s}f)ds \Big]
    = \Pi_x [ f(\xi_t)\mathbf 1_{t<\zeta} ],
    \quad t \geq 0, x \in E.
  \end{align}
\end{itemize}
We refer our readers to \cite{Li2011Measure-valued} for more discussions about the definition and the existence of superprocesses.
To avoid triviality, we assume that $\psi(x,z)$ is not identically equal to $-\rho_1(x)z$.

Notice that the branching mechanism $\psi$ can be extended into a map from $E \times \mathbb C_+$ to $\mathbb C$ using \eqref{eq: branching mechanism}.
Define
\begin{align}
  \psi'(x,z)
  := - \rho_1(x) + 2\rho_2(x) z + \int_{(0,\infty)} (1-e^{-zy})y\pi(x,dy),
  \quad x\in E, z\in \mathbb C_+.
\end{align}
Then according to Lemma \ref{lem: extension lemma for branching mechanism}, for each $x \in E$, $z \mapsto \psi(x,z)$ is a holomorphic function on $\mathbb C_+^0$ with derivative $z \mapsto \psi'(x,z)$.
Define $\psi_0(x,z) := \psi(x,z)+ \rho_1(x)z $ and $\psi'_0(x,z) := \psi'(x,z) + \rho_1(x)$.

Denote by $\mathbb W$ the space of $\mathcal M(E)$-valued c\`{a}dl\`{a}g paths with its canonical path denoted by $(W_t)_{t\geq 0}$.
We say $X$ is \emph{non-persistent} if $\mathbf P_{\delta_x}(\|X_t\|= 0) > 0$ for all $x\in E$ and $t> 0$.
Suppose that $(X_t)_{t\geq 0}$ is non-persistent, then according to \cite[Section 8.4]{Li2011Measure-valued}, there is a unique family of measures $(\mathbb N_x)_{x\in E}$ on $\mathbb W$ such that
(i) $\mathbb N_x (\forall t > 0, \|W_t\|=0) =0$;
(ii) $\mathbb N_x(\|W_0 \|\neq 0) = 0$;
and (iii) if $\mathcal N$ is a Poisson random measure defined on some probability space with intensity $\mathbb N_\mu(\cdot):= \int_E \mathbb N_x(\cdot )\mu(dx)$, then the superprocess $\{X;\mathbf P_\mu\}$ can be realized by $\widetilde X_0 := \mu$ and $\widetilde X_t(\cdot) := \mathcal N[W_t(\cdot)]$ for each $t>0$.
We refer to $(\mathbb N_x)_{x\in E}$ as the \emph{Kuznetsov measures} of $X$.

% ** Semigroup for superproccesses
\subsection{Semigroups for superprocesses}
\label{sec: definition of vf}
Let $X$ be a non-persistent superprocess with its Kuznetsov measure denoted by $(\mathbb N_x)_{x\in E}$.
We define the mean semigroup
\begin{align}
  P_t^{\rho_1} f(x)
  := \Pi_{x}[e^{\int_0^t \rho_1(\xi_s)ds}f(\xi_t) \mathbf 1_{t< \zeta}],
  \quad t\geq 0, x\in E, f\in \mathcal B_b(E,\mathbb R_+).
\end{align}
It is known from \cite[Proposition 2.27]{Li2011Measure-valued} and \cite[Theorem 2.7]{Kyprianou2014Fluctuations} that for all $t > 0$, $\mu \in \mathcal M(E)$ and $f\in \mathcal B_b(E,\mathbb R_+)$,
\begin{align}
  \label{eq: mean formula for superprocesses}
  \mathbb N_{\mu}[\langle W_t, f\rangle]
  =\mathbf P_{\mu}[\langle X_t, f\rangle]
  =\mu(P^{\rho_1}_t f).
\end{align}

Define
\begin{align}
  L_1(\xi)
  &:= \{f\in \mathcal B(E): \forall x\in E, t\geq 0, \quad \Pi_x[|f(\xi_t)|]< \infty\}, \\
  L_2(\xi)
  &:= \{f \in \mathcal B(E): |f|^2 \in L_1(\xi)\}.
\end{align}
Using monotonicity and linearity, we get from \eqref{eq: mean formula for superprocesses}  that
\begin{align}
  \mathbb N_x[\langle W_t, f\rangle]
  = \mathbf P_{\delta_x}[\langle f, X_t\rangle]
  = P^{\rho_1}_t f(x) \in \mathbb R,
  \quad f\in L_1(\xi), t > 0,x\in E.
\end{align}
This says that the random variable $\langle X_t, f\rangle$ is well defined under probability $\mathbf P_{\delta_x}$ provided $f\in L_1(\xi)$.
By the branching property of the superprocess, $\langle X_t, f\rangle$ is an infinitely divisible random variable.
Therefore, we can write
\[
  U_t(\theta f)(x)
  := \operatorname{Log} \mathbf P_{\delta_x}[e^{i \theta \langle X_t, f\rangle}],
  \quad t\geq 0, f\in L_1(\xi), \theta \in \mathbb R, x\in E,
\]
as its characteristic exponent.
According to Campbell's formula, see \cite[Theorem 2.7]{Kyprianou2014Fluctuations} for example, we have
\[
  \mathbf P_{\delta_x} [e^{i\theta \langle X_t, f\rangle}]
  = \exp(\mathbb N_x[ e^{i\theta \langle W_t, f\rangle} - 1]),
  \quad t>0, f\in L_1(\xi), \theta \in \mathbb R, x\in E.
\]
Noticing that $\theta \mapsto \mathbb N_x[e^{i\theta W_t(f)} - 1]$ is a continuous function on $\mathbb R$ and that $\mathbb N_x[e^{i\theta \langle W_t, f\rangle} - 1] = 0$ if $\theta = 0$, according to \cite[Lemma 7.6]{Sato2013Levy}, we have
\begin{align}
  \label{eq: N and characteristic exponent}
  U_t(\theta f)(x)
  = \mathbb N_x[e^{i \langle W_t, \theta f\rangle} - 1],
  \quad t>0, f\in L_1(\xi), \theta \in \mathbb R, x\in E.
\end{align}

\begin{lem}
  There exists a constant $C\geq 0$ such that
  for all $f \in L_1(\xi),x\in E$ and $t\geq 0$, we have
  \begin{align}
    \label{eq: upper bound of psi(v)}
    \big|\psi\big(x,-U_tf\big)\big|
    \leq C P^{\rho_1}_t |f|(x) + C (P^{\rho_1}_t |f| (x))^2.
  \end{align}
\end{lem}

\begin{proof}
  Noticing that
  \[
    e^{\operatorname{Re} U_tf(x)}
    = |e^{U_tf(x)}|
    = |\mathbf P_{\delta_x}[e^{i \langle X_t, f\rangle}]|
    \leq 1,
  \]
  we have
  \begin{align}
    \label{eq: -v has positive real part}
    \operatorname{Re} U_tf(x)
    \leq 0.
  \end{align}
  Therefore, we can speak of $\psi(x,-U_tf)$ since $z\mapsto \psi(x,z)$ is well defined on $\mathbb C_+$.
  According to Lemma \ref{lem: estimate of exponential remaining}, we have that
  \begin{align}
    \label{eq: upper bound for vf}
    |U_tf(x)|
    \leq \mathbb N_x[|e^{i \langle W_t, f\rangle} - 1|]
    \leq \mathbb N_x[|i \langle W_t, f\rangle|]
    \leq (P^{\rho_1}_t |f|)(x).
  \end{align}
  Notice that, for any compact $K \subset \mathbb R$,
  \begin{align}
    \label{eq: estimate of deriavetive of v(theta)}
    \mathbb N_x \Big[\sup_{\theta \in K} \Big|\frac{\partial}{\partial \theta} (e^{i\theta \langle W_t, f\rangle} - 1) \Big|\Big]
    \leq \mathbb N_x[|\langle W_t, f\rangle|] \sup_{\theta \in K}|\theta|
    \leq (P^{\rho_1}_t |f|)(x) \sup_{\theta \in K}|\theta| < \infty.
  \end{align}
  Therefore, according to \cite[Theorem A.5.2]{Durrett2010Probability} and \eqref{eq: N and characteristic exponent}, $ U_t( \theta f)( x )$ is differentiable in $\theta \in \mathbb R$ with
  \[
    \frac{\partial}{\partial \theta} U_t(\theta f)(x)
    = i\mathbb N_x[\langle W_t, f\rangle e^{i\theta \langle W_t, f\rangle}],
    \quad \theta \in \mathbb R.
  \]
  Moreover, from the above, it is clear that
  \begin{align}
    \label{eq: upper bounded for derivative of v(theta)}
    \sup_{\theta \in \mathbb R}\Big| \frac{\partial}{\partial \theta}U_t(\theta f)(x)\Big|
    \leq ( P^{\rho_1}_t |f|)(x).
  \end{align}
  It follows from the dominated convergence theorem that $(\partial/\partial \theta)U_t(\theta f)(x)$ is continuous in $\theta$.
  In other words, $\theta \mapsto -U_t(\theta f)(x)$ is a $C^1$ map from $\mathbb R$ to $\mathbb C_+$.
  Thus,
  \begin{align}
    \label{eq: path integration representation of psi(v)}
    \psi(x,-U_tf)
    = -\int_0^1 \psi'\big(x,-U_t(\theta f)\big) \frac{\partial}{\partial \theta} U_t(\theta f)(x)~d\theta.
  \end{align}
  Notice that
  \begin{align}
    & |\psi'(x, -U_tf)| \\
    & = \Big| -\rho_1(x)- 2\rho_2(x) U_tf(x)+ \int_{(0,\infty)} y (1- e^{y U_tf(x)} ) \pi(x,dy)\Big| \\
    & = \Big| - \rho_1(x)- 2\rho_2(x)\mathbb N_x[e^{i \langle W_t, f\rangle} - 1]  + \int_{(0,\infty)} y \mathbf P_{y \delta_x}[1-e^{i \langle X_t, f\rangle}] \pi(x,dy) \Big| \\
    & \leq \|\rho_1\|_\infty + 2\rho_2(x)\mathbb N_x[\langle W_t, |f|\rangle]+ \int_{(0,\infty)} y\mathbf P_{y\delta_x}[2\wedge \langle X_t, |f|\rangle] \pi(x,dy) \\
    & \leq \|\rho_1\|_\infty + 2\|\rho_2\|_\infty P^{\rho_1}_t |f|(x) + \Big(\sup_{x\in E}\int_{(0,1]}y^2 \pi(x,dy)\Big)~P^{\rho_1}_t |f|(x) + 2\sup_{x\in E}\int_{(1,\infty)} y \pi(x,dy) \\
    & =: C_1 + C_2(P^{\rho_1}_t |f|)(x), \label{eq: upper bound of psi'(v)}
  \end{align}
  where $C_1, C_2$ are constants independent on $f,x$ and $t$.
  Now, combining the display above with \eqref{eq: path integration representation of psi(v)} and \eqref{eq: upper bounded for derivative of v(theta)} we get the desired result.
\end{proof}

This lemma also says that if $f\in L^2(\xi)$ then
\[
  \Pi_x\Big[\int_0^t \psi(\xi_s,- U_{t-s}f)ds\Big]
  \in \mathbb C,
  \quad x\in E, t\geq 0.
\]
is well defined.
In fact, using Jensen's inequality and the Markov property, we have
\begin{align}
  \label{eq: domination of psi(v)}
  & \Pi_x\Big[\int_0^t |\psi \big(\xi_s,-U_{t-s}f\big)|ds\Big]
  \leq \Pi_x\Big[\int_0^t \big(C_1 P_{t-s}^{\rho_1}|f|(\xi_s)+C_2 P_{t-s}^{\rho_1}|f|(\xi_s)^2\big)ds\Big] \\
  & \leq \int_0^t \big(C_1 e^{t\|\rho_1\|}\Pi_x \big[ \Pi_{\xi_s}[|f(\xi_{t-s})|] \big]+C_2 e^{2t\|\rho_1\|}\Pi_x \big[ \Pi_{\xi_s}[|f (\xi_{t-s})|]^2 \big]\big)~ds \\
  & \leq \int_0^t (C_1 e^{t\|\rho_1\|}\Pi_x [ |f(\xi_{t})|]+C_2e^{2t\|\rho_1\|}\Pi_x [ |f (\xi_{t})|^2 ])~ds < \infty.
\end{align}

% ** A complex-valued non-linear integral equation
\subsection{A complex-valued non-linear integral equation}
Let $X$ be a non-persistent superprocess.
In this subsection, we will prove the following:

\begin{prop}
  \label{prop: complex FKPP-equation}
  If $f\in L_2(\xi)$,  then for all $t\geq 0$ and $x\in E$,
\begin{align}
  \label{eq: complex FKPP-equation}
  U_tf(x) - \Pi_x \Big[\int_0^t \psi\big(\xi_s, - U_{t-s}f\big) ds \Big]
  = i \Pi_x [f(\xi_t)].
\end{align}
\begin{align}
  \label{eq: complex FKPP-equation with FK-transform}
  U_tf(x) -  \int_0^t P_{t-s}^{\rho_1} \psi_0\big(\cdot,-U_sf\big) (x)~ds
  = iP_t^{\rho_1} f(x).
\end{align}
\end{prop}

To prove this, we will need the generalized spine decomposition theorem from \cite{RenSongSun2017Spine}.
Let $f\in \mathcal B_b(E,\mathbb R_+)$, $T >0$ and $x\in E$.
Suppose that $\mathbf P_{\delta_x}[\langle X_T, f\rangle] = \mathbb N_x[\langle W_T, f\rangle] = P^{\rho_1}_T f(x) \in (0,\infty)$, then we can define the following probability transforms:
\begin{align}
  d\mathbf P_{\delta_x}^{\langle X_T, f\rangle}
  := \frac{\langle X_T, f\rangle}{P_T^{\rho_1} f(x)} d\mathbf P_{\delta_x};
  \quad d\mathbb N_x^{\langle W_T, f\rangle}
  :=  \frac{\langle W_T, f\rangle}{P_T^{\rho_1} f(x)} d\mathbb N_x.
\end{align}
Following the definition in \cite{RenSongSun2017Spine}, we say that $\{\xi, \mathbf n;\mathbf Q_{x}^{(f,T)}\}$ is a spine representation of $\mathbb N_x^{\langle W_T, f\rangle}$ if
\begin{itemize}
\item
  the spine process $\{(\xi_t)_{0\leq t\leq T}; \mathbf Q^{(f,T)}_x\}$ is a copy of $\{(\xi_t)_{0\leq t\leq T}; \Pi^{(f,T)}_{x}\}$, where
  \begin{align}
    d\Pi_x^{(f,T)}
    := \frac{f(\xi_T)e^{\int_0^T \rho_1(\xi_s)ds}}{P^{\rho_1}_T f(x)} d \Pi_x;
  \end{align}
\item
  given $\{(\xi_t)_{0\leq t\leq T}; \mathbf Q^{(f,T)}_x\}$, the immigration measure
\[
  \{\mathbf n(\xi,ds,dw); \mathbf Q^{(f,T)}_x[\cdot |(\xi_t)_{0\leq t\leq T}]\}
\]
is a Poisson random measure on $[0,T] \times \mathbb W$ with intensity
\begin{align}
  \label{eq: conditional intensity}
  \mathbf m(\xi,ds,dw)
  := 2 \rho_2(\xi_s) ds \cdot \mathbb N_{\xi_s}(dw) + ds \cdot \int_{y\in (0,\infty)} y \mathbf P_{y\delta_{\xi_s}}(X\in dw) \pi(\xi_s,dy);
\end{align}
\item
  $\{(Y_t)_{0\leq t\leq T}; \mathbf Q^{(f,T)}_x\}$ is an $\mathcal M(E)$-valued process defined by
  \begin{align}
    Y_t
    := \int_{(0,t] \times \mathbb W} w_{t-s} \mathbf n(\xi,ds,dw),
    \quad 0 \leq t\leq T.
  \end{align}
\end{itemize}
According to the spine decomposition theorem in \cite{RenSongSun2017Spine}, we have that
\begin{align}
  \label{eq: Spine decomposition 1}
  \{(X_s)_{s \geq 0};\mathbf P_{\delta_x}^{\langle X_T, f\rangle}\}
  \overset{f.d.d.}{=} \{(X_s + W_s)_{s \geq 0};\mathbf P_{\delta_x} \otimes \mathbb N_x^{\langle W_T, f\rangle} \},
\end{align}
\begin{align}
  \label{eq: Spine decomposition 2}
  \{(W_s)_{0\leq s\leq T};\mathbb N_x^{\langle W_T, f\rangle}\}
  \overset{f.d.d.}{=} \{(Y_s)_{s \geq 0};\mathbf Q_x^{(f,T)}\}.
\end{align}

\begin{proof}[Proof of Proposition \ref{prop: complex FKPP-equation}]
  Assume that $f\in \mathcal B_b(E)$.
  Fix $t>0, r\in [0,t), x\in E$ and a strictly positive $g\in \mathcal B_b(E)$.
  Denote by $\{\xi, \mathbf n; \mathbf Q_x^{(g,t)}\}$ the spine representation of $\mathbb N_x^{\langle W_t, g\rangle}$.
  Conditioned on $\{\xi; \mathbf Q_x^{(g,t)}\}$, denote by $\mathbf m(\xi, ds,dw)$ the conditional intensity of $\mathbf n$ in \eqref{eq: conditional intensity}.
  Denote by $\Pi_{r,x}$ the probability of Hunt process $\{\xi; \Pi\}$ initiated at time $r$ and position $x$.
  From Lemma \ref{lem: estimate of exponential remaining}, we have $\mathbf Q^{(g,t)}_{x}$-almost surely
  \begin{align}
    & \int_{[0,t]\times \mathbb W}|e^{i \langle w_{t-s}, f\rangle} - 1| \mathbf m(\xi, ds,dw)
      \leq \int_{[0,t]\times \mathbb W}\big(| \langle w_{t-s}, f\rangle| \wedge 2\big) \mathbf m(\xi, ds,dw) \\
    & \leq \int_0^t \Big(2\rho_2(\xi_s)\mathbb N_{\xi_s}\big( \langle W_{t-s}, |f|\rangle\big)  + \int_{(0,1]} y \mathbf P_{y \delta_{\xi_s}}[\langle X_{t-s}, |f|\rangle] \pi(\xi_s,dy) \\
    & \qquad\qquad + 2\int_{(1,\infty)}y\pi(\xi_s,dy)\Big) ds
    \\ & \leq \int_0^t (P_{t-s}^{\rho_1} |f|)(\xi_s)\Big(2\rho_2(\xi_s)  + \int_{(0,1]} y^2 \pi(\xi_s,dy)\Big) ds + 2t \sup_{x\in E}\int_{(1,\infty)}y\pi(x,dy)
    \\ & \leq \Big(2\|\rho_2\|_\infty +\sup_{x\in E}\int_{(0,1]} y^2 \pi(x,dy)\Big) t e^{t\|\rho_1\|_\infty}\|f\|_\infty + 2t \sup_{x\in E}\int_{(1,\infty)}y\pi(x,dy)
         < \infty.
  \end{align}
  Using this, Fubini's theorem, \eqref{eq: N and characteristic exponent} and \eqref{eq: -v has positive real part} we have $\mathbf Q^{(g,t)}_{x}$-almost surely,
  \begin{align}
    & \int_{[0,t]\times \mathbb N}(e^{i \langle w_{t-s}, f\rangle} - 1) \mathbf m(\xi, ds,dw)
    \\ & =\int_0^t \Big(2\rho_2(\xi_s)\mathbb N_{\xi_s}(e^{i \langle W_{t-s}, f\rangle} - 1)  + \int_{(0,\infty)} y \mathbf P_{y \delta_{\xi_s}}[e^{i \langle X_{t-s}, f\rangle} - 1] \pi(\xi_s,dy)\Big) ds
    \\ & =\int_0^t \Big( 2\rho_2(\xi_s) U_{t-s} f(\xi_s) + \int_{(0,\infty)} y (e^{y U_{t-s}f(\xi_s)} - 1) \pi(\xi_s,dy) \Big) ds
    \\ & = -\int_0^t \psi'_0 \big(\xi_s, -U_{t-s}f\big)ds.
  \end{align}
  Therefore, according to \eqref{eq: Spine decomposition 2}, Campbell's formula and above, we have that
  \begin{align}
    \label{eq: N to Pi}
    & \mathbb N_x^{\langle W_{t}, g\rangle}[e^{i \langle W_t, f\rangle}]
      = \mathbf Q_x^{(g,t)} \Big[\exp\Big\{\int_{[0,t]\times \mathbb N}(e^{i \langle w_{t-s}, f\rangle} - 1) \mathbf m(\xi, ds,dw)\Big\}\Big]
    \\ & = \Pi_x^{(g,t)} [e^{-\int_0^t \psi'_0(\xi_s, -U_{t-s}f)ds}]
    = \frac{1}{P_t^{\rho_1} g (x)} \Pi_x[ g(\xi_t) e^{-\int_0^t \psi'(\xi_s, -U_{t-s}f)ds} ].
  \end{align}
  Let $\epsilon >0$.
  Define $f^+ = (f \vee 0) + \epsilon$ and $f^- = (-f) \vee 0 + \epsilon$, then $f^\pm$ are strictly positive and $f = f^+ - f^-$.
  According to \eqref{eq: Spine decomposition 1}, we have that
  \begin{align}
    \frac{\mathbf P_{\delta_x}[\langle X_t,f^{\pm}\rangle e^{i \langle X_t,f\rangle}]}{\mathbf P_{\delta_x}[\langle X_t,f^{\pm}\rangle ]}
    = \mathbf P_{\delta_x}[e^{i \langle X_t,f\rangle}] \mathbb N_x^{\langle W_t,f^{\pm}\rangle}[e^{i \langle X_t,f\rangle}].
  \end{align}
  Using \eqref{eq: N to Pi} and the above, we have
  \begin{align}
    \frac{\mathbf P_{\delta_x}[\langle X_t, f\rangle e^{i \langle X_t, f\rangle}] }{\mathbf P_{\delta_x}[e^{i \langle X_t, f\rangle}]}
    & = \mathbf P_{\delta_x}[\langle X_t, f^+\rangle] \mathbb N_x^{\langle W_t, f^+\rangle} [e^{i \langle X_t, f\rangle}] - \mathbf P_{\delta_x}[\langle X_t, f^-\rangle]\mathbb N_x^{\langle W_t, f^-\rangle}[e^{i \langle X_t, f\rangle}]
    \\ & = \Pi_x[ f(\xi_t) e^{- \int_0^t \psi'(\xi_s, -U_{t-s}f) ds}  ].
  \end{align}
  Therefore, we have
  \begin{align}
    \frac{\partial}{\partial \theta} {U_t(\theta f)(x)}
    = \frac{\mathbf P_{\delta_x}[i\langle X_t, f\rangle e^{i \langle X_t, f\rangle}] }{\mathbf P_{\delta_x}[e^{i \langle X_t, f\rangle}]}
    = \Pi_x[ if(\xi_t) e^{ - \int_0^t \psi'(\xi_s, -U_{t-s}(\theta f)) ds} ].
  \end{align}
  Since $\{(\xi_{r+t})_{t \geq 0}; \Pi_{r,x}\} \overset{d}{=} \{(\xi_{t})_{t\geq 0}; \Pi_{x}\} $, we have
  \begin{align}
    & \frac{\partial}{\partial \theta} U_{t-r}(\theta f)( x)
      = \Pi_x[ i f(\xi_{t-r}) e^{-\int_0^{t-r} \psi'(\xi_s, -U_{t-r-s}(\theta f)) ds} ] \\
    & = \Pi_{r,x}[i f(\xi_t)e^{-\int_0^{t-r} \psi'(\xi_{r+s}, -U_{t-r-s}(\theta f)) ds} ]
      = \Pi_{r,x}[if(\xi_t)e^{-\int_r^t \psi'(\xi_{s}, -U_{t-s}(\theta f)) ds} ].
  \end{align}
  
  From \eqref{eq: upper bound of psi'(v)}, we know that for each $\theta\in \mathbb R$, $(t,x) \mapsto |\psi'(x,-U_tf(x))|$ is locally bounded (i.e. bounded on $[0,T]\times E$ for each $T \geq 0$).
  Therefore, we can apply Lemma \ref{eq: complex FK} and get that
  \[
    \frac{\partial}{\partial \theta} U_{t-r}(\theta f)(x) + \Pi_{r,x} \Big[\int_r^t \psi'\big(\xi_s,- U_{t-s}(\theta f)\big)\frac{\partial}{\partial \theta} U_{t-s}(\theta f)(\xi_s)~ds\Big]
    = \Pi_{r,x} [i f(\xi_t)]
  \]
  and
  \begin{align}
    & \frac{\partial}{\partial \theta} U_{t-r}(\theta f)(x) + \Pi_{r,x} \Big[\int_r^t e^{\int_r^s \rho_1(\xi_u)du}\psi_0'\big(\xi_s,- U_{t-s}(\theta f)\big)\frac{\partial}{\partial \theta} U_{t-s}(\theta f)(\xi_s)~ds\Big]\\
    & = \Pi_{r,x} [i e^{\int_r^t \rho_1(\xi_s)ds}f(\xi_t)].
  \end{align}
  Integrating the two displays above with respect to $\theta$  on [0,1], using
  Fubini's theorem, \eqref{eq: upper bounded for derivative of v(theta)}, \eqref{eq: path integration representation of psi(v)} and \eqref{eq: upper bound of psi'(v)}, we get
  \begin{align}
    U_{t-r}f(x) - \Pi_{r,x} \Big[\int_r^t \psi\big(\xi_s,-U_{t-s}f\big) ~ds\Big]
    = i \theta \Pi_{r,x} [f(\xi_t)]
  \end{align}
  and
  \begin{align}
    U_{t-r}f(x) - \Pi_{r,x} \Big[\int_r^t e^{\int_r^s \rho_1(\xi_u)du} \psi_0\big(\xi_s,- U_{t-s}f\big) ~ds\Big]
    = i \Pi_{r,x} [e^{\int_r^t\rho_1(\xi_u)du}f(\xi_t)].
  \end{align}
  Taking $r = 0$, we get that \eqref{eq: complex FKPP-equation} and \eqref{eq: complex FKPP-equation with FK-transform} are true if $f\in \mathcal B_b(E)$.

  The rest of the proof is to evaluate \eqref{eq: complex FKPP-equation} and \eqref{eq: complex FKPP-equation with FK-transform} for all $f\in L_2(\xi)$. We only do this for \eqref{eq: complex FKPP-equation} since the argument for \eqref{eq: complex FKPP-equation with FK-transform} is similar.
  Let $n \in \mathbb N$.
  Writing $f_n := (f^+ \wedge n) - (f^- \wedge n)$, then $f_n \xrightarrow[n\to \infty]{} f$ pointwise.
  From what we have proved, we have
  \begin{align}
    \label{eq: complex FKPP-equation for fn}
    U_tf_n(x) - \Pi_{x} \Big[\int_0^t \psi\big(\xi_s, - U_{t-s}f_n\big) ~ds\Big]
    = i \Pi_{x} [f_n(\xi_t)].
  \end{align}
  Note that
  (i) $\Pi_{x}[f_n(\xi_t)] \xrightarrow[n\to \infty]{} \Pi_{x}[f(\xi_t)]$;
  (ii) by \eqref{eq: N and characteristic exponent}, the dominated convergence theorem and the fact that
  \[
    |e^{i W_t(f_n)} - 1| \leq \langle W_t, |f|\rangle;
    \quad \mathbb N_x[\langle W_t, |f|\rangle] = (P_t^{\rho_1} |f|)(x) < \infty,
  \]
  we have $U_tf_n(x) \xrightarrow[n\to \infty]{} U_tf(x)$, and (iii) by the dominated convergence theorem, \eqref{eq: domination of psi(v)} and the fact (see \eqref{eq: upper bound of psi(v)}) that
  \[
    big|\psi(\xi_s,- U_{t-s}f_n)\big|
    \leq C_1 P_{t-s}^{\rho_1}|f|(\xi_s)+C_2 P_{t-s}^{\rho_1}|f|(\xi_s)^2,
  \]
  we get that $\Pi_{x} [\int_0^t \psi(\xi_s,- U_{t-s}f_n)ds] \xrightarrow[n\to \infty]{} \Pi_{x} [\int_0^t \psi(\xi_s,- U_{t-s}f)ds]$.
  Using these, letting $n \to \infty$ in \eqref{eq: complex FKPP-equation for fn}, we get the desired result.
\end{proof}

\subsection*{Acknowledgment}
We thank Zenghu Li and Rui Zhang for helpful conversations.
We also thank the referee for very helpful comments.

\begin{thebibliography}{10}

\bibitem{AdamczakMilos2015CLT}
  R. Adamczak and P. Mi{\l}o\'{s}, \emph{C{LT} for {O}rnstein-{U}hlenbeck branching particle system},
  Electron. J. Probab. \textbf{20} (2015), no. 42, 35 pp.

\bibitem{Asmussen76Convergence}
  S. Asmussen, \emph{Convergence rates for branching processes},
  Ann. Probab.  \textbf{4} (1976), no. 1, 139--146.

\bibitem{AsmussenHering1983Branching}
  S. Asmussen and H. Hering, \emph{Branching processes},
  Progress in Probability and Statistics, 3. Birkh\"{a}user Boston, Inc., Boston, MA, 1983.

\bibitem{Athreya1969Limit}
  K. B. Athreya,
  \emph{Limit theorems for multitype continuous time {M}arkov branching processes. {I}. {T}he case of an eigenvector linear functional},
  Z. Wahrscheinlichkeitstheorie und Verw. Gebiete \textbf{12} (1969), 320--332.

\bibitem{Athreya1969LimitB}
  K. B. Athreya,
  \emph{Limit theorems for multitype continuous time {M}arkov branching processes. {II}. {T}he case of an arbitrary linear functional},
  Z. Wahrscheinlichkeitstheorie und Verw. Gebiete \textbf{13} (1969), 204--214.

\bibitem{Athreya1971Some}
  K. B. Athreya,
  \emph{Some refinements in the theory of supercritical multitype {M}arkov branching processes},
  Z. Wahrscheinlichkeitstheorie und Verw. Gebiete \textbf{20} (1971), 47--57.

\bibitem{ChenRenWang2008An-almost}
  Z.-Q. Chen, Y.-X. Ren, and H. Wang,
  \emph{An almost sure scaling limit theorem for {D}awson-{W}atanabe superprocesses},
  J. Funct. Anal. \textbf{254} (2008), no. 7, 1988--2019.

 \bibitem{ChenRenSongZhang2015Strong-law}
   Z.-Q. Chen, Y.-X. Ren, R. Song, and R. Zhang,
   \emph{Strong law of large numbers for supercritical superprocesses under second moment condition},
   Front. Math. China \textbf{10} (2015), no. 4, 807--838.

 \bibitem{ChenRenYang2019Skeleton}
   Z.-Q. Chen, Y.-X. Ren, and T. Yang,
   \emph{Skeleton decomposition and law of large numbers for supercritical superprocesses},
   Acta Appl. Math. \textbf{159} (2019), 225--285.

 \bibitem{Durrett2010Probability}
   R. Durrett,
   \emph{Probability: theory and examples},
   Fourth edition. Cambridge Series in Statistical and Probabilistic Mathematics, 31. Cambridge University Press, Cambridge, 2010.

\bibitem{Dynkin1993Superprocesses}
  E. B. Dynkin,
  \emph{Superprocesses and partial differential equations},
  Ann. Probab. \textbf{21} (1993), no. 3, 1185--1262.

\bibitem{EckhoffKyprianouWinkel2015Spines}
  M. Eckhoff, A. E. Kyprianou, and M. Winkel,
  \emph{Spines, skeletons and the strong law of large numbers for superdiffusions},
  Ann. Probab. \textbf{43} (2015), no. 5, 2545--2610.

\bibitem{Englander2009Law}
  J. Engl\"{a}nder,
  \emph{Law of large numbers for superdiffusions: the non-ergodic case},
  Ann. Inst. Henri Poincar\'{e} Probab. Stat. \textbf{45} (2009), no. 1, 1--6.

\bibitem{EnglanderWinter2006Law}
  J. Engl\"{a}nder and A. Winter,
  \emph{Law of large numbers for a class of superdiffusions},
  Ann. Inst. H. Poincar\'{e} Probab. Statist. \textbf{42} (2006), no. 2, 171--185.

\bibitem{EnglanderTuraev2002A-scaling}
  J. Engl\"{a}nder and  D. Turaev,
  \emph{A scaling limit theorem for a class of superdiffusions},
  Ann. Probab. \textbf{30} (2002), no. 2, 683--722.

\bibitem{Heyde1970A-rate}
  C. C. Heyde,
  \emph{A rate of convergence result for the super-critical {G}alton-{W}atson process},
  J. Appl. Probability \textbf{7} (1970), 451--454.

\bibitem{Heyde1971Some}
  C. C. Heyde,
    \emph{some central limit analogues for supercritical {G}alton-{W}atson processes},
  J. Appl. Probability \textbf{8} (1971), 52--59.

\bibitem{HeydeBrown1871An-invariance}
  C. C. Heyde and B. M. Brown,
  \emph{An invariance principle and some convergence rate results for branching processes},
  Z. Wahrscheinlichkeitstheorie und Verw. Gebiete, \textbf{20} (1971), 271--278.

\bibitem{HeydeLeslie1971Improved}
  C. C. Heyde and J. R. Leslie,
  \emph{Improved classical limit analogues for {G}alton-{W}atson processes with or without immigration},
  Bull. Austral. Math. Soc. \textbf{5} (1971), 145--155.

\bibitem{IksanovKoleskoMeiners2018Stable-like}
  A. Iksanov, K. Kolesko, and M. Meiners,
  \emph{Stable-like fluctuations of {B}gins' martingales},
  Stochastic Process. Appl. (2018).

\bibitem{Janson2004Functional}
  S. Janson,
  \emph{Functional limit theorems for multitype branching processes and generalized {P}\'{o}lya urns},
  Stochastic Process. Appl. \textbf{110} (2004), no. 2, 177--245.

\bibitem{KestenStigum1966Additional}
  H. Kesten and B. P. Stigum,
  \emph{Additional limit theorems for indecomposable multidimensional {G}alton-{W}atson processes},
  Ann. Math. Statist. \textbf{37} (1966), 1463--1481.

\bibitem{KestenStigum1966A-limit}
  H. Kesten and B. P. Stigum,
  \emph{A limit theorem for multidimensional {G}alton-{W}atson processes},
  Ann. Math. Statist. \textbf{37} (1966), 1211--1223.

\bibitem{KouritzinRen2014A-strong}
  M. A. Kouritzin, and Y.-X. Ren,
  \emph{A strong law of large numbers for super-stable processes},
  Stochastic Process. Appl. \textbf{121} (2014), no. 1, 505--521.

\bibitem{Kyprianou2014Fluctuations}
  A. E. Kyprianou,
  \emph{Fluctuations of {L}\'{e}vy processes with applications},
  % 2 ed., Universitext, Springer, Heidelberg, 2014.
  Introductory lectures. Second edition. Universitext. Springer, Heidelberg, 2014.

\bibitem{Li2011Measure-valued}
  Z. Li,
  \emph{Measure-valued branching {M}arkov processes},
  Probability and its Applications (New York). Springer, Heidelberg, 2011.

\bibitem{LiuRenSong2009Llog}
  % R.~Liu, Y.-X. Ren, and R.~Song,
  R.-L. Liu, Y.-X. Ren, and R. Song,
  \emph{{$L\log L$} criterion for a class of superdiffusions},
  J. Appl. Probab. \textbf{46} (2009), no. 2, 479--496.

\bibitem{LiuRenSong2013Strong}
  % R.~Liu, Y.-X. Ren, and R.~Song,
  R.-L. Liu, Y.-X. Ren, and R. Song,
  \emph{Strong law of large numbers for a class of superdiffusions},
  Acta Appl. Math. \textbf{123} (2013), 73--97.

\bibitem{MarksMilos2018CLT}
  R. Marks and P. Mi{\l}o{\'s},
  \emph{C{LT} for supercritical branching processes with heavy-tailed branching law},
  arXiv:1803.05491.

\bibitem{MetafunePallaraPriola2002Spectrum}
  G. Metafune, D. Pallara, and E. Priola,
  \emph{Spectrum of {O}rnstein-{U}hlenbeck operators in {$L^p$} spaces with respect to invariant  measures},
  J. Funct. Anal. \textbf{196} (2002), no. 1, 40--60.

\bibitem{Milos2012Spatial}
  P. Mi{\l}o{\'s},
  \emph{Spatial central limit theorem for supercritical superprocesses},
  J. Theoret. Probab. \textbf{31} (2018), no. 1, 1--40.

\bibitem{RenSongSun2017Spine}
  Y.-X. Ren, R. Song, and Z. Sun,
  \emph{Spine decompositions and limit theorems for a class of critical superprocesses},
  Acta Appl. Math. (2019).

\bibitem{RenSongSun2018Limit}
  Y.-X. Ren, R. Song, and Z. Sun,
  \emph{Limit theorems for a class of critical superprocesses with stable branching},
  arXiv:1807.02837.

\bibitem{RenSongZhang2014Central}
  Y.-X. Ren, R. Song, and R. Zhang,
  \emph{Central limit theorems for super {O}rnstein-{U}hlenbeck processes},
  Acta Appl. Math. \textbf{130} (2014), 9--49.

\bibitem{RenSongZhang2014CentralB}
  Y.-X. Ren, R. Song, and R. Zhang,
  \emph{Central limit theorems for supercritical branching {M}arkov processes},
  J. Funct. Anal. \textbf{266} (2014), no. 3, 1716--1756.

\bibitem{RenSongZhang2015Central}
  Y.-X. Ren, R. Song, and R. Zhang,
  \emph{Central limit theorems for supercritical superprocesses},
  Stochastic Process. Appl. \textbf{125} (2015), no. 2, 428--457.

\bibitem{RenSongZhang2017Central}
  Y.-X. Ren, R. Song, and R. Zhang,
  \emph{Central limit theorems for supercritical branching nonsymmetric {M}arkov processes},
  Ann. Probab. \textbf{45} (2017), no. 1, 564--623.

\bibitem{RenSongZhang2017Functional}
  Y.-X. Ren, R. Song, and R. Zhang,
  \emph{Functional central limit theorems for supercritical superprocesses},
  Acta Appl. Math. \textbf{147} (2017), 137--175.

\bibitem{Sato2013Levy}
  K. Sato,
  \emph{L{\'e}vy processes and infinitely divisible distributions},
  % Cambridge Studies in Advanced Mathematics, vol.~68, Cambridge University Press, Cambridge, 2013.
  Translated from the 1990 Japanese original. Revised by the author. Cambridge Studies in Advanced Mathematics, 68. Cambridge University Press, Cambridge, 1999.

\bibitem{SchillingSongVondravcek2010Bernstein}
  R. L. Schilling, R. Song, and Z. Vondra\v{c}ek,
  \emph{Bernstein functions.}
  Theory and applications. Second edition. De Gruyter Studies in Mathematics, 37. Walter de Gruyter \& Co., Berlin, 2012.

\bibitem{SteinShakarchi2003Complex}
  E. M. Stein and R. Shakarchi, \emph{Complex analysis},
  Princeton Lectures in Analysis, 2. Princeton University Press, Princeton, NJ, 2003.

\bibitem{Wang2010An-almost}
  L. Wang, \emph{An almost sure limit theorem for super-{B}rownian motion},
  J. Theoret. Probab. \textbf{23} (2010), no. 2, 401--416.

\end{thebibliography}
\end{document}
