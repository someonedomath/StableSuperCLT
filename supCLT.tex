% * The preamble
\documentclass[12pt,a4paper]{amsart}
\setlength{\textwidth}{\paperwidth}
\addtolength{\textwidth}{-2in}
\calclayout
\usepackage[utf8]{inputenc}
\usepackage[T1]{fontenc}
\usepackage{mathtools}
\mathtoolsset{showonlyrefs}
\usepackage{stackrel}
\usepackage{mathrsfs}
\usepackage{hyperref}
\usepackage{comment}
\usepackage{amsthm}
\theoremstyle{plain}
\newtheorem{thm}{Theorem}[section]
\newtheorem{lem}[thm]{Lemma}
\newtheorem{prop}[thm]{Proposition}
\newtheorem{cor}[thm]{Corollary}
\newtheorem{conj}[thm]{Conjecture}
\theoremstyle{definition}
\newtheorem{defi}[thm]{Definition}
\newtheorem{rem}[thm]{Remark}
\newtheorem{exa}[thm]{Example}
\newtheorem{asp}{Assumption}
\numberwithin{equation}{section}
\allowdisplaybreaks
% * Top matter
\begin{document}
\title
[stable CLT for super-OU processes]
{Stable Central Limit Theorems for Super Ornstein-Uhlenbeck Processes}
\author
[Y.-X. Ren, R. Song, Z. Sun and J. Zhao]
{Yan-Xia Ren, Renming Song, Zhenyao Sun and Jianjie Zhao}
\address{
  Yan-Xia Ren \\
  LMAM School of Mathematical Sciences \& Center for Statistical Science \\
  Peking University \\
  Beijing, P. R. China, 100871}
\email{yxren@math.pku.edu.cn}
\thanks{The research of Yan-Xia Ren is supported in part by NSFC (Grant Nos. 11671017  and 11731009) and LMEQF.}
\address{
  Renming Song \\
  Department of Mathematics \\ 
  University of Illinois at Urbana-Champaign \\ 
  % Urbana, IL 61801, USA}
  Urbana, IL, USA, 61801}
\email{rsong@illinois.edu}
\thanks{The Research of Renming Song is support in part by a grant from the Simons Foundation (\#429343, Renming Song)}
\address{
  Zhenyao Sun \\ 
  % School of Mathematical Sciences \\ 
  School of Mathematics and Statistics\\ 
  % Peking University \\ 
  Wuhan University \\ 
  % Beijing, P. R. China, 100871}
  Hubei, P. R. China, 100871}
% \email{zhenyao.sun@pku.edu.cn}
\email{zhenyao.sun@gmail.com}
\address{
  Jianjie Zhao \\ 
  School of Mathematical Sciences \\ 
  Peking University \\ 
  Beijing, P. R. China, 100871}
\email{zhaojianjie@pku.edu.cn}
\thanks{Jianjie Zhao is the corresponding author}
\begin{abstract}
  Let $\xi$ be an Ornstein-Uhlenbeck process on $\mathbb R^d$ with generator $L = \frac{1}{2}\sigma^2\Delta - b x \cdot \nabla$, where $\sigma, b >0$. 
  Let $\psi$ be a branching mechanism which is close to a function of the form $\widetilde{\psi}(z)=-\alpha z +\rho z^2+ \eta z^{1+\beta}$ with $\alpha>0$, $\rho\ge 0$, $\eta>0$ and $\beta\in (0, 1)$, in some sense. 
  In this paper, we study asymptotic behaviors of $(\xi, \psi)$-superprocesses $(X_t)_{t\geq 0}$. 
  For any testing function $f$ of polynomial growth, denote by $\kappa_f$ the order of $f$ in the spectral decomposition of $f$ in terms of the spectrum of the mean semigroup of $X$. 
  Conditioned on non-extinction, we establish some stable central limit theorems for $\langle f, X_t \rangle$ in three different regimes: 
  the small branching rate regime $\alpha\beta< \kappa_f b(1+\beta)$; 
  the critical branching rate regime $\alpha\beta = \kappa_f b(1+\beta)$; 
  and the large branching rate regime $\alpha\beta > \kappa_f b(1+\beta)$.
\end{abstract}
\subjclass[2010]{60J68, 60F05}
\keywords{Superprocesses, Ornstein-Uhlenbeck processes, Stable distribution, Central limit theorem, Law of large numbers, Branching rate regime}
\maketitle
% * Contents
\section{Introduction}
% ** Motivation
\subsection{Motivation}
\label{sec: Motivation}
Let $d \in \mathbb N:= \{1,2,\dots\}$ and $\mathbb R_+:= [0,\infty)$.
Let $\xi=\{(\xi_t)_{t\geq 0}; (\Pi_x)_{x\in \mathbb R^d}\}$ be an $\mathbb R^d$-valued Ornstein-Uhlenbeck process (OU process) with generator
\begin{align}
  Lf(x)
  = \frac{1}{2}\sigma^2\Delta f(x)-b x \cdot \nabla f(x)
  , \quad  x\in \mathbb R^d, f \in C^2(\mathbb R^d),
\end{align}
where $\sigma > 0$ and $b > 0$ are constants.
Let $\psi$ be a function on $\mathbb R_+$ of the form 
\begin{align} 
  \label{eq: honogeneou branching mechanism}
  \psi(z)=
  - \alpha z + \rho z^2 + \int_{(0,\infty)} (e^{-zy} - 1 + zy) \pi(dy)
  , \quad  z \in \mathbb R_+,
\end{align}
where $\alpha > 0 $, $\rho \geq0$ and $\pi$ is a measure on $(0,\infty)$ with $\int_{(0,\infty)}(y\wedge y^2) \pi(dy)< \infty$.
$\psi$ is referred to as a branching mechanism and $\pi$ is referred to as the L\'evy measure of $\psi$.
Denote by $\mathcal M(\mathbb R^d)$ the space of all finite Borel measures on $\mathbb R^d$.
For each $f,g\in \mathcal B(\mathbb R^d, \mathbb R)$ and $\mu \in \mathcal M(\mathbb R^d)$, write $\langle f,\mu\rangle = \int f(x)\mu(dx)$ and $\langle f, g\rangle = \int f(x)g(x) dx$ whenever the integrals make sense.
We say a real-valued Borel function $f:(t,x)\mapsto f(t,x)$ on $\mathbb R_+\times \mathbb R^d$ is \emph{locally bounded} if, for each $t\in \mathbb R_+$, we have $ \sup_{s\in [0,t],x\in \mathbb R^d} |f(s,x)|<\infty. $
We say that an $\mathcal M(\mathbb R^d)$-valued Hunt process $X = \{(X_t)_{t\geq 0}; (\mathbb{P}_{\mu})_{\mu \in \mathcal M(\mathbb R^d)}\}$ on a measurable space $(\Omega, \mathcal{F})$ is a \emph{super Ornstein-Uhlenbeck process (super-OU process)} with branching mechanism $\psi$, or a $(\xi, \psi)$ superprocess, if for each non-negative bounded Borel function $f$ on $\mathbb R^d$, we have
\begin{align}
  \label{eq: def of V_t}
  \mathbb{P}_{\mu}[e^{-\langle f,X_t \rangle}]
  = e^{-\langle V_tf, \mu \rangle}
  , \quad t\geq 0, \mu \in \mathcal M(\mathbb R^d),
\end{align}
where $(t,x) \mapsto V_tf(x)$ is the unique locally bounded positive solution to the equation
\begin{align}
  V_tf(x) + \Pi_x \Big[ \int_0^t\psi\big(V_{t-s}f(\xi_s)\big)~ds\Big]
	= \Pi_x [f(\xi_t)]
  , \quad x\in \mathbb R^d, t\geq 0.
\end{align}	
The existence of such super-OU process $X$ is well known, see \cite{Dynkin1993Superprocesses} for instance.

Recently, there have been quite a few papers on laws of large numbers for superdiffusions. 
In \cite{Englander2009Law, EnglanderWinter2006Law, EnglanderTuraev2002A-scaling}, some weak laws of large numbers (convergence in law or in probability) were established. 
The strong law of large numbers for superprocesses was first studied in \cite{ChenRenWang2008An-almost}, followed by \cite{ChenRenSongZhang2015Strong-law, ChenRenYang2019Skeleton, EckhoffKyprianouWinkel2015Spines, KouritzinRen2014A-strong, LiuRenSong2013Strong, Wang2010An-almost} under different settings.
For a good survey on recent developments in laws of large numbers for branching Markov processes and superprocesses, see \cite{EckhoffKyprianouWinkel2015Spines}.

The strong law of large numbers for the super-OU process $X$ can be stated as follows:
Under some conditions on the branching mechanism $\psi$ (these conditions are satisfied under our Assumptions 1 and 2 below), there exists an $\Omega_0$ of $\mathbb{P}_\mu$-full probability for every $\mu\in\mathcal M(\mathbb R^d)$ such that on $\Omega_0$, for every Lebesgue-a.\/e. continuous bounded non-negative function $f$ on $\mathbb R^d$, we have $\lim_{t\to\infty} e^{-\alpha t} \langle f, X_t\rangle =H_\infty\langle f, \varphi\rangle $, where $H_\infty$ is the limit of the martingale $e^{-\alpha t}\langle 1,X_t\rangle$ and $\varphi$ is the invariant density of the OU process $\xi$ defined in \eqref{invariantdensity} below.
See \cite[Theorem 2.13 \& Example 8.1]{ChenRenYang2019Skeleton} and \cite[Theorem 1.2 \& Example 4.1]{EckhoffKyprianouWinkel2015Spines}.

In this paper, we will establish some spatial central limit theorems (CLT) for the super-OU process $X$ above.
Our key assumption is that the branching mechanism $\psi(z)$ takes the form of $-\alpha z +\rho z^2+ \psi_1(z)$, where $\psi_1$ is close to $\eta z^{1+\beta}$ with $\eta>0$ and $\beta\in (0, 1)$, in some sense.
We want to find $(F_t)_{t\geq 0}$ and $(G_t)_{t\geq 0}$ such that $ (\langle f, X_t \rangle -G_t)/F_t $ converges weakly to some non-degenerate random variable as $t\rightarrow\infty$, for a large class of functions $f$.
It turns out that the statements of the CLT's are different in three regimes depending on the sign of $\alpha\beta-\kappa_f b (1+\beta)$: 
the small branching rate regime $\alpha\beta< \kappa_f b(1+\beta)$; 
the critical branching rate regime $\alpha\beta = \kappa_f b(1+\beta)$; 
and the large branching rate regime $\alpha\beta > \kappa_f b(1+\beta)$. 
Here, $\kappa_f$ is the order of $f$ in the spectral decomposition in terms of the spectrum of $L$.
Note that, in the setting of this paper, $\langle f,X_t\rangle$ typically have infinite second moment.

There are many papers studying central limit theorems for branching processes, branching diffusions and superprocesses under the second moment condition.
See \cite{Heyde1970A-rate, HeydeBrown1871An-invariance, HeydeLeslie1971Improved} for supercritical Galton-Watson processes (GW processes), \cite{KestenStigum1966Additional,KestenStigum1966A-limit} for supercritical multi-type GW processes \cite{Athreya1969Limit,Athreya1969LimitB,Athreya1971Some} for supercritical multi-type continuous time branching processes and \cite{AsmussenHering1983Branching} for general supercritical branching Markov processes under certain conditions.
Some spatial central limit theorems for   supercritical branching OU processes with binary branching mechanism were proved in \cite{AdamczakMilos2015CLT} and some spatial central limit theorems for supercritical super-OU processes with branching mechanisms satisfying a fourth moment condition were proved in \cite{Milos2012Spatial}.
These two papers made connections between the CLT and  the branching rate regimes.
Some spatial central limit theorems for supercritical super-OU  processes with branching mechanisms satisfying only a second moment condition were established in \cite{RenSongZhang2014Central}.
Moreover, compared with the results of \cite{AdamczakMilos2015CLT,Milos2012Spatial}, the limit distributions in \cite{RenSongZhang2014Central} are non-degenerate.
Since then, a series of spatial central limit theorems for a large class of general supercritical branching Markov processes and superprocesses with spatially dependent branching mechanisms were proved in \cite{RenSongZhang2014CentralB,RenSongZhang2015Central,RenSongZhang2017Central}.
The functional version of the central limit theorems was established in \cite{Janson2004Functional} for supercritical multitype branching processes, and  in \cite{RenSongZhang2017Functional} for supercritical superprocesses.

There are also many limiting theorem type results for supercritical branching processes and branching Markov processes with branching mechanisms of infinite second moments.
Heyde \cite{Heyde1971Some} established a central limit type  theorem for supercritical GW processes when the offspring distribution belongs to the domain of attraction of a stable law of index $\alpha\in (1, 2]$, and proved that the limit law is  a stable law. 
Similar results  for supercritical multi-type GW processes and supercritical  continuous time branching processes, 
under some $p$-th ($p\in(1,2]$) moment condition on the offspring distribution, were given in Asmussen \cite{Asmussen76Convergence}.
Recently, Marks and Milo\'s \cite{MarksMilos2018CLT} considered the limit behavior of supercritical branching OU processes with a special stable branching law.
They established some spatial central limit theorems in the small and critical branching rate regimes, but they did not prove any central limit theorem type result in the large branching rate regime.
We also mention here that very recently \cite{IksanovKoleskoMeiners2018Stable-like} considered stable fluctuations of Biggins' martingales in the context of branching random walks and \cite{RenSongSun2018Limit} considered the asymptotic behavior of a class of critical superprocesses with spatially dependent stable branching.

As far as we know, this paper is the first to study spatial central limit theorems for supercritical superprocesses without the second moment condition.

% ** Main results
\subsection{Main results}
\label{sec: main results}{}
We will always assume that the following assumption holds.
\begin{asp}
  \label{asp: Greys condition}
  The branching mechanism satisfies Grey's condition, i.e., there exists $z' > 0$ such that $\psi(z) > 0$ for all $z>z'$ and  $\int_{z'}^\infty \psi(z)^{-1}dz < \infty$.
\end{asp}
For each $\mu \in \mathcal M(\mathbb R^d)$, write $\|\mu\| = \langle 1, \mu\rangle$.
It is known (see \cite[Theorems 12.5 \& 12.7]{Kyprianou2014Fluctuations} for example) that, under Assumption \ref{asp: Greys condition}, the \emph{extinction event} $D :=\{\exists t\geq 0,~\text{s.t.}~ \|X_t\| =0 \}$ has positive probability, with respect to $\mathbb P_\mu$ for each  $\mu \in \mathcal M(\mathbb R^d)$.
In fact, $ \mathbb{P}_{\mu} (D) = e^{-\bar v \|\mu\|}, $ for $\mu\in \mathcal M(\mathbb R^d)$, where $ \bar v := \sup\{\lambda \geq 0: \psi(\lambda) = 0\} \in (0,\infty) $ is the largest root of $\psi$.

Denote by $\Gamma$ the gamma function.
For any $\sigma$-finite signed measure $\mu$, we use $|\mu|$ to denote the total variation measure of $\mu$.
In this paper, we will also assume the following:
\begin{asp}
  \label{asp: branching mechanism}
  There exist constants $\eta > 0$ and $\beta \in (0,1)$ such that
  \begin{align}
    \label{eq: asp of branching mechanism}
    \int_{(1,\infty)}y^{1+\beta +\delta}~\Big|\pi(dy)-\frac{\eta~dy}{\Gamma(-1-\beta)y^{2+\beta}}\Big| <\infty,
  \end{align}
	for some $\delta > 0$.
\end{asp}
We will show in Subsection \ref{sec: branching mechanism} that if Assumption \ref{asp: branching mechanism} holds, then $\eta$ and $\beta$ are uniquely determined by the L\'evy measure $\pi$.
In the reminder of the paper, we will always use $\eta$ and $\beta$ to denote the constants in Assumption  \ref{asp: branching mechanism}.

\begin{rem}
  \label{rem: small enough delta}
	Note that $\delta$ is not uniquely determined by $\pi$.
	In fact, if $\delta>0$ is a constant such that \eqref{eq: asp of branching mechanism} holds, then replacing $\delta$ by any smaller positive number, \eqref{eq: asp of branching mechanism} still holds.
	Therefore, Assumption \ref{asp: branching mechanism} is equivalent to the following statement: 
	There exist constants $\eta > 0$ and $\beta \in (0,1)$ such that, for each small enough $\delta>0$, \eqref{eq: asp of branching mechanism} holds.
\end{rem}

Assumption \ref{asp: branching mechanism} says that there exist constants $\eta>0$ and $\beta > 0$ such that the L\'evy measure $\pi(dy)$ is not too far away from the measure $\eta \Gamma(-1-\beta)^{-1}y^{-2-\beta} dy$.
In particular, if $\pi(dy)$ is equal to $\eta \Gamma(-1-\beta)^{-1}y^{-2-\beta} dy$, then the branching mechanism $\psi(z)$ takes the form of $-\alpha z + \rho z^2 + \eta z^{1+\beta}$.
It will be proved in Lemma \ref{lem: LlogL criterion} that, under Assumption \ref{asp: branching mechanism}, the branching mechanism $\psi$ satisfies the $L \log L$ condition, i.e., $ \int_{(1,\infty)} y\log y~\pi(dy) < \infty. $
This guarantees that $H_\infty$, the limit of the non-negative martingale $(e^{-\alpha t} \|X_t\|)_{t\geq 0}$, is non-degenerate.

Denote by $\mathcal B(\mathbb R^d, \mathbb R)$ the space of all $\mathbb R$-valued Borel functions on $\mathbb R^d$.
Denote by $\mathcal B(\mathbb R^d, \mathbb R_+)$ the space of all $\mathbb R_+$-valued Borel functions on $\mathbb R^d$.
We use  $(P_t)_{t\geq 0}$ to denote the transition semigroup of $\xi$.	
Define
\begin{align}
  \label{eq: meansemigroup}
  P^{\alpha}_t f(x)
  :=
  e^{\alpha t} P_t f(x) 
  = \Pi_x [e^{\alpha t}f(\xi_t)]
  , \quad x\in \mathbb R^d,t\geq 0, f\in \mathcal B(\mathbb R^d, \mathbb R_+).
\end{align}
It is known that, see \cite[Proposition 2.27]{Li2011Measure-valued} for example, $(P^\alpha_t)_{t\geq 0}$ is the \emph{mean semigroup} of $X$, in the sense that
\begin{equation}
  \mathbb{P}_{\mu}[\langle f, X_t \rangle]
  = \langle P^\alpha_t f, \mu \rangle,
  \quad t\geq 0, f\in \mathcal B(\mathbb R^d, \mathbb R_+), \mu \in \mathcal M(\mathbb R^d).
\end{equation}
The limiting behavior of the super-OU process is closely related to the asymptotic property of this mean semigroup $(P^\alpha_t)_{t\geq 0}$, and therefore, to the property of the OU semigroup $(P_t)_{t\geq 0}$.

We now recall some basic spectral properties of $(P_t)_{t\geq 0}$ from \cite{MetafunePallaraPriola2002Spectrum}.
It is known that the OU process $\xi$ has an invariant density
\begin{align}
\label{invariantdensity}
  \varphi(x)dx
  :=\Big (\frac{b}{\pi \sigma^2}\Big )^{d/2}\exp \Big(-\frac{b}{\sigma^2}|x|^2 \Big)dx,
  \quad x\in \mathbb R^d.
\end{align}
Let $L^2(\varphi):= \left\{ h  \in \mathcal B(\mathbb R^d, \mathbb R): \int_{\mathbb R^d} |h(x)|^2 \varphi(x) dx < \infty \right\}$.
Then $L^2(\varphi)$ is a Hilbert space with inner product
\begin{align}
  \langle f_1, f_2 \rangle_{\varphi}
  := \int_{\mathbb R^d}f_1(x)f_2(x)\varphi(x) dx, \quad f_1,f_2 \in L^2(\varphi).
\end{align}
Let $\mathbb Z_+ := \mathbb N\cup\{0\}$.
For each $p = (p_k)_{k = 1}^d \in \mathbb{Z}_+^{d}$, write $|p|:=\sum_{k=1}^d p_k$, $p!:= \prod_{k= 1}^d p_k!$ and $\partial x^p:= \prod_{k = 1}^d\partial x_k^{p_k}$.
The \emph{Hermite polynomials} are defined by
\begin{align}
  H_p(x)
  :=(-1)^{|p|}\exp(|x|^2) \frac{\partial ^{|p|}}{\partial x^p} \exp(-|x|^2) 
  , \quad x\in \mathbb R^d, p \in \mathbb{Z}_+^{d}.
\end{align}
It is known that $(P_t)_{t\geq 0}$ is a strongly continuous semigroup in $L^2(\varphi)$ and its generator $L$ has discrete spectrum $\sigma(L)= \{-bk: k \in \mathbb Z_+\}$.
For each $k \in \mathbb Z_+$, denote by $\mathcal{A}_k$ the eigenspace corresponding to the eigenvalue $-bk$, then
$
\mathcal{A}_k
= \operatorname{Span} \{\phi_p : p\in \mathbb Z_+^d, |p|=k\},
$
where
\begin{align}
  \label{eigenfunction}
  \phi_p(x)
  := \frac{1}{\sqrt{ p! 2^{|p|} }} H_p \Big(\frac{ \sqrt{b} }{\sigma}x \Big)
  , \quad x\in \mathbb R^d, p\in \mathbb Z_+^d.
\end{align}
In other words,
\begin{equation}
  P_t\phi_p(x)
  =e^{-b|p|t}\phi_p(x),
  \quad t\geq 0, x\in \mathbb R^d, p\in \mathbb Z_+^d.
\end{equation}
Moreover, $\{\phi_p: p \in \mathbb Z_+^d\}$ forms a complete orthonormal basis of $L^2(\varphi)$.
Thus for each $f\in L^2(\varphi)$, we have
\begin{equation}
\label{semicomp1}
f
=\sum_{k=0}^{\infty}\sum_{p\in \mathbb Z_+^d:|p|=k}\langle f, \phi_p \rangle_{\varphi} \phi_p
, \quad \text{in~} L^2(\varphi).
\end{equation}
For each function $f\in L^2(\varphi)$, denote by
\begin{equation}
  \kappa_f
  :=\inf \left \{k\geq 0: \exists ~ p\in \mathbb Z_+^d ,{\rm ~s.t.~}|p|=k {\rm ~and~}  \langle f, \phi_p \rangle_{\varphi}\neq 0\right \},
\end{equation}
the order of the function $f$ in the spectral decomposition \eqref{semicomp1}.
$\kappa_f$ corresponds to the lowest non-trivial frequency in the eigen-expansion of $f$.
Note that $ \kappa_f\geq 0$ and that, if $f\in L^2(\varphi)$ is non-trivial, then $\kappa_f<\infty$.
In particular, the order of any constant non-zero function is zero.
Denote by $\mathcal P$ the class of functions of polynomial growth on $\mathbb R^d$, i.e.,
\begin{equation}
  \label{eq: polynomial growth function}
  \mathcal{P}
  :=\big \{f\in \mathscr B(\mathbb R^d, \mathbb R):\exists~ C>0, n \in \mathbb Z_+, {\rm such  \, \, that \, } \forall x\in \mathbb R^d,~ |f(x)|\leq C(1+|x|)^n\big\}.
\end{equation}
It is clear that $\mathcal{P} \subset L^2(\varphi)$.
In this paper, we are mainly interested in the asymptotic behavior of $\langle f, X_t\rangle$ for non-trivial $f\in \mathcal P$.
The asymptotic behavior varies with the following three regimes:
(1) the small branching rate regime where $\alpha\beta<\kappa_fb(1+\beta)$; 
(2) the critical branching rate regime where $\alpha\beta=\kappa_fb(1+\beta)$; and
(3) the large branching rate regime where $\alpha\beta>\kappa_fb(1+\beta)$.

% *** Large branching rate regime
\subsubsection{Large branching rate regime}
Denote by $\mathcal M_c(\mathbb R^d)$ the space of all finite Borel measures of compact support on $\mathbb R^d$.
For each $p\in \mathbb{Z}_+^d$, define
\[
  H_t^p
  := e^{-(\alpha-|p|b)t}\langle\phi_p,X_t\rangle,
  \quad t\geq 0.
\]
If $\alpha\beta>|p|b(1+\beta)$, then for all $\gamma\in (0, \beta)$ and $\mu\in \mathcal M_c(\mathbb R^d)$, we will prove in Lemma \ref{lemma26} that $(H_t^p)_{t\geq 0}$ is a $\mathbb{P}_{\mu}$-martingale bounded in $L^{1+\gamma}(\mathbb{P}_{\mu})$.
Thus the limit $H^p_{\infty}:=\lim_{t\rightarrow \infty}H_t^p$ exists $\mathbb{P}_{\mu}$-almost surely and in $L^{1+\gamma}(\mathbb{P}_{\mu})$.
\begin{thm}
  \label{thm: law of large number}
  If $f \in \mathcal{P}$ satisfies $\alpha\beta>\kappa_fb(1+\beta)$, then for all $\gamma\in (0, \beta)$ and  $\mu\in \mathcal M_c(\mathbb R^d)$,
  \[
    e^{-(\alpha-\kappa_fb)t}\langle f, X_t\rangle
    \xrightarrow[t\to \infty]{}\sum_{p\in \mathbb Z_+^d:|p|=\kappa_f}\langle f, \phi_p\rangle_{\varphi} H_{\infty}^p
    \quad in~ L^{1+\gamma}(\mathbb{P}_{\mu}).
  \]
  Moreover, if $f$ is twice differentiable and all its second order partial derivatives are in $\mathcal{P}$, then we also have almost sure convergence.
\end{thm}
\begin{rem}
  If $f\in \mathcal B(\mathbb R^d, \mathbb R_+)$ is non-trivial and  bounded, then $\kappa_f=0$.
  Hence, Theorem \ref{thm: law of large number} says that for any $\gamma\in (0, \beta)$ and  $\mu\in \mathcal M_c(\mathbb R^d)$, as $t\rightarrow \infty$,
  \[
    e^{-\alpha t}\langle f, X_t\rangle
    \rightarrow \langle f, \varphi\rangle H_{\infty}
    \quad \mbox{in } L^{1+\gamma}(\mathbb{P}_{\mu}).
  \]
  Moreover, if $f$ is twice differentiable and all its second order partial derivatives are in $\mathcal{P}$, then we also have a.\/s.\ convergence.
  However, to get  a.\/s.\ convergence for bounded non-negative 
  Lebesgue-a.\/e.\/\ continuous functions $f$, we do not need $f$ to be twice differentiable. See \cite[Theorem 2.13 \& Example 8.1]{ChenRenYang2019Skeleton} and \cite[Theorem 1.2 \& Example 4.1]{EckhoffKyprianouWinkel2015Spines}.
\end{rem}
\begin{rem}
  For general $f\in \mathcal{P}$ with $\alpha\beta>\kappa_fb(1+\beta)$, Theorem \ref{thm: law of large number} extends the strong laws of large numbers of \cite{ChenRenYang2019Skeleton, EckhoffKyprianouWinkel2015Spines} in which the first order asymptotic ($\kappa_f=0$) was identified.
\end{rem}

The fluctuation result associated to the law of large numbers in this case is given in Theorem \ref{thm: large clt} below.
Define
\begin{align}
  \label{eq: def of N}
  \mathcal{N}
  :=\{p\in \mathbb{Z}_+^d: \alpha\beta>|p|(1+\beta)b\},
  \quad K : = \sup\{|p|: p \in \mathcal N\}
\end{align}
and
\begin{equation}
  \label{eq: def of Cl}
  \mathcal{C}_l
  :=\Big\{f\in \mathcal P: f(x)=\sum_{p\in\mathcal{N}}\langle f, \phi_p\rangle_{\varphi}\phi_p(x)\Big\}.
\end{equation}
For each $t\ge 0$, we define an operator $I_t$ on $\mathcal{C}_l$ by
\begin{align}
  \label{definition of Itf}
  I_t f(x)
  := \sum_{p\in \mathcal{N}}\langle f, \phi_p\rangle_{\varphi} e^{-(\alpha-|p|b)t}\phi_p(x)
  , \quad x\in \mathbb{R}^d, f\in \mathcal C_l.
\end{align}
It is easy to see that $I_sf(x)=\mathbb{P}_{\delta_x}[\langle I_t f, X_{t-s}\rangle]=P_{t-s}^{\alpha}I_tf(x)$.
Define
\begin{align}
  \label{bar-m}
  \bar{m}[f]
  :=\eta \int_{0}^{\infty} e^{\alpha s}~ds \int_{\mathbb R^d} \big(iI_sf(x)\big)^{1+\beta}\varphi(x)~dx,
  \quad f\in \mathcal C_l.
\end{align}
We will show in Lemma \ref{lem: def of all m} that $\bar{m}[f]$ is well-defined for each $f\in \mathcal C_l$; 
and in Lemma \ref{prop: alpha stable rv} that, when $f\in \mathcal C_l$ is non-trivial, $\theta \mapsto \exp( \bar m[\theta f])$ is the characteristic function of a $(1+\beta)$-stable random variable.
\begin{thm}
  \label{thm: large clt}
  For non-trivial $f\in\mathcal{C}_l$ and $\mu\in \mathcal{M}_c(\mathbb{R}^d)$, under $\mathbb{P}_{\mu}(\cdot|D^c)$, it holds that,
  \begin{align}
    \label{thm: large rate}
    \frac{\langle f, X_t\rangle-\sum_{p\in\mathcal{N}}\langle f,\phi_p\rangle_\varphi e^{(\alpha-|p|b)t}H^p_{\infty}}{\|X_t\|^\frac{1}{1+\beta}}
    \xrightarrow[t\to \infty]{d}\bar{\zeta},
  \end{align}
  where $\bar{\zeta}$ is a $(1+\beta)$-stable random variable with characteristic function $\theta\mapsto \exp(\bar{m}[\theta f])$.
\end{thm}

% *** Critical branching rate regime
\subsubsection{Critical branching rate regime}
For each $f\in \mathcal{P}$ satisfying $\alpha\beta=\kappa_f b(1+\beta)$, define
\begin{equation}
  \label{tilde-m}
  \widetilde{m}[f]
  := \eta\int_{\mathbb R^d} \Big(-i\sum_{p\in \mathbb Z_+^d:|p|=\kappa_f}\langle f,\phi_p\rangle\phi_p(x)\Big)^{1+\beta} \varphi(x)~dx.
\end{equation}
According to Lemma \ref{lem: def of all m} below, $\widetilde m[f]$ is well-defined.
We will prove in Lemma \ref{prop: alpha stable rv} that $\theta \mapsto \exp( \widetilde m[\theta f])$ is the characteristic function of a $(1+\beta)$-stable random variable.
\begin{thm}
  \label{thm: critical clt}
  If $f\in\mathcal{P}$ satisfies  $\alpha\beta=\kappa_fb(1+\beta)$, then for any $\mu\in \mathcal M_c(\mathbb R^d)$, under $\mathbb{P}_{\mu}(\cdot|D^c)$, it holds that
  \[
    \frac{\langle f,X_t\rangle}{\left(t\|X_t\|\right)^{\frac{1}{1+\beta}}}
    \xrightarrow[t\to \infty]{d} \widetilde{\zeta},
  \]
  where $\widetilde{\zeta}$ is a $(1+\beta)$-stable random variable with characteristic function $\theta\mapsto \exp(\widetilde{m}[\theta f])$.
\end{thm}

% *** Small branching rate regime
\subsubsection{Small branching rate regime}
\label{msmallcase}
If $f\in \mathcal{P}$ is non-trivial and satisfies  $\alpha\beta<\kappa_f b(1+\beta)$, we define
\begin{equation}
  m[f]
  := \eta \int_0^{\infty} e^{-\alpha s} ~ds\int_{\mathbb R^d} \big(-iP_s^\alpha f(x)\big)^{1+\beta} \varphi(x)~dx.
\end{equation}
According to Lemma \ref{lem: def of all m} below, $m[f]$ is well-defined.
We will prove  in Lemma \ref{prop: alpha stable rv} that $\theta \mapsto \exp( m[\theta f])$ is the characteristic function of a $(1+\beta)$-stable random variable.
\begin{thm}
  \label{thm: small clt}
  If $f\in\mathcal{P}$ is non-trivial and satisfies $\alpha\beta<\kappa_f b(1+\beta)$, then for any $\mu\in \mathcal M_c(\mathbb R^d)$, under $\mathbb{P}_{\mu}(\cdot|D^c)$, it holds that
  \[
    \frac{\langle f,X_t\rangle}{\|X_t\|^{\frac{1}{1+\beta}}}\xrightarrow[t\rightarrow \infty]{d} \zeta,
  \]
  where $\zeta$ is a $(1+\beta)$-stable random variable with characteristic function $\theta\mapsto \exp(m[\theta f])$.
\end{thm}

% ** Some intuitive explanation
\subsection{Some intuitive explanation}
Here we give some intuitive explanation of the central limit theorems above.
The main ideas are similar to those given in \cite{MarksMilos2018CLT} for branching OU processes.
For any $\mu\in \mathcal M_c(\mathbb R^d)$ and any random variable $Y$ with finite mean, we define
\begin{equation}
  \label{Ist}
  \mathcal I_s^t Y
  := \mathcal I_s^t [Y, \mu]
  := \mathbb P_\mu[Y|\mathscr F_t] - \mathbb P_\mu[Y|\mathscr F_s],
  \quad 0 \leq s \leq t <\infty.
\end{equation}
We will use the shorter notation $\mathcal I_s^t Y$ when there is no danger of confusion.
For each $f\in \mathcal{P}$, consider the following decomposition over the time interval $[0,t]$:
\begin{align}
  \langle f,X_t\rangle
  := \sum_{k=0}^{\lfloor t \rfloor-1} \mathcal I_{t-k-1}^{t-k}\langle f ,X_t\rangle+\mathcal I_0^{t-\lfloor t \rfloor}\langle f ,X_t\rangle + \langle P^\alpha_tf,X_0\rangle,
  \quad t\geq 0.
\end{align}
To find the fluctuation of $\langle f,X_t\rangle$, we will investigate the fluctuation of each term on the right hand side above.

Suppose $(\zeta_k)_{k \in \mathbb N}$ are independent $(1+\beta)$-stable random variables with characteristic functions $\theta\mapsto \exp( m_k[\theta f])$, where $m_k[f]$ is given in Lemma \ref{lem: def of all m} below.
Recall $\|X_t\|\sim e^{\alpha t}$ as $t\to\infty$.
Lemma \ref{lem: mainlemma} and Proposition \ref{cor: indepedent of the limit zeta for critical and small} below imply that if $\alpha\beta \leq \kappa_f b(1+\beta) $ then
\[
  \bigg(\frac{\mathcal I^{t-k}_{t-k-1} \langle f,X_t\rangle}{\|X_t\|^{\frac{1}{1+\beta}}} \bigg)_{k=1}^n
  \xrightarrow [t\to \infty]{d} (\zeta_k)_{k=1}^n.
\]
So if $f$ is in the small branching rate regime, i.e., $\alpha \beta < \kappa_f b(1+\beta)$, we have roughly that
\[
  \frac{\langle f,X_t\rangle}{\|X_t\|^{\frac{1}{1+\beta}}} 
  \xrightarrow[t\to \infty]{d} \zeta\overset{d}{=}\sum_{k=0}^\infty \zeta_k,
\]
where $\zeta$ is a $(1+\beta)$-stable random variable with characteristic function $\theta\mapsto \exp(m[\theta f])$, with $m[\theta f]$ given in Lemma \ref{lem: def of all m} below.
In the explanation above, we have used the facts that $(\zeta_k)_{k\in \mathbb N}$, $\zeta$ are all well defined, and
\begin{align}
  \label{eq: equatlity for mf for small rate}
  m[f] 
  = \sum_{k=0}^\infty m_k[ f].
\end{align}
These facts will be made clear in Subsection \ref{sec: stable distributions} below.

If $f$ is in the critical branching rate regime, i.e., $\alpha \beta = \kappa_f b(1+\beta)$, then the right hand side of \eqref{eq: equatlity for mf for small rate} may not converge.
Instead, see Lemma \ref{lem: def of all m} below, there is another quantity $\widetilde m[f]$ satisfying that
\[
  \widetilde{m}[f] 
  = \lim_{t\rightarrow \infty}\frac{1}{t}\sum_{k=0}^{\lfloor t \rfloor}m_k[f].
\]
In this case, we roughly have that
\[
 	\frac{\langle f,X_t\rangle}{(t\|X_t\|)^{\frac{1}{1+\beta}}} 
  \overset{d}{\approx} \frac{1}{t^{\frac{1}{1+\beta}}} \sum_{k=0}^{\lfloor t\rfloor} \zeta_k
  \xrightarrow[t\to \infty]{d} \widetilde \zeta
\]
where $\widetilde \zeta$ is a $(1+\beta)$-stable random variable with characteristic function $\theta\mapsto \exp(\widetilde m[\theta f])$.

For $f\in \mathcal C_l$, the general idea is almost the same, except that we need to consider the decomposition over the time interval $[t,\infty)$. 
Taking $f = \phi_p$ with $\alpha \beta > |p|b(1+\beta)$ as an example, we do the following decomposition:
\[
  H^p_t-H^p_\infty
  =\sum^{\infty}_{n=1}(H^p_{t+n-1}-H^p_{t+n}).
\]
The fluctuation behaviors of each of the terms $H_{t+n-1}^p - H_{t+n}^p$ and their asymptotic independence will be established in Lemma \ref{large-central} and \ref{lem: independency for large rate} below, respectively.
These lemmas will eventually lead us to the fluctuation result in the large branching rate regime.

For a general $f$, we have a unique decomposition: $f=f_l+f_c+f_s$, where 
\[
  f_l
  = \sum_{p\in \mathcal{N}}a_p\phi_p(x)\in \mathcal{C}_l
  , \quad x\in \mathbb{R}^d,
\]
and $f_c=\langle f, \phi_p\rangle_{\varphi} \phi_p(x)$  with $p=\frac{\alpha\beta}{b(1+\beta)}$.
Note that there may be no $p$ such that $p=\frac{\alpha\beta}{b(1+\beta)}$. In this case $f_c=0$.
Our main results above give central limit type results for $\langle f_l, X_t\rangle$, $\langle f_c, X_t\rangle$ and $\langle f_s, X_t\rangle$, respectively.
We conjecture that the limits of these three terms, normalized properly, are independent, because intuitively these limits come from small time intervals, intermediate time intervals and large time intervals, respectively.
If this is valid, we can get a central limit type result for $\langle f, X_t\rangle$ for general $f\in\mathcal{P}$.
This independence was proved under the second moment condition, see \cite{RenSongZhang2015Central}.
We leave the question of independence for stable branching mechanism to a future project.

This paper is our first attempt on the stable CLT for superprocesses.
There are still many open questions.
Ren, Song and Zhang have established some spatial  central limit theorems in \cite{RenSongZhang2015Central} for a class of superprocesses with general spatial motions under the assumption that the branching mechanisms satisfy a second moment condition.
We hope to prove spatial CLT's for superprocesses with general motions without the second moment assumption on the branching mechanism in a future project.

Recall that our Assumption \ref{asp: branching mechanism} says that the branching mechanism $\psi$ is $-\alpha z + \rho z^2+\eta z^{1+\beta}$ plus a small perturbation
\begin{align}
  \label{eq: psi 1}
	\psi_1(z)
	& := \int_{(0,\infty)}(e^{-yz}-1+yz) \Big(\pi(dy) - \frac{\eta~dy}{\Gamma(-1-\beta) y^{2+\beta}}\Big)
\end{align}
where $\psi_1$ satisfies \eqref{eq: asp of branching mechanism} with some $\delta>0$.
It would be interesting to consider more general branching mechanisms.

Here are some examples of branching mechanisms satisfying Assumptions \ref{asp: Greys condition} and \ref{asp: branching mechanism}:
If $h$ is a complete Bernstein function which is regularly varying at 0 with index $\beta_1\in (\beta, 1)$, then
\[
  \psi(z)
  := -\alpha z + \rho z^2+\eta z^{1+\beta}+zh(z)
  , \qquad z>0
\]
satisfies Assumptions \ref{asp: Greys condition} and \ref{asp: branching mechanism}.
If $\beta_1\in (\beta, 1)$, $c_1\in (0, \eta/\Gamma(-1-\beta))$ and $c_2\ge 1$, then
\[
  \psi(z)
  :=-\alpha z + \rho z^2+\eta z^{1+\beta}-\int^\infty_{c_2} (e^{-yz}-1+yz)\frac{c_1dy}{y^{1+\beta_1}}
  , \qquad z\in \mathbb R_+
\]
satisfies Assumptions \ref{asp: Greys condition} and \ref{asp: branching mechanism}.

The rest of the paper is organized as follows: 
In Subsection \ref{sec: branching mechanism} we will give some preliminary results for our branching mechanism $\psi$.
In Subsections \ref{sec: controller} and \ref{sec: h-controller} we will give some estimates for some operators related to the super-OU process $X$ that will be used in our proofs.
In Subsection \ref{sec: stable distributions} we will give the definitions of all the $(1+\beta)$-stable random variables involved in this paper.
In Subsection \ref{sec: Small value probability} we will give some estimates for the small value probability of continuous state branching processes. 
In Subsection \ref{sec: Moments for super-OU processes} we will give upper bounds for the $(1+\gamma)$-moments for our superprocesses.
These estimates and upper bounds will be crucial in the proofs of our main results.
In Subsection \ref{sec: large rate lln}, we will give the proof of Theorem \ref{thm: law of large number}.
In Subsection \ref{sec:critical}, we will give the proof of Theorem \ref{thm: critical clt}.
In Subsection \ref{sec: small rate}, we will give the proof of Theorem \ref{thm: small clt}.
In Subsection \ref{sec: large rate clt}, we will give the proof of Theorem \ref{thm: large clt}.
In the Appendix, we consider a general superprocess $(X_t)_{t\geq 0}$, and we prove there that the characteristic exponent of $\langle f,X_t\rangle$ satisfies a complex-valued non-linear integral equation.
This fact will be used at several places in this paper, and we think it is of independent interest.
\begin{comment}
  We end this section with a list of frequently used notations:
  \begin{itemize}
  \item
    $\varphi$ and $\phi_p$ are given in \eqref{invariantdensity} and \eqref{eigenfunction}, respectively.
  \item
    $P_t$, $P^\alpha_t$, $V_t$ and $U_t$ are given in \eqref{eq: meansemigroup}, \eqref{eq: meansemigroup}, \eqref{eq: def of V_t}, \eqref{eq: def of U_t}, respectively.
  \item
    $\mathcal N$, $K$, $\mathcal C_l$ and $I_t$ are given in \eqref{eq: def of N}, \eqref{eq: def of N},  \eqref{eq: def of Cl} and \eqref{definition of Itf}, respectively.
  \item
    $\psi$, $\psi_1$ and $\psi_0$ are given in \eqref{eq: honogeneou branching mechanism}, \eqref{eq: psi 1} and \eqref{eq: psi 0}, respectively.
  \item
    $\Psi$, $\Psi_1$ and $\Psi_0$ are given in \eqref{eq: def of operator Psi}.
  \item
    $Z_t$, $Z'_t$, $Z''_t$ and $Z'''_t$ are given in \eqref{eq: def of Zf}.
  \item
    $Q_\kappa$ and $Q$ are given in \eqref{Q_k} and \eqref{Q}, respectively.
  \item
    $m_t[f]$, $\bar m_t[f]$, $m[f]$, $\widetilde m[f]$ and $\bar m[f]$ are given in Lemma \ref{lem: def of all m}.
  \end{itemize}
\end{comment}

% ** Preliminaries
\section{Preliminaries}
% *** Branching mechanism
\subsection{Branching mechanism}
\label{sec: branching mechanism}
Let $\psi$ be the branching mechanism given in \eqref{eq: honogeneou branching mechanism}.
Suppose that Assumptions \ref{asp: Greys condition} and \ref{asp: branching mechanism} hold.
In this subsection, we give some preliminary results about the branching mechanism $\psi$.
Recall that $\eta$ and $\beta$ are the constants in Assumption \ref{asp: branching mechanism}.
Let $\mathbb C_+:= \{x+iy: x\in \mathbb R_+, y \in \mathbb R\}$ and $\mathbb C^0_+:= \{x+iy: x\in (0,\infty), y \in \mathbb R\}$.
\begin{lem}
  \label{lem: complex extension for psi1}
	The function $\psi_1$ on $\mathbb R_+$ can be uniquely extended as a complex-valued continuous function on $\mathbb C_+$ which is holomorphic on $\mathbb C^0_+$.
  Moreover, for each $\delta > 0$ small enough, there exists $C>0$ such that for all $z\in \mathbb C_+$, we have
  $
	|\psi_1(z)| \leq C |z|^{1+\beta+\delta} + C|z|^2.
  $
\end{lem}
\begin{proof}
  According to Lemma \ref{lem: extension lemma for branching mechanism} below and the uniqueness of holomorphic extensions,
	we know that $\psi_1$ can be uniquely extended as a complex-valued continuous function on $\mathbb C_+$ which is holomorphic on $\mathbb C^0_+$.
	The extended $\psi_1$ has the following form:
  \[
    \psi_1(z)
    = \int_{(0,\infty)}(e^{-yz}-1+yz) \Big(\pi(dy) - \frac {\eta~dy} {\Gamma(-1-\beta)y^{2+\beta}} \Big)
    , \quad z\in \mathbb C_+.
  \]
	Now, according to  Assumption \ref{asp: branching mechanism} and Remark \ref{rem: small enough delta}, for each small enough $\delta > 0$, we have
  \begin{align}
    |\psi_1(z)|
    & \leq \int_{(0,\infty)} (|yz|\wedge |yz|^2) \Big|\pi(dy) - \frac{\eta~dy}{\Gamma(-1-\beta)y^{2+\beta}}\Big| \\
    & \leq  |z|^2\int_{(0,1)} y^2 \Big|\pi(dy) - \frac{\eta~dy}{\Gamma(-1-\beta)y^{2+\beta}}\Big| \\
    & \quad + |z|^{1+\beta +\delta}\int_{(1,\infty)} y^{1+\beta + \delta} \Big|\pi(dy) - \frac{\eta~dy}{\Gamma(-1-\beta)y^{2+\beta}}\Big|,
      \quad z \in \mathbb C_+,
  \end{align}
	as desired.
\end{proof}
The following lemma says that the constants $\eta, \beta$ in Assumption \ref{asp: branching mechanism} are uniquely determined by the L\'evy measure $\pi$.
\begin{lem}
  \label{lem: unique of beta and eta}
  Suppose Assumption  \ref{asp: branching mechanism} holds. Suppose that there are $\eta', \delta'>0$ and $\beta'\in (0,1)$ such that
  \[
    \int_{(1,\infty)} y^{ 1 + \beta'  + \delta' }~ \Big| \pi(dy) - \frac {\eta' ~dy} {\Gamma (- 1 - \beta ) y^{2 + \beta'}} \Big| 
    < \infty.
  \]
	Then $\eta'= \eta$ and $\beta ' = \beta$.
\end{lem}
\begin{proof}
	Without loss of generality, we assume that $\beta+\delta \leq \beta'+ \delta'$.
	Using  the fact that $y^{1+\beta+ \delta} \leq y^{1+\beta'+\delta'}$ with $y \geq 1$, we get
  \[
    \int_{(1, \infty)} y^{1 + \beta + \delta}   \Big| \pi(dy) - \frac {\eta' ~dy} {\Gamma( - 1 - \beta)y^{2 + \beta'}} \Big| 
    < \infty .
  \]
	Comparing this with Assumption \ref{asp: branching mechanism}, we get
  \[
    \int_{(1,\infty)} y^{ 1 + \beta + \delta} \Big| \frac { \eta ~dy} {\Gamma (- 1 - \beta) y^{2 + \beta}} - \frac {\eta' ~dy} {\Gamma (- 1 - \beta) y^{2 + \beta'}} \Big| < \infty.
  \]
	In other words, if we denote by $\widetilde \pi(dy)$ the measure $\eta' \Gamma(-1-\beta)^{-1} y^{-2-\beta'} dy$, then $\widetilde \pi$ is a L\'evy measure which satisfies Assumption \ref{asp: branching mechanism}.
	Applying Lemma \ref{lem: complex extension for psi1} to this measure $\widetilde \pi$, we have that there exists $c>0$ such that
  \[
    | \eta z^{ 1 + \beta } - \eta' z^{ 1 + \beta' } |
    \leq c z^{ 1 + \beta + \delta } + c z^2
    , \quad z \in \mathbb R_+.
  \]
  Dividing both sides by $z^{1+\beta}$ we have
  \[
    | \eta - \eta' z^{ \beta' - \beta } |
    \leq cz^{\delta}+cz^{1-\beta}
    ,	\quad z \in \mathbb R_+.
  \]
	This implies that $ \eta' z^{\beta' - \beta} \xrightarrow[\mathbb R^+\ni z\to 0]{} \eta >0. $
	So we must have $\beta'= \beta$ and $\eta'= \eta$.
\end{proof}
\begin{lem}
  \label{lem: LlogL criterion}
  If $\psi$ satisfies Assumption  \ref{asp: branching mechanism}, then $\psi$ satisfies the $L \log L$ condition, i.e.,
  \[
    \int_{(1,\infty)} y \log y~\pi(dy)
    < \infty.
  \]
\end{lem}
\begin{proof}
	Using  Assumption \ref{asp: branching mechanism} and the fact that $y\log y \leq y^{1+\beta+\delta}$ for $y$ large enough, we get
  \[
    \int_{(1,\infty)} y \log y ~\Big| \pi(dy) - \frac { \eta ~dy } { \Gamma ( - 1 - \beta ) y^{ 2 + \beta } } \Big| 
    < \infty.
  \]
	Therefore we have
  \[
    \int_{ ( 1, \infty ) } y \log y ~\Big( \pi(dy) - \frac { \eta ~dy } { \Gamma ( - 1 - \beta ) y^{ 2 + \beta } } \Big) 
    < \infty.
  \]
  Combining this with
  \[
    \int_{ ( 1, \infty ) } \frac { \eta \log y ~dy } { \Gamma ( - 1 - \beta ) y^{ 1 + \beta } } 
    < \infty,
  \]
  we immediately get the desired result.
\end{proof}

% ** Definition of controller
\subsection{Definition of controller}
\label{sec: controller}
Denote by $\mathcal B(\mathbb R^d, \mathbb C)$ the space of all $\mathbb C$-valued Borel functions on $\mathbb R^d$.
Recall that $\mathcal P$ is given in \eqref{eq: polynomial growth function}.
Define $\mathcal P^+:= \mathcal P \cap \mathcal B(\mathbb R^d, \mathbb R_+)$ and $\mathcal P^*:= \{f\in \mathcal B(\mathbb R^d, \mathbb C): |f|\in \mathcal P\}$.

% Modified
% For any function $h: [0,\infty) \to [0,\infty)$, we say an operator $R: \mathcal P^+ \to \mathcal P^+$ is an \emph{$h$-controller} if
% (1) $f, g\in \mathcal P^+$ and $f\leq g$ imply that $Rf \leq Rg$; and (2) $f \in \mathcal P^+$ and $\theta \in [0,\infty)$ imply that $ R (\theta f)\leq h(\theta) Rf$.
% For a subset $\mathcal D$ of $\mathcal P^*$ and an $h$-controller $R$, we say an operator $A : \mathcal D \to \mathcal P^*$ is \eminnershapeph{$h$-controlled by $R$ on $\mathcal D$} if $|Af| \leq R|f|$ for each $f\in \mathcal D$.
% We say an operator $A: \mathcal D \to \mathcal P^*$ is \emph{$h$-controllable} if there exists an $h$-controller $R$ such that $A$ is $h$-controlled by $R$.
% We say a family of operators $(A_s)_{s\in \Lambda}$ from $\mathcal D $ to $\mathcal P^*$ is \emph{uniformly $h$-controllable on $\mathcal D$} if there exists an $h$-controller $R$ such that, for each $s\in \Lambda$, $A_s$ is $h$-controlled by $R$ on $\mathcal D$.



In this paper, we say $R$ is a \emph{monotone} operator on $\mathcal P^+$ if $R:\mathcal P^+ \to \mathcal P^+$ satisfies that $Rf\leq Rg$ for each $f\leq g$ in $\mathcal P^+$.
For a function $h: [0,\infty) \to [0,\infty)$, we say $R$ is an \emph{$h$-controller} if $R$ is a monotone operator on $\mathcal P^+$ and that $R(\theta f)\leq h(\theta) Rf$ for each $f\in \mathcal P^+$ and $\theta \in [0,\infty)$.
For subsets $\mathcal D, \mathcal I\subset \mathcal P^*$ and an operator $R$ on $\mathcal P^+$, we say an operator $A$ is \emph{controlled by $R$ from $\mathcal D$ to $\mathcal I$} if $A:\mathcal D \to \mathcal I$ and that $|Af| \leq R|f|$ for each $f\in \mathcal D$; say a family of operators $\mathscr O$ is \emph{uniformly controlled by $\mathcal R$ from $\mathcal D$ to $\mathcal I$} if for each $A\in \mathcal O$, $A$ is controlled by $R$ from $\mathcal D$ to $\mathcal I$.
For subsets $\mathcal D, \mathcal I\subset \mathcal P^*$ and a function $h:[0,\infty) \to [0,\infty)$, we say an operator $A$ (resp. a family of operators $\mathscr O$) is \emph{$h$-controllable} (resp. \emph{uniformly $h$-controllable}) from $\mathcal D$ to $\mathcal I$ if there exists an $h$-controller $R$ such that $A$ (resp. $\mathscr O$) is controlled (resp. uniformly controlled) by $R$ from $\mathcal D$ to $\mathcal I$. 
% End Modified

For two operators $A: \mathcal D_A \subset \mathcal P^*\to \mathcal P^*$ and $B: \mathcal D_B \subset \mathcal P^*\to \mathcal P^*$, define $(A \times B)f (x):= Af(x) \times Bf(x)$ for all $f\in \mathcal D_A \cap \mathcal D_B$ and $x\in \mathbb{R}^d$.
For any $a \in \mathbb R$ and any operator $A :\mathcal D_A \to \mathcal B(\mathbb R^d, \mathbb C\setminus (-\infty, 0])$, define $A^{\times a}f(x):= (Af(x))^a$ for all $f\in \mathcal D_A$ and $x\in \mathbb R^d$.

% Modified
\begin{comment}
\begin{lem}
  \label{lem: property of controllable operators}
  Suppose that $\Lambda$ is an index set and $(A_\lambda)_{\lambda\in \Lambda}$ is a
  family of operators, from $\mathcal D\subset \mathcal P^*$ to $ \mathcal P^*$,
  which is uniformly 
  $h$-controlled by $R_A$ on $\mathcal D$ for a given function $h:[0,\infty) \to [0, \infty)$.
  Suppose that $\Delta$ is another index set and $(B_\delta)_{\delta\in \Delta}$ is a family of operators, from $\mathcal D_0\subset \mathcal P^*$ to $ \mathcal D$,
  which is uniformly $g$-controlled by $R_B$ on $\mathcal D_0$ 
  for some function $g: [0,\infty) \to [0,\infty)$. 
  Suppose that $(C_\lambda)_{\lambda\in \Lambda} $ is a family of operators, from $\mathcal D\subset \mathcal P^*$ to $ \mathcal P^*$,
  which is uniformly $k$-controlled by $R_C$ on $\mathcal D_0$ for a given function $k:[0,\infty) \to [0, \infty)$.
  Then 
  \begin{itemize}
  \item[(1)]
    $(A_\lambda B_\delta)_{\delta\in \Delta, \lambda \in \Lambda}$ is uniformly $(h \circ g)$-controllable on $\mathcal D_0$;
  \item[(2)]
    $(B_\delta\times A_\lambda)_{\delta \in \Delta, \lambda \in \Lambda}$ is uniformly $(h\times g)$-controllable on $\mathcal D$ and $(B_\delta + A_\lambda)_{\delta \in \Delta, \lambda \in \Lambda}$ is uniformly $(h\vee g)$-controllable on $\mathcal D$;
  \item[(3)] 
    For any $a>0$, $(C^{\times a}_\lambda)_{\lambda \in \Lambda}$ is uniformly $(k^a)$-controllable.
  \item[(4)]
    Suppose further that $(\Lambda, \mathscr F)$ is a measurable space and that $(\lambda,x)\mapsto A_\lambda f(x)$ is $\mathscr F \otimes \mathscr B(\mathbb R^d)$-measurable for each $f\in \mathcal D$.
    For any probability measure $\nu$ on $(\Lambda, \mathscr F)$,  we write
    \[
      A_\nu f(x)
      := \int_{\Lambda} A_\lambda f (x)~\nu(d\lambda),
      \quad f\in \mathcal D, x\in \mathbb R^d.
    \]
    Then  $\{A_\nu: \nu \text{ is  a probability measure on } (\Lambda, \mathscr F)\}$ is uniformly $h$-controllable on $\mathcal D$.
  \end{itemize}
\end{lem}
\end{comment}
The following lemma is easy to verify.
\begin{lem}
  \label{lem: property of controllable operators}
  For each $i \in \{0,1\}$, let $\mathscr O_i$ be a family of operators which is  uniformly controlled by an $h_i$-controller $R_i$ from $\mathcal D_i \subset \mathcal P^*$ to $ \mathcal I_i \subset \mathcal P^*$. 
  Then the followings hold: 
  \begin{enumerate}
  \item 
    If $\mathcal I_0 \subset \mathcal D_1$, then $\{A_1A_0: A_i \in \mathscr O_i, i = 0,1\}$ is uniformly controlled by the $(h_1 \circ h_0)$-controller $R_1R_0$ from $\mathcal D_0$ to $\mathcal I_1$;
  \item
    $\{ A_1 \times A_0: A_i \in \mathscr O_i, i = 0,1\}$ is uniformly controlled by the $(h_1\times h_0)$-controller $R_1 \times R_0$ from $\mathcal D_0 \cap \mathcal D_1$ to $\mathcal P^*$.
  \item
    $\{ A_1 + A_0: A_i \in \mathscr O_i, i = 0,1\}$ is uniformly controlled by the $(h_1 \vee h_0)$-controller $R_1 + R_0$ from $\mathcal D_0 \cap \mathcal D_1$ to $\mathcal P^*$. 
\item 
    If $\mathcal I_0 \subset \mathcal B(\mathbb R^d, \mathbb C \setminus (\infty, 0])$ and $a>0$, then $\{A^{\times a} : A \in \mathscr O_0\}$ is uniformly controlled by the $(h_0^a)$-controller $R_0^{\times a}$ from $\mathcal D_0$ to $\mathcal P^*$.
  \item
    Suppose that $\mathcal O_0 = \{A_\theta: \theta \in \Theta \}$ where $\Theta$ is an index set. 
    Further suppose that $(\Theta, \mathcal J )$ is a measurable space and that $(\theta,x) \mapsto A_\theta f(x)$ is $\mathcal J \otimes \mathcal B(\mathbb R^d)$-measurable for each $f\in \mathcal D$.
    Then the following space of operators
    \[
      \Big\{ f \mapsto \int_{\Theta} A_\theta f~\nu(d\theta) : \nu \text{ is a probability measure on } (\Theta, \mathcal J) \Big\}
    \] 
    is uniformly controlled by $R_0$ from $\mathcal D_0$ to $\mathcal P^*$.
\end{enumerate}  
\end{lem}
% END MODIFIED

% DELETED
\begin{comment}
\begin{proof}
  (1). 
  If $\lambda \in \Lambda$, $\delta \in \Delta$ and $f\in \mathcal D_0$, then $|A_\lambda B_\delta f| \leq R_A |B_\delta f| \leq R_A R_B |f|$.
  If $f,g \in \mathcal P^+$ with $f\leq g$, then $R_AR_Bf \leq R_A R_B g$.
  If $f \in \mathcal P^+$ and $\theta \in [0,\infty)$, then $R_AR_B(\theta f) \leq R_A(g(\theta) R_Bf) \leq (h\circ g)(\theta) R_A R_B f $.
  % Therefore 
  Combining these, we get that
  $(A_\lambda B_\delta)_{\delta\in \Delta, \lambda \in \Lambda}$ is uniformly $(h \circ g)$-controllable on $\mathcal D_0$.
  
  (2). 
  If $\lambda \in \Lambda$, $\delta \in \Delta$ and $f\in \mathcal D$, then $|A_\lambda \times B_\delta f| \leq |A_\lambda f| \cdot |B_\delta f| \leq R_A \times R_B |f|$.
  If $f,g \in \mathcal P^+$ with $f\leq g$, then $R_A\times R_Bf \leq R_A\times R_B g$.
  If $f \in \mathcal P^+$ and $\theta \in [0,\infty)$, then $R_A\times R_B (\theta f) \leq h(\theta) g(\theta) (R_A \times R_B) f $.
  Combining these, we get that $(A_\lambda \times B_\delta)_{\delta\in \Delta, \lambda \in \Lambda}$ is uniformly $(h \times g)$-controllable on $\mathcal D$.
  
  If $\lambda \in \Lambda$, $\delta \in \Delta$ and $f\in \mathcal D$, then $|(A_\lambda + B_\delta) f| \leq |A_\lambda f| + |B_\delta f| \leq (R_A + R_B) |f|$.
  If $f,g \in \mathcal P^+$ with $f\leq g$, then $(R_A + R_B)f \leq (R_A + R_B) g$.
  If $f \in \mathcal P^+$ and $\theta \in [0,\infty)$, then $(R_A + R_B) (\theta f) \leq h(\theta) R_Af + g(\theta) R_Bf \leq (h(\theta) \vee g(\theta)) (R_A+R_B)f $.
  Combining these, we get that $(A_\lambda + B_\delta)_{\delta\in \Delta, \lambda \in \Lambda}$ is uniformly $(h \vee g)$-controllable on $\mathcal D$.
  
  (3). 
  If $\lambda \in \Lambda$, $f\in \mathcal D$, then $|C_\lambda^{\times a} f| = |C_\lambda f|^a  \leq R_C^{\times a} |f|$.
  If $f,g \in \mathcal P^+$ with $f\leq g$, then $R_C^{\times a}f \leq R_C^{\times a} g$.
  If $f \in \mathcal P^+$ and $\theta \in [0,\infty)$, then $R_C^{\times a}(\theta f) \leq (k(\theta) R_C f)^a = k(\theta)^a R_C^{\times a}f$.
  Combining these, we get that $(C_\lambda^{\times a})_{\lambda \in \Lambda}$ is uniformly $(k^a)$-controllable on $\mathcal D$.
  
  (4).     
  For all $f \in \mathcal{D}$, $x\in \mathbb R^d$ and  probability measure $\nu$ on $(\Lambda, \mathscr F)$,
  \[
    | A_{\nu}f(x) | 
    \leq \int_{\Lambda} | A_{\lambda}f(x) | \nu(d\lambda) 
    \leq \int_{\Lambda}R|f|(x)\nu(d\lambda) \leq R|f|(x). 
    \qedhere
  \]
\end{proof}
\end{comment}
% END DELETED
% ** h-controller for super-OU processes% \subsection{h-controller for super-OU processes}
\subsection{Controllers for super-OU processes}
\label{sec: h-controller}
Let $X$ be the super-OU process introduced in Subsection \ref{sec: Motivation} with branching mechanism $\psi$ satisfying
Assumptions \ref{asp: Greys condition} and \ref{asp: branching mechanism}.
In this subsection, we will define several operators and study their properties that will be used in this paper.

Define $\psi_0(z) = \psi(z) + \alpha z$ for each $z\in \mathbb{R}_+$.
% Using an argument similar to the proof of Lemma \ref{lem: complex extension for psi1}, $\psi, \psi_0$ and $\psi_1$ can be uniquely extended as complex-valued continuous functions on $\mathbb C_+$ which are holomorphic on $\mathbb C^0_+$.
According to Lemma \ref{lem: complex extension for psi1}, $\psi, \psi_1$ and $\psi_0$ can all be uniquely extended as complex-valued continuous functions on $\mathbb C_+$ which are also holomorphic on $\mathbb C^0_+$.
For each $f\in \mathcal B(\mathbb R^d, \mathbb C_+)$ and $x\in \mathbb R^d$, define
$\Psi f (x) = \psi\circ f(x)$, $\Psi_0 f(x)= \psi_0 \circ f(x)$ and $\Psi_1 f(x)= \psi_1 \circ f(x)$.

For all $t\in [0,\infty), x\in \mathbb R^d $ and $f \in \mathcal{P}$, let
\begin{align}
  \label{eq: def of U_t}
  U_tf(x) 
  := \operatorname{Log} \mathbb P_{\delta_x}[e^{i\theta \langle f, X_t\rangle}]|_{\theta = 1}
\end{align}
be the value of the characteristic exponent of the infinitely divisible random variable $\langle f, X_t\rangle$ at $1$.
% new
(See the paragraph after Lemma \ref{lem: Lip of power function}.)
% end new
It follows from \eqref{eq: -v has positive real part} that $-U_tf(x)$ takes values in $\mathbb C_+$. Furthermore, we know from Proposition \ref{prop: complex FKPP-equation} that
\begin{align}
  \label{eq:chareq2}
  U_tf(x) - \int_0^t P^\alpha_{t-s} \Psi_0(-U_sf)(x)ds
  = i P^{\alpha}_t f(x)
  , \quad t\in [0,\infty), x\in \mathbb{R}^d, f\in \mathcal P.
\end{align}

For all $t\geq 0$ and $f\in \mathcal P$, we define
\begin{align}
  \label{eq: def of Zf}
  Z_t f
    := \int_0^t P^\alpha_{t-s}\big( \eta (-i P^\alpha_sf)^{1+\beta}\big)ds,
    & \qquad Z'_t f
    := \int_0^t P^\alpha_{t-s}\big( \eta (-U_s f)^{1+\beta}\big)ds,\\
   Z''_t f
    := \int_0^t P^\alpha_{t-s}\Psi_1(-U_s f)ds,
    &\qquad\  Z'''_t f
    := (Z'_t - Z_t+ Z''_t)f.
\end{align}
Then we have that
\begin{equation}
  \label{eq: key equality}
  U_t - i P^\alpha_t
  = Z'_t + Z''_t
  = Z_t + Z'''_t
  , \quad t\geq 0.
\end{equation}
For all $\kappa \in \mathbb Z_+$ and $f\in \mathcal P$, define
\begin{equation}
  \label{Q_k}
  Q_\kappa f
  := \sup_{t\geq 0} e^{\kappa b t}|P_t f|, 
  \qquad  Q f
  := Q_{\kappa_f}f.
\end{equation}
Then according to \cite[Fact 1.2]{MarksMilos2018CLT}, $Q$ is an operator from $\mathcal P$ to $\mathcal P$.

\begin{lem}
  \label{lem: upper bound for usgx}
  Under Assumptions \ref{asp: Greys condition} and \ref{asp: branching mechanism}, the following statements are true:
  \begin{itemize}
  \item[(1)]
    $(-U_t)_{0\leq t\leq 1}$ is uniformly $\theta$-controllable from $\mathcal P$ to $\mathcal P^*\cap \mathcal B(\mathbb R^d, \mathbb C_+)$.
  \item[(2)]
    $(P^\alpha_t)_{0\leq t\leq 1}$ is uniformly $\theta$-controllable on $\mathcal P^*$.
  \item[(3)]
    $\Psi_0$ is $(\theta^2\vee \theta^{1+\beta})$-controllable on $\mathcal P^* \cap \mathcal B(\mathbb R^d, \mathbb C_+)$.
  \item[(4)]
    $(U_t- iP_t^{\alpha})_{0\leq t\leq 1}$ is uniformly $(\theta^2\vee \theta^{1+\beta})$-controllable on $\mathcal P$.
  \item[(5)]
    $(Z'_t-Z_t)_{0\leq t\leq 1}$ is uniformly $(\theta^{2+\beta}\vee \theta^{1+2\beta})$-controllable on $\mathcal P$.
  \item[(6)]
    For any $\delta > 0$ small enough, we have that $(Z''_t)_{0\leq t\leq 1}$ is uniformly $(\theta^2\vee \theta^{1+\beta+\delta})$-controllable on $\mathcal P$.
  \item[(7)]
    For any $\delta > 0$ small enough, we have that $(Z'''_t)_{0\leq t\leq 1}$ is uniformly $(\theta^{2+\beta}\vee \theta^{1+\beta+\delta})$-controllable on $\mathcal P$.
  \end{itemize}
\end{lem}

\begin{proof}
  (1). According to \eqref{eq: -v has positive real part}, $U_t$ is an operator from $\mathcal P$ to $\mathcal B(\mathbb R^d, \mathbb C_+)$.
  It follows from \eqref{eq: upper bound for vf} that for all $g\in \mathcal P$, $0\leq t\leq 1$ and $x\in \mathbb R^d$,
  \[
    |U_t g(x)|
    \leq \sup_{0\leq u\leq 1}P_u^\alpha |g| (x).
  \]
  We claim that $f\mapsto\sup_{0\leq u\leq 1}P^{\alpha}_u f$ is a map from $\mathcal P^+$ to $\mathcal P^+$. In fact, if $f\in \mathcal P^+$, there exists constant $c>0$ such that
  \[
    0
    \leq \sup_{0\leq u\leq 1}P^{\alpha}_u f
    \leq \sup_{0\leq u\leq 1} P_u (e^{\alpha u} e^{-\kappa_f u} e^{\kappa_f u} f )
    \leq c \sup_{0\leq u\leq 1} (e^{\kappa_fu}P_u f) \leq c Qf \in \mathcal P.
  \]
	It is clear that $f\mapsto\sup_{0\leq u\leq 1}P^{\alpha}_u f$ is a $\theta$-controller.
  
  (2). Similar to the proof of (1).
  
  (3). By Lemma \ref{lem: complex extension for psi1}, there exist $C, \delta >0$ satisfying $\beta+\delta< 1$ such that for all $ f \in \mathcal P^* \cap \mathcal B( \mathbb R^d, \mathbb C_+ )$,
  \begin{align}
    &|\Psi_0 f|
      \leq \eta |f|^{1+\beta} + |\Psi_1 f|
      \leq \eta |f|^{1+\beta} + C|f|^2+ C |f|^{1+\beta + \delta}.
  \end{align}
  Note that
  \[
    f \mapsto \eta f^{1+\beta} + Cf^2+ Cf^{1+\beta + \delta}
    , \quad f\in \mathcal P^+,
  \]
  is a $(\theta^2 \vee \theta^{1+\beta})$-controller.
  
  (4). 
  From (1)--(3) above and Lemma \ref{lem: property of controllable operators}.(1), we know that the operators
  \[
    f \mapsto P^\alpha_{t-s}\Psi_0(-U_sf)
    , \quad 0\leq s\leq t\leq 1,
  \]
  are uniformly $(\theta^2\vee \theta^{1+\beta})$-controllable.
  Combining this with \eqref{eq:chareq2} and 
  Lemma \ref{lem: property of controllable operators}.(4), we get the desired result.
  
  (5). Notice that from Lemma \ref{lem: Lip of power function},
  \[
    |(-U_t f)^{1+\beta} - (-iP^\alpha_t f)^{1+\beta} |
    \leq  (1+\beta) |U_t f-iP^\alpha_t f|(|U_t f|^{\beta}+|i P^\alpha_t f|^{\beta}).
  \]
  Now using (1), (2) and  (4) above, and Lemma \ref{lem: property of controllable operators}.(2)--(3), we get that the operators
  \[
    f \mapsto (-U_t f)^{1+\beta} - (-iP^\alpha_t f)^{1+\beta},\quad 0\leq t\leq 1,
  \]
  are uniformly $(\theta^{2+\beta}\vee \theta^{1+2\beta})$-controllable.
  Combining with Lemma \ref{lem: property of controllable operators}.(1) and (4), and
  \begin{equation}
    (Z'_t - Z_t)f 
    = \int_0^t P^\alpha_{t-s}\Big( \eta \big((-U_s f)^{1+\beta} - (-iT_s^\alpha f)^{1+\beta} \big)\Big)ds
    , \quad 0\leq t\leq 1, f\in \mathcal P,
  \end{equation}
  we get the desired result.
  
  (6). By  Lemma \ref{lem: complex extension for psi1}, for any $\delta > 0$ small enough, there exists  $C>0$ such that
  \[
    |\Psi_1(f)|
    \le C(|f|^2+|f|^{1+\beta+ \delta})
    , \quad f\in \mathcal P^*\cap\mathcal B(\mathbb R^d, \mathbb C_+).
  \]
  Note that, for all $\delta, C>0$,
  \[
    f \mapsto C(f^2+f^{1+\beta+\delta})
    , \quad f\in \mathcal P^+
  \]
  is a $(\theta^2 \vee \theta^{1+\beta+\delta})$-controller.
  Therefore, for any $\delta > 0$ small enough, we have that $\Psi_1$ is a $(\theta^2 \vee \theta^{1+\beta+\delta})$-controllable operator from $\mathcal P^*\cap\mathcal B(\mathbb R^d, \mathbb C_+)$ to $\mathcal P^*$.
  Combining  this  (1)--(2) above, and Lemma \ref{lem: property of controllable operators}.(1) and (4), we get that, for any $\delta > 0$ small enough, the operators
  \[
    f
    \mapsto Z_t'' f
    = \int_0^t P_{t-s}^\alpha \Psi_1(-U_sf)ds
    , \quad 0\leq t\leq 1,
  \]
  are uniformly $(\theta^2 \vee \theta^{1+\beta+\delta})$-controllable from $\mathcal P$ to $\mathcal P^*$.
  
  (7). Since $Z'''_t = (Z'_t-Z_t)+Z''_t$, the desired result follows from (5)-- (6) above and Lemma \ref{lem: property of controllable operators}.(2).
\end{proof}

% *** Stable distributions
\subsection{Stable distributions}
\label{sec: stable distributions}

In this subsection, we will show that the $(1+\beta)$-stable random variables in Theorems \ref{thm: large clt}, \ref{thm: critical clt} and \ref{thm: small clt} are all well defined.

Recall that $(P^\alpha_t)_{t\geq 0}$ is defined in \eqref{eq: meansemigroup}, $\mathcal P$ is defined in \eqref{eq: polynomial growth function}, $\mathcal N$ is given in \eqref{eq: def of N} and $\mathcal C_l$ is defined in \eqref{eq: def of Cl}.
\begin{lem}
\label{lem: def of all m}
\begin{enumerate}
\item
  If $f\in\mathcal{P}$, then the following integrals are well defined:
  \[
    m_t[f]
    := \eta \int_t^{t+1}e^{-\alpha s}~ds\int_{\mathbb R^d} (-iP_{s}^\alpha f(x))^{1+\beta} \varphi(x)~dx,
    \quad t\geq 0.
  \]
	Furthermore, there exists  $C>0$ such that
  \begin{equation}
    \label{domi-m}
    |m_t[f]|
    \leq C e^{(\alpha\beta-\kappa_fb(1+\beta))t},
    \quad t\geq 0.
  \end{equation}
\item
  If $f\in\mathcal C_l$, then the following integrals are well defined:
  \[
    \bar{m}_t[f]
    := \eta \int_{t-1}^{t} e^{\alpha s}~ds \int_{\mathbb R^d}\big(iI_sf(x)\big)^{1+\beta} \varphi(x)~dx, \quad t\geq 1.
  \]
  Furthermore, there exists  $C>0$ such that
  \begin{equation}
    |\bar{m}_t[f]| 
    \leq C e^{-(\alpha\beta-K(1+\beta)b)t}, \quad t\geq 1,
  \end{equation}
  where $K$ is defined in \eqref{eq: def of N}.
\item
  If $f \in \mathcal C_l$, then the following integral is well defined:
  \[
    \bar{m}[f]
    :=\eta \int_{0}^{\infty} e^{\alpha s}~ds \int_{\mathbb R^d} \big( iI_sf(x) \big)^{ 1 + \beta } \varphi( x ) ~dx.
  \]
  Furthermore, we have
  \begin{equation}
    \label{sum-bar-m}
    \bar{m}[f]
    = \sum_{n=1}^{\infty} \bar{m}_n[f].
  \end{equation}
\item
  If $f \in \mathcal{P}$ satisfies $\alpha\beta=\kappa_f b(1+\beta)$, then the following integral is well defined:
  \[
    \widetilde{m}[f]
    := \eta\int_{\mathbb R^d} \Big(-i\sum_{p\in \mathbb Z_+^d:|p|=\kappa_f}\langle f,\phi_p\rangle\phi_p(x)\Big)^{1+\beta} \varphi(x)~dx.
  \]
	Furthermore, we have
  \begin{equation}
    \label{para: critical case}
    \widetilde{m}[f] 
    = \lim_{t\rightarrow \infty}\frac{1}{t}\sum_{k=0}^{\lfloor t \rfloor}m_k[f].
  \end{equation}
\item
  If $f\in \mathcal{P}$ satisfies $\alpha\beta<\kappa_fb(1+\beta)$, then the following integral is well defined:
  \[
    m[f]
    := \eta \int_0^{\infty} e^{-\alpha s} ~ds\int_{\mathbb R^d} \big(-iP_s^\alpha f(x)\big)^{1+\beta} \varphi(x)~dx.
  \]
	Furthermore, we have
  \begin{equation}
    \label{sum-m}
    m[f]
    = \sum_{k=0}^\infty m_k[f].
  \end{equation}
\end{enumerate}
\end{lem}
\begin{proof}
  (1) is  \cite[Lemma 2.7]{MarksMilos2018CLT}; (3) is obvious from (2) noticing that $\alpha\beta>K(1+\beta)b$; (4) is \cite[Lemma 4.2]{MarksMilos2018CLT}; and (5) is obvious from (1).
  So we only need to show (2). 
  We first claim that there exists $h\in \mathcal{P}$ such that
  \begin{align}
    \label{ineq: control of Itg}
    |I_tf(x)|
    \leq e^{-(\alpha-Kb)t}h(x),
    \quad t\geq 0, x\in \mathbb{R}^d.
  \end{align}
  In fact, if we put $h:= \sum_{p\in \mathcal N} |\langle f,\phi_p\rangle_{\varphi}\phi_p|$, then
\begin{equation}
  \label{eq: supper for Itf}
  |I_tf(x)|
  =\Big|\sum_{p\in\mathcal{N}} \langle f,\phi_p\rangle_{\varphi} e^{-(\alpha-|p|b)t}\phi_p(x)\Big|
  \leq e^{-(\alpha-Kb)t}h(x),
  \quad t\geq 0,x\in \mathbb{R}^d.
\end{equation}
Therefore there exists  $C>0$ such that
\begin{align}
  & |\bar{m}_t[f]|
    \leq \eta \int_{t-1}^{t} e^{\alpha s}~ds \int_{\mathbb R^d} | I_sf(x)|^{1+\beta}\varphi(x)~dx \\
  & \leq \eta \int_{t-1}^{t} e^{\alpha s}e^{-(\alpha-Kb)(1+\beta)s}~ds\int_{\mathbb R^d} h(x)^{1+\beta}\varphi(x)~dx \\
  & \leq C e^{-(\alpha\beta-K(1+\beta)b)t}
    , \quad t\geq 1.
\end{align}
\end{proof}
The following proposition says that $m_t[f]$, $\bar m_t[f]$, $m[f]$, $\bar m[f]$ and  $\widetilde m[f]$ are all related to $(1+\beta)$-stable distributions.
\begin{prop}
  \label{prop: alpha stable rv}
	For all $t\geq 0$ and non-trivial $f\in \mathcal P$,
  \begin{enumerate}
  \item
    \label{item: stable 1}
    $\theta \mapsto \exp(m_t[\theta f])$ is the characteristic function of an $\mathbb R$-valued $(1+\beta)$-stable random variable;
  \item
    \label{item: stable 2}
    $\theta \mapsto \exp(\bar{m}_t[\theta f])$ is the characteristic function of an $\mathbb R$-valued $(1+\beta)$-stable random variable, provided $f \in \mathcal C_l$;
  \item
    \label{item: stable 3}
    $\theta \mapsto \exp(\bar m[\theta f])$ is the characteristic function of an $\mathbb R$-valued $(1+\beta)$-stable random variable, provided $f \in \mathcal C_l$;
  \item
    \label{item: stable 4}
    $\theta \mapsto \exp(\widetilde m[\theta f])$ is the characteristic function of an $\mathbb R$-valued $(1+\beta)$-stable random variable, provided $\alpha\beta=\kappa_f b(1+\beta)$;
  \item
    \label{item: stable 5}
    $\theta \mapsto \exp(m[\theta f])$ is the characteristic function of an $\mathbb R$-valued $(1+\beta)$-stable random variable, provided $\alpha\beta < \kappa_f b(1+\beta)$.
  \end{enumerate}
\end{prop}
This proposition says that the $(1+\beta)$-stable random variables  in Theorems \ref{thm: large clt}, \ref{thm: critical clt} and \ref{thm: small clt} are all well defined.
The proof of this proposition relies on the following lemma:
\begin{lem}
  \label{lem: charactreisticfunction}
  Let $q$ be a measure on $\mathbb R^d\setminus\{0\}$ with
  $\int_{\mathbb R^d\setminus\{0\}} |x|^{1+\beta} q(dx) \in (0,\infty)$.
  Then
  \[
    \theta 
    \mapsto  \exp\Big\{\int_{\mathbb R^d\setminus\{0\}} (i\theta \cdot x)^{1+\beta} q(dx)\Big\},
    \quad \theta \in \mathbb R^d
  \]
  is the characteristic function of an $\mathbb R^d$-valued $(1+\beta)$-stable random variable.
\end{lem}
\begin{proof}
  It follows from disintegration that there exist a measure $\lambda$ on $S:= \{\xi\in \mathbb R^d:|\xi| = 1\}$ and a kernel $k(\xi,dt)$ from $S$ to $\mathbb R_+$ such that
  \[
    \int_{\mathbb R^d\setminus \{0\}} f(x)q(dx) 
    = \int_S \lambda(d\xi) \int_{\mathbb R^+} f(\xi t)k(\xi,dt),\quad
    f\in \mathcal B(\mathbb R^d\setminus \{0\}, \mathbb R_+).
  \]
  We define another measure $\lambda_0$ on $S$ by
  \[
    \lambda_0(d\xi) 
    := \frac1{\Gamma(-1-\beta)}\int_0^\infty t^{1+\beta}k(\xi,dt) \lambda (d\xi),
  \]
  where $\Gamma$ is the Gamma function.
  Then $\lambda_0$ is a non-zero finite measure, since
  \begin{align}
    &\lambda_0(S) 
      = \frac{1}{\Gamma(-1-\beta)} \int_S \lambda (d\xi) \int_0^\infty |t\xi|^{1+\beta}k(\xi,dt) \\
    & = \frac{1}{\Gamma(-1-\beta)} \int_{\mathbb R^d\setminus\{0\}} |x|^{1+\beta} q(dx) \in (0,\infty).
  \end{align}
  Define a measure $\nu$ on $\mathbb R^d\setminus\{0\}$ by
  \[
    \int_{\mathbb R^d\setminus\{0\}}f(x)\nu(dx)
    = \int_{S} \lambda_0(d\xi) \int_0^\infty f(r\xi) \frac{dr}{r^{2+\beta}} .
  \]
  Then, according to \cite[Remark 14.4]{Sato2013Levy}, $\nu$ is the L\'evy measure of a $(1+\beta)$-stable distribution on $\mathbb R^d$, say $\mu$, whose characteristic function is
  \[
    \hat \mu(\theta)
    = \exp \Big \{ \int_{\mathbb R^d\setminus\{0\}} (e^{-i\theta \cdot y}-1+i\theta \cdot y) \nu(dy) \Big \}
    , \quad \theta \in \mathbb R.
  \]
  Finally, according to \eqref{eq: stable branching on C+}, we have
  \begin{align}
    & \int_{\mathbb R^d\setminus\{0\}} (e^{-i\theta \cdot y}-1+i\theta \cdot y) \nu(dy)
      = \int_S \lambda_0(d\xi) \int_0^\infty (e^{-ir\theta \cdot \xi}-1+ir\theta \cdot \xi) \frac{dr}{r^{2+\beta}} \\
    & = \int_S \lambda (d\xi) \int_0^\infty (e^{-ir\theta \cdot \xi}-1+ir\theta \cdot \xi) \frac{dr}{\Gamma(-1-\beta)r^{2+\beta}}\int_0^\infty t^{1+\beta} k(\xi,dt) \\
    & = \int_S \lambda (d\xi) \int_0^\infty (i\theta\cdot \xi)^{1+\beta} t^{1+\beta} k(\xi,dt)
      = \int_S \lambda(d\xi) \int_0^\infty (i\theta \cdot t\xi)^{1+\beta} k(\xi,dt) \\
    & = \int_{\mathbb R^d} (i\theta \cdot x)^{1+\beta} q(dx).
      \qedhere
  \end{align}
\end{proof}
\begin{proof}[Proof of Proposition \ref{prop: alpha stable rv}]
  \eqref{item: stable 1}.
	Fix $t\geq 0$ and $f\in \mathcal P$.
	Note that $m_t[\theta f]$ can be rewritten as
  \[
    m_t[\theta f]
    = \eta \int_t^{t+1}e^{-\alpha s}~ds\int_{\mathbb R^d} \big(-i\theta (P_{s}^\alpha f)(x)\big)^{1+\beta} \varphi(x)~dx,
    \quad \theta \in \mathbb R.
  \]
	Therefore, according to Lemma \ref{lem: charactreisticfunction}, we only need to show that
  \begin{equation}
    \label{eq: what I want to proof}
    \int_t^{t+1}e^{-\alpha s} ~ds\int_{\mathbb R^d} | P_{s}^\alpha f(x)|^{1+\beta} \varphi(x)~dx
    \in (0, \infty).
  \end{equation}
	According to Lemma \ref{lem: upper bound for usgx}.(2), Lemma \ref{lem: property of controllable operators}.(1)--(2) and Lemma \ref{lem: property of controllable operators}.(4), we know that
  \[
    f 
    \mapsto \int_t^{t+1}e^{-\alpha s} |P_{s}^\alpha f|^{1+\beta}~ds
  \]
  is a $\theta^{1+\beta}$-controllable operator on $\mathcal P$.
  This implies that \[ x \mapsto \int_t^{t+1}e^{-\alpha s} |P_{s}^\alpha f(x)|^{1+\beta}~ds\] is an element of $\mathcal P$.
	Therefore the integral on the left-hand side of \eqref{eq: what I want to proof} is finite.
	Since $f$ is non-trivial, the integral on the left-hand side of \eqref{eq: what I want to proof} is positive.
  
  The proofs of \eqref{item: stable 2}--\eqref{item: stable 5} are similar to that of \eqref{item: stable 1}. 
  We omit the details here.
\end{proof}

Before we end this subsection, we  give  two more lemmas which will be used in the proofs of our main results in Section \ref{proofs of main results}.

\begin{lem}{\cite[Lemma 2.8]{MarksMilos2018CLT}}\label{lem: control of gk}
  For each $f\in \mathcal{P}$, let
  \begin{align}
    g_k
    := \frac{Z_1 P^{\alpha}_k f-\langle  Z_1P^{\alpha}_k f,\varphi\rangle}{e^{(\alpha-\kappa_f b)(1+\beta)t}},
    \quad k \geq 0.
  \end{align}
  Then there exists $h\in \mathcal{P}$ such that $|P_sg_k|\leq e^{-bs}h$ for all $s, k \geq 0.$
\end{lem}

\begin{lem}
  \label{control of gn} 
  Suppose that $g\in \mathcal{C}_l$.
  Put
  \begin{align}
    \bar{g}_t
    :=\frac{Z_1(-I_tg)-\langle Z_1(-I_tg),\varphi\rangle}{e^{-(\alpha-Kb)(1+\beta)t}},\quad t\geq 0.
  \end{align}
  Then there exists $h\in \mathcal{P}$ such that $|P_s\bar{g}_t| \leq e^{-bs}h$ for all $s, t\geq 0.$
\end{lem}
\begin{proof}
  First we claim that there exists $h_1\in \mathcal P$ such that
\begin{align}
  \label{ineq: control of partial derivative of Itg}
  \Big|\frac{\partial}{\partial x_i}I_tg(x)\Big| \leq e^{-(\alpha-Kb)t}h_1(x),
  \quad  t\geq 0, x\in \mathbb R^d.
\end{align}
In fact, if we put $h_1(x)=\sum_{p\in \mathcal N}|\langle g, \phi_p\rangle_\varphi\frac{\partial \phi_p}{\partial x_i}(x)| \in \mathcal P$, then
\begin{align}
  \Big|\frac{\partial}{\partial x_i}I_tg(x)\Big| =
  \Big|\sum_{p\in \mathcal N}\langle g, \phi_p\rangle_\varphi e^{-(\alpha-|p|b)t} \frac{\partial}{\partial x_i}\phi_p(x)\Big|
  \leq e^{-(\alpha-Kb)t}h_1(x)
  , \quad t\geq 0, x\in \mathbb R^d.
\end{align}
It is well known that
\[
  P_t f(x) = \int_{\mathbb R^d} f\big(x e^{-bt} + y \sqrt{1-e^{-2bt}}\big) \varphi(y) dy,
  \quad t\geq 0, x\in \mathbb R^d, f\in \mathcal B(\mathbb R^d, \mathbb R_+).
\]
It is clear that the above equality also holds for each $f\in \mathcal P$.
From this, one can easily check that for each differential $f \in \mathcal P$ with $|\nabla  f| \in \mathcal P$, we have
\[
  \frac{\partial P_t^\alpha f(x)}{\partial x_i} = e^{-bt} P_t^\alpha\Big(\frac{\partial f}{\partial x_i}\Big)(x), \quad t\geq 0, x\in \mathbb R^d.
\]
Note that if $g \in \mathcal C_l$, then $g^{1+\beta}\in \mathcal P$ and $|\nabla (g^{1+\beta})| \in \mathcal P$.
Therefore,
\begin{align}
  & \frac{\partial}{\partial x_i} P_{1-s}^\alpha  (P_s^\alpha I_t g)^{1+\beta}
    = \frac{\partial}{\partial x_i} P_{1-s}^\alpha (I_{t-s} g)^{1+\beta}
    = e^{-b(1-s)} P_{1-s}^\alpha \frac{\partial}{\partial x_i} (I_{t-s}g)^{1+\beta} \\
  & =(1+\beta)e^{-b(1-s)} P_{1-s}^\alpha \Big( (I_{t-s}g)^\beta \times \frac{\partial}{\partial x_i} (I_{t-s}g) \Big),\quad s\in [0,1], t\geq 0.
\end{align}
Using \eqref{eq: supper for Itf}, \eqref{ineq: control of partial derivative of Itg} and the above, we have that there exist a bounded function $a$ on $[0,1]$ and an $h_2 \in \mathcal P$ such that
\begin{equation}
  \label{eq: key estimation for partial}
  \Big| \frac{\partial P_{1-s}^\alpha  (P_s^\alpha I_t g)^{1+\beta}}{\partial x_i}  (x)\Big| \leq a(s) e^{-(\alpha - Kb)(1+\beta)t}h_2(x),\quad s\in [0,1],t\geq 0, x\in \mathbb R^d.
\end{equation}
Therefore, for each compact subset $A \subset \mathbb R^d$, we have
\[
  \int_0^1 \sup_{x\in A} \Big|\frac{\partial P_{1-s}^\alpha  (P_s^\alpha I_t g)^{1+\beta}}{\partial x_i}  (x)\Big| ds 
  < \infty
  , \quad t\geq 0.
\]
Using this and \cite[Theorem A.5.2]{Durrett2010Probability}, we can get
\[
  \frac{\partial}{\partial x_i} Z_1(-I_tg)
  = \eta i^{1+\beta}  \frac{\partial}{\partial x_i} \int_0^1 P_{1-s}^\alpha (I_{t-s}g)^{1+\beta} ds
  = \eta i^{1+\beta}  \int_0^1  \frac{\partial}{\partial x_i} P_{1-s}^\alpha (I_{t-s}g)^{1+\beta} ds.
\]
Therefore, using \eqref{eq: key estimation for partial} and the above, there exists  $C> 0$ such that for all $t\geq 0$,
\begin{align}
  & \Big|\frac{\partial}{\partial x_i}\bar{g}_t(x)\Big|
    = e^{(\alpha-Kb)(1+\beta)t}\Big|\frac{\partial}{\partial x_i}Z_1(-I_tg)\Big| \leq C h_2(x),
    \quad x\in \mathbb R^d.
\end{align}
Therefore there exist $c,n>0$ such that
\begin{align}
  \label{ineq: control of sup gt}
  | \nabla \bar g_t (z)|
  \leq c(1+|z|)^n
  , \quad t\geq 0, z\in \mathbb R^d.
\end{align}
Define
\[
  D(x,y)
  := \{ax+by: a,b\in [0,1]\}
  , \quad x, y \in \mathbb R^d.
\]
Then for all $x,y\in \mathbb R^d$, $D(x,y)$ is a convex set on $\mathbb R^d$.
In particular, it contains the interval connecting $xe^{-bs}+y\sqrt{1-e^{-2bs}}$ with $y$ for each $s\geq 0$.
Also, note that $ | \sqrt { 1 - \theta } - 1 | \leq \theta $ for all $ \theta \in [ 0, 1 ] $, and that $ 1 + \theta + \lambda \leq ( 1 + \theta ) ( 1 + \lambda ) $ for all $ \theta, \lambda \geq 0$.
Using the fact that $\langle \bar{g}_t,\varphi\rangle = 0$, we get that for all $s,t\geq 0, x\in \mathbb R^d$,
\begin{align}
  & |P_s\bar{g}_t(x)|
    = \Big|\int_{\mathbb R^d} \big(\bar {g}_t(xe^{-bs}+y\sqrt{1-e^{-2bs}})-\bar {g}_t (y)\big) \varphi(y)~dy \Big| \\
  & \leq \int_{\mathbb R^d} \sup_{z\in D(x,y)} |\nabla \bar g_t (z)|\Big|xe^{-bs}+y\sqrt{1-e^{-2bs}}-y \Big| \varphi(y)~dy \\
  & \leq e^{-bs}\int_{\mathbb R^d} c(1+|x|+|y|)^n (|x|+|y|) \varphi(y)~dy \\
  & \leq ce^{-bs}(1+|x|)^n\Big(|x|\int_{\mathbb R^d} (1+|y|)^n\varphi(y)dy + \int_{\mathbb R^d}(1+|y|)^n|y|\varphi(y)dy\Big).
  \qedhere
\end{align}
\end{proof}

% *** small value probability
\subsection{Small value probability}
\label{sec: Small value probability}
In this subsection, we digress briefly from our super-OU process and consider a (supercritical) \emph{continuous-state branching process (CSBP)} $\{(Y_t)_{t\geq 0}; \mathbf P_x\}$ with branching mechanism $\psi$ given by \eqref{eq: honogeneou branching mechanism}.
Such a process $\{(Y_t)_{t\geq 0}; \mathbf P_x\}$ is defined as an $\mathbb R^+$-valued Hunt process satisfying 
\[
  \mathbf P_x[e^{-\lambda Y_t}] = e^{- x v_t(\lambda)},
  \quad x\in \mathbb R^+, t\geq 0, \lambda \in \mathbb R^+,
\]
where for each $\lambda\geq 0$, $t\mapsto v_t(\lambda)$ is the unique positive solution to the equation
\begin{equation}
  \label{eq: fkpp equation for CSBP}
  v_t(\lambda) - \int_0^t \psi(v_s(\lambda))~ds = \lambda,
  \quad t\geq 0.
\end{equation}
It can be verified that for each $\mu \in \mathcal M(\mathbb R^d)$ with $x = \| \mu \|$, we have
\[
  \{(\|X_t\|)_{t\geq 0}; \mathbb P_\mu\}
  \overset{\text{law}}{=} \{(Y_t)_{t\geq 0}; \mathbf P_x\}.
\]

Our goal in this subsection is to determine how fast the probability $\mathbf P_x(0<e^{-\alpha t}Y_t \leq k_t)$ converges to $0$ when $t\mapsto k_t$ is a strictly positive function on $[0,\infty)$ such that $k_t \to 0$ and $k_t e^{\alpha t} \to \infty$ as $t\to \infty$.
Suppose that Grey's condition is satisfied, i.e., there is some constant $z' > 0$ such that $\psi(z) > 0$ for all $z>z'$, and that $\int_{z'}^\infty \psi(z)^{-1}dz < \infty$.
Also suppose that the $L \log L$ condition is satisfied, i.e.,
\[
  \int_1^\infty y \log y~\pi(dr)
  < \infty.
\]
We write $W_t = e^{-\alpha t}Y_t$ for each $t\geq 0$.
\begin{prop}
  \label{lem: control of XT}
  Suppose that $t\mapsto k_t$ is a strictly positive function on $[0,\infty)$ such that $k_t \to 0$ and $k_t e^{\alpha t} \to \infty$ as $t\to \infty$.
  Then, for each $x\geq 0$, there exist $C,\delta>0$ such that
  \[
    \mathbf P_x(0<W_t\leq k_t)
    \leq C(k_t^\delta + e^{-\delta t}), \quad t\geq 0.
  \]
\end{prop}
\begin{proof}
  Step 1. We recall some known facts about the CSBP $(Y_t)$.
  For each $\lambda \geq 0$, we denote by $t\mapsto v_t(\lambda)$ the unique positive solution of \eqref{eq: fkpp equation for CSBP}.
  Letting $\lambda \to \infty$ in \eqref{eq: fkpp equation for CSBP}, we have by monotonicity that $\bar v_t:= \lim_{\lambda \to \infty}v_t(\lambda)$ exists in $(0,\infty]$ for all $t\geq 0$, and that
  \begin{equation}
    \label{eq: svp1}
    \mathbf P_x(Y_t = 0)=e^{-x\bar v_t}, \quad t\geq 0, x\ge 0.
  \end{equation}
  It is known, see \cite[Theorems 3.5--3.8]{Li2011Measure-valued} for example, that under Grey's condition, we have (i) $0\leq \bar v_t < \infty$ for all $t>0$; and (ii) $t\mapsto \bar v_t$ is decreasing and $\bar v:= \lim_{t\to \infty} \bar v_t \in [0,\infty)$ is the largest root of $\psi(z) = 0$.
  Letting $t \to \infty$ in \eqref{eq: svp1}, we have by monotonicity that
  \[
    \mathbf P_x(\exists t \geq 0, Y_t = 0)
    = e^{-x\bar v}, \quad x\geq 0.
  \]
  
  Note that $\psi$ has derivative
  \[
    \psi'(z)
    = -\alpha + 2\rho z + \int_{(0,\infty)}(1-e^{-zy})y\pi(dy),\quad z\geq 0,
  \]
  which is increasing in $z$.
  This says that $\psi$ is a convex function.
  Also notice that $\psi'(0+)=-\alpha <0$ and that there exists $z>0$ such that $\psi(z)>0$.
  Therefore we have (i) $\bar v > 0$; (ii) $\psi(z) < 0$ on $z\in (0,\bar v)$; and (iii) $\psi(z) > 0 $ on $z\in (\bar v, \infty)$.
  It is also known, see \cite[Proposition 3.3]{Li2011Measure-valued} for example, that
  \begin{itemize}
  \item
    if $\lambda \in (0,\bar v)$, then $0<\lambda \leq v_t(\lambda)<\bar v $ and
    \begin{align}
      \label{CSBP: int}
    \int_{\lambda}^{v_t(\lambda)} \frac{dz}{-\psi(z)} = t, \quad t\geq 0;
    \end{align}
  \item
    if $\lambda \in (\bar v, \infty)$, then $\bar v < v_t(\lambda)\leq \lambda< \infty $ and
    \[
      \int_{v_t(\lambda)}^\lambda\frac{dz}{\psi(z)} = t, \quad t\geq 0.
    \]
  \end{itemize}
  By monotonicity, we have that
  \begin{equation}
    \label{eq:svp2}
    \int_{\bar v_t}^\infty \frac{dz}{\psi(z)} = t, \quad t\geq 0.
  \end{equation} 
    
  Step 2. We will show that, for each $x \geq 0$ there exists a constant $c_1>0$ such that
  \[
    \mathbf P_{x}(0< W_t\leq k_t)
    \leq c_1\big(|\bar v- v_t(k_t^{-1}e^{-\alpha t})|+|\bar v_t - \bar v|\big),
    \quad t\geq 0.
  \]
  In fact, for all $x\geq 0$ and $t\geq 0$, we have
  \begin{align}
    & \mathbf P_{x}(0<W_t \leq k_t)
      = \mathbf P_{x}( e^{-k_t^{-1}W_t}\geq e^{-1},W_t > 0) \\
    & \leq e \mathbf P_{x}[e^{-k_t^{-1} W_t};W_t > 0]
      =  e\big(\mathbf P_x[e^{-k_t^{-1} W_t}]-\mathbf P_x(W_t = 0)\big) \\ 
    & = e\big(e^{-xv_t(k_t^{-1} e^{-\alpha t})}-e^{-x\bar v_t}\big)
      \leq ex \big(|\bar v-v_t(k_t^{-1} e^{-\alpha t})|+ |\bar v_t- \bar v|\big),
  \end{align}
  as desired in this step.
  
  Step 3. We will show that there exist $c_2, \delta_1, t_0 > 0$ such that
  \[
    |\bar v_t-\bar v|
    \leq c_2e^{-\delta_1 t}
    , \quad t\geq t_0.
  \]
  In fact, since $\psi$ is a convex function, we must have $\tau:=\psi'(\bar v)>0$ and that  $\psi(z) \geq (z-\bar v)\tau$ for each $z\geq \bar v$.
  According to Grey's condition, we can find a constant $z_0 >\bar v $ such that $t_0 := \int^\infty_{z_0}\psi(z)^{-1}dz<\infty$.
  For each $t > t_0$, according to \eqref{eq:svp2}, we have
  \begin{align}
    & t - t_0 =
      \int^\infty_{\bar v_t} \frac{dz}{\psi(z)} - \int_{z_0}^\infty \frac{dz}{\psi(z)}
      = \int_{\bar v_t}^{z_0} \frac{dz}{\psi(z)} \\ 
    & \leq \int_{\bar v_t}^{z_0} \frac{dz}{(z-\bar v)\tau}
      = \frac{1}{\tau} \big(\log (z_0-\bar v) - \log(\bar v_t-\bar v)\big).
  \end{align}
  Rearranging, we get $ \bar v_t - \bar v \leq (z_0 - \bar v)e^{-\tau(t-t_0)}, $ for all $t\geq t_0$.
  This implies the desired result in this step.
  
  Step 4: 
  We will show that there exist $c_3, \delta_2, t_1>0$ such that
  \[
    |\bar v - v_t(k_t^{-1} e^{-\alpha t})|\leq
    c_3k_t^{\delta_2}, \quad t\geq t_1.
  \]
  Define $\rho_t := 1+(\log k_t)/(t\alpha)$ for all $t\geq 0$.
  By the fact that $k_t^{-1}e^{-\alpha t} = e^{-\alpha \rho_t t}$ for  all $t\geq 0$ and the condition that $k_t e^{\alpha t} \xrightarrow[t\to \infty]{} \infty$, we have $\rho_t t \xrightarrow[t\to \infty]{} \infty $.
  Since the $L\log L$ condition is satisfied, we have (see \cite{LiuRenSong2009Llog} for example), $W_t \xrightarrow[t\to \infty]{a.s.} W_\infty$, where the martingale limit $W_\infty$ is a non-degenerate positive random variable. This implies that
\[
  v_t(e^{-\alpha t}) 
  = -\log \mathbf P_1[e^{-W_t}]\xrightarrow[t\to \infty]{} - \log \mathbf P_{1}[e^{-W_\infty}] 
  =: z^* \in (0,\infty).
\]
The $L \log L$ condition also guarantees that (see again \cite{LiuRenSong2009Llog} for example) $\{W_\infty = 0\} = \{\exists t \geq 0, X_t= 0\}$  a.s. in $\mathbf P_1$. This and the non-degeneracy of $W_\infty$ imply that
\[
  z^*
  = -\log \mathbf P_1[e^{-W_\infty}] 
  < -\log \mathbf P_1(W_\infty = 0) = \bar v.
\]

Fix an arbitrary $\epsilon \in (0,\tau)$.
According to the fact that $\tau=\psi'(\bar v)>0$, there exists $z_0 \in (0,\bar v)$ such that for all $z\in (z_0, \bar v)$, we have $-\psi(z)\geq (\bar v - z)(\tau- \epsilon)$.     
Fix this $z_0$.
For $t$ large enough, we have $0<k_t^{-1}e^{-\alpha t} < v_t(k_t^{-1}e^{-\alpha t})< \bar v$.
Then using \eqref{CSBP: int} with $\lambda=k_t^{-1} e^{-\alpha t}$, we have for $t>0$ large enough,
\begin{align}
  t 
  & =\int^{v_t(k_t^{-1} e^{-\alpha t})}_{k_t^{-1} e^{-\alpha t}}\frac{dz}{-\psi(z)}
    = \Big(\int^{v_{\rho_t t}(e^{-\alpha \rho_t t})}_{e^{-\alpha \rho_t t}}  + \int^{z_0}_{v_{\rho_t t}(e^{-\alpha \rho_t t})} +\int^{v_t(k_t^{-1}e^{-\alpha  t})}_{z_0}\Big)\frac{dz}{-\psi(z)} \\
  & = \rho_t t + O(1) +\int^{v_t(k_t^{-1}e^{-\alpha t})}_{z_0} \frac{dz}{-\psi(z)},
\end{align}
where we used the fact that
\[
  \int_{v_{\rho_t t}(e^{-\alpha \rho_tt})}^{z_0} \frac{dz}{-\psi(z)} 
  \xrightarrow[t\to \infty] {} \int_{z^*}^{z_0} \frac{dz}{-\psi(z)}.
\]
Now we have, for $t$ large enough,
\begin{align}
  & t 
    \leq  \rho_t t + O(1) + \int_{z_0}^{v_t(k_t^{-1}e^{-\alpha t})} \frac{dz}{(\bar v-z)(\tau - \epsilon)} \\
  & =  \rho_t t +O(1)- \frac{1}{\tau-\epsilon}\Big( \log \big(\bar v-v_t(e^{-\alpha \rho_t t})\big) - \log(\bar v-z_0)\Big).
\end{align}
Rearranging, we get, for $t$ large enough,
\[
  e^{-t(\tau - \epsilon)} 
  \geq e^{-\rho_t t(\tau - \epsilon)+O(1)}(\bar v - v_t(e^{-\alpha \rho_t t})).
\]
Therefore, there exist $c_3>0$ and $t_1>0$ such that for all $t\geq t_1$,
\[
  \bar v - v_t(k_t^{-1} e^{-\alpha t}) 
  \leq e^{-t(\tau -\epsilon)+ (1+\frac{\log k_t}{t\alpha})t(\tau - \epsilon)+O(1)}
  \leq c_3k_t^{\frac{\tau - \epsilon}{\alpha}}.
\]
This implies the desired result in this step.

Finally, by Steps 2-4, we have for each $x\geq 0$, there exist $c_4, \delta_3, t_2 > 0$ such that
\[
  \mathbf P_{x}(0< W_t\leq k_t) 
  \leq c_4(k_t^{\delta_3}+e^{-\delta_3 t})
  , \quad t\geq t_2.
\]
Note that the left side is always bounded by $1$, so we can take $t_2 =0$ in the above statement.
\end{proof}

% ** Moments for super-OU processes
\subsection{Moments for super-OU processes}
\label{sec: Moments for super-OU processes}
In this subsection,  we want to find some upper bound for the $(1+\gamma)$-th moment of $\langle g ,X_t \rangle$, where $\gamma \in (0,\beta)$, $g\in \mathcal P$ and $\{(X_t)_{t\geq 0}; (\mathbb P_\mu)_{\mu \in \mathcal M(\mathbb R^d)}\}$ is the super OU process considered in Subsection \ref{sec: main results} satisfying Assumption \ref{asp: Greys condition} and \ref{asp: branching mechanism}.
Recall that the operator $\mathcal{I}^t_s$ is defined in \eqref{Ist}.

\begin{lem}
  \label{lem: control pair for P(M>lambda)}
  There is a $(\theta^2\vee\theta^{1+\beta})$-controller $R$ such that for all $0\leq t\leq 1$, $g\in \mathcal P$, $\lambda >0$ and $\mu\in \mathcal M_c(\mathbb R^d)$, we have
  \[
    \mathbb P_\mu ( |\mathcal{I}_0^t\langle g,X_t\rangle| > \lambda)
    \leq \frac{\lambda}{2}\int_{-2/\lambda}^{2/\lambda}\langle R|\theta g|,\mu\rangle d\theta.
  \]
\end{lem}

\begin{proof}
  Denote by $R$ the $(\theta^2\vee\theta^{1+\beta})$-controller in Lemma \ref{lem: upper bound for usgx}.(4).
  Using Lemma \ref{lem: estimate of exponential remaining} and the argument in the proof of \cite[Theorem 3.3.6]{Durrett2010Probability}, we get
  \begin{align}
    & \big|\mathbb P_\mu (|\mathcal{I}_0^t\langle g,X_t\rangle| > \lambda)\big|
      \leq \Big|\frac{\lambda}{2}\int_{-2/\lambda}^{2/\lambda}(1 - \mathbb P_\mu[e^{i\theta \mathcal{I}_0^t\langle g,X_t\rangle}])d\theta\Big| \\
    & \leq \frac{\lambda}{2}\int_{-2/\lambda}^{2/\lambda}|1-e^{\langle U_t(\theta g)-iP^\alpha_t (\theta g),\mu \rangle}|d\theta
    \leq \frac{\lambda}{2}\int_{-2/\lambda}^{2/\lambda}\langle |U_t(\theta g) - iP^\alpha_t(\theta g)|,\mu\rangle d\theta \\ 
    & \leq \frac{\lambda}{2}\int_{-2/\lambda}^{2/\lambda}\langle R|\theta g|,\mu\rangle d\theta.
      \qedhere
  \end{align}
\end{proof}
\begin{lem}
  \label{lem: temp}
  For all $h \in \mathcal P^+$ and $\mu \in \mathcal M_c(\mathbb R^d)$, there exists $C > 0$ such that for all $\kappa \in \mathbb Z_+ $, $\lambda > 0$ and $0\leq r\leq s\leq t<\infty$ with $s-r \leq 1$, we have
  \[
    \sup_{g \in \mathcal P: Q_\kappa g\leq h}\mathbb P_{\mu}(|\mathcal I_r^s\langle g, X_t\rangle|>\lambda)
    \leq C e^{\alpha r} \bigg(\Big( \frac{e^{(t-s)(\alpha - \kappa b)}}{\lambda}\Big)^{1+\beta} + \Big( \frac{e^{(t-s)(\alpha - \kappa b)}}{\lambda}\Big)^{2} \bigg).
  \]
\end{lem}

\begin{proof}
  Denote by $R$ the $(\theta^2\vee\theta^{1+\beta})$-controller in Lemma \ref{lem: control pair for P(M>lambda)}.
  Fix $h \in \mathcal P^+$, $\mu \in \mathcal M_c(\mathbb R^d)$ $\kappa \in \mathbb Z_+ $ and $0\leq r\leq s\leq t < \infty$ with $s-r \leq 1$.
  Suppose that $g\in \mathcal P$ satisfies $Q_\kappa g \leq h$.
  Using the Markov property of $X$, we get
\begin{align}
  & \mathbb P_{\mu}(|\mathcal I_r^s\langle g, X_t\rangle|>\lambda)
    = \mathbb P_\mu \Big[\mathbb P_\mu\big[|\langle P_{t-s}^\alpha g, X_{s}\rangle - \langle P_{t-r}^\alpha g, X_{r}\rangle|> \lambda\big| \mathscr F_r\big]\Big] \\
  & = \mathbb P_\mu \big[\mathbb P_{X_r}(|\langle P_{t-s}^\alpha g, X_{s-r}\rangle - \langle P_{t-r}^\alpha g, X_{0}\rangle|> \lambda)\big] \\
  & = \mathbb P_\mu \big[\mathbb P_{X_r}(|\mathcal I_0^{s-r}\langle P_{t-s}^\alpha g, X_{s-r}\rangle |> \lambda)\big]
  \leq \mathbb P_\mu \Big[ \frac{\lambda}{2}\int_{-2/\lambda}^{2/\lambda}\langle R|\theta P^\alpha_{t-s}g|,X_r\rangle d\theta \Big] \\
  & \leq \mathbb P_\mu \Big[ \frac{\lambda}{2}\int_{-2/\lambda}^{2/\lambda}\langle R|\theta e^{(t-s)(\alpha- \kappa b)}h|,X_r\rangle d\theta \Big] \\
  & \leq \mathbb P_\mu [ \langle Rh,X_r\rangle ] \frac{\lambda}{2}\int_{-2/\lambda}^{2/\lambda}(|\theta e^{(t-s)(\alpha- \kappa b)}|^{1+\beta} + |\theta e^{(t-s)(\alpha- \kappa b)}|^{2})d\theta
  \\ & =  \langle P_r^\alpha Rh,\mu\rangle \bigg(  \frac{2^{2+\beta}}{2+\beta}\Big(\frac{e^{(t-s)(\alpha- \kappa b)}}{\lambda}\Big)^{1+\beta} + \frac{2^{3}}{3}\Big(\frac{e^{(t-s)(\alpha- \kappa b)}}{\lambda}\Big)^2\bigg)
  \\ & \leq C e^{\alpha r} \bigg(\Big( \frac{e^{(t-s)(\alpha - \kappa b)}}{\lambda}\Big)^{1+\beta} + \Big( \frac{e^{(t-s)(\alpha - \kappa b)}}{\lambda}\Big)^{2} \bigg),
\end{align}
where the $C>0$ above is chosen as
\[
  C := \Big(\frac{2^{2+\beta}}{2+\beta} + \frac{2^{3}}{3} \Big)\langle Q_0Rh, \mu\rangle.
  \qedhere
\]
\end{proof}

\begin{lem}
  \label{lem: control of mgtrs}
  For all $h \in \mathcal P$, $\mu \in \mathcal M_c(\mathbb R^d)$ and $\gamma\in (0, \beta)$, there exists $C > 0$ such that for all $\kappa \in \mathbb Z_+$ and $0\leq r\leq s\leq t<\infty$ with $s-r \leq 1$, we have
  \[
    \sup_{g \in \mathcal P: Q_\kappa g\leq h} \mathbb P_\mu\big[|\mathcal I_r^s\langle g, X_t\rangle|^{1+\gamma}\big]
    \leq C e^{t\alpha+(t-s) (\gamma\alpha- (1+\gamma)\kappa b)}.
  \]
\end{lem}

\begin{proof}
  Fix $h \in \mathcal P$ and $\mu \in \mathcal M_c(\mathbb R^d)$. Let $C_0$ be the constant in the Lemma \ref{lem: temp}.
  For all $\kappa \in \mathbb Z_+$,  $0\leq r\leq s\leq t$ with $s-r \leq 1$,  $g\in \mathcal P$ with $Q_{\kappa} g \leq h$, and $c>0$, we have
\begin{align}
  & \mathbb P_\mu\big[|\mathcal I_r^s\langle g, X_t\rangle|^{1+\gamma}\big]
    = (1+\gamma)\int_0^\infty \lambda^{\gamma} \mathbb P_{\mu}(|\mathcal I_r^s\langle g, X_t\rangle|>\lambda) d\lambda \\ 
  & \leq (1+\gamma)\int_0^c \lambda^{\gamma} d\lambda +(1+\gamma)\int_c^\infty \lambda^{\gamma}\mathbb P_\mu(|\mathcal I_r^s\langle g, X_t\rangle|> \lambda) d\lambda \\
  & \leq c^{1+\gamma} + C_0  e^{\alpha r}(1+\gamma)\int_c^\infty \bigg(\Big(\frac{e^{(t-s)(\alpha - \kappa b)}}{\lambda}\Big)^{1+\beta}+\Big(\frac{e^{(t-s)(\alpha - \kappa b)}}{\lambda}\Big)^{2}\bigg)\lambda^{\gamma}d\lambda \\
  & \leq c^{1+\gamma} e^{\alpha r} + C_0e^{\alpha r}(1+\gamma)\Big(  \frac{e^{(1+\beta)(t-s)(\alpha- \kappa b)}}{(\beta - \gamma)c^{\beta - \gamma}}  + \frac{e^{2(t-s)(\alpha- \kappa b)}}{(1 - \gamma)c^{1 - \gamma}} \Big).
\end{align}
Taking $c = e^{(t-s)(\alpha- \kappa b)}$, we get
\begin{align}
  & \mathbb P_\mu\big[|\mathcal I_r^s\langle g, X_t\rangle|^{1+\gamma}\big]
    \leq e^{(1+\gamma)(t-s)(\alpha- \kappa b)} e^{\alpha r}\Big(1+ C_0 \frac{1+\gamma}{\beta - \gamma}+ C_0 \frac{1+\gamma}{1 - \gamma}\Big).
\end{align}
Note that
\begin{align}
  & (1+\beta)(t-s)(\alpha- \kappa b) + \alpha r
    = (t-s)\alpha+(t-s) (\beta\alpha- (1+\beta)\kappa b) \\
  & \leq t\alpha+(t-s) (\beta\alpha- (1+\beta)\kappa b).
\end{align}
So the desired result is true with
\[
  C
  := 1+ C_0 \frac{1+\gamma}{\beta - \gamma}+ C_0 \frac{1+\gamma}{1 - \gamma}.
  \qedhere
\]
\end{proof}
For each random variable $\{Y; \mathbb P\}$ and $p \in [1,\infty)$, we write $ \|Y\|_{\mathbb P;p} := \mathbb P[|Y|^p]^{1/p}. $
\begin{lem}
  \label{lem: control moment}
  For all $h \in \mathcal P$, $\mu \in \mathcal M_c(\mathbb R^d)$, $\gamma\in (0, \beta)$ and $\kappa \in \mathbb Z_+$, there exists a constant $C > 0$ such that for each $t\geq 0$, we have
  \begin{itemize}
  \item[(1)]
    $\sup_{g\in \mathcal P: Q_\kappa g \leq h}\|\langle g,X_t\rangle\|_{\mathbb{P}_{\mu};1+\gamma}\leq C e^{(\alpha-\kappa b)t}$ provided $\alpha\gamma > \kappa (1+\gamma)b$;
  \item[(2)]
    $\sup_{g\in \mathcal P: Q_\kappa g \leq h}\|\langle g,X_t\rangle\|_{\mathbb{P}_{\mu};1+\gamma}\leq C te^{\frac{\alpha}{1+\gamma}t}$ provided $\alpha\gamma = \kappa (1+\gamma)b$;
  \item[(3)]
    $\sup_{g\in \mathcal P: Q_\kappa g \leq h} \|\langle g,X_t\rangle\|_{\mathbb{P}_{\mu};1+\gamma}\leq C e^{\frac{\alpha}{1+\gamma}t}$ provided $\alpha\gamma < \kappa (1+\gamma)b$.
  \end{itemize}
\end{lem}
\begin{proof}
  Fix $\gamma \in (0,\beta)$ and $\mu \in \mathcal M_c(\mathbb R^d)$.
  Let $C$ be the constant in Lemma \ref{lem: control of mgtrs}.
  Using the triangle inequality, for all $\kappa\in \mathbb Z_+$, $g \in \mathcal P$ with $Q_\kappa g \leq h$ and $t\geq 0$, we have
\begin{align}
  & \|\langle g,X_t\rangle\|_{\mathbb P_\mu;1+\gamma}
    \leq \sum_{l=0}^{\lfloor t\rfloor - 1}\big\| \mathcal{I}_{t-l-1}^{t-l}\langle g,X_t\rangle \big\|_{\mathbb P_\mu;1+\gamma}+\big\| \mathcal{I}_{0}^{t-\lfloor t \rfloor}\langle g,X_t\rangle  \big\|_{\mathbb P_\mu;1+\gamma} + |\langle P^\alpha_t g,\mu\rangle| \\ 
  & \leq C^{\frac{1}{1+\gamma}} e^{\frac{\alpha}{1+\gamma}t} \sum_{l=0}^{\lfloor t\rfloor} e^{\frac{\gamma\alpha-\kappa (1+\gamma)b}{1+\gamma} l} + e^{(\alpha - \kappa b)t} \langle h,\mu\rangle.
\end{align}
By calculating the sum on the right, we get the desired result.
\end{proof}

% * Proofs of main results
\section{Proofs of main results}
\label{proofs of main results}
In this section, we will prove the main results of this paper. Recall that $\mathcal{M}_c(\mathbb{R}^d)$ is the space of all finite Borel measures of compact support on $\mathbb{R}^d$.
For simplicity, we will write $\mathbb{\widetilde{P}}_{\mu}=\mathbb{P}_{\mu}(\cdot|D^c)$ in this section.

% ** The large branching rate regime: law of large numbers
\subsection{The large branching rate regime: law of large numbers}
\label{sec: large rate lln}

In this subsection, we prove Theorem \ref{thm: law of large number}.
We will return to the large branching rate regime in Subsection \ref{sec: large rate clt} to prove Theorem \ref{thm: large clt}.

To prove Theorem \ref{thm: law of large number}, we first prove the almost sure and $L^{1+\gamma}(\mathbb{P}_{\mu})$ convergence of a family of martingales for $\gamma\in (0, \beta)$. Recall that $L$ is the infinitesimal generator of the OU-process.  For $f\in \mathcal{P}\cap C^2(\mathbb R^d)$ and  $a\in \mathbb R$, we define
\begin{align}
  \label{defmartingale}
  M_t^{f,a}
  :=e^{-(\alpha-ab)t}\langle f,X_t\rangle-\int_0^t e^{-(\alpha-ab)s}\langle (L+ab)f, X_s\rangle~ ds.
\end{align}
Let $(\mathscr{F}_t)_{t\geq 0}$ be the natural filtration of $X$.  The following lemma says that $\{M_t^{f,a}: t\geq 0\}$ is a martingale with respect to $(\mathscr{F}_t)_{t\geq 0}$.
\begin{lem}
  \label{lemma25}
  For all $f\in \mathcal{P}\cap C^2(\mathbb R^d)$, $a\in \mathbb R$ and $\mu\in \mathcal M_c(\mathbb R^d)$, the process $(M_t^{f,a})_{t\geq 0}$ is a $\mathbb P_\mu$-martingale with respect to $(\mathscr F_t)_{t\geq 0}$.
\end{lem}
\begin{proof}
  Put$\bar{f} :=(L+ab)f$.
  It follows easily from Ito's formula that
\begin{equation}
  \label{Theorem55}
  P_t^{ab}f(x)
  = f(x)+\int_0^t P_s^{ab}\bar{f}(x)~ds,\quad t\geq 0,x\in \mathbb R^d,
\end{equation}
where $P_t^{ab} := e^{abt}P_t$.
For $0\leq s\leq t$, we have
\begin{align}
  \label{martingale1}
  & \quad\mathbb{P}_{\mu}[M_t^{f,a}|\mathscr{F}_s]
    =e^{-(\alpha-ab)t}\mathbb{P}_{\mu}\left[\langle f,X_t\rangle|\mathscr{F}_s\right]-\mathbb{P}_{\mu}\Big[\int_0^t e^{-(\alpha-ab)u}\langle \bar{f}, X_u\rangle~ du\Big|\mathscr{F}_s\big] \\
  & =e^{-(\alpha-ab)t}\langle P_{t-s}^{\alpha}f, X_s\rangle-\int_0^s e^{-(\alpha-ab)u}\langle \bar{f}, X_u\rangle~ du - \int_s^t e^{-(\alpha-ab)u}\langle P_{u-s}^{\alpha} \bar{f},X_s\rangle~ du.
\end{align}
Using \eqref{Theorem55} and Fubini's theorem, we have
\begin{align}
  & \int_s^t e^{-(\alpha-ab)u}\langle P_{u-s}^{\alpha} \bar{f},X_s\rangle~ du=e^{-(\alpha-ab)s}\int_s^t\langle P_{u-s}^{ab}\bar{f},X_s\rangle~du\\
  & = e^{ - ( \alpha - ab ) s } \Big \langle \int_0^{t-s} P_{u}^{ab} \bar{f}~ du, X_s \Big \rangle
    = e^{-(\alpha-ab)s}\left(\langle P_{t-s}^{ab}f,X_s\rangle-\langle
    f, X_s \rangle \right) \\
  & = e^{-(\alpha-ab)t}\langle P_{t-s}^{\alpha}f, X_s\rangle - e^{ - ( \alpha - ab ) s} \langle
    f,X_s\rangle.
\end{align}
Using this and \eqref{martingale1}, we get the desired result.
\end{proof}

Recall that, for $p=(p_1,...,p_d)\in \mathbb Z_+^d$,  $\phi_p$ is an eigenfunction of $L$ corresponding to the eigenvalue $-|p|b$. Define
\[
  H_t^p
  :=e^{-(\alpha-|p|b)t}\langle\phi_p,X_t\rangle, \quad t\geq 0.
\]

\begin{lem}
  \label{lemma26}
  For any $\mu\in \mathcal M_c(\mathbb R^d)$, $(H^p_t)_{t\geq 0}$ is a $\mathbb P_\mu$-martingale with respect to $(\mathscr F_t)$. Moreover, if $\alpha\beta>|p|b(1+\beta)$, then, for all $\gamma\in (0, \beta)$, we have $\sup_{t\geq 0}\|H_t^p\|_{\mathbb P_\mu;1+\gamma}< \infty$ and $ H_{\infty}^p := \lim_{t\rightarrow \infty}H_t^p $ exists $\mathbb{P}_{\mu}$-a.s and in $L^{1+\gamma}(\mathbb{P}_{\mu}).$
\end{lem}
\begin{proof}
  It follows from Lemma \ref{lemma25} that $(H_t^p)_{t\geq 0}$ is a $\mathbb P_\mu$-martingale for any $\mu\in \mathcal M_c(\mathbb R^d)$.
  There exists $\gamma_0 \in (0,\beta)$ close enough to $\beta$ such that for $\gamma\in [\gamma_0, \beta)$, we have $\alpha\gamma>|p|(1+\gamma)b$.
  Using  Lemma \ref{lem: control moment} and the fact $\kappa_{\phi_p}=|p|$, we get that, for $\gamma\in [\gamma_0, \beta)$, there exists $C_{\gamma, \mu, p}>0$ (depending on $\gamma$, $\mu$ and $p$) such that
\[
 	\|H_t^p\|_{\mathbb P_\mu;1+\gamma}
  \leq C_{\gamma, \mu, p} e^{-(\alpha-|p|b)t}e^{(\alpha-|p|b)t}
  = C_{\gamma, \mu, p}
  , \quad t\geq 0.
\]
For all $\gamma\in (0, \gamma_0)$ and any $\mu\in \mathcal M_c(\mathbb R^d)$, we have
\[
	\| H_t^p \|_{\mathbb P_\mu;1+\gamma}
	\leq \| H_t^p \|_{\mathbb P_\mu;1+\gamma_0}
  < C_{\gamma_0, \mu, p},
	\quad t\geq 0.
\]

Hence, for all $\gamma \in (0,\beta)$ and $\mu\in \mathcal M_c(\mathbb R^d)$, the martingale is bounded in $L^{1+\gamma}(\mathbb{P}_{\mu})$ and hence converges in $L^{1+\gamma}(\mathbb{P}_{\mu}) $ and almost surely, by \cite[Theorem 5.4.5]{Durrett2010Probability}.
\end{proof}

In particular, when $p=0$, $H_t^0$ reduces to $H_t:=e^{-\alpha t}\|X_t\|$, thus, as $t\rightarrow \infty$, $H_t$ converges to $H_{\infty}$, $\mathbb{P}_{\mu}$-almost surely and in $L^{1+\gamma}(\mathbb{P}_{\mu})$  for any $\mu\in \mathcal M_c(\mathbb R^d)$.

\begin{lem}
  \label{lem: control of wt}
  Let $\mu\in \mathcal M_c(\mathbb R^d)$.
  For all $\gamma\in (0,\beta)$ and $p\in \mathbb{Z}_+^d$ with $\alpha \gamma > |p|b(1+\gamma)$, there exists $C> 0$ such that for all $0\leq s<t$,
\[
  \|H^p_t-H^p_s\|_{\mathbb{P}_{\mu};1+\gamma}
  \leq C e^{-\frac{ 1}{1+\gamma}(\alpha\gamma-|p|b(1+\gamma))s}.
\]
Moreover, we have
\[
  \|H^p_\infty-H^p_s\|_{\mathbb{P}_{\mu};1+\gamma}
  \leq C e^{-\frac{ 1}{1+\gamma}(\alpha\gamma-|p|b(1+\gamma))s},\quad s\geq 0.
\]
\end{lem}

\begin{proof}
  The second assertion follows immediately from the first, so we only prove the first assertion.
  Fix $\gamma \in (0,\beta)$, $p\in \mathbb{Z}_+^d$ and $\mu\in \mathcal M_c(\mathbb R^d)$.
  By Lemma \ref{lem: control of mgtrs} with $g=\phi_p$ and $k=|p|$, there exists $C_1>0$ such that for all $0\leq r\leq s $ with $s-r\leq1$, we have
\begin{align}
  \mathbb{P}_{\mu}\Big[\big|e^{(\alpha-|p|b)(t-s)}\langle\phi_p, X_s\rangle-e^{(\alpha-|p|b)(t-r)}\langle\phi_p, X_r\rangle\big|^{1+\gamma}\Big]
  \leq C_1e^{\alpha t+(t-s)(\alpha\gamma-(1+\gamma)|p|b)}.
\end{align}
Dividing both sides by $e^{(\alpha-|p|b) t(1+\gamma)}$, we get
\begin{align}
  \mathbb{P}_{\mu}\big[|H^p_s-H^p_r|^{1+\gamma}\big]
  \leq  C_1 e^{-(\alpha\gamma-(1+\gamma)|p|b)s}.
\end{align}
Thus there exists $C_2>0$ such that for any $0\leq s<t$,
\begin{align}
	& \|H^p_t-H^p_s\|_{\mathbb{P}_{\mu};1+\gamma} \\
	& \leq \|H^p_{\lfloor s \rfloor+1}-H^p_s\|_{\mathbb{P}_{\mu};1+\gamma}+\sum_{k=\lfloor s \rfloor+1}^{\lfloor t \rfloor}\|H^p_{k+1}-H^p_{k}\|_{\mathbb{P}_{\mu};1+\gamma}+\|H^p_t-H^p_{\lfloor t \rfloor+1}\|_{\mathbb{P}_{\mu};1+\gamma} \\
	& \leq C_1^{\frac{1}{1+\gamma}} \Big(e^{-\frac{(\alpha \gamma-(1+\gamma)|p|b) s}{1+\gamma}}+\sum_{k=\lfloor s \rfloor+1}^{\lfloor t \rfloor}e^{-\frac{(\alpha \gamma-(1+\gamma)|p|b) k}{1+\gamma}}+ e^{-\frac{(\alpha \gamma-(1+\gamma)|p|b t}{1+\gamma}}\Big)
   \leq C_2e^{-\frac{(\alpha \gamma-(1+\gamma)|p|b)}{1+\gamma}s}.
   \qedhere
\end{align}	
\end{proof}

\begin{proof}[Proof of Theorem \ref{thm: law of large number}]
	Fix $f \in \mathcal P$ such that $\alpha \beta > \kappa_f b (1+\beta)$ and $\mu \in \mathcal M_c(\mathbb R^d)$.
	Write
  \begin{equation}
    f
    = \sum_{p\in \mathbb Z_+^d:|p|\geq \kappa_f}\langle f,\phi_p\rangle_\varphi \phi_p
    =: \sum_{p\in \mathbb Z_+^d:|p|= \kappa_f}\langle f,\phi_p\rangle_\varphi \phi_p+\widetilde{f}.
  \end{equation}
	Then
  \begin{align}
    & e^{-(\alpha-\kappa_fb)t}\langle f,X_t\rangle=
      \sum_{p\in \mathbb Z_+^d:|p|= \kappa_f}\langle f,\phi_p\rangle_\varphi H_t^p+e^{-(\alpha-\kappa_fb)t} \langle \widetilde{f},X_t\rangle,
      \quad t\geq 0.
  \end{align}
	According to Lemma \ref{lemma26}, we have
\begin{equation}
  \label{as convergence}
  \sum_{p\in \mathbb{Z}_+^d:|p|= \kappa_f}\langle f,\phi_p\rangle_\varphi H_t^p
  \xrightarrow[t\to \infty]{} \sum_{p\in \mathbb{Z}_+^d:|p|=\kappa_f}\langle f, \phi_p\rangle_{\varphi} H_{\infty}^p,
\end{equation}
$\mathbb{P}_{\mu}$-a.s. and in $L^{1+\gamma}(\mathbb{P}_{\mu})$ for any $\gamma\in(0,\beta)$.
Therefore, it suffices to show that
\begin{equation}
  J_t
  :=e^{-(\alpha-\kappa_fb)t}\langle \widetilde{f},X_t\rangle,
  \quad t\geq 0,
\end{equation}
converges in $L^{1+\gamma}(\mathbb{P}_{\mu})$ for all $\gamma\in(0,\beta)$, and converges a.s. provided $f\in C^2(\mathbb R^d)$ satisfies $D^2f\in \mathcal{P}$.

Step 1. Let $g\in \mathcal P$.
Let $\kappa > 0$ be such that $\kappa < \kappa_g$ and $\kappa < \frac{\alpha \beta}{b(1+\beta)}$.
We will show that for each $\gamma \in (0,\beta)$ there exist $C_1,\delta_1 > 0$ such that
\[
	\|e^{-(\alpha - \kappa b)t} \langle g, X_t\rangle\|_{\mathbb P_\mu;1+\gamma}
	\leq C_1 e^{-\delta_1 t},
	\quad t\geq 0.
\]
In order to do this, we choose a $\gamma_0 \in (0,\beta)$ close enough to $\beta$ such that, for each $\gamma \in [\gamma_0, \beta)$, we have $\kappa < \frac{\alpha\gamma}{b(1+\gamma)}$.
Then, according to Lemma \ref{lem: control moment}, we have, for each $\gamma \in (0,\beta)$,
\begin{enumerate}
\item
	if $\gamma \in [\gamma_0, \beta)$ and $\alpha\gamma> \kappa_g (1+\gamma)b$, then there exists $C_2>0$ such that
  \[
    \|e^{-(\alpha - \kappa b)t} \langle g, X_t\rangle\|_{\mathbb P_\mu;1+\gamma}
    \leq C_2 e^{-(\alpha-\kappa b)t}e^{(\alpha-\kappa_g b)t}
    \leq C_2  e^{-(\kappa_g - \kappa )bt},
    \quad t\geq 0;
  \]
\item
	if $\gamma \in [\gamma_0, \beta)$ and $\alpha\gamma=\kappa_g(1+\gamma)b$, then there exists $C_3>0$ such that
  \[
    \|e^{-(\alpha - \kappa b)t} \langle g, X_t\rangle\|_{\mathbb P_\mu;1+\gamma}
    \leq C_3 t e^{-(\alpha - \kappa b)t}e^{\frac{\alpha}{1+\gamma}t}
    = C_3 t e^{-(\frac{\alpha \gamma}{1+\gamma} - \kappa b)t},
    \quad t\geq 0;
  \]
\item
	if $\gamma \in [\gamma_0, \beta)$ and $\alpha\gamma < \kappa_g (1+\gamma)b$, then there exists $C_4>0$ such that
  \[
    \|e^{-(\alpha - \kappa b)t} \langle g, X_t\rangle\|_{\mathbb{P}_{\mu};1+\gamma}
    \leq C_4  e^{-(\alpha - \kappa b)t}e^{\frac{\alpha}{1+\gamma}t}
    = C_4  e^{-(\frac{\alpha \gamma}{1+\gamma} - \kappa b)t},
    \quad t\geq 0;
  \]
\item
	if $\gamma \in (0,\gamma_0)$ then, thanks to (1)--(3) above and the fact that \[\|e^{-(\alpha - \kappa b)t} \langle g, X_t\rangle\|_{\mathbb{P}_{\mu};1+\gamma}
    \leq \|e^{-(\alpha - \kappa b)t} \langle g, X_t\rangle\|_{\mathbb{P}_{\mu};1+\gamma_0},\] there exist $C_5, \delta_2 >0$ such that
  \[
    \|e^{-(\alpha - \kappa b)t} \langle g, X_t\rangle\|_{\mathbb{P}_{\mu};1+\gamma}
    \leq C_5e^{-\delta_2 t},
    \quad t\geq 0.
  \]
\end{enumerate}
Thus, the desired conclusion in this step is valid.
In particular, by taking $g = \widetilde f$ and $\kappa = \kappa_f$, we get that $J_t$ converges to $0$ in $L^{1+\gamma}(\mathbb{P}_{\mu})$ for any $\gamma\in(0,\beta)$.

Step 2.
We further assume that $f\in C^2(\mathbb R^d)$ and $D^2f \in \mathcal{P}$.
We will show that $J_t$ converges to $0$ almost surely.
For $a \geq 0$, $ t\geq 0$, and $g\in \mathcal{P}\cap C^2(\mathbb{R}^d)$ satisfying $D^2g\in \mathcal{P}$, we define
\begin{align}
	L_t^{g,a}
  & :=\int_0^t e^{-(\alpha-ab)s}\langle (L+ab)g,X_s\rangle ds,\\
  Y_t^{g,a}
  & :=\int_0^t e^{-(\alpha-ab)s}|\langle (L+ab)g,X_s\rangle|ds.
\end{align}
Now choose $a_0 \in (\kappa_{f}, \kappa_f + 1)$ close enough to $\kappa_f$ so that $a_0 < \frac{\alpha \beta}{b(1+\beta)}$.
According to \eqref{defmartingale}, 
\begin{align}
  J_t
  = e^{-(a_0-\kappa_f)bt} (M_t^{\widetilde{f}, a_0}+L_t^{\widetilde{f}, a_0}),
  \quad t\geq 0.
\end{align}
So we only need to show that
\begin{align}
  e^{-(a_0-\kappa_f)b t}M_t^{\widetilde{f},a_0}
  \xrightarrow[t\to \infty]{} 0,
  \quad e^{-(a_0-\kappa_f)b t}L_t^{\widetilde{f},a_0}
  \xrightarrow[t\to \infty]{} 0
  \quad \mathbb{P}_{\mu}\text{-a.s.}
\end{align}
Notice that $\kappa_{(L+a_0 b)\widetilde{f}}\geq \kappa_{\widetilde{f}}\geq \kappa_f+1 > a_0$.
By Step 1, for any fixed $\gamma\in (0,\beta)$, there exist $C_6, \delta_3>0$ such that for each $t\geq 0$,
\begin{equation}
  \| e^{-(\alpha-a_0 b)t}\langle \widetilde{f},X_t\rangle)\|_{\mathbb{P}_{\mu};1+\gamma}
  \leq C_6 e^{-\delta_3 t},
  \quad \|e^{-(\alpha-a_0 b)t}\langle L\widetilde{f}+a_0 b\widetilde{f},X_t\rangle\|_{\mathbb{P}_{\mu};1+\gamma}
  \leq C_6 e^{-\delta_3 t}.
\end{equation}
Now, by the triangle inequality, for each $t\geq 0$,
\begin{align}
  & \|L_t^{\widetilde{f},a_0}\|_{\mathbb{P}_{\mu};1+\gamma}
    \leq\|Y_t^{\widetilde{f},a_0}\|_{\mathbb{P}_{\mu};1+\gamma} \\
  & \leq \int_0^t \|e^{-(\alpha-a_0 b)s}\langle L\widetilde{f}+a_0 b\widetilde{f},X_s\rangle\|_{\mathbb{P}_{\mu};1+\gamma}ds\leq C_6 \int_0^t e^{-\delta_3 s}ds\leq\frac{C_6}{\delta_3}.
\end{align}
Since $Y_t^{\widetilde{f},a_0}$ is increasing in $t$, it converges to some finite random variable $Y_{\infty}^{\widetilde{f},a_0}$ almost surely and in $L^{1+\gamma}(\mathbb{P}_{\mu})$.
Consequently,  we have
\begin{align}
  \lim_{t\rightarrow \infty}e^{-(a_0 - \kappa_f)bt}|L_t^{\widetilde{f},a_0}|
  \leq  \lim_{t\rightarrow \infty}e^{-(a_0 - \kappa_f)bt}|Y_t^{\widetilde{f},a_0}|=0, \quad
  \mathbb P_\mu\text{-a.s.}
\end{align}
On the other hand, the martingale $M_t^{\widetilde{f},a_0}$ satisfies
\begin{align}
  \|M_t^{\widetilde{f},a_0}\|_{\mathbb{P}_{\mu};1+\gamma}
  \leq \|e^{-(\alpha-a_0 b)t}\langle \widetilde{f},X_t\rangle)\|_{\mathbb{P}_{\mu};1+\gamma}+\|L_t^{\widetilde{f},a_0}\|_{\mathbb{P}_{\mu};1+\gamma}
  \leq C_6(e^{-\delta_3 t}+\frac{1}{\delta_3}),
  \quad t\geq 0.
\end{align}
This implies that the martingale converges almost surely.
Consequently,
\[
	\lim_{t\rightarrow\infty} e^{-(a_0-\kappa_f)bt}M_t^{\widetilde{f},a_0}
	= 0,
	\quad \mathbb P_\mu\text{-a.s.}.
  \qedhere
\]
\end{proof}

% *** The critical branching rate regime
\subsection{The critical branching rate regime}
\label{sec:critical}

Recall that $\mathcal I_s^t$ is defined in \eqref{Ist}.
\begin{lem}
  \label{lem: mainlemma}
  Let $f\in \mathcal{P}$ be non-trivial and satisfy $\alpha\beta\leq \kappa_fb(1+\beta)$.
  Then for all $k\geq 0$ and $\mu \in \mathcal M_c(\mathbb R^d)$, under $\mathbb{P}_{\mu}(\cdot | D ^c)$, we have
\begin{equation}
    \gamma_{t,k}
    := \frac{\mathcal I_{t-k-1}^{t-k}\langle f ,X_t\rangle}{(e^{\alpha (k+1)}\|X_{t-k-1}\|)^{\frac{1}{1+\beta}}}\xrightarrow{d}\zeta_k, \quad t\rightarrow \infty, \label{limitdistribution1}
\end{equation}
where $\zeta_k$ is a $(1+\beta)$-stable random variable with characteristic function $\theta\mapsto \exp(m_k[\theta f])$.
\end{lem}
\begin{proof}
  We only need to show that for all $\mu \in \mathcal M_c(\mathbb R^d), f\in \mathcal P, k \geq 0$,
\begin{equation}
  \mathbb{P}_{\mu}[\exp(i\gamma_{t,k}); D^c]
  \xrightarrow[t\rightarrow \infty]{}\mathbb{P}_{\mu}(D^c)\exp(m_k[f]).
\end{equation}
In fact, once we prove this, we can replace $f$ by $\theta f$, with $\theta \in \mathbb R$ being  arbitrary,  to get the desired result.
In the rest of the proof we fix a $\mu \in \mathcal M_c(\mathbb R^d)$ and an $f\in \mathcal P$.
Define
\[
  A_t(\epsilon)
  :=\{ \|X_t\| \geq e^{(\alpha - \epsilon)t} \},
  \quad t\geq 0, \epsilon > 0.
\]

Step 1. We will show that for all $\epsilon > 0, k\geq 0$ and $t>k+1$, we have
\begin{equation}
  \big|\mathbb{P}_{\mu}\big[e^{i\gamma_{t,k}}-e^{m_k[f]}; D^c\big]\big|
  \leq J_1(t,k,\epsilon)+J_2(t,k,\epsilon)+J_3(t,k,\epsilon),
\end{equation}
where
\begin{align}
\label{eq: Def of Ji}
  J_1(t,k,\epsilon)
  & := \mathbb{P}_{\mu}\big[|\langle Z'''_1(\theta_{t,k}P^\alpha_k f), X_{t-k-1}\rangle|; A_{t-k-1}(\epsilon) \big],
  \\ J_2(t,k,\epsilon)
  & := \mathbb{P}_{\mu}\big[|\langle Z_1(\theta_{t,k}P^\alpha_k f),X_{t-k-1}\rangle-m_k[f]|; A_{t-k-1}(\epsilon)\big],
  \\ J_3(t,k, \epsilon)
  & :=2\mathbb{P}_{\mu}(A_{t-k-1}(\epsilon)\Delta D^c),
  \\ \theta_{t,k}
  & := (e^{\alpha( k+1)}\|X_{t-k-1}\|)^{-\frac{1}{1+\beta}}.
\end{align}
In fact, it follows from \eqref{eq: key equality}, the definitions of $U_1$, $Z'''_1$ and $Z_1$, that for all $k\geq 0, t\geq k+1$,
\begin{align}
  \label{eq: need1}
  & \mathbb{P}_{\mu}[e^{i\gamma_{t,k}}|\mathscr{F}_{t-k-1}]
    =\mathbb{P}_{\mu}[e^{i\theta_{t,k}\langle P^\alpha_k f,X_{t-k}\rangle-i\theta_{t,k}\langle P^\alpha_{k+1} f, X_{t-k-1}\rangle}|\mathscr{F}_{t-k-1}] \\
  & =e^{\langle (U_1 - iP^\alpha_1 ) (\theta_{t,k}P^\alpha_k f),X_{t-k-1}\rangle}
    =e^{\langle (Z_1 + Z'''_1) (\theta_{t,k}P^\alpha_k f),X_{t-k-1}\rangle}.
\end{align}
From Proposition \ref{prop: alpha stable rv}.(1), we have $\operatorname {Re} m_t[f] < 0$. 
Using this, \eqref{eq: need1}, \eqref{eq: -v has positive real part} and the fact $|e^{-x} - e^{-y}| \leq |x-y|$ for all $x,y \in \mathbb C_+$, we get that for all $k\geq 0$, $t\geq k+1$ and $\epsilon> 0$,
\begin{align}
  \label{eq: inequality that will used later}
  & \big|\mathbb{P}_{\mu}\big[e^{i\gamma_{t,k}}-e^{m_k[f]}; D^c\big]\big|
    \leq \mathbb{P}_{\mu}\Big[\big| \mathbb{P}_{\mu}[e^{i\gamma_{t,k}}-e^{m_k[f]}; D^c | \mathscr F_{t-k-1}]\big|\Big] \\
  & \leq \mathbb{P}_{\mu}\Big[\big| \mathbb{P}_{\mu}[e^{i\gamma_{t,k}}-e^{m_k[f]}; A_{t-k-1}(\epsilon)| \mathscr F_{t-k-1}]\big| + 2\mathbb P_\mu(A_{t-k-1}(\epsilon) \Delta D^c| \mathscr F_{t-k-1})\Big] \\
  & = \mathbb{P}_{\mu}\Big[ \big|\mathbb{P}_{\mu}[e^{i\gamma_{t,k}}| \mathscr F_{t-k-1}]-e^{m_k[f]}\big|;A_{t-k-1}(\epsilon)\Big] + J_3(t,k,\epsilon) \\
  & \leq \mathbb{P}_{\mu}\big[|e^{\langle (Z_1+Z'''_1) (\theta_{t,k}P^\alpha_k f),X_{t-k-1}\rangle}-e^{m_k[f]}|;A_{t-k-1}(\epsilon)\big]+  J_3(t,k,\epsilon) \\
  & \leq \mathbb{P}_{\mu}\big[|\langle (Z_1+Z'''_1)(\theta_{t,k}P^\alpha_k f),X_{t-k-1}\rangle-m_k[f]|;A_{t-k-1}(\epsilon)\big]+  J_3(t,k,\epsilon) \\
  & \leq J_1(t,k,\epsilon) + J_2(t,k,\epsilon)+J_3(t,k,\epsilon).
\end{align}

Step 2. We will show that for $\epsilon>0$ small enough, there exist  $C_1, \delta_1>0$ such that
\begin{equation}
  \label{lemma31q}
  J_1(t,k,\epsilon)
  \leq C_1e^{-\delta_1 (t-k)},
  \quad k\geq 0, t\geq k+1.
\end{equation}
In fact, let $\delta_0 >0$ be the constant in Lemma \ref{lem: upper bound for usgx}.(7) and let $R$ be the corresponding $(\theta^{2+\beta}\vee \theta^{1+\beta+\delta_0})$-controller.
Then, we have for all $k\geq 0$, $t\geq k+1$ and $\epsilon> 0$,
\begin{align}
  & |Z'''_1(\theta_{t,k}P^\alpha_k f)|\mathbf{1}_{A_{t-k-1}(\epsilon)}
    \leq R(|\theta_{t,k}P^\alpha_k f|)\mathbf{1}_{A_{t-k-1}(\epsilon)}
  \leq R \Big(\frac{e^{(\alpha-\kappa_fb)k} Qf}{(e^{\alpha (k+1)}e^{(\alpha-\epsilon)(t-k-1)})^\frac{1}{1+\beta}}\Big) \\
  & \leq \sum_{\rho \in \{\delta_0, 1\}}\Big(\frac{e^{(\alpha-\kappa_fb)k}}{(e^{\alpha (k+1)}e^{(\alpha-\epsilon)(t-k-1)})^\frac{1}{1+\beta}}\Big)^{1+\beta+ \rho} RQf \\
  & \leq\sum_{\rho \in \{\delta_0, 1\}}e^{\frac{1+\beta + \rho}{1+\beta}(\alpha\beta-\kappa_fb(1+\beta))k}e^{-\frac{1+\beta+\rho}{1+\beta} (\alpha-\epsilon)(t-k-1)}RQf
    \leq \sum_{\rho \in \{\delta_0,1\}}e^{-\frac{1+\beta+\rho}{1+\beta}(\alpha-\epsilon)(t-k-1)}RQf,
\end{align}
where $Q$ is defined by \eqref{Q_k}.
Thus for all $k\geq 0$, $t\geq k+1$ and $\epsilon> 0$,
\begin{align}
  \label{eq: estimate of J1}
  J_1(t,k,\epsilon)
  & \leq \sum_{\rho \in \{\delta_0,1\}}e^{-\frac{1+\beta+\rho}{1+\beta}(\alpha-\epsilon)(t-k-1)}\mathbb{P}_{\mu}[\langle RQf,X_{t-k-1}\rangle]\\
  & \leq \sum_{\rho \in \{\delta_0,1\}} \langle Q_0 RQf, \mu \rangle e^{-(\alpha\frac{\rho}{1+\beta}-\epsilon\frac{1+\beta+\rho}{1+\beta})(t-k-1)},
\end{align}
where $Q$ is defined by \eqref{Q_k}.
By taking $\epsilon>0$ small enough, we get the desired result in this step.

Step 3.
We will show that for $\epsilon>0$ small enough there exist $C_2, \delta_2>0$ such that $ J_2(t,k,\epsilon) \leq C_2e^{-\delta_2 (t-k)},$ for each $k\geq 0, t\geq k+1.$

We first claim that for each $k\geq0$ and $g\in \mathcal P$, $m_k[g]=e^{-\alpha(k+1)}\langle Z_1P_k^{\alpha}g, \varphi \rangle$. 
In fact, by Fubini's theorem and the definitions of $Z_1$ and $m_k$, we have for all $k\geq 0$ and $g\in \mathcal P$.
\begin{align}
	& m_k[g]
   = \eta \int_k^{k+1}e^{-\alpha s}~ds\int_{\mathbb R^d} (-iP_{s}^\alpha g(x))^{1+\beta} \varphi(x)~dx\\
  & = \eta \int_0^1e^{-\alpha (k+s)}~ds\int_{\mathbb R^d} (-iP_{s+k}^\alpha g)^{1+\beta}\varphi(x)~dx\\
  & = e^{-\alpha(k+1)} \cdot \eta \int_0^{1}e^{\alpha(1- s)}~ds\int_{\mathbb R^d} (-iP_{s+k}^\alpha        g(x))^{1+\beta} \varphi(x)~dx\\
  & =e^{-\alpha(k+1)}\langle \eta \int_0^1P_{1-s}^{\alpha}(-iP_{s+k}^\alpha g(x))^{1+\beta}~ds, \varphi \rangle
  =e^{-\alpha(k+1)}\langle Z_1(P_k^{\alpha}g), \varphi \rangle.
\end{align}

Therefore, for all $k\geq 0$, $t\geq k+1$ and $\epsilon> 0$,
\begin{align}
  \langle Z_1(\theta_{t,k}P^\alpha_k f),X_{t-k-1}\rangle-m_k[f]
    & = \theta_{t,k}^{1+\beta} \langle Z_1P^\alpha_k f,X_{t-k-1}\rangle - e^{-\alpha (k+1)}\langle  Z_1P^\alpha_k f,\varphi\rangle
  \\ & =e^{-\alpha (k+1)}\Big(\frac{\langle Z_1P^\alpha_k f ,X_{t-k-1}\rangle}{\|X_{t-k-1}\|}-\langle  Z_1P^\alpha_k f ,\varphi\rangle\Big),
\end{align}
and
\begin{align}
  \label{eq: prevJ2}
  J_2(t,k,\epsilon)
  & = \mathbb P_\mu\big[|\langle Z_1(\theta_{t,k}P^\alpha_k f),X_{t-k-1}\rangle-m_k[f]|;A_{t-k-1}(\epsilon)\big] \\ 
  & =e^{-\alpha( k+1)}\mathbb{P}_{\mu}\bigg[\Big|\frac{\langle Z_1P^{\alpha}_k f,X_{t-k-1}\rangle}{\|X_{t-k-1}\|}-\langle  Z_1P^{\alpha}_k f,\varphi\rangle\Big|;A_{t-k-1}(\epsilon)\bigg]\nonumber\\
  & \leq e^{-\alpha (k+1)}e^{-(\alpha-\epsilon)(t-k-1)}e^{(\alpha-\kappa_f b)(1+\beta)k} \mathbb{P}_{\mu}\left[\left|\langle g_k,X_{t-k-1}\rangle\right|\right],
\end{align}
where
\[
  g_k
  = \frac{Z_1 P^{\alpha}_k f-\langle  Z_1P^{\alpha}_k f,\varphi\rangle}{e^{(\alpha-\kappa_f b)(1+\beta)k}},
  \quad k \geq 0.
\]
It follows from Lemma \ref{lem: control of gk} that there exists $h\in \mathcal{P}$ such that
\[
  Q_1 (\operatorname{Re} g_k) \leq h
  \text{ and } Q_1 (\operatorname{Im} g_k)\leq h,
  \quad k \geq 0,
\]
where $Q_1$ is defined by \eqref{Q_k} with $\kappa=1$.

Chose a $\gamma\in(0,\beta)$ small enough such that $\alpha \gamma < b < (1+\gamma)b$.
According to Lemma \ref{lem: control moment}.(3) (with $\kappa=1$), there exists $C_3>0$ such that for all $t\geq 0$ and $k\geq 0$,
\begin{align}
  & \mathbb{P}_{\mu}\left[\left|\langle g_k,X_{t}\rangle\right|\right]
    \leq \|\langle \operatorname{Re} g_k, X_{t}\rangle\|_{\mathbb{P}_{\mu,1+\gamma}} + \|\langle \operatorname{Im} g_k, X_{t}\rangle\|_{\mathbb{P}_{\mu,1+\gamma}} \\
  & \leq 2\sup_{g\in \mathcal P: Q_1 g\leq h} \|\langle g, X_t\rangle\|_{\mathbb P_\mu; 1+\gamma} \leq C_3 e^{\frac{\alpha t}{1+\gamma}}.
\end{align}
Therefore, for all $k\geq 0$, $t\geq k+1$ and $\epsilon> 0$, we have
\begin{align}
  \label{eq: right bound for J2}
  & J_2(t,k, \epsilon)
    \leq  e^{-\alpha (k+1)}e^{-(\alpha-\epsilon)(t-k-1)}e^{(\alpha-\kappa_f b)(1+\beta)k} \mathbb{P}_{\mu}\left[\left|\langle g_k,X_{t-k-1}\rangle\right|\right] \\
  & \leq C_3 e^{-\alpha k}e^{-(\alpha-\epsilon)(t-k-1)}e^{(\alpha-\kappa_f b)(1+\beta)k} e^{\frac{\alpha}{1+\gamma}(t-k-1)} \\
  & = C_3 e^{(\alpha \beta - \kappa_f b(1+\beta))k}e^{-(\frac{\alpha\gamma}{1+\gamma}-\epsilon)(t-k-1)}
    \leq C_3 e^{-(\frac{\alpha\gamma}{1+\gamma}-\epsilon)(t-k-1)}.
\end{align}
By taking $\epsilon >0$ small enough, we get the required result in this step.

Step 4.
We will show that, for each $\epsilon\in (0,  \alpha)$, there exist $C_4,\delta_3>0$ such that for all $k\geq0, t\geq k+1$, $
J_3(t,k,\epsilon)\leq C_4e^{-\delta_3 (t-k)}.
% \end{equation}
$
In fact, we have  for all $t\geq 0, \epsilon >0$,
\[   
  \mathbb P_{\mu}(A_{t}(\epsilon), D) 
  = \mathbb P_{\mu}[\mathbb P_{\mu}(D|\mathscr F_t);A_t(\epsilon)]
  = \mathbb P_\mu[e^{-\bar v\|X_t\|};A_t(\epsilon)]
  \leq \exp({-\bar v \|\mu\|e^{(\alpha - \epsilon)t}}).
\]
By Proposition \ref{lem: control of XT}, for each $\epsilon \in (0, \alpha)$, there exists  $C_5, \delta_4>0$ such that for all $t\geq 0$,
\begin{equation}
  \mathbb P_\mu(A_t(\epsilon)^c,D^c) 
  \leq \mathbb P_\mu(0 < e^{-\alpha t}\|X_t\| 
  \leq e^{ - \epsilon t}) \leq C_5 (e^{-\epsilon \delta_4 t}+e^{-\delta_4 t}).
\end{equation}
Combining these results, we get the desired result in step 4.

Finally, combining the results in Steps 1--4, noticing that, if $\epsilon>0$ is chosen small enough then $J_{i}(t,k,\epsilon), i = 1,2,3,$ converge to $0$ exponentially fast as $t\rightarrow\infty$, we immediately get the desired result.
\end{proof}

\begin{cor}\label{cor: used in next corollary}
  Suppose $f\in \mathcal{P}$ is non-trivial and satisfies $\alpha\beta\leq \kappa_fb(1+\beta)$.
  Then for all $\Theta >0$ and $\mu\in \mathcal M_c(\mathbb R^d)$, there exist $C, \delta>0$ such that for $k \geq 0, t\geq k+1$ and $|\theta|\leq \Theta$,
\begin{equation}
  \mathbb{P}_{\mu}\Big[\big|\mathbb{P}_{\mu}[e^{i\theta\gamma_{t,k}}-e^{m_k[\theta f]}; D^c | \mathscr F_{t-k-1}]\big|\Big]
  \leq Ce^{-\delta(t-k)}.
\end{equation}
\end{cor}
\begin{proof}
	For $f\in \mathcal P$, define $J_1^f(t,k,\epsilon), J_2^f(t,k,\epsilon)$ and $J_3(t,k,\epsilon)$ as the $J_1, J_2$ and $J_3$ in \eqref{eq: Def of Ji}.
	Now, fix a $\mu \in \mathcal M_c(\mathbb R^d)$ and an $f\in \mathcal P$.
  According to \eqref{eq: inequality that will used later},  we have for all $\theta\in \mathbb R$, $k\geq 0$, $t\geq k+1$ and $\epsilon> 0$,
  \begin{align}
    & \mathbb{P}_{\mu}\Big[\big| \mathbb{P}_{\mu}[e^{i\theta \gamma_{t,k}}-e^{m_k[\theta f]}; D^c | \mathscr F_{t-k-1}]\big|\Big]
    \leq J^{\theta f}_1(t,k,\epsilon) + J^{\theta f}_2(t,k,\epsilon)+J_3(t,k,\epsilon).
  \end{align}
  
  Let $\delta_0 >0$ be the constant in Lemma \ref{lem: upper bound for usgx}.(7) and let $R$ be the corresponding $(\theta^{2+\beta}\vee \theta^{1+\beta+\delta_0})$-controller.
	According to \eqref{eq: estimate of J1}, we have for all $\theta\in \mathbb R$, $k\geq 0$, $t\geq k+1$ and $\epsilon> 0$,
  \begin{align}
    & J^{\theta f}_1(k,t,\epsilon)
      \leq \sum_{\rho \in \{\delta_0,1\}} \langle Q_0 RQ(\theta f), \mu \rangle e^{-(\alpha\frac{\rho}{1+\beta}-\epsilon\frac{1+\beta+\rho}{1+\beta})(t-k-1)} \\
    & \leq(|\theta|^{2+\beta}\vee |\theta|^{1+\beta+\delta_0}) \sum_{\rho \in \{\delta_0,1\}} \langle Q_0 RQf, \mu \rangle e^{ - ( \alpha \frac{ \rho} {1 + \beta} - \epsilon \frac{ 1 + \beta + \rho}{ 1 + \beta}) (t-k-1)}.
\end{align}
From the definitions of $Z_1$ and $m_t$ we get that for all $g\in \mathcal P, \theta \geq 0, t\geq 0$,
\[
	Z_1( \pm \theta g) 
  = \theta^{1+\beta} Z_1(\pm g), \quad m_t[\pm \theta g] = \theta^{1+\beta} m_t[\pm g].
\]
Therefore, we have for all $\theta >0, k \geq 0, t\geq k+1, \epsilon > 0$, $ J^{\pm \theta f}_2(t,k,\epsilon)
= \theta^{1+\beta} J_2^{\pm f}(t,k,\epsilon). $
According to this and \eqref{eq: right bound for J2},
we have that there exists $C > 0$ such that for all $\theta\in \mathbb R$, $k\geq 0$, $t\geq k+1$ and $\epsilon> 0$,
\begin{equation}
  \label{eq:31step3b}
  J^{\theta f}_2(t,k,\epsilon)
  \leq C |\theta|^{1+\beta}\exp\Big(-(\frac{\alpha\gamma}{1+\gamma}-\epsilon)(t-k-1)\Big).
\end{equation}
Finally, noticing that $|\theta| < \Theta$, using the estimates of $J^{\theta f}_{i}, i = 1,2$ and the estimate of $J_3$ in Step 4 of the proof of the previous lemma, we get the desired result by choosing $\epsilon$ small enough.
\end{proof}

\begin{prop}
  \label{cor: indepedent of the limit zeta for critical and small}
  Suppose that $f\in \mathcal{P}$ is non-trivial and satisfies $\alpha\beta\leq\kappa_fb(1+\beta)$.
  Then for all $\Theta >0$ and $\mu\in \mathcal M_c(\mathbb R^d)$, there exist $C,\delta>0$ such that for all $t\geq 0$, $n \in \{0, \cdots, \lfloor t \rfloor\}$ and $(\theta_0, \cdots, \theta_n)\in \mathbb R^{n+1}$satisfying $|\theta_i|\leq \Theta$, we have
\begin{equation}
  \label{32corollary}
  \Big|\mathbb{\widetilde{P}}_{\mu}\Big[\prod_{k=0}^n\exp\Big(i\theta_k \frac {\mathcal I_{t-k-1}^{t-k}\langle f ,X_t\rangle}{(e^{\alpha (k+1)}\|X_{t-k-1}\|)^\frac{1}{1+\beta}}\Big)-\prod_{k=0}^n\exp(m_k[\theta_k f])\Big]\Big|\leq C e^{-\delta(t-n)}.
\end{equation}
\end{prop}
\begin{proof}
  Recall that \[\gamma_{t,k}:=\frac {\mathcal I_{t-k-1}^{t-k}\langle f ,X_t\rangle}{(e^{\alpha( k+1)}\|X_{t-k-1}\|)^\frac{1}{1+\beta}},\quad k \geq 0, t\geq k+1. \]
  Fix $t\geq 0$, $n \in \{0, \cdots, \lfloor t \rfloor\}$ and $(\theta_0, \cdots, \theta_n)\in \mathbb R^{n+1}$satisfying $|\theta_i|\leq \Theta$.
  For each $k\in\{-1,...,n\}$, we define
  \[
    a_k
    :=\Big(\prod_{l=0}^{k}\exp(m_l[\theta_lf])\Big) \times \mathbb{\widetilde{P}}_{\mu}\Big(\prod_{l=k+1}^{n}\exp\left(i\theta_l\gamma_{t,l}\right)\Big),
  \]
  where by convention the first product is $1$ for $k=-1$. Then we get for each $k > -1$,
  \begin{align}
    & a_{k-1} - a_k
    =\mathbb{P}_{\mu}(D^c)^{-1}\Big(\prod_{l=0}^{k-1}e^{m_l[\theta_l f]}\Big) \times \mathbb{P}_{\mu}\Big[(e^{i\theta_{k}\gamma_{t,k}}-e^{m_k[\theta_k f]})\prod_{l=k+1}^ne^{i\theta_{l}\gamma_{t,l}};D^c\Big] \\ 
    & =\mathbb{P}_{\mu}(D^c)^{-1} \Big(\prod_{l=0}^{k-1}e^{m_l[\theta_l f]}\Big) \times \mathbb{P}_{\mu}\Big[\mathbb P_\mu[e^{i\theta_{k}\gamma_{t,k}}-e^{m_k[\theta_k f]}; D^c|\mathscr F_{t-k-1}]\prod_{l=k+1}^ne^{i\theta_{l}\gamma_{t,l}}\Big].
  \end{align}
  By Corollary \ref{cor: used in next corollary}, there exist $C,\delta>0$ such that for any $k\in\{0, 1, \cdots, n\}$, we have
  \begin{align}
    | a_{k-1} - a_k|
    & \leq \frac{1}{\mathbb{P}_{\mu}(D^c)}\mathbb{P}_{\mu}\Big[\big|\mathbb P_\mu[e^{i\theta_{k}\gamma_{t,k}}-e^{m_k[\theta_k f]}; D^c\big|\mathscr{F}_{t-k-1}]\big|\Big]
    \leq C e^{-\delta(t-k)}.
  \end{align}
  Therefore,
  \begin{align}
    \text{LHS of \eqref{32corollary}}
    & = \left|a_{-1}-a_n\right|
    \leq \sum_{k=0}^n\left|a_{k-1}-a_k\right|
    \leq \sum_{k=0}^n C e^{-\delta(t-k)}.
  \end{align}
	Recall that $C, \delta>0$ are independent of the choice of $t\geq 0$, $n \in \{0,...,\lfloor t \rfloor\}$ and $(\theta_0,...,\theta_n)\in \mathbb R^{n+1}$ with $|\theta_i|\leq \Theta$.
  This implies the desired result.
\end{proof}
We now present the proof of Theorem \ref{thm: critical clt}.
\medskip

\begin{proof}[Proof of Theorem \ref{thm: critical clt}]
  By assumption, $f\in\mathcal{P}$ is non-trivial and satisfies $\alpha\beta=\kappa_fb(1+\beta)$.
  Fix $\mu \in \mathcal M_c(\mathbb R^d)$.
  Choose $t_0 > 0$ large enough so that $\lceil t_0-\ln t_0\rceil \leq \lfloor t_0 \rfloor - 1.$
  We write
  \begin{align}
    &(t\|X_t\|)^{-\frac{1}{1+\beta}} \langle f,X_t \rangle \\
    & = \sum_{k=0}^{\lfloor t-\ln t \rfloor} \frac {\mathcal I_{t-k-1}^{t-k}\langle f ,X_t\rangle} {(t\|X_t\|)^{\frac{1}{1+\beta}}} + \Big( \sum_{k=\lceil t-\ln t \rceil}^{\lfloor t \rfloor-1} \frac {\mathcal I_{t-k-1}^{t-k} \langle f, X_t\rangle} {(t\|X_t\|)^{\frac{1}{1+\beta}}} + \frac {\mathcal I_0^{t-\lfloor t \rfloor}\langle f ,X_t\rangle} {(t\|X_t\|)^{\frac{1}{1+\beta}}} \Big) + \frac {\langle P^\alpha_tf,X_0\rangle} { ( t \| X_t \| )^{ \frac { 1 } { 1 + \beta } } } \\ 
    & =: I_1(t) + I_2(t) + I_3(t),
      \quad t\geq t_0.
  \end{align}
  Define
  \[
    \widetilde I_1(t)
    := \sum_{k=0}^{\lfloor t-\ln t \rfloor} \frac{\mathcal I_{t-k-1}^{t-k} \langle f ,X_t\rangle} {(t e^{\alpha(k+1)}\|X_{t-k-1}\|)^{\frac{1}{1+\beta}} },
    \quad t\geq t_0.
  \]
  Fix $\theta\in \mathbb R$.
  Taking $\theta_k=t ^{-\frac{1}{1+\beta}} \theta $ and $n={\lfloor t-\ln t \rfloor}$ in Proposition \ref{cor: indepedent of the limit zeta for critical and small}, we get that there exist $C_1,\delta_1>0$ such that,
  \begin{align}
    \Big| \mathbb{\widetilde{P}}_{\mu}  [ e^{ i  \theta \widetilde{I}_1(t) } ] - \exp \Big( \frac { 1 } { t } \sum_{k=0}^{\lfloor t - \ln t \rfloor} m_k [ \theta f ] \Big) \Big| 
    \leq \frac{C_1}{t^{\delta_1}},
    \quad t\geq t_0.
  \end{align}
  According to \eqref{para: critical case},  $\widetilde{I}_1(t)\xrightarrow[t\to \infty]{d}\widetilde \zeta$ under $\widetilde {\mathbb P}_\mu$.
  Therefore, we only need to prove
  \begin{equation}
    \label{toprove-1}
    |\mathbb{\widetilde{P}}_{\mu}[e^{i\theta I_1(t)}]-\mathbb{\widetilde{P}}_{\mu}[e^{i\theta\widetilde{I}_1(t)}]|
    \xrightarrow[t\to \infty]{} 0
  \end{equation}
  and
  \begin{equation}
    \label{toprove-2}
    I_i(t)\xrightarrow[t\to \infty]{d} 0,
    \quad i = 2,3,
    \mbox{ under } \widetilde {\mathbb P}_\mu.
  \end{equation}

  By \cite[Lemma 3.4.3]{Durrett2010Probability},
  \begin{equation}
    \label{ineq: control of I1t}
    |\mathbb{\widetilde{P}}_{\mu}[e^{i\theta I_1(t)}] - \mathbb{\widetilde{P}}_{\mu} [e^{i\theta\widetilde{I}_1(t)}]|
    \leq \sum_{k=0}^{\lfloor t-\ln t \rfloor}\mathbb{\widetilde{P}}_{\mu}\big[|Y_{t,k}|\big],
    \quad t\geq t_0,
  \end{equation}
  where for all $k \geq 0$ and $t\geq k+1$,
  \begin{align}
    Y_{t,k}
    :=\exp\Big(i\theta\frac{\mathcal I_{t-k-1}^{t-k}\langle f ,X_t\rangle}{(t e^{\alpha(k+1)}\|X_{t-k-1}\|)^{\frac{1}{1+\beta}}}\Big)-\exp\Big(i\theta\frac{\mathcal I_{t-k-1}^{t-k}\langle f ,X_t\rangle}{(t\|X_t\|)^{\frac{1}{1+\beta}}}\Big).
  \end{align}
  Let $\gamma \in (0,\beta)$ be close enough to $\beta$ such that
  \[
    \frac{\alpha \gamma}{1+\gamma} > \frac{\alpha}{1+\gamma} - \frac{\alpha}{1+\beta} > 0.
  \]
  Fix this $\gamma$, then choose $\eta_0,\eta_1>0$ such that
  \[
    \frac{\alpha \gamma}{1+\gamma} >\eta_0 > \eta_0 - 3\eta_1 > \frac{\alpha}{1+\gamma} - \frac{\alpha}{1+\beta} > 0.
  \]
  Recall that $H_t := e^{-\alpha t}\|X_t\|$.
  Define for all $k \geq 0$ and $t\geq k+1$,
  \begin{equation}
    \label{def: Dtk}
    \mathcal{D}_{t,k}
    :=\left\{|H_t-H_{t-k-1}|\leq  e^{-\eta_0 (t-k-1)}, H_{t-k-1}> 2e^{-\eta_1(t-k-1)}\right\}.
  \end{equation}

  Step 1. We will show that there exist $C_2,\delta_2 >0$ such that for all $k \geq 0$ and $t\geq k+1$,
  \begin{equation}
    \label{thm121}
    \mathbb{\widetilde{P}}_{\mu}\big[|Y_{t,k}|;\mathcal{D}^c_{t,k}\big]
    \leq C_2 e^{-\delta_2 (t-k)}.
  \end{equation}
  It follows from Proposition \ref{lem: control of XT}, Lemma \ref{lem: control of wt} with $|p|=0$ and Chebyshev's inequality that there exist $C_3, \delta_3>0$ such that for all $k \geq 0$ and $t\geq k+1$,
  \begin{align}
    \label{eq: prob of Dtkc}
    & \mathbb{\widetilde{P}}_{\mu}(\mathcal{D}_{t,k}^c)
    \leq \mathbb{\widetilde{P}}_{\mu}(|H_t-H_{t-k-1}| > e^{-\eta_0 (t-k-1)})+\mathbb{\widetilde{P}}_{\mu}(H_{t-k-1}\leq 2e^{-\eta_1(t-k-1)}) \\
    & \leq \mathbb{P}_{\mu}(D^c)^{-1}e^{\eta_0(t-k-1)}\mathbb{P}_{\mu}[|H_t-H_{t-k-1}|] \\
    & \quad 
      + \mathbb{P}_{\mu}(D^c)^{-1} \mathbb P_\mu(H_{t-k-1}\leq 2e^{-\eta_1(t-k-1)}; D^c) \\
    & \leq \mathbb{P}_{\mu}(D^c)^{-1}  e^{\eta_0(t-k-1)}\|H_t - H_{t-k-1}\|_{\mathbb P_\mu; 1+\gamma} \\
    & \quad + \mathbb{P}_{\mu}(D^c)^{-1} \mathbb P_\mu(0<H_{t-k-1}\leq 2e^{-\eta_1(t-k-1)}) \\
    & \leq C_3 e^{-(\frac{\alpha \gamma}{1+\gamma} - \eta_0)(t-k-1)}+C_3 e^{-\delta_3(t-k-1)}.
  \end{align}
  This implies the desired result in this step, since $|Y_{t,k}| \leq 2$ a.s..
  
  Step 2. We will show that there exist $C_4,\delta_4 > 0$ such that for all $k\geq 0$ and $t\geq k+1$,
  \begin{equation}
    \label{thm122}
    \mathbb{\widetilde{P}}_{\mu}\big[|Y_{t,k}|\mathbf{1}_{\mathcal{D}_{t,k}}\big]
    \leq  C_4 e^{-\delta_4 (t-k)}.
  \end{equation}
  In fact, since $|e^{ix}-e^{iy}|\leq|x-y|$ for all $x,y\in \mathbb R$, we have for all $k \geq 0$ and $t\geq k+1$,
  \begin{align}
    \label{eq: control of Ykt}
    & \mathbb{\widetilde{P}}_{\mu}\big[|Y_{t,k}|\mathbf{1}_{\mathcal{D}_{t,k}}\big] \\
    & \leq|\theta|t^{-\frac{1}{1+\beta}} \mathbb{\widetilde{P}}_{\mu}\bigg[|\mathcal I_{t-k-1}^{t-k}\langle f , X_t\rangle | \cdot \Big| \frac {1} {(e^{\alpha(k+1)} \| X_{t-k-1}\| )^{ \frac {1} {1+\beta} } } - \frac {1} {\|X_t\|^{\frac{1}{1+\beta}}}\Big|\mathbf{1}_{\mathcal{D}_{t,k}}\bigg] \\
    & \leq |\theta| e^{-\frac{\alpha}{1+\beta}t}\mathbb{\widetilde{P}}_{\mu}\big[|\mathcal I_{t-k-1}^{t-k}\langle f ,X_t\rangle|\cdot K_{t,k}\big],
  \end{align}
  where
  \begin{equation}
    \label{def: Ktk}
    K_{t,k}
    := \Big| \frac {H_t^{\frac{1}{1+\beta}}-H_{t-k-1}^{\frac{1}{1+\beta}}} {H_t^{\frac{1}{1+\beta}} H_{t-k-1}^{ \frac {1} {1+\beta} }} \Big| \mathbf{1}_{\mathcal{D}_{t,k}}.
  \end{equation}
  Note that, since $\eta_1 < \eta_0$, we have on $\mathcal D_{t,k}$,
  \begin{align}
    H_t
    & \geq H_{t-k-1}- e^{-\eta_0(t-k-1)}
      \geq 2e^{-\eta_1(t-k-1)}-e^{-\eta_0(t-k-1)}
      \geq e^{-\eta_1(t-k-1)}.
  \end{align}
  Therefore, for each $k \geq 0$ and $t\geq k+1$,  on $\mathcal D_{t,k}$,
  \begin{align}
    & \Big|H_t^{\frac{1}{1+\beta}}-H_{t-k-1}^{\frac{1}{1+\beta}}\Big|
      \leq \frac {1} {1+\beta} \max \Big \{ H_t^{-\frac{\beta}{1+\beta} }, H_{t-k-1}^{ -\frac{\beta}{1+\beta} } \Big\} \left| H_t - H_{t-k-1} \right| \\
    & \leq \frac{1}{1+\beta} \max\{e^{\eta_1 (t-k-1)}, \frac{1}{2}e^{\eta_1(t-k-1)}\}^{\frac{\beta}{1+\beta}}e^{-\eta_0(t-k-1)} \\
    & \leq \frac{1}{1+\beta} e^{\eta_1 (t-k-1)} e^{-\eta_0(t-k-1)}
    = \frac{1}{1+\beta}  e^{-(\eta_0 - \eta_1)(t-k-1)}
  \end{align}
  and
  \begin{align}
    |H_t^{\frac{1}{1+\beta}}H_{t-k-1}^{\frac{1}{1+\beta}}|
    \geq 2^{\frac{1}{1+\beta}} e^{-2\eta_1(t-k-1)}.
  \end{align}
  Thus, there exists  $C_5> 0$ such that for all $k \geq 0, t\geq k+1$,
  \begin{equation}
    \label{ineq: control of Kkt}
    K_{t,k}
    \leq C_5 e^{-(\eta_0 - 3\eta_1)(t-k-1)}.
  \end{equation}
  By Lemma \ref{lem: control of mgtrs}, \eqref{eq: control of Ykt} and \eqref{ineq: control of Kkt}, there exist $C_6>0$ such that for all $k\geq 0$ and $t\geq k+1$,
  \begin{align}
    \label{eq: Y in D}
    & \mathbb{\widetilde{P}}_{\mu}\big[|Y_{t,k}|\mathbf{1}_{\mathcal{D}_{t,k}}\big]
    \leq C_5 | \theta | e^{-\frac{\alpha}{1+\beta}t} \mathbb{\widetilde{P}}_{\mu} \big[ | \mathcal{I}_{t-k-1}^{t-k}\langle f, X_t\rangle| \big] e^{-(\eta_0 - 3\eta_1)(t-k-1)} \\
    & \leq  \frac{C_5} {\mathbb{P}_{\mu}(D^c)} | \theta | e^{-\frac{\alpha} {1+\beta}t}\mathbb{P}_{\mu} \big[ |\mathcal{I}_{t-k-1}^{t-k}\langle f, X_t\rangle|\big]e^{-(\eta_0 - 3\eta_1)(t-k-1)} \\
    & \leq \frac{C_5} {\mathbb{P}_{\mu}(D^c)} | \theta | e^{ - \frac{\alpha} {1 + \beta} t} \|\mathcal{I}_{t-k-1}^{t-k}\langle f,X_t\rangle\|_{\mathbb P_\mu; 1+\gamma} e^{-(\eta_0 - 3\eta_1)(t-k - 1)} \\
    & \leq C_6 e^{-\frac{\alpha}{1+\beta}t}e^{\frac{\alpha}{1+\gamma}t}e^{\frac{\gamma \alpha-\kappa_f(1+\gamma)b}{1+\gamma}k}e^{-(\eta_0 - 3\eta_1)(t-k)} \\
    & = C_6 e^{(\frac{\alpha}{1+\gamma}-\frac{\alpha}{1+\beta})(t-k)}e^{-(\eta_0 - 3\eta_1)(t-k)},
  \end{align}
  as desired in this step.
  In the last equality, we used the fact that
  \[
    -(\frac{\alpha}{1+\gamma}-\frac{\alpha}{1+\beta})
    = \alpha(1-\frac{1}{1+\gamma}) - \alpha(1-\frac{1}{1+\beta})
    = \frac{\gamma \alpha}{1+\gamma} - k_f b
    =\frac{\alpha \gamma-\kappa_f(1+\gamma)b}{1+\gamma}.
  \]
  
  Step 3.
  We will show that there exist $C_7, \delta_5> 0$ such that for $t$ large enough,
  \begin{equation}
    \label{domi-difference of Is}
    |\mathbb{\widetilde{P}}_{\mu}[e^{i\theta I_1(t)}] - \mathbb{\widetilde{P}}_{\mu} [ e^{ i \theta \widetilde{I}_1(t) } ] |
    \leq C_7 t^{-\delta_5}.
  \end{equation}
  In fact, according to \eqref{thm121} and \eqref{thm122}, there exist $C_8,\delta_6 > 0$ such that for all $k \geq 0$ and $t\geq k+1$ we have,
  \[
    \widetilde{\mathbb P}_\mu\big[|Y_{t,k}|\big]
    \leq C_8 e^{-\delta_6(t-k)}.
  \]
  Therefore, there exists $C_9> 0$  such that for all $t \geq 1$,
  \begin{align}
    & \sum_{k=0}^{\lfloor t-\ln t \rfloor} \widetilde {\mathbb P}_\mu\big[|Y_{t,k}|\big]
      \leq C_8\sum_{k=0}^{\lfloor t-\ln t \rfloor} e^{-\delta_6(t-k)}
      = C_8 e^{-\delta_6 t}\sum_{k=0}^{\lfloor t-\ln t \rfloor} e^{\delta_6 k}
      \leq C_9 e^{-\delta_6 t}e^{\delta_6 (t-\ln t)}
      = \frac{C_9}{t^{\delta_6}}.
  \end{align}
  Letting $t\to\infty$ in \eqref{domi-difference of Is}, we get \eqref{toprove-1}.
  
  Step 4.
  We will show that $I_2(t) \xrightarrow[t\to \infty]{d} 0$ with respect to $\mathbb{\widetilde{P}}_{\mu}$.
  In fact, let $\mathcal{E}_t:=\{\|X_t\|>t^{-1/2}e^{\alpha t}\}$. According to Proposition \ref{lem: control of XT}, there exist $C_{10}, \delta_7>0$ such that
  \begin{equation}
    \mathbb{\widetilde{P}}_{\mu}(\mathcal{E}^c_t)
    \leq \frac{1}{\mathbb{P}_{\mu}(D^c)}\mathbb{P}_{\mu}(0<e^{-\alpha t}\|X_t\|\leq t^{-1/2})\leq C_{10}( t^{-\delta_7}+e^{-\delta_7 t})
    , \quad t\geq0.
  \end{equation}
  Therefore,
  \begin{equation}
    \label{Theorem123}
    |\mathbb{\widetilde{P}}_{\mu}[(e^{i\theta I_2(t)}-1)\mathbf{1}_{\mathcal{E}^c_t}]|
    \leq 2\mathbb{\widetilde{P}}_{\mu}(\mathcal{E}^c_t)\leq C_{10}(t^{-\delta_7}+e^{-\delta_7 t}),
    \quad t\geq t_0.
  \end{equation}
  Choose a $\gamma\in (0,\beta)$ close enough to $\beta$ such that $\alpha(\frac{1}{1+\gamma}-\frac{1}{1+\beta})\leq \frac{1}{2(1+\beta)}$.
	According to Lemma \ref{lem: control of mgtrs}, there exist $C_{11},C_{12},C_{13}>0$ such that
  \begin{align}
    & |\mathbb{\widetilde{P}}_{\mu} [ (e^{i\theta I_2(t)}-1)\mathbf{1}_{\mathcal{E}_t}]|
      \leq |\theta| \mathbb{\widetilde{P}}_{\mu} \big[ |I_2(t)|\mathbf{1}_{\mathcal{E}_t}\big] \\
    & \leq | \theta| t^{-\frac{1}{2(1+\beta)}}e^{-\frac{\alpha}{1+\beta}t}\Big(\sum_{k=\lceil t-\ln t \rceil}^{\lfloor t \rfloor - 1}\mathbb{\widetilde{P}}_{\mu}\big[| \mathcal{I}_{t-k-1}^{t-k}\langle f,X_t\rangle|\big] + \mathbb{\widetilde{P}}_{\mu}\big[| \mathcal{I}_{0}^{t-\lfloor t\rfloor}\langle f,X_t\rangle|\big]\Big) \\
    & \leq C_{11} |\theta| t^{-\frac{1}{2(1+\beta)}}e^{-\frac{\alpha}{1+\beta}t}\Big(\sum_{k=\lceil t-\ln t \rceil}^{\lfloor t \rfloor - 1}\|\mathcal{I}_{t-k-1}^{t-k}\langle f,X_t\rangle\|_{\mathbb P_\mu; 1+\gamma} + \|\mathcal I_0^{t-\lfloor t \rfloor} \langle f, X_t\rangle\|_{\mathbb P_\mu;1+\gamma}\Big) \\ 
    & \leq C_{12} |\theta| t^{-\frac{1}{2(1+\beta)}}e^{-\frac{\alpha}{1+\beta}t}\sum_{k=\lceil t-\ln t \rceil}^{\lfloor t \rfloor}e^{\frac{\alpha}{1+\gamma}t}e^{\frac{\alpha\gamma-\kappa_f(1+\gamma)b}{1+\gamma}k}\\ 
    & = C_{12} |\theta| t^{-\frac{1}{2(1+\beta)}}e^{(\frac{\alpha }{1+\gamma}-\frac{\alpha }{1+\beta})t} \sum_{k=\lceil t-\ln t \rceil}^{\lfloor t \rfloor}e^{-(\frac{\alpha}{1+\gamma}-\frac{\alpha}{1+\beta})k}\\
    & \leq C_{12} |\theta| t^{-\frac{1}{2(1+\beta)}}e^{(\frac{\alpha }{1+\gamma}-\frac{\alpha }{1+\beta})(t - \lceil t - \ln t\rceil)} \sum_{j=0}^{\infty}e^{-(\frac{\alpha}{1+\gamma}-\frac{\alpha}{1+\beta})j}\\
    & \leq C_{13}|\theta| t^{-\frac{1}{2(1+\beta)}}t^{(\frac{\alpha}{1+\gamma}- \frac{\alpha}{1+\beta})},
      \quad t\geq t_0.
  \end{align}
  From this and \eqref{Theorem123}, we get the desired result in this step.
  
  Step 5.
  We will show that $I_3(t) \xrightarrow[t\to \infty]{\widetilde {\mathbb P}_\mu \text{-} a.s.} 0$.
  In fact, we have
  \begin{align}
    & |I_3(t)|
      \leq \frac{\langle |P^\alpha_tf|,X_0\rangle}{(t\|X_t\|)^{\frac{1}{1+\beta}}}
      \leq \frac{\langle e^{\alpha t - \kappa_f b t}Qf,X_0\rangle}{(te^{\alpha t} H_t)^{\frac{1}{1+\beta}}} \\
    & = t^{-\frac{1}{1+\beta}} e^{\frac{\beta \alpha t}{1+\beta} - k_fbt} H_t^{-\frac{1}{1+\beta}} \langle Qf,X_0\rangle
      = t^{-\frac{1}{1+\beta}} H_t^{-\frac{1}{1+\beta}} \langle Qf,X_0\rangle
      \xrightarrow[t\to \infty]{\widetilde {\mathbb P}_\mu \text{-} a.s.} 0.
  \end{align}
  
	Finally, combining Steps 3--5, we complete the proof of Theorem \ref{thm: critical clt}.
\end{proof}

% *** The small branching rate regime
\subsection{The small branching rate regime}
\label{sec: small rate}

In this subsection, we prove the central limit theorem for the small branching rate regime.

\begin{proof}[Proof of Theorem \ref{thm: small clt}]
  By assumption, $f\in \mathcal P$ is non-trivial and satisfies $\alpha \beta < \kappa_f b(1+\beta)$.
	Let $t_0 > 1$ be large enough so that $ \lceil t - \ln t\rceil \leq \lfloor t \rfloor - 1, $ for all $t\geq t_0$.
  Fix $\mu\in \mathcal M_c(\mathbb R^d)$ and write
  \begin{align}
    & \frac{\langle f,X_t\rangle}{\|X_t\|^{\frac{1}{1+\beta}}}
    \\ & =\sum_{k=0}^{\lfloor t-\ln t \rfloor} \frac{\mathcal I_{t-k-1}^{t-k}\langle f ,X_t\rangle}{\|X_t\|^{\frac{1}{1+\beta}}}+ \Big(\sum_{k=\lceil t-\ln t \rceil}^{\lfloor t \rfloor-1} \frac{\mathcal I_{t-k-1}^{t-k}\langle f ,X_t\rangle}{\|X_t\|^{\frac{1}{1+\beta}}}+\frac{\mathcal I_0^{t-\lfloor t \rfloor}\langle f ,X_t\rangle}{\|X_t\|^{\frac{1}{1+\beta}}}\Big) +
\frac{\langle P^\alpha_t f, X_0\rangle}{\|X_t\|^{\frac{1}{1+\beta}}}
    \\ & =:I'_1(t)+I'_2(t)+I'_3(t),
         \quad t\geq t_0.
  \end{align}
	Define
  \[
    \widetilde I'_1(t)
    := \sum_{k=0}^{\lfloor t-\ln t \rfloor}\frac{\mathcal I_{t-k-1}^{t-k}\langle f ,X_t\rangle}{( e^{\alpha(k+1)}\|X_{t-k-1}\|)^{\frac{1}{1+\beta}}},
    \quad t > t_0.
  \]
  Fix a $\theta\in \mathbb R$.
  Taking $\theta_k=\theta $ and $n={\lfloor t-\ln t \rfloor}$ in Proposition \ref{cor: indepedent of the limit zeta for critical and small}, we get that there exist $C_1,\delta_1 > 0$ such that
  \begin{align}
    \Big|\mathbb{\widetilde{P}}_{\mu} [e^{i\theta\widetilde I'_1(t)} ]-\exp\Big(\sum_{k=0}^{\lfloor t-\ln t \rfloor}m_k[\theta f]\Big)\Big|
    \leq C_1 e^{-\delta_1(t - \lfloor t - \ln t\rfloor)}
    \leq \frac{C_1}{t^{\delta_1}},
    \quad t\geq 0.
  \end{align}
  Hence, according to \eqref{sum-m}, we have $\widetilde I'_1(t)\xrightarrow[t\to \infty]{d} \zeta$ under $\widetilde {\mathbb P}_\mu$.
  So we only need to prove that $|\mathbb{\widetilde{P}}_{\mu}[e^{i\theta I'_1(t)}]-\mathbb{\widetilde{P}}_{\mu}[e^{i\theta\widetilde I'_1(t)}]|\xrightarrow[t\to \infty]{} 0$ and $I'_i(t)\xrightarrow[t\to \infty]{d} 0,~i=2,3,$ under $\widetilde {\mathbb P}_\mu$.

  Step 1.
  We will show that $|\mathbb{\widetilde{P}}_{\mu}[e^{i\theta I'_1(t)}]-\mathbb{\widetilde{P}}_{\mu}[e^{i\theta\widetilde I'_1(t)}]|\xrightarrow[t\to \infty]{} 0$.
  Define for $k\geq 0$ and $t\geq k+1$,
  \begin{align}
    Y'_{t,k}
    :=\exp\Big(i\theta\frac{\mathcal I_{t-k-1}^{t-k}\langle f ,X_t\rangle}{( e^{\alpha(k+1)}\|X_{t-k-1}\|)^{\frac{1}{1+\beta}}}\Big)-\exp\Big(i\theta\frac{\mathcal I_{t-k-1}^{t-k}\langle f ,X_t\rangle}{\|X_t\|^{\frac{1}{1+\beta}}}\Big).
  \end{align}
	Then we have
  \begin{equation}
    \label{ineq: control of I1tb}
    |\mathbb{\widetilde{P}}_{\mu}[e^{i\theta I'_1(t)}] - \mathbb{\widetilde{P}}_{\mu} [e^{i\theta\widetilde{I}'_1(t)}]|
    \leq \sum_{k=0}^{\lfloor t-\ln t \rfloor}\mathbb{\widetilde{P}}_{\mu}\big[|Y'_{t,k}|\big],
    \quad t\geq t_0.
  \end{equation}
  Let $\gamma \in (0,\beta)$ be close enough to $\beta$ such that
  \[
    \frac{\alpha \gamma}{1+\gamma} 
    > \frac{\alpha}{1+\gamma} - \frac{\alpha}{1+\beta} 
    > 0.
  \]
  Fix this $\gamma$, then chose $\eta_0,\eta_1>0$ such that
  \[
    \frac{\alpha \gamma}{1+\gamma} 
    > \eta_0 
    > \eta_0 - 3\eta_1 
    > \frac{\alpha}{1+\gamma} - \frac{\alpha}{1+\beta} 
    > 0.
  \]
	Define $\mathcal D_{t,k}$ and $K_{t,k}$ as in \eqref{def: Dtk} and \eqref{def: Ktk} respectively.
	Using Lemma \ref{lem: control of mgtrs}, \eqref{eq: prob of Dtkc}, \eqref{ineq: control of Kkt} and an argument similar to that used in proving \eqref{eq: control of Ykt}, we get that there exist $C_2,C_3,\delta_2>0$ such that for $k\geq 0$ and $t\geq k+1$,
  \begin{align}
    & \mathbb{\widetilde{P}}_{\mu}\big[|Y'_{t,k}|\big]
      = \mathbb{\widetilde{P}}_{\mu}\big[|Y'_{t,k}|; \mathcal D_{t,k}\big] + \mathbb{\widetilde{P}}_{\mu}\big[|Y'_{t,k}|; \mathcal D_{t,k}^c \big] \\
    & \leq |\theta| e^{-\frac{\alpha}{1+\beta} t}\mathbb{\widetilde{P}}_{\mu}\big[|\mathcal I_{t-k-1}^{t-k}\langle f ,X_t\rangle|\cdot K_{t,k}\big] + 2\mathbb{\widetilde{P}}_{\mu}( \mathcal D_{t,k}^c ) \\
    & \leq C_2 e^{-\frac{\alpha}{1+\beta} t} \|\mathcal I_{t-k-1}^{t-k}\langle f, X_t\rangle \|_{\mathbb P_\mu; 1+\gamma}e^{-(\eta_0 - 3\eta_1)(t-k)} + Ce^{-\delta_2(t-k)} \\
    & \leq C_3( e^{-\frac{\alpha}{1+\beta}t}e^{\frac{\alpha}{1+\gamma}t}e^{\frac{\gamma \alpha-\kappa_f(1+\gamma)b}{1+\gamma}k}e^{-(\eta_0 - 3\eta_1)(t-k)}+ e^{-\delta_2(t-k)}).
  \end{align}
  Since $\alpha\beta<\kappa_f(1+\beta)b$,  we have
  \begin{align}
    \label{eq: condition for supercritical}
    & -(\frac{\alpha}{1+\gamma}-\frac{\alpha}{1+\beta})
      = \alpha(1-\frac{1}{1+\gamma}) - \alpha(1-\frac{1}{1+\beta}) \\
    & > \frac{\gamma \alpha}{1+\gamma} - k_f b
      =\frac{\alpha \gamma-\kappa_f(1+\gamma)b}{1+\gamma}.
  \end{align}
  Using this, we have that there exist $C_4, \delta_3 > 0$ such that for $k\geq 0$ and $t\geq k+1$,
  \begin{align}
    \mathbb{\widetilde{P}}_{\mu}\big[|Y'_{t,k}|\big]
    & \leq C_3( e^{(\frac{\alpha}{1+\gamma} - \frac{\alpha}{1+\beta})(t-k)}e^{-(\eta_0 - 3\eta_1)(t-k)}+ e^{-\delta_2(t-k)})
    \leq C_4e^{-\delta_3 (t-k)}.
  \end{align}
  Now we can  use the  same argument used in the proof of Theorem \ref{thm: critical clt} to prove $|\mathbb{\widetilde{P}}_{\mu}[e^{i\theta I'_1(t)}]-\mathbb{\widetilde{P}}_{\mu}[e^{i\theta\widetilde I'_1(t)}]|\xrightarrow[t\to \infty]{} 0$.

	Step 2.
	We will show that $I'_2(t)\xrightarrow[t\to \infty]{d} 0$.
	Let $\gamma \in (0,\beta)$.
	According to \eqref{eq: condition for supercritical}, we can choose $\epsilon > 0$ small enough so that
  \[
    q
    := - \frac{\alpha \gamma-\kappa_f(1+\gamma)b}{1+\gamma}
    > \frac{\alpha}{1+\gamma}-\frac{\alpha}{1+\beta} + \frac{2\epsilon}{1+\beta} > 0.
  \]
	Recall that $A_t(\epsilon)=\{\|X_t\|>e^{(\alpha-\epsilon )t}\}$.
  By \eqref{lem: control of XT}, there exist $C_5,\delta_4>0$ such that
  \begin{align}
    & |\mathbb{\widetilde{P}}_{\mu}[(e^{i\theta I'_2(t)}-1)\mathbf{1}_{A_t(\epsilon)^c}]|
      \leq 2\mathbb{\widetilde{P}}_{\mu}(A_t(\epsilon)^c)\leq \frac{2}{\mathbb{P}_{\mu}(D^c)}\mathbb{P}_{\mu}(0<e^{-\alpha t}\|X_t\|\leq e^{-\epsilon t}) \\
    & \leq C_5(e^{-\epsilon\delta_4 t}+e^{-\delta_4 t}),
      \quad t\geq t_0.
  \end{align}
	According to Lemma \ref{lem: control of mgtrs}, there exist $C_6,C_7,C_8>0$ such that for all $t\ge t_0$,
  \begin{align}
    & \big|\mathbb{\widetilde{P}}_{\mu} [ (e^{i\theta I'_2(t)}-1)\mathbf{1}_{A_t(\epsilon)}]\big|
      \leq |\theta| \mathbb{\widetilde{P}}_{\mu} \big[ |I'_2(t)|\mathbf{1}_{A_t(\epsilon)}\big] \\
    & \leq|\theta| e^{-\frac{(\alpha - \epsilon )t}{1+\beta}} \Big(\sum_{k=\lceil t-\ln t \rceil}^{\lfloor t \rfloor - 1}\mathbb{\widetilde{P}}_{\mu}\big[| \mathcal{I}_{t-k-1}^{t-k}\langle f,X_t\rangle|\big] + \mathbb{\widetilde{P}}_{\mu}\big[| \mathcal{I}_{0}^{t-\lfloor t\rfloor}\langle f,X_t\rangle|\big]\Big) \\
    & \leq C_6  e^{-\frac{(\alpha - \epsilon )t}{1+\beta}} \Big(\sum_{k=\lceil t-\ln t \rceil}^{\lfloor t \rfloor - 1}\|\mathcal{I}_{t-k-1}^{t-k}\langle f,X_t\rangle\|_{\mathbb P_\mu; 1+\gamma} + \|\mathcal I_0^{t-\lfloor t \rfloor} \langle f, X_t\rangle\|_{\mathbb P_\mu;1+\gamma}\Big) \\ 
    & \leq C_7  e^{-\frac{(\alpha - \epsilon )t}{1+\beta}} \sum_{k=\lceil t-\ln t \rceil}^{\lfloor t \rfloor}e^{\frac{\alpha}{1+\gamma}t}e^{\frac{\alpha\gamma-\kappa_f(1+\gamma)b}{1+\gamma}k}\\
    & \leq C_7 e^{-\frac{\epsilon}{1+\beta} t}e^{(\frac{\alpha }{1+\gamma}-\frac{\alpha }{1+\beta} + \frac{2\epsilon}{1+\beta})t} \sum_{k=\lceil t-\ln t \rceil}^{\lfloor t \rfloor}e^{\frac{\alpha\gamma-\kappa_f(1+\gamma)b}{1+\gamma}k}\\
    & \leq C_7 e^{-\frac{\epsilon}{1+\beta} t} e^{qt} \sum_{k=\lceil t-\ln t \rceil}^{\lfloor t \rfloor}e^{-qk}\\
    & \leq C_7 e^{-\frac{\epsilon}{1+\beta} t} e^{q(t - \lceil t - \ln t\rceil)} \sum_{j=0}^{\infty}e^{-qj}\leq C_8 e^{-\frac{\epsilon}{1+\beta} t} t^q.
  \end{align}
	Therefore, we get the desired result in this step.
  
	Step 3. We will show that $I'_3(t) \xrightarrow[t\to \infty]{\widetilde {\mathbb P}_\mu \text{-} a.s.} 0$.
  In fact, we have
  \begin{align}
    & |I'_3(t)|
      \leq \frac{\langle |P^\alpha_tf|,X_0\rangle}{\|X_t\|^{\frac{1}{1+\beta}}}
      \leq \frac{\langle e^{\alpha t - \kappa_f b t}Qf,X_0\rangle}{(e^{\alpha t} H_t)^{\frac{1}{1+\beta}}}
      = e^{(\frac{\beta \alpha }{1+\beta} - k_fb)t} H_t^{-\frac{1}{1+\beta}} \langle Qf,X_0\rangle
      \xrightarrow[t\to \infty]{\widetilde {\mathbb P}_\mu \text{-} a.s.} 0.
      \qedhere
  \end{align}
\end{proof}

% *** The large branching rate regime: CLT
\subsection{The large branching rate regime: CLT}
\label{sec: large rate clt}
In this subsection, we revisit the large branching rate regime and prove Theorem \ref{thm: large clt}.
For $g\in \mathcal{C}_l$, recall the definition of $I_tg$ in \eqref{definition of Itf}.
By Fubini's theorem, we have
\begin{align}
  \label{equ: transform of mn}
  & \bar{m}_n[g]
    = e^{\alpha(n-1)}\int_0^1 e^{\alpha s}\langle \eta(iI_{s+n-1}g)^{1+\beta}, \varphi\rangle ds \\   
  & = e^{\alpha(n-1)}\langle \int_0^1 P_s^{\alpha}\langle \eta(iI_{s+n-1}g)^{1+\beta}ds, \varphi\rangle\\
  & =e^{\alpha(n-1)}\langle \int_0^1 P_{1-s}^{\alpha}\langle \eta(iI_{n-s}g)^{1+\beta}ds, \varphi\rangle=e^{\alpha(n-1)}\langle Z_1(-I_ng), \varphi\rangle,
    \quad n\geq 1.
\end{align}

Recall that $H^p_t$ are defined before Lemma \ref{lemma26}. We write $a_p:= \langle \phi_p, g\rangle_\varphi$ for each $p \in \mathbb Z_+^d$.

\begin{lem}
  \label{large-central}
  For all $n\ge 1$ and $\mu\in\mathcal{M}_c(\mathbb{R}^d)$, under $\mathbb{P}(\cdot|D^c)$,
\begin{align}
  \bar{\gamma}_{t,n}
  := \sum_{p\in\mathcal{N}} a_p \frac { H^p_{t+n-1} - H^p_{t+n} } { e^{ - ( \alpha - | p | b ) t } ( e^{ - \alpha ( n - 1 ) } \| X_{ t + n - 1 } \| )^{ \frac { 1 } { 1 + \beta } } }
  \xrightarrow[t\to \infty]{d}\bar{\zeta}_n,
\end{align}
	where $\bar{\zeta}_n$ is a $(1+\beta)$-stable random variable with characteristic function $\theta\mapsto \exp(\bar{m}_n[\theta g])$.
\end{lem}
\begin{proof}
  Fix $n\ge 1$ and $\mu\in\mathcal{M}_c(\mathbb{R}^d)$.
  We only need to show that
  \begin{align}
    \mathbb{P}_{\mu}[\exp(i\bar{\gamma}_{t,n}); D^c]
    \xrightarrow[t\rightarrow \infty]{}\mathbb{P}_{\mu}(D^c)\exp(\bar{m}_n[g]).
  \end{align}
  Put $ \theta_{ t, p, n} := a_p \frac { e^{ - ( \alpha - | p | b ) n } } { ( e^{ - \alpha ( n - 1 ) } \| X_{ t + n - 1 } \| )^{ \frac { 1 } { 1 + \beta } } } $ and $ A_t(\epsilon):=\{\|X_t\| > e^{(\alpha-\epsilon)t}\}$.
  We have
  \begin{align}
    \bar{\gamma}_{t,n}
    =\sum_{p\in \mathcal{N}}\theta_{t,p,n}(e^{\alpha-|p|b}\langle \phi_p, X_{t+n-1}\rangle-\langle \phi_p, X_{t+n}\rangle).
  \end{align}

	Step 1. We will show that for all $\epsilon > 0, n\geq 1$, and $t\geq 0$, we have
  \begin{align}
    \big|\mathbb{P}_{\mu}\big[e^{i\bar{\gamma}_{t,n}}-e^{\bar{m}_n[g]}; D^c\big]\big|
    \leq J'_1(t,n,\epsilon)+J'_2(t,n,\epsilon)+J'_3(t,n,\epsilon),
  \end{align}
	where
  \begin{align}
    \label{eq: Def of JJ1}
    J'_1(t,n,\epsilon)
    & := \mathbb{P}_{\mu}\big[|\langle Z'''_1(-\sum_{p\in \mathcal{N}}\theta_{t,p,n}\phi_p), X_{t+n-1}\rangle|; A_{t+n-1}(\epsilon) \big], \\ 
    J'_2(t,n,\epsilon)
    & := \mathbb{P}_{\mu}\big[|\langle Z_1(-\sum_{p\in \mathcal{N}}\theta_{t,p,n}\phi_p),X_{t+n-1}\rangle-\bar{m}_n[g]|; A_{t+n-1}(\epsilon)\big] \\
    J'_3(t,n, \epsilon)
    & := 2\mathbb{P}_{\mu}(A_{t+n-1}(\epsilon)\Delta D^c).
  \end{align}
  In fact, it follows from \eqref{eq: key equality} that
  \begin{align}
    \label{eq: need11}
    & \displaystyle\mathbb{P}_{\mu}[e^{i\bar{\gamma}_{t,n}}|\mathscr{F}_{t+n-1}]
      = \mathbb{P}_{\mu}[e^{i\sum_{p\in \mathcal{N}}\theta_{t,p,n}(e^{\alpha-|p|b}\langle \phi_p,X_{t+n-1}\rangle-\langle \phi_p, X_{t+n}\rangle)}|\mathscr{F}_{t+n-1}] \\
    & = \displaystyle e^{\langle (Z'''_1+Z_1)(-\sum_{p\in \mathcal{N}}\theta_{t,p,n}\phi_p), X_{t+n-1} \rangle}.
  \end{align}

  According to \eqref{eq: need11} and the fact that $|e^{-x} - e^{-y}| \leq |x-y|$ for all $x,y \in \mathbb C_+$,
  we have for all $n\geq 1, t\geq 0$ and $\epsilon> 0$,
  \begin{align}
    \label{eq: inequality that will used later1}
    & \big|\mathbb{P}_{\mu}\big[e^{i\bar{\gamma}_{t,n}}-e^{\bar{m}_n[g]}; D^c\big]\big|
    \leq \mathbb{P}_{\mu}\Big[\big| \mathbb{P}_{\mu}[e^{i\bar{\gamma}_{t,n}}-e^{\bar{m}_n[g]}; D^c | \mathscr F_{t+n-1}]\big|\Big] \\
    & \leq \mathbb{P}_{\mu}\Big[\big| \mathbb{P}_{\mu}[e^{i\bar{\gamma}_{t,n}}-e^{\bar{m}_n[g]}; A_{t+n-1}(\epsilon)| \mathscr F_{t+n-1}]\big| + 2\mathbb P_\mu(A_{t+n-1}(\epsilon) \Delta D^c| \mathscr F_{t+n-1})\Big] \\
    & = \mathbb{P}_{\mu}\Big[ \big|\mathbb{P}_{\mu}[e^{i\bar{\gamma}_{t,n}}| \mathscr F_{t+n-1}]-e^{\bar{m}_n[g]}\big|;A_{t+n-1}(\epsilon)\Big] + J'_3(t,n,\epsilon) \\
    & = \mathbb{P}_{\mu}\Big[\big|e^{\langle (Z'''_1+Z_1)(-\sum_{p\in \mathcal{N}}\theta_{t,p,n}\phi_p), X_{t+n-1}\rangle}-e^{\bar{m}_n[g]}\big|;A_{t+n-1}(\epsilon)\Big]+ J'_3(t,n,\epsilon) \\
    & \leq \mathbb{P}_{\mu}\Big[\big|\langle (Z'''_1+Z_1)(-\sum_{p\in \mathcal{N}}
 \theta_{t,p,n}\phi_p), X_{t+n-1}\rangle-\bar{m}_n[g]\big|;A_{t+n-1}(\epsilon)\Big]+ J'_3(t,n,\epsilon) \\
    & \leq J'_1(t,n,\epsilon)+J'_2(t,n,\epsilon)+J'_3(t,n,\epsilon).
  \end{align}
  
  Step 2.
  We will show that for $\epsilon>0$ small enough, there exist $C_1,\delta_1>0$ such that for all $n\geq 1$ and $t\geq 0$, $ J'_1(t,n,\epsilon)\leq C_1 e^{-\delta_1(t+n)}. $
  Let $\delta_0$ be the constant in Lemma \ref{lem: upper bound for usgx}.(7) and $R$ be the corresponding $(\theta^{2+\beta}\vee \theta^{1+\beta+\delta_0})$-controller.
  Then we have for all $n\geq 1, t\geq 0, \epsilon > 0$,
  \begin{align}
    & |Z'''_1(-\sum_{p\in \mathcal{N}}\theta_{t,p,n}\phi_p)|\mathbf{1}_{A_{t+n-1}(\epsilon)}
      \leq R(|\sum_{p\in\mathcal{N}}\theta_{t,p,n}\phi_p|)\mathbf{1}_{A_{t+n-1}(\epsilon)} \\
    & \leq R \Big( \frac{ \sum_{p\in\mathcal{N} } | a_p | e^{ - ( \alpha - | p | b ) n } | \phi_p | } { \big( e^{ - \alpha ( n - 1 ) } e^{ ( \alpha - \epsilon ) ( t + n - 1 ) } \big)^{ \frac { 1 } { 1 + \beta } } } \Big) \\
    & \leq \sum_{ \rho \in \{ \delta_0, 1 \} } \Big( \frac { e^{ - ( \alpha - K b ) n } }  { \big( e^{ - \alpha ( n - 1 ) } e^{ ( \alpha - \epsilon ) ( t + n - 1 ) } \big)^{ \frac { 1 } { 1 + \beta } } } \Big)^{ 1 + \beta + \rho } R h \\
    & = \sum_{ \rho \in \{ \delta_0, 1 \} } e^{ - \alpha \frac{ 1 + \beta + \rho } { 1 + \beta } } e^{ - \frac { 1 + \beta + \rho } { 1 + \beta } ( \alpha \beta - K ( 1 + \beta ) b ) n } e^{ - \frac{ 1 + \beta + \rho } { 1 + \beta } ( \alpha - \epsilon ) ( t + n - 1 ) } R h \\
    & \leq \sum_{\rho\in\{\delta_0,1\}}e^{-\frac{1+\beta+\rho}{1+\beta}(\alpha-\epsilon)(t+n-1)}Rh,
  \end{align}
  where $h=\sum_{p\in \mathcal{N}}|a_p\phi_p|$.
  Thus there exists $C_3>0$ such that for $n\geq 1,  t\geq 0$ and $\epsilon > 0$,
  \begin{align}
    \label{eq: estimate of J11}
    J'_1(t,n,\epsilon)
    & \leq  \sum_{ \rho \in \{ \delta_0, 1\} } e^{ - \frac{ 1 + \beta + \rho } { 1 + \beta } ( \alpha - \epsilon ) ( t + n - 1 ) } \mathbb{ P }_{ \mu }[ \langle R h, X_{ t + n - 1 } \rangle] \\
    & \leq C_1 \sum_{ \rho \in \{ \delta_0, 1\} } \exp \Big \{ - \Big( \alpha \frac{ \rho } { 1 + \beta } - \epsilon \frac{ 1 + \beta + \rho } { 1 + \beta } \Big) ( t + n - 1 ) \Big\}.
  \end{align}
  By choosing $\epsilon>0$ small enough, we get the desired result in this step.
  
  Step 3.
  We will show that for each $\epsilon > 0$ small enough, there exist $C_2,\delta_2>0$ such that for all $n\geq 1, t\geq 0$, $ J'_2(t,n,\epsilon) \leq C_2 e^{-\delta_2(t+n)}.$
  In fact, according to the definitions of $Z_1$ and $\bar{m}_n$, and \eqref{equ: transform of mn}, we have for all $t\geq 0, n\geq1$,
  \begin{align}
    & \langle Z_1(-\sum_{p\in \mathcal{N}}\theta_{t,p,n}\phi_p), X_{t+n-1}\rangle-\bar{m}_n[g]
      = e^{\alpha(n-1)}\Big(\frac{\langle Z_1(-I_ng),X_{t+n-1}\rangle}{\|X_{t+n-1}\|}-\langle Z_1(-I_ng),\varphi\rangle\Big).
  \end{align}
  Therefore, for all $t\geq 0, n\geq1$ and $\epsilon > 0$,
  \begin{align}
    & J'_2(t,n,\epsilon)
      = \mathbb{P}_{\mu}\big[|\langle Z_1(-\sum_{p\in\mathcal{N}}\theta_{t,p,n}\phi_p),X_{t+n-1}\rangle-\bar{m}_n[g]|; A_{t+n-1}(\epsilon)\big]
    \\ & \leq e^{\alpha(n-1)}\mathbb{P}_{\mu}\Big[\big|\frac{\langle Z_1(-I_ng),X_{t+n-1}\rangle}{\|X_{t+n-1}\|}-\langle Z_1(-I_ng),\varphi\rangle\big|; A_{t+n-1}(\epsilon)\Big]
    \\ & \leq e^{\alpha(n-1)}e^{-(\alpha-Kb)(1+\beta)n}e^{-(\alpha-\epsilon)(t+n-1)}\mathbb{P}_{\mu} [ \langle \bar {g}_n, X_{t+n-1} \rangle],
  \end{align}
  where
  \begin{align}
    \bar{g}_n
    := \frac{Z_1(-I_ng)-\langle Z_1(-I_ng),\varphi\rangle}{e^{-(\alpha-Kb)(1+\beta)n}}.
  \end{align}
  Fix a $\gamma>0$ small enough such that $\alpha \gamma < (1+\gamma)b$.
  Using Lemma \ref{control of gn} and Lemma \ref{lem: control moment}.(3) with $\kappa=1$, there exists $C_3>0$ such that
  \begin{align}
    \mathbb{P}_{\mu}[\langle \bar{g}_n, X_{t+n-1}\rangle]
    \leq C_3e^{\frac{\alpha}{1+\gamma}(t+n-1)},\quad t\geq0, n\geq 1.
  \end{align}
  Therefore, for all $t\geq 0, n\geq1$ and $\epsilon >0$,
  \begin{align}
    & J'_2(t,n,\epsilon)\leq e^{\alpha(n-1)}e^{-(\alpha-Kb)(1+\beta)n}e^{-(\alpha-\epsilon)(t+n-1)}\mathbb{P}_{\mu}[\langle \bar{g}_n, X_{t+n-1}\rangle]
    \\ & \leq C_3 e^{\alpha(n-1)}e^{-(\alpha-Kb)(1+\beta)n}e^{-(\alpha-\epsilon)(t+n-1)}e^{\frac{\alpha }{1+\gamma}(t+n-1)} \\
   & = C_3 e^{-\alpha} e^{ - ( \alpha \beta - K ( 1 + \beta ) b ) n } e^{ - ( \frac{ \alpha \gamma } { 1 + \gamma } - \epsilon ) ( t + n - 1 ) }
     \leq C_3 e^{-( \frac{ \alpha \gamma } { 1 + \gamma } - \epsilon ) ( t + n - 1 ) }.
  \end{align}
  Now, by choosing $\epsilon >0$ small enough, we get the desired result in this step.

  Step 4. According to the Step 4 of the proof of  Lemma \ref{lem: mainlemma}, we have for any $\epsilon\in (0,  \alpha)$ there exist $C_4,\delta_3>0$ such that $ J'_3(t,n,\epsilon)\leq C_4e^{-\delta_3 (t+n)}$ for all $ t\geq0, n\geq 1$.

  Finally, combining the results in Steps 1--4. and noticing that, if $\epsilon>0$ is chosen small enough then $J_{i}, i = 1,2,3$, converge to $0$ exponentially fast when $t\rightarrow\infty$, we get the desired result.
\end{proof}

\begin{lem}
  \label{lem: independency for large rate}
  For all $\theta\in \mathbb{R}$ and $\mu \in \mathcal{M}_c(\mathbb{R}^d)$, there exist $C,\delta>0$ such that for all $t\geq 0$, $m\in\mathbb{N}$, we have
  \begin{align}
    \label{ineq: next we will need}
    \bigg| \widetilde{ \mathbb{P} }_{ \mu } \Big[ \prod_{ l = 1 }^m \exp ( i \theta \bar{ \gamma }_{t, l} ) - \prod_{ l = 1 }^m \exp( \bar{m}_l[\theta g])\Big]\bigg|\leq C e^{-\delta t}.
  \end{align}
\end{lem}
\begin{proof}
  Step 1. Fix $\theta \in \mathbb R$ and $\mu \in \mathcal M_c(\mathbb R^d)$.
  Similar to the proof of Corollary \ref{cor: used in next corollary}, we can show that  for all $\theta\in \mathbb{R}$ and $\mu\in \mathcal{M}_c(\mathbb{R}^d)$, there exist $C_1,\delta_1>0$ such that, for any $t\geq 0$, $n\geq 1$, we have
  \begin{align}
    \mathbb{P}_{\mu}\Big[\big|\mathbb{P}_{\mu}[e^{i\theta\bar{\gamma}_{t,n}}-e^{\bar{m}_n[\theta g]}; D^c | \mathscr F_{t+n-1}]\big|\Big]\leq C_1e^{-\delta_1(t+n-1)}.
  \end{align}

  Step 2.
  Fix $t\geq 0$ and $m\in \mathbb{N}$.
  For each $n=0,\cdots,m$, we define
  \[
    \widetilde{a}_n
    := \mathbb{\widetilde{P}}_{\mu}\Big[\prod_{l=0}^{n}\exp(i\theta\bar{\gamma}_{t,l})\Big] \times \prod_{l=n+1}^{m}\exp(\bar{m}_l[\theta g]),
  \]
  where by convention the first product is $1$ for $n=0$. Then we get for each $n \ge 1$,
  \begin{align}
    & \widetilde{a}_{n-1} - \widetilde{a}_n
      = \mathbb{P}_{ \mu } ( D^c )^{ - 1 } \mathbb{P}_{ \mu } \Big[ \Big( \prod_{ l=0 }^{n - 1 } e^{ i \theta \bar{ \gamma }_{ t, l} } \Big) \times ( e^{ \bar{m}_n [ \theta g ] } - e^{ i \theta \bar{ \gamma }_{t, n}}) ; D^c \Big] \times \prod_{ l = n + 1}^{ m } e^{\bar{ m }_l[ \theta g ] } \\ 
    & = \mathbb{ P }_{ \mu }( D^c )^{ - 1 } \mathbb{P}_{ \mu } \Big[ \Big( \prod_{ l = 0 }^{n - 1} e^{ i \theta \bar{ \gamma }_{t, l} } \Big) \times \mathbb{P}_{\mu}[e^{\bar{m}_n[\theta g]}-e^{i\theta \bar{\gamma}_{t,n}};D^c|\mathscr{F}_{t+n-1}]\Big]\times \prod_{l=n+1}^{m}e^{\bar{m}_l[\theta g]}.
  \end{align}
  According to Step 1 and Proposition \ref{prop: alpha stable rv}, there exists $C_2>0$ such that for any $n=1,\cdots, m$ and $t\geq 0$, we have
  \begin{align}
    | \widetilde{a}_{n-1}- \widetilde{a}_n|
    & \leq \frac{1}{\mathbb{P}_{\mu}(D^c)}\mathbb{P}_{\mu}\Big[\big|\mathbb P_\mu[e^{i\theta\bar{\gamma}_{t,n}}-e^{\bar{m}_n[\theta g]}; D^c\big|\mathscr{F}_{t+n-1}]\big|\Big]
    \leq C_2 e^{-\delta_1(t+n-1)}.
  \end{align}
  Therefore,
  \begin{align}
    \text{LHS of \eqref{ineq: next we will need}}
    & = \left|\widetilde{a}_{0}-\widetilde{a}_m\right|
      \leq\sum_{n=1}^m\left|\widetilde{a}_{n-1}-\widetilde{a}_n\right|
      \leq \sum_{n=1}^m C_2 e^{-\delta_1(t+n-1)}.
  \end{align}
	Notice that $C_2, \delta_1>0$ are independent of the choice of $t\geq 0$, $m\in \mathbb{N}$.
\end{proof}
Now we define
\begin{align}
  \widetilde{\gamma}_{t,n}
  :=\sum_{p\in \mathcal{N} } a_p \frac { H^p_{t+n-1} - H^p_{t+n} } { e^{-(\alpha-|p|b)t} \| X_{t} \|^{ \frac{1}{1+\beta}}},
  \quad t\geq 0, n\geq 1.
\end{align}
\begin{lem}
  \label{lem: lemma04}
  For all $\theta\in \mathbb{R}$ and $\mu\in \mathcal{M}_c(\mathbb{R}^d)$, there exist $C,\delta>0$ such that for all $m\in \mathbb{N}$ and $t\geq 0$, we have
  \begin{align}
    \Big|\widetilde{\mathbb{P}}_{\mu}[\prod_{n=1}^m e^{i\theta \widetilde{\gamma}_{t,n}}]-\widetilde{\mathbb{P}}_{\mu}[\prod_{n=1}^me^{i\theta \bar{\gamma}_{t,n}}]\Big|
    \leq C m e^{-\delta t}.
  \end{align}
\end{lem}
\begin{proof}
  Fix a $\theta \in \mathbb R$ and a $\mu \in \mathcal M_c(\mathbb R^d)$.
  According to \cite[Lemma 3.4.3]{Durrett2010Probability},
  \begin{align}
    \label{ineq: used next 3}
    \Big|\widetilde{\mathbb{P}}_{\mu}[\prod_{n=1}^me^{i\theta \widetilde{\gamma}_{t,n}}]-\widetilde{\mathbb{P}}_{\mu}[\prod_{n=1}^me^{i\theta \bar{\gamma}_{t,n}}]\Big|\leq \sum_{n=1}^m\widetilde{\mathbb{P}}_{\mu}[|Y''_{t,n}|],
  \end{align}
  where $Y''_{t,n}:=e^{i\theta\widetilde{\gamma}_{t,n}}-e^{i\theta\bar{\gamma}_{t,n}}$.
  Let $\gamma \in (0,\beta)$ be close enough to $\beta$ such that
  \[
    \frac{\alpha \gamma}{1+\gamma}
    > \frac{\alpha}{1+\gamma} - \frac{\alpha}{1+\beta} > 0,
    \qquad
    \alpha\gamma
    >K(1+\gamma)b,
  \]
  where $K$ is defined in \eqref{eq: def of N}.
  Fix this $\gamma$, and then choose $\eta_0,\eta_1>0$ such that
  \[
    \frac{\alpha \gamma}{1+\gamma}
    > \eta_0
    > \eta_0 - 3\eta_1
    > \frac{\alpha}{1+\gamma} - \frac{\alpha}{1+\beta}
    > 0.
  \]
  Define for all $n \geq 1$ and $t\geq 0$,
  \begin{align}
    \label{def: Dtk1}
    \mathcal{D}_{t,n}
    & :=\left\{|H_t-H_{t+n-1}|\leq  e^{-\eta_0 t}, H_{t}> 2e^{-\eta_1t}\right\}.
  \end{align}

  Step 1. 
  Similar to Step 1 in the proof of Theorem \ref{thm: critical clt}, we can show that there exist $C_1,\delta_1 >0$ such that for all $n \geq 1$ and $t\geq 0$, $ \mathbb{ \widetilde{ P } }_{ \mu } \big[ | Y''_{ t, n } | \mathbf{ 1 }_{ \mathcal{ D }^c_{t, n } } \big] \leq C_1 e^{-\delta_1 t}. $
  
  Step 2.
  We will show that there exist $C_2,\delta_2 > 0$ such that for all $n\geq 1$ and $t\geq 0$,
  \begin{align}
    \label{thm12211}
    \mathbb{\widetilde{P}}_{\mu}\big[|Y''_{t,n}|\mathbf{1}_{\mathcal{D}_{t,n}}\big]
    \leq C_1e^{-\delta_2 t}.
  \end{align}
  In fact, since $|e^{ix}-e^{iy}|\leq|x-y|$ for all $x,y\in \mathbb R$, we have, for all $t \geq 0$ and $n\geq 1$,
  \begin{align}
    \label{large: used next}
    & \widetilde{\mathbb{P}}_{\mu}\big[|Y''_{t,n}|\mathbf{1}_{\mathcal{D}_{t,n}}\big]\\
    & \leq |\theta|\sum_{p\in\mathcal{N}}|a_p|e^{(\alpha-|p|b)t}\widetilde{\mathbb{P}}_{\mu}\Big[|H_{t+n-1}^p-H_{t+n}^p|\cdot\Big|\frac{1}{(e^{-\alpha(n-1)}\|X_{t+n-1}\|)^{\frac{1}{1+\beta}}}-\frac{1}{\|X_t\|^{\frac{1}{1+\beta}}}\Big|\mathbf{1}_{\mathcal{D}_{t,n}}\Big]\\
    & \leq |\theta| \sum_{p\in\mathcal{N}} |a_p| e^{(\alpha-|p|b)t}e^{-\frac{\alpha}{1+\beta}t}\widetilde{\mathbb{P}}_{\mu}\Big[|H_{t+n-1}^p-H_{t+n}^p|\cdot K'_{t,n}\Big],
  \end{align}
  where
  \begin{align}
    K'_{t,n}
    := \Big| \frac{ H_t^{ \frac { 1 } { 1 + \beta } } - H_{ t + n - 1 }^{ \frac { 1 }{ 1 + \beta } } } { H_t^{ \frac{ 1 }{ 1 + \beta }} H_{t+n-1}^{\frac{1}{1+\beta}}}\Big|\mathbf{1}_{\mathcal{D}_{t,n}}.
  \end{align}
	Since $\eta_1 < \eta_0$, using an argument similar to that used in Step 2 of the proof of Theorem \ref{thm: critical clt}, we can show that, there is a constant $C_3> 0$ such that,
  \begin{align}
    \label{ineq: control of Kkt1}
    K'_{t,n}
    \leq C_3 e^{-(\eta_0 - 3\eta_1) t},
    \quad t \geq 0, n\geq 1.
  \end{align}
  According to Lemma \ref{lem: control of wt}, there exists $C_4>0$ such that for all $t\geq 0$ and $n\geq 1$,
  \begin{align}
    \label{ineq:used next 2}
    & \widetilde{\mathbb{P}}_{\mu}\big[|Y''_{t,n}|\mathbf{1}_{\mathcal{D}_{t,n}}\big]
      \leq \frac{C_3}{\mathbb{P}_{\mu}(D^c)} |\theta| \sum_{ p \in \mathcal{N} } |a_p| e^{ ( \alpha - |p| b ) t } e^{ - \frac{ \alpha } { 1 + \beta } t } e^{ - ( \eta_0 - 3 \eta_1 ) t } \mathbb{ P }_{ \mu } \Big[ | H_{t+n-1}^p - H_{t+n}^p | \Big]\\
    & \leq \frac{ C_3 }{ \mathbb{P}_{ \mu }( D^c ) } |\theta| \sum_{p\in\mathcal{N}}| a_p | e^{(\alpha-|p|b)t} e^{ - \frac{ \alpha } { 1 + \beta } t } e^{ - ( \eta_0 - 3 \eta_1 ) t } \| H_{ t + n - 1 }^p - H_{ t + n}^p \|_{ \mathbb{ P }_{ \mu }; 1 + \gamma} \\
    & \leq C_4 | \theta | \sum_{ p \in \mathcal{N} } | a_p | e^{ ( \alpha - |p| b ) t} e^{ - \frac{ \alpha } { 1 + \beta } t } e^{ - ( \eta_0 - 3 \eta_1 )t  } e^{ - \frac{ 1 } { 1 + \gamma } ( \alpha \gamma - | p |(1+\gamma)b)(t+n-1)}\\
    & = C_4 | \theta | \sum_{ p \in \mathcal{ N } } | a_p | \exp \Big \{ \Big( \frac { \alpha } { 1 + \gamma } - \frac{ \alpha } { 1 + \beta } - ( \eta_0 - 3 \eta_1 ) \Big) t \Big \} \cdot e^{ - \frac { 1 } { 1 + \gamma }( \alpha \gamma - | p | ( 1 + \gamma ) b ) ( n - 1 ) }.
  \end{align}
  \eqref{thm12211} now follows  since $\frac{\alpha}{1+\gamma}-\frac{\alpha}{1+\beta}<\eta_0-3\eta_1$ and $\alpha\gamma>Kb(1+\gamma)\geq |p|b(1+\gamma)$.

  Combining \eqref{ineq: used next 3} with Steps 1-2, we get the result immediately. Notice that $\delta$ is independent of $m$ and $t$.
\end{proof}
\begin{lem}
  \label{lem: lemma05}
  For each $\theta\in \mathbb{R}$ and $\mu \in \mathcal{M}_c(\mathbb{R}^d)$, there exist  $C,\delta_1,\delta_2>0$ such that
  \begin{align}
    \Big|\widetilde{\mathbb{P}}_{\mu}\Big[\exp(i\theta \sum_{n=m}^{\infty}\widetilde{\gamma}_{t,n})-1\Big]\Big|
    \leq C(e^{-\delta_1 t}+e^{\delta_1 t}e^{-\delta_2 m}),
    \quad m\in \mathbb{N}, t\geq 0.
  \end{align}
\end{lem}
\begin{proof}
  Fix $\theta \in \mathbb R$, $\mu \in \mathcal M_c(\mathbb R^d)$, $\epsilon \in (0,\alpha)$ and $\gamma \in (0,\beta)$ with $\alpha \gamma > Kb(1+\gamma)$.
  Recall that $A_t(\epsilon):=\{\|X_t\|> e^{(\alpha-\epsilon) t}\}$.
  According to Proposition \ref{lem: control of XT}, there exist $C_1,\delta>0$ such that for all $t\geq 0$ and $m\in \mathbb N$,
  \begin{align}
    \label{ineq: control of gamma_t on dc}
    \Big| \widetilde{\mathbb{P}}_{\mu}\Big[\big(\exp(i\theta \sum_{n=m}^{\infty}\widetilde{\gamma}_{t,n})-1\big)\mathbf{1}_{A_t(\epsilon)^c}\Big]
    \leq 2 \widetilde{P}_{\mu}(A_t(\epsilon)^c)
    \leq C_1 e^{-\delta t}.
  \end{align}
  According to Lemma \ref{lem: control of wt}, there exist $C_2,C_3>0$ such that for all $m\in \mathbb{N}, t\geq$,
  \begin{align}
  \label{ineq: control of gamma_t on d}
    & \Big|\widetilde{\mathbb{P}}_{\mu}\Big[\big(\exp(i\theta \sum_{n=m}^{\infty} \widetilde{ \gamma }_{ t, n}) - 1 \big) \mathbf{1}_{ A_t( \epsilon ) } \Big] \Big| \\
    & \leq \frac{1}{\mathbb{P}_{\mu}(D^c)}|\theta|\sum_{n=m}^{\infty}\sum_{p\in \mathcal{N}}|a_p|\mathbb{P}_{\mu}\Big[\frac{|H_{t+n-1}^p-H_{t+n}^p|}{e^{-(\alpha-|p|b)t}\|X_t\|^{\frac{1}{1+\beta}}}\mathbf{1}_{A_t(\epsilon)}\Big]\\
    & \leq  \frac{1}{\mathbb{P}_{\mu}(D^c)}|\theta|\sum_{p\in \mathcal{N}}|a_p|e^{(\alpha-|p|b)t}e^{-\frac{\alpha-\epsilon}{1+\beta}t}\sum_{n=m}^{\infty}\|H_{t+n-1}^p-H_{t+n}^p\|_{\mathbb{P}_{\mu};1+\gamma}\\
    & \leq C_2|\theta|\sum_{p\in \mathcal{N} } |a_p| e^{ ( \alpha - |p| b ) t} e^{- \frac{ \alpha - \epsilon } { 1 + \beta } t } \sum_{ n = m}^{ \infty } e^{ - \frac{ 1 }{ 1 +\ gamma} ( \alpha \gamma - | p | ( 1 + \gamma ) b ) ( t + n - 1 ) } \\
    & \leq C_3 | \theta | \exp \Big\{ \Big( \frac { \alpha } { 1 + \gamma } - \frac { \alpha } { 1 + \beta } + \frac{ \epsilon } { 1 + \beta } \Big) t \Big\}
    e^{-\frac{1}{1+\gamma}(\alpha\gamma-K(1+\gamma)b)m}.
  \end{align}
  Combining \eqref{ineq: control of gamma_t on dc} and \eqref{ineq: control of gamma_t on d}, we complete the proof.
\end{proof}
Now we are ready to prove Theorem \ref{thm: large clt}.

\begin{proof}[Proof of Theorem \ref{thm: large clt}]
  Fix $\mu \in \mathcal M_c(\mathbb R^d)$ and $\theta \in \mathbb R$.
  Let $\delta_0$ be the constant $\delta$ in Lemma \ref{lem: lemma04}.
  Take $m=\lfloor e^{\frac{\delta_0}{2}t}\rfloor$.
  First note that for $t\ge 0$,
  \begin{align}\label{decompose}
    \text{LHS of \eqref{thm: large rate}}
    =\sum_{n=1}^{\infty}\sum_{p\in \mathcal{N}}a_p\frac{H^p_{t+n-1}-H^p_{t+n}}{e^{-(\alpha-|p|b)t}\|X_t\|^{\frac{1}{1+\beta}}}=\sum_{n=1}^m\widetilde{\gamma}_{t,n}+\sum_{n=m+1}^{\infty}\widetilde{\gamma}_{t,n}.
  \end{align}
  According to Lemma \ref{lem: independency for large rate}, there exist $C_1,\delta_1>0$ such that for all $t\geq 0$,
  \begin{align}
    \label{ineq: last1}
    \Big| \widetilde{ \mathbb{ P } }_{ \mu } \Big[ \exp( i \theta \sum_{ n = 1 }^m \gamma_{ t, n} ) - \exp( \sum_{ n = 1 }^m \bar{ m }_n [ \theta g ] ) \Big]
    \Big|\leq C_1 e^{-\delta_1 t}.
  \end{align}
  According to Lemma \ref{lem: lemma04}, there exists $C_2>0$ such that for all $t\geq 0$,
  \begin{align}
    \label{ineq: last2}
    \Big|\widetilde{\mathbb{P}}_{\mu}[e^{i\theta\sum_{n=1}^m \widetilde{\gamma}_{t,n}}]-\widetilde{\mathbb{P}}_{\mu}[e^{i\theta \sum_{n=1}^m\gamma_{t,n}}]\Big|\leq C_2  e^{-\frac{\delta_0}{2} t}.
  \end{align}
  According to Lemma \ref{lem: lemma05}, there exist $C_3,\delta_2,\delta_3>0$ such that for all $t\geq 0$,
  \begin{align}
    \label{ineq: last3}
    \Big|\widetilde{\mathbb{P}}_{\mu}\Big[\exp(i\theta \sum_{n=m}^{\infty}\widetilde{\gamma}_{t,n})-1\Big]\Big|\leq C_3(e^{-\delta_2 t}+e^{\delta_2 t}e^{-\delta_3 e^{\frac{\delta_0}{2}t}}).
  \end{align}
  Also note that $\sum_{n=1}^\infty\bar{m}_n[\theta g]=\bar{m}[\theta g]$, see \eqref{sum-bar-m}.

  Combining \eqref{decompose},\eqref{ineq: last1}, \eqref{ineq: last2} and \eqref{ineq: last3}, and letting $t\to\infty$, we get the desired result of Theorem \ref{thm: large clt}.
\end{proof}

% ** Appendix
\appendix
\section{ }
\subsection{Analytic facts}
In this subsection, we collect some useful analytic facts.
\begin{lem}
  \label{lem: estimate of exponential remaining}
  For $z\in \mathbb C_+$,  we have
  \begin{equation}
    \label{eq: estimate of exponential remaining}
    \Big|e^{-z} - \sum_{k=0}^n \frac{(-z)^k}{k!} \Big|
    \leq \frac{|z|^{n+1}}{(n+1)!} \wedge \frac{2|z|^{n}}{n!}, \quad n\in \mathbb Z_+.
  \end{equation}
\end{lem}
\begin{proof}
  Notice that $|e^{-z}| = e^{- \operatorname{Re} z} \leq 1$.
  Therefore,
\begin{equation}
  |e^{-z} - 1|
  = \Big| \int_0^1 e^{-\theta z} z d\theta\Big|
  \leq |z|.
\end{equation}
Also, notice that $|e^{-z} - 1| \leq |e^{-z}|+1 \leq 2$.
Thus \eqref{eq: estimate of exponential remaining} is true when $n = 0$.
Now, suppose that \eqref{eq: estimate of exponential remaining} is true when $n = m$ for some $m \in \mathbb Z_+$.
Then
\begin{align}
  &\Big|e^{-z} - \sum_{k=0}^{m+1} \frac{(-z)^k}{k!}\Big|
    = \Big| \int_0^1\Big(e^{-\theta z} - \sum_{k=0}^m \frac{(-\theta z)^k}{k!} \Big) z d\theta \Big| \\
  & \quad \leq  \Big(\int_0^1 \frac{|\theta z|^{m+1}}{(m+1)!} |z| d\theta\Big) \wedge \Big(\int_0^1 \frac{2|\theta z|^{m}}{m!} |z| d\theta\Big)
    = \frac{|z|^{m+2}}{(m+2)!} \wedge \frac{2|z|^{m+1}}{(m+1)!},
\end{align}
which says that \eqref{eq: estimate of exponential remaining} is true for $n = m + 1$.
\end{proof}

\begin{lem}
  \label{lem: extension lemma for branching mechanism}
  Suppose that  $\pi$ is a measure on $(0,\infty)$ with $\int_{(0,\infty)} (y \wedge y^2) \pi(dy)< \infty$.
  Then the functions
  \begin{align}
    & h (z) 
      = \int_{(0,\infty)} (e^{-zy} - 1 + zy) \pi(dy), \quad z \in \mathbb C_+ \\ 
    & h'(z) 
      = \int_{(0,\infty)}(1- e^{-zy})y \pi(dy), \quad z \in \mathbb C_+
  \end{align}
  are well defined, continuous on $\mathbb C_+$ and holomorphic on $\mathbb C_+^0$.
  Moreover,
  \[
    \frac{h(z)-h(z_0)}{z-z_0} 
    \xrightarrow[\mathbb C_+\ni z \to z_0]{} h'(z_0),\quad z_0 \in \mathbb C_+.
  \]
\end{lem}
\begin{proof}
  It follows from Lemma \ref{lem: estimate of exponential remaining} that $h$ and $h'$ are well defined on $\mathbb C_+$.
  According to \cite[Theorems 3.2. \& Proposition 3.6]{SchillingSongVondravcek2010Bernstein}, $h'$ is continuous on $\mathbb C_+$ and holomorphic on $\mathbb C_+^0$.
  
  It follows from Lemma \ref{lem: estimate of exponential remaining} that, for each $z_0 \in \mathbb C_+$,  there exists $C>0$ such that for $z \in \mathbb C_+$ close enough to $z_0$ and any $y>0$,
\begin{align}
  & \Big| \frac{e^{-zy} - e^{-z_0 y}+(z-z_0) y}{z-z_0} \Big|
    = \frac{1}{|z-z_0|}\Big| \int_0^1 \big(-y e^{-(\theta z+(1-\theta)z_0)y}+y\big)(z-z_0)d\theta\Big| \\ 
  & \leq y\int_0^1 |1-e^{-(\theta z +(1-\theta)z_0)y}| d\theta
    \leq (2y) \wedge\Big( y^2\int_0^1|\theta z+(1-\theta)z_0|d\theta\Big)
    \leq C(y\wedge y^2).
\end{align}
Using this and the dominated convergence theorem, we have
\begin{align}
  & \frac{h(z)-h(z_0)}{z-z_0} = \int_{(0,\infty)} \frac{e^{-zy}+zy -(e^{-z_0 y}+z_0 y)}{z-z_0}  \pi(dy) \\
  & \xrightarrow[\mathbb C_+\ni z\to z_0]{} \int_{(0,\infty)}(1 - e^{-z_0 y} )y\pi(dy) = h'(z_0),
\end{align}
which says that $h$ is continuous on $\mathbb C_+$ and holomorphic on $\mathbb C_+^0$.
\end{proof}

For each $z\in \mathbb C\setminus (-\infty,0]$, we define
$
\log z := \log |z| + i \arg z
$
where $\arg z \in (-\pi,\pi)$ is uniquely determined by
$
z = |z|e^{i \arg z}.
$ 	
For all $z\in \mathbb C\setminus (-\infty,0]$ and $\gamma \in \mathbb C$, we define
$
z^\gamma := e^{\gamma \log z}.
$
Then it is known, see \cite[Theorem 6.1]{SteinShakarchi2003Complex} for example, that $z\mapsto \log z$ is holomorphic in $\mathbb C\setminus (-\infty,0]$.
Therefore, for each $\gamma \in \mathbb C$, $z\mapsto z^\gamma$ is holomorphic in $\mathbb C\setminus (-\infty,0]$. (We use the convention that  $0^\gamma := \mathbf 1_{\gamma = 0}$.)
Using the definition above we can easily show that $(z_1z_0)^\gamma = z_1^\gamma z_0^\gamma$ provided $\arg (z_1z_0)=\arg (z_1) + \arg(z_0)$.

% It is known, see, for instance, \cite[Theorem 6.1.3]{SteinShakarchi2003Complex} and the remark following it, that the function $\Gamma$ has an unique analytic extension in $\mathbb C\setminus\{0, -1,-2,\dots\}$ and that
It is known, see, for instance, \cite[Theorem 6.1.3]{SteinShakarchi2003Complex} and the remark following it, that the Gamma function $\Gamma$ has an unique analytic extension in $\mathbb C\setminus\{0, -1,-2,\dots\}$ and that
\[
	\Gamma(z+1) = z \Gamma(z),\quad z\in \mathbb C\setminus\{0, -1,-2,\dots\}.
\]
Using this recursively, one gets that
\begin{equation}
  \label{eq: definition of Gamma function}
  \Gamma(x)
  := \int_0^\infty t^{x-1} \Big(e^{-t} - \sum_{k=0}^{n-1} \frac{(-t)^k}{k!}\Big) dt,
  \quad -n< x< -n+1, n\in \mathbb N.
\end{equation}

Fix a $\beta \in (0,1)$.
Using the uniqueness of holomorphic extension and Lemma \ref{lem: extension lemma for branching mechanism}, we get that
\begin{equation}
  z^{\beta}
	= \int_0^\infty (e^{-zy}-1) \frac{dy}{\Gamma(-\beta)y^{1+\beta}},
  \quad z\in \mathbb C_+,
\end{equation}
% are (i)  extension of the real function $x\mapsto x^{\beta}$ defined on $[0,\infty)$; (ii) holomorphic on $\mathbb C_+^0$ and (iii) continuous on $\mathbb C_+$.
% Similarly, we get that
and similarly that
\begin{equation}
  \label{eq: stable branching on C+}
  z^{1+\beta}
  = \int_0^\infty (e^{-zy}-1+zy)\frac{dy}{\Gamma(-1-\beta)y^{2+\beta}},
  \quad z\in \mathbb C_+.
\end{equation}
Lemma \ref{lem: extension lemma for branching mechanism} also says that the derivative of $z^{1+\beta}$ is $(1+\beta)z^{\beta}$ on $\mathbb C^0_+$.
\begin{lem}
  \label{lem: Lip of power function}
  For all $z_0,z_1 \in \mathbb C_+$, we have
\begin{equation}
  \label{eq: Lip of power function}
  |z_0^{1+\beta} - z_1^{1+\beta}|
  \leq (1+\beta)(|z_0|^{\beta}+|z_1|^{\beta})|z_0 - z_1|.
\end{equation}
\end{lem}
\begin{proof}
  Since $z^{1+\beta}$ is continuous on $\mathbb C_+$, we only need to prove the lemma assuming $z_0,z_1 \in \mathbb C^0_+$.
  Notice that
\begin{equation}
  \label{eq: upper bound for beta power of z}
	|z^\beta|
	= |e^{\beta \log |z| +i\beta \operatorname {arg}z}| = e^{\beta \log |z|} = |z|^\beta,
	\quad z \in \mathbb C\setminus (-\infty, 0].
\end{equation}
Define a path $\gamma: [0,1] \to \mathbb C^0_+$ such that
\[
  \gamma(\theta)
  = z_0 (1-\theta) + \theta z_1,
  \quad \theta \in [0,1].
\]
Then, we have
\begin{align}
  |z_0^{1+\beta} - z_1^{1+\beta}|
  & \leq (1+\beta) \int_0^1 |\gamma(\theta)^{\beta}|\cdot |\gamma'(\theta)|d\theta
    \leq (1+\beta)  \sup_{\theta \in [0,1]} |\gamma(\theta)|^{\beta} \cdot |z_1-z_0| \\
  & \leq (1+\beta)  ( |z_1|^{\beta}+|z_0|^{\beta} ) |z_1-z_0|.
    \qedhere
\end{align}
\end{proof}

Suppose that $\varphi(\theta)$ is a continuous function from $\mathbb R$ into $\mathbb C$ such that $\varphi(0) = 1$ and $\varphi(\theta) \neq 0$ for all $\theta \in \mathbb R$.
Then according to \cite[Lemma 7.6]{Sato2013Levy}, there is a unique continuous function $f(\theta)$ from $\mathbb R$ into $\mathbb C$ such that $f(0) = 0$ and $e^{f(\theta)} = \varphi(\theta)$.
Such a function $f$ is called the distinguished logarithm of the function $\varphi$ and is denoted as $\operatorname{Log} \varphi(\theta)$.
In particular, when $\varphi$ is the characteristic function of an infinitely divisible random variable $Y$,  $\operatorname{Log} \varphi(\theta)$ is called the L\'evy exponent of $Y$.
This distinguished logarithm should not be confused with the $\log$ function defined on $\mathbb C\setminus (-\infty, 0]$.
See the paragraph immediately after \cite[Lemma 7.6]{Sato2013Levy}.

% *** Feynman-Kac formula with complex valued functions
\subsection{Feynman-Kac formula with complex valued functions}
\label{seq: complex Feynman-Kac transform}
In this subsection we give a version of the Feynman-Kac formula with complex valued functions.
Suppose that $\{(\xi_t)_{t \in [r,\infty)}; (\Pi_{r,x})_{r\in [0,\infty), x\in E}\}$ is a (possibly non-homogeneous) Hunt process in a locally compact separable metric space $E$.
We write
\begin{equation}
  H^{(h)}_{(s,t)}
  := \exp\Big\{\int_s^t h(u,\xi_u) du\Big\},
  \quad 0 \leq s \leq t, h \in \mathcal B_b([0,t] \times E,\mathbb C).
\end{equation}
\begin{lem}
  \label{eq: complex FK}
  Let $t \geq 0$. Suppose that $\rho_1, \rho_2\in \mathcal B_b([0,t] \times E, \mathbb C)$ and $f\in \mathcal B_b(E, \mathbb C)$.
  Then
\begin{equation}
  \label{eq: expresion of g}
  g(r,x) 
  := \Pi_{r,x}[ H_{(r,t)}^{(\rho_1+\rho_2)} f(\xi_t)],\quad r \in [0,t], x\in E,
\end{equation}
is the unique locally bounded solution to the equation
\[
  g(r,x)
  = \Pi_{r,x} [ H_{(r,t)}^{(\rho_1)} f(\xi_t)]+\Pi_{r,x} \Big[ \int_r^tH_{(r,s)}^{(\rho_1)}\rho_2(s,\xi_s) g(s,\xi_s)~ds \Big],\quad r \in [0,t], x\in E.
\]
\end{lem}

\begin{proof}
  The proof is similar to that of \cite[Lemma A.1.5]{Dynkin1993Superprocesses}. We include it here for the sake of completeness.
  We first verify that \eqref{eq: expresion of g} is a solution.
  Notice that
  \begin{equation}
    \Pi_{r,x} \Big[ \int_r^t | H_{(r,t)}^{(\rho_1)}\rho_2(s,\xi_s) H_{(s,t)}^{(\rho_2)} f(\xi_t)| ~ds \Big]
    \leq  \int_r^t e^{(t-r)\|\rho_1\|_\infty}e^{(t-s)\|\rho_2\|_\infty}\|\rho_2\|_\infty\|f\|_\infty ~ds
    < \infty.
  \end{equation}
  Also notice that
  \begin{equation}
    \label{eq: crucial for Feynman-Kac}
    \frac{\partial}{\partial s} H^{(\rho_2)}_{(s,t)}= -H^{(\rho_2)}_{(s,t)}\rho_2(s,\xi_s),
    \quad s\in (0,t).
  \end{equation}
  Therefore, from the Markov property of $\xi$ and Fubini's theorem we get that
  \begin{align}
    & \Pi_{r,x} \Big[ \int_r^tH_{(r,s)}^{(\rho_1)}~(\rho_2 g)(s,\xi_s)~ds \Big]
      =\Pi_{r,x} \Big[ \int_r^t H_{(r,s)}^{(\rho_1)}\rho_2(s,\xi_s) \Pi_{s,\xi_s}[ H_{(s,t)}^{(\rho_1+\rho_2)} f(\xi_t)]~ds \Big] \\
    & = \Pi_{r,x} \Big[ \int_r^t H_{(r,t)}^{(\rho_1)}\rho_2(s,\xi_s) H_{(s,t)}^{(\rho_2)} f(\xi_t) ~ds \Big]
      = \Pi_{r,x} [ H_{(r,t)}^{(\rho_1)}f(\xi_t)(H_{(r,t)}^{(\rho_2)} - 1)] \\
    & = g(r,x) - \Pi_{r,x} [ H_{(r,t)}^{(\rho_2)} f(\xi_t)].
  \end{align}
  For uniqueness, suppose  $\widetilde g$ is another solution. Put $h(r) = \sup_{x\in E}|g(r,x) - \widetilde g(r,x)|$. 
  Then
  \[
    h(r) 
    \leq e^{t\|\rho_1\|_\infty}\|\rho_2\|_\infty \int_r^t h(s)ds,
    \quad r\le t.
  \]
  Applying Gronwall's inequality, we get that $h(r) =  0$ for $r\in [0,t]$.
\end{proof}

% ** Superprocesses
\subsection{Superprocesses}
\label{sec: definition of superprocess}
In this subsection, we will give the definition of a general superprocess.
Let $E$ be locally compact separable metric space. Denote by $\mathcal M(E)$ the collection of all the finite measures on $E$ equipped with the topology of weak convergence.
For each function $F(x,z)$ on $E\times \mathbb R_+$ and each $\mathbb R_+$-valued function $f$ on $E$, we use the following convention in this subsection:
\[
  F(x,f):= F(x,f(x)),\quad x\in E.
\]
A process $X=\{(X_t)_{t\geq 0}; (\mathbf P_\mu)_{\mu \in \mathcal M(E)}\}$ is said to be a $(\xi,\psi)$-superprocess if
\begin{itemize}
\item
  the spatial motion $\xi=\{(\xi_t)_{t\geq 0};(\Pi_x)_{x\in E}\}$ is an $E$-valued Hunt process with its lifetime denoted by $\zeta$;
\item
  the branching mechanism $\psi: E\times[0,\infty) \to \mathbb R$ is given by
\begin{equation}
  \label{eq: branching mechanism}
  \psi(x,z)=
  -\rho_1(x) z + \rho_2 (x) z^2 + \int_{(0,\infty)} (e^{-zy} - 1 + zy) \pi(x,dy).
\end{equation}
where $\rho_1 \in \mathcal B_b(E)$, $\rho_2 \in \mathcal B_b(E, \mathbb R_+)$ and $\pi(x,dy)$ is a kernel from $E$ to $(0,\infty)$ such that $\sup_{x\in E} \int_{(0,\infty)} (y\wedge y^2) \pi(x,dy) < \infty$;
\item
  $X=\{(X_t)_{t\geq 0}; (\mathbf P_\mu)_{\mu \in \mathcal M(E)}\}$ is an $\mathcal M(E)$-valued Hunt process with transition probability determined by
  \begin{equation}
    \mathbf P_\mu [e^{-X_t(f)}] = e^{-\mu(V_tf)},
    \quad t\geq 0, \mu \in \mathcal M(E), f\in \mathcal B^+_b(E),
  \end{equation}
  where for each $f\in \mathcal B_b(E)$, the function $(t,x)\mapsto V_tf(x)$ on $[0,\infty) \times E$ is the unique locally bounded positive solution to the equation
  \begin{equation}
    \label{eq:FKPP_in_definition}
    V_tf(x) + \Pi_x \Big[  \int_0^{t\wedge \zeta} \psi(\xi_s,V_{t-s}f)ds \Big]
    = \Pi_x [ f(\xi_t)\mathbf 1_{t<\zeta} ],
    \quad t \geq 0, x \in E.
  \end{equation}
\end{itemize}
We refer our readers to \cite{Li2011Measure-valued} for more discussions about the definition and the existence of superprocesses.
To avoid triviality, we assume that $\psi(x,z)$ is not identically equal to $-\rho_1(x)z$.

Notice that the branching mechanism $\psi$ can be extended into a map from $E \times \mathbb C_+$ to $\mathbb C$ using \eqref{eq: branching mechanism}.
Define
\begin{equation}
  \psi'(x,z)
  := - \rho_1(x) + 2\rho_2(x) z + \int_{(0,\infty)} (1-e^{-zy})y\pi(x,dy),
  \quad x\in E, z\in \mathbb C_+.
\end{equation}
Then according to Lemma \ref{lem: extension lemma for branching mechanism}, for each $x \in E$, $z \mapsto \psi(x,z)$ is a holomorphic function on $\mathbb C_+^0$ with derivative $z \mapsto \psi'(x,z)$.
Define $\psi_0(x,z) := \psi(x,z)+ \rho_1(x)z $ and $\psi'_0(x,z) := \psi'(x,z) + \rho_1(x)$.

Denote by $\mathbb W$ the space of $\mathcal M(E)$-valued c\`{a}dl\`{a}g paths with its canonical path denoted by $(W_t)_{t\geq 0}$.
We say $X$ is \emph{non-persistent} if $\mathbf P_{\delta_x}(\|X_t\|= 0) > 0$ for all $x\in E$ and $t> 0$.
Suppose that $(X_t)_{t\geq 0}$ is non-persistent, then according to \cite[Section 8.4]{Li2011Measure-valued}, there is a unique family of measures $(\mathbb N_x)_{x\in E}$ on $\mathbb W$ such that
(i) $\mathbb N_x (\forall t > 0, \|W_t\|=0) =0$; 
(ii) $\mathbb N_x(\|W_0 \|\neq 0) = 0$; 
and (iii) if $\mathcal N$ is a Poisson random measure defined on some probability space with intensity $\mathbb N_\mu(\cdot):= \int_E \mathbb N_x(\cdot )\mu(dx)$, then the superprocess $\{X;\mathbf P_\mu\}$ can be realized by $\widetilde X_0 := \mu$ and $\widetilde X_t(\cdot) := \mathcal N[W_t(\cdot)]$ for each $t>0$.
We refer to $(\mathbb N_x)_{x\in E}$ as the \emph{Kuznetsov measures} of $X$.

% ** Semigroup for superproccesses
\subsection{Semigroups for superprocesses}
\label{sec: definition of vf}
Let $X$ be a non-persistent superprocess with its Kuznetsov measure denoted by $(\mathbb N_x)_{x\in E}$.
We define the mean semigroup
\begin{equation}
  P_t^{\rho_1} f(x)
  := \Pi_{x}[e^{\int_0^t \rho_1(\xi_s)ds}f(\xi_t) \mathbf 1_{t< \zeta}],
  \quad t\geq 0, x\in E, f\in \mathcal B_b(E,\mathbb R_+).
\end{equation}
It is known from \cite[Proposition 2.27]{Li2011Measure-valued} and \cite[Theorem 2.7]{Kyprianou2014Fluctuations} that for all $t > 0$, $\mu \in \mathcal M(E)$ and $f\in \mathcal B_b(E,\mathbb R_+)$,
\begin{equation}
  \label{eq: mean formula for superprocesses}
  \mathbb N_{\mu}[\langle W_t, f\rangle]
  =\mathbf P_{\mu}[\langle X_t, f\rangle]
  =\mu(P^{\rho_1}_t f).
\end{equation}

Define
\begin{align}
  L_1(\xi)
  &:= \{f\in \mathcal B(E): \forall x\in E, t\geq 0, \quad \Pi_x[|f(\xi_t)|]< \infty\}, \\
  L_2(\xi)
  &:= \{f \in \mathcal B(E): |f|^2 \in L_1(\xi)\}.
\end{align}
Using monotonicity and linearity, we get from \eqref{eq: mean formula for superprocesses}  that
\begin{equation}
  \mathbb N_x[\langle W_t, f\rangle]
  = \mathbf P_{\delta_x}[\langle f, X_t\rangle]
  = P^{\rho_1}_t f(x) \in \mathbb R,
  \quad f\in L_1(\xi), t > 0,x\in E.
\end{equation}
This says that the random variable $\langle X_t, f\rangle$ is well defined under probability $\mathbf P_{\delta_x}$ provided $f\in L_1(\xi)$.
By the branching property of the superprocess, $\langle X_t, f\rangle$ is an infinitely divisible random variable.
Therefore, we can write
\[
  U_t(\theta f)(x) 
  := \operatorname{Log} \mathbf P_{\delta_x}[e^{i \theta \langle X_t, f\rangle}],
  \quad t\geq 0, f\in L_1(\xi), \theta \in \mathbb R, x\in E,
\]
as its characteristic exponent.
According to Campbell's formula, see \cite[Theorem 2.7]{Kyprianou2014Fluctuations} for example, we have
\[
  \mathbf P_{\delta_x} [e^{i\theta \langle X_t, f\rangle}]
  = \exp(\mathbb N_x[ e^{i\theta \langle W_t, f\rangle} - 1]),
  \quad t>0, f\in L_1(\xi), \theta \in \mathbb R, x\in E.
\]
Noticing that $\theta \mapsto \mathbb N_x[e^{i\theta W_t(f)} - 1]$ is a continuous function on $\mathbb R$ and that $\mathbb N_x[e^{i\theta \langle W_t, f\rangle} - 1] = 0$ if $\theta = 0$, according to \cite[Lemma 7.6]{Sato2013Levy}, we have
\begin{equation}
  \label{eq: N and characteristic exponent}
  U_t(\theta f)(x) 
  = \mathbb N_x[e^{i \langle W_t, \theta f\rangle} - 1],
  \quad t>0, f\in L_1(\xi), \theta \in \mathbb R, x\in E.
\end{equation}

\begin{lem}
  There exists a constant $C\geq 0$ such that
  for all $f \in L_1(\xi),x\in E$ and $t\geq 0$, we have
\begin{equation}
  \label{eq: upper bound of psi(v)}
  \big|\psi\big(x,-U_tf\big)\big|
  \leq C P^{\rho_1}_t |f|(x) + C (P^{\rho_1}_t |f| (x))^2.
\end{equation}
\end{lem}
\begin{proof}
  Noticing that
  \[
    e^{\operatorname{Re} U_tf(x)}
    = |e^{U_tf(x)}|
    = |\mathbf P_{\delta_x}[e^{i \langle X_t, f\rangle}]|
    \leq 1,
  \]
  we have
  \begin{equation}
    \label{eq: -v has positive real part}
    \operatorname{Re} U_tf(x)
    \leq 0.
  \end{equation}
  Therefore, we can speak of $\psi(x,-U_tf)$ since $z\mapsto \psi(x,z)$ is well defined on $\mathbb C_+$.
  According to Lemma \ref{lem: estimate of exponential remaining}, we have that
  \begin{equation}
    \label{eq: upper bound for vf}
    |U_tf(x)| 
    \leq \mathbb N_x[|e^{i \langle W_t, f\rangle} - 1|]
    \leq \mathbb N_x[|i \langle W_t, f\rangle|]
    \leq (P^{\rho_1}_t |f|)(x).
  \end{equation}
  Notice that, for any compact $K \subset \mathbb R$,
  \begin{equation}
    \label{eq: estimate of deriavetive of v(theta)}
    \mathbb N_x \Big[\sup_{\theta \in K} \Big|\frac{\partial}{\partial \theta} (e^{i\theta \langle W_t, f\rangle} - 1) \Big|\Big]
    \leq \mathbb N_x[|\langle W_t, f\rangle|] \sup_{\theta \in K}|\theta|
    \leq (P^{\rho_1}_t |f|)(x) \sup_{\theta \in K}|\theta| < \infty.
  \end{equation}
  Therefore, according to \cite[Theorem A.5.2]{Durrett2010Probability} and \eqref{eq: N and characteristic exponent}, $ U_t( \theta f)( x )$ is differentiable in $\theta \in \mathbb R$ with
  \[
    \frac{\partial}{\partial \theta} U_t(\theta f)(x)
    = i\mathbb N_x[\langle W_t, f\rangle e^{i\theta \langle W_t, f\rangle}],
    \quad \theta \in \mathbb R.
  \]
  Moreover, from the above, it is clear that
  \begin{equation}
    \label{eq: upper bounded for derivative of v(theta)}
    \sup_{\theta \in \mathbb R}\Big| \frac{\partial}{\partial \theta}U_t(\theta f)(x)\Big|
    \leq ( P^{\rho_1}_t |f|)(x).
  \end{equation}
  It follows from the dominated convergence theorem that $(\partial/\partial \theta)U_t(\theta f)(x)$ is continuous in $\theta$.
  In other words, $\theta \mapsto -U_t(\theta f)(x)$ is a $C^1$ map from $\mathbb R$ to $\mathbb C_+$.
  Thus,
  \begin{equation}
    \label{eq: path integration representation of psi(v)}
    \psi(x,-U_tf)
    = -\int_0^1 \psi'\big(x,-U_t(\theta f)\big) \frac{\partial}{\partial \theta} U_t(\theta f)(x)~d\theta.
  \end{equation}
  Notice that
  \begin{align}
    & |\psi'(x, -U_tf)| \\
    & = \Big| -\rho_1(x)- 2\rho_2(x) U_tf(x)+ \int_{(0,\infty)} y (1- e^{y U_tf(x)} ) \pi(x,dy)\Big| \\
    & = \Big| - \rho_1(x)- 2\rho_2(x)\mathbb N_x[e^{i \langle W_t, f\rangle} - 1]  + \int_{(0,\infty)} y \mathbf P_{y \delta_x}[1-e^{i \langle X_t, f\rangle}] \pi(x,dy) \Big| \\ 
    & \leq \|\rho_1\|_\infty + 2\rho_2(x)\mathbb N_x[\langle W_t, |f|\rangle]+ \int_{(0,\infty)} y\mathbf P_{y\delta_x}[2\wedge \langle X_t, |f|\rangle] \pi(x,dy) \\ 
    & \leq \|\rho_1\|_\infty + 2\|\rho_2\|_\infty P^{\rho_1}_t |f|(x) + \Big(\sup_{x\in E}\int_{(0,1]}y^2 \pi(x,dy)\Big)~P^{\rho_1}_t |f|(x) + 2\sup_{x\in E}\int_{(1,\infty)} y \pi(x,dy) \\ 
    & =: C_1 + C_2(P^{\rho_1}_t |f|)(x), \label{eq: upper bound of psi'(v)}
  \end{align}
  where $C_1, C_2$ are constants independent on $f,x$ and $t$.
  Now, combining the display above with \eqref{eq: path integration representation of psi(v)} and \eqref{eq: upper bounded for derivative of v(theta)}
  we get the desired result.
\end{proof}
This lemma also says that if $f\in L^2(\xi)$ then
\[
  \Pi_x\Big[\int_0^t \psi(\xi_s,- U_{t-s}f)ds\Big]
  \in \mathbb C,
  \quad x\in E, t\geq 0.
\]
is well defined.
In fact, using Jensen's inequality and the Markov property, we have
\begin{align}
  \label{eq: domination of psi(v)}
  & \Pi_x\Big[\int_0^t \big|\psi \big(\xi_s,-U_{t-s}f\big)\big|ds\Big]
  \leq \Pi_x\Big[\int_0^t \big(C_1 P_{t-s}^{\rho_1}|f|(\xi_s)+C_2 P_{t-s}^{\rho_1}|f|(\xi_s)^2\big)ds\Big] \\ 
  & \leq \int_0^t \big(C_1 e^{t\|\rho_1\|}\Pi_x \big[ \Pi_{\xi_s}[|f(\xi_{t-s})|] \big]+C_2 e^{2t\|\rho_1\|}\Pi_x \big[ \Pi_{\xi_s}[|f (\xi_{t-s})|]^2 \big]\big)~ds \\ 
  & \leq \int_0^t (C_1 e^{t\|\rho_1\|}\Pi_x [ |f(\xi_{t})|]+C_2e^{2t\|\rho_1\|}\Pi_x [ |f (\xi_{t})|^2 ])~ds < \infty.
\end{align}

% ** A complex-valued non-linear integral equation
\subsection{A complex-valued non-linear integral equation}
Let $X$ be a non-persistent superprocess.
In this subsection, we will prove the following:
\begin{prop}
  \label{prop: complex FKPP-equation}
  If $f\in L_2(\xi)$,  then for all $t\geq 0$ and $x\in E$,
\begin{equation}
  \label{eq: complex FKPP-equation}
  U_tf(x) - \Pi_x \Big[\int_0^t \psi\big(\xi_s, - U_{t-s}f\big) ds \Big] 
  = i \Pi_x [f(\xi_t)].
\end{equation}
\begin{equation}
  \label{eq: complex FKPP-equation with FK-transform}
  U_tf(x) -  \int_0^t P_{t-s}^{\rho_1} \psi_0\big(\cdot,-U_sf\big) (x)~ds
  = iP_t^{\rho_1} f(x).
\end{equation}
\end{prop}

To prove this, we will need the generalized spine decomposition theorem from \cite{RenSongSun2017Spine}.
Let $f\in \mathcal B_b(E,\mathbb R_+)$, $T >0$ and $x\in E$.
Suppose that $\mathbf P_{\delta_x}[\langle X_T, f\rangle] = \mathbb N_x[\langle W_T, f\rangle] = P^{\rho_1}_T f(x) \in (0,\infty)$, then we can define the following probability transforms:
\begin{equation}
  d\mathbf P_{\delta_x}^{\langle X_T, f\rangle}
  := \frac{\langle X_T, f\rangle}{P_T^{\rho_1} f(x)} d\mathbf P_{\delta_x};
  \quad d\mathbb N_x^{\langle W_T, f\rangle}
  :=  \frac{\langle W_T, f\rangle}{P_T^{\rho_1} f(x)} d\mathbb N_x.
\end{equation}
Following the definition in \cite{RenSongSun2017Spine}, we say that $\{\xi, \mathbf n;\mathbf Q_{x}^{(f,T)}\}$ is a spine representation of $\mathbb N_x^{\langle W_T, f\rangle}$ if
\begin{itemize}
\item
  the spine process $\{(\xi_t)_{0\leq t\leq T}; \mathbf Q^{(f,T)}_x\}$ is a copy of $\{(\xi_t)_{0\leq t\leq T}; \Pi^{(f,T)}_{x}\}$, where
  \begin{equation}
    d\Pi_x^{(f,T)} 
    := \frac{f(\xi_T)e^{\int_0^T \rho_1(\xi_s)ds}}{P^{\rho_1}_T f(x)} d \Pi_x;
  \end{equation}
\item
  given $\{(\xi_t)_{0\leq t\leq T}; \mathbf Q^{(f,T)}_x\}$, the immigration measure
\[ 
  \{\mathbf n(\xi,ds,dw); \mathbf Q^{(f,T)}_x[\cdot |(\xi_t)_{0\leq t\leq T}]\}
\]
is a Poisson random measure on $[0,T] \times \mathbb W$ with intensity
\begin{equation}
  \label{eq: conditional intensity}
  \mathbf m(\xi,ds,dw)
  := 2 \rho_2(\xi_s) ds \cdot \mathbb N_{\xi_s}(dw) + ds \cdot \int_{y\in (0,\infty)} y \mathbf P_{y\delta_{\xi_s}}(X\in dw) \pi(\xi_s,dy);
\end{equation}
\item
  $\{(Y_t)_{0\leq t\leq T}; \mathbf Q^{(f,T)}_x\}$ is an $\mathcal M(E)$-valued process defined by
  \begin{equation}
    Y_t
    := \int_{(0,t] \times \mathbb W} w_{t-s} \mathbf n(\xi,ds,dw),
    \quad 0 \leq t\leq T.
  \end{equation}
\end{itemize}
According to the spine decomposition theorem in \cite{RenSongSun2017Spine}, we have that
\begin{equation}
  \label{eq: Spine decomposition 1}
  \{(X_s)_{s \geq 0};\mathbf P_{\delta_x}^{\langle X_T, f\rangle}\}
  \overset{f.d.d.}{=} \{(X_s + W_s)_{s \geq 0};\mathbf P_{\delta_x} \otimes \mathbb N_x^{\langle W_T, f\rangle} \},
\end{equation}
\begin{equation}
  \label{eq: Spine decomposition 2}
  \{(W_s)_{0\leq s\leq T};\mathbb N_x^{\langle W_T, f\rangle}\}
  \overset{f.d.d.}{=} \{(Y_s)_{s \geq 0};\mathbf Q_x^{(f,T)}\}.
\end{equation}

\begin{proof}[Proof of Proposition \ref{prop: complex FKPP-equation}]
  Assume that $f\in \mathcal B_b(E)$.
  Fix $t>0, r\in [0,t), x\in E$ and a strictly positive $g\in \mathcal B_b(E)$.
  Denote by $\{\xi, \mathbf n; \mathbf Q_x^{(g,t)}\}$ the spine representation of $\mathbb N_x^{\langle W_t, g\rangle}$.
  Conditioned on $\{\xi; \mathbf Q_x^{(g,t)}\}$, denote by $\mathbf m(\xi, ds,dw)$ the conditional intensity of $\mathbf n$ in \eqref{eq: conditional intensity}.
  Denote by $\Pi_{r,x}$ the probability of Hunt process $\{\xi; \Pi\}$ initiated at time $r$ and position $x$.
  From Lemma \ref{lem: estimate of exponential remaining}, we have $\mathbf Q^{(g,t)}_{x}$-almost surely
\begin{align}
  & \int_{[0,t]\times \mathbb W}|e^{i \langle w_{t-s}, f\rangle} - 1| \mathbf m(\xi, ds,dw)
    \leq \int_{[0,t]\times \mathbb W}\big(| \langle w_{t-s}, f\rangle| \wedge 2\big) \mathbf m(\xi, ds,dw) \\ 
  & \leq \int_0^t \Big(2\rho_2(\xi_s)\mathbb N_{\xi_s}\big( \langle W_{t-s}, |f|\rangle\big)  + \int_{(0,1]} y \mathbf P_{y \delta_{\xi_s}}[\langle X_{t-s}, |f|\rangle] \pi(\xi_s,dy) \\
  & \qquad\qquad + 2\int_{(1,\infty)}y\pi(\xi_s,dy)\Big) ds
  \\ & \leq \int_0^t (P_{t-s}^{\rho_1} |f|)(\xi_s)\Big(2\rho_2(\xi_s)  + \int_{(0,1]} y^2 \pi(\xi_s,dy)\Big) ds + 2t \sup_{x\in E}\int_{(1,\infty)}y\pi(x,dy)
  \\ & \leq \Big(2\|\rho_2\|_\infty +\sup_{x\in E}\int_{(0,1]} y^2 \pi(x,dy)\Big) t e^{t\|\rho_1\|_\infty}\|f\|_\infty + 2t \sup_{x\in E}\int_{(1,\infty)}y\pi(x,dy)
       < \infty.
\end{align}
Using this, Fubini's theorem, \eqref{eq: N and characteristic exponent} and \eqref{eq: -v has positive real part} we have $\mathbf Q^{(g,t)}_{x}$-almost surely,
\begin{align}
  & \int_{[0,t]\times \mathbb N}(e^{i \langle w_{t-s}, f\rangle} - 1) \mathbf m(\xi, ds,dw)
  \\ & =\int_0^t \Big(2\rho_2(\xi_s)\mathbb N_{\xi_s}(e^{i \langle W_{t-s}, f\rangle} - 1)  + \int_{(0,\infty)} y \mathbf P_{y \delta_{\xi_s}}[e^{i \langle X_{t-s}, f\rangle} - 1] \pi(\xi_s,dy)\Big) ds
  \\ & =\int_0^t \Big( 2\rho_2(\xi_s) U_{t-s} f(\xi_s) + \int_{(0,\infty)} y (e^{y U_{t-s}f(\xi_s)} - 1) \pi(\xi_s,dy) \Big) ds
  \\ & = -\int_0^t \psi'_0 \big(\xi_s, -U_{t-s}f\big)ds.
\end{align}
Therefore, according to \eqref{eq: Spine decomposition 2}, Campbell's formula and above, we have that
\begin{align}
  \label{eq: N to Pi}
    & \mathbb N_x^{\langle W_{t}, g\rangle}[e^{i \langle W_t, f\rangle}]
      = \mathbf Q_x^{(g,t)} \Big[\exp\Big\{\int_{[0,t]\times \mathbb N}(e^{i \langle w_{t-s}, f\rangle} - 1) \mathbf m(\xi, ds,dw)\Big\}\Big]
    \\ & = \Pi_x^{(g,t)} [e^{-\int_0^t \psi'_0(\xi_s, -U_{t-s}f)ds}]
    = \frac{1}{P_t^{\rho_1} g (x)} \Pi_x[ g(\xi_t) e^{-\int_0^t \psi'(\xi_s, -U_{t-s}f)ds} ].
\end{align}
Let $\epsilon >0$.
Define $f^+ = (f \vee 0) + \epsilon$ and $f^- = (-f) \vee 0 + \epsilon$, then $f^\pm$ are strictly positive and $f = f^+ - f^-$.
According to \eqref{eq: Spine decomposition 1}, we have that
\begin{equation}
  \frac{\mathbf P_{\delta_x}[\langle X_t,f^{\pm}\rangle e^{i \langle X_t,f\rangle}]}{\mathbf P_{\delta_x}[\langle X_t,f^{\pm}\rangle ]}
  = \mathbf P_{\delta_x}[e^{i \langle X_t,f\rangle}] \mathbb N_x^{\langle W_t,f^{\pm}\rangle}[e^{i \langle X_t,f\rangle}].
\end{equation}
Using \eqref{eq: N to Pi} and the above, we have
\begin{align}
  \frac{\mathbf P_{\delta_x}[\langle X_t, f\rangle e^{i \langle X_t, f\rangle}] }{\mathbf P_{\delta_x}[e^{i \langle X_t, f\rangle}]}
  & = \mathbf P_{\delta_x}[\langle X_t, f^+\rangle] \mathbb N_x^{\langle W_t, f^+\rangle} [e^{i \langle X_t, f\rangle}] - \mathbf P_{\delta_x}[\langle X_t, f^-\rangle]\mathbb N_x^{\langle W_t, f^-\rangle}[e^{i \langle X_t, f\rangle}]
  \\ & = \Pi_x[ f(\xi_t) e^{- \int_0^t \psi'(\xi_s, -U_{t-s}f) ds}  ].
\end{align}
Therefore, we have
\begin{equation}
  \frac{\partial}{\partial \theta} {U_t(\theta f)(x)}
  = \frac{\mathbf P_{\delta_x}[i\langle X_t, f\rangle e^{i \langle X_t, f\rangle}] }{\mathbf P_{\delta_x}[e^{i \langle X_t, f\rangle}]}
  = \Pi_x[ if(\xi_t) e^{ - \int_0^t \psi'(\xi_s, -U_{t-s}(\theta f)) ds} ].
\end{equation}
Since $\{(\xi_{r+t})_{t \geq 0}; \Pi_{r,x}\} \overset{d}{=} \{(\xi_{t})_{t\geq 0}; \Pi_{x}\} $, we have
\begin{align}
  & \frac{\partial}{\partial \theta} U_{t-r}(\theta f)( x)
    = \Pi_x[ i f(\xi_{t-r}) e^{-\int_0^{t-r} \psi'(\xi_s, -U_{t-r-s}(\theta f)) ds} ] \\ 
  & = \Pi_{r,x}[i f(\xi_t)e^{-\int_0^{t-r} \psi'(\xi_{r+s}, -U_{t-r-s}(\theta f)) ds} ]
    = \Pi_{r,x}[if(\xi_t)e^{-\int_r^t \psi'(\xi_{s}, -U_{t-s}(\theta f)) ds} ].
\end{align}

 From \eqref{eq: upper bound of psi'(v)}, we know that for each $\theta\in \mathbb R$, $(t,x) \mapsto |\psi'(x,-U_tf(x))|$ is locally bounded (i.e. bounded on $[0,T]\times E$ for each $T \geq 0$).
Therefore, we can apply Lemma \ref{eq: complex FK} and get that
\[
  \frac{\partial}{\partial \theta} U_{t-r}(\theta f)(x) + \Pi_{r,x} \Big[\int_r^t \psi'\big(\xi_s,- U_{t-s}(\theta f)\big)\frac{\partial}{\partial \theta} U_{t-s}(\theta f)(\xi_s)~ds\Big]
  = \Pi_{r,x} [i f(\xi_t)]
\]
and
\begin{align}
  & \frac{\partial}{\partial \theta} U_{t-r}(\theta f)(x) + \Pi_{r,x} \Big[\int_r^t e^{\int_r^s \rho_1(\xi_u)du}\psi_0'\big(\xi_s,- U_{t-s}(\theta f)\big)\frac{\partial}{\partial \theta} U_{t-s}(\theta f)(\xi_s)~ds\Big]\\
  & = \Pi_{r,x} [i e^{\int_r^t \rho_1(\xi_s)ds}f(\xi_t)].
\end{align}
Integrating the two displays above with respect to $\theta$  on [0,1], using 
Fubini's theorem, \eqref{eq: upper bounded for derivative of v(theta)}, \eqref{eq: path integration representation of psi(v)} and \eqref{eq: upper bound of psi'(v)}, we get
\begin{equation}
  U_{t-r}f(x) - \Pi_{r,x} \Big[\int_r^t \psi\big(\xi_s,-U_{t-s}f\big) ~ds\Big]
  = i \theta \Pi_{r,x} [f(\xi_t)]
\end{equation}
and
\begin{equation}
  U_{t-r}f(x) - \Pi_{r,x} \Big[\int_r^t e^{\int_r^s \rho_1(\xi_u)du} \psi_0\big(\xi_s,- U_{t-s}f\big) ~ds\Big]
  = i \Pi_{r,x} [e^{\int_r^t\rho_1(\xi_u)du}f(\xi_t)].
\end{equation}
Taking $r = 0$, we get that \eqref{eq: complex FKPP-equation} and \eqref{eq: complex FKPP-equation with FK-transform} are true if $f\in \mathcal B_b(E)$.

The rest of the proof is to evaluate \eqref{eq: complex FKPP-equation} and \eqref{eq: complex FKPP-equation with FK-transform} for all $f\in L_2(\xi)$. We only do this for \eqref{eq: complex FKPP-equation} since the argument for \eqref{eq: complex FKPP-equation with FK-transform} is similar.
Let $n \in \mathbb N$.
Writing $f_n := (f^+ \wedge n) - (f^- \wedge n)$, then $f_n \xrightarrow[n\to \infty]{} f$ pointwise.
From what we have proved, we have
\begin{equation}
  \label{eq: complex FKPP-equation for fn}
  U_tf_n(x) - \Pi_{x} \Big[\int_0^t \psi\big(\xi_s, - U_{t-s}f_n\big) ~ds\Big]
  = i \Pi_{x} [f_n(\xi_t)].
\end{equation}
Note that 
(i) $\Pi_{x}[f_n(\xi_t)] \xrightarrow[n\to \infty]{} \Pi_{x}[f(\xi_t)]$;
(ii) by \eqref{eq: N and characteristic exponent}, the dominated convergence theorem and the fact that
\[
  |e^{i W_t(f_n)} - 1| \leq \langle W_t, |f|\rangle;
  \quad \mathbb N_x[\langle W_t, |f|\rangle] = (P_t^{\rho_1} |f|)(x) < \infty,
\]
we have $U_tf_n(x) \xrightarrow[n\to \infty]{} U_tf(x)$, and (iii) by the dominated convergence theorem, \eqref{eq: domination of psi(v)} and the fact (see \eqref{eq: upper bound of psi(v)}) that
\[
  big|\psi(\xi_s,- U_{t-s}f_n)\big|
  \leq C_1 P_{t-s}^{\rho_1}|f|(\xi_s)+C_2 P_{t-s}^{\rho_1}|f|(\xi_s)^2,
\]
we get that $\Pi_{x} [\int_0^t \psi(\xi_s,- U_{t-s}f_n)ds] \xrightarrow[n\to \infty]{} \Pi_{x} [\int_0^t \psi(\xi_s,- U_{t-s}f)ds]$.
Using these, letting $n \to \infty$ in \eqref{eq: complex FKPP-equation for fn}, we get the desired result.
\end{proof}

\subsection*{Acknowledgment}
We thank Zenghu Li and Rui Zhang for helpful conversations.
We also thank the referee for very helpful comments.

\begin{thebibliography}{10}

\bibitem{AdamczakMilos2015CLT}
  R. Adamczak and P. Mi{\l}o\'{s}, \emph{C{LT} for {O}rnstein-{U}hlenbeck branching particle system},
  Electron. J. Probab. \textbf{20} (2015), no. 42, 35 pp.

\bibitem{Asmussen76Convergence}
  S. Asmussen, \emph{Convergence rates for branching processes}, 
  Ann. Probab.  \textbf{4} (1976), no. 1, 139--146.

\bibitem{AsmussenHering1983Branching}
  S. Asmussen and H. Hering, \emph{Branching processes}, 
  Progress in Probability and Statistics, 3. Birkh\"{a}user Boston, Inc., Boston, MA, 1983.

\bibitem{Athreya1969Limit}
  K. B. Athreya,  
  \emph{Limit theorems for multitype continuous time {M}arkov branching processes. {I}. {T}he case of an eigenvector linear functional}, 
  Z. Wahrscheinlichkeitstheorie und Verw. Gebiete \textbf{12} (1969), 320--332.

\bibitem{Athreya1969LimitB}
  K. B. Athreya, 
  \emph{Limit theorems for multitype continuous time {M}arkov branching processes. {II}. {T}he case of an arbitrary linear functional}, 
  Z. Wahrscheinlichkeitstheorie und Verw. Gebiete \textbf{13} (1969), 204--214.

\bibitem{Athreya1971Some}
  K. B. Athreya, 
  \emph{Some refinements in the theory of supercritical multitype {M}arkov branching processes}, 
  Z. Wahrscheinlichkeitstheorie und Verw. Gebiete \textbf{20} (1971), 47--57.

\bibitem{ChenRenWang2008An-almost}
  Z.-Q. Chen, Y.-X. Ren, and H. Wang, 
  \emph{An almost sure scaling limit theorem for {D}awson-{W}atanabe superprocesses}, 
  J. Funct. Anal. \textbf{254} (2008), no. 7, 1988--2019.

 \bibitem{ChenRenSongZhang2015Strong-law} 
   Z.-Q. Chen, Y.-X. Ren, R. Song, and R. Zhang,
   \emph{Strong law of large numbers for supercritical superprocesses under second moment condition}, 
   Front. Math. China \textbf{10} (2015), no. 4, 807--838.

 \bibitem{ChenRenYang2019Skeleton} 
   Z.-Q. Chen, Y.-X. Ren, and T. Yang,
   \emph{Skeleton decomposition and law of large numbers for supercritical superprocesses}, 
   Acta Appl. Math. \textbf{159} (2019), 225--285.

 \bibitem{Durrett2010Probability}
   R. Durrett, 
   \emph{Probability: theory and examples}, 
   Fourth edition. Cambridge Series in Statistical and Probabilistic Mathematics, 31. Cambridge University Press, Cambridge, 2010.

\bibitem{Dynkin1993Superprocesses}
  E. B. Dynkin, 
  \emph{Superprocesses and partial differential equations}, 
  Ann. Probab. \textbf{21} (1993), no. 3, 1185--1262.

\bibitem{EckhoffKyprianouWinkel2015Spines}
  M. Eckhoff, A. E. Kyprianou, and M. Winkel, 
  \emph{Spines, skeletons and the strong law of large numbers for superdiffusions}, 
  Ann. Probab. \textbf{43} (2015), no. 5, 2545--2610.

\bibitem{Englander2009Law}
  J. Engl\"{a}nder, 
  \emph{Law of large numbers for superdiffusions: the non-ergodic case}, 
  Ann. Inst. Henri Poincar\'{e} Probab. Stat. \textbf{45} (2009), no. 1, 1--6.

\bibitem{EnglanderWinter2006Law}
  J. Engl\"{a}nder and A. Winter, 
  \emph{Law of large numbers for a class of superdiffusions}, 
  Ann. Inst. H. Poincar\'{e} Probab. Statist. \textbf{42} (2006), no. 2, 171--185.

\bibitem{EnglanderTuraev2002A-scaling}
  J. Engl\"{a}nder and  D. Turaev, 
  \emph{A scaling limit theorem for a class of superdiffusions}, 
  Ann. Probab. \textbf{30} (2002), no. 2, 683--722.

\bibitem{Heyde1970A-rate}
  C. C. Heyde, 
  \emph{A rate of convergence result for the super-critical {G}alton-{W}atson process}, 
  J. Appl. Probability \textbf{7} (1970), 451--454.

\bibitem{Heyde1971Some}
  C. C. Heyde, 
  % \emph{some central limit analogues for supercritical {g}alton-{w}atson processes}, 
  \emph{some central limit analogues for supercritical {G}alton-{W}atson processes}, 
  J. Appl. Probability \textbf{8} (1971), 52--59.

\bibitem{HeydeBrown1871An-invariance}
  C. C. Heyde and B. M. Brown, 
  \emph{An invariance principle and some convergence rate results for branching processes}, 
  Z. Wahrscheinlichkeitstheorie und Verw. Gebiete, \textbf{20} (1971), 271--278.

\bibitem{HeydeLeslie1971Improved}
  C. C. Heyde and J. R. Leslie, 
  \emph{Improved classical limit analogues for {G}alton-{W}atson processes with or without immigration}, 
  Bull. Austral. Math. Soc. \textbf{5} (1971), 145--155.

\bibitem{IksanovKoleskoMeiners2018Stable-like}
  A. Iksanov, K. Kolesko, and M. Meiners, 
  \emph{Stable-like fluctuations of {B}iggins' martingales}, 
  Stochastic Process. Appl. (2018).

\bibitem{Janson2004Functional}
  S. Janson, 
  \emph{Functional limit theorems for multitype branching processes and generalized {P}\'{o}lya urns}, 
  Stochastic Process. Appl. \textbf{110} (2004), no. 2, 177--245.

\bibitem{KestenStigum1966Additional}
  H. Kesten and B. P. Stigum, 
  \emph{Additional limit theorems for indecomposable multidimensional {G}alton-{W}atson processes},
  Ann. Math. Statist. \textbf{37} (1966), 1463--1481.

\bibitem{KestenStigum1966A-limit}
  H. Kesten and B. P. Stigum, 
  \emph{A limit theorem for multidimensional {G}alton-{W}atson processes}, 
  Ann. Math. Statist. \textbf{37} (1966), 1211--1223.

\bibitem{KouritzinRen2014A-strong}
  M. A. Kouritzin, and Y.-X. Ren,  
  \emph{A strong law of large numbers for super-stable processes},  
  Stochastic Process. Appl. \textbf{121} (2014), no. 1, 505--521.

\bibitem{Kyprianou2014Fluctuations}
  A. E. Kyprianou, 
  \emph{Fluctuations of {L}\'{e}vy processes with applications},
  % 2 ed., Universitext, Springer, Heidelberg, 2014.
  Introductory lectures. Second edition. Universitext. Springer, Heidelberg, 2014.

\bibitem{Li2011Measure-valued}
  Z. Li, 
  \emph{Measure-valued branching {M}arkov processes}, 
  Probability and its Applications (New York). Springer, Heidelberg, 2011.

\bibitem{LiuRenSong2009Llog}
  % R.~Liu, Y.-X. Ren, and R.~Song, 
  R.-L. Liu, Y.-X. Ren, and R. Song, 
  \emph{{$L\log L$} criterion for a class of superdiffusions}, 
  J. Appl. Probab. \textbf{46} (2009), no. 2, 479--496.

\bibitem{LiuRenSong2013Strong}
  % R.~Liu, Y.-X. Ren, and R.~Song, 
  R.-L. Liu, Y.-X. Ren, and R. Song, 
  \emph{Strong law of large numbers for a class of superdiffusions}, 
  Acta Appl. Math. \textbf{123} (2013), 73--97.

\bibitem{MarksMilos2018CLT}
  R. Marks and P. Mi{\l}o{\'s}, 
  \emph{C{LT} for supercritical branching processes with heavy-tailed branching law}, 
  arXiv:1803.05491.

\bibitem{MetafunePallaraPriola2002Spectrum}
  G. Metafune, D. Pallara, and E. Priola, 
  \emph{Spectrum of {O}rnstein-{U}hlenbeck operators in {$L^p$} spaces with respect to invariant  measures}, 
  J. Funct. Anal. \textbf{196} (2002), no. 1, 40--60.

\bibitem{Milos2012Spatial}
  P. Mi{\l}o{\'s},
  \emph{Spatial central limit theorem for supercritical superprocesses},
  J. Theoret. Probab. \textbf{31} (2018), no. 1, 1--40.

\bibitem{RenSongSun2017Spine}
  Y.-X. Ren, R. Song, and Z. Sun, 
  \emph{Spine decompositions and limit theorems for a class of critical superprocesses},
  Acta Appl. Math. (2019).

\bibitem{RenSongSun2018Limit}
  Y.-X. Ren, R. Song, and Z. Sun, 
  \emph{Limit theorems for a class of critical superprocesses with stable branching},
  arXiv:1807.02837.

\bibitem{RenSongZhang2014Central}
  Y.-X. Ren, R. Song, and R. Zhang, 
  \emph{Central limit theorems for super {O}rnstein-{U}hlenbeck processes}, 
  Acta Appl. Math. \textbf{130} (2014), 9--49.

\bibitem{RenSongZhang2014CentralB}
  Y.-X. Ren, R. Song, and R. Zhang, 
  \emph{Central limit theorems for supercritical branching {M}arkov processes}, 
  J. Funct. Anal. \textbf{266} (2014), no. 3, 1716--1756.

\bibitem{RenSongZhang2015Central}
  Y.-X. Ren, R. Song, and R. Zhang, 
  \emph{Central limit theorems for supercritical superprocesses},
  Stochastic Process. Appl. \textbf{125} (2015), no. 2, 428--457.

\bibitem{RenSongZhang2017Central}
  Y.-X. Ren, R. Song, and R. Zhang, 
  \emph{Central limit theorems for supercritical branching nonsymmetric {M}arkov processes}, 
  Ann. Probab. \textbf{45} (2017), no. 1, 564--623.

\bibitem{RenSongZhang2017Functional}
  Y.-X. Ren, R. Song, and R. Zhang, 
  \emph{Functional central limit theorems for supercritical superprocesses}, 
  Acta Appl. Math. \textbf{147} (2017), 137--175.

\bibitem{Sato2013Levy}
  K. Sato, 
  \emph{L{\'e}vy processes and infinitely divisible distributions},
  % Cambridge Studies in Advanced Mathematics, vol.~68, Cambridge University Press, Cambridge, 2013.
  Translated from the 1990 Japanese original. Revised by the author. Cambridge Studies in Advanced Mathematics, 68. Cambridge University Press, Cambridge, 1999.

\bibitem{SchillingSongVondravcek2010Bernstein}
  R. L. Schilling, R. Song, and Z. Vondra\v{c}ek,
  \emph{Bernstein functions.}
  Theory and applications. Second edition. De Gruyter Studies in Mathematics, 37. Walter de Gruyter \& Co., Berlin, 2012.

\bibitem{SteinShakarchi2003Complex}
  E. M. Stein and R. Shakarchi, \emph{Complex analysis}, 
  Princeton Lectures in Analysis, 2. Princeton University Press, Princeton, NJ, 2003.

\bibitem{Wang2010An-almost}
  L. Wang, \emph{An almost sure limit theorem for super-{B}rownian motion}, 
  J. Theoret. Probab. \textbf{23} (2010), no. 2, 401--416.

\end{thebibliography}
\end{document}
