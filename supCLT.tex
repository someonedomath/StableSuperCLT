%%%----Versions-----------------------------
% supCLT5.tex ......... by Zhenyao
% supCLT4.tex ......... by Jianjie
% supCLT3.tex 2018/8/25 by Zhenyao
% supCLT2.tex 2018/8/13 by Jianjie
% supCLT1.tex 2018/7/24 by Zhenyao
% supCLT.tex 2018/6/29 by Jianjie
%---The preamble----------------------
\documentclass[12pt,oneside,english]{amsart}
\usepackage[T1]{fontenc}
\usepackage[utf8]{inputenc}
\usepackage{geometry}
\geometry{tmargin=1in,bmargin=1in,lmargin=1in,rmargin=1in}
\usepackage{mathtools}
\mathtoolsset{showonlyrefs}
\usepackage{amsthm}
\usepackage{stackrel}
\usepackage[backref=section]{hyperref}
\def\MR#1{\href{http://www.ams.org/mathscinet-getitem?mr=#1}{MR-#1}}
\def\ARXIV#1{\href{https://arxiv.org/abs/#1}{arXiv:#1}}
\usepackage{mathrsfs}
\usepackage{color}
\usepackage{comment}
\theoremstyle{plain}
\newtheorem{thm}{Theorem}[section]
\newtheorem{lem}[thm]{Lemma}
\newtheorem{prop}[thm]{Proposition}
\newtheorem{cor}[thm]{Corollary}
\newtheorem{conj}[thm]{Conjecture}
\theoremstyle{definition}
\newtheorem{defi}[thm]{Definition}
\newtheorem{rem}[thm]{Remark}
\newtheorem{exa}[thm]{Example}
\newtheorem{asp}{Assumption}
\numberwithin{equation}{section}
\allowdisplaybreaks
\newcommand{\added}[1]{{\color{blue}#1}}\newcommand{\deleted}[1]{{\color{red}#1}}
%\newcommand{\added}[1]{#1}\newcommand{\deleted}[1]{}

%---Top matter------------------------------
\begin{document}
\title
    [CLT for Super-OU processes]
    {CLT for Super-OU Processes with stable Branching}
\author
    [Y.-X. Ren, R. Song, Z. Sun and J. Zhao]
    {Yan-Xia Ren, Renming Song, Zhenyao Sun and Jianjie Zhao}
\address
    {Yan-Xia Ren\\
    School of Mathematical Sciences\\
    Peking University\\
    Beijing, P. R. China, 100871}
\email{yxren@math.pku.edu.cn}
\thanks{The research of Yan-Xia Ren is supported in part by NSFC (Grant Nos. 11671017  and 11731009).}
\address
    {Zhenyao Sun\\
    School of Mathematical Sciences\\
    Peking University\\
    Beijing, P. R. China, 100871}
\email{zhenyao.sun@pku.edu.cn}
\address
    {Jianjie Zhao\\
    School of Mathematical Sciences\\
    Peking University\\
    Beijing, P. R. China, 100871}
\email{zhaojianjie@pku.edu.cn}
%\begin{abstract}
%\end{abstract}
%\subjclass[2010]{60J80, 60F05}
%\keywords{}
%\date{}
\maketitle
%---Contents-----------------------
\section{Introduction}
\subsection{Model}
    Let $d \in \mathbb N:= \{1,2,\dots\}$ and
    %\deleted{For each $x\in \mathbb{R}^d$, write $|x|$ for its Euclidean norm.}
    $\xi=\{(\xi_t)_{t\geq 0}; (\Pi_x)_{x\in \mathbb R^d}\}$ be an $\mathbb R^d$-valued Ornstein-Uhlenbeck prcess (OU process) with infinitesimal generator $L$ such that
\begin{equation}
\label{eq: OU generator}
    Lf(x)
        = \frac{1}{2}\sigma^2\Delta f(x)-b x \cdot \nabla f(x),
        \quad  x\in \mathbb R^d,
        f \in C^2(\mathbb{R}^d),
\end{equation}
    where $\sigma>0$ and $b>0$.
    Denote by $(T_t)_{t\geq 0}$ the transition semigroup of $\xi$.
    Let $\psi$ be a function on \added{$\mathbb R^d \times \mathbb R_+$}\deleted{$\mathbb R_+:= [0,\infty)$} with
\deleted{
\begin{equation}\label{mechanism}
    \psi(z)
    := - \alpha z + z^{1+\beta},
    \quad z \in \mathbb R_+,
\end{equation}
}
\added{
\begin{equation}\label{mechanism}
    \psi(x, z)
    := - \alpha z + z^{1+\beta},
    \quad x\in \mathbb R^d, z \in \mathbb R_+,
\end{equation}
}
        where $\alpha > 0$ and $\beta \in (0,1) $.
    Denote by $\mathcal{M}(\mathbb{R}^d)$ the space of all finite measures on $\mathbb{R}^d$.
    For each measurable function $f$ and measure $\mu$ write $\langle f,\mu\rangle = \int f(x)\mu(dx)$.
    Put $\|\mu\|=\langle 1,\mu\rangle$ \added{if it is well defined}.
    We say a $\mathcal{M}(\mathbb{R}^d)$-valued Markov process $X = \{(X_t)_{t\geq 0}; (\mathbb{P}_{\mu})_{\mu \in \mathcal M(\mathbb R^d)}\}$ is a super Ornstein-Uhlenbeck process (super-OU process) with branching mechanism $\psi$, if for each non-negative bounded measurable function $f$ on $\mathbb{R}^d$, we have
\deleted{
\begin{equation} \label{super}
    \mathbb{P}_{\mu}[e^{-\langle f,X_t \rangle}]
        = e^{-\langle u_f(t,\cdot), \mu \rangle},
        \quad t\geq 0, \mu \in \mathcal M(\mathbb R^d),
\end{equation}
}
\added{
\begin{equation} \label{super}
    \mathbb{P}_{\mu}[e^{-\langle f,X_t \rangle}]
    = e^{-\langle V_tf, \mu \rangle},
    \quad t\geq 0, \mu \in \mathcal M(\mathbb R^d),
\end{equation}
}
    where \deleted{$(t,x) \mapsto u_f(t,x)$} \added{$(t,x) \mapsto V_tf(x)$} is the unique locally bounded positive solution to the equation
\deleted{
\begin{equation}\label{eq1}
	u_f(t,x) + \Pi_x \Big[ \int_0^t\psi\big(u_f(t-s, \xi_s)\big)~ds\Big]
	= \Pi_x [f(\xi_t)],
	\quad x\in \mathbb R^d, t\geq 0.
\end{equation}
}
\added{
\begin{equation}\label{eq1}
	V_tf(x) + \Pi_x \Big[ \int_0^t\psi\big(\xi_s,V_{t-s}f(\xi_s)\big)~ds\Big]
	= \Pi_x [f(\xi_t)],
    \quad x\in \mathbb R^d, t\geq 0.
\end{equation}	
}

\deleted{
	The existence of such superprocesses is well known, see \cite{EB} for instance.
}
\added{
	The existence of superprocesses are well known, see \cite{EB} for instance.
}
    Note that $(\|X_t\|)_{t\geq 0}$ is a continuous-state branching process with branching mechanism $\psi$.
    According to \cite[Theorem 12.5 \& Theorem 12.7]{Kyprianou2014Fluctuations}, we have
\begin{equation}
    \mathbb{P}_{\mu} (\exists t\geq 0,~\text{s.t.}~\|X_t\|=0)
    = e^{-\alpha^* \|\mu\|},
\end{equation}
    where $\alpha^* := \sup\{\lambda \geq 0: \psi(\lambda) = 0\}$.

    Define the \emph{Fenyman-Kac semigroup} by
\begin{equation}\label{meansemigroup}
    T^{\alpha}_t f(x)
    :=
    e^{\alpha t} T_t f(x) =
    \Pi_x [e^{\alpha t}f(\xi_t)],
    \quad x\in \mathbb{R}^d,t\geq 0, f\in \mathscr B^+(\mathbb R^d).
\end{equation}
    It is konwn that, see \cite[Proposition 2.27]{Li2011Measure-valued} for example, $(T^\alpha_t)_{t\geq 0}$ is the \emph{mean semigroup} of $X$, in the sense that
\begin{equation}\label{eq:meanformula}
    \mathbb{P}_{\mu}[\langle f, X_t \rangle]
    = \langle T_t^\alpha f, \mu \rangle,
    \quad t\geq 0, f\in \mathscr B^+(\mathbb R^d)\added{, \mu \in \mathcal M(\mathbb R^d)}.
\end{equation}

\subsection{OU semigroup}
    In this subsection, we recall some results on the spectrum property of the OU operator $L$ from \cite{GD}.
    Let $\xi$ be an OU process with generator $L$ given by \eqref{eq: OU generator}.
    It is known that $\xi$ has an invariant measure
\begin{equation}
\label{invariantdensity}
    \varphi(x)dx
    =\Big (\frac{b}{\pi \sigma^2}\Big )^{d/2}\exp \Big(-\frac{b}{\sigma^2}|x|^2 \Big)dx,
    \quad x\in \mathbb{R}^d.
\end{equation}
    Let $L^2(\varphi):= \left\{ h  \in \mathscr B(\mathbb R^d): \int_{\mathbb{R}^d} |h(x)|^2 \varphi(x) dx < \infty \right\}$, then $L^2(\varphi)$ is an Hilbert space with inner product
\begin{equation}
    \langle f_1, f_2 \rangle_{\varphi}
    := \int_{\mathbb{R}^d}f_1(x)f_2(x)\varphi(x) dx, \quad f_1,f_2 \in L^2(\varphi).
\end{equation}
     Let $\mathbb N = \{1,2,\dots\}$ and $\mathbb N_0 := \mathbb N\cup\{0\}$.
    For each $p = (p_k)_{k = 1}^d \in \mathbb{N}_0^{d}$, let $|p| \added{ :}=\sum_{k=1}^d p_k$, $p!:= \prod_{k= 1}^d p_k !$ and $\partial x^p:= \prod_{k = 1}^d\partial x_k^{p_k}$.
    The Hermite polynomials are defined by
\begin{equation}
    H_p(x)
    :=(-1)^{|p|}\exp(|x|^2) \frac{\partial ^{|p|}}{\partial x^p} \exp(-|x|^2) ,
    \quad x\in \mathbb R^d, p \in \mathbb{N}_0^{d}.
\end{equation}
    It is known that $(T_t)$ is a strong continuous semigroup on $L^2(\varphi)$ and its generator $L$ has eigenvalues $\sigma(L)= \{-bk: k \in \mathbb N_0\}$.
    For each $k \in \mathbb N_0$, denote $A_k$ the eigenspace corresponding to eigenvalue $-bk$, then
\begin{equation}
    A_k
    = \operatorname{Span} \{\phi_p : p\in \mathbb N_0^d, |p|=k\},
\end{equation}
where
\begin{equation}\label{eigenfunction}
    \phi_p(x)
    := \frac{1}{\sqrt{ p! 2^{|p|} }} H_p \Big(\frac{ \sqrt{b} }{\sigma}x \Big),
    \quad x\in \mathbb{R}^d, p\in \mathbb N_0^d.
\end{equation}
    It is known that
\begin{equation}\label{semigroupformula}
    T_t\phi_p(x)
    =e^{-b|p|t}\phi_p(x),
    \quad t\geq 0, x\in \mathbb{R}^d, p\in \mathbb N_0^d.
\end{equation}
    Moreover, $\{\phi_p: p \in \mathbb N_0^d\}$ forms a complete orthogonal basis for $L^2(\varphi)$.
    That is, for each $f\in L^2(\varphi)$,
\begin{equation}\label{semicomp1}
    f
    =\sum_{k=0}^{\infty}\sum_{|p|=k}\langle f, \phi_p \rangle_{\varphi} \phi_p,
    \quad \text{in~} L^2(\varphi).
\end{equation}
    Denote by
\begin{equation}\label{order}
    \kappa_f
    :=\inf \left \{k\geq 0: \exists ~ p\in \mathbb N_0^d ,{\rm ~s.t.~}|p|=k {\rm ~and~}  \langle f, \phi_p \rangle_{\varphi}\neq 0\right \}
\end{equation}
    the order of function $f$ in $L^2(\varphi)$.
    Note that $ \kappa_f\geq 0$.
    In particular, the order of any constant function is zero.
{
    Denote by $\mathcal P$ the class of functions of polynomial growth on $\mathbb R^d$ with
\begin{equation}
    \mathcal{P}
    :=\big \{f\in \mathscr B(\mathbb{R}^d):\exists~ C>0, n \in \mathbb N_0, {\rm ~s.t.~} \forall x\in \mathbb{R}^d,~ |f(x)|\leq C(1+|x|)^n\big\}.
\end{equation}
    It is clear that $\mathcal{P} \subset L^2(\varphi)$.
    According to \cite[Fact 1.2]{MM}, for any $f \in \mathcal{P}$, there exists $R \in \mathcal{P}$ such that,
\begin{equation}
\label{eq:semigroupineq}
    |T_tf(x)|
    \leq e^{-\kappa_f bt} R(x),
    \quad t\geq 0, x\in \mathbb R^d.
\end{equation}
	For each $f\in \mathcal {P}$, define an operator $Q$ by $Qf(x):=\sup_{t\geq 0}e^{\kappa(f)bt} |T_tf(x)|$.
	Then $Q$ is an operator form $\mathcal P$ to $\mathcal P \cap \mathscr B
	(\mathbb R^d,\mathbb R_+)$.
}
\deleted{
\subsection{}
    Let $(\Omega, \mathscr F)$ be a measurable space.
    Let $\mathcal F$ be a linear subspace of real valued measurable functions on $\Omega$.
\begin{defi}
\label{def: alpha-positive operator}
    We say an operator $A: \mathcal F \to \mathcal F$ is $\alpha$-scalable for some $\alpha \in \mathbb R$ if for each $\lambda \in \mathbb R$ and $h \in \mathcal F$ we have $A (\lambda h) = |\lambda|^\alpha~ Ah$.
\end{defi}
    Suppose that $\alpha, \beta \in \mathbb R$.
    Let $A_\alpha$ and $A_\beta$ be an $\alpha$-scalable and a $\beta$-scalable operator on $\mathcal F$.
    Then it is clear that $A_\alpha A_\beta$ is an $(\alpha\beta)$-scalable operator.
    Denote by $\mathcal P$ the class of functions of polynomial growth on $\mathbb R^d$ defined by
\begin{equation}
    \mathcal{P}
    :=\big \{f\in \mathscr B(\mathbb{R}^d):\exists~ C>0, n \in \mathbb N_0, {\rm ~s.t.~} \forall x\in \mathbb{R}^d,~ |f(x)|\leq C(1+|x|)^n\big\}.
\end{equation}
    It is clear that $\mathcal{P} \subset L^2(\varphi)$.
    For each $f\in \mathcal {P}$, define an operator $Q$ by $Qf(x):=\sup_{t\geq 0}e^{\kappa(f)bt} |T_tf(x)|$.
    According to \cite[Fact 1.2]{MM}, $Q$ is an $1$-scalable operator on $\mathcal P$.
    According to \cite[Fact 1.2]{MM}, for any $f \in \mathcal{P}$, there exists $R_0 \in \mathcal{P}$ such that,
\begin{equation}
\label{eq:semigroupineq}
    |T_tf(x)|
    \leq e^{-\kappa(f)bt} R_0(x),
    \quad t\geq 0, x\in \mathbb R^d.
\end{equation}
}
{\bf Zhenyao: I moved the next subsection in the appendix.}
\deleted{
\subsection{}
    Recall $\Gamma$ the Gamma function be defined by
\begin{equation}
    \Gamma (x) := \int_0^\infty t^{x-1} e^{-t}dt,
    \quad x>0,
\end{equation}
and
\begin{align}
\label{eq: definition of Gamma function}
    \Gamma(x)
    := \int_0^\infty t^{x-1} \Big(e^{-t} - \sum_{k=0}^{n-1} \frac{(-t)^k}{k!}\Big) dt,
    \quad -n< x< -n+1, n\in \mathbb N.
\end{align}
    It is known that
\begin{equation}
    \Gamma(x+1) = x\Gamma(x),
    \quad x>0~\text{or}~-n< x< -n+1, n\in \mathbb N.
\end{equation}

    Fix a $\beta \in (0,1)$.
    According to \cite[Theorem 3.2. \& Theorem 3.5.]{SchillingSongVondracek2010Bernstein}, we can verify that the following function:
\begin{equation}
    z
    \mapsto z^{\beta}
    := \int_0^\infty (e^{-zy}-1) \frac{dy}{\Gamma(-\beta)y^{1+\beta}},
    \quad z\in \mathbb C_+,
\end{equation}
\begin{itemize}
\item
    is an extension of the real function $x\mapsto x^{\beta}$ on $[0,\infty)$;
\item
    is continuous on $\mathbb C_+ :=\{x+iy: x\in [0,\infty), y\in \mathbb R\}$;
\item
    is holomorphic on $\mathbb C_+^0:= \{x+iy:x\in (0,\infty), y\in \mathbb R\}$.
\end{itemize}
    According to Lemma \ref{lem: extension lemma for branching mechanism} in the Appendix, we can verify that the following function:
\begin{equation}
\label{eq: stable branching on C+}
    z\mapsto
    z^{1+\beta}
    := \int_0^\infty (e^{-zy}-1+zy)\frac{dy}{\Gamma(-1-\beta)y^{2+\beta}},
    \quad z\in \mathbb C_+,
\end{equation}
\begin{itemize}
\item
    is an extension of the real function $x\mapsto x^{1+\beta}$ on $[0,\infty)$;
\item
    is continuous on $\mathbb C_+$;
\item
    is holomorphic on $\mathbb C_+^0$.
\end{itemize}
    Moreover, according to \eqref{eq: path integration representation of h}, for any $C^1$ path $\gamma:[0,1]\to \mathbb C_+$ ,
\begin{align}
\label{eq: integration formula for 1+beta-th power of z}
    &\gamma(1)^{1+\beta} - \gamma(0)^{1+\beta}
    = \int_0^1 \gamma'(\theta)d\theta \int_{(0,\infty)}(1-e^{-\gamma(\theta)y})\frac{ydy}{\Gamma(-1-\beta)y^{2+\beta}}
    \\&=\int_0^1 \gamma'(\theta)d\theta \int_{(0,\infty)}(1-e^{-\gamma(\theta)y})\frac{(-1-\beta)dy}{\Gamma(-\beta)y^{1+\beta}}
    = \int_0^1 (1+\beta) \gamma(\theta)^{\beta} \gamma'(\theta)d\theta.
\end{align}
    This indicates that the derivative of $z\mapsto z^{1+\beta}$ is $z\mapsto (1+\beta)z^{\beta}$ on $\mathbb C^0_+$.
\begin{lem}
\label{lem: Lip of power function}
    There is a constant $C_{\ref{lem: Lip of power function}}$ such that for any $z_0,z_1 \in \mathbb C_+$,
\begin{equation}
\label{eq: Lip of power function}
    |z_0^{1+\beta} - z_1^{1+\beta}|
    \leq C_{\ref{lem: Lip of power function}}(|z_0|^{\beta}+|z_1|^{\beta})|z_0 - z_1|.
\end{equation}
    In particular, for each $z\in \mathbb C$, we have
$
    |z^{1+\beta}|
    \leq  C_{\ref{lem: Lip of power function}} |z|^{1+\beta}.
$

\end{lem}
\begin{proof}
    Notice that
\begin{align}
\label{eq: upper bound for beta power of z}
    &|z^{\beta}|
    = \Big|\int_0^\infty (e^{-zy}-1) \frac{dy}{\Gamma(-\beta)y^{1+\beta}}\Big|
    \\&\leq \frac{1}{|\Gamma(-\beta)|}\int_0^\infty \frac{2\wedge (|z|y)}{y^{1+\beta}}dy
    = \frac{|z|^{\beta}}{|\Gamma(-\beta)|}\int_0^\infty \frac{2\wedge t}{t^{1+\beta}}dt =: c |z|^{\beta},
    \quad z\in \mathbb C_+.
\end{align}
    Define a path $\gamma: [0,1] \to \mathbb C_+$ such that
\[
    \gamma(\theta)
    = z_0 (1-\theta) + \theta z_1,
    \quad \theta \in [0,1].
\]
    Then, according to \eqref{eq: integration formula for 1+beta-th power of z}, we have
\begin{align}
    |z_0^{1+\beta} - z_1^{1+\beta}|
    &\leq (1+\beta) \int_0^1 |\gamma(\theta)^{\beta}|\cdot |\gamma'(\theta)|d\theta
    \leq (1+\beta) c_0 \sup_{\theta \in [0,1]} |\gamma(\theta)|^{\beta} \cdot |z_1-z_0|
    \\&\leq (1+\beta) c_0 ( |z_1|^{\beta}+|z_0|^{\beta} ) |z_1-z_0|.
    \qedhere
\end{align}
\end{proof}
}
\subsection{}
    Let $X$ be the super OU-process discussed in the \deleted{previous subsections} \added{first subsection}.
    Its branching mechanism $\psi$ can be extended on \deleted{$\mathbb C_+$ by $\psi(z) := -\alpha z + z^{1+\beta}$ for each $z\in \mathbb C_+$} \added{$\mathbb R^d\times \mathbb C_+$ by $\psi(x, z) := -\alpha z + z^{1+\beta}$ for each $x\in \mathbb R^d$ and $z\in \mathbb C_+$}.
    According to the above analysis, \added{for each $x\in \mathbb R^d$}, $\psi\added{(x,\cdot)}$ is continuous on $\mathbb C_+$ and is holomorphic on $\mathbb C_+^0$ with derivative
\begin{equation}
\label{eq: deriavetive of the Poission part}
    \psi'(\added{x,}z) := -\alpha + (1+\beta)z^{\beta},
    \quad \added{x\in \mathbb R^d}, z\in \mathbb C_+.
\end{equation}
    Write $\psi_0(\added{x,}z) := \psi(\added{x,}z) + \alpha z$ and $\psi'_0(\added{x,}z) := \psi'(\added{x,}z) + \alpha$.
    For each $f\in \mathscr B(\mathbb R^d, \mathbb C_+)$ and $x\in \deleted{E}\added{\mathbb R^d}$, define $\Psi_0f(x) := \psi_0(x,f(x))$ and $\Psi'_0 f(x) := \psi'_0(x,f(x))$.
    For each $t\in [0,\infty), x\in \mathbb E$ and $f \in \mathcal{P}$, define $U_tf(x) := \operatorname{Log} \mathbb P_{\delta_x}[e^{i\theta \langle f, X_t\rangle}]|_{\theta = 1}$
    (See the argument in Subsection \ref{sec: definition of characteristic exponent} of Appendix).
    From \eqref{eq: -v has positive real part}, we know that $-U_tf(x)$ take values in $\mathbb C_+$. Furthermore, we know from Proposition \ref{prop: complex FKPP-equation} that
\begin{align}
\label{eq:chareq2}
    U_tf(x)-\int_0^t T^{\alpha}_{t-s} \Psi_0(-U_sf)(x)ds
    =i T^{\alpha}_t f(x),
    \quad t\in [0,\infty), x\in E, f\in \mathcal P.
\end{align}

    {\bf Zhenyao: There are notation inconsistency in the Appendix and other sections. Mainly on the branching mechanism: In the appendix, we use
\[
    \psi(x,z)= -\beta(x)z+\alpha(x)z^2 + \int_0^t \dots;
\]
    But in other subsections we use
\[
    \psi(z) = - \alpha z + z^{1+\beta}.
\]
    Readers can easily be confused by which $\alpha$ and $\beta$ we mean in the context.}

\subsection{Main Result}

In this subsection, we will give the main results of this paper. In the remainder of this paper,whenever we deal with an initial configuration $\mu \in \mathcal{M}(\mathbb{R}^d)$, we are implicitly assuming that it has a compact support.
\subsubsection{Large Branching Rate}

For each $p\in \mathbb{Z}_+^d$, we define
$$H_t^p:= e^{-(\alpha-|p|b)t}\langle\phi_p,X_t\rangle,\quad t\geq 0.$$
 We will prove (see Lemma \ref{lemma26}) that, if $\alpha\beta>|p|b(1+\beta)$, $H_t^p$ is a $\mathbb{P}_{\mu}$-martingale bounded in $L^{1+\gamma}(\mathbb{P}_{\mu})$ for any $0<\gamma<\beta$, thus the limit $H^p_{\infty}:=\lim_{t\rightarrow \infty}H_t^p$ exists $\mathbb{P}_{\mu}$-as and in $L^{1+\gamma}(\mathbb{P}_{\mu})$.
 \begin{thm}\label{Theorem11}
     If $f \in \mathcal{P}$ satisfies $\alpha\beta>\kappa(f)b(1+\beta)$, then as $t\rightarrow \infty$,
     $$e^{-(\alpha-\kappa(f)b)t}\langle f, X_t\rangle \rightarrow\sum_{|p|=\kappa(f)}\langle f, \phi_p\rangle_{\varphi} H_{\infty}^p \quad in~ L^{1+\gamma}(\mathbb{P}_{\mu})$$
     for any $0<\gamma<\beta$.

     Moreover, if $f$ is twice differentiable and its twice derivative $D^2 f$ satisfies $|D^2 f| \in \mathcal{P}$, then the convergence is almost sure.
 \end{thm}
For any $t>0$, let $W_t:=e^{-\alpha t}\|X_t\|$, then $W_t$ is equal to $H_t^0$ and is a non-negative martingale with limit $W_{\infty}:=\lim_{t\rightarrow\infty}W_t$,  $\mathbb{P}_{\mu}$-a.s. and in $L^{1+\gamma}(\mathbb{P}_{\mu})$.
 \begin{rem}
    If $\kappa(f)=0$, $\langle f, \phi_{\kappa(f)}\rangle_{\varphi}$ reduces to $\langle f,\varphi\rangle$. Hence by Theorem \ref{Theorem11}, we get, as $t\rightarrow \infty$
     $$e^{-\alpha t}\langle f, X_t\rangle \rightarrow \langle f, \varphi\rangle W_{\infty} \quad in~ L^{1+\gamma}(\mathbb{P}_{\mu})$$
    for any $0<\gamma<\beta$.

    Moreover, if $f$ is twice differentiable and $|D^2 f| \in \mathcal{P}$, then the convergence is almost sure.
 \end{rem}

\subsubsection{Critical Branching Rate}
For $f\in \mathcal{P}$, define
$$\tilde{m}[f]:= \langle(-i\phi)^{1+\beta},\varphi\rangle$$
where
$$\phi(x)=\sum_{|p|=\kappa(f)}\langle f,\phi_p\rangle\phi_p(x).$$
See Lemma \ref{lemma210} for more details about $\tilde{m}[f]$.
\begin{thm}\label{Theorem12}
Let $f\in\mathcal{P}$, assume that  $\alpha\beta=\kappa(f)b(1+\beta)$, then, under $\mathbb{P}_{\mu}(\cdot|D^c)$, it holds that
$$\frac{\langle f,X_t\rangle}{\left(t\|X_t\|\right)^{\frac{1}{1+\beta}}}\xrightarrow{d} \eta_1, \quad t\rightarrow \infty,$$
where $\eta_1$ is a $(1+\beta)$-stable random variable with characteristic function given by
$$\mathbb{E} [e^{i\theta \eta_1}]=\exp(\theta^{1+\beta}\tilde{m}[f]).$$
\end{thm}

\subsubsection{Small Branching Rate}

For $f\in \mathcal{P}$, define
$$m[f]:=\int_{\mathbb{R}^d}\int_0^{\infty} e^{-\alpha s}(-iT_{s}^{\alpha}f)^{1+\beta}(x)~ds~\varphi(x)~dx$$
In Lemma \ref{lemma211}, we will prove that $|m[f]|<\infty$.
\begin{thm}\label{Theorem13}
    If $f\in\mathcal{P}$ satisfies  $\alpha\beta<\kappa(f)b(1+\beta)$, then, under $\mathbb{P}_{\mu}(\cdot|D^c)$, it holds that
    $$\frac{\langle f,X_t\rangle}{\|X_t\|^{\frac{1}{1+\beta}}}\xrightarrow{d} \eta_2, \quad t\rightarrow \infty,$$
    where $\eta_2$ is a $(1+\beta)$-stable random variable with characteristic function given by
    $$\mathbb{E} [e^{i\theta \eta_2}]=\exp(\theta^{1+\beta}m[f]).$$
\end{thm}

\section{Preliminaries}
    In this section, we will estimate the characteristic function and the $(1+\gamma)$-th moments of super OU-process, where $\gamma \in (0,\beta)$.
\deleted{
    Consider an incremental decomposition of $\langle g,X_t \rangle$.
    For each $t\geq 0$, denote by $\{\mathscr{F}_t:t\geq 0\}$ the natrual filtration of $X$.
    For each $g\in \mathcal P$ and $0 \leq a \leq b \leq t <\infty$, define random variable
\[
   	M_g^t(a,b)
    := \mathbb P_{\mu}[\langle g,X_t\rangle|\mathscr F_b] - \mathbb P_{\mu}[\langle g,X_t\rangle|\mathscr F_a]
    =\langle T_{t-b}^\alpha g, X_b\rangle - \langle T_{t-a}^\alpha g, X_a\rangle.
\]
    Thus, we have
\begin{align}
    M^t_g(0,t)
    = \sum_{k = 0}^{\lfloor t \rfloor - 1} M^t_g(k,k+1) + M^t_g(\lfloor t \rfloor, t).
\end{align}
    Note that $I_a^b[\langle g,X_t \rangle] = I_a^b[\langle g,X_t\rangle]$
\begin{align}
    M_g^t(a,b) = M_{(T_{t-b}^\alpha g)}^b (a,b).
\end{align}
}

\added{
	For each $0 \leq a \leq b <\infty$ and  random variable $Y$ with finite mean, define random variable
$
   	I_a^b[Y]
    := \mathbb P[Y|\mathscr F_b] - \mathbb P[Y|\mathscr F_a].
$
	Then we have an nature incremental decomposition of $Y$:
\begin{align}
    I_0^t[Y]
    = \sum_{k = 0}^{\lfloor t \rfloor - 1} I_k^{k+1}[Y] + I_{\lfloor t \rfloor}^t[Y],
    \quad t\geq 0.
\end{align}
	The random variable we are interested in is $Y:= \langle g,X_t\rangle$.
	In this case we have
\begin{align}
    I_a^b[\langle g,X_t\rangle]
    = I_a^b\big[\mathbb P[\langle g,X_t\rangle|\mathscr F_b]\big]
    = I_a^b[ \langle T_{t-b}^\alpha g, X_b \rangle ].
\end{align}
}

    Define $\mathcal P^+:= \mathcal P \cap \mathcal B(\mathbb R^d, \mathbb R_+)$ and $\mathcal P^*:= \{f\in \mathscr B(\mathbb R^d, \mathbb C): |f|\in \mathcal P\}$.
    We say $S$ is a $\gamma$-scalable operator for some $\gamma\in \mathbb R$ if $S: \mathcal P^+ \to \mathcal P^+$ and $S(\theta f) = \theta^\gamma Sf$ for each $\theta \geq 0$ and $f \in \mathcal P^+$.
    We say $R$ is a monotonic operator if $R:\mathcal P^+ \to \mathcal P^+$ and $Rf \leq Rg$ for each $f, g \in \mathcal P^+$ with $f\leq g$.
    We say $(R,S)$ is a $\gamma$-control-pair for some $\gamma \in \mathbb R$ if $R$ is a monotonic operator and $S$ is a $\gamma$-scalable operator and $Rf\leq Sf$ for each $f\in \mathcal P^+$.
    We say an operator $A$ is $\gamma$-controllable on $\mathcal D \added{\subset \mathcal P^*}$ for some $\gamma \in \mathbb R$ \deleted{and $\mathcal D \subset \mathcal P^*$} if $A: \mathcal D \to \mathcal P^*$ and there \deleted{exits} \added{is} a $\gamma$-control pair $(R,S)$ such that $|Af|\leq R|f|$ for each $f\in \mathcal D$.
    In this case we say $A$ is $\gamma$-controlled by the $\gamma$-control-pair $(R,S)$ on $\mathcal D$.
    We say a family of operator $(A_s)_{s\in \Lambda}$ is uniformly $\gamma$-controllable on $\mathcal D\added{\subset \mathcal P^*}$ for some $\gamma \in \mathbb R$ \deleted{and $\mathcal D \subset \mathcal P^*$} if there \deleted{exits} \added{is} a $\gamma$-control pair $(R,S)$ such that $A_s$ is $\gamma$-controlled by $(\added{R}\deleted{M}, S)$ on $\mathcal D$, for each $s\in \Lambda$.

    The first reason for considering $\gamma$-controllable operators is the following:
\added{
\begin{lem}
    Suppose that operators $(A_\lambda)_{\lambda\in \Lambda}$ are uniformly $\gamma$-controllable on $\mathcal D$ for some $\gamma \in \mathbb R$ and $\mathcal D \subset \mathcal P^*$.
    Then there exists a $\gamma$-scalable operator $S$ such that
\[
    |A_\lambda T_t^\alpha f|
    \leq e^{\gamma t (\alpha  - \kappa_f b)} SQf,
    \quad \lambda \in \Lambda, t\geq 0, f\in \mathcal D.
\]
\end{lem}
}
\added{
\begin{proof}
    Let $(R,S)$ be the $\gamma$-control-pair for $(A_\lambda)_{\lambda\in \Lambda}$.
    Then
\[
    |A_\lambda T_t^\alpha f| \leq R|T_t^\alpha f|
    \leq R (e^{\alpha t  - \kappa_f bt}Qf)
    \leq S (e^{\alpha t  - \kappa_f bt}Qf)
    \leq e^{\gamma t (\alpha  - \kappa_f b)} SQf.
    \qedhere
\]
\end{proof}
}
\deleted{
    For two operators $A$ and $B$ defined on same domain $\mathcal D\subset \mathcal P^*$, write $(A\times B)f (x):= Af(x) \times Bf(x)$.
    Let $a \in \mathbb R$, write $A^{\times a}f(x):= (Af(x))^a$ as long as it is well defined for all .
}
\added{
	For two operators $A: \mathcal D_A \to \mathcal P^*$ and $B: \mathcal D_B \to \mathcal P^*$, define $(A\times B)f (x):= Af(x) \times Bf(x)$ for each $f\in \mathcal D_A \cap \mathcal D_B$ and $x\in \mathbb R^d$.
    Let $a > 0$, define $A^{\times a}f(x):= (Af(x))^a$ for each $f\in \mathcal D_A$ and $x\in \mathbb R^d$.
}
    The second reason for considering $\gamma$-controllable operators is that they have good algebra properties:
\begin{lem}
\label{lem: property of controllable operators}
    Let operators $(A_\lambda)_{\lambda\in \Lambda}$ be uniformly $\gamma$-controllable on $\mathcal D$ from some $\gamma \in \mathbb R$ and $\mathcal D\subset \mathcal P^*$:
\begin{itemize}
\item[(1)]
    Suppose that $(\Lambda, \mathscr F)$ is a measurable space.
    Also suppose that $(\lambda,x)\mapsto A_\lambda f(x)$ is $\mathscr F \otimes \mathscr B(\mathbb R^d)$-measurable for each $f\in \mathcal D$.
    For each probability measure $\mu$ on $(\Lambda, \mathscr F)$ write
\[
    A_\mu f(x):= \int_{\Lambda} A_\lambda f (x)~\mu(d\lambda), \quad f\in \mathcal D, x\in \mathbb R^d.
\]
    Then operators $\{A_\mu: \mu \text{ is  a probability measure on } (\Lambda, \mathscr F)\}$ are uniformly $\gamma$-controllable on $\mathcal D$.
\item[(2)]
    Suppose that operators $(B_\delta)_{\delta\in \Delta}$ is uniformly $\beta$-controllable on $\mathcal D_0 \subset \mathcal P^*$ \added{for some $\beta \in \mathbb R$}.
    Also suppose that for each $\lambda$, $A_\lambda:\mathcal D \to \mathcal D_0$.
    Then operators $(B_\delta A_\lambda)_{\delta\in \Delta, \lambda \in \Lambda}$ is uniformly $(\gamma\beta)$-controllable on $\mathcal D$.
\item[(3)]
    Suppose that operators $(B_\delta)_{\delta \in \Delta}$ is uniformly $\beta$-controllable on $\mathcal D$ for some $\beta\in \mathbb R$.
    Then operators $(B_\delta\times A_\lambda)_{\delta \in \Delta, \lambda \in \Lambda}$ are uniformly $(\gamma+\beta)$-controllable.
\item[(4)]
    Let \deleted{$a\in \mathbb R$} \added{$a>0$}. Suppose that, for each $\lambda \in \Lambda$, $A_\lambda : \mathcal D \to \mathcal P^+$.
    Then operators $(A^{\times a}_\lambda)_{\lambda \in \Lambda}$ are uniformly $(a\gamma)$-controllable.
\end{itemize}
\end{lem}
\begin{proof}
    Proof of (1): Let $(R,S)$ be the $\gamma$-control-pair of $(A_\lambda)_{\lambda\in\Lambda}$ on $\mathcal{D}$. For each $f \in \mathcal{D}$ and $\mu$ the probability measure on $(\Lambda, \mathscr F)$.
\[
   |A_{\mu}f(x)|\leq \int_{\Lambda}|A_{\lambda}f(x)|\mu(d\lambda) \leq \int_{\Lambda}R|f|(x)\mu(d\lambda) \leq R|f|(x).
\]

   	Proof of (2): Let $(R_A, S_A)$ be the $\gamma$-control-pair of $(A_\lambda)_{\lambda\in\Lambda}$ on $\mathcal{D}$ and $(R_B, S_B)$ be the $\beta$-control-pair of $(B_{\delta})_{\delta\in\Delta}$ on $\mathcal{D}_0$.
\added{
	Note that $(R_BR_A, S_BS_A)$ is a $\beta \gamma$-control-pair.
	In fact:
\begin{itemize}
\item
	For each $f,g \in \mathcal P^+$ with $f\leq g$, since $R_Af \leq R_A g$ we have $R_B(R_A f)\leq R_B(R_A g)$.
\item
	For each $f\in \mathcal{P}^+$ and $\theta \geq 0$, we have $S_BS_A(\theta f)=S_B(\theta^{\gamma}S_Af)=\theta^{\beta\gamma}S_BS_Af$.
\item
	For each $f,g \in \mathcal P^+$ with $f\leq g$, we have $R_B R_A f \leq R_B S_A f \leq S_BS_A f$.
\end{itemize}
	Finally, note that for each $\delta\in \Delta, \lambda\in\Lambda$ and $f\in \mathcal D$, $|B_{\delta}A_{\lambda}f|\leq R_B|A_{\lambda}f|\leq R_BR_A|f|$ which says that operators $(B_\delta\times A_\lambda)_{\delta \in \Delta, \lambda \in \Lambda}$ are uniformly $(\beta\gamma)$-controlled by $(R_BR_A,S_BS_A)$.
}

   Proof of (3) and (4): Similar to the Proof of (2).
\end{proof}
\subsection{}
    For each $f \in \mathcal{P}$, $x\in E$ and $t\geq 0$, define
\begin{align}
\label{eq: def of Zf}
    \tilde U_t f(x)
    &:= i T^\alpha_t f(x) + \int_0^t T^\alpha_{t-s} \Psi_0(-i T_s^{\alpha}f)(x)ds
    \\Z_t f (x)
    &:= \int_0^t T^\alpha_{t-s} \Psi_0(-i T_s^{\alpha}f)(x)ds.
\end{align}

\begin{lem}
\label{lem: upper bound for usgx}
\added{The following statements are true for the super-OU process considered in the first subsection:}
\begin{itemize}
\item[(1)]
    Operators $(-U_t)_{0\leq t\leq 1}$ are uniformly $1$-controllable from $\mathcal P$ to $\mathcal P^*\cap \mathcal B(\mathbb R^d, \mathbb C_+)$.
\item[(2)]
    Operators $(T^\alpha_t)_{0\leq t\leq 1}$ are uniformly $1$-controllable on $\mathcal P$.
\item[(3)]
    Operator $\Psi_0$ is $(1+\beta)$-controllable on $\mathcal P^* \cap \mathcal B(\mathbb R^d, \mathbb C_+)$.
\item[(4)]
    Operators $(U_t- iT_t^{\alpha})_{0\leq t\leq 1}$ are uniformly $(1+\beta)$-controllable on $\mathcal P$.
\item[(5)]
    Operators $\{\Psi_0(-U_t) - \Psi_0(-iT_t^\alpha): 0\leq t\leq 1\}$ are uniformly $(1+2\beta)$-controllable on $\mathcal P$.
\item[(6)]
    Operator $(U_t-\tilde U_t)_{0\leq t\leq 1}$ are uniformly $(1+2\beta)$-controllable on $\mathcal P$.
\item[(7)]
    Operators $(Z_t)_{0\leq t\leq 1}$ are uniformly $(1+\beta)$-controallable on $\mathcal P$.
\end{itemize}
\end{lem}

\begin{proof}
    Proof of (1): From \eqref{eq: upper bound for vf} in the appendix, we have for each $g\in \mathcal P$, $0\leq t\leq 1$ and $x\in E$,
\[
    |U_t g(x)|
    \leq \sup_{0\leq u\leq 1}T_u^\alpha |g| (x).
\]
    Note that $f\mapsto\sup_{0\leq u\leq 1}T^{\added{\alpha}}_u|f|$ is monotonic and $1$-scalable.
    \added{Combine this with \eqref{eq: -v has positive real part}, we get the desired result.}

    Proof of (2): Similar to the Proof of (1).

    Proof of (3): According to Lemma \ref{lem: Lip of power function}, for each $f\in \mathcal P^* \cap \mathcal B(\mathbb R^d, \mathbb C_+)$,
\[
    |\Psi_0 f(x)| = |f(x)^{1+\beta}| \leq C_{\ref{lem: Lip of power function}} |f(x)|^{1+\beta}.
\]
    Also note that $f\mapsto |f|^{1+\beta}$ is monotonic and $(1+\beta)$-scalable.

    Proof of (4): From (1), (2), (3) and Lemma \ref{lem: property of controllable operators}(2) we know that operators
\[
    f
    \mapsto T^{\alpha}_{t-s}\Psi_0(-U_sf)(x),
    \quad 0\leq s\leq t\leq 1
\]
    are uniformly $(1+\beta)$-controllable.
    From \added{\eqref{eq:chareq2},} Lemma \ref{lem: property of controllable operators}(1) and this we get the desired result.

    Proof of (5): Notice that from Lemma \ref{lem: Lip of power function},
\[
    |\Psi_0(-U_t f) - \Psi_0(-iT_t^\alpha f) |
    \leq  C_{\ref{lem: Lip of power function}} |U_u f-iT_u^{\alpha}f|(|U_u f|^{\beta}+|i T_u^{\alpha}f|^{\beta}).
\]
    From (1), (2), (4) and Lemma \ref{lem: property of controllable operators} (3,4) we get the desired result.

    Proof of (6): Note that
\[
    U_sf - \tilde U_sf
    = \int_0^s T_{s-u}^{\alpha}\big(\Psi_0(-U_t f)-\Psi_0(-i T_t^{\alpha}f)\big)~du.
\]
    Then from (2), (5) and Lemma \ref{lem: property of controllable operators}(1,2) we get the desired result.

    Proof of (7): Similar to the proof of (4).
\end{proof}
\begin{comment}
\textbf{ Zhenyao: I think the following Corollary is incorrect. In its proof, we seems used Jensen's inequality assuming $\mu$ is a probability measure (which is not the case). At first, I think I may find a way to fix this. But then I realize we don't really need this Corrollary to calculate $\|X_t(g)\|_{1+\gamma}$. So I didn't touch it. Yet, I'm not sure if this Corrollary is needed somewhere else in section 3.}

\begin{cor}
\label{cor: corollary1}
    There exists an $R^g_3 \in \mathcal{P}$ such that for any $\mu\in \mathcal M(\mathbb R^d)$ with compact support, $k \in \mathbb{N}$,$\theta \in \mathbb{R}$ and $0\leq s\leq 1$, we have
\[
    \mathbb P_{\mu}[e^{i\theta(\langle g_k, X_s\rangle-\langle T_s^{\alpha}g_k,\mu \rangle)}]
    =1+\langle Z_{g_k}(s,\cdot,\theta),\mu\rangle+ err(k,\mu,\theta),
\]
    where $|err(k,\mu,\theta)| \leq (C_g|\theta e^{(\alpha-\kappa b)k}|^{2+2\beta} + C_g|\theta e^{(\alpha-\kappa b)k}|^{1+2\beta} )\langle R^g_3,\mu\rangle$.
\end{cor}
\begin{proof}
    Let
\begin{align}
\label{eq: definition of varphi-mu-k-theta}
    \varphi_{k,\mu,s}(\theta)
    =\mathbb{P}_{\mu}[e^{i\theta(\langle g_k, X_s\rangle-\langle T_s^{\alpha}g_k,\mu \rangle)}]
    =e^{(\langle v_{g_k}(s,\cdot,\theta),\mu \rangle-i\theta \langle T_s^{\alpha} g_k, \mu \rangle)},
    \quad \theta \in \mathbb R.
\end{align}
    Then we have
\begin{align*}
    &|\varphi_{k,\mu,s}(\theta)-1-\langle Z_{g_k}(s,\cdot, \theta),\mu\rangle|\\
    &\leq|\varphi_{\mu,k,s}(\theta)-1-\left( \langle v_{g_k}(s,\cdot,\theta), \mu \rangle-i\theta \langle T_s^{\alpha}g_k,\mu\rangle\right)| + |\langle v_{g_k}(s,\cdot,\theta)-\tilde{v}_{g_k}(s,\cdot,\theta),\mu \rangle|.
\end{align*}
    Let $R^g_1\in \mathcal P$ be the control function in Lemma \ref{lemma2}.
    Let $\tilde R^g$ be the control function in \eqref{eq: Zineq}.
    Notice that
$
    \operatorname{Re} \big(v_{g_k}(s,x,\theta) - i\theta T_s^\alpha g_k(x)\big)
    = \operatorname{Re} v_{g_k}(s,x,\theta)
    \leq 0,
$
    (see \eqref{eq: -v has positive real part} in the appendix); and if $ \operatorname{Re} z\leq 0$, we have $|e^z-1-z|\leq |z|^2$, (see Lemma \ref{lem: estimate of exponential remaining} in the appendix).
    Therefore, for any $\mu\in \mathcal M(\mathbb R^d)$ with compact support, $k \in \mathbb{N}$, $\theta \in \mathbb{R}$ and $0\leq s\leq1$, using Jensen's inequality,
\begin{align*}
    &|\varphi_{k,\mu,s}(\theta)-1- \langle Z_{g_k}(s,\cdot, \theta),\mu\rangle|\\
    &\leq |\langle v_{g_k}(s,\cdot,\theta)-i\theta T_s^{\alpha}g_k, \mu \rangle|^2 + \langle |v_{g_k}(s,\cdot,\theta)-\tilde{v}_{g_k}(s,\cdot,\theta)|, \mu \rangle
    \\&\leq \langle |v_{g_k}(s,\cdot,\theta)-i\theta T_s^{\alpha}g_k|^2, \mu \rangle + \langle |v_{g_k}(s,\cdot,\theta)-\tilde{v}_{g_k}(s,\cdot,\theta)|, \mu \rangle
    \\&\leq e^{2\alpha}|C_g\theta e^{(\alpha-\kappa b)k+\alpha}|^{2+2\beta}\langle (\tilde R^g)^2,\mu\rangle + |C_g\theta e^{(\alpha-\kappa b)k}|^{1+2\beta}\langle R^g_1,\mu\rangle.
    \qedhere
\end{align*}
\end{proof}
\end{comment}

\subsection{}

 In this subsection, for any $\gamma \in (0,\beta)$, we want to bound the $(1+\gamma)$-th moment of $\langle g ,X_t \rangle$. Denote $\mathcal{M}_c(\mathbb{R}^d)$ by the space of all finite measures on $\mathbb{R}^d$ with compact support.
\begin{lem}
\label{lem: control pair for P(M>lambda)}
    There is an $(1+\beta)$-control-pair $(R,S)$ such that for each $0\leq t\leq 1$, $g\in \mathcal P$, $\lambda >0$ and $\mu\in \mathcal M_c(\mathbb R^d)$ we have
\[
    \mathbb P_\mu (|\deleted{M^t_g(0,t)} \added{I_0^t[\langle g,X_t\rangle]}| > \lambda)
    \leq \frac{\lambda}{2}\int_{-2/\lambda}^{2/\lambda}\langle R|\theta g|,\mu\rangle d\theta
    \deleted{\leq \langle S| g|,\mu\rangle \frac{2^{2+\beta/(2+\beta)}}{\lambda^{1+\beta}}}.
\]
\end{lem}
\added{
\begin{proof}
    Denote by $(R,S)$ the $(1+\beta)$-control pair for Lemma \ref{lem: upper bound for usgx}(4).
    Using Lemma \ref{lem: estimate of exponential remaining} and the argument in \cite[Proof of Theorem 3.3.6]{Durrett2010Probability}, we have
\begin{align}
    &\big|\mathbb P_\mu (|I_0^t[\langle g,X_t\rangle]| > \lambda)\big|
    \leq \Big|\frac{\lambda}{2}\int_{-2/\lambda}^{2/\lambda}(1 - \mathbb P_\mu[e^{i\theta I_0^t[\langle g,X_t\rangle]}])d\theta\Big|
    \\&\leq \frac{\lambda}{2}\int_{-2/\lambda}^{2/\lambda}|1-e^{\langle U_t(\theta g)-iT_t^{\alpha}(\theta g),\mu \rangle}|d\theta
    \leq \frac{\lambda}{2}\int_{-2/\lambda}^{2/\lambda}\langle |U_t(\theta g) - iT_t^\alpha(\theta g)|,\mu\rangle d\theta
    \\&\leq \frac{\lambda}{2}\int_{-2/\lambda}^{2/\lambda}\langle R|\theta g|,\mu\rangle d\theta.
      \qedhere
\end{align}
\end{proof}
}

{\bf Zhenyao: I will change $M_g^t(a,b)$ to $I_a^b[\langle g,X_t\rangle]$ without mention. Here ``I'' means incremental.}

\begin{lem}
    There is an $(1+\beta)$-scalable operator $S$ such that for each $g\in \mathcal P$, $\lambda >0$, $\mu\in \mathcal M_c(\mathbb R^d)$ and $0\leq r\leq s\leq t$ with $s-r \leq 1$ we have
\[
    \mathbb P_{\mu}(|\added{I_r^s[\langle g, X_t\rangle]}|>\lambda)
    \leq \frac{ 1}{\lambda^{1+\beta}} e^{(1+\beta)(t-s)(\alpha- \kappa_gb)+ \alpha r} \langle S\added{Q}g,\mu\rangle.
\]
\end{lem}

\begin{proof}
    Denote by $(R_0,S_0)$ the $(1+\beta)$-control-pair in Lemma \ref{lem: control pair for P(M>lambda)}.
    Then using Markovian property of $(X_t)_{t\geq 0}$ we get
\added{
\begin{align}
    &\mathbb P_{\mu}(|I_r^s[\langle g, X_t\rangle]|>\lambda)
    = \mathbb P_\mu \big[\mathbb P_\mu[|\langle T_{t-s}^\alpha g, X_{s}\rangle - \langle T_{t-r}^\alpha g, X_{r}\rangle|> \lambda\big| \mathscr F_r]\big]
    \\&= \mathbb P_\mu \big[\mathbb P_{X_r}(|\langle T_{t-s}^\alpha g, X_{s-r}\rangle - \langle T_{t-r}^\alpha g, X_{0}\rangle|> \lambda)\big]
    = \mathbb P_\mu \big[\mathbb P_{X_r}(|I_0^{s-r}[\langle T_{t-s}^\alpha g, X_{s-r}\rangle] |> \lambda)\big]
    \\&\leq \mathbb P_\mu \Big[ \frac{\lambda}{2}\int_{-2/\lambda}^{2/\lambda}\langle R_0|\theta T^\alpha_{t-s}g|,X_r\rangle d\theta \Big]
    \leq \mathbb P_\mu \Big[ \frac{\lambda}{2}\int_{-2/\lambda}^{2/\lambda}\langle R_0|\theta e^{(t-s)(\alpha- \kappa_gb)}Qg|,X_r\rangle d\theta \Big]
	\\&\leq \mathbb P_\mu \Big[ \frac{\lambda}{2}\int_{-2/\lambda}^{2/\lambda}\langle S_0|\theta e^{(t-s)(\alpha- \kappa_gb)}Qg|,X_r\rangle d\theta \Big]
    = e^{(1+\beta)(t-s)(\alpha- \kappa_gb)} \mathbb P_\mu [ \langle S_0Qg,X_r\rangle ] \frac{\lambda}{2}\int_{-2/\lambda}^{2/\lambda}|\theta|^{1+\beta}d\theta
    \\& = e^{(1+\beta)(t-s)(\alpha- \kappa_gb)} \langle T_r^\alpha S_0Qg,\mu\rangle  \frac{2^{2+\beta}}{2+\beta}\frac{1}{\lambda^{1+\beta}}
    \leq e^{(1+\beta)(t-s)(\alpha- \kappa_gb)+ \alpha r} \langle QS_0Qg,\mu\rangle  \frac{2^{2+\beta}}{2+\beta}\frac{1}{\lambda^{1+\beta}}.
\end{align}
}
    Note that
\[
    S := \frac{2^{2+\beta}}{2+\beta}QS_0\deleted{Q}
\]
    is an $(1+\beta)$-scalable operator which is independent of the choice of $g,\mu, \lambda, r, s$ and $t$.
\end{proof}

\begin{lem}\label{lem: control of mgtrs}
    For each $g\in \mathcal P$, $\mu \in \mathcal M_c(\mathbb{R}^d)$ and $0<\gamma < \beta$, there exists a constant $C>0$ such that for each $0\leq r\leq s\leq t$ with $s-r \leq 1$ we have
\[
    \mathbb P_\mu\big[|I_r^s[\langle g, X_t\rangle]|^{1+\gamma}\big]
    \leq C e^{t\alpha+(t-s) (\gamma\alpha- (1+\gamma)\kappa_gb)}.
\]
\end{lem}

\begin{proof}
	\added{Let $S$ be the $(1+\beta)$-scalable operator in the previous Lemma.}
    For each $c>0$ we have
\begin{align}
    &\mathbb P_\mu\big[|I_r^s[\langle g, X_t\rangle]|^{1+\gamma}\big]
    = (1+\gamma)\int_0^\infty \lambda^{\gamma} \mathbb P_{\mu}(|I_r^s[\langle g, X_t\rangle]|>\lambda) d\lambda
    \\&\leq (1+\gamma)\int_0^c \lambda^{\gamma} d\lambda +(1+\gamma)\int_c^\infty \lambda^{\gamma}\mathbb P_\mu(|I_r^s[\langle g, X_t\rangle]|> \lambda) d\lambda
    \\& \leq c^{1+\gamma} + e^{(1+\beta)(t-s)(\alpha- \kappa_gb) + \alpha r} \langle S\added{Q}g,\mu\rangle  (1+\gamma)\int_c^\infty \frac{1}{\lambda^{1+(\beta-\gamma)}}d\lambda
    \\&\leq c^{1+\gamma} e^{\alpha r} + e^{(1+\beta)(t-s)(\alpha- \kappa_gb) + \alpha r} \langle S\added{Q}g,\mu\rangle   \frac{1+\gamma}{\beta - \gamma} \frac{1}{c^{\beta - \gamma}}.
\end{align}
    Taking $c = e^{(t-s)(\alpha- \kappa_gb)}$ we have
\begin{align}
    &\mathbb P_\mu\big[|I_r^s[\langle g, X_t\rangle]|^{1+\gamma}\big]
    \leq e^{(1+\gamma)(t-s)(\alpha- \kappa_gb)} e^{\alpha r}\Big(1+ \langle SQg,\mu\rangle \frac{1+\gamma}{\beta - \gamma}\Big).
\end{align}
    Note that
\[
    (1+\gamma)(t-s)(\alpha- \kappa_gb) + \alpha r
    \leq t\alpha+(t-s) (\gamma\alpha- (1+\gamma)\kappa_gb).
    \qedhere
\]
\end{proof}
\begin{comment}
\end{comment}
    \deleted{As $\langle g,X_t\rangle,~g\in \mathcal{P}$ is the sum of $M_g^t(k,k+1), k=0,1,...,\lfloor t\rfloor$ and $M_g^t(\lfloor t\rfloor, t)$. We obtain the following lemma.}

\begin{lem}
\label{lemma24}
    For each $0 < \gamma < \beta$ and $\mu\in \mathcal M_c(\mathbb R^d)$ and $g\in \mathcal P$ there exists a constant $C>0$ such that for each $t\geq 0$:
\begin{itemize}
\item[(1)]
    $\|\langle g,X_t\rangle\|_{\mathbb{P}_{\mu};1+\gamma}\leq C e^{(\alpha-\kappa_g b)t}$ provided $\alpha\gamma > \kappa_g (1+\gamma)b$.
\item[(2)]
    $\|\langle g,X_t\rangle\|_{\mathbb{P}_{\mu};1+\gamma}\leq C te^{\frac{\alpha}{1+\gamma}t}$ provided $\alpha\gamma = \kappa_g (1+\gamma)b$.
\item[(3)]
    $\|\langle g,X_t\rangle\|_{\mathbb{P}_{\mu};1+\gamma}\leq C e^{\frac{\alpha}{1+\gamma}t}$ provided $\alpha\gamma < \kappa_g (1+\gamma)b$.
\end{itemize}
\end{lem}
\begin{proof}
    Fix $\gamma \in (0,\beta)$ and $\mu \in \mathcal M_c(\mathbb R^d)$ and $g\in \mathcal P$.
    Let $C$ be the constant in Lemma \ref{lem: control of mgtrs}.
    Using the triangle inequality,
\begin{align}
    \|\langle g,X_t\rangle\|_{\mathbb P_\mu;1+\gamma}
    &\leq \sum_{k=0}^{\lfloor t\rfloor - 1}\big\| I_{t-k-1}^{t-k}[\langle g,X_t\rangle] \big\|_{\mathbb P_\mu;1+\gamma}+\big\| I_{0}^{t-\lfloor t \rfloor}[\langle g,X_t\rangle]  \big\|_{\mathbb P_\mu;1+\gamma}
    \\ &\leq C e^{\frac{\alpha}{1+\gamma}t}\sum_{k=0}^{\lfloor t\rfloor} e^{\frac{\gamma\alpha-\kappa_g (1+\gamma)b}{1+\gamma} k} .
\end{align}
    By calculating the sum on the right, we get the desired result.
\end{proof}

\subsection{Martingales}
    In this subsection, we will prove the almost sure and $L^{1+\gamma}(\mathbb{P}_{\mu})$ convergence of some martingales with $0<\gamma<\beta$. Recall that $L$ is the infinitesimal generator of OU-process. Let $a\in \mathbb{R}$ and $f\in \mathcal{P}\cap C^2(\mathbb{R}^d)$ such that $Lf \in \mathcal{P}$ ,we define that
\begin{align}
\label{defmartingale}
    M_t^{f,a}:=e^{-(\alpha-ab)t}\langle f,X_t\rangle-\int_0^t e^{-(\alpha-ab)s}\langle (L+ab)f, X_s\rangle~ ds.
\end{align}
{\bf Zhenyao: we should show that it is well defined.}
    The following lemma says that $\{M_t^{f,a}: t\geq 0\}$ is a martingale corresponding to the $\sigma$-algebra \added{$\{\mathscr{F}_t\}_{t\geq 0}$}.
{\bf Zhenyao: I changed all $\mathcal F$ to $\mathscr F$ without mention below. I want to use mathscr for $\sigma$-fields and mathcal for function classes.}
\begin{lem}
\label{lemma25}
    Let $a\in \mathbb R$ and $f\in \mathcal{P}\cap C^2(\mathbb{R}^d)$ such that $Lf \in \mathcal{P}$. Then the process $(M_t^{f,a})_{t\geq 0}$ is an $(\mathscr F_t)$-martingale.
\end{lem}
\begin{proof}
    Write $\bar{f}=(L+ab)f$. According to \cite[Theorem A.55]{Li2011Measure-valued}, we have
\begin{align}\label{Theorem55}
    T_t^{ab}f(x)= f(x)+\int_0^t T_s^{ab}\bar{f}(x)~ds,\quad t\geq 0,x\in \mathbb{R}^d.
\end{align}
Let $0\leq s\leq t$,
\begin{align}
\label{martingale1}
    &\mathbb{P}_{\mu}[M_t^{f,a}|\mathscr{F}_s]
    =e^{-(\alpha-ab)t}\mathbb{P}_{\mu}\left[\langle f,X_t\rangle|\mathscr{F}_s\right]-\mathbb{P}_{\mu}\Big[\int_0^t e^{-(\alpha-ab)u}\langle \bar{f}, X_u\rangle~ du\Big|\mathscr{F}_s\big]
    \\&=e^{-(\alpha-ab)t}\langle T_{t-s}^{\alpha}f, X_s\rangle-\int_0^s e^{-(\alpha-ab)u}\langle \bar{f}, X_u\rangle~ du -\int_s^t e^{-(\alpha-ab)u}\langle T_{u-s}^{\alpha} \bar{f},X_s\rangle~ du.
\end{align}
    Using \eqref{Theorem55} and Fubini's theorem, we have
\begin{align}
    &\int_s^t e^{-(\alpha-ab)u}\langle T_{u-s}^{\alpha} \bar{f},X_s\rangle~ du=e^{-(\alpha-ab)s}\int_s^t\langle T_{u-s}^{ab}\bar{f},X_s\rangle~du\\
    &=e^{-(\alpha-ab)s}\Big\langle\int_0^{t-s}T_{u}^{ab}\bar{f}~du,X_s\Big\rangle=e^{-(\alpha-ab)s}\left(\langle T_{t-s}^{ab}f,X_s\rangle-\langle
    f,X_s\rangle\right)\\
    &=e^{-(\alpha-ab)t}\langle T_{t-s}^{\alpha}f, X_s\rangle-e^{-(\alpha-ab)s}\langle
    f,X_s\rangle.
\end{align}
   Using this and \eqref{martingale1}, we get the desired result.
\end{proof}

    Let $p=(p_1,...,p_d)\in \mathbb N_0^d$, recall that $\phi_p$ is one of the eigenfunctions of $L$ corresponding to the eigenvalue $-|p|b$. Define
$$H_t^p:=e^{-(\alpha-|p|b)t}\langle\phi_p,X_t\rangle, \quad t\geq 0.$$

\begin{lem}\label{lemma26}
    $(H^p_t)_{t\geq 0}$ is a martingale with respect to filtration $(\mathscr F_t)$.
    Moreover, if $\alpha\beta>|p|b(1+\beta)$, then for any $0< \gamma<\beta$ and $\mu \in \mathcal M_c(\mathbb R^d)$, we have $\sup_{t\geq 0}\|H_t^p\|_{\mathbb P_\mu;1+\gamma}< \infty$ and
$$H_{\infty}^p:=\lim_{t\rightarrow \infty}H_t^p$$
exists $\mathbb{P}_{\mu}$-a.s and in $L^{1+\gamma}(\mathbb{P}_{\mu}).$
\end{lem}
\begin{proof}
    According to Lemma \ref{lemma25}, we get $(H_t^p)_{t\geq 0}$ is a martingale.

    There exists $\gamma_0 \in [0,\beta)$ close enough to $\beta$ such that for each $\gamma_0 \leq \gamma < \beta$, $\alpha\gamma>|p|(1+\gamma)b$.
    Using  Lemma \ref{lemma24} and the fact $\kappa_{\phi_p}=|p|$, we get that, \deleted{for any $\gamma_0\leq\gamma<\beta$,} there exists a constant $C_{\gamma, \mu, p} \added{>0}$ \added{(which depends on the choice of $\gamma$, $\mu$ and $p$)}  such that
 \added{
 $$
 	\|H_t^p\|_{\mathbb P_\mu;1+\gamma}
    \leq C_{\gamma, \mu, p} e^{-(\alpha-|p|b)t}e^{(\alpha-|p|b)t}
    =C_{\gamma, \mu, p}, \quad t\geq 0.
 $$
 }
    For any $0\leq\gamma<\gamma_0$, \deleted{consider the concave function $x\mapsto x^{(1+\gamma)/(1+\gamma_0)}$ on $[0,\infty)$, using Jensen's inequality, we have} \added{note that}
\added{
$$
	\|H_t^p\|_{\mathbb P_\mu;1+\gamma}
	\leq\|H_t^p\|_{\mathbb P_\mu;1+\gamma_0}
	<C_{\gamma_0, \mu, p},
	\quad t\geq 0.
$$
}

    Hence \added{, for each $\gamma \in (0,\beta)$,} the martingale is bounded in $L_{1+\gamma}(\mathbb{P}_{\mu})$ and this indicate\added{s} the convergence in $L_{1+\gamma}(\mathbb{P}_{\mu}) $ and almost sure,
%added
    according to \cite[Theorem 5.4.5]{Durrett2010Probability}.
%end added
\end{proof}


    Recall that $\|X_t\|=\langle 1,X_t\rangle$.
    By Lemma \ref{lemma26},
    %it is not to hard to prove that,
    it is clear that,
    under $\mathbb{P}_{\mu}$, the process $W_t=e^{-\alpha t}\|X_t\|$ is a positive martingale satisfying
$\lim_{t\rightarrow \infty} W_t= W_{\infty},$
$\mathbb{P}_{\mu}$-a.s and in $L^{1+\gamma}(\mathbb{P}_{\mu})$. So $W_{\infty}$ is non-degenerate and the $(1+\gamma)$-th moment is finite. Moreover,  we have $\mathbb{P}_{\mu}(W_{\infty})=\|\mu\|$ and $\{W_{\infty}=0\}=D,$ \added{$\mathbb{P}_{\mu}$-a.s} \deleted{which is the extinction event}.
%comment
    {\bf Zhenyao: we need references for the last assertion. Also, we used three different notation $M, H$ and $W$ to denote one martingale. This is unnessary.}
%end comment
\begin{lem}\label{lem: control of Wt}
 %For any $\gamma\in [0,\beta)$ and $0\leq s<t$, we have
 For any $\gamma\in (0,\beta)$ and $\mu\in \mathcal M_c(\mathbb R^d)$, there exists a \added{$C> 0$} such that for each $0\leq s<t$,
 $$\|W_t-W_s\|_{\mathbb{P}_{\mu};1+\gamma}\leq \added{C} e^{-\alpha s/2}.$$
\end{lem}

\begin{proof}
%new added
    Fix $\gamma \in (0,\beta)$ and $\mu\in \mathcal M_c(\mathbb R^d)$.
    According to Lemma \ref{lem: control of mgtrs}, there exists a constant \added{$C_0>0$}, such that for each $0\leq r\leq s $ with $s-r\leq1$, we have
    \begin{align}
        \mathbb{P}_{\mu}\big[\big|e^{\alpha(t-s)}\|X_s\|-e^{\alpha(t-r)}\|X_r\|\big|^{1+\gamma}\big]
        \deleted{=\mathbb{P}_{\mu}[|M_1^t(r,s)|^{1+\gamma}]}
        \leq \added{C_0}e^{\alpha t+(t-s)\deleted{(}\alpha\gamma\deleted{-(1+\gamma)b)}}.
    \end{align}
    Dividing both sides by $e^{\alpha t(1+\gamma)}$, we get
    \begin{align}
        \mathbb{P}_{\mu}\left|W_s-W_r\right|^{1+\gamma}\leq  \added{C_0} e^{-\alpha \gamma s}.
    \end{align}
    Then \added{there is a $C>0$,} for any $0\leq s<t$,
\deleted{
    \begin{align}
       & \|W_t-W_s\|_{\mathbb{P}_{\mu};1+\gamma}\leq \|W_{\lfloor s \rfloor +1}-W_s\|_{\mathbb{P}_{\mu};1+\gamma}+\sum_{k=\lfloor s \rfloor+2}^{\lfloor t \rfloor}\|W_k-W_{k-1}\|_{\mathbb{P}_{\mu};1+\gamma}+\|W_t-W_{\lfloor t \rfloor}\|_{\mathbb{P}_{\mu};1+\gamma}
       \\&\leq C_{\ref{lem: control of Wt}}e^{-\frac{\alpha \gamma}{1+\gamma}s}\leq C_{\ref{lem: control of Wt}}e^{-\frac{\alpha}{2}s}.
    \end{align}
}
\added{
\begin{align}
	& \|W_t-W_s\|_{\mathbb{P}_{\mu};1+\gamma}
	\leq \|W_{\lfloor s \rfloor+1}-W_s\|_{\mathbb{P}_{\mu};1+\gamma}+\sum_{k=\lfloor s \rfloor+1}^{\lfloor t \rfloor}\|W_{k+1}-W_{k}\|_{\mathbb{P}_{\mu};1+\gamma}+\|W_t-W_{\lfloor t \rfloor+1}\|_{\mathbb{P}_{\mu};1+\gamma}
	\\& \leq C_0^{\frac{1}{1+\gamma}} \Big(e^{-\frac{\alpha \gamma s}{1+\gamma}}+\sum_{k=\lfloor s \rfloor+1}^{\lfloor t \rfloor}e^{-\frac{\alpha \gamma k}{1+\gamma}}+ e^{-\frac{\alpha \gamma t}{1+\gamma}}\Big)
	\leq Ce^{-\frac{\alpha \gamma}{1+\gamma}s}\leq Ce^{-\frac{\alpha}{2}s}.
\end{align}	
}
\end{proof}

\deleted{
As $\|X_t\|$ is an continuous state branching process with initial mass $\|\mu\|$.
for any $\lambda>0$, $\|X_t\|$ satisfies, see \cite[Chapter 3]{Li2011Measure-valued}, for example.
\begin{align}\label{CSBP}
    \mathbb{P}_{\mu}e^{-\lambda\|X_t\|}=e^{-\|\mu\|v_t(\lambda)}
\end{align}
where $v_t(\lambda)$ is the unique local bounded solution to the equation
\begin{align*}
    \frac{\partial}{\partial t}v_t(\lambda)=-\psi(v_t(\lambda)), \quad  v_0(\lambda)=\lambda,  \quad \lambda>0, t\geq 0.
\end{align*}

In this paper, for any $\beta\in (0,1)$, $v_t(\lambda)$ has explicit expression \cite[Example 3.1]{Li2011Measure-valued}
\begin{align*}
    v_t(\lambda)=\frac{e^{\alpha t} \lambda}{[1+\frac{1}{\alpha}(e^{\alpha \beta t}-1)\lambda^{\beta}]^{1/\beta}}=\frac{\alpha^{1/\beta}}{(\alpha\lambda^{-\beta}e^{-\alpha \beta t}+1-e^{-\alpha \beta t})^{1/\beta}},\quad t\geq 0,\lambda\geq 0.
\end{align*}
If we fix $t>0$, $v_t(\lambda)$ is increasing corresponding to $\lambda$.
}
\added{
    According to \cite[Example 3.1]{Li2011Measure-valued}, we have
\begin{align}\label{CSBP}
    \mathbb{P}_{\mu} [e^{-\lambda\|X_t\|}]
    =e^{-\|\mu\|v_t(\lambda)},
    \quad \lambda \geq 0, t\geq 0,
\end{align}
    where
\[
    v_t(\lambda)=\frac{\alpha^{1/\beta}}{(\alpha\lambda^{-\beta}e^{-\alpha \beta t}+1-e^{-\alpha \beta t})^{1/\beta}},\quad t\geq 0,\lambda\geq 0.
\]
}

    Recall that the \deleted{superprocess $X$'s} extinction probability \added{of the superprocess is} $\mathbb{P}_{\mu}(D)=e^{-\alpha^*\|\mu\|}$ where $\alpha^*=\alpha^{1/\beta}$ is the larger roots of $\psi(z)=0$. 
    Denote \added{by} $\mathbb{\tilde{P}}_{\mu}=\mathbb{P}_{\mu}(\cdot|D^c)$. 
    \added{According to \eqref{CSBP} and the markovian property we have for each $\mu \in \mathcal M_c(\mathbb R^d), \lambda \geq 0$ and 
    $t\geq 0$,}
\added{
\begin{align}
    \label{laplaceexpress}
    \mathbb{\tilde{P}}_{\mu}[e^{-\lambda \|X_t\|}]
    &=\frac{ \mathbb P_\mu[e^{-\lambda \|X_t\| }\mathbb P_\mu(D^c| \mathscr F_t)] }{\mathbb P_\mu(D^c)}
    =\frac{ \mathbb P_\mu[e^{-\lambda \|X_t\| }(1-e^{-\alpha^*\|X_t\|})] }{\mathbb P_\mu(D^c)}
    \\&=\frac{e^{-\|\mu\|v_t(\lambda)}-e^{-\|\mu\|v_t(\lambda+\alpha^*)}}{1-e^{-\|\mu\|\alpha^*}}
    \leq \frac{\|\mu\|}{1-e^{-\|\mu\|\alpha^*}}\left|v_t(\lambda)-v_t(\lambda+\alpha^*)\right|. 
\end{align}
}
    \deleted{We want to prove the following Lemma}
\added{
\begin{lem}\label{lem: control of XT}
	For each $\mu\in \mathcal M_c(\mathbb R^d)$, there is a constant $C>0$ such that for each function $f:[0,\infty)\to [1,\infty)$ and each $t\geq 0$ we have
\begin{align}
    \mathbb{\tilde{P}}_{\mu}\big(\|X_t\|\leq f(t)\big)\leq C \Big(\frac{f(t)}{e^{\alpha t}}\Big)^{\beta}.
\end{align}
\end{lem}
}
\begin{proof}
 \deleted{Without loss of generality, we assume that $\|\mu\|=1$,} By Chebyshev's Inequality and formula \eqref{laplaceexpress}, we get
\added{
\begin{align*}
    \mathbb{\tilde{P}}_{\mu}[\|X_t\|\leq f(t)]&=\mathbb{\tilde{P}}_{\mu}[e^{-\lambda\|X_t\|}\geq e^{-\lambda f(t)}]\leq e^{\lambda f(t)}\mathbb{\tilde{P}}_{\mu}[e^{-\lambda \|X_t\|}]\\
    &\leq \frac{\|\mu\|}{1-e^{-\|\mu\|\alpha^*}}e^{\lambda f(t)}\left|v_t(\lambda)-v_t(\lambda+\alpha^*)\right|.
\end{align*}
}
	Let $w_t(\lambda)=\alpha \lambda^{-\beta}e^{-\alpha \beta t}+1-e^{-\alpha\beta t}$ where $t\geq 0,\lambda\geq0$ then $v_t(\lambda)=\alpha^{1/\beta}w_t(\lambda)^{-1/\beta}$.
\added{Note that}
\begin{align*}
    \left|v_t(\lambda)-v_t(\lambda+\alpha^*)\right|&\leq\frac {\alpha^{1/\beta}}{\beta} \left|w_t(\lambda)-w_t(\lambda+\alpha^*)\right| \sup_{s\in [\lambda,\lambda+\alpha^*]}w_t(s)^{-1/\beta-1}\\
    &\leq  \frac{\alpha^{1/\beta+1}}{\beta}\left|\frac{1}{\lambda^{\beta}}-\frac{1}{(\lambda+\alpha^*)^{\beta}}\right|e^{-\alpha\beta t} \sup_{s\in [\lambda,\lambda+\alpha^*]}w_t(s)^{-1/\beta-1}\\
    &\leq  \frac{\alpha^{1/\beta+1}}{\beta}\frac{2}{\lambda^{\beta}}e^{-\alpha\beta t} \sup_{s\in [\lambda,\lambda+\alpha^*]}w_t(s)^{-1/\beta-1}.
\end{align*}
Therefore, \added{for each $\mu\in \mathcal M_c(\mathbb R^d)$, there is a constant $C_0>0$ such that for each non-negative function $f$ on $[0,\infty),$ each $\lambda \geq 0$ and each $t\geq 0$ we have }
\begin{align}
\label{eq: inequality before take lambda}
    \mathbb{\tilde{P}}_{\mu}\big(\|X_t\|\leq f(t)\big)
    \leq C_0e^{\lambda f(t)}\frac{1}{\lambda^{\beta}}e^{-\alpha\beta t} \sup_{s\in [\lambda,\lambda+\alpha^*]}w_t(s)^{-1/\beta-1}.
\end{align}
\deleted{
Take $\lambda=\frac{1}{f(t)}$ into the above inequality, we have
\begin{align*}
    \mathbb{\tilde{P}}_{\mu}[\|X_t\|\leq f(t)]\leq \frac{2 e \alpha^{1+1/\beta}}{\beta (1-e^{-\alpha^*})}\cdot\sup_{s\in [1/f(t),1/f(t)+\alpha^*]}w_t(s)^{-1/\beta-1}\exp(\beta(\ln f(t)-\alpha t))
\end{align*}
}
Notice that
\begin{align*}
    w_t(s)^{-1/\beta-1}&\leq \frac{1}{(\alpha s^{-\beta}e^{-\alpha \beta t})^{1/
    \beta+1}}\mathbf{1}_{\{t\leq 1\}}+\frac{1}{(1-e^{-\alpha\beta t})^{1/\beta+1}}\mathbf{1}_{t\geq 1}\\
    &\added{\leq \frac{s^{1+\beta}e^{\alpha(1+\beta)}}{\alpha^{1/\beta +1}}+\frac{1}{(1-e^{-\alpha\beta })^{1/\beta+1}}.}
\end{align*}
\added{Therefore, for each $\lambda \in [0,1]$,
\begin{align*}
    \sup_{s\in [\lambda,\lambda+\alpha^*]}w_t(s)^{-1/\beta-1}
    &\leq\sup_{s\in [\lambda,\lambda+\alpha^*]}\frac{s^{1+\beta}e^{\alpha(1+\beta)}}{\alpha^{1/\beta +1}}+\frac{1}{(1-e^{-\alpha\beta })^{1/\beta+1}}\\
    &\leq \frac{(1+\alpha^*)^{1+\beta}e^{\alpha(1+\beta)}}{\alpha^{1/\beta+1}}+\frac{1}{(1-e^{-\alpha\beta })^{1/\beta+1}} =: C_1.
\end{align*}
}
\added{Finally, taking $\lambda = f(t)^{-1} \in [0,1]$ back to \eqref{eq: inequality before take lambda} and using the above inequality, we get the desired result.}
\end{proof}
	Define $\mathcal{A}_t(\epsilon):=\{ \|X_t\|\in \|\mu\|(e^{(\alpha-\epsilon)t},e^{(\alpha+\epsilon)t})\}$ where $\epsilon,t>0$.
\deleted{
\begin{lem}\label{lem: control of AT}
  For any $\epsilon,t>0$ and $\mu \in \mathcal{M}_c(\mathbb{R}^d)$, there exists $C>0$ and $\delta>0$ such that
  $$\mathbb{\tilde{P}}_{\mu}\left(\mathcal{A}_t(\epsilon)^c\right)\leq C e^{-\delta t}.$$
\end{lem}
}
\added{
\begin{lem}\label{lem: control of AT}
  For each $\mu \in \mathcal{M}_c(\mathbb{R}^d)$ there exists $C>0$ such that for each $t\geq 0$ we have
  $
  \mathbb{\tilde{P}}_{\mu}\left(\mathcal{A}_t(\epsilon)^c\right)\leq C e^{-\beta\epsilon t}.
  $
\end{lem}
}

\begin{proof}
    First, for any $\epsilon>0$, using Markov inequality, \added{we get for each $\epsilon>0$ and $t\geq 0$,}
\added{
\begin{align}
    \mathbb{\tilde{P}}_{\mu}(\|X_t\|\geq\|\mu\|e^{(\alpha+\epsilon)t})
    \leq\frac{\mathbb{\tilde{P}}_{\mu}[\|X_t\|]}{\|\mu\|e^{(\alpha+\epsilon)t}}
	= \frac{\mathbb P_\mu[\|X_t\|\mathbf 1_{D^c}]}{\mathbb P_\mu(D^c)\|\mu\|e^{(\alpha+\epsilon)t}}
    \leq \frac{1}{\mathbb{P}_{\mu}(D^c)}e^{-\epsilon t}.
\end{align}
}
    For another side, using Lemma \ref{lem: control of XT}, we have
    $$\mathbb{\tilde{P}}_{\mu}\left(\|X_t\|\leq \|\mu\|e^{(\alpha-\epsilon)t}\right)\leq C_0 e^{-\beta\epsilon t}, $$
    \added{for some constant $C_0>0$ not related to $\epsilon>0$ and $t\geq 0$.}
\deleted{To finish the proof we write $C_{\ref{lem: control of AT}}=C_{\ref{lem: control of XT}}+1/\mathbb{P}_{\mu}(D^c)$ and $\delta_{\ref{lem: control of AT}}=\beta\epsilon $.}
\end{proof}


\subsection{Parameters}
 Recall the definition of operator $Z_1$ in \eqref{eq: def of Zf} and the invariant density $\varphi$ in \eqref{invariantdensity}, define
 \begin{align}\label{parameter_mk}
      m_t[f]:=e^{-\alpha (t+1)}\int_{\mathbb R^d} (Z_1T_t^{\alpha}f)(x)\varphi(x)dx,\quad t\geq 0, f\in \mathcal P.
 \end{align}
 \deleted{Using the same argument as in the proof of \cite[Lemma2.7, Lemma4.2 \& Lemma 5.1]{MM}, we can show that}
\added{According to \cite[Lemma 2.7]{MM}, for each $f\in\mathcal{P}$, there is a constant $C>0$ such that 
\[
    |m_t[f]|
    \leq C e^{(\alpha\beta-\kappa_fb(1+\beta))t},
    \quad t\geq 0.
\]
}
\deleted{Therefore, $m_t[f]$ is well-defined.}

\added{
    Let $f \in \mathcal{P}$.
    Suppose that $\kappa_f\geq 1$ and $\alpha\beta=\kappa_f b(1+\beta)$.
    Denote by
\[
    \phi(x)
    =\sum_{|p|=\kappa_f}\langle f, \phi_p\rangle\phi_p(x),
    \quad x\in \mathbb{R}^d, p \in\mathbb Z_+^d,
\]
    the projection of $f$ onto the $\kappa_f$-eigenspace.
    Then accodting to \cite[Lemma 4.2]{MM} the following limit exists:
\begin{align}
\label{lemma210}
    \tilde{m}[f]:=\lim_{t\rightarrow \infty}\frac{1}{t}\sum_{k=0}^{\lfloor t \rfloor}m_k[f]=\langle\Psi_0(-i\phi),\varphi\rangle.
\end{align}

    Let $f\in \mathcal{P}$.
    Suppose that $\kappa_f\geq 1$ and $\alpha\beta<\kappa_fb(1+\beta)$. Then, according to \cite[Lemma 5.1]{MM}, the following series and integral absolutely convergence:
\begin{align}
    m[f]
    :=e^{-\alpha}\sum_{k=0}^\infty m_k[f]
    =\int_{\mathbb{R}^d}\int_0^{\infty} e^{-\alpha s}\Psi_0(-iT_{s}^{\alpha}f)(x)~ds~\varphi(x)~dx. \label{msmallcase}
\end{align}
}

\added{
    Note that $m_t[\theta f]$ can be rewritten as follows:
\[
    m_t[\theta f]= e^{-\alpha (t+1)}\int_{\mathbb R^d} dx~\varphi(x)\int_0^1 T_{1-s}^\alpha (-i\theta T_{s+t}^\alpha f)^{1+\beta}(x)~ds,
    \quad \theta \in \mathbb R.
\]
    The following Lemma says that function $\theta \mapsto \exp(m_t[\theta f])$ is the characteristic function of a $(1+\beta)$-stable random variable.
\begin{lem}
    Let $q$ be a measure on $\mathbb R^d\setminus\{0\}$ with $\int |x|^{1+\beta} q(dx) \in (0,\infty)$. 
    Then $$\theta \mapsto  \exp\Big\{\int_{\mathbb R^d\setminus\{0\}} (i\theta \cdot x)^{1+\beta} q(dx)\Big\}$$
    is the characteristic function of a $(1+\beta)$-stable random variable.
\end{lem}
\begin{proof}
    From the measure theory, there is a measure $\lambda$ on $S:= \{\xi\in \mathbb R^d:|\xi| = 1\}$ and a kernel $k(\xi,dt)$ from $S$ to $\mathbb R_+$ such that 
\[
    \int_{\mathbb R^d\setminus \{0\}} f(x)q(dx) = \int_S \int_{\mathbb R^+} f(\xi t)k(\xi,dt) \lambda(d\xi),\quad 
    f\in \mathscr B(\mathbb R^d\setminus \{0\}, \mathbb R_+).
\]
    Define another measure $\lambda_0$ on $S$ by
\[
    \lambda_0(d\xi) := \frac{1}{\Gamma(-1-\beta)} \int_0^\infty t^{1+\beta}k(\xi,dt) \lambda (d\xi).
\]
    Then $\lambda_0$ is a non-zero finite measure, since
\[
    \lambda_0(S) = \frac{1}{\Gamma(-1-\beta)} \int_S \int_0^\infty |t\xi|^{1+\beta}k(\xi,dt) \lambda (d\xi)
    = \frac{1}{\Gamma(-1-\beta)} \int_{\mathbb R^d\setminus\{0\}} |x|^{1+\beta} q(dx) \in (0,\infty).
\]
    Define a measure $\nu$ on $\mathbb R^d\setminus\{0\}$ by
\[
    \int_{\mathbb R^d/\{0\}}f(x)\nu(dx)= \int_{S} \int_0^\infty f(r\xi) \frac{dr}{r^{2+\beta}} \lambda_0(d\xi).
\]
    Then, according to \cite[Remark 14.4]{Sato1999Levy}, $\nu$ is the L\`evy measure of an $(1+\beta)$-stable distribution on $\mathbb R^d$, say $\mu$, whose characteristic function is \[\hat \mu(\theta)=\exp\Big\{\int_{\mathbb R^d/\{0\}} (e^{-i\theta \cdot y}-1+i\theta \cdot y) \nu(dy)\Big\}.\]
    The final step, we verify that
\begin{align}
    &\int_{\mathbb R^d/\{0\}} (e^{-i\theta \cdot y}-1+i\theta \cdot y) \nu(dy)
    = \int_S \int_0^\infty (e^{-ir\theta \cdot \xi}-1+ir\theta \cdot \xi) \frac{dr}{r^{2+\beta}}\lambda_0(d\xi)
\\&\quad = \int_S \int_0^\infty (e^{-ir\theta \cdot \xi}-1+ir\theta \cdot \xi) \frac{dr}{\Gamma(-1-\beta)r^{2+\beta}}\int_0^\infty t^{1+\beta} k(\xi,dt) \lambda (d\xi)
\\&\quad = \int_S (i\theta\cdot \xi)^{1+\beta} \int_0^\infty t^{1+\beta} k(\xi,dt) \lambda (d\xi)
= \int_S \int_0^\infty (i\theta \cdot t\xi)^{1+\beta} k(\xi,dt)\lambda(d\xi)
\\&\quad = \int_{\mathbb R^d} (i\theta \cdot x)^{1+\beta} q(dx).
\qedhere
\end{align}
\end{proof}
}
\section{Proof of main results}

In this section, we will prove the main results of this paper. Recall that $\mathbb{\tilde{P}}_{\mu}=\mathbb{P}_{\mu}(\cdot|D^c)$ and let $F_k(\nu)=\left(e^{\alpha k}\|\nu\|\right)^{\frac{1}{1+\beta}},~ \nu\in \mathcal{M}(\mathbb{R}^d)$.

\begin{lem}\label{lemma31}
For any $x,y\in \mathbb{C}$, $|e^x-e^y|\leq (e^{Re(x)}\vee e^{Re(y)})|x-y|$.
\end{lem}
\begin{proof}
let $z=Re(y)+iIm(x)$,
\begin{align}\label{estim}
    |e^x-e^y|\leq|e^x-e^z|+|e^z-e^y|\leq |e^{Re(x)}-e^{Re(y)}|+e^{Re(y)}|x-y|\leq (e^{Re(x)}\vee e^{Re(y)})|x-y|.
\end{align}
\end{proof}
{\bf Zhenyao: The proof of the above lemma is not right. The proof of the following lemma is not accurate.}
\begin{lem}\label{lemma32}
Recall $\mathcal{A}_t(\epsilon)=\{\|X_t\|\in\|\mu\|(e^{(\alpha-\epsilon_0)t},e^{(\alpha+\epsilon_0)t})\},~\epsilon,t>0$, then for any $\epsilon>0$, we have
\begin{align}
    \lim_{t\rightarrow\infty}\mathcal{A}_t(\epsilon)=D^c, ~~\mathbb{P}_{\mu}\text{-}{\rm a.s}.
\end{align}
\end{lem}
\begin{proof}
    We just need to prove
    \begin{align}
        D^c \subset \liminf_{t\rightarrow\infty}\mathcal{A}_t(\epsilon)\subset \limsup_{t\rightarrow\infty}\mathcal{A}_t(\epsilon)\subset D^c,~~\mathbb{P}_{\mu}\text{-}{\rm a.s}.
    \end{align}
    for any $\omega\in D$, there exists $S>0$, such that $\|X_t(\omega)\|=0,~t\geq T$. Thus $\omega \notin \cup_{t\geq S}\mathcal{A}_t(\epsilon)$ and $\omega\notin  \limsup_{t\rightarrow\infty}\mathcal{A}_t(\epsilon):=\cap_{S\geq 0}\cup_{t\geq S}\mathcal{A}_t(\epsilon)$ imply that $\limsup_{t\rightarrow\infty}\mathcal{A}_t(\epsilon)\subset D^c$.

    For any $\omega \in D^c$, there exists $\delta>0$ such that $W_{\infty}(\omega)\in(\delta,\frac{1}{\delta})$,\deleted{$~\mathbb{P}_{\mu}$-a.s}. As $W_t$ converges to $W_{\infty}$ $\mathbb{P}_{\mu}$-a.s, there exists $S_1>0$ such that for any $t\geq S_1$, we have $\frac{\delta}{2}\leq W_t(\omega)\leq 2\delta,$\deleted{$~\mathbb{P}_{\mu}$-a.s}. Taking $S_2>0$ such that for any $t>S_2$, $\|\mu\|e^{-\epsilon t}<\frac{\delta}{2}$ and $\|\mu\|e^{-\epsilon t}>2\delta$, then $\omega\in \mathcal{A}_t(\epsilon),~t>S:=\max\{S_1,S_2\}$ implies $\omega \in \liminf_{t\rightarrow \infty}\mathcal{A}_t(\epsilon):=\cup_{S>0}\cap_{t\geq S}\mathcal{A}_t(\epsilon)$.
\end{proof}
\deleted{
\begin{lem}
For any $f\in \mathcal{P}$ and $k\in \mathbb{N}$, let $m_k[f]$ be the parameter defined in \eqref{parameter_mk}, then the function $\theta\mapsto \exp(\theta^{1+\beta}e^{\alpha}m_k[f])$ defined on $\theta\in(0,\infty)$ is a characteristic function of some $(1+\beta)$-stable random variable.
\end{lem}
\begin{proof}
    Let $J=\langle\int_0^1 T_{1-s}^{\alpha}(-T^{\alpha}_s f)^{1+\beta}(\cdot)ds,\varphi\rangle$, then
    \begin{align}
        m_k[f]=J e^{-\alpha(k+1)}\cdot i^{1+\beta}=J e^{-\alpha(k+1)}\cos\frac{(1+\beta)\pi}{2}\big(1+i\tan\frac{(1+\beta)\pi}{2}\big).
    \end{align}
Thus
\begin{align}
    \exp(\theta^{1+\beta}e^{\alpha}m_k[f])=\exp\big(J e^{-\alpha k}\cos\frac{(1+\beta)\pi}{2}\cdot\theta^{1+\beta}\big(1+i\tan\frac{(1+\beta)\pi}{2}\big)\big)
\end{align}
B \cite[Theorem 14.15]{Sato1999Levy}, we get the result.
\end{proof}
}
\begin{lem}\label{lemma33}

 Let $f\in \mathcal{P}$, assume that $\alpha\beta\leq \kappa(f)b(1+\beta)$, then for any  $k\in\mathbb{N}$ and $\mu \in \mathcal{M}(\mathbb{R}^d)$ with compact support, under $\mathbb{P}_{\mu}(\cdot | D ^c)$, we have
 \begin{align}
      \frac{M_k^t[f]}{F_k(X_{t-k-1})}\xrightarrow{d}\zeta_k, \quad t\rightarrow \infty, \label{limitdistribution1}
 \end{align}
 where $\zeta_k$ is a $(1+\beta)$-stable random variable with characteristic function
 $$\mathbb{E}e^{i\theta\zeta_k}=\exp\left[\theta^{1+\beta}e^{\alpha}m_k[f]\right].$$
 \end{lem}

 \begin{proof}
     Let $v_{f_k}(x,\theta)=v_{f_k}(1,x,\theta),\tilde{v}_{f_k}(x,\theta)=\tilde{v}_{f_k}(1,x,\theta)$ and $\tilde{\theta}_{t,k}=\frac{\theta}{F_k(X_{t-k-1})}$. For any $k\in\mathbb{N}$ and $t>k$, we have
         \begin{align}
        &\mathbb{P}_{\mu}\big[\exp(i\theta\frac {M_k^t[f]}{F_k(X_{t-k-1})})\big|\mathscr{F}_{t-k-1}\big]\\
        &=\exp\big(\langle v_{f_k}(\cdot,\tilde{\theta}_{t,k}),X_{t-k-1}\rangle-i\tilde{\theta}_{t,k}\langle T_{k+1}^{\alpha}f, X_{t-k-1}\rangle\big).\\
        \label{condition mean}
        &=\exp\big(\langle v_{f_k}(\cdot,\tilde{\theta}_{t,k})-\tilde{v}_{f_k}(\cdot, \tilde{\theta}_{t,k}),X_{t-k-1}\rangle+\langle Z_{f_k}(\cdot,\tilde{\theta}_{t,k}),X_{t-k-1}\rangle\big)\\
       \end{align}
         Fix $\gamma_0\in (0,\beta)$, let $p=\min\{b,\frac{\alpha\gamma_0}{2(1+\gamma_0)}\}$ and $\epsilon_0=\min\{\frac{\alpha\beta}{2(1+2\beta)},\frac{p}{2}\}$.
       To get the conclusion of Lemma \ref{lemma33}, it suffices to show that
       \begin{align}
           \mathbb{P}_{\mu}\Big(\exp(i\theta\frac{M_k^t[f]}{F_k(X_{t-k-1}}))\mathbf{1}_{D^c}\Big)\xrightarrow{t\rightarrow \infty}\mathbb{P}_{\mu}(D^c)\exp(\theta^{1+\beta}e^{\alpha}m_k[f]).
       \end{align}
\deleted{
        \begin{align}
        &\left|\mathbb{P}_{\mu}\left[\left(\exp(i\theta\frac{M_k^t[f]}{F_k(X_{t-k-1}})-\exp(\theta^{1+\beta}e^{\alpha}m_k[f])\right)\mathbf{1}_{D^c}\right]\right| \\
        &\leq\left|\mathbb{P}_{\mu}\left[\left(\exp(i\theta\frac{M_k^t[f]}{F_k(X_{t-k-1}})-\exp(\theta^{1+\beta}e^{\alpha}m_k[f])\right)\mathbf{1}_{\mathcal{A}_{t-k-1}(\epsilon_0)}\right]\right|\\
        &+\left|\mathbb{P}_{\mu}\left[\left(\exp(i\theta\frac{M_k^t[f]}{F_k(X_{t-k-1}})-\exp(\theta^{1+\beta}e^{\alpha}m_k[f])\right)\mathbf{1}_{D^c  \cap\mathcal{A}^c_{t-k-1}(\epsilon_0)}\right]\right|
        \end{align}
         Let $c=|\theta^{1+\beta}e^{\alpha}m_k[f]|$. By lemma \ref{lemm28}, there exists $\delta_1>0$ and $C_5>0$ such that
       \begin{align}
            &\left|\mathbb{P}_{\mu}\left[\left(\exp(i\theta\frac{M_k^t[f]}{F_k(X_{t-k-1}})-\exp(\theta^{1+\beta}e^{\alpha}m_k[f])\right)\mathbf{1}_{D^c  \cap\mathcal{A}^c_{t-k-1}(\epsilon_0)}\right]\right| \\
            &\leq \mathbb{P}_{\mu}(D^c)\mathbb{\tilde{P}}_{\mu}(\mathcal{A}_{t-k-1})\leq (1+e^c)C_5\mathbb{P}_{\mu}(D^c)e^{-\delta_1(t-k-1)}.
       \end{align}
       }
     Equivalently, by Lemma \ref{lemma32}, we just need to prove
     \begin{align}
        \Big|\mathbb{P}_{\mu}\Big[\big(\exp(i\theta \frac {M_k^t[f]}{F_k(X_{t-k-1})})-\exp(\theta^{1+\beta}e^{\alpha}m_k[f])\big)\mathbf{1}_{\mathcal{A}_{t-k-1}(\epsilon_0)}\Big]\Big|\xrightarrow{t\rightarrow\infty} 0.
    \end{align}

      \deleted{ On the other hand, by formula \eqref{condition mean} and Lemma \ref{estim}}
      By formula \eqref{condition mean} and lemma \ref{lemma31},
    \begin{align}
        &\Big|\mathbb{P}_{\mu}\Big[\big(\exp(i\theta \frac {M_k^t[f]}{F_k(X_{t-k-1})})-\exp(\theta^{1+\beta}e^{\alpha}m_k[f])\big)\mathbf{1}_{\mathcal{A}_{t-k-1}(\epsilon_0)}\Big]\Big|\\
        &=\Big|\mathbb{P}_{\mu}\Big[\mathbb{P}_{\mu}\Big(\big(\exp(i\theta \frac {M_k^t[f]}{F_k(X_{t-k-1})})-\exp(\theta^{1+\beta}e^{\alpha}m_k[f])\big)\mathbf{1}_{\mathcal{A}_{t-k-1}(\epsilon_0)}\Big|\mathscr{F}_{t-k-1}\Big)\Big]\Big|\\
        &\leq e^c\mathbb{P}_{\mu}\big[\big|\langle v_{f_k}(\cdot,\tilde{\theta}_{t,k})-\tilde{v}_{f_k}(\cdot,\tilde{\theta}_{t,k}), X_{t-k-1}\rangle\big|\mathbf{1}_{\mathcal{A}_{t-k-1}(\epsilon_0)}\big]\\
        \label{ineqinlem31}
        &+e^c\mathbb{P}_{\mu}\big[\big|\langle Z_{f_k}(\cdot,\tilde{\theta}_{t,k}),X_{t-k-1}\rangle-\theta^{1+\beta}e^{\alpha}m_k[f]\big|\mathbf{1}_{\mathcal{A}_{t-k-1}(\epsilon_0)}\big].
    \end{align}
Step 1.  We will show that for each $\mu \in\mathcal{M}(\mathbb{R}^d)$ with compact support, there exists $C_6>0$, such that for any $\theta\in \mathbb{R}$, $k\in\mathbb{N}$ and $t>k$,
    \begin{align}
    \label{lemma31q}
        \mathbb{P}_{\mu}\big[\big|\langle v_{f_k}(\cdot,\tilde{\theta}_{t,k})-\tilde{v}_{f_k}(\cdot,\tilde{\theta}_{t,k}), X_{t-k-1}\rangle\big|\mathbf{1}_{\mathcal{A}_{t-k-1}(\epsilon_0)}\big]\leq C_6|\theta|^{1+2\beta}e^{-\frac{\alpha\beta}{2(1+\beta)}(t-k-1)}.
    \end{align}
Combining with \eqref{eq:semigroupineq}, \eqref{eq:gcontrol1} and Lemma \ref{lemma2}, there exist $R_1^f,R_6>0$ such that
\begin{align}
    & |v_{f_k}(x,\tilde{\theta}_{t,k})-\tilde{v}_{f_k}(x,\tilde{\theta}_{t,k})|\leq |\tilde{\theta}_{t,k} e^{(\alpha-\kappa(f) b)k}|^{1+2\beta}R^f_1(x),\\
    & |T_t (R^f_1)(x)|\leq R_6(x), \quad x\in \mathbb{R}^d,t\geq 0, \theta \in \mathbb{R}.
\end{align}
%Let $R_1$ be the control function in Lemma \ref{lemma2}. According to \eqref{eq:semigroupineq}, there exists $R_6\in \mathcal{P}$, satisfying $|T_t %(R_1)(x)|\leq R_6(x)$, for each $t>0$ and $x\in \mathbb{R}^d$.

For any $\theta\in \mathbb{R}$, $k\in \mathbb{N}$ and $t>k$, on the event $\mathcal{A}_{t-k-1}(\epsilon_0)$, we have
    \begin{align*}
        &\mathbb{P}_{\mu}\big[\big|\langle v_{f_k}(\cdot,\tilde{\theta}_{t,k})-\tilde{v}_{f_k}(\cdot,\tilde{\theta}_{t,k}), X_{t-k-1}\rangle\big|\mathbf{1}_{\mathcal{A}_{t-k-1}(\epsilon_0)}\big]
        \\ &\leq \Big(\frac{|\theta|e^{(\alpha-\kappa(f)b)k}}{(e^{\alpha k}\|\mu\|e^{(\alpha-\epsilon)(t-k-1)})^\frac{1}{1+\beta}}\Big)^{1+2\beta}\mathbb{\tilde{P}}_{\mu}\langle R^f_1,X_{t-k-1}\rangle\\
        &\leq\frac{1}{\|\mu\|^{(1+2\beta)/(1+\beta)}}\langle R_6,\mu\rangle|\theta|^{1+2\beta}e^{\alpha(t-k-1)} e^{-\frac{\alpha(1+2\beta)(t-k-1)}{1+\beta}}e^{\frac{\epsilon_0(1+2\beta)(t-k-1)}{1+\beta}}\\
        &\leq \frac{1}{\|\mu\|^{(1+2\beta)/(1+\beta)}}\langle R_6,\mu\rangle|\theta|^{1+2\beta}e^{-\frac{\alpha\beta}{2(1+\beta)}(t-k-1)}.
    \end{align*}
     \deleted{
 Step 2. Let $h_k(x):=Z_{f_k}(x)-\langle Z_{f_k},\varphi\rangle,~x\in\mathbb{R}^d$.According to \cite[Lemma 2.8]{MM}, there exists $R_7\in \mathcal{P}$ such that for any $k\in\mathbb{N}$, $x\in \mathbb{R}^d$ and $s\geq 0$,
 \begin{align}
 \label{eq:31step2}
     |T_s h_k(x)|\leq e^{-bs}e^{(\alpha-\kappa(f)b)(1+\beta)k}R_7(x).
 \end{align}

 Let $\tilde{f}=T_1^{\alpha}f$, then $\kappa(\tilde{f})=\kappa(f)$. It is easy to show that $\tilde{f}\in C^{\infty}$ and $\tilde{f}^{(p)}\in\mathcal{P},~p\in\mathbb{Z}_+^d$, thus by the inequality \eqref{eq:semigroupineq} and the fact $\kappa\big(\frac{\partial \tilde{f}}{\partial x_i}\big)\geq \kappa(\tilde{f})-1$, see \cite[Lemma 2.3]{RSZ}, we get there exists $\tilde{R}_7\in \mathcal{P}$ such that for each $i=1,...,d$
 \begin{align}
 \label{ineq-temp}
     \Big|T_t\big(\frac{\partial\tilde{f}}{\partial x_i}\big)(x)\Big|\leq e^{(\alpha-(\kappa(f)-1)b)t}\tilde{R}_7(x), \quad x\in \mathbb {R}^d, t\geq 0.
 \end{align}
 As $\langle h_k,\varphi\rangle=0$,
 \begin{align}
     &T_s h_k(x)= T_s [h_k(\cdot+xe^{-bs})] (0)-\langle h_k,\varphi\rangle
     \\ &= \int_{\mathbb{R}^d}\big(h_k(xe^{-bs}+y\sqrt{1-e^{-2bs}})-h_k(y)\big)\varphi (y)~dy
     \\ &\leq \int_{\mathbb{R}^d}|\nabla h_k(z)||xe^{-bs}+y\sqrt{1-e^{-2bs}}-y|\varphi (y)~dy.
  \end{align}
 Where $z$ is a point on the line segment connecting $y$ and $xe^{-bs}+y\sqrt{1-e^{-2bs}}$. For each $i=1,...,d$
 \begin{align}
     \frac{\partial h_k}{\partial x_i}(x)=\frac{\partial Z_{g_k}}{\partial x_i}(x)=\int_0^1 T^{\alpha}_{1-s}\Big(\frac{\partial}{\partial x_i}(-i T_{s+k-1}^{\alpha} \tilde{f})^{1+\beta}\Big)(x)~ds.
 \end{align}
  Let $R_0$ be the control function in \eqref{eq:semigroupineq},
      \begin{align}
        \Big|\frac{\partial}{\partial x_i}(-i T_{s+k-1}^{\alpha}\tilde{f})^{\beta}\Big|&=\Big|(1+\beta)(-i T_{s+k-1}^{\alpha}\tilde{f})^{\beta}(-i e^{-b(s+k-1)}T_{s+k-1}^{\alpha}\big(\frac{\partial \tilde{f}}{\partial x_i}\big))\Big|\\
        &\leq e^{(\alpha-\kappa(\tilde{f})b)(s+k-1)(1+\beta)}\cdot (1+\beta)e^{(\alpha-\kappa(f)b)\beta}R_0^{\beta}(x)\tilde{R}_7(x).
    \end{align}
Thus there exists $R_8\in \mathcal{P}$,
\begin{align}
\label{eq:31temp }
    \Big|\frac{\partial}{\partial x_i}h_k(x)\Big|\leq e^{(\alpha-\kappa(f)b)(1+\beta)k}R_8(x),\quad x\in \mathbb{R}^d, k\in \mathbb{N}.
\end{align}
By \eqref{eq:31temp } and the fact $|xe^{-bs}+y\sqrt{1-e^{-2bs}}-y|\leq e^{-bs}(|x|+|y|)$,
\begin{align}
    |T_s h_k(x)|\leq e^{-bs}e^{(\alpha-\kappa(f)b)(1+\beta)k}R_8(x)(|x|+\int_{\mathbb{R}^d}|y|\varphi(y)~dy).
\end{align}
}
Step 2, for any $\mu \in \mathcal{M}_c(\mathbb{R}^d)$ and $\theta\in \mathbb{R}$, we will show that there exists $C_7>0$ such that
\begin{align}
\label{eq:31step3}
    &\mathbb{P}_{\mu}\big[\big|\langle Z_{f_k}(\cdot,\tilde{\theta}_{t,k}),X_{t-k-1}\rangle-\theta^{1+\beta}e^{\alpha}m_k[f]\big|\mathbf{1}_{\mathcal{A}_{t-k-1}(\epsilon_0)}\big]\\
    &\leq C_7|\theta|^{1+\beta} e^{-\frac{p}{2}(t-k-1)},\quad k\in \mathbb{N}, t\geq k.
\end{align}
    Notice that $\langle Z_{f_k}(\cdot,\tilde{\theta}_{t,k}),X_{t-k-1}\rangle-\theta^{1+\beta}e^{\alpha}m_k[f]=\theta^{1+\beta}e^{-\alpha k}(\frac{\langle Z_{f_k},X_{t-k-1}\rangle}{\|X_{t-k-1}\|} \rangle-\langle Z_{f_k},\varphi\rangle)$. Therefore,
    %and $\theta^{1+\beta}e^{\alpha}m_k[f]=\theta^{1+\beta}e^{-\alpha k}\langle Z_{f_k},\varphi\rangle$.
\begin{align}
        &\mathbb{P}_{\mu}\left[\left|\langle Z_{f_k}(\cdot,\tilde{\theta}_{t,k}),X_{t-k-1}\rangle-\theta^{1+\beta}e^{\alpha}m_k[f]\right|\mathbf{1}_{\mathcal{A}_{t-k-1}(\epsilon_0)}\right]\\
    &=|\theta|^{1+\beta}e^{-\alpha k}\mathbb{P}_{\mu}\left[\left|\frac{\langle Z_{f_k},X_{t-k-1}\rangle}{\|X_{t-k-1}\|}-\langle Z_{f_k},\varphi\rangle\right|\mathbf{1}_{\mathcal{A}_{t-k-1}(\epsilon_0)}\right]\nonumber\\
    &\leq |\theta|^{1+\beta}e^{-\alpha k}e^{-(\alpha-\epsilon_0)(t-k-1)}\mathbb{P}_{\mu}\left[\left|\langle h_k,X_{t-k-1}\rangle\right|\right],\label{II1}
\end{align}
\added{
where $h_k(x):=Z_{f_k}(x)-\langle Z_{f_k},\varphi\rangle,~x\in\mathbb{R}^d$.According to \cite[Lemma 2.8]{MM}, there exists $R_7\in \mathcal{P}$ such that for any $k\in\mathbb{N}$, $x\in \mathbb{R}^d$ and $s\geq 0$,
 \begin{align}
 \label{eq:31step2}
     |T_s h_k(x)|\leq e^{-bs}e^{(\alpha-\kappa(f)b)(1+\beta)k}R_7(x).
 \end{align}
}
Thus $h_k$ satisfies \eqref{eq:gcontrol1} with $C_{h_k}=e^{(\alpha-\kappa(f))(1+\beta)k}$, $\kappa=1$ and $R_0^{h_k}(x)=R_7(x)$, fix some $\gamma_0\in(0,\beta)$, according to Lemma \ref{lemma24}, there exist $C_8>0$,
\begin{align}
    &\mathbb{P}_{\mu}\left[\left|\langle h_k,X_{t-k-1}\rangle\right|\right]\leq \|\langle h_k, X_{t-k-1}\rangle\|_{\mathbb{P}_{\mu,1+\gamma_0}}\\
    &\leq C_8 e^{(\alpha-\kappa(f)b)(1+\beta)k}
    \begin{cases}
    e^{(\alpha-b)(t-k-1)}\quad &\alpha\gamma_0>(1+\gamma_0)b\\
    e^{(\alpha-\frac{\gamma_0}{2(1+\gamma_0)}\alpha)(t-k-1)}\quad &\alpha\gamma_0=(1+\gamma_0)b\\
    e^{(\alpha-\frac{\gamma_0}{(1+\gamma_0)}\alpha)(t-k-1)}\quad &\alpha\gamma_0<(1+\gamma_0)b\\
    \end{cases}
    \\& \leq C_8 e^{(\alpha-\kappa(f)b)(1+\beta)k}e^{(\alpha-p)(t-k-1)}, \quad k\in \mathbb{N}, t\geq k,
\end{align}
We want to emphasize that, on the case $\alpha\gamma_0=(1+\gamma_0)b$, the second equation holds when $t$ is large enough. Thus
\begin{align}
    &\mathbb{P}_{\mu}\left[\left|\langle Z_{f_k}(\cdot,\tilde{\theta}_{t,k}),X_{t-k-1}\rangle-\theta^{1+\beta}e^{\alpha}m_k[f]\right|\mathbf{1}_{\mathcal{A}_{t-k-1}(\epsilon_0)}\right]
    \\ & \leq C_8 |\theta|^{1+\beta}e^{(\alpha\beta-\kappa(f)b(1+\beta))k}e^{-(p-\epsilon_0)(t-k-1)} \leq C_8 |\theta|^{1+\beta}e^{-\frac{p}{2}(t-k-1)}.
\end{align}

At last, combine \eqref{lemma31q} and \eqref{eq:31step3} and let $t\rightarrow\infty$, we get the result.
\end{proof}
\begin{cor}\label{corollary31}
Let $f\in \mathcal{P}$ and $\Theta>0$, assume that $\alpha\beta\leq\kappa(f)b(1+\beta)$. There exists $C_9,\delta_2>0$ such that for any $n \in \{0,...,t\}$ and $(\theta_0,...,\theta_n)\in \mathbb{R}^{n+1}$ satisfying $|\theta_i|\leq \Theta$, we have
\begin{align}
\label{32corollary}
    \left|\mathbb{\tilde{P}}_{\mu}\left(\prod_{k=0}^n\exp(i\theta_k \frac {M_k^t[f]}{(e^{\alpha k}\|X_{t-k-1}\|)^\frac{1}{1+\beta}})-\prod_{k=0}^n\exp(\theta_k^{1+\beta}e^{\alpha}m_k[f])\right)\right|\leq C_9 e^{-\delta_2(t-n)}
\end{align}
\end{cor}
\begin{proof}
    Denote $\gamma_{t,k}=\frac {M_k^t[f]}{(e^{\alpha k}\|X_{t-k-1}\|)^\frac{1}{1+\beta}} $, we define
    $$\varphi^k_t(\theta_0,...,\theta_n):=\prod_{l=0}^{k}\exp\left(\theta_l^{1+\beta}e^{\alpha}m_l[f]\right)\mathbb{\tilde{P}}_{\mu}\left(\prod_{l=k+1}^{n}\exp\left(i\theta_l\gamma_{t,l}\right)\right)$$
    for $k\in\{-1,...,n\}$, if $k=-1$, the product is one. Then Conditioning with $\mathscr{F}_{t-k-1}$ we get
    \begin{align*}
        \varphi^{k-1}_t(\theta_0,...,\theta_n)-&\varphi^{k}_t(\theta_0,...,\theta_n)=\prod_{l=0}^{k-1}\exp\left(\theta_l^{1+\beta}e^{\alpha}m_l[f]\right)\\
        &\mathbb{\tilde{P}}_{\mu}\left[\left(\mathbb{\tilde{P}}_{\mu}\left(e^{i\theta_k \gamma_{t,k}}|\mathscr{F}_{t-k-1}\right)-e^{\theta_k^{1+\beta}e^{\alpha}m_k[f]}\right)\prod_{l=k+1}^n\exp(i\theta_l \gamma_{t,l})\right]
    \end{align*}
    Hence,
    \begin{align*}
        \left|\varphi^{k-1}_t(\theta_0,...,\theta_n)-\varphi^{k}_t(\theta_0,...,\theta_n)\right| \leq \mathbb{\tilde{P}}_{\mu}\left|\mathbb{\tilde{P}}_{\mu}\left(e^{i\theta_k \gamma_{t,k}}|\mathscr{F}_{t-k-1}\right)-e^{\theta_k^{1+\beta}e^{\alpha}m_k[f]}\right|.
    \end{align*}
    Step 1, we will prove that there exists $C_{10},\delta_3>0$ such that for any $|\theta|<\Theta$ and $k\in \{0,...,t\}$,
    \begin{align}
    \label{eq:32step1}
        \left|\mathbb{\tilde{P}}_{\mu}\left(e^{i\theta \gamma_{t,k}}|\mathscr{F}_{t-k-1}\right)-e^{\theta^{1+\beta}e^{\alpha}m_k[g]}\right|\leq C_{10} e^{-\delta_3(t-k-1)}.
    \end{align}
    %we completed our proof, as the left side of thesis of the lemma is equal to %$\left|\varphi^{-1}_t(\theta_0,...,\theta_n)-\varphi^{n}_t(\theta_0,...,\theta_n)\right|$.

    Recall that
    $$\mathbb{\tilde{P}}_{\mu}\left(e^{i\theta \gamma_{t,k}}|\mathscr{F}_{t-k-1}\right)=\exp(\langle v_{f_k}(\cdot,\tilde{\theta}_{t,k})-\tilde{v}_{f_k}(\cdot, \tilde{\theta}_{t,k}),X_{t-k-1}\rangle+\langle Z_{f_k}(\cdot,\tilde{\theta}_{t,k}),X_{t-k-1}\rangle).$$
 By \eqref{lemma31q} and \eqref{eq:31step3}, we easily get, if $|\theta|\leq\Theta$ and $k\in\{0,...,t\}$,
    \begin{align*}
        &\mathbb{\tilde{P}}_{\mu}\left[\left|\langle v_{f_k}(\cdot,\tilde{\theta}_{t,k})-\tilde{v}_{f_k}(\cdot,\tilde{\theta}_{t,k}), X_{t-k-1}\rangle\right|\mathbf{1}_{\mathcal{A}_{t-k-1}(\epsilon_0)}\right]\leq \frac{C_{6}}{\mathbb{P}_{\mu}(D^c)}|\Theta|^{1+2\beta} e^{-\frac{\alpha\beta}{2(1+\beta)}(t-k-1)}\\
        &\mathbb{\tilde{P}}_{\mu}\left[\left|\langle Z_{f_k}(\cdot,\tilde{\theta}_{t,k}),X_{t-k-1}\rangle-\theta^{1+\beta}e^{\alpha}m_k[f]\right|\mathbf{1}_{\mathcal{A}_{t-k-1}(\epsilon_0)}\right]\leq \frac{C_{7}}{\mathbb{P}_{\mu}(D^c)}|\Theta|^{1+\beta} e^{-\frac{p}{2}(t-k-1)}.
    \end{align*}
 Define
\begin{align*}
    \mathcal{B}_{t-k-1}:=\left\{\left|\langle v_{f_k}(\cdot,\tilde{\theta}_{t,k})-\tilde{v}_{f_k}(\cdot,\tilde{\theta}_{t,k}), X_{t-k-1}\rangle\right|\mathbf{1}_{\mathcal{A}_{t-k-1}(\epsilon_0)}\leq\frac{C_{6}}{\mathbb{P}_{\mu}(D^c)}|\Theta|^{1+2\beta} e^{-\frac{\alpha\beta}{4(1+\beta)}(t-k-1)}\right\}.
\end{align*}
Applying Markov's inequality, we get $\mathbb{\tilde{P}}_{\mu}(\mathcal{B}^c_{t-k-1})\leq e^{-\frac{\alpha\beta}{4(1+\beta)}(t-k-1)}$. Similarly, define
\begin{align*}
    \mathcal{C}_{t-k-1}:=\left\{\left|\langle Z_{f_k}(\cdot,\tilde{\theta}_{t,k}),X_{t-k-1}\rangle-\theta^{1+\beta}e^{\alpha}m_k[f]\right|\mathbf{1}_{\mathcal{A}_{t-k-1}(\epsilon_0)}\leq \frac{C_{7}}{\mathbb{P}_{\mu}(D^c)}|\Theta|^{1+\beta} e^{-\frac{p}{4}(t-k-1)}\right\},
\end{align*}
 which satisfies  $\mathbb{\tilde{P}}_{\mu}(\mathcal{C}^c_{t-k-1})\leq e^{-\frac{p}{4}(t-k-1)}$.

Hence, for any $k\in\{0,...,t\}$, on the event $\mathcal{A}_{t-k-1}(\epsilon_0)\cap\mathcal{B}_{t-k-1}\cap\mathcal{C}_{t-k-1}$, we get
\begin{align*}
   &\left|\mathbb{\tilde{P}}_{\mu}\left(e^{i\theta \gamma_{t,k}}|\mathscr{F}_{t-k-1}\right)-e^{\theta^{1+\beta}e^{\alpha}m_k[f]}\right|\\
   &\leq \left|\langle v_{f_k}(\cdot,\tilde{\theta}_{t,k})-\tilde{v}_{f_k}(\cdot,\tilde{\theta}_{t,k}), X_{t-k-1}\rangle\right|
   +\left|\langle Z_{f_k}(\cdot,\tilde{\theta}_{t,k}),X_{t-k-1}\rangle-\theta^{1+\beta}e^{\alpha}m_k[f]\right|\\
   &\leq \frac{1}{\mathbb{P}_{\mu}(D^c)}(C_6|\Theta|^{1+2\beta}+C_7|\Theta|^{1+\beta}) e^{-\min\{\frac{\alpha\beta}{4(1+\beta)},\frac{p}{4}\}(t-k-1)}.
\end{align*}
On the event $(\mathcal{A}_{t-k-1}(\epsilon_0)\cap\mathcal{B}_{t-k-1}\cap\mathcal{C}_{t-k-1})^c$, we get
\begin{align*}
    &\left|\mathbb{\tilde{P}}_{\mu}\left(e^{i\theta \gamma_{t,k}}|\mathscr{F}_{t-k-1}\right)-e^{\theta^{1+\beta}e^{\alpha}m_k[f]}\right|\\
    &\leq 2(\mathbb{\tilde{P}}_{\mu}(\mathcal{A}^c_{t-k-1}(\epsilon_0))+\mathbb{\tilde{P}}_{\mu}(\mathcal{B}^c_{t-k-1})+\mathbb{\tilde{P}}_{\mu}(\mathcal{C}^c_{t-k-1}))\\
    &\leq 6e^{-\min\{\delta_1,\frac{\alpha\beta}{2(1+\beta)},\frac{p}{4}\}(t-k-1)}.
\end{align*}

Step 2, notice that the left side of \eqref{32corollary} is equal to $\left|\varphi^{-1}_t(\theta_0,...,\theta_n)-\varphi^{n}_t(\theta_0,...,\theta_n)\right|$, moreover,
\begin{align}
    &\left|\varphi^{-1}_t(\theta_0,...,\theta_n)-\varphi^{n}_t(\theta_0,...,\theta_n)\right|
    \\&\leq\sum_{k=0}^n\left|\varphi^{k-1}_t(\theta_0,...,\theta_n)-\varphi^{k}_t(\theta_0,...,\theta_n)\right|
    \\&\leq \sum_{k=0}^n C_{10} e^{-\delta_3(t-k-1)}.
\end{align}
Thus, we complete our proof.
\end{proof}


In next three subsections, we just use $C$ and $\delta$ as a constant positive real number, which may be different in different formula.
\subsection{Proof of Theorem \ref{Theorem12}}

    For any $f\in\mathcal{P}$, we write
    \begin{align*}
        (t\|X_t\|)^{-\frac{1}{1+\beta}}\langle f,X_t\rangle&=\sum_{k=0}^{\lfloor t-\ln t \rfloor} (t\|X_t\|)^{-\frac{1}{1+\beta}}M_k^t[f]+\sum_{k=\lceil t-\ln t \rceil}^t (t\|X_t\|)^{-\frac{1}{1+\beta}}M_k^t[f]\\
        &=I_t+J_t.
    \end{align*}
    Denote
    $$\tilde{I}_t=\sum_{k=0}^{\lfloor t-\ln t \rfloor}\frac{M_k^t[f]}{(t e^{\alpha(k+1)}\|X_{t-k-1}\|)^{\frac{1}{1+\beta}}}.$$
    Taking $\theta_k=(t e^{\alpha})^{-\frac{1}{1+\beta}} \theta $ and $n={\lfloor t-\ln t \rfloor}$ in Corollary \ref{corollary31}, we get
    \begin{align*}
        \left|\mathbb{\tilde{P}}_{\mu}e^{i\theta\tilde{I}_t}-\exp\left(\theta^{1+\beta}\frac{1}{t}\sum_{k=0}^{\lfloor t-\ln t \rfloor}m_k[f]\right)\right|\leq C \frac{1}{t^{\delta}},
    \end{align*}
    for some $C,\delta>0$. Hence we obtain that $\tilde{I}_t\rightarrow\eta_1$ as $t\rightarrow \infty$, by using Lemma \ref{lemma210}.

    Therefore, to prove this theorem ,we just to prove $\left|\mathbb{\tilde{P}}_{\mu}e^{i\theta I_t}-\mathbb{\tilde{P}}_{\mu}e^{i\theta\tilde{I}_t}\right|\rightarrow 0$ and $J_t\rightarrow^d 0$.

    Denote
    \begin{align*}
        Y_{t,k}=\exp\left(i\theta\frac{M_k^t[f]}{(t e^{\alpha(k+1)}\|X_{t-k-1}\|)^{\frac{1}{1+\beta}}}\right)-\exp\left(i\theta\frac{M_k^t[f]}{\left(t\|X_t\|\right)^{\frac{1}{1+\beta}}}\right).
    \end{align*}

    Using the element inequality $|\prod x_i-\prod y_i|\leq\sum |x_i-y_i|$ for $x_i,y_i \in \mathbb{C}$ with $|x_i|,|y_i|\leq 1$, we get
    \begin{align*}
        \left|\mathbb{\tilde{P}}_{\mu}e^{i\theta I_t}-\mathbb{\tilde{P}}_{\mu}e^{i\theta\tilde{I}_t}\right|\leq \sum_{k=0}^{\lfloor t-\ln t \rfloor}\mathbb{\tilde{P}}_{\mu}|Y_{t,k}|.
    \end{align*}
    Recall Lemma \ref{lemma27}, define
    \begin{align*}
        \mathcal{D}_{t,k}:=\left\{|W_t-W_{t-k-1}|\leq \|\mu\| e^{-\frac{\alpha}{4}(t-k-1)}, W_{t-k-1}>\|\mu\|e^{-\frac{\alpha}{32}(t-k-1)}\right\},
    \end{align*}
    then follow Lemma \ref{lemma27} and Lemma \ref{lemma28}, we get
    \begin{align*}
        \mathbb{\tilde{P}}_{\mu}(\mathcal{D}_{t,k}^c)&\leq \mathbb{\tilde{P}}_{\mu}(|W_t-W_{t-k-1}|\geq \|\mu\| e^{-\frac{\alpha}{4}(t-k-1)})+\mathbb{\tilde{P}}_{\mu}(W_{t-k-1}\leq \|\mu\|e^{-\frac{\alpha}{32}(t-k-1)}),\\
        &\leq C e^{-\delta(t-k-1)}
    \end{align*}
    for some $C,\delta>0$. Hence
    \begin{align}
        \mathbb{\tilde{P}}_{\mu}|Y_{t,k}|\mathbf{1}_{\mathcal{D}^c_{t,k}}\leq 2 C e^{-\delta(t-k-1)}.\label{thm121}
    \end{align}

    Using the element inequality $|e^{ix}-e^{iy}|\leq|x-y|$,
    \begin{align*}
        \mathbb{\tilde{P}}_{\mu}|Y_{t,k}|\mathbf{1}_{\mathcal{D}_{t,k}}&\leq|\theta|t^{-\frac{1}{1+\beta}}\mathbb{\tilde{P}}_{\mu}|M_k^t[f]|\left|\frac{1}{\left(e^{\alpha(k+1)\|X_{t-k-1}\|}\right)^{\frac{1}{1+\beta}}}-\frac{1}{\|X_t\|^{\frac{1}{1+\beta}}}\right|\mathbf{1}_{\mathcal{D}_{t,k}}\\
        &\leq|\theta|t^{-\frac{1}{1+\beta}}e^{-\frac{\alpha}{1+\beta}t}\mathbb{\tilde{P}}_{\mu}|M_k^t[f]|\left|\frac{W_t^{\frac{1}{1+\beta}}-W_{t-k-1}^{\frac{1}{1+\beta}}}{W_t^{\frac{1}{1+\beta}}W_{t-k-1}^{\frac{1}{1+\beta}}}\right|\mathbf{1}_{\mathcal{D}_{t,k}}.
    \end{align*}
    When $t$ is large enough, $t^{-\frac{1}{1+\beta}}\leq 1$. What's more, by Corollary \ref{cor27}
    $$\mathbb{\tilde{P}}_{\mu}|M_k^t[f]|\leq \frac{\mathbb{P}_{\mu}(|M_k^t[f]|)}{\mathbb{P}(D^c)}\leq C e^{\frac{\alpha}{1+\gamma}t}e^{\frac{\gamma \alpha-\kappa(f)(1+\gamma)b}{1+\gamma}k},$$
for some constant $C \geq 0$ and any $0<\gamma<\beta$.
We denote
$$K_{t,k}:=\left|\frac{W_t^{\frac{1}{1+\beta}}-W_{t-k-1}^{\frac{1}{1+\beta}}}{W_t^{\frac{1}{1+\beta}}W_{t-k-1}^{\frac{1}{1+\beta}}}\right|\mathbf{1}_{\mathcal{D}_{t,k}}.$$
 On the event $\mathcal{D}_{t,k}$, by mean value theorem of integrals, we get
 \begin{align*}
     \left|W_t^{\frac{1}{1+\beta}}-W_{t-k-1}^{\frac{1}{1+\beta}}\right|\leq \max \left\{W_t^{-\frac{\beta}{1+\beta}},W_{t-k-1}^{-\frac{\beta}{1+\beta}}\right\}\left|W_t-W_{t-k-1}\right|,
 \end{align*}
 what's more, when $t$ is sufficiently large,
 \begin{align*}
     W_t\geq W_{t-k-1}-\|\mu\| e^{-\frac{\alpha}{4}(t-k-1)}&\geq\|\mu\|\left(e^{-\frac{\alpha}{32}(t-k-1)}-e^{-\frac{\alpha}{4}(t-k-1)}\right)\\
     &\geq \|\mu\| e^{-\frac{\alpha}{16}(t-k-1)}.
 \end{align*}
Hence,
\begin{align*}
    \left|W_t^{\frac{1}{1+\beta}}-W_{t-k-1}^{\frac{1}{1+\beta}}\right|&\leq C \max\left\{e^{\frac{\alpha}{16}(t-k-1)}, e^{\frac{\alpha}{32}(t-k-1)}\right\}e^{-\frac{\alpha}{4}(t-k-1)}\\
    &\leq C e^{-\frac{3\alpha}{16}(t-k-1)},
\end{align*}
\begin{align*}
    \left|W_t^{\frac{1}{1+\beta}}W_{t-k-1}^{\frac{1}{1+\beta}}\right|\geq C e^{-\frac{3\alpha}{32}(t-k-1)}.
\end{align*}
This is to say on the event $\mathcal{D}_{t,k}$, we get $K_{t,k}\leq C e^{-\frac{3\alpha}{32}(t-k-1)}$. Moreover, recall that $\alpha\beta=\kappa(f)(1+\beta)b$, we get
\begin{align}
    \mathbb{\tilde{P}}_{\mu}|Y_{t,k}|\mathbf{1}_{\mathcal{D}_{t,k}}&\leq C e^{-\frac{\alpha}{1+\beta}t}e^{\frac{\alpha}{1+\gamma}t}e^{\frac{\gamma \alpha-\kappa(f)(1+\gamma)b}{1+\gamma}k}e^{-\frac{3\alpha}{32}(t-k-1)}\label{thm125}\\
    &\leq C e^{(\frac{\alpha}{1+\gamma}-\frac{\alpha}{1+\beta})(t-k)}e^{-\frac{3\alpha}{32}(t-k-1)}.\nonumber
\end{align}

Now we take $\gamma$ so close to $\beta$ such that $\frac{1}{1+\gamma}-\frac{1}{1+\beta}<\frac{1}{32}$, then
\begin{align}
     \mathbb{\tilde{P}}_{\mu}|Y_{t,k}|\mathbf{1}_{\mathcal{D}_{t,k}}\leq  e^{-\frac{\alpha}{16}(t-k)}.\label{thm122}
\end{align}
At last, combining \eqref{thm121} and \eqref{thm122}, we get
$$\left|\mathbb{\tilde{P}}_{\mu}e^{i\theta I_t}-\mathbb{\tilde{P}}_{\mu}e^{i\theta\tilde{I}_t}\right|\leq C t^{-\delta},$$
for some $C,\delta>0$, which tend to zero, as $t\rightarrow \infty$.

Next, we will prove $J_t \rightarrow^d 0$ as $t\rightarrow \infty$, this is to say
\begin{align*}
    \left|\mathbb{\tilde{P}}_{\mu}e^{i\theta J_t}-1\right|\rightarrow 0.
\end{align*}

Let $\mathcal{E}_t:=\{\|X_t\|>\|\mu\|t^{-1/2}e^{\alpha t}\}$. Using Lemma \ref{lemma2_13}, we get
\begin{align}
    \mathbb{\tilde{P}}_{\mu}(\mathcal{E}^c_t)\leq C t^{-\delta}, \quad t\geq0,\label{Theorem123}
\end{align}
    for some $C,\delta>0$. Hence $\left|\mathbb{\tilde{P}}_{\mu}e^{i\theta J_t}-1\right|\mathbf{1}_{\mathcal{E}^c_{t,k}}\leq 2\mathbb{\tilde{P}}_{\mu}(\mathcal{E}^c_t)\leq Ct^{-\delta}$.

On the event $\mathcal{E}_t$, by using the element inequality $|e^{i z}-1|\leq |z|$,
\begin{align*}
    \left|\mathbb{\tilde{P}}_{\mu}e^{i\theta J_t}-1\right|&\leq|\theta| \mathbb{\tilde{P}}_{\mu}\left|\sum_{k=\lceil t-\ln t \rceil}^t \left(t\|X_t\|\right)^{-\frac{1}{1+\beta}}M_k^t[f]\right|\\
    &\leq t^{-\frac{1}{2(1+\beta)}}e^{-\frac{\alpha}{1+\beta}t}\mathbb{\tilde{P}}_{\mu}\left|\sum_{k=\lceil t-\ln t \rceil}^t M_k^t[f]\right|.
\end{align*}
By the triangle inequality and Corollary \ref{cor27}, for any $\gamma\in(0,\beta)$, we know that
\begin{align*}
    \mathbb{\tilde{P}}_{\mu}\left|\sum_{k=\lceil t-\ln t \rceil}^t M_k^t[f]\right|&\leq \sum_{k=\lceil t-\ln t \rceil}^t \|M_k^t[f]\|_{1+\gamma}\leq\sum_{k=\lceil t-\ln t \rceil}^t C_{\gamma}e^{\frac{\alpha}{1+\gamma}t}e^{\frac{\gamma \alpha-\kappa(f)(1+\gamma)b}{1+\gamma}k}\\
    &=\sum_{k=\lceil t-\ln t \rceil}^t C_{\gamma}e^{\frac{\alpha}{1+\gamma}(t-k)}e^{\frac{\alpha}{1+\beta}k}\leq C_{\gamma}e^{\frac{\alpha}{1+\beta}}t^{-\alpha(\frac{1}{1+\beta}-\frac{1}{1+\gamma})}.
\end{align*}
Hence, on the event $\mathcal{E}_t$
\begin{align*}
    \left|\mathbb{\tilde{P}}_{\mu}e^{i\theta J_t}-1\right|\leq C_{\gamma}t^{-\alpha(\frac{1}{1+\beta}-\frac{1}{1+\gamma})} t^{-\frac{1}{2(1+\beta)}}
\end{align*}
We take $\gamma$ close enough to $\beta$ such that $\alpha(\frac{1}{1+\gamma}-\frac{1}{1+\beta})$ smaller than $\frac{1}{2(1+\beta)}$ and combine with \eqref{Theorem123}, we proved that $J_t \rightarrow^d 0$ as $t\rightarrow \infty$.

\subsection{Proof of Theorem \ref{Theorem11}}
For any $f\in \mathcal{P}$, $f=\sum_{|p|\geq \kappa(f)}\langle f,\phi_p\rangle_\varphi \phi_p$. Therefore, we have the decomposition.
\begin{align*}
    &e^{-(\alpha-\kappa(f)b)t}\langle f,X_t\rangle=I_t+J_t\\
    &=e^{-(\alpha-\kappa(f)b)t}\sum_{|p|= \kappa(f)}\langle f,\phi_p\rangle_\varphi \langle \phi_p,X_t\rangle+e^{-(\alpha-\kappa(f)b)t}\sum_{|p|> \kappa(f)}\langle f,\phi_p\rangle_\varphi \langle \phi_p,X_t\rangle
\end{align*}
By Lemma \ref{lemma26}, we easily get that
\begin{align*}
    I_t \rightarrow \sum_{|p|=\kappa(f)}\langle f, \phi_p\rangle_{\varphi} H_{\infty}^p  \quad as~ t\rightarrow \infty
\end{align*}
$\mathbb{P}_{\mu}$-as and in $L^{1+\gamma}(\mathbb{P}_{\mu})$, for any $0< \gamma<\beta$. Thus, we just prove that $J_t$ tend to zero by this two ways.

{\em Convergence in $L^{1+\gamma}(\mathbb{P}_{\mu})$}:
Let
\begin{align*}
    \tilde{f}=\sum_{|p|> \kappa(f)}\langle f,\phi_p\rangle_\varphi \langle \phi_p,X_t\rangle
\end{align*}
By the definition of $\kappa(f)$, we know that $\kappa(\tilde{f})\geq \kappa(f)+1$. What's more, $J_t=e^{-(\alpha-\kappa(f)b)t}\langle \tilde{f},X_t\rangle$.
Using Corollary \ref{cor27}, for some constant $C$, we get

1) when $\alpha\gamma>\kappa(\tilde{f})(1+\gamma)b$
\begin{align*}
    \|J_t\|_{1+\gamma}\leq C e^{-(\alpha-\kappa(f)b)t}e^{(\alpha-\kappa(\tilde{f})b)t}\leq C  e^{-(\alpha-\kappa(f)b)t}e^{(\alpha-(\kappa(f)+1)b)t}=C e^{-bt}
\end{align*}

2) when $\alpha\gamma=\kappa(\tilde{f})(1+\gamma)b\geq (\kappa(f)+1)(1+\gamma)b$
\begin{align*}
     \|J_t\|_{1+\gamma}\leq C t e^{-(\alpha-\kappa(f)b)t}e^{\frac{\alpha}{1+\gamma}t}\leq C t e^{-bt}
\end{align*}

3) when $\alpha\gamma>\kappa(\tilde{g})(1+\gamma)b\geq (\kappa(g)+1)(1+\gamma)b$
\begin{align*}
    \|J_t\|_{1+\gamma}\leq C e^{-(\alpha-\kappa(g)b)t}e^{\frac{\alpha}{1+\gamma}t}\leq C e^{-bt}
\end{align*}
In conclusion, $\lim_{t\rightarrow \infty}\|J_t\|=0$.

{\em Almost sure convergence}: For each $a\in \mathbb{R}$, let $\{M_t^{f,a},t\geq0\}$ be the martingale in \eqref{defmartingale} and let
$$L_t^{f,a}:=\int_0^t e^{-(\alpha-ab)s}\langle \bar{f},X_s\rangle ds.$$
Thus, $M_t^{\tilde{f},a}=e^{-(\alpha-ab)t}\langle \tilde{f}, X_t\rangle-L_t^{\tilde{f},a}$. When take $\alpha=\kappa(f)+\frac{1}{2}$, we get

%We take $a=\kappa(g)+\frac{1}{2}$ and $f=\tilde{g}$ into this martingale and get
\begin{align*}
    J_t=e^{-\frac{1}{2}}M_t^{\tilde{f},a}+e^{-\frac{1}{2}}L_t^{\tilde{f},a}.
\end{align*}
Equivalently, we will prove that
\begin{align*}
    e^{-\frac{1}{2}bt}M_t^{\tilde{f},a}\rightarrow 0, \quad e^{-\frac{1}{2}bt}L_t^{\tilde{f},a}\rightarrow 0 \quad as~t\rightarrow \infty
\end{align*}
$\mathbb{P}_{\mu}$-as.
As $\kappa(L\tilde{f}+ab\tilde{f})=\kappa(\tilde{f})\geq \kappa(f)+1$, following the similar way to prove the case $L^{1+\gamma}(\mathbb{P}_{\mu})$-convergence, we get there exists some $C, \delta>0$ such that for any $t\geq 0$
\begin{align*}
    \|e^{-(\alpha-ab)t}\langle \tilde{f},X_t\rangle)\|_{1+\gamma}\leq C e^{-\delta t},\quad \|e^{-(\alpha-ab)t}\langle L\tilde{f}+ab\tilde{f},X_t\rangle\|_{1+\gamma}\leq C e^{-\delta t}.
\end{align*}
Therefore, $|L_t^{\tilde{f},a}|\leq Y_t^{\tilde{f},a}:=\int_0^t e^{-(\alpha-ab)s}|\langle L\tilde{f}+ab\tilde{f},X_s\rangle|ds$ satisfying that, for ant $t\geq 0$,
\begin{align*}
    \|L_t^{\tilde{f},a}\|_{1+\gamma}\leq\|Y_t^{\tilde{f},a}\|_{1+\gamma}\leq \int_0^t \|e^{-(\alpha-ab)s}\langle L\tilde{f}+ab\tilde{f},X_s\rangle\|_{1+\gamma}ds\leq C \int_0^t e^{-\delta s}ds\leq\frac{C}{\delta}.
\end{align*}
As $Y_t^{\tilde{f},a}$ is positive increasing, we get that it convergence some random variable $Y_{\infty}^{\tilde{f},a}$ almost sure and in $L^{1+\gamma}(\mathbb{P}_{\mu})$. This is to say
\begin{align*}
    \lim_{t\rightarrow \infty}e^{-\frac{1}{2}bt}|L_t^{\tilde{f},a}|\leq  \lim_{t\rightarrow \infty}e^{-\frac{1}{2}bt}|Y_t^{\tilde{f},a}|=0.
\end{align*}
On the other hand, the martingale $M_t^{\tilde{f},a}$ satisfies
\begin{align*}
    \|M_t^{\tilde{f},a}\|_{1+\gamma}\leq  \|e^{-(\alpha-ab)t}\langle \tilde{f},X_t\rangle)\|_{1+\gamma}+\|L_t^{\tilde{f},a}\|_{1+\gamma}\leq C,\quad t\geq 0.
\end{align*}
for some constant $C$, which implies that $\lim_{t\rightarrow\infty} e^{-\frac{1}{2}bt}M_t^{\tilde{f},a}=0$. we complete our proof.
\subsection{Proof of Theorem \ref{Theorem13}}
The proof of Theorem \ref{Theorem13} is similar with that of Theorem \ref{Theorem12} or much easier.
     For each $f\in \mathcal{P}$, we write
    \begin{align*}
        (\|X_t\|)^{-\frac{1}{1+\beta}}\langle f,X_t\rangle&=\sum_{k=0}^{\lfloor t-\ln t \rfloor} (\|X_t\|)^{-\frac{1}{1+\beta}}M_k^t[f]+\sum_{\lceil t-\ln t \rceil}^t (\|X_t\|)^{-\frac{1}{1+\beta}}M_k^t[f]\\
        &=I_t+J_t.
    \end{align*}
    We denote
    $$\tilde{I}_t=\sum_{k=0}^{\lfloor t-\ln t \rfloor}\frac{M_k^t[f]}{( e^{\alpha(k+1)}\|X_{t-k-1}\|)^{\frac{1}{1+\beta}}}.$$
    Taking $\theta_k=( e^{\alpha})^{-\frac{1}{1+\beta}} \theta $ and $n={\lfloor t-\ln t \rfloor}$ in Corollary \ref{corollary31}, then we get
    \begin{align*}
        \left|\mathbb{\tilde{P}}_{\mu}e^{i\theta\tilde{I}_t}-\exp\left(\theta^{1+\beta}\sum_{k=0}^{\lfloor t-\ln t \rfloor}m_k[f]\right)\right|\leq C \frac{1}{t^{\delta}},
    \end{align*}
    for some $C,\delta>0$. Hence We obtain that $\tilde{I}_t\rightarrow\eta_2$ as $t\rightarrow \infty$, by using  \eqref{msmallcase}.

    Therefore, to prove the theorem ,we just to prove $\left|\mathbb{\tilde{P}}_{\mu}e^{i\theta I_t}-\mathbb{\tilde{P}}_{\mu}e^{i\theta\tilde{I}_t}\right|\rightarrow 0$ and $J_t\rightarrow^d 0$.

    We denote
    \begin{align*}
        Y_{t,k}:=\exp\left(i\theta\frac{M_k^t[f]}{( e^{\alpha(k+1)}\|X_{t-k-1}\|)^{\frac{1}{1+\beta}}}\right)-\exp\left(i\theta\frac{M_k^t[f]}{\left(\|X_t\|\right)^{\frac{1}{1+\beta}}}\right)
    \end{align*}
With completely same method before \eqref{thm125} in the proof of Theorem \ref{Theorem12}, we can prove that
\begin{align*}
    \mathbb{\tilde{P}}_{\mu}|Y_{t,k}|\leq C( e^{-\frac{\alpha}{1+\beta}t}e^{\frac{\alpha}{1+\gamma}t}e^{\frac{\gamma \alpha-\kappa(f)(1+\gamma)b}{1+\gamma}k}e^{-\frac{3\alpha}{32}(t-k-1)}+ e^{-\delta(t-k-1)}), \quad t\geq 0.
\end{align*}
Moreover, on the case $\alpha\beta<\kappa(f)(1+\beta)b$,  $\frac{\alpha}{1+\gamma}-\frac{\alpha}{1+\beta}< - \frac{\alpha\gamma-\kappa(f)(1+\gamma)b}{1+\gamma}$, this is to say,
\begin{align*}
    \mathbb{\tilde{P}}_{\mu}|Y_{t,k}|\leq C e^{-(t-k)}
\end{align*}
Then we can prove $\left|\mathbb{\tilde{P}}_{\mu}e^{i\theta I_t}-\mathbb{\tilde{P}}_{\mu}e^{i\theta\tilde{I}_t}\right|\rightarrow 0$, by following the lines of the proof in the critical case.

To prove $J_t\rightarrow^d 0$, We let $\mathcal{E}_t:=\{\|X_t\|>\|\mu\|e^{(\alpha-\epsilon )t}\}$ instead, for any $\epsilon>0$, we know that, see Lemma \ref{lemma28}, there exists
$C,\delta >0$ such that  $\mathbb{\tilde{P}}_{\mu}(\mathcal{E}_t^c)\leq C e^{-\delta t}$. What's more, for any $0< \gamma <\beta$, we let $q=\frac{\alpha}{1+\beta}-\frac{\alpha}{1+\gamma} - \frac{\alpha\gamma-\kappa(f)(1+\gamma)b}{1+\gamma}>0$, then by Corollary \ref{cor27}, we have
\begin{align*}
    \|M_k^t[f]\|\leq \|M_k^t[f]\|_{1+\gamma}\leq \tilde{C}_1 e^{\frac{\alpha}{1+\gamma}t}e^{\frac{\gamma \alpha-\kappa(f)(1+\gamma)b}{1+\gamma}k}\leq \tilde{C}_1 e^{-qk}e^{\frac{\alpha}{1+\gamma}(t-k)}e^{\frac{\alpha}{1+\beta}k}
\end{align*}
Then we get $  \left|\mathbb{\tilde{P}}_{\mu}e^{i\theta J_t}-1\right|\rightarrow 0$ as $ t\rightarrow \infty$, by following the similar way in critical case.

\appendix
\section{}

\subsection{}
    In this subsection, we present two analytic results which are useful.
\begin{lem}
\label{lem: estimate of exponential remaining}
    Suppose that $z\in \mathbb C_+:= \{x+iy: x \in [0,\infty), y \in \mathbb R\}$. Then
\begin{equation}
\label{eq: estimate of exponential remaining}
    \Big|e^{-z} - \sum_{k=0}^n \frac{(-z)^k}{k!} \Big|
    \leq \frac{|z|^{n+1}}{(n+1)!} \wedge \frac{2|z|^{n}}{n!}, \quad n\in \mathbb N_0.
\end{equation}
\end{lem}
\begin{proof}
    Notice that $|e^{-z}| = e^{- \operatorname{Re} z} \leq 1$.
    Therefore, according to \cite[Theorem 7.20.]{Rudin1987Real},
\begin{equation}
    |e^{-z} - 1| = \Big| \int_0^1 e^{-\theta z} z d\theta\Big|
    \leq |z|.
\end{equation}
    Also, notice that $|e^{-z} - 1| \leq |e^{-z}|+1 \leq 2$.
    So we have \eqref{eq: estimate of exponential remaining} is true when $n = 0$.
    Now, suppose that \eqref{eq: estimate of exponential remaining} is true when $n = m$ for some $m \in \mathbb N_0$.
    According to \cite[Theorem 7.20.]{Rudin1987Real},
\begin{align}
    &\Big|e^{-z} - \sum_{k=0}^{m+1} \frac{(-z)^k}{k!}\Big|
    = \Big| \int_0^1\Big(e^{-\theta z} - \sum_{k=0}^m \frac{(-\theta z)^k}{k!} \Big) z d\theta \Big|
    %\\&\quad \leq  \Big(\int_0^1 \frac{|\theta z|^{m+1}}{(m+1)!} z d\theta\Big) \wedge \Big(\int_0^1 \frac{2|\theta z|^{m}}{m!} z d\theta\Big)
    \\&\quad \leq  \Big(\int_0^1 \frac{|\theta z|^{m+1}}{(m+1)!} |z| d\theta\Big) \wedge \Big(\int_0^1 \frac{2|\theta z|^{m}}{m!} |z| d\theta\Big)
    = \frac{|z|^{m+2}}{(m+2)!} \wedge \frac{2|z|^{m+1}}{(m+1)!},
\end{align}
    which says that \eqref{eq: estimate of exponential remaining} is true for $n = m + 1$.
    Finally, use induction.
\end{proof}

\begin{lem}
\label{lem: extension lemma for branching mechanism}
    Suppose that  $\nu$ is a measure on $(0,\infty)$ such that $\int_{(0,\infty)} (u \wedge u^2) \nu(du)< \infty$. Then the following function $h: \mathbb C_+ \to \mathbb C$ is well defined:
\begin{equation}
    h (z) = \int_{(0,\infty)} (e^{-zu} - 1 + zu) \nu(du), \quad z \in \mathbb C_+.
\end{equation}
    Moreover, $h$ is continuous on $\mathbb C_+$ and is holomorphic on $\mathbb C_+^0:=\{x+iy:x \in (0,\infty), y\in \mathbb R\}$ with deriavative
\begin{equation}
\label{eq: deriavetive of the Poission part}
    h'(z) = \int_{(0,\infty)}(1- e^{-uz})u \nu(du).
\end{equation}
\end{lem}
\begin{proof}
    From Lemma \ref{lem: estimate of exponential remaining}, we know that $h$ is well defined on $\mathbb C_+$.
    According to \cite[Theorem 3.2. \& Theorem 3.5.]{SchillingSongVondracek2010Bernstein}, \eqref{eq: deriavetive of the Poission part} defines a continuous function $h'$ on $\mathbb C_+$ which is holomorphic on $\mathbb C_+^0$.
    Let $z_0, z \in \mathbb C_+$.
    Let $\gamma: [0,1]\mapsto \mathbb C_+$ be a $C^1$ path with $\gamma(0) = z_0$ and $\gamma(1) = z$.
    Notice that, according to Lemma \ref{lem: estimate of exponential remaining},
\begin{align}
    &\int_0^1 \int_0^\infty |1-e^{-u\gamma(\theta)}|u~\nu(du)~d\theta
    \\ &= \int_0^1~d\theta~ \Big( \int_0^1 |1-e^{-u\gamma(\theta)}|u~\nu(du) + \int_1^\infty |1-e^{-u\gamma(\theta)}|u~\nu(du) \Big)
    \\ &\leq \int_0^1~d\theta~ \Big( |\gamma(\theta)|\int_0^1 u^2~\nu(du) + 2\int_1^\infty u~\nu(du) \Big)
    < \infty.
\end{align}
    Therefore, according to Fubini's theorem,
\begin{align}
\label{eq: path integration representation of h}
    \int_\gamma h'(z)dz
    &=\int_0^1 h'(\gamma(\theta))\gamma'(\theta) ~d\theta
    = \int_0^1 \int_0^\infty (1-e^{-u\gamma(
    \theta)}) u~\nu(du)~ \gamma'(\theta) ~d\theta
    \\&= \int_0^\infty \int_0^1 (1-e^{-u\gamma(
    \theta)})~ \gamma'(\theta) u ~d\theta~ \nu(du)
    = h(z) - h(z_0).
\end{align}
    The rest of the proof follows the arguement in \cite[Section 10.14]{Rudin1987Real}.
\end{proof}

{ \color{blue}
\subsection{}
	In this subsection, we give some analytical facts that will be used in this paper.
	For each $z\in \mathbb C\setminus (-\infty,0]$ define
\[
	\log z := \log |z| + i \arg z
\]
	where $\arg z \in (-\pi,\pi)$ is uniquely determined by
\[
	z = |z|e^{i \arg z}.
\] 	
	For each $z\in \mathbb C\setminus (-\infty,0]$ and $\gamma \in \mathbb C$ define
\[
	z^\gamma := e^{\gamma \log z}.
\]
	Then it is known, see \cite[Theorem 6.1]{SteinShakarchi2003Complex} for example, $z\mapsto \log z$ is holomophic on $\mathbb C\setminus (-\infty,0]$.
	Therefore, so is $z\mapsto z^\gamma$, for each $\gamma \in \mathbb C$.
	For convention, define $0^\gamma := \mathbf 1_{\gamma = 0}$ for each $\gamma \in \mathbb C$.
    From this definition we can verify that $(z_1z_0)^\gamma = z_1^\gamma z_0^\gamma$ provided $\arg (z_1z_0)=\arg z_1 + \arg(z_0)$.


    Recall $\Gamma$ the Gamma function defined by
\begin{equation}
    \Gamma (x) := \int_0^\infty t^{x-1} e^{-t}dt,
    \quad x>0.
\end{equation}
	It is known, see \cite[Theorem 6.1.3 and its following remark]{SteinShakarchi2003Complex} for example, function $\Gamma$ has an unique analytic extension on $\mathbb C\setminus\{0, -1,-2,\dots\}$ and that
\[
	\Gamma(z+1) = z \Gamma(z),\quad z\in \mathbb C\setminus\{0, -1,-2,\dots\}.
\]
	Using this recursively, one can verify that
\begin{align}
\label{eq: definition of Gamma function}
    \Gamma(x)
    := \int_0^\infty t^{x-1} \Big(e^{-t} - \sum_{k=0}^{n-1} \frac{(-t)^k}{k!}\Big) dt,
    \quad -n< x< -n+1, n\in \mathbb N.
\end{align}

    Fix a $\beta \in (0,1)$.
    Using \cite[Theorem 3.2. \& Theorem 3.5.]{SchillingSongVondracek2010Bernstein}, we can verify that
\begin{equation}
    z^{\beta}
	= \int_0^\infty (e^{-zy}-1) \frac{dy}{\Gamma(-\beta)y^{1+\beta}},
    \quad z\in \mathbb C_+,
\end{equation}
	by showing that the both sides
\begin{itemize}
\item
    are extension of the real function $x\mapsto x^{\beta}$ on $[0,\infty)$;
\item
    are holomorphic on $\mathbb C_+^0:= \{x+iy:x\in (0,\infty), y\in \mathbb R\}$;
\item
    are continuous on $\mathbb C_+ :=\{x+iy: x\in [0,\infty), y\in \mathbb R\}$.
\end{itemize}
    Similarly, using Lemma \ref{lem: extension lemma for branching mechanism} in the Appendix, we can verify that
\begin{equation}
\label{eq: stable branching on C+}
    z^{1+\beta}
    = \int_0^\infty (e^{-zy}-1+zy)\frac{dy}{\Gamma(-1-\beta)y^{2+\beta}},
    \quad z\in \mathbb C_+.
\end{equation}
    Moreover, according to \eqref{eq: path integration representation of h}, for any $C^1$ path $\gamma:[0,1]\to \mathbb C_+$ ,
\begin{align}
\label{eq: integration formula for 1+beta-th power of z}
    &\gamma(1)^{1+\beta} - \gamma(0)^{1+\beta}
    = \int_0^1 \gamma'(\theta)d\theta \int_{(0,\infty)}(1-e^{-\gamma(\theta)y})\frac{ydy}{\Gamma(-1-\beta)y^{2+\beta}}
    \\&=\int_0^1 \gamma'(\theta)d\theta \int_{(0,\infty)}(1-e^{-\gamma(\theta)y})\frac{(-1-\beta)dy}{\Gamma(-\beta)y^{1+\beta}}
    = \int_0^1 (1+\beta) \gamma(\theta)^{\beta} \gamma'(\theta)d\theta.
\end{align}
    This indicates that the derivative of $z\mapsto z^{1+\beta}$ is $z\mapsto (1+\beta)z^{\beta}$ on $\mathbb C^0_+$.
\begin{lem}
\label{lem: Lip of power function}
    There is a constant $C_{\ref{lem: Lip of power function}}$ such that for any $z_0,z_1 \in \mathbb C_+$,
\begin{equation}
\label{eq: Lip of power function}
    |z_0^{1+\beta} - z_1^{1+\beta}|
    \leq C_{\ref{lem: Lip of power function}}(|z_0|^{\beta}+|z_1|^{\beta})|z_0 - z_1|.
\end{equation}
    In particular, for each $z\in \mathbb C_+$, we have
$
    |z^{1+\beta}|
    \leq  C_{\ref{lem: Lip of power function}} |z|^{1+\beta}.
$

\end{lem}
\begin{proof}
    Notice that
\begin{align}
\label{eq: upper bound for beta power of z}
    &|z^{\beta}|
    = \Big|\int_0^\infty (e^{-zy}-1) \frac{dy}{\Gamma(-\beta)y^{1+\beta}}\Big|
    \\&\leq \frac{1}{|\Gamma(-\beta)|}\int_0^\infty \frac{2\wedge (|z|y)}{y^{1+\beta}}dy
    = \frac{|z|^{\beta}}{|\Gamma(-\beta)|}\int_0^\infty \frac{2\wedge t}{t^{1+\beta}}dt =: c |z|^{\beta},
    \quad z\in \mathbb C_+.
\end{align}
    Define a path $\gamma: [0,1] \to \mathbb C_+$ such that
\[
    \gamma(\theta)
    = z_0 (1-\theta) + \theta z_1,
    \quad \theta \in [0,1].
\]
    Then, according to \eqref{eq: integration formula for 1+beta-th power of z}, we have
\begin{align}
    |z_0^{1+\beta} - z_1^{1+\beta}|
    &\leq (1+\beta) \int_0^1 |\gamma(\theta)^{\beta}|\cdot |\gamma'(\theta)|d\theta
    \leq (1+\beta) c_0 \sup_{\theta \in [0,1]} |\gamma(\theta)|^{\beta} \cdot |z_1-z_0|
    \\&\leq (1+\beta) c_0 ( |z_1|^{\beta}+|z_0|^{\beta} ) |z_1-z_0|.
    \qedhere
\end{align}
\end{proof}
}

\subsection{}
\label{seq: complex Feynman-Kac transform}
%new added
    In this subsection we give a version of the Feynman-Kac formula with complex values.
%end new added
    Suppose that $\{(\xi_t)_{t \in [r,\infty)}; (\Pi_{r,x})_{r\in [0,\infty), x\in E}\}$ is a (possibly non-homogeneous) Hunt process in a locally compact seperable metric space $E$.
    %For brevity, for each $0< s< t< \infty$ and each bounded complex valued measurable function $\rho$ on $[0,\infty) \times E$, we write
    Fix a time $t >0$.
    For each $0< s< t$ and each bounded complex valued measurable function $\rho$ on $[0,t) \times E$, we write
\begin{equation}
    H^{(\rho)}_{(s,t)}:= \exp\Big\{\int_s^t \rho(u,\xi_u) du\Big\}.
\end{equation}
    Notice that
\begin{equation}
\label{eq: crucial for Feynman-Kac}
    \frac{\partial}{\partial s} H^{(\rho)}_{(s,t)}= -H^{(\rho)}_{(s,t)}\rho(s,\xi_s),
    %\quad s> 0.
    \quad s\in (0,t).
\end{equation}
    %Suppose that $\beta$, $\rho$ are complex valued bounded measurable functions on $[0,\infty) \times E$; $F$ is a complex valued bounded measurable function on $E$.
    Suppose that $\beta$, $\rho$ are complex valued bounded measurable functions on $[0,t) \times E$; $F$ is a complex valued bounded measurable function on $E$.
%delete
    %Fix a time $t >0$.
%end delete
    Define
\begin{equation}
    g(r,x) := \Pi_{r,x}[ H_{(r,t)}^{(\beta+\rho)} F(\xi_t)],\quad r \in [0,t), x\in E.
\end{equation}
    Notice that
\begin{align}
    \Pi_{r,x} \Big[ \int_r^t | H_{(r,t)}^{(\beta)}\rho(s,\xi_s) H_{(s,t)}^{(\rho)} F(\xi_t)| ~ds \Big]
    \leq  \int_r^t e^{(t-r)\|\beta\|_\infty}e^{(t-s)\|\rho\|_\infty}\|\rho\|_\infty\|F\|_\infty ~ds
    < \infty.
\end{align}
    Therefore, from the Markov property of $\xi$, \eqref{eq: crucial for Feynman-Kac}, Fubini's theorem and above, we can verify that
\begin{align}
    &\Pi_{r,x} \Big[ \int_r^tH_{(r,s)}^{(\beta)}~(\rho g)(s,\xi_s)~ds \Big]
    =\Pi_{r,x} \Big[ \int_r^t H_{(r,s)}^{(\beta)}\rho(s,\xi_s) \Pi_{s,\xi_s}[ H_{(s,t)}^{(\beta+\rho)} F(\xi_t)]~ds \Big]
    \\&= \Pi_{r,x} \Big[ \int_r^t H_{(r,t)}^{(\beta)}\rho(s,\xi_s) H_{(s,t)}^{(\rho)} F(\xi_t) ~ds \Big]
    = \Pi_{r,x} [ H_{(r,t)}^{(\beta)}F(\xi_t)(H_{(r,t)}^{(\rho)} - 1)]
    \\&= g(r,x) - \Pi_{r,x} [ H_{(r,t)}^{(\beta)} F(\xi_t)].
\end{align}

\subsection{}
\label{sec: definition of characteristic exponent}
    According to Levy-Khintchine formula \cite[Theorem 8.1]{Sato1999Levy}, an $\mathbb R$-valued random variable $\{Y; P\}$ is infinitely divisible if and only if
\begin{equation}
\label{eq: characteristic function of an infinitely divisible random variable}
    P[e^{i\theta Y}]
    = \exp\Big( ia \theta - \frac{1}{2}\sigma^2 \theta^2 + \int_{\mathbb R}(e^{i\theta x} - 1 - i \theta x\mathbf 1_{|x|< 1}) \nu(dx)\Big),
    \quad \theta \in \mathbb R,
\end{equation}
    where $a\in \mathbb R$, $\sigma \in \mathbb R$ and $\nu$ is a measure on $\mathbb R\setminus\{0\}$ with $\int_{\mathbb R}(1\wedge x^2) \nu(dx)< \infty$.
    Suppose that $\{Y;P\}$ is an infinitely divisible random variable with characteristic function \eqref{eq: characteristic function of an infinitely divisible random variable}, then
\begin{equation}
\label{eq: Characteristic exponent}
    h(\theta)
    := ia \theta - \frac{1}{2}\sigma^2 \theta^2 + \int_{\mathbb R}(e^{i\theta x} - 1 - i \theta x\mathbf 1_{|x|< 1}) \nu(dx),
    \quad \theta \in \mathbb R,
\end{equation}
    is a continuous function.
    %In fact, for any compact $K \subset \mathbb R$, since due to Lemma \ref{lem: estimate of exponential remaining},
    In fact, for any compact $K \subset \mathbb R$, due to Lemma \ref{lem: estimate of exponential remaining},
\begin{equation}
\int_{\mathbb R}\sup_{\theta \in K} |e^{i\theta x} - 1 - i \theta x\mathbf 1_{|x|< 1}|~\nu(dx)
\leq (\sup_{\theta \in K} \theta^2)\int_{[0,1)} | x|^2 \nu(dx) + 2 \int_{[1,\infty)} \nu(dx) < \infty,
\end{equation}
    we can verify the continuity of $h$ from the dominant convergence theorem.
    Now, according to \cite[Proposition 1.9.1.]{Linde1986Probability},
    $h$ is the unique continuous function from $\mathbb R$ to $\mathbb C$ with $h(0) = 0$ and $P[e^{i\theta Y}] = \exp\big(h(\theta)\big)$.
    We refer to function $h$ the characteristic exponent of random variable $Y$ and write $\operatorname{Log} P[e^{i\theta Y}] = h(\theta)$.

\subsection{}

    Suppose that $\{Y;P\}$ is an infinitely divisible random variable with characteristic exponent \eqref{eq: Characteristic exponent}.
    Suppose that $P[|Y|]< \infty$.
    According to \cite[Corollary 25.8.]{Sato1999Levy}, $\int_{|x|>1} |x|\nu(dx) < \infty$.

    We claim that, $h$ is differentiable on $\mathbb R$.
    In fact, according to Lemma \ref{lem: estimate of exponential remaining}, we have for any compact $K \subset \mathbb R$,
\begin{align}
    &\int_{\mathbb R} \sup_{\theta \in K} \Big|\frac{\partial}{\partial \theta}(e^{i\theta x} - 1 - i \theta x\mathbf 1_{|x|< 1})\Big| \nu(dx)
    = \int_{\mathbb R} \sup_{\theta \in K} | e^{
    i\theta x} ix - ix \mathbf 1_{|x|< 1} | \nu(dx)
    \\&\leq(\sup_{\theta \in K}|\theta|) \int_{[0,1)}|x^2| \nu(dx)+\int_{[1,\infty)} |x|  \nu(dx)
    %< \infty,
    %\quad \epsilon > 0.
    < \infty.
\end{align}
    Therefore, according to \cite[Theorem A.5.2.]{Durrett2010Probability},
    we have that $h$ is differentiable on $\mathbb R$ with derivative
\begin{equation}
    h'(\theta) = ia - \sigma^2 \theta + i \int_{\mathbb R} (xe^{i\theta x} - x\mathbf 1_{|x|<1}) \nu(dx).
\end{equation}
    %Therefore, according to \cite[Theorem 3.2.4.]{Cuppens1975Decomposition}, we have that map $\theta \mapsto \mathbf P[e^{i\theta Y}]$ from $\mathbb R$ to $\mathbb C$ is differentiable with
    Now, according to \cite[Theorem 3.2.4.]{Cuppens1975Decomposition}, we have that map $\theta \mapsto \mathbf P[e^{i\theta Y}]$ from $\mathbb R$ to $\mathbb C$ is differentiable with
\begin{equation}
\label{eq: derivative of characteristic exponent}
    iP[Y e^{i\theta Y}]
    =\frac{d}{d \theta} P[e^{i\theta Y}]
    = P[e^{i\theta Y}] \frac{d}{d \theta} h(\theta),
    \quad \theta \in \mathbb R.
\end{equation}

\subsection{}
\label{sec: definition of superprocess}
    We say $X=\{(X_t)_{t\geq 0}; (\mathbf P_\mu)_{\mu \in \mathcal M^1_E}\}$ is a $(\xi,\psi)$-superprocess if
\begin{itemize}
\item
    The sample space $E$ is a locally compact separable metric space.
    Denote by $\mathcal M_E^1$ the collection of all the finite measures on $E$ equipped with weak topology.
\item
    The spatial motion $\xi=\{(\xi_t)_{t\geq 0};(\Pi_x)_{x\in E}\}$ is an $E$-valued Hunt process with its lifetime denoted by $\zeta$.
\item
    %the branching mechanism $\psi: E\times[0,\infty) \to \mathbb R$ is given by
    The branching mechanism $\psi: E\times[0,\infty) \to \mathbb R$ is given by
\begin{equation}
\label{eq: branching mechanism}
    \psi(x,z)=
    - \beta(x) z + \alpha (x) z^2 + \int_{(0,\infty)} (e^{-zy} - 1 + zy) \pi(x,dy).
\end{equation}
    where $\beta \in \mathscr B_b(E)$, $\alpha \in \mathscr B^+_b(E)$ and $\pi(x,dy)$ is a kernel from $E$ to $(0,\infty)$ such that $\sup_{x\in E} \int_{(0,\infty)} (y\wedge y^2) \pi(x,dy) < \infty$.
\item
    $X=\{(X_t)_{t\geq 0}; (\mathbf P_\mu)_{\mu \in \mathcal M^1_E}\}$ is an $\mathcal M^1_E$-valued Hunt process with is transition probability determined by
\begin{align}
    \mathbf P_\mu [e^{-X_t(f)}] = e^{-\mu(V_tf)},
    \quad t\geq 0, \mu \in \mathcal M_E^1, f\in \mathscr B^+_b(E),
\end{align}
    where for each $f\in b\mathscr B_E$, the function $(t,x)\mapsto V_tf(x)$ on $[0,\infty) \times E$ is the unique locally bounded positive solution to the equation
\begin{align}\label{eq:FKPP_in_definition}
    V_tf(x) + \Pi_x \Big[  \int_0^{t\wedge \zeta} \psi(\xi_s,V_{t-s}f)ds \Big]
    = \Pi_x [ f(\xi_t)\mathbf 1_{t<\zeta} ],
    \quad t \geq 0, x \in E.
\end{align}
\end{itemize}
    We refer our reader to \cite{Li2011Measure-valued} for more discussion about the definition and the existence of superprocesses.
    To avoid triviality, we assume that $\psi(x,z)\neq -\beta(x)z$ for some $x \in E$ and $z \geq 0$; and that $\Pi_x(\zeta>t)>0$ for each $x \in E$ and $t \geq 0$.

\subsection{}
\label{sec: extension of branching mechanism}
    Let $X$ be the superprocess defined in Subsection \ref{sec: definition of superprocess}.
    Notice that, the branching mechanism $\psi$ can be extended as a map from $E \times \mathbb C_+$ to $\mathbb C$ using \eqref{eq: branching mechanism}.
    Define
\begin{equation}
    \psi'(x,z):= - \beta(x) + 2\alpha(x) z + \int_{(0,\infty)} (1-e^{-zy})y\pi(x,dy),
    \quad x\in E, z\in \mathbb C_+.
\end{equation}
    Then according to Lemma \ref{lem: extension lemma for branching mechanism}, for each $x \in E$, $z \mapsto \psi(x,z)$ is a holomorphic function on $\mathbb C_+^0$ with deriavetive $z \mapsto \psi'(x,z)$.
    Define $\psi_0(x,z) := \psi(x,z)+ \beta(x)z $ and $\psi'_0(x,z) := \psi'(x,z) + \beta(x)$.

\subsection{}
    Let $X$ be the superprocess defined in Subsection \ref{sec: definition of superprocess}.
    %Denote by $\mathbb W$ the space of $\mathcal M_E^1$-valued c\`{a}dl\`{a}g paths whis its conanical path denoted by $(W_t)_{t\geq 0}$.
    Denote by $\mathbb W$ the space of $\mathcal M_E^1$-valued c\`{a}dl\`{a}g paths with its conanical path denoted by $(W_t)_{t\geq 0}$.
    We say $X$ is \emph{non-persistent} if $\mathbf P_{\delta_x}(\|X_t\|= 0) > 0$ for all $x\in E$ and $t> 0$.
    Suppose that $(X_t)_{t\geq 0}$ is non-persistent, then according to \cite[Section 8.4]{Li2011Measure-valued},
    there is a family of measures $(\mathbb N_x)_{x\in E}$ on $\mathbb W$ such that
\begin{itemize}
\item
    $\mathbb N_x (\forall t \geq 0, \|W_t\|=0) =0$;
\item
    $\mathbb N_x(\|W_0 \|\neq 0) = 0$;
\item
    For any $\mu \in \mathcal M_E^1$, if $\mathcal N$ is a Poisson random measure defined on some probability space
    with intensity $\mathbb N_\mu(\cdot):= \int_E \mathbb N_x(\cdot )\mu(dx)$,
    then the superprocess $\{X;\mathbf P_\mu\}$ can be realized by $\widetilde X_0 := \mu$ and $\widetilde X_t(\cdot) := \mathcal N[W_t(\cdot)]$ for each $t>0$.
\end{itemize}
    We refer to $(\mathbb N_x)_{x\in E}$ the Kuznetsov measures of $X$.
\subsection{}
\label{sec: definition of vf}
    Let $X$ be the superprocess defined in Subsection \ref{sec: definition of superprocess}.
    Define the mean semigroup
\begin{equation}
    P_t^\beta f(x)
    := \Pi_{x}[e^{\int_0^t \beta(\xi_s)ds}f(\xi_t) \mathbf 1_{t< \zeta}],
    \quad t\geq 0, x\in E, f\in b\mathscr B^+_E.
\end{equation}
    Assume that the superprocess $X$ is non-presistent with its Kuznestov measure denoted by $(\mathbb N_x)_{x\in E}$.
    It is know from \cite[Proposition 2.27]{Li2011Measure-valued} and \cite[Theorem 2.7]{Kyprianou2014Fluctuations} that for any $t > 0$, $\mu \in \mathcal M_E^1$ and $f\in b\mathscr B^+_E$,
\begin{equation}
\label{eq: mean formula for superprocesses}
    \mathbb N_{\mu}[W_t(f)]
    =\mathbf P_{\mu}[X_t(f)]=\mu(P^\beta_t f).
\end{equation}
    Define
\begin{align}
    L_1(\xi)
    &:= \{f\in \mathscr B_E: \forall x\in E, t\geq 0, \quad \Pi_x[|f(\xi_t)|]< \infty\},
    \\L_2(\xi)
    &:= \{f \in \mathscr B_E: |f|^2 \in L_1(\xi)\}.
\end{align}
    Let $f\in L_1(\xi), t >0$ and $x\in E$.
    Evaluating \eqref{eq: mean formula for superprocesses} by monotonicity and linearity, we have
\begin{equation}
    \mathbb N_x[W_t(f)]
    =\mathbf P_{\delta_x}[X_t(f)]=P^\beta_t f(x) \in \mathbb R.
\end{equation}
    Notice that, from the branching property of the superprocess $X$, $\{X_t(f); \mathbf P_{\delta_x}\}$ is an infinitely divisible random variable with finite moment.
    Denote by $U_t(\theta f)(x) := \operatorname{Log} \mathbf P_{\delta_x}[e^{i \theta X_t(f)}]$  the charateristic exponent of random variable $\{X_t(f); \mathbf P_{\delta_x}\}$.
    According to Campbell's formula, see \cite[Theorem 2.7]{Kyprianou2014Fluctuations} for example, for each $\theta \in \mathbb R$,
$   \mathbf P_{\delta_x} [e^{i\theta X_t(f)}]
    = \exp(\mathbb N_x[ e^{i\theta W_t(f)} - 1]).
$
    Notice that $\theta \mapsto \mathbb N_x[e^{i\theta W_t(f)} - 1]$ is a continuous function.
    Also notice that $\mathbb N_x[e^{i\theta W_t(f)} - 1] = 0$ if $\theta = 0$.
    Therefore, according to the discussion in Subsection \ref{sec: definition of characteristic exponent}, we have
\begin{equation}
\label{eq: N and characteristic exponent}
    U_tf(x) = \mathbb N_x[e^{i W_t(f)} - 1]
    \quad t \geq 0.
\end{equation}

\subsection{}
    Let $X$ be the non-persistent superprocess discussed in Subsection \ref{sec: definition of vf}.
    In this subsection, we want to show that if $f\in L^2(\xi)$ then the following expectation
\[
    \Pi_x\Big[\int_0^t \psi(\xi_s,- U_{t-s}f)ds\Big]
    \in \mathbb C
\]
    is well defined.

    Firstly, fix $t > 0,x\in E$ and $f\in L^1(\xi)$.
    Noticing that
\[
     e^{\operatorname{Re} U_tf(x)}
    = |e^{U_tf(x)}|
    = |\mathbf P_{\delta_x}[e^{i X_t(f)}]|
    \leq 1,
\]
    we have
\begin{equation}
\label{eq: -v has positive real part}
 \operatorname{Re} U_tf(x)
    \leq 0.
\end{equation}
%new added
    Therefore, it is legal to talk about $\psi(x,-U_tf)$since $z\mapsto \psi(x,z)$ is well defined on $\mathbb C_+$.
    According to Lemma \ref{lem: estimate of exponential remaining}, we have that
\begin{equation}
\label{eq: upper bound for vf}
    |U_tf(x)| \leq \mathbb N_x[|e^{i W_t(f)} - 1|]
    \leq \mathbb N_x[|i W_t(f)|]
    \leq (P^\beta_t |f|)(x).
\end{equation}
    Notice that, for any compact $K \subset \mathbb R$,
\begin{equation}
\label{eq: estimate of deriavetive of v(theta)}
    \mathbb N_x\Big[\sup_{\theta \in K} \Big|\frac{\partial}{\partial \theta} (e^{i\theta W_t(f)} - 1) \Big|\Big]
    \leq \mathbb N_x[|W_t(f)|] \leq (P^\beta_t |f|)(x) < \infty.
\end{equation}
    Therefore, according to \cite[Theorem A.5.2.]{Durrett2010Probability},
    $U_t(\theta f)(x)$ is differentiable in $\theta \in \mathbb R$ with
\[
    \frac{\partial}{\partial \theta} U_t(\theta f)(x)
    = i\mathbb N_x[W_t(f)e^{i\theta W_t(f)}],
    \quad \theta \in \mathbb R.
\]
    Moreover, from the above, it is clear that
\begin{equation}
\label{eq: upper bounded for derivative of v(theta)}
    \sup_{\theta \in \mathbb R}\Big| \frac{\partial}{\partial \theta}U_t(\theta f)(x)\Big|
    \leq ( P^\beta_t |f|)(x).
\end{equation}
    From the dominate convergence theorem, we can verify that $(\partial/\partial \theta)U_tf(x)$ is continuous in $\theta$.
    In another words, $\theta \mapsto -U_t(\theta f)(x)$ is a $C^1$ map from $\mathbb R$ to $\mathbb C_+$.
    According to this and \eqref{eq: path integration representation of h}, we can write
\begin{equation}
\label{eq: path integration representation of psi(v)}
    \psi(x,-U_tf) = -\int_0^1 \psi'\big(x,-U_t(\theta f)\big) \frac{\partial}{\partial \theta} U_t(\theta f)(x)~d\theta.
\end{equation}
    Notice that
\begin{equation}
\label{eq: upper bound of psi'(v)}
\begin{split}
    &|\psi'(x, -U_tf)|
    \\&= \Big| -\beta(x)- 2\alpha(x) U_tf(x)+ \int_{(0,\infty)} y (1- e^{y U_tf(x)} ) \pi(x,dy)\Big|
    \\&= \Big| - \beta(x)- 2\alpha(x)\mathbb N_x[e^{i W_{t}(f)} - 1]  + \int_{(0,\infty)} y \mathbf P_{y \delta_x}[1-e^{i X_{t}(f)}] \pi(x,dy) \Big|
\\ &\leq \|\beta\|_\infty + 2\alpha(x)\mathbb N_x[W_t(|f|)]+ \int_{(0,\infty)} y\mathbf P_{y\delta_x}[2\wedge X_t(|f|)] \pi(x,dy)
\\ &\leq \|\beta\|_\infty + 2\|\alpha\|_\infty  P^\beta_t |f|(x) + \Big(\sup_{x\in E}\int_{(0,1]}y^2 \pi(x,dy)\Big)~P^\beta_t |f|(x) + 2\sup_{x\in E}\int_{(1,\infty)} y \pi(x,dy)
\\ &=: C_{\ref{eq: upper bound of psi'(v)}.1} + C_{\ref{eq: upper bound of psi'(v)}.2}(P^\beta_t |f|)(x).
\end{split}
\end{equation}
    where $C_{\ref{eq: upper bound of psi'(v)}.1}, C_{\ref{eq: upper bound of psi'(v)}.2}$ are constants not dependent on $f,x$ and $t$.
    Now, using \eqref{eq: path integration representation of psi(v)}, \eqref{eq: upper bounded for derivative of v(theta)} and \eqref{eq: upper bound of psi'(v)}, we have

\begin{align}
\label{eq: upper bound of psi(v)}
    \big|\psi\big(x,-U_tf\big)\big|
    \leq C_{\ref{eq: upper bound of psi'(v)}.1} P^\beta_t |f|(x)+C_{\ref{eq: upper bound of psi'(v)}.2} P^\beta_t |f| (x)^2.
\end{align}
    Now assume that $f \in L_2(\xi) \subset L_1(\xi)$.
    Then, using Jensen's inequality and the above, we have that
\begin{align}
\label{eq: domination of psi(v)}
    &\Pi_x\Big[\int_0^t \big|\psi \big(\xi_s,-U_{t-s}f\big)\big|ds\Big]
    \\&\leq \Pi_x\Big[\int_0^t \big(C_{\ref{eq: upper bound of psi'(v)}.1} P_{t-s}^\beta|f|(\xi_s)+C_{\ref{eq: upper bound of psi'(v)}.2} P_{t-s}^\beta|f|(\xi_s)^2\big)ds\Big]
    \\ &\leq \int_0^t \big(C_{\ref{eq: upper bound of psi'(v)}.1} e^{t\|\beta\|}\Pi_x \big[ \Pi_{\xi_s}[|f(\xi_{t-s})|] \big]+C_{\ref{eq: upper bound of psi'(v)}.2} e^{2t\|\beta\|}\Pi_x \big[ \Pi_{\xi_s}[|f (\xi_{t-s})|]^2 \big]\big)~ds
    \\ &\leq \int_0^t (C_{\ref{eq: upper bound of psi'(v)}.1} e^{t\|\beta\|}\Pi_x [ |f(\xi_{t})|]+C_{\ref{eq: upper bound of psi'(v)}.2}e^{2t\|\beta\|}\Pi_x [ |f (\xi_{t})|^2 ])~ds < \infty.
\end{align}

    As a consequence, expectation
\[
     \Pi_x\Big[\int_0^t \psi(\xi_s,-U_{t-s}f)ds\Big]
    \in \mathbb C
\]
    is well defined for all $f\in L_2(\xi)$.
\subsection{}
     In this subsection, we recall the generalized spine decomposition theorem for the superprocesses.
    Let $X$ be the non-persistent superprocess discussed in Subsection \ref{sec: definition of vf}.
    Let $f\in b\mathscr B_E^{++}$, $T >0$ and $x\in E$.
    The fact that $\mathbf P_{\delta_x}[X_T(f)] = \mathbb N_x[W_T(f)] = P^\beta_T f(x) \in (0,\infty)$ allows us to define the following probability transforms:
\begin{equation}
    d\mathbf P_{\delta_x}^{X_T(f)}
    := \frac{X_T(f)}{P_T^\beta f(x)} d\mathbf P_{\delta_x};
    \quad d\mathbb N_x^{W_T(f)}
    :=  \frac{W_T(f)}{P_T^\beta f(x)} d\mathbb N_x.
\end{equation}
    Following the definition in \cite{RenSongSun2017Spine}, we say that $\{\xi, \mathbf n;\mathbf Q_{x}^{(f,T)}\}$ is a spine representation of $\mathbb N_x^{W_T(f)}$ if
\begin{itemize}
\item
    The spine process $\{(\xi_t)_{0\leq t\leq T}; \mathbf Q^{(f,T)}_x\}$ is a copy of $\{(\xi_t)_{0\leq t\leq T}; \Pi^{(f,T)}_{x}\}$,
    where
\begin{equation}
    d\Pi_x^{(f,T)} := \frac{f(\xi_T)e^{\int_0^T \beta(\xi_s)ds}}{P^\beta_T f(x)} d \Pi_x;
\end{equation}
\item
    Given $\{(\xi_t)_{0\leq t\leq T}; \mathbf Q^{(f,T)}_x\}$, the immigration measure $\{\mathbf n(\xi,ds,dw); \mathbf Q^{(f,T)}_x[\cdot |(\xi_t)_{0\leq t\leq T}]\}$ is a Poisson random measure on $[0,T] \times \mathbb W$ with intensity
\begin{align}
    \mathbf m(\xi,ds,dw)
    %:= 2 \alpha(\xi_s) ds \cdot \mathbb N_{\xi_s}(dw) + ds \cdot \int_{(0,\infty)} y \mathbf P_{y\delta_{\xi_s}}(X\in dw) \pi(\xi_s,dy);
    := 2 \alpha(\xi_s) ds \cdot \mathbb N_{\xi_s}(dw) + ds \cdot \int_{y\in (0,\infty)} y \mathbf P_{y\delta_{\xi_s}}(X\in dw) \pi(\xi_s,dy);
\end{align}
\item
    $\{(Y_t)_{0\leq t\leq T}; \mathbf Q^{(f,T)}_x\}$ is an $\mathcal M^1_E$-valued process defined by
\begin{align}
    Y_t
    := \int_{(0,t] \times \mathbb W} w_{t-s} \mathbf n(\xi,ds,dw),
    \quad 0 \leq t\leq T.
\end{align}
\end{itemize}
    According to the spine decomposition theorem \cite{RenSongSun2017Spine}, we have that
\begin{align}
\label{eq: Spine decomposition 1}
    \{(X_s)_{s \geq 0};\mathbf P_{\delta_x}^{X_T(f)}\}
    \overset{f.d.d.}{=} \{(X_s + W_s)_{s \geq 0};\mathbf P_{\delta_x} \otimes \mathbb N_x^{W_T(f)} \}
\end{align}
    and
\begin{align}
\label{eq: Spine decomposition 2}
    \{(W_s)_{0\leq s\leq T};\mathbb N_x^{W_T(f)}\}
    \overset{f.d.d.}{=} \{(Y_s)_{s \geq 0};\mathbf Q_x^{(f,T)}\}.
\end{align}
\subsection{}
    Let $X$ be the non-persistent superprocess discussed in Subsection \ref{sec: definition of vf}.

\begin{prop}
\label{prop: complex FKPP-equation}
    Let $f\in L_2(\xi)$. Then for each $t\geq 0$ and $x\in E$,
\begin{equation}
\label{eq: complex FKPP-equation}
    U_tf(x) - \Pi_x \Big[\int_0^t \psi\big(\xi_s, - U_{t-s}f\big) ds \Big]
    \\= i \Pi_x [f(\xi_t)],
\end{equation}
and
\begin{equation}
\label{eq: complex FKPP-equation with FK-transform}
    U_tf(x) -  \int_0^t P_{t-s}^{\beta} \psi_0\big(\cdot,-U_sf\big) (x)~ds
    \\= iP_t^\beta f(x).
\end{equation}
\end{prop}

\begin{proof}
    Assume that $f\in b\mathscr B_E$.
    Set $t>0, r\in [0,t), x\in E$ and $g\in b\mathscr B_E^{++}$.
    Denoted by $\{\xi, \mathbf n; \mathbf Q_x^{(g,t)}\}$ the spine representation of $\mathbb N_x^{W_t(g)}$.
    %Given $\{\xi; \mathbf Q_x^{(g,t)}\}$, denote by $\mathbf m(\xi, ds,dw)$ the conditional intensity of $\mathbf n$.
    Conditioned on $\{\xi; \mathbf Q_x^{(g,t)}\}$, denote by $\mathbf m(\xi, ds,dw)$ the conditional intensity of $\mathbf n$.
    Denote by $\Pi_{r,x}$ the probability of Hunt process $\{\xi; \Pi\}$ initiated at time $r$ and position $x$.
    From Lemma \ref{lem: estimate of exponential remaining}, we have $\mathbf Q^{(g,t)}_{x}$-a.s.ly
\begin{align}
&\int_{[0,t]\times \mathbb W}|e^{i w_{t-s}(f)} - 1| \mathbf m(\xi, ds,dw)
    \leq \int_{[0,t]\times \mathbb W}\big(| w_{t-s}(f)| \wedge 2\big) \mathbf m(\xi, ds,dw)
    \\&\leq \int_0^t \Big(2\alpha(\xi_s)\mathbb N_{\xi_s}\big( W_{t-s}(|f|)\big)  + \int_{(0,1]} y \mathbf P_{y \delta_{\xi_s}}[ X_{t-s}(|f|)] \pi(\xi_s,dy)
    \\&\qquad\qquad+ 2\int_{(1,\infty)}y\pi(\xi_s,dy)\Big) ds
     \\&\leq \int_0^t (P_{t-s}^\beta |f|)(\xi_s)\Big(2\alpha(\xi_s)  + \int_{(0,1]} y^2 \pi(\xi_s,dy)\Big) ds + 2t \sup_{x\in E}\int_{(1,\infty)}y\pi(x,dy)
    \\&\leq \Big(2\|\alpha\|_\infty +\sup_{x\in E}\int_{(0,1]} y^2 \pi(x,dy)\Big) t e^{t\|\beta\|_\infty}\|f\|_\infty + 2t \sup_{x\in E}\int_{(1,\infty)}y\pi(x,dy)
    < \infty.
\end{align}
    According to this, Fubini's theorem, \eqref{eq: N and characteristic exponent}, \eqref{eq: -v has positive real part} and the argument in Subsection \ref{sec: extension of branching mechanism}, we have $\mathbf Q^{(g,t)}_{x}$-a.s.ly,
\begin{align}
    &\int_{[0,t]\times \mathbb N}(e^{i w_{t-s}(f)} - 1) \mathbf m(\xi, ds,dw)
    \\&=\int_0^t \Big(2\alpha(\xi_s)\mathbb N_{\xi_s}(e^{i W_{t-s}(f)} - 1)  + \int_{(0,\infty)} y \mathbf P_{y \delta_{\xi_s}}[e^{i X_{t-s}(f)} - 1] \pi(\xi_s,dy)\Big) ds
    \\&=\int_0^t \Big( 2\alpha(\xi_s) U_{t-s} f(\xi_s) + \int_{(0,\infty)} y (e^{y U_{t-s}f(\xi_s)} - 1) \pi(\xi_s,dy) \Big) ds
    \\&= -\int_0^t \psi'_0 \big(\xi_s, -U_{t-s}f\big)ds.
\end{align}
    Therefore, according to \eqref{eq: Spine decomposition 2}, Campbell's formula and above, we have that
\begin{align}
\label{eq: N to Pi}
    \mathbb N_x^{W_t(g)}[e^{i W_t(f)}]
    &=\mathbf Q_x^{(g,t)} \Big[\exp\Big\{\int_{[0,t]\times \mathbb N}(e^{i w_{t-s}(f)} - 1) \mathbf m(\xi, ds,dw)\Big\}\Big]
    \\&= \Pi_x^{(g,t)} [e^{-\int_0^t \psi'_0(\xi_s, -U_{t-s}f)ds}]
    \\&= \frac{1}{T_t^\beta g (x)} \Pi_x[ g(\xi_t) e^{-\int_0^t \psi'(\xi_s, -U_{t-s}f)ds} ].
\end{align}
    Let $\epsilon >0$.
    Define $f^+ = (f \vee 0) + \epsilon$ and $f^- = (-f) \vee 0 + \epsilon$, then $f^\pm \in b\mathscr B^{++}_E$ and $f = f^+ - f^-$.
    According to \eqref{eq: Spine decomposition 1}, we have that
\begin{equation}
    \frac{\mathbf P_{\delta_x}[X_t(f^{\pm})e^{i X_t(f)}]}{\mathbf P_{\delta_x}[X_t(f^{\pm})]}
    = \mathbf P_{\delta_x}[e^{i X_t(f)}] \mathbb N_x^{W_t(f^{\pm})}[e^{i X_t(f)}].
\end{equation}
    Using \eqref{eq: N to Pi} and above, we have
\begin{align}
    \frac{\mathbf P_{\delta_x}[X_t(f)e^{i X_t(f)}] }{\mathbf P_{\delta_x}[e^{i X_t(f)}]}
    &= \mathbf P_{\delta_x}[X_t(f^+)] \mathbb N_x^{W_t(f^+)} [e^{i X_t(f)}] - \mathbf P_{\delta_x}[X_t(f^-)]\mathbb N_x^{W_t(f^-)}[e^{i X_t(f)}]
    \\& = \Pi_x[ f(\xi_t) e^{- \int_0^t \psi'(\xi_s, -U_{t-s}f) ds}  ].
\end{align}
    Therefore, according to \eqref{eq: derivative of characteristic exponent} and above, we have
\begin{align}
    \frac{\partial}{\partial \theta} {U_t(\theta f)(x)}
    =  \Pi_x[ if(\xi_t) e^{ - \int_0^t \psi'(\xi_s, -U_{t-s}(\theta f)) ds} ]
\end{align}
    Since $\{(\xi_{r+t})_{t \geq 0}; \Pi_{r,x}\} \overset{d}{=} \{(\xi_{t})_{t\geq 0}; \Pi_{x}\} $, we have

\begin{align}
    &\frac{\partial}{\partial \theta} U_{t-r}(\theta f)( x)
    = \Pi_x[ i f(\xi_{t-r}) e^{-\int_0^{t-r} \psi'(\xi_s, -U_{t-r-s}(\theta f)) ds} ]
    \\&= \Pi_{r,x}[i f(\xi_t)e^{-\int_0^{t-r} \psi'(\xi_{r+s}, -U_{t-r-s}(\theta f)) ds} ]
    = \Pi_{r,x}[if(\xi_t)e^{-\int_r^t \psi'(\xi_{s}, -U_{t-s}(\theta f)) ds} ].
\end{align}

    From \eqref{eq: upper bound of psi'(v)}, we know that for each $\theta\in \mathbb R$, $(t,x) \mapsto |\psi'(x,-U_tf(x))|$ is locally bounded (i.e. bounded on $[0,T]\times E$ for each $T \geq 0$).
    Therefore, we can apply the argument in Subsection \ref{seq: complex Feynman-Kac transform} and get that

\[
    \frac{\partial}{\partial \theta} U_{t-r}(\theta f)(x) + \Pi_{r,x} \Big[\int_r^t \psi'\big(\xi_s,- U_{t-s}(\theta f)\big)\frac{\partial}{\partial \theta} U_{t-s}(\theta f)(\xi_s)~ds\Big]
    = \Pi_{r,x} [i f(\xi_t)],
\]
    and
\begin{align}
    \frac{\partial}{\partial \theta} U_{t-r}(\theta f)(x) + \Pi_{r,x} \Big[\int_r^t e^{\int_r^s \beta(\xi_u)du}\psi_0'\big(\xi_s,- U_{t-s}(\theta f)\big)\frac{\partial}{\partial \theta} U_{t-s}(\theta f)(\xi_s)~ds\Big]
    = \Pi_{r,x} [i e^{\int_r^t \beta(\xi_s)ds}f(\xi_t)].
\end{align}
    Integrating the aboves with respect to $\theta$  on [0,1], using \eqref{eq: path integration representation of psi(v)}, \eqref{eq: upper bound of psi'(v)}, \eqref{eq: upper bounded for derivative of v(theta)} and Fubini's theorem, we have

\begin{equation}
    U_{t-r}f(x) - \Pi_{r,x} \Big[\int_r^t \psi\big(\xi_s,-U_{t-s}f\big) ~ds\Big]
    = i\theta \Pi_{r,x} [f(\xi_t)],
\end{equation}
    and similarly,
\begin{equation}
    U_{t-r}f(x) - \Pi_{r,x} \Big[\int_r^t e^{\int_r^s \beta(\xi_u)du} \psi_0\big(\xi_s,- U_{t-s}f\big) ~ds\Big]
    = i\Pi_{r,x} [e^{\int_r^t\beta(\xi_u)du}f(\xi_t)].
\end{equation}
    Taking $r = 0$, we get that \eqref{eq: complex FKPP-equation} and \eqref{eq: complex FKPP-equation with FK-transform} is true if $f\in b\mathscr B_E$.

    %Now assume that $f\in L_2(\xi)$.
    The rest of the proof is to evaluate \eqref{eq: complex FKPP-equation} and \eqref{eq: complex FKPP-equation with FK-transform} for all $f\in L_2(\xi)$. We only do this for \eqref{eq: complex FKPP-equation} since the argument for \eqref{eq: complex FKPP-equation with FK-transform} is similar.
    Set $n \in \mathbb N$.
    Writing $f_n := (f^+ \wedge n) - (f^- \wedge n)$, then $f_n \xrightarrow[n\to \infty]{} f$ pointwisely.
    From what we have proved, we have
\begin{equation}
\label{eq: complex FKPP-equation for fn}
    U_tf_n(x) - \Pi_{x} \Big[\int_0^t \psi\big(\xi_s, - U_{t-s}f_n\big) ~ds\Big]
    = i \Pi_{x} [f_n(\xi_t)].
\end{equation}
    Notice the following:
\begin{itemize}
\item
    It is clear that $\Pi_{x}[f_n(\xi_t)] \xrightarrow[n\to \infty]{} \Pi_{x}[f(\xi_t)]$.
\item
     $U_tf_n(x) \xrightarrow[n\to \infty]{} U_tf(x)$ due to \eqref{eq: N and characteristic exponent}, dominated convergence theorem and the fact that
\[
    |e^{i W_t(f_n)} - 1| \leq W_t(|f|);
    \quad \mathbb N_x[W_t(|f|)] = (P_t^\beta |f|)(x) < \infty.
\]
\item
     $\Pi_{x} [\int_0^t \psi(\xi_s,- U_{t-s}f_n)ds] \xrightarrow[n\to \infty]{} \Pi_{x} [\int_0^t \psi(\xi_s,- U_{t-s}f)ds]$ due to donimated convergence theorem, \eqref{eq: domination of psi(v)} and the fact (see \eqref{eq: upper bound of psi(v)}) that
\begin{align}
    \big|\psi(\xi_s,- U_{t-s}f_n)\big|
    \leq C_{\ref{eq: upper bound of psi'(v)}.1} P_{t-s}^\beta|f|(\xi_s)+C_{\ref{eq: upper bound of psi'(v)}.2} P_{t-s}^\beta|f|(\xi_s)^2.
\end{align}
\end{itemize}
    Using the above arguements, letting $n \to \infty$ in \eqref{eq: complex FKPP-equation for fn}, we get the desired result.
\end{proof}
%end new add



\begin{thebibliography} {10}

\bibitem{BAM}
Berestycki, J., Kyrianou, A.E., Murillo-Salas, A
\newblock{\em The prolific backbone for supercritical superprocesses}. Stoch. Proc. Appl. 121, 1315-1331(2011)

\bibitem{Cuppens1975Decomposition}
Cuppens, R.:
\emph{Decomposition of multivariate probabilities.}
Probability and Mathematical Statistics, Vol. 29. Academic Press [Harcourt Brace Jovanovich, Publishers], New York-London, 1975.
\MR{0517412}

\bibitem{Durrett2010Probability}
Durrett, R.:
\emph{Probability: theory and examples.}
Fourth edition. Cambridge Series in Statistical and Probabilistic Mathematics, 31. Cambridge University Press, Cambridge, 2010.
\MR{2722836}

\bibitem{EB}
Dynkin,~E.B.
\newblock {\em Superprocesses and partial differential equations}, Ann. Probab (1993): 1185-1262.

\bibitem{DK}
Dynkin, E. B., Kuznetsov, S. E.:
\newblock {\em $\mathbb{N}$-measure  for branching exit Markov system and their applications to differential equations}, Probab. Theory Related Fields 130(2004) 135-150

\bibitem{Kyprianou2014Fluctuations}
Kyprianou, A. E.:
\emph{Fluctuations of L\'{e}vy processes with applications.}
Introductory lectures. Second edition. Universitext. Springer, Heidelberg, 2014.
\MR{3155252}

\bibitem{Li2011Measure-valued}
Li, Z.:
\emph{Measure-valued branching Markov processes.}
Probability and its Applications (New York), Springer, Heidelberg, 2011.
\MR{2060602}

\bibitem{Linde1986Probability}
Linde, W.:
\emph{Probability in Banach spaces—stable and infinitely divisible distributions.}
Second edition. A Wiley-Interscience Publication. John Wiley \& Sons, Ltd., Chichester, 1986.

\bibitem{MM}
Marks, R., Mil\'{o}s, P.:
\newblock {\em CLT for supercritical branching processes with heavy-tailed branching law}.
\ARXIV{1803.05491}

\bibitem{GD}
Metafune,~G., Pallara,D.
\newblock {\em Specturm of Ornstein-Uhlenbeck operators in $\mathcal{L}^p$ space with respect to invariant measures}. J. Funct. Anal. 196, 40-60(2002)

\bibitem{RSZ}
Ren, Y.-X., Song, R., Zhang, R.:
\newblock {\em Central limit theorems for super Ornstein-Uhlenbeck processes}, Acta Appl. Math. 130(2014)9-49.

\bibitem{RenSongSun2017Spine}
Ren, Y.-X., Song, R., Sun, Z.:
\newblock{\em Spine decompositions and limit theorems for a class of critical superpeocesses,} arXiv preprint arXiv:1711.09188(2017).

\bibitem{Rudin1987Real}
Rudin, W.:
\emph{Real and complex analysis.}
Third edition. McGraw-Hill Book Co., New York, 1987.
\MR{0924157}

\bibitem{SchillingSongVondracek2010Bernstein}
Schilling, R., Song, R., Vondra\v{c}ek, Z.:
\emph{Bernstein functions. Theory and applications.}
De Gruyter Studies in Mathematics, 37. Walter de Gruyter \& Co., Berlin, 2010.
\MR{2598208}


\bibitem{Sato1999Levy}
Sato, K.:
\emph{Lévy processes and infinitely divisible distributions.}
Cambridge Studies in Advanced Mathematics, 68. Cambridge University Press, Cambridge, 1999.

\bibitem{SteinShakarchi2003Complex}
Stein, E. M. and Shakarchi, Rami:
\emph{Complex analysis.}
Princeton Lectures in Analysis, 2. Princeton University Press, Princeton, NJ, 2003.

\end{thebibliography}




\end{document}
