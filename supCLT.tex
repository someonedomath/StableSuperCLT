%%----Versions-----------------------------
% supCLT0127.tex 2018/01/27 by Yanxia
% supCLT0124.tex 2018/01/24 by Jianjie
% supCLT0114.tex 2018/01/14 by Yanxia
% supCLT0107.tex 2018/01/07 by Jianjie
% supclt1231.tex 2018/12/14 by Zhenyao
% supclt1213.tex 2018/12/14 by Yanxia
% supclt1213.tex 2018/12/13 by Renming
% supclt1209.tex 2018/12/07 by Zhenyao
% supclt1207.tex 2018/12/07 by Renming
% supCLT1204.tex 2018/12/04 by Zhenyao
% supCLT1130.tex 2018/11/30 by Jianjie
% supCLT1124.tex 2018/11/24 by Zhenyao
% supCLT1115.tex 2018/11/15 by Jianjie
% supCLT9.tex ......... by Zhenyao
% supCLT8.tex ......... by Jianjie
% supCLT7.tex ......... by Zhenyao
% supCLT6.tex ......... by Jianjie
% supCLT5.tex ......... by Zhenyao
% supCLT4.tex ......... by Jianjie
% supCLT3.tex 2018/8/25 by Zhenyao
% supCLT2.tex 2018/8/13 by Jianjie
% supCLT1.tex 2018/7/24 by Zhenyao
% supCLT.tex 2018/6/29 by Jianjie
%---The preamble----------------------
\documentclass[12pt,a4paper]{amsart}
	\setlength{\textwidth}{\paperwidth}
	\addtolength{\textwidth}{-2in}
	\calclayout
\usepackage[utf8]{inputenc}
\usepackage{mathtools}
	\mathtoolsset{showonlyrefs}
\usepackage{stackrel}
\usepackage{mathrsfs}
\usepackage{cite}
\usepackage[backref=section]{hyperref}
	\def\MR#1{\href{http://www.ams.org/mathscinet-getitem?mr=#1}{MR-#1}}
	\def\ARXIV#1{\href{https://arxiv.org/abs/#1}{arXiv:#1}}
\usepackage{comment}
\usepackage{amsthm}
	\theoremstyle{plain}
	\newtheorem{thm}{Theorem}[section]
	\newtheorem{lem}[thm]{Lemma}
	\newtheorem{prop}[thm]{Proposition}
	\newtheorem{cor}[thm]{Corollary}
	\newtheorem{conj}[thm]{Conjecture}
	\theoremstyle{definition}
	\newtheorem{defi}[thm]{Definition}
	\newtheorem{rem}[thm]{Remark}
	\newtheorem{exa}[thm]{Example}
	\newtheorem{asp}{Assumption}
	\numberwithin{equation}{section}
	\allowdisplaybreaks
%\newcommand{\added}[1]{{\color{blue}#1}}\newcommand{\deleted}[1]{{\color{red}#1}}
%\newcommand{\added}[1]{#1}\newcommand{\deleted}[1]{}

%---Top matter------------------------------
\begin{document}
\title
    [CLT for Super-OU processes]
 %   {CLT for Super-OU Processes with stable Branching}
 {Central Limit Theorems for Super Ornstein-Uhlenbeck Processes with Stable Branching}
\author
    [Y.-X. Ren, R. Song, Z. Sun and J. Zhao]
    {Yan-Xia Ren, Renming Song, Zhenyao Sun and Jianjie Zhao}
\address
    {Yan-Xia Ren\\
    LMAM School of Mathematical Sciences \& Center for
Statistical Science\\
    Peking University\\
    Beijing, P. R. China, 100871}
\email{yxren@math.pku.edu.cn}
\thanks{The research of Yan-Xia Ren is supported in part by NSFC (Grant Nos. 11671017  and 11731009).}
\address
    {Zhenyao Sun\\
    School of Mathematical Sciences\\
    Peking University\\
    Beijing, P. R. China, 100871}
\email{zhenyao.sun@pku.edu.cn}
\address
    {Jianjie Zhao\\
    School of Mathematical Sciences\\
    Peking University\\
    Beijing, P. R. China, 100871}
\email{zhaojianjie@pku.edu.cn}
%\begin{abstract}
%\end{abstract}
%\subjclass[2010]{60J80, 60F05}
%\keywords{}
%\date{}
\maketitle
%---Contents-----------------------
\begin{abstract}
    %Consider a super-OU process $\{X_t, t\geq 0\}$ with branching mechanism $\psi(z)=-\alpha z +z^{1+\beta}$, where $\alpha >0$ and $\beta\in (0,1)$.
    Consider a super Ornstein-Uhlenbeck processes (super-OU process) $(X_t)_{t\geq 0}$ with branching mechanism $\psi(z)=-\alpha z +\eta z^{1+\beta} + \psi_1(z)$, where $\alpha >0$ and $\beta\in (0,1)$ and that there exists $C >0$ and $\delta >0$ such that
$
    |\psi_1(z)| \leq C(|z|^2+|z|^{1+\gamma + \delta}).
$
    For each testing function $f$ of at most polynomial growth, denote by $\kappa_f$ the order of $f$ in the spectral decomposition of $f$ in terms of the spectrum of the the mean semigroup of $X$.
    Conditioned on no-extinction, we establish some spatial central limit theorems for $\langle f, X_t \rangle$ in three different regimes:
    %In the small branching rate case, $\alpha\beta< \kappa_f b(1+\beta)$, $\langle f,X_t\rangle/\|X_t\|^{\frac{1}{+\beta}}$ converges weakly.
    In the small branching rate case  $\alpha\beta< \kappa_f b(1+\beta)$, $\langle f,X_t\rangle/\|X_t\|^{\frac{1}{1+\beta}}$ converges weakly to a $(1+\beta)$-stable random variable.
    %In the critical branching rate case,  $\alpha\beta= \kappa_f b(1+\beta)$,  $\langle f,X_t\rangle/(t\|X_t\|)^{\frac{1}{+\beta}}$ converges weakly.
    In the critical branching rate case  $\alpha\beta= \kappa_f b(1+\beta)$,  $\langle f,X_t\rangle/(t\|X_t\|)^{\frac{1}{1+\beta}}$ converges weakly another $(1+\beta)$-stable random variable.
    In the large branching rate case  $\alpha\beta> \kappa_f b(1+\beta)$,
    %new added
    we first prove that
    %end new
    $e^{-(\alpha-\kappa(f)b)t}\langle f,X_t\rangle$ converges almost surely and in $L^{1+\gamma}$, for any $\gamma \in (0,\beta)$,
    %new added
   and then established a central limit theorem, which say that after proper centering and scaling, we get a $(1+\beta)$-stable random variable again.
    %end new
\end{abstract}
%\subjclass[2010]{60J80, 60F05}
%\keywords{}
%\date{}
\maketitle
\section{Introduction}
\subsection{Model}
\label{ss:1.1}
	Let $d \in \mathbb N:= \{1,2,\dots\}$.
    Let $\xi=\{(\xi_t)_{t\geq 0}; (\Pi_x)_{x\in \mathbb R^d}\}$ be an $\mathbb R^d$-valued Ornstein-Uhlenbeck prcess (OU process) with generator
\begin{equation}
\label{eq: OU generator}
    Lf(x)
        = \frac{1}{2}\sigma^2\Delta f(x)-b x \cdot \nabla f(x),
        \quad  x\in \mathbb R^d,
        f \in C^2(\mathbb R^d),
\end{equation}
    %where $a>0$ and $b>0$ are constants.
    where $\sigma$ and $b$ are positive constants.
    Denote by $(P_t)_{t\geq 0}$ the transition semigroup of $\xi$.
    Let $\psi$ be the function on $\mathbb R_+$ defined by
\begin{equation}
\label{mechanism}
       \psi(z)=
    - \alpha z + \rho z^2 + \int_{(0,\infty)} (e^{-zy} - 1 + zy) \pi(dy).
    \quad  z \in \mathbb R_+,
\end{equation}
	%where $\alpha > 0$ and $\beta \in (0,1) $ are constants.
	where $\alpha > 0 $, $\rho \geq0$ and $\pi$ is a measure on $(0,\infty)$ with $\int_{(0,\infty)}y\wedge y^2 \pi(dy)< \infty$.
    Denote by $\mathcal M(\mathbb R^d)$ the space of all finite Borel measures on $\mathbb R^d$.
%added%
    Denote by $\mathcal B(\mathbb R^d)$ the space of all real-valued Borel measurable functions on $\mathbb R^d$.
%%end added%
	%For all Borel function $f$ and Borel measure $\mu$, write $\langle f,\mu\rangle = \int f(x)\mu(dx)$ whenever the integral makes sense.
	For each $f\in \mathcal B(\mathbb R^d)$ and $\mu \in \mathcal M(\mathbb R^d)$, write $\langle f,\mu\rangle = \int f(x)\mu(dx)$ whenever the integral makes sense.
    Put $\|\mu\|=\langle 1,\mu\rangle$.

    We say an $\mathcal M(\mathbb R^d)$-valued Markov process $X = \{(X_t)_{t\geq 0}; (\mathbb{P}_{\mu})_{\mu \in \mathcal M(\mathbb R^d)}\}$ is a super Ornstein-Uhlenbeck process (super-OU process) with branching mechanism $\psi$, if for each non-negative bounded Borel function $f$ on $\mathbb R^d$, we have
\begin{equation} \label{super}
    \mathbb{P}_{\mu}[e^{-\langle f,X_t \rangle}]
    = e^{-\langle V_tf, \mu \rangle},
    \quad t\geq 0, \mu \in \mathcal M(\mathbb R^d),
\end{equation}
	where $(t,x) \mapsto V_tf(x)$ is the unique locally bounded positive solution to the equation
\begin{equation}\label{eq1}
	V_tf(x) + \Pi_x \Big[ \int_0^t\psi\big(\xi_s,V_{t-s}f(\xi_s)\big)~ds\Big]
	= \Pi_x [f(\xi_t)],
    \quad x\in \mathbb R^d, t\geq 0.
\end{equation}	
    The existence of superprocesses are well known, see \cite{Dynkin1993Superprocesses} for instance.

   According to  Subsection A.1, the branching mechanism $\psi$ can be extended to
    $\mathbb C_+:=\{x+iy: x\in [0, \infty), y\in \mathbb R\}$ by
 \[\psi(z)=
    - \alpha z + \rho z^2 + \int_{(0,\infty)} (e^{-zy} - 1 + zy) \pi(dy),
    \quad  z \in \mathbb C_+,\]
    which is continuous on $\mathbb C_+$ and is holomorphic on $\mathbb C_+^0:=
    \{x+iy: x\in (0, \infty), y\in \mathbb R\}$ with derivative
\begin{equation}
\label{eq: deriavetive of the Poission part}
    \psi'(z) := -\alpha + 2\rho z+ \int_{(0,\infty)} (1-e^{-zy}) y\pi(dy),
    \quad z\in \mathbb C^0_+.
\end{equation}
{ \bf
\begin{asp}
\label{asp: branching mechanism}
    The branching mechanism takes the form of
\[
    \psi(z) = -\alpha z + \eta z^{1+\beta} + \psi_1(z),\quad z\in \mathbb C_+,
\]
    where $\eta > 0$, $\gamma \in (0,1)$ and that there exists $C >0$ and $\delta >0$ such that
\[
    |\psi_1(z)| \leq C(|z|^2+|z|^{1+\beta + \delta}),\quad z\in \mathbb C_+.
\]
\end{asp}
\begin{lem}
    Under Assumption \ref{asp: branching mechanism}, we have that $\psi$ satisfies the Grey's condition i.e. there is some constant $z' > 0$ such that $\psi(z) > 0$ for all $z>z'$, and that $\int_{z'}^\infty \psi(z)^{-1}dz < \infty$.
    Also, $\psi$ satisfies the $L \log L$ critierion  i.e.
\[
    \int_1^\infty y \log y~\pi(dr)< \infty.
\]
\end{lem}
\begin{proof}
    { Need a proof.}
\end{proof}
}
    According to \cite[Theorem 12.5 \& Theorem 12.7]{Kyprianou2014Fluctuations}, we have
\begin{equation}
    \mathbb{P}_{\mu} (\exists t\geq 0,~\text{s.t.}~\|X_t\|=0)
    = e^{-\bar v \|\mu\|},
\end{equation}
    where $\bar v := \sup\{\lambda \geq 0: \psi(\lambda) = 0\}$.
    Define
\begin{equation}\label{meansemigroup}
    P^{\alpha}_t f(x)
    :=
    e^{\alpha t} P_t f(x) =
    \Pi_x [e^{\alpha t}f(\xi_t)],
    \quad x\in \mathbb R^d,t\geq 0, f\in \mathcal B(\mathbb R^d, \mathbb R_+).
\end{equation}
    It is konwn that, see \cite[Proposition 2.27]{Li2011Measure-valued} for example, $(P^\alpha_t)_{t\geq 0}$ is the \emph{mean semigroup} of $X$, in the sense that
\begin{equation}\label{eq:meanformula}
    \mathbb{P}_{\mu}[\langle f, X_t \rangle]
    = \langle P^\alpha_t f, \mu \rangle,
    \quad t\geq 0, f\in \mathcal B(\mathbb R^d, \mathbb R_+), \mu \in \mathcal M(\mathbb R^d).
\end{equation}


 In this paper, we will establish some spatial central limit theorems for the super-OU process with stable branching mechanism. Precisely, we want to find $F_t$ and $G_t$ such that
 $$\frac{\langle f, X_t \rangle -G_t}{F_t}$$ converges weakly to some non-degenerate random variable as $t\rightarrow\infty$, for some test functions $f$. It turns out that the normalization $F_t$ depends on what we call branching rate regimes determined by the sign of $\alpha\beta-\kappa_f b (1+\beta)$, where $\kappa_f$ is the order of $f$ in the spectral decomposition.

 There are many papers studying central limit theorems for branching processes, branching diffusions and superprocesses. See \cite{HKBPS1,HKBPS2} for supercritical multi-type Galton-Waston processes, \cite{KBA1,KBA2,KBA3} for supercritical multi-type continuous time branching processes and \cite{ASHH} for general supercritical branching Markov processes under some certain conditions. In \cite{ARMP}, Adamcazk and Milo\'s proved some central limit theorems for supercritical branching Ornstein-Uhlenbeck processes with binary branching mechanism and in \cite{MP2012} Milo\'s proved some central limit theorems for supercritical super-OU processes with branching mechanisms satisfying a forth moment condition. In this two paper, Milo\'s with his coauthor pictured connection between CLT and what they call branching rate regimes. In \cite{RSZ}, Ren, Song and Zhang proved some central limit thoerems for supercritical super-OU processes with branching mechanisms satisfying only a second moment. Moverover, Compare with the results in \cite{ARMP,MP2012}, the limit distributions in \cite{RSZ} are non-degenerate. In \cite{RSZ1,RSZ2,RSZ3,RSZ4}, They also establish a series of central limit theorems for a large class of general supercritical branching Markov processes and super processes with spatially dependent branching mechanisms.

 All the results described above hold for systems with branching mechanisms satisfying finite variance. It's a natural  to ask that how to establish counterparts of the central limit theorems for branching Markov processes and superprocesses with infinite variance.  In \cite{MM}, Marks and Milo\'s considered the limit behavior of supercritical branching Ornstein-Uhlenbeck processes with a special stable branching law. They established some central limit theorems for small and critical branching rate cases,  but leave the central limit result unsolved for large rate case. In this paper, we consider a class of super-OU processes with more general stable branching mechanisms. For small and critical branching cases, we obtained central limit theorems which are similar to corresponding results for branching Ornstein-Uhlenbeck processes in Marks and Milo\'s \cite{MM}. For large rate case, we first establish  an almost sure convergent result, and then establish a central limit theorem, which has no corresponds in \cite{MM}.

 %Except for the main results proved in section 4, there are some other significative results in our paper. In appendix, we show that the characteristic exponent of supercritical general super processes $U_t f(x)=-{\rm Log} \mathbb{P}_{\delta_x}e^{i\theta X_t(f)}$ can be defined by corresponding equation by using Spain Decomposition. In section 3, we establish a small value probability estimation of supercritical continuous-state branching processes with branching mechanism satisfying Grey's condition and $L\log L$ critierion. This all results are necessary for us to prove the central limit theorems.


\subsection{OU Semigroup}
    In this subsection, we recall some results on the spectral properties of the OU operator $L$ from \cite{GD}.
    Let $\xi$ be an OU process with generator $L$ given by \eqref{eq: OU generator}.
    It is known that $\xi$ has an invariant measure
\begin{equation}
\label{invariantdensity}
    \varphi(x)dx
    =\Big (\frac{b}{\pi \sigma^2}\Big )^{d/2}\exp \Big(-\frac{b}{\sigma^2}|x|^2 \Big)dx,
    \quad x\in \mathbb R^d.
\end{equation}
    Let $L^2(\varphi):= \left\{ h  \in \mathcal B(\mathbb R^d, \mathbb R): \int_{\mathbb R^d} |h(x)|^2 \varphi(x) dx < \infty \right\}$. $L^2(\varphi)$ is a Hilbert space with inner product
\begin{equation}
    \langle f_1, f_2 \rangle_{\varphi}
    := \int_{\mathbb R^d}f_1(x)f_2(x)\varphi(x) dx, \quad f_1,f_2 \in L^2(\varphi).
\end{equation}
     Let $\mathbb N = \{1,2,\dots\}$ and $\mathbb Z_+ := \mathbb N\cup\{0\}$.
     For each $p = (p_k)_{k = 1}^d \in \mathbb{Z}_+^{d}$,
    let $|p|:=\sum_{k=1}^d p_k$, $p!:= \prod_{k= 1}^d p_k !$ and $\partial x^p:= \prod_{k = 1}^d\partial x_k^{p_k}$.
    The Hermite polynomials are defined by
\begin{equation}
    H_p(x)
    :=(-1)^{|p|}\exp(|x|^2) \frac{\partial ^{|p|}}{\partial x^p} \exp(-|x|^2) ,
    \quad x\in \mathbb R^d,
    p \in \mathbb{Z}_+^{d}.
\end{equation}
    It is known that $(P_t)$ is a strongly continuous semigroup on $L^2(\varphi)$ and $L$ has
    discrete spectrum $a(L)= \{-bk: k \in \mathbb Z_+\}$.
    For each $k \in \mathbb Z_+$, denote by $\mathcal{A}_k$ the eigenspace corresponding to the eigenvalue $-bk$, then
\begin{equation}
    \mathcal{A}_k
    = \operatorname{Span} \{\phi_p : p\in \mathbb Z_+^d, |p|=k\},
\end{equation}
where
\begin{equation}\label{eigenfunction}
    \phi_p(x)
    := \frac{1}{\sqrt{ p! 2^{|p|} }} H_p \Big(\frac{ \sqrt{b} }{a}x \Big),
    \quad x\in \mathbb R^d, p\in \mathbb Z_+^d.
\end{equation}
    It is known that
\begin{equation}\label{semigroupformula}
    P_t\phi_p(x)
    =e^{-b|p|t}\phi_p(x),
    \quad t\geq 0, x\in \mathbb R^d, p\in \mathbb Z_+^d.
\end{equation}
    Moreover, $\{\phi_p: p \in \mathbb Z_+^d\}$ forms a complete orthogonal basis for $L^2(\varphi)$.
    Thus, for each $f\in L^2(\varphi)$,
\begin{equation}\label{semicomp1}
    f
    =\sum_{k=0}^{\infty}\sum_{|p|=k}\langle f, \phi_p \rangle_{\varphi} \phi_p,
    \quad \text{in~} L^2(\varphi).
\end{equation}
    Denote by
\begin{equation}\label{order}
    \kappa_f
    :=\inf \left \{k\geq 0: \exists ~ p\in \mathbb Z_+^d ,{\rm ~s.t.~}|p|=k {\rm ~and~}  \langle f, \phi_p \rangle_{\varphi}\neq 0\right \},
\end{equation}
    the order of the function $f$ in $L^2(\varphi)$.
    Note that $ \kappa_f\geq 0$.
    In particular, the order of any constant function is zero.
    Denote by $\mathcal P$ the class of functions of polynomial growth on $\mathbb R^d$, i.e.,
\begin{equation}
    \mathcal{P}
    :=\big \{f\in \mathscr B(\mathbb R^d):\exists~ C>0, n \in \mathbb Z_+, {\rm ~s.t.~} \forall x\in \mathbb R^d,~ |f(x)|\leq C(1+|x|)^n\big\}.
\end{equation}
    It is clear that $\mathcal{P} \subset L^2(\varphi)$.
\subsection{Main Results}
We will use $\mathcal M_c(\mathbb R^d)$ to denote the space of all
finite Borel measures of compact support on $\mathbb R^d$ and $D$ to denote the extinction event of super-OU process.
In this subsection, we will give the main results of this paper.
\subsubsection{Large branching rate}

For each $p\in \mathbb{Z}_+^d$, we define
$$H_t^p:= e^{-(\alpha-|p|b)t}\langle\phi_p,X_t\rangle,\quad t\geq 0.$$
  For all $\gamma\in (0, \beta)$ and $\mu\in \mathcal M_c(\mathbb R^d)$,
  under the condition
   $\alpha\beta>|p|b(1+\beta)$,
   $H_t^p$ is a $\mathbb{P}_{\mu}$-martingale bounded in
   $L^{1+\gamma}(\mathbb{P}_{\mu})$ (see Lemma \ref{lemma26}).
  Thus the limit $H^p_{\infty}:=\lim_{t\rightarrow \infty}H_t^p$ exists $\mathbb{P}_{\mu}$-almost surely and in $L^{1+\gamma}(\mathbb{P}_{\mu})$.
 \begin{thm}\label{Theorem11}
     If $f \in \mathcal{P}$ satisfies $\alpha\beta>\kappa_fb(1+\beta)$, then for
any $\gamma\in (0, \beta)$ and any $\mu\in \mathcal M_c(\mathbb R^d)$,
     as $t\rightarrow \infty$,
     $$e^{-(\alpha-\kappa_fb)t}\langle f, X_t\rangle \rightarrow\sum_{|p|=\kappa_f}\langle f, \phi_p\rangle_{\varphi} H_{\infty}^p \quad in~ L^{1+\gamma}(\mathbb{P}_{\mu}).$$

     Moreover, if $f$ is twice differentiable and all its second order partial derivatives are in $\mathcal{P}$, then we also have almost sure convergence.
 \end{thm}
For any $t\geq 0$, let $H_t:=e^{-\alpha t}\|X_t\|$. Then $H_t$ is equal to $H_t^0$ and is a non-negative martingale with limit $H_{\infty}:=\lim_{t\rightarrow\infty}H_t$,  $\mathbb{P}_{\mu}$-a.s. and in $L^{1+\gamma}(\mathbb{P}_{\mu})$, for all $\gamma\in (0, \beta)$ and $\mu\in \mathcal M_c(\mathbb R^d)$.
 \begin{rem}
Let $f \in \mathcal{P}$.  If $\kappa_f=0$,
    $\langle f, \phi_{\kappa_f}\rangle_{\varphi}$ reduces to $\langle f,\varphi\rangle$. Hence by Theorem \ref{Theorem11},
for any $\gamma\in (0, \beta)$ and any $\mu\in \mathcal M_c(\mathbb R^d)$, as $t\rightarrow \infty$,
     $$e^{-\alpha t}\langle f, X_t\rangle \rightarrow \langle f, \varphi\rangle H_{\infty} \quad in~ L^{1+\gamma}(\mathbb{P}_{\mu}).$$
    Moreover, if $f$ is twice differentiable and all its second order partial derivatives are in $\mathcal{P}$, then we also have almost sure convergence.
 \end{rem}

\subsubsection{Critical branching rate}
    For $f\in \mathcal{P}$ satisfying $\alpha\beta=\kappa_f b(1+\beta)$, define
\begin{equation}\label{tilde-m}
    \widetilde{m}[f]
    := \langle \eta (-i\phi)^{1+\beta},\varphi\rangle
\end{equation}
    where
\[
    \phi(x)
    =\sum_{|p|=\kappa_f}\langle f,\phi_p\rangle\phi_p(x).
\]

\begin{thm}
\label{Theorem12}
    If $f\in\mathcal{P}$ satisfies  $\alpha\beta=\kappa_fb(1+\beta)$,
   then for any $\mu\in \mathcal M_c(\mathbb R^d)$, under
    $\tilde{\mathbb P}_\mu := \mathbb{P}_{\mu}(\cdot|D^c)$, it holds that
\[
    \frac{\langle f,X_t\rangle}{\left(t\|X_t\|\right)^{\frac{1}{1+\beta}}}
    \xrightarrow{d} \tilde{\zeta},
    \quad t\rightarrow \infty,
\]
    where $\tilde{\zeta}$ is a $(1+\beta)$-stable random variable with
\[
    \mathbb{E} [e^{i\theta \tilde{\zeta}}]
    =\exp(\widetilde{m}[\theta f]),
    \quad \theta\in \mathbb R,
\]
where $\widetilde{m}$ is defined by \eqref{tilde-m}.
\end{thm}

\subsubsection{Small branching rate}

For $f\in \mathcal{P}$ satisfying $\alpha\beta<\kappa_f b(1+\beta)$, define
\begin{equation}
     m[f]
    =\int_0^{\infty} e^{-\alpha s} \langle \eta (-iP_s^\alpha f)^{1+\beta}, \varphi \rangle~ds, \label{msmallcase}
\end{equation}
According to Lemmd \ref{Lemma2.7} below, $m[f]$ is well-defined.
\begin{thm}
\label{Theorem13}
    If $f\in\mathcal{P}$ satisfies  $\alpha\beta<\kappa_f b(1+\beta)$,
   then for any $\mu\in \mathcal M_c(\mathbb R^d)$, under
    $\mathbb{P}_{\mu}(\cdot|D^c)$, it holds that
    $$\frac{\langle f,X_t\rangle}{\|X_t\|^{\frac{1}{1+\beta}}}\xrightarrow{d} \zeta, \quad t\rightarrow \infty,$$
    where $\zeta$ is a $(1+\beta)$-stable random variable with
    $$\mathbb{E} [e^{i\theta \zeta}]=\exp(m[\theta f]), \quad \theta\in \mathbb R,$$
    where $m$ is defined by \eqref{msmallcase}.
\end{thm}

\subsubsection{Further results in the large branching rate case}
Define
\begin{align}
   \mathcal{N}:=\{p\in \mathbb{Z}_+^d:  \alpha\beta>|p|(1+\beta)b\},
\end{align}
and
\begin{align}
    \mathcal{C}_l:=\Big\{f(x)=\sum_{p\in\mathcal{N}}a_p\phi_p(x): a_p\in \mathbb{R}, \text{and}~ x\in\mathbb{R}^d \Big\}.
\end{align}
For each fixed $f\in \mathcal{C}_l$, $f$ can be expressed as
\begin{align}\label{discompose}
    f(x)=\sum_{p\in \mathcal{N}}a_p\phi_p(x),\quad x\in \mathbb{R}^d,
\end{align}
where $a_p=\langle f, \phi_p\rangle_{\varphi}$. Let
\begin{align}\label{definition of Itf}
    I_tf(x):=\sum_{p\in \mathcal{N}}a_pe^{-(\alpha-|p|b)t}\phi_p(x), \quad t\geq 0, x\in \mathbb{R}^d.
\end{align}
It is easy to get that $I_sf(x)=\mathbb{P}_{\delta_x}[\langle I_t f, X_{t-s}\rangle]=P_{t-s}^{\alpha}I_tf(x)$.

Define
\begin{align}\label{bar-m}
   \bar{m}[f]:=\int_{0}^{\infty} e^{\alpha s}\langle \eta(iI_sf)^{1+\beta},\varphi\rangle ds.
\end{align}
According to Lemma \ref{lem: control of mn} in Subsection 4, $\bar{m}[f]$ is well-defined.
\begin{thm}\label{theorem 1.6}
For each $f\in\mathcal{C}_l$ expressed by \eqref{discompose} and $\mu\in \mathcal{M}_c(\mathbb{R}^d)$, under $\mathbb{P}_{\mu}(\cdot|D^c)$ it holds that, as $t\to\infty$,
\begin{align}\label{thm: large rate}
    \frac{\langle f, X_t\rangle-\sum_{p\in\mathcal{N}}a_pe^{(\alpha-|p|b)t}H^p_{\infty}}{\|X_t\|^\frac{1}{1+\beta}}\xrightarrow{d}\bar{\zeta},
\end{align}
where $\bar{\zeta}$ is an $(1+\beta)$-stable random variable with characteristic function
\begin{align}
    \mathbb{E}[e^{i\theta\bar{\zeta}}]=\exp[\bar{m}[\theta f]],\quad \theta\in \mathbb{R},
\end{align}
where $\bar{m}$ is defined by \eqref{bar-m}.
\end{thm}

\section{Preliminaries}
\subsection{Definition of $h$-controller}
    Define $\mathcal P^+:= \mathcal P \cap \mathcal B(\mathbb R^d, \mathbb R_+)$ and $\mathcal P^*:= \{f\in \mathcal B(\mathbb R^d, \mathbb C): |f|\in \mathcal P\}$.
    For each function $h: [0,\infty) \to [0,\infty)$, we say an operator $R$ is an $h$-controller if
\begin{itemize}
\item
    $R: \mathcal P^+ \to \mathcal P^+$;
\item
    $f, g\in \mathcal P^+$ and $f\leq g$ implies that $Rf \leq Rg$; and
\item
    $f \in \mathcal P^+$ and $\theta \in [0,\infty)$ implies that $ R (\theta f)\leq h(\theta) Rf$.
\end{itemize}
    For each function $h: [0,\infty) \to [0,\infty)$ and $\mathcal D \subset \mathcal P^*$, we say $A$ is an \emph{$h$-controllable} operator on $\mathcal D$ if
\begin{itemize}
\item
    $A : \mathcal D \to \mathcal P^*$;
\item
    There exists an $h$-controller $R$ such that $|Af| \leq R|f|$ for each $f\in \mathcal D$.
\end{itemize}
    We also say operator $A$ is \emph{$h$-controlled by $R$ on $\mathcal D$} in this case.
    We say a family of operators $(A_s)_{s\in \Lambda}$ is \emph{uniformly $h$-controllable on $\mathcal D$} if there exists an $h$-controller $R$ such that, for each $s\in \Lambda$, $A_s$ is $h$-controlled by operator $R$ on $\mathcal D$.
    For two operators $A: \mathcal D_A \subset \mathcal P^*\to \mathcal P^*$ and $B: \mathcal D_B \subset \mathcal P^*\to \mathcal P^*$, define $(A \times B)f (x):= Af(x) \times Bf(x)$ for all $f\in \mathcal D_A \cap \mathcal D_B$ and $x\in \mathbb{R}^d$.
    For any $a \in \mathbb R$, suppose that $A :\mathcal D_A \to \mathcal B(\mathbb R^d, \mathbb C\setminus (-\infty, 0])$, define $A^{\times a}f(x):= (Af(x))^a$ for all $f\in \mathcal D_A$ and $x\in \mathbb R^d$.

\begin{lem}
    \label{lem: property of controllable operators}
    Let $\mathcal D \subset \mathcal P^*$, $\Lambda$ be an index set, and $(A_\lambda)_{\lambda\in \Lambda}$ be a family of operators from $\mathcal D$ to $ \mathcal P^*$. Assume that $(A_\lambda)_{\lambda\in \Lambda}$ is uniformly $h$-controllable on $\mathcal D$ for a given function $h:[0,\infty) \to [0, \infty)$.
\begin{itemize}
\item[(1)]
    Suppose that $(\Lambda, \mathscr F)$ is a measurable space
    and that $(\lambda,x)\mapsto A_\lambda f(x)$ is $\mathscr F \otimes \mathscr B(\mathbb R^d)$-measurable for each $f\in \mathcal D$.
    For any probability measure $\nu$ on $(\Lambda, \mathscr F)$, write
\[
    A_\nu f(x):= \int_{\Lambda} A_\lambda f (x)~\nu(d\lambda), \quad f\in \mathcal D, x\in \mathbb R^d.
\]
    Then  $\{A_\nu: \nu \text{ is  a probability measure on } (\Lambda, \mathscr F)\}$ is uniformly $h$-controllable on $\mathcal D$.
\item[(2)]
    Suppose that $\Delta$ is another index set and $(B_\delta)_{\delta\in \Delta}$ is a family of operators from $\mathcal D_0\subset \mathcal P^*$ to $ \mathcal D$.
    Assume that $(B_\delta)_{\delta\in \Delta}$ is uniformly $g$-controllable on $\mathcal D_0$ for some function $g: [0,\infty) \to [0,\infty)$.
    Then  $(A_\lambda B_\delta)_{\delta\in \Delta, \lambda \in \Lambda}$ is uniformly $(h \circ g)$-controllable on $\mathcal D_0$.
\item[(3)]
    Suppose that $\Delta$ is another index set and $(B_\delta)_{\delta\in \Delta}$ is a family of operators from $\mathcal D$ to $ \mathcal P^*$.
    Assume that $(B_\delta)_{\delta\in \Delta}$ is uniformly $g$-controllable on $\mathcal D$ for some function $g: [0,\infty) \to [0,\infty)$.
    Then  $(B_\delta\times A_\lambda)_{\delta \in \Delta, \lambda \in \Lambda}$ is uniformly $(h\times g)$-controllable on $\mathcal D$, and $(B_\delta + A_\lambda)_{\delta \in \Delta, \lambda \in \Lambda}$ is uniformly $(h\vee g)$-controllable on $\mathcal D$.
\item[(4)]
    Let $a>0$. Suppose that, for each $\lambda \in \Lambda$, $A_\lambda : \mathcal D \to \mathcal B(\mathbb R^d, \mathbb C_+)$.
    Then $(A^{\times a}_\lambda)_{\lambda \in \Lambda}$ is uniformly $(h^a)$-controllable.
\end{itemize}
\end{lem}
\begin{proof}
    (1). Let $(A_\lambda)_{\lambda\in\Lambda}$ be uniformly controlled by an $h$-controller $R$ on $\mathcal D$. For all $f \in \mathcal{D}$, $x\in \mathbb R^d$ and  probability measure $\nu$ on $(\Lambda, \mathscr F)$,
\[
   |A_{\nu}f(x)|\leq \int_{\Lambda}|A_{\lambda}f(x)|\nu(d\lambda) \leq \int_{\Lambda}R|f|(x)\nu(d\lambda) \leq R|f|(x).
\]

    (2). Let $(A_\lambda)_{\lambda\in\Lambda}$ be uniformly controlled by  an $h$-controller $R_A$ on $\mathcal D$.
    Let $(B_\delta)_{\delta\in\Delta}$ be uniformly controlled by $g$-controller $R_B$ on $\mathcal D_0$.
    We verify the following:
\begin{itemize}
\item
    For each $\lambda \in \Lambda$, $\delta \in \Delta$ and $f\in \mathcal D_0$, we have $|A_\lambda B_\delta f| \leq R_A |B_\delta f| \leq R_A R_B |f|$.
\item
    If $f,g \in \mathcal D_0$ with $|f|\leq |g|$, then $R_AR_B|f| \leq R_A R_B |g|$.
\item
    If $f \in \mathcal D_0$ and $\theta \in [0,\infty)$, then $R_AR_B|\theta f| \leq R_A(g(\theta) R_B|f|) \leq (h\circ g)(\theta) R_A R_B |f| $.
\end{itemize}
    Therefore $(A_\lambda B_\delta)_{\delta\in \Delta, \lambda \in \Lambda}$ is uniformly $(h \circ g)$-controllable on $\mathcal D_0$.

    (3). Let $(A_\lambda)_{\lambda\in\Lambda}$ be uniformly controlled by an $h$-controller $R_A$ on $\mathcal D$.
    Let $(B_\delta)_{\delta\in\Delta}$ be uniformly controlled by a $g$-controller $R_B$ on $\mathcal D$.

    We verify the following:
\begin{itemize}
\item
    For each $\lambda \in \Lambda$, $\delta \in \Delta$ and $f\in \mathcal D$, we have $|A_\lambda \times B_\delta f| \leq |A_\lambda f| \cdot |B_\delta f| \leq R_A \times R_B |f|$.
\item
    If $f,g \in \mathcal D$ with $|f|\leq |g|$, then $R_A\times R_B|f| \leq R_A\times R_B |g|$.
\item
    If $f \in \mathcal D$ and $\theta \in [0,\infty)$, then $R_A\times R_B|\theta f| \leq h(\theta) g(\theta) (R_A \times R_B) |f| $.
\end{itemize}
    Therefore $(A_\lambda \times B_\delta)_{\delta\in \Delta, \lambda \in \Lambda}$ is uniformly $(h \times g)$-controllable on $\mathcal D$.

    We also verify the following:
\begin{itemize}
\item
    For each $\lambda \in \Lambda$, $\delta \in \Delta$ and $f\in \mathcal D$, we have $|(A_\lambda + B_\delta) f| \leq |A_\lambda f| + |B_\delta f| \leq (R_A + R_B) |f|$.
\item
    If $f,g \in \mathcal D$ with $|f|\leq |g|$, then $(R_A + R_B)|f| \leq (R_A + R_B) |g|$.
\item
    If $f \in \mathcal D$ and $\theta \in [0,\infty)$, then $(R_A + R_B) |\theta f| \leq h(\theta) R_A|f| + g(\theta) R_B|f| \leq (h(\theta) \vee g(\theta)) (R_A+R_B)|f| $.
\end{itemize}
    Therefore $(A_\lambda + B_\delta)_{\delta\in \Delta, \lambda \in \Lambda}$ is uniformly $(h \vee g)$-controllable on $\mathcal D$.

    (4). Let $(A_\lambda)_{\lambda\in\Lambda}$ be uniformly controlled by an $h$-controller $R$ on $\mathcal D$.
    We verify the following:
\begin{itemize}
\item
    For each $\lambda \in \Lambda$, $f\in \mathcal D$, we have $|A_\lambda^{\times a} f| = |A_\lambda f|^a  \leq R^{\times a} |f|$.
\item
    If $f,g \in \mathcal D$ with $|f|\leq |g|$, then $R^{\times a}|f| \leq R^{\times a} |g|$.
\item
    If $f \in \mathcal D$ and $\theta \in [0,\infty)$, then $R^{\times a}|\theta f| \leq (h(\theta) R |f|)^a = h(\theta)^a R^{\times a}|f|$.
\end{itemize}
    Therefore $(A_\lambda^{\times a})_{\lambda \in \Lambda}$ is uniformly $(h^a)$-controllable on $\mathcal D$.
\end{proof}

\subsection{h-controller  for super-OU process.}
Let $X$ be the super-OU process introduced in Subsection 1.1. Write $\psi_0(z)=\psi(z)+\alpha z$, for each $z\in \mathbb{C}_+$, and define operators on $\mathcal{B}(\mathbb{R}^d,\mathbb{C}_+)$ by
\begin{equation}\begin{split}
    \Psi f (x) &:= -\alpha f(x) + \eta f(x)^{1+\beta} + \psi_1\big(f(x)\big),
    \\\Psi_0 f(x) &:= \Psi f(x) + \alpha f(x)
    \\ \Psi_1 f(x) &:= \psi_1 \big(f(x)\big),
    \quad  f\in \mathcal B(\mathbb R^d, \mathbb C_+), x\in \mathbb R^d.
\end{split}\end{equation}
For all $t\in [0,\infty), x\in \mathbb R^d $ and $f \in \mathcal{P}$, recall that $U_tf(x) := \operatorname{Log} \mathbb P_{\delta_x}[e^{i\theta \langle f, X_t\rangle}]|_{\theta = 1}$,
the value of the characteristic exponent of the infinite divisible random variable
$\langle f, X_t\rangle$ at $1$.
    From \eqref{eq: -v has positive real part}, we know that $-U_tf(x)$ takes values in $\mathbb C_+$. Furthermore, we know from Proposition \ref{prop: complex FKPP-equation} that
\begin{equation}\begin{split}
\label{eq:chareq2}
    U_tf(x)-\int_0^t P^\alpha_{t-s} \Psi_0(-U_sf)(x)ds
    =i P^{\alpha}_t f(x),
    \quad t\in [0,\infty), x\in \mathbb{R}^d, f\in \mathcal P.
\end{split}\end{equation}
If let
\begin{equation}\begin{split}
\label{eq: def of Zf}
    Z_t f
    &:= \int_0^t P^\alpha_{t-s}\big( \eta (-i P^\alpha_sf)^{1+\beta}\big)ds,
    \\ Z'_t f
    &:= \int_0^t P^\alpha_{t-s}\big( \eta (-U_s f)^{1+\beta}\big)ds,
    \\ Z''_t f
    &:= \int_0^t P^\alpha_{t-s}\Psi_1(-U_s f)ds,
    \\ K_t f
    &:= (Z'_t - Z_t+ Z''_t)f,
\end{split}\end{equation}
    Then we have that
\[
    U_t - i P^\alpha_t
    = Z'_t + Z''_t
    = Z_t+ (Z'_t - Z_t) + Z''_t
    = Z_t+K_t, \quad t\geq 0.
\]

\begin{lem}
\label{lem: upper bound for usgx}
    Under Assumption \ref{asp: branching mechanism}, the following statements are true:
\begin{itemize}
\item[(1)]
   $(-U_t)_{0\leq t\leq 1}$ is uniformly $\theta$-controllable from $\mathcal P$ to $\mathcal P^*\cap \mathcal B(\mathbb R^d, \mathbb C_+)$.
\item[(2)]
    $(P^\alpha_t)_{0\leq t\leq 1}$ is uniformly $\theta$-controllable on $\mathcal P^*$.
\item[(3)]
    $\Psi_0$ is $(\theta^2\vee \theta^{1+\beta})$-controllable from $\mathcal P^* \cap \mathcal B(\mathbb R^d, \mathbb C_+)$ to $\mathcal P^*$.
\item[(4)]
    $(U_t- iP_t^{\alpha})_{0\leq t\leq 1}$ is uniformly $(\theta^2\vee \theta^{1+\beta})$-controllable on $\mathcal P$.
\item[(5)]
    $(Z'_t-Z_t)_{0\leq t\leq 1}$ is uniformly $(\theta^{2+\beta}\vee \theta^{1+2\beta})$-controllable from $\mathcal P$ to $\mathcal P^*$.
\item[(6)]
    There is a constant $\delta > 0$ such that $(Z''_t)_{0\leq t\leq 1}$ is uniformly $(\theta^2\vee \theta^{1+\beta+\delta})$-controllable from $\mathcal P$ to $\mathcal P^*$.
\item[(7)]
    There is a constant $\delta > 0$ such that $(K_t)_{0\leq t\leq 1}$ is uniformly $(\theta^{2+\beta}\vee \theta^{1+\beta+\delta})$-controllable from $\mathcal P$ to $\mathcal P^*$.
\end{itemize}
\end{lem}

\begin{proof}
    (1). According to \eqref{eq: -v has positive real part}, $U_t: \mathcal P \to \mathcal B(\mathbb R^d, \mathbb C_+)$.
    It follows from \eqref{eq: upper bound for vf} that for all $g\in \mathcal P$, $0\leq t\leq 1$ and $x\in \mathbb R^d$,
\[
    |U_t g(x)|
    \leq \sup_{0\leq u\leq 1}P_u^\alpha |g| (x).
\]
    Note that $f\mapsto\sup_{0\leq u\leq 1}P^{\alpha}_u f$ is a $\theta$-controller.

    (2). Similar to the proof of (1).

    (3). Let $C>0$ and $\delta > 0$ be the constant in Assumption \ref{asp: branching mechanism}.
    For each $f\in \mathcal P^* \cap \mathcal B(\mathbb R^d, \mathbb C_+)$,
\begin{equation}\begin{split}
    &|\Psi_0 f|
    \leq \eta |f|^{1+\beta} + |\Psi_1 f|
    \leq \eta |f|^{1+\beta} + C|f|^2+ C|f|^{1+\beta + \delta}
\end{split}\end{equation}
    Note that
\[
    f \mapsto \eta f^{1+\beta} + Cf^2+ Cf^{1+\beta + \delta},\quad f\in \mathcal P^+
\]
     is a $(\theta^2 \vee \theta^{1+\beta})$-controller.

    (4). From (1), (2), (3) and Lemma \ref{lem: property of controllable operators}.(2), we know that the operators
\[
    f
    \mapsto P^\alpha_{t-s}\Psi_0(-U_sf)(x),
    \quad 0\leq s\leq t\leq 1
\]
    are uniformly $(\theta^2\vee \theta^{1+\beta})$-controllable.
    Combining this with \eqref{eq:chareq2} and Lemma \ref{lem: property of controllable operators}.(1), we get the desired result.

    (5). Notice that from Lemma \ref{lem: Lip of power function},
\[
    |(-U_t f)^{1+\beta} - (-iP^\alpha_t f)^{1+\beta} |
    \leq  (1+\beta) |U_t f-iP^\alpha_t f|(|U_t f|^{\beta}+|i P^\alpha_t f|^{\beta}).
\]
    Now using (1), (2), (4) and Lemma \ref{lem: property of controllable operators}.(3)-(4), we get that
\[
    f \mapsto (-U_t f)^{1+\beta} - (-iP^\alpha_t f)^{1+\beta},\quad 0\leq t\leq 1,
\]
    are unifromly $(\theta^{2+\beta}\vee \theta^{1+2\beta})$-controllable.
    As a consequence of this, Lemma \ref{lem: property of controllable operators}.(1)-(2), and
\begin{equation}\begin{split}
    (Z'_t - Z_t)f = \int_0^t P^\alpha_{t-s}\Big( \eta \big((-U_s f)^{1+\beta} - (-iT_s^\alpha f)^{1+\beta} \big)\Big)ds,
    \quad 0\leq t\leq 1, f\in \mathcal P,
\end{split}\end{equation}
    we get the desired result.

    (6). Suppose that $C>0, \delta>0$ are constants in Assumption \ref{asp: branching mechanism}.
    Noticing that
\[
    |\Psi_1(f)|
    \le C(|f|^2+|f|^{1+\beta+ \delta}),
    \quad f\in \mathcal P^*\cap\mathcal B(\mathbb R^d, \mathbb C_+)
\]
    and that
\[
    f\mapsto C(f^2+f^{1+\beta+\delta}),
    \quad f\in \mathcal P^+
\]
    is a $(\theta^2 \vee \theta^{1+\beta+\delta})$-controllor, we have that $\Psi_1$ is $(\theta^2 \vee \theta^{1+\beta+\delta})$-controllable from $\mathcal P^*\cap\mathcal B(\mathbb R^d, \mathbb C_+)$ to $\mathcal P^*$.
    According to this, (1), (2), and Lemma \ref{lem: property of controllable operators} (1)-(2), we have that
\[
    f
    \mapsto Z_t'' f
    = \int_0^t P_{t-s}^\alpha \Psi_1(-U_sf)ds,
    \quad 0\leq t\leq 1,
\]
    are uniformly $(\theta^2 \vee \theta^{1+\beta+\delta})$-controllable from $\mathcal P$ to $\mathcal P^*$.

(7). Since $K_t = (Z'_t-Z_t)+Z''_t$, we get the desired result from (5), (6), Lemma \ref{lem: property of controllable operators}(3).
\end{proof}

\subsection{Parameters of the stable limits}
   Define
 \begin{equation}\begin{split}
 \label{parameter_mk}
      m_t[f]
      :=e^{-\alpha (t+1)} \langle (Z_1P^\alpha_t f), \varphi\rangle,\quad t\geq 0, f\in \mathcal P.
 \end{split}\end{equation}

 We first collect some lemmas of \cite{MM}, which will be used in this paper.
 \begin{lem}(\cite[Lemma 2.7]{MM})\label{Lemma2.7} For each $f\in\mathcal{P}$, there is a constant $C>0$ such that
\begin{equation}\label{domi-m}
    |m_t[f]|
    \leq C e^{(\alpha\beta-\kappa_fb(1+\beta))t},
    \quad t\geq 0.
\end{equation}
\end{lem}
 Recall the definition of $\widetilde{m}$ in \eqref{tilde-m}.
  \begin{lem}(\cite[Lemma 4.2]{MM})
    If $f \in \mathcal{P}$ satisfies $\alpha\beta=\kappa_f b(1+\beta)$,
    we have
\begin{equation}\begin{split}
\label{para: critical case}
 \widetilde{m}[f]:=\lim_{t\rightarrow \infty}\frac{1}{t}\sum_{k=0}^{\lfloor t \rfloor}m_k[f]=\langle\eta(-i\phi)^{1+\beta},\varphi\rangle,
\end{split}\end{equation}
    where
\[
    \phi
    :=\sum_{p\in \mathbb Z_+^d:|p|=\kappa_f}\langle f, \phi_p\rangle\phi_p.
\]
\end{lem}

%moved from below
\begin{lem}\label{lem: control of gk}(\cite[Lemma 2.8]{MM})
    For each $f\in \mathcal{P}$, let
    \begin{align}
        g_k
    := \frac{Z_1 P^{\alpha}_k f-\langle  Z_1P^{\alpha}_k f,\varphi\rangle}{e^{(\alpha-\kappa_f b)(1+\beta)t}},
    \quad k \geq 0.
    \end{align}
    There exist $h\in \mathcal{P}$ such that
    \begin{align}
        |P_sg_k|\leq e^{-bs}h,\quad s\geq 0, k \geq 0.
    \end{align}
\end{lem}
% end move
%moved from below
\begin{lem}\label{control of gn} Suppose that $g\in \mathcal{C}_l$.
    Put
    \begin{align}
        \bar{g}_t:=\frac{Z_1(-I_tg)-\langle Z_1(-I_tg),\varphi\rangle}{e^{-(\alpha-Kb)(1+\beta)t}},\quad t\geq 0.
    \end{align}
    There exists $h\in \mathcal{P}$ such that
    \begin{align}
        |P_s\bar{g}_t|\leq e^{-bs}h,\quad s\geq 0, t\geq 0.
    \end{align}
\end{lem}
\begin{proof}
    The proof  is similar to that of \cite[Lemma 2.8]{MM}.
\end{proof}
% end move



Next lemma is an criterion of $(1+\beta)$-stable distribution.

\begin{lem}
\label{lem: charactreisticfunction}
    Let $q$ be a measure on $\mathbb R^d\setminus\{0\}$ with $\int |x|^{1+\beta} q(dx) \in (0,\infty)$.
    Then $$\theta \mapsto  \exp\Big\{\int_{\mathbb R^d\setminus\{0\}} (i\theta \cdot x)^{1+\beta} q(dx)\Big\},\quad \theta \in \mathbb R^d$$
    is the characteristic function of an $\mathbb R^d$-valued $(1+\beta)$-stable random variable.
\end{lem}
\begin{proof}
    From measure theory, there exist a measure $\lambda$ on $S:= \{\xi\in \mathbb R^d:|\xi| = 1\}$ and a kernel $k(\xi,dt)$ from $S$ to $\mathbb R_+$ such that
\[
    \int_{\mathbb R^d\setminus \{0\}} f(x)q(dx) = \int_S \lambda(d\xi) \int_{\mathbb R^+} f(\xi t)k(\xi,dt),\quad
    f\in \mathcal B(\mathbb R^d\setminus \{0\}, \mathbb R_+).
\]
Recall the definition of $\Gamma$ function in \eqref{eq: definition of Gamma function}, and define another measure $\lambda_0$ on $S$ by
\[
    \lambda_0(d\xi) := \frac{\int_0^\infty t^{1+\beta}k(\xi,dt)}{\Gamma(-1-\beta)} \lambda (d\xi).
\]
    Then $\lambda_0$ is a non-zero finite measure, since
\[
    \lambda_0(S) = \frac{1}{\Gamma(-1-\beta)} \int_S \lambda (d\xi) \int_0^\infty |t\xi|^{1+\beta}k(\xi,dt)
    = \frac{1}{\Gamma(-1-\beta)} \int_{\mathbb R^d\setminus\{0\}} |x|^{1+\beta} q(dx) \in (0,\infty).
\]
    Define a measure $\nu$ on $\mathbb R^d\setminus\{0\}$ by
\[
    \int_{\mathbb R^d\setminus\{0\}}f(x)\nu(dx)= \int_{S} \lambda_0(d\xi) \int_0^\infty f(r\xi) \frac{dr}{r^{2+\beta}} .
\]
    Then, according to \cite[Remark 14.4]{Sato1999Levy}, $\nu$ is the L\`evy measure of a $(1+\beta)$-stable distribution on $\mathbb R^d$, say $\mu$, whose characteristic function is \[\hat \mu(\theta)=\exp\Big\{\int_{\mathbb R^d\setminus\{0\}} (e^{-i\theta \cdot y}-1+i\theta \cdot y) \nu(dy)\Big\}.\]
	Finally,
\begin{equation}\begin{split}
    &\int_{\mathbb R^d\setminus\{0\}} (e^{-i\theta \cdot y}-1+i\theta \cdot y) \nu(dy)
    = \int_S \lambda_0(d\xi) \int_0^\infty (e^{-ir\theta \cdot \xi}-1+ir\theta \cdot \xi) \frac{dr}{r^{2+\beta}}
\\&\quad = \int_S \lambda (d\xi) \int_0^\infty (e^{-ir\theta \cdot \xi}-1+ir\theta \cdot \xi) \frac{dr}{\Gamma(-1-\beta)r^{2+\beta}}\int_0^\infty t^{1+\beta} k(\xi,dt)
\\&\quad = \int_S \lambda (d\xi) \int_0^\infty (i\theta\cdot \xi)^{1+\beta} t^{1+\beta} k(\xi,dt)
= \int_S \lambda(d\xi) \int_0^\infty (i\theta \cdot t\xi)^{1+\beta} k(\xi,dt)
\\&\quad = \int_{\mathbb R^d} (i\theta \cdot x)^{1+\beta} q(dx).
\qedhere
\end{split}\end{equation}
\end{proof}

For each $f\in \mathcal{P}$ satisfying $\alpha\beta<\kappa_fb(1+\beta)$,
According to \eqref{domi-m}, $\sum_{k=0}^\infty |m_k[f]|<\infty$, and  then
\begin{equation}\label{sum-m}
m[f]
    :=\sum_{k=0}^\infty m_k[f]
    =\int_0^{\infty} e^{-\alpha s} \langle \eta (-iP_s^\alpha f)^{1+\beta}, \varphi \rangle~ds.
\end{equation}

\begin{prop}
\label{cor: alpha stable rv}
	For all $t\geq 0$ and $f\in \mathcal P$,
\begin{enumerate}
\item
\label{it: first stable}
    $\theta \mapsto \exp(m_t[\theta f])$ is the characteristic function of a $(1+\beta)$-stable random variable;
\item
\label{it: second stable}
    $\theta \mapsto \exp(\tilde m[\theta f])$ is the characteristic function of a $(1+\beta)$-stable random variable, provided $\alpha\beta=\kappa_f b(1+\beta)$;
\item
\label{it: third stable}
    $\theta \mapsto \exp(m[\theta f])$ is the characteristic function of a $(1+\beta)$-stable random variable, provided $\alpha\beta < \kappa_f b(1+\beta)$.
\end{enumerate}
\end{prop}
\begin{proof}
    \eqref{it: first stable}:
	Fix $t\geq 0$ and $f\in \mathcal P$.
	Note that $m_t[\theta f]$ can be rewritten as
\[
    m_t[\theta f]= e^{-\alpha (t+1)}\int_{\mathbb R^d} dx~\varphi(x)\int_0^1 P_{1-s}^\alpha (-i\theta P_{s+t}^\alpha f)^{1+\beta}(x)~ds,
    \quad \theta \in \mathbb R.
\]
	Therefore, according to Lemma \ref{lem: charactreisticfunction}, we only need to show that
\begin{equation}
\label{eq: what I want to proof}
	\int_{\mathbb R^d} dx~\varphi(x)\int_0^1 P_{1-s}^\alpha (|P_{s+t}^\alpha f|^{1+\beta})(x)~ds < \infty.
\end{equation}
	According to Lemma \ref{lem: upper bound for usgx}.(2), Lemma \ref{lem: property of controllable operators}.(1)--(2) and Lemma \ref{lem: property of controllable operators}.(4),  we know that
$
	g \mapsto \int_0^1 P_{1-s}^\alpha (|P_{s}^\alpha g|^{1+\beta})~ds
$
	is a $(1+\beta)$-controllable operator on $\mathcal P$.
	This implies that $x \mapsto \int_0^1 P_{1-s}^\alpha (|P_{s+t}^\alpha f|^{1+\beta})(x) ds$ is an element of $\mathcal P$.
	Therefore \eqref{eq: what I want to proof} is true.

    The proofs of \eqref{it: second stable} and \eqref{it: third stable}  is similar to that of \eqref{it: first stable}.
\end{proof}
Recall that
\begin{align}
   \mathcal{N}=\{p\in \mathbb{Z}_+^d:  \alpha\beta>|p|(1+\beta)b\},
\end{align}
and
\begin{align}
    \mathcal{C}_l=\Big\{g(x)=\sum_{p\in\mathcal{N}}a_p\phi_p(x): a_p\in \mathbb{R}, \text{and}~ x\in\mathbb{R}^d \Big\}.
\end{align}

For each $g\in \mathcal{C}_l$ we have the following expression:
    $$g(x)=\sum_{p\in \mathcal{N}}a_p\phi_p(x), \quad x\in\mathbb{R}^d.$$

Also recall that the definition of $I_tg(x)$:
\begin{align}
    I_tg(x)=\sum_{p\in \mathcal{N}}a_pe^{-(\alpha-|p|b)t}\phi_p(x), \quad t\geq 0, x\in \mathbb{R}^d.
\end{align}
Define
\begin{align}
    K=\sup\{k\in \mathbb{N}: \alpha\beta>k(1+\beta)b\}.
\end{align}

\begin{lem}\label{lem: control of Isg}
 For each $g\in \mathcal{C}_l$, there exists $h\in \mathcal{P}$ such that  for any $t\geq 0$,
\begin{align}
    |I_tg(x)|\leq e^{-(\alpha-Kb)t}h(x),\quad x\in \mathbb{R}^d.
\end{align}
\end{lem}
\begin{proof}
Suppose $g(x)=\sum_{p\in \mathcal{N}}a_p\phi_p(x)$, $x\in\mathbb{R}^d$.  Put  $h=\sum_{p\in \mathcal{N}}|a_p\phi_p|$.
\begin{align}
    |I_tg(x)|=\Big|\sum_{p\in\mathcal{N}}a_pe^{-(\alpha-|p|b)t}\phi_p(x)\Big|\leq e^{-(\alpha-Kb)t}h(x),\quad t\geq 0,x\in \mathbb{R}^d.
\end{align}
\end{proof}
For each  $n\geq 1$, let
\begin{align}
    \bar{m}_n[g]:=\int_{n-1}^n e^{\alpha s}\langle \eta(iI_sg)^{1+\beta},\varphi\rangle ds.
\end{align}
\begin{lem}\label{lem: control of mn}
For each $g\in \mathcal{C}_l$,  there exists constant $C>0$ such that
 \begin{align}
     |\bar{m}_n[g]|< C e^{-(\alpha\beta-K(1+\beta)b)n}, \quad n\geq 1,
 \end{align}
which implies that
\begin{equation}\label{sum-bar-m}\bar{m}[g]:=\int_{0}^\infty e^{\alpha s}\langle \eta(iI_sg)^{1+\beta},\varphi\rangle ds=\sum_{n=1}^{\infty}\bar{m}_n[g]\end{equation}
is well-defined.
\end{lem}
\begin{proof}
Let $h$ be the control function in Lemma \ref{lem: control of Isg}.
\begin{align}
    &|\bar{m}_n[g]|\leq \int_{n-1}^n e^{\alpha s}\langle |I_sg|^{1+\beta}, \varphi\rangle ds\leq \int_{n-1}^n e^{\alpha s}e^{-(\alpha-Kb)(1+\beta)s}ds\cdot \langle h^{1+\beta}, \varphi\rangle\\
    &\leq \int_{n-1}^n e^{-(\alpha\beta-K(1+\beta)b)s}ds\cdot\langle h^{1+\beta}, \varphi\rangle
    \leq C e^{-(\alpha\beta-K(1+\beta)b)n}.
\end{align}
By the fact that $\alpha\beta>K(1+\beta)b$, we easily get
$\sum_{n=1}^{\infty}|\bar{m}_n[g]<\infty$ and that  $$\int_{0}^\infty e^{\alpha s}\langle \eta(iI_sg)^{1+\beta},\varphi\rangle ds=\sum_{n=1}^{\infty}\bar{m}_n[g].$$
\end{proof}
\begin{prop}
\label{cor: alpha stable rv 1}
For any $n\geq 0$ and $g\in \mathcal C_l$ given by \eqref{discompose}, we have
\begin{enumerate}
\item
\label{it: first stable}
    $\theta \mapsto \exp(\bar{m}_n[\theta g])$ is the characteristic function of a $(1+\beta)$-stable random variable;
\item
\label{it: second stable}
    $\theta \mapsto \exp(\bar m[\theta g])$ is the characteristic function of a $(1+\beta)$-stable random variable.
\end{enumerate}
\end{prop}
\begin{proof}
The proof is similar to that of Proposition \ref{cor: alpha stable rv}. We omit the details here.
\end{proof}

\section{Tail probability}
In this section, we will introduce a small value probability result of supercritical continuous-state branching process with general branching mechanism and
$(1+\gamma)$-moments for the super-OU process. These will be needed to prove our main results.


\subsection{Small value probability}
    In this subsection, we digress briefly from super-OU process and consider a supercritical continuous-state branching process (CSBP)
        $\{(Y_t)_{t\geq 0}; \mathbf P_x\}$.
    Our goal is to determining how fast the probability $\mathbf P_x(0<e^{-\alpha t}Y_t \leq k_t)$ convergence to $0$ while $k_t \to 0$ as $t\to \infty$.
   Assume that the branching mechanism of the CSBP $\{(Y_t)_{t\geq 0}; \mathbf P_x\}$ is given by \eqref{mechanism}.
    Suppose that the Grey's condition is satisfied, i.e. there is some constant $z' > 0$ such that $\psi(z) > 0$ for all $z>z'$, and that $\int_{z'}^\infty \psi(z)^{-1}dz < \infty$.
    Also suppose that the $L \log L$ critierion is satisfied, i.e.
\[
    \int_1^\infty y \log y~\pi(dr)< \infty.
\]
    Write $W_t = e^{-\alpha t}Y_t$ for each $t\geq 0$.
\begin{prop}
\label{lem: control of XT}
    Suppose that $t\mapsto k_t$ is a strictly positive map on $[0,\infty)$ with $k_t \to 0$ and $k_t e^{\alpha t} \to \infty$ as $t\to \infty$.
    Then, for each $x\geq 0$, there are constants $C,\delta>0$ such that
\[
    \mathbf P_x(0<W_t\leq k_t) \leq C(k_t^\delta + e^{-\delta t}), \quad t\geq 0.
\]
\end{prop}
\begin{proof}
    Step 1. We recall some known facts about the CSBP $(Y_t)$.
    For each $\lambda \geq 0$, denote by $t\mapsto v_t(\lambda)$ the unique positive solution of
\[
    v_t(\lambda)
    = \lambda - \int_0^t \psi(v_s(\lambda))ds,\quad t\geq 0.
\]
    Then according to the definition of CSBP, we have
\[
    \mathbf P_x[e^{-\lambda Y_t}] = e^{-xv_t(\lambda)}, \quad t\geq 0, \lambda \geq 0, x\geq 0.
\]
    Taking $\lambda \to \infty$ in the above equation, we have by monotonicity that $\bar v_t:= \lim_{\lambda \to \infty}v_t(\lambda)$ exists in $(0,\infty]$ for all $t\geq 0$, and that
\begin{equation}
\label{eq: svp1}
    \mathbf P_x(Y_t = 0)=e^{-x\bar v_t}, \quad t\geq 0, x\ge 0.
\end{equation}
    It is known, see \cite[Theorem 3.5-3.8]{Li2011Measure-valued} for example, that under the Grey's condition, the following statements are true:
\begin{itemize}
\item
    $0\leq \bar v_t < \infty$ for all $t>0$.
\item
    $t\mapsto \bar v_t$ is decreasing and $\bar v:= \lim_{t\to \infty} \bar v_t \in [0,\infty)$ is the largest root of $\psi(z) = 0$.
\end{itemize}
    Taking $t \to \infty$ in \eqref{eq: svp1}, we have by monotonicity that
\[
    \mathbf P_x(\exists t \geq 0, Y_t = 0)
    = e^{-x\bar v}, \quad x\geq 0.
\]

    Note that $\psi$ has derivative
\[
    \psi'(z)
    = -\alpha + 2\rho z + \int_{(0,\infty)}(1-e^{-zy})y\pi(dy),\quad z\geq 0,
\]
    which is increasing in $z$.
    This says that $\psi$ is a convex function.
    Also notice that $\psi'(0+)=-\alpha <0$ and that there exists $z>0$ such that $\psi(z)>0$.
    So, we can verify that the following are true:
\begin{itemize}
\item
    $\bar v > 0$ ,
\item
    $\psi(z) < 0$ on $z\in (0,\bar v)$,
\item
    $\psi(z) > 0 $ on $z\in (\bar v, \infty)$.
\end{itemize}
    It is also known, see \cite[Proposition 3.3]{Li2011Measure-valued} for example, that
\begin{itemize}
\item
    if $\lambda \in (0,\bar v)$ then $0<\lambda \leq v_t(\lambda)<\bar v $, and that
\begin{align}
\label{CSBP: int}
    \int_{\lambda}^{v_t(\lambda)} \frac{dz}{-\psi(z)} = t, \quad t\geq 0.
\end{align}
\item
    if $\lambda \in (\bar v, \infty)$ then $\bar v < v_t(\lambda)\leq \lambda< \infty $, and that
\[
  \int_{v_t(\lambda)}^\lambda\frac{dz}{\psi(z)} = t, \quad t\geq 0.
\]
\end{itemize}
    By monotonicity, we have that
\begin{equation}
\label{eq:svp2}
    \int_{\bar v_t}^\infty \frac{dz}{\psi(z)} = t, \quad t\geq 0.
\end{equation}

    Step 2. We will show that, for each $x \geq 0$ there exists a constant $C>0$ such that
\[
    \mathbf P_{x}(0< W_t\leq k_t)
    \leq C\big(|\bar v- v_t(k_t^{-1}e^{-\alpha t})|+|\bar v_t - \bar v|\big),
    \quad t\geq 0.
\]
In fact, for each $x\geq 0$ and $t\geq 0$ we have
\begin{equation}\begin{split}
    &\mathbf P_{x}(0<W_t \leq k_t)
    = \mathbf P_{x}( e^{-k_t^{-1}W_t}\geq e^{-1},W_t > 0)
    \\&\leq e \mathbf P_{x}[e^{-k_t^{-1} W_t};W_t > 0]
    =  e\big(\mathbf P_x[e^{-k_t^{-1} W_t}]-\mathbf P_x(W_t = 0)\big)
    \\ &= e\big(e^{-xv_t(k_t^{-1} e^{-\alpha t})}-e^{-x\bar v_t}\big)
    \\&\leq ex \big(|\bar v-v_t(k_t^{-1} e^{-\alpha t})|+ |\bar v_t- \bar v|\big)
\end{split}\end{equation}
    as required in this step.

    Step 3. We will show that there exists constants $c, \delta, t_0 > 0$ such that
\[
    |\bar v_t-\bar v|
    \leq ce^{-\delta t},
    \quad t\geq t_0.
\]
    In fact, since $\psi$ is a convex function, $\psi(z)$ is positive for $z$ large enough, and $\bar v$ is the largest root of $\psi(z)=0$, we must have $\tau:=\psi'(\bar v)>0$, and that  $\psi(z) \geq (z-\bar v)\tau$ for any $z\geq 0$.
    According to Grey's condition, we can find a constant $z_0 >\bar v $ such that
\[
    t_0
    := \int^\infty_{z_0} \frac{dz}{\psi(z)} < \infty.
\]
    For each $t > t_0$, according to \eqref{eq:svp2}, we have
\begin{equation}\begin{split}
    &t-t_0 = \int^\infty_{\bar v_t} \frac{dz}{\psi(z)} - \int_{z_0}^\infty \frac{dz}{\psi(z)}
    =\int_{\bar v_t}^{z_0} \frac{dz}{\psi(z)}
    \\ &\leq \int_{\bar v_t}^{z_0} \frac{dz}{(z-\bar v)\tau}
    = \frac{1}{\tau} \big(\log (z_0-\bar v) - \log(\bar v_t-\bar v)\big).
\end{split}\end{equation}
    Rearrange this, we have
\[
    \bar v_t - \bar v \leq (z_0 - \bar v)e^{-\tau(t-t_0)},
    \quad t\geq t_0.
\]
    This implies the desired result in this step.

    Step 4: We will show that there exists constant $C, \delta,t_0>0$ such that
\[
    |\bar v - v_t(k_t^{-1} e^{-\alpha t})|\leq Ck_t^\delta, \quad t\geq t_0.
\]
    Define
\[
    \rho_t := 1+ \frac{\log k_t}{t\alpha}, \quad t\geq 0.
\]
    According to the fact that
\[
    k_t^{-1}e^{-\alpha t} = e^{-\alpha \rho_t t}, \quad t\geq 0,
\]
    and the condition that $k_t e^{\alpha t} \xrightarrow[t\to \infty]{} \infty$, we have $\rho_t t \xrightarrow[t\to \infty]{} \infty $.
    Since the $L\log L$ criterion is satisfied, we have (see \cite{LiuRenSong2009Llog} for example), $W_t \xrightarrow[t\to \infty]{a.s.} W_\infty$ where the martingale limit $W_\infty$ is a non-degenerate random variable. This implies that
\[
    v_t(e^{-\alpha t}) = -\log \mathbf P_1[e^{-W_t}]\xrightarrow[t\to \infty]{} - \log \mathbf P_{1}[e^{-W_\infty}] =: z^* \in (0,\infty).
\]
    The $L \log L$ critierioin also garuntees that (see again \cite{LiuRenSong2009Llog} for example) $\{W_\infty = 0\} = \{\exists t \geq 0, X_t= 0\}$  a.s.ly in $\mathbf P_1$. This and the non-degeneracy of $W_\infty$ implies that
\[
    z^*=-\log \mathbf P_1[e^{-W_\infty}] < -\log \mathbf P_1(W_\infty = 0) = \bar v.
\]

    Fix an arbitrary $\epsilon \in (0,\tau)$.
    According to the fact that $\tau=\psi'(\bar v)>0$ , there exists $z_0 \in (0,\bar v)$ such that for all $z\in (z_0, \bar v)$, we have $-\psi(z)\geq (\bar v - z)(\tau- \epsilon)$.     Fix this $z_0$.
    For $t$ large enough, we have $0<k_t^{-1}e^{-\alpha t} < v_t(k_t^{-1}e^{-\alpha t})< \bar v$. Then  using \eqref{CSBP: int} with $\lambda=k_t^{-1} e^{-\alpha t}$, we have
\begin{equation}\begin{split}
    &t =\int^{v_t(k_t^{-1} e^{-\alpha t})}_{k_t^{-1} e^{-\alpha t}}\frac{dz}{-\psi(z)}
    \\&= \Big(\int^{v_{\rho_t t}(e^{-\alpha \rho_t t})}_{e^{-\alpha \rho_t t}}  + \int^{z_0}_{v_{\rho_t t}(e^{-\alpha \rho_t t})} +\int^{v_t(k_t^{-1}e^{-\alpha  t})}_{z_0}\Big)\frac{dz}{-\psi(z)}
     \\&= \rho_t t + O(1) +\int^{v_t(k_t^{-1}e^{-\alpha t})}_{z_0} \frac{dz}{-\psi(z)},
\end{split}\end{equation}
    where we used the fact that
\[
    \int_{v_{\rho_t t}(e^{-\alpha \rho_tt})}^{z_0} \frac{dz}{-\psi(z)} \xrightarrow[t\to \infty] {}\int_{z^*}^{z_0} \frac{dz}{-\psi(z)}.
\]
    Now we have, for $t$ large enough,
\begin{equation}\begin{split}
    &t\leq  \rho_t t + O(1)+ \int_{z_0}^{v_t(k_t^{-1}e^{-\alpha t})} \frac{dz}{(\bar v-z)(\tau - \epsilon)}
    \\&=  \rho_t t +O(1)- \frac{1}{\tau-\epsilon}\Big( \log \big(\bar v-v_t(e^{-\alpha \rho_t t})\big) - \log(\bar v-z_0)\Big).
\end{split}\end{equation}
    Rearrange this, we have, for $t$ large enough,
\[
    e^{-t(\tau - \epsilon)} \geq e^{-\rho_t t(\tau - \epsilon)+O(1)}(\bar v - v_t(e^{-\alpha \rho_t t})).
\]
    Therefore, there exists constants $C>0$ and $t_0>0$ such that for all $t\geq t_0$,
\[
    \bar v - v_t(k_t^{-1} e^{-\alpha t}) \leq e^{-t(\tau -\epsilon)+ (1+\frac{\log k_t}{t\alpha})t(\tau - \epsilon)+O(1)}
    \leq Ck_t^{\frac{\tau - \epsilon}{\alpha}}.
\]
    This implies the required result in this step.

    Finally, according to the results in Steps 2-4, we have for each $x\geq 0$, there exists constant $C, \delta, t_0 > 0$ such that
\[
    \mathbf P_{x}(0< W_t\leq k_t) \leq C(k_t^{\delta}+e^{-\delta t}),\quad t\geq t_0.
\]
Note that the left side is always bounded by $1$, so we can take $t_0 =0$ in the above statement.
\end{proof}
\subsection{$(1+\gamma)$-moments for the super-OU process}

 In this subsection,  we want to bound the $(1+\gamma)$-th moment of $\langle g ,X_t \rangle$ for $\gamma \in (0,\beta)$.
    For all $0 \leq s \leq t <\infty$ and  random variable $Y$ with finite mean, write
$$
    \mathcal I_s^t Y
  := \mathbb P_\mu[Y|\mathscr F_t] - \mathbb P_\mu[Y|\mathscr F_s].
$$

\begin{lem}
\label{lem: control pair for P(M>lambda)}
    There is a $(\theta^2\vee\theta^{1+\beta})$-controller $R$ such that for all $0\leq t\leq 1$, $g\in \mathcal P$, $\lambda >0$ and $\mu\in \mathcal M_c(\mathbb R^d)$, we have
\[
    \mathbb P_\mu ( |\mathcal{I}_0^t\langle g,X_t\rangle| > \lambda)
    \leq \frac{\lambda}{2}\int_{-2/\lambda}^{2/\lambda}\langle R|\theta g|,\mu\rangle d\theta.
\]
\end{lem}

\begin{proof}
    Denote by $R$ the $(\theta^2\vee\theta^{1+\beta})$-controller for Lemma \ref{lem: upper bound for usgx}.(4).
    Using Lemma \ref{lem: estimate of exponential remaining} and the argument in the proof \cite[Theorem 3.3.6]{Durrett2010Probability}, we get
\begin{equation}\begin{split}
    &\big|\mathbb P_\mu (|\mathcal{I}_0^t\langle g,X_t\rangle| > \lambda)\big|
    \leq \Big|\frac{\lambda}{2}\int_{-2/\lambda}^{2/\lambda}(1 - \mathbb P_\mu[e^{i\theta \mathcal{I}_0^t\langle g,X_t\rangle}])d\theta\Big|
    \\&\leq \frac{\lambda}{2}\int_{-2/\lambda}^{2/\lambda}|1-e^{\langle U_t(\theta g)-iP^\alpha_t (\theta g),\mu \rangle}|d\theta
    \leq \frac{\lambda}{2}\int_{-2/\lambda}^{2/\lambda}\langle |U_t(\theta g) - iP^\alpha_t(\theta g)|,\mu\rangle d\theta
    \\&\leq \frac{\lambda}{2}\int_{-2/\lambda}^{2/\lambda}\langle R|\theta g|,\mu\rangle d\theta.
      \qedhere
\end{split}\end{equation}
\end{proof}
    For all $\kappa \in \mathbb Z_+$ and $f\in \mathcal P$, define
\begin{equation}\label{Q_k}
Q_\kappa f := \sup_{t\geq 0} e^{\kappa b t}|P_t f|,
\end{equation}
and
\begin{equation}\label{Q}
Q f:= Q_{\kappa_f}f.
\end{equation}
    Then according to \cite[Fact 1.2]{MM}, $Q$ is an operator from $\mathcal P$ to $\mathcal P$.
\begin{lem}\label{lem: temp}
    For all $h \in \mathcal P^+$ and $\mu \in \mathcal M_c(\mathbb R^d)$, there exists a constant $C > 0$ such that for all $\kappa \in \mathbb Z_+ $ and $0\leq r\leq s\leq t<\infty$ with $s-r \leq 1$, we have
\[
    \sup_{g \in \mathcal P: Q_\kappa g\leq h}\mathbb P_{\mu}(|\mathcal I_r^s\langle g, X_t\rangle|>\lambda)
       \leq C e^{\alpha r} \bigg(\Big( \frac{e^{(t-s)(\alpha - \kappa b)}}{\lambda}\Big)^{1+\beta} + \Big( \frac{e^{(t-s)(\alpha - \kappa b)}}{\lambda}\Big)^{2} \bigg).
\]
   \end{lem}
\begin{proof}
    Denote by $R$ the $(\theta^2\vee\theta^{1+\beta})$-controller in Lemma \ref{lem: control pair for P(M>lambda)}.
    Fix $h \in \mathcal P^+$, $\mu \in \mathcal M_c(\mathbb R^d)$, $\kappa \in \mathbb Z_+ $ and $0\leq r\leq s\leq t < \infty$ with $s-r \leq 1$.
    Suppose that $g\in \mathcal P$ satisfies $Q_\kappa g \leq h$.
    Using the Markov property of $X$, we get
\begin{equation}\begin{split}
    &\mathbb P_{\mu}(|\mathcal I_r^s\langle g, X_t\rangle|>\lambda)
    = \mathbb P_\mu \big[\mathbb P_\mu[|\langle P_{t-s}^\alpha g, X_{s}\rangle - \langle P_{t-r}^\alpha g, X_{r}\rangle|> \lambda\big| \mathscr F_r]\big]
    \\&= \mathbb P_\mu \big[\mathbb P_{X_r}(|\langle P_{t-s}^\alpha g, X_{s-r}\rangle - \langle P_{t-r}^\alpha g, X_{0}\rangle|> \lambda)\big]
    = \mathbb P_\mu \big[\mathbb P_{X_r}(|\mathcal I_0^{s-r}\langle P_{t-s}^\alpha g, X_{s-r}\rangle |> \lambda)\big]
    \\&\leq \mathbb P_\mu \Big[ \frac{\lambda}{2}\int_{-2/\lambda}^{2/\lambda}\langle R|\theta P^\alpha_{t-s}g|,X_r\rangle d\theta \Big]
    \leq \mathbb P_\mu \Big[ \frac{\lambda}{2}\int_{-2/\lambda}^{2/\lambda}\langle R|\theta e^{(t-s)(\alpha- \kappa b)}h|,X_r\rangle d\theta \Big]
    \\&\leq \mathbb P_\mu [ \langle Rh,X_r\rangle ] \frac{\lambda}{2}\int_{-2/\lambda}^{2/\lambda}(|\theta e^{(t-s)(\alpha- \kappa b)}|^{1+\beta} + |\theta e^{(t-s)(\alpha- \kappa b)}|^{2})d\theta
    \\& =  \langle P_r^\alpha Rh,\mu\rangle \bigg(  \frac{2^{2+\beta}}{2+\beta}\Big(\frac{e^{(t-s)(\alpha- \kappa b)}}{\lambda}\Big)^{1+\beta} + \frac{2^{3}}{3}\Big(\frac{e^{(t-s)(\alpha- \kappa b)}}{\lambda}\Big)^2\bigg)
    \\ & \leq C e^{\alpha r} \bigg(\Big( \frac{e^{(t-s)(\alpha - \kappa b)}}{\lambda}\Big)^{1+\beta} + \Big( \frac{e^{(t-s)(\alpha - \kappa b)}}{\lambda}\Big)^{2} \bigg),
\end{split}\end{equation}
    where the constatn $C$ is chosen as
\[
    C := \Big(\frac{2^{2+\beta}}{2+\beta} + \frac{2^{3}}{3} \Big)\langle Q_0Rh, \mu\rangle.
    \qedhere
\]
\end{proof}


\begin{lem}
\label{lem: control of mgtrs}
    For all $h \in \mathcal P$, $\mu \in \mathcal M_c(\mathbb R^d)$ and $\gamma\in (0, \beta)$, there exists a constant $C > 0$ such that for each $\kappa \in \mathbb Z_+$ and $0\leq r\leq s\leq t<\infty$ with $s-r \leq 1$, we have
\[
    \sup_{g \in \mathcal P: Q_\kappa g\leq h} \mathbb P_\mu\big[|\mathcal I_r^s\langle g, X_t\rangle|^{1+\gamma}\big]
    \leq C e^{t\alpha+(t-s) (\gamma\alpha- (1+\gamma)\kappa b)}.
\]
\end{lem}

\begin{proof}
    Fix $h \in \mathcal P$ and $\mu \in \mathcal M_c(\mathbb R^d)$. Let $C_0$ be the constant in the Lemma \ref{lem: temp}.
    For all $\kappa \in \mathbb Z_+$,  $0\leq r\leq s\leq t$ with $s-r \leq 1$,  $g\in \mathcal P$ with $Q_{\kappa} g \leq h$, and $c>0$, we have
\begin{equation}\begin{split}
&\mathbb P_\mu\big[|\mathcal I_r^s\langle g, X_t\rangle|^{1+\gamma}\big]
    = (1+\gamma)\int_0^\infty \lambda^{\gamma} \mathbb P_{\mu}(|\mathcal I_r^s\langle g, X_t\rangle|>\lambda) d\lambda
    \\&\leq (1+\gamma)\int_0^c \lambda^{\gamma} d\lambda +(1+\gamma)\int_c^\infty \lambda^{\gamma}\mathbb P_\mu(|\mathcal I_r^s\langle g, X_t\rangle|> \lambda) d\lambda
    \\& \leq c^{1+\gamma} + C_0  e^{\alpha r}(1+\gamma)\int_c^\infty \bigg(\Big(\frac{e^{(t-s)(\alpha - \kappa b)}}{\lambda}\Big)^{1+\beta}+\Big(\frac{e^{(t-s)(\alpha - \kappa b)}}{\lambda}\Big)^{2}\bigg)\lambda^{\gamma}d\lambda
    \\&\leq c^{1+\gamma} e^{\alpha r} + C_0e^{\alpha r}(1+\gamma)\Big(  \frac{e^{(1+\beta)(t-s)(\alpha- \kappa b)}}{(\beta - \gamma)c^{\beta - \gamma}}  + \frac{e^{2(t-s)(\alpha- \kappa b)}}{(1 - \gamma)c^{1 - \gamma}} \Big).
\end{split}\end{equation}
    Taking $c = e^{(t-s)(\alpha- \kappa b)}$, we get
\begin{equation}\begin{split}
    &\mathbb P_\mu\big[|\mathcal I_r^s\langle g, X_t\rangle|^{1+\gamma}\big]
    \leq e^{(1+\gamma)(t-s)(\alpha- \kappa b)} e^{\alpha r}\Big(1+ C_0 \frac{1+\gamma}{\beta - \gamma}+ C_0 \frac{1+\gamma}{1 - \gamma}\Big).
\end{split}\end{equation}
    Note that
\begin{equation}\begin{split}
    &(1+\beta)(t-s)(\alpha- \kappa b) + \alpha r
    = (t-s)\alpha+(t-s) (\beta\alpha- (1+\beta)\kappa b)
    \\&\leq t\alpha+(t-s) (\beta\alpha- (1+\beta)\kappa b).
\end{split}\end{equation}
    So the desired result is true with
\[
    C := 1+ C_0 \frac{1+\gamma}{\beta - \gamma}+ C_0 \frac{1+\gamma}{1 - \gamma}.
    \qedhere
\]
\end{proof}

\begin{lem}
\label{lem: control moment}
    For any $h \in \mathcal P$, $\mu \in \mathcal M_c(\mathbb R^d)$, $\gamma\in (0, \beta)$ and $\kappa \in \mathbb Z_+$, there exists a constant $C > 0$ such that for each $t\geq 0$, we have
\begin{itemize}
\item[(1)]
    $\sup_{g\in \mathcal P: Q_\kappa g \leq h}\|\langle g,X_t\rangle\|_{\mathbb{P}_{\mu};1+\gamma}\leq C e^{(\alpha-\kappa b)t}$ provided $\alpha\gamma > \kappa (1+\gamma)b$;
\item[(2)]
    $\sup_{g\in \mathcal P: Q_\kappa g \leq h}\|\langle g,X_t\rangle\|_{\mathbb{P}_{\mu};1+\gamma}\leq C te^{\frac{\alpha}{1+\gamma}t}$ provided $\alpha\gamma = \kappa (1+\gamma)b$;
\item[(3)]
    $\sup_{g\in \mathcal P: Q_\kappa g \leq h} \|\langle g,X_t\rangle\|_{\mathbb{P}_{\mu};1+\gamma}\leq C e^{\frac{\alpha}{1+\gamma}t}$ provided $\alpha\gamma < \kappa (1+\gamma)b$.
\end{itemize}
\end{lem}
\begin{proof}
    Fix $\gamma \in (0,\beta)$ and $\mu \in \mathcal M_c(\mathbb R^d)$.
    Let $C$ be the constant in Lemma \ref{lem: control of mgtrs}.
    Using the triangle inequality, for all $\kappa\in \mathbb Z_+$, $g \in \mathcal P$ with $Q_\kappa g \leq h$ and $t\geq 0$, we have
\begin{equation}\begin{split}
    &\|\langle g,X_t\rangle\|_{\mathbb P_\mu;1+\gamma}
        \\ &\leq \sum_{l=0}^{\lfloor t\rfloor - 1}\big\| \mathcal{I}_{t-l-1}^{t-l}\langle g,X_t\rangle \big\|_{\mathbb P_\mu;1+\gamma}+\big\| \mathcal{I}_{0}^{t-\lfloor t \rfloor}\langle g,X_t\rangle  \big\|_{\mathbb P_\mu;1+\gamma}
    + |\langle P^\alpha_t g,\mu\rangle|
    \\ &\leq C^{\frac{1}{1+\gamma}} e^{\frac{\alpha}{1+\gamma}t} \sum_{l=0}^{\lfloor t\rfloor} e^{\frac{\gamma\alpha-\kappa (1+\gamma)b}{1+\gamma} l} + e^{(\alpha - \kappa b)t} \langle h,\mu\rangle.
\end{split}\end{equation}
    By calculating the sum on the right, we get the desired result.
\end{proof}

\section{Proof of main results}

In this section, we will prove the main results of this paper. Recall that $\mathcal{M}_c(\mathbb{R}^d)$ is the space of all finite Borel measures of compact support on $\mathbb{R}^d$. Denote  $\mathbb{\tilde{P}}_{\mu}=\mathbb{P}_{\mu}(\cdot|D^c)$ in the following section.

\subsection{The large rate case: $\alpha \beta > \kappa_f b (1+\beta)$}

In this subsection, we prove Theorem \ref{Theorem11}. We will be back to the large branching rate case in Subsection \ref{large rate again} to prove Theorem \ref{theorem 1.6}.

To prove Theorem \ref{Theorem11}, we first prove the almost sure and $L^{1+\gamma}(\mathbb{P}_{\mu})$ convergence of a family martingales for $\gamma\in (0, \beta)$. Recall that $L$ is the infinitesimal generator of the OU-process. Let $f\in \mathcal{P}\cap C^2(\mathbb R^d)$, and let $a\in \mathbb R$, we define that
\begin{equation}\begin{split}
\label{defmartingale}
    M_t^{f,a}:=e^{-(\alpha-ab)t}\langle f,X_t\rangle-\int_0^t e^{-(\alpha-ab)s}\langle (L+ab)f, X_s\rangle~ ds.
\end{split}\end{equation}
    The following lemma says that $\{M_t^{f,a}: t\geq 0\}$ is a martingale with respect to $(\mathscr{F}_t)_{t\geq 0}$.
\begin{lem}
\label{lemma25}
    For each $f\in \mathcal{P}\cap C^2(\mathbb R^d)$ and $a\in \mathbb R$.
Then, for any $\mu\in \mathcal M_c(\mathbb R^d)$, the process $(M_t^{f,a})_{t\geq 0}$ is a $\mathbb P_\mu$-martingale with respect to $(\mathscr F_t)_{t\geq 0}$.
\end{lem}
\begin{proof}
    Write $\bar{f}=(L+ab)f$. According to \cite[Theorem A.55]{Li2011Measure-valued}, we have
\begin{equation}\begin{split}\label{Theorem55}
    P_t^{ab}f(x)= f(x)+\int_0^t P_s^{ab}\bar{f}(x)~ds,\quad t\geq 0,x\in \mathbb R^d.
\end{split}\end{equation}
Let $0\leq s\leq t$,
\begin{equation}\begin{split}
\label{martingale1}
    &\mathbb{P}_{\mu}[M_t^{f,a}|\mathscr{F}_s]
    \\&=e^{-(\alpha-ab)t}\mathbb{P}_{\mu}\left[\langle f,X_t\rangle|\mathscr{F}_s\right]-\mathbb{P}_{\mu}\Big[\int_0^t e^{-(\alpha-ab)u}\langle \bar{f}, X_u\rangle~ du\Big|\mathscr{F}_s\big]
    \\&=e^{-(\alpha-ab)t}\langle P_{t-s}^{\alpha}f, X_s\rangle-\int_0^s e^{-(\alpha-ab)u}\langle \bar{f}, X_u\rangle~ du
    \\&\quad -\int_s^t e^{-(\alpha-ab)u}\langle P_{u-s}^{\alpha} \bar{f},X_s\rangle~ du.
\end{split}\end{equation}
    Using \eqref{Theorem55} and Fubini's theorem, we have
\begin{equation}\begin{split}
    &\int_s^t e^{-(\alpha-ab)u}\langle P_{u-s}^{\alpha} \bar{f},X_s\rangle~ du=e^{-(\alpha-ab)s}\int_s^t\langle P_{u-s}^{ab}\bar{f},X_s\rangle~du\\
    &=e^{-(\alpha-ab)s}\Big\langle\int_0^{t-s}P_{u}^{ab}\bar{f}~du,X_s\Big\rangle=e^{-(\alpha-ab)s}\left(\langle P_{t-s}^{ab}f,X_s\rangle-\langle
    f,X_s\rangle\right)\\
    &=e^{-(\alpha-ab)t}\langle P_{t-s}^{\alpha}f, X_s\rangle-e^{-(\alpha-ab)s}\langle
    f,X_s\rangle.
\end{split}\end{equation}
   Using this and \eqref{martingale1}, we get the desired result.
\end{proof}

    Let $p=(p_1,...,p_d)\in \mathbb Z_+^d$, recall that $\phi_p$ is an eigenfunctions of $L$ corresponding to the eigenvalue $-|p|b$. Define
\[
    H_t^p
    :=e^{-(\alpha-|p|b)t}\langle\phi_p,X_t\rangle, \quad t\geq 0.
\]

\begin{lem}\label{lemma26}
    For any $\mu\in \mathcal M_c(\mathbb R^d)$, $(H^p_t)_{t\geq 0}$ is a $\mathbb P_\mu$-martingale with respect to $(\mathscr F_t)$. Moreover, if $\alpha\beta>|p|b(1+\beta)$, then for all $\gamma\in (0, \beta)$ we have $\sup_{t\geq 0}\|H_t^p\|_{\mathbb P_\mu;1+\gamma}< \infty$ and
\[
    H_{\infty}^p
    :=\lim_{t\rightarrow \infty}H_t^p
\]
exists $\mathbb{P}_{\mu}$-a.s and in $L^{1+\gamma}(\mathbb{P}_{\mu}).$
\end{lem}
\begin{proof}
  It follows from Lemma \ref{lemma25} that $(H_t^p)_{t\geq 0}$ is a $\mathbb P_\mu$-martingale for any $\mu\in \mathcal M_c(\mathbb R^d)$.

    There exists $\gamma_0 \in (0,\beta)$ close enough to $\beta$ such that for $\gamma\in [\gamma_0, \beta)$, $\alpha\gamma>|p|(1+\gamma)b$.
    Using  Lemma \ref{lem: control moment} and the fact $\kappa_{\phi_p}=|p|$,
   we get, for $\gamma\in [\gamma_0, \beta)$, that there exists
    a constant $C_{\gamma, \mu, p}>0$ (depending on $\gamma$, $\mu$ and $p$)  such that
\[
 	\|H_t^p\|_{\mathbb P_\mu;1+\gamma}
    \leq C_{\gamma, \mu, p} e^{-(\alpha-|p|b)t}e^{(\alpha-|p|b)t}
    =C_{\gamma, \mu, p}, \quad t\geq 0.
\]
   For any $\gamma\in (0, \gamma_0)$ and any $\mu\in \mathcal M_c(\mathbb R^d)$, note that
\[
	\|H_t^p\|_{\mathbb P_\mu;1+\gamma}
	\leq\|H_t^p\|_{\mathbb P_\mu;1+\gamma_0}
	<C_{\gamma_0, \mu, p},
	\quad t\geq 0.
\]

   Hence, for each $\gamma \in (0,\beta)$ and any $\mu\in \mathcal M_c(\mathbb R^d)$,
    the martingale is bounded in $L^{1+\gamma}(\mathbb{P}_{\mu})$ and hence converges in $L^{1+\gamma}(\mathbb{P}_{\mu}) $ and almost surely, by \cite[Theorem 5.4.5]{Durrett2010Probability}.
\end{proof}

In particular, when $p=0$, $H_t^0$ reduces to $H_t:=e^{-\alpha t}\|X_t\|$, thus, as $t\rightarrow \infty$, $H_t$ converges to $H_{\infty}$, $\mathbb{P}_{\mu}$-almost surely and in $L^{1+\gamma}(\mathbb{P}_{\mu})$  for any $\mu\in \mathcal M_c(\mathbb R^d)$.

\begin{lem}\label{lem: control of wt} For all $\gamma\in (0,\beta)$, $p\in \mathbb{Z}_+^d$ and $\mu\in \mathcal M_c(\mathbb R^d)$, there exists a $C> 0$ such that for all $0\leq s<t$,
\[
    \|H^p_t-H^p_s\|_{\mathbb{P}_{\mu};1+\gamma}
    \leq C e^{-\frac{ 1}{1+\gamma}(\alpha\gamma-|p|b(1+\gamma))s}.
\]
Moreover, letting $t\to\infty$ in above inequality, we have
\[
    \|H^p_\infty-H^p_s\|_{\mathbb{P}_{\mu};1+\gamma}
    \leq C e^{-\frac{ 1}{1+\gamma}(\alpha\gamma-|p|b(1+\gamma))s},\quad s\geq 0.
\]
\end{lem}

\begin{proof}
     Fix $\gamma \in (0,\beta)$, $p\in \mathbb{Z}_+^d$ and $\mu\in \mathcal M_c(\mathbb R^d)$.
Using Lemma \ref{lem: control of mgtrs} with $g=\phi_p$ and $k=|p|$,
 there exists a constant $C_0>0$ such that for all $0\leq r\leq s $ with $s-r\leq1$, we have
    \begin{align}
        \mathbb{P}_{\mu}\big[\big|e^{(\alpha-|p|b)(t-s)}\langle\phi_p, X_s\rangle-e^{(\alpha-|p|b)(t-r)}\langle\phi_p, X_r\rangle\big|^{1+\gamma}\big]
        \leq C_0e^{\alpha t+(t-s)(\alpha\gamma-(1+\gamma)|p|b\gamma}.
    \end{align}
    Dividing both sides by $e^{(\alpha-|p|b) t(1+\gamma)}$, we get
    \begin{align}
        \mathbb{P}_{\mu}\big[|H^p_s-H^p_r|^{1+\gamma}\big]\leq  C_0 e^{-(\alpha\gamma-(1+\gamma)|p|b)s}.
    \end{align}
    Thus there is a $C>0$ for any $0\leq s<t$,
\begin{align}
	& \|H^p_t-H^p_s\|_{\mathbb{P}_{\mu};1+\gamma}
	\\&\leq \|H^p_{\lfloor s \rfloor+1}-H^p_s\|_{\mathbb{P}_{\mu};1+\gamma}+\sum_{k=\lfloor s \rfloor+1}^{\lfloor t \rfloor}\|H^p_{k+1}-H^p_{k}\|_{\mathbb{P}_{\mu};1+\gamma}+\|H^p_t-H^p_{\lfloor t \rfloor+1}\|_{\mathbb{P}_{\mu};1+\gamma}
	\\& \leq C_0^{\frac{1}{1+\gamma}} \Big(e^{-\frac{(\alpha \gamma-(1+\gamma)|p|b) s}{1+\gamma}}+\sum_{k=\lfloor s \rfloor+1}^{\lfloor t \rfloor}e^{-\frac{(\alpha \gamma-(1+\gamma)|p|b) k}{1+\gamma}}+ e^{-\frac{(\alpha \gamma-(1+\gamma)|p|b t}{1+\gamma}}\Big)
	\leq Ce^{-\frac{(\alpha \gamma-(1+\gamma)|p|b)}{1+\gamma}s}.
    \qedhere
\end{align}	
\end{proof}

{\it Proof of Theorem \ref{Theorem11}.}\quad
	Fix  $f \in \mathcal P$ such that $\alpha \beta > \kappa_f b (1+\beta)$.
	Fix $\mu \in \mathcal M_c(\mathbb R^d)$.
	Write
\begin{equation}\begin{split}
    f
    =\sum_{p\in \mathbb Z_+^d:|p|\geq \kappa_f}\langle f,\phi_p\rangle_\varphi \phi_p
    =:
    \sum_{p\in \mathbb Z_+^d:|p|= \kappa_f}\langle f,\phi_p\rangle_\varphi \phi_p+\tilde{f}.
\end{split}\end{equation}
	Then
\begin{align*}
    &e^{-(\alpha-\kappa_fb)t}\langle f,X_t\rangle=
    \sum_{p\in \mathbb Z_+^d:|p|= \kappa_f}\langle f,\phi_p\rangle_\varphi H_t^p+e^{-(\alpha-\kappa_fb)t} \langle \tilde{f},X_t\rangle,
    \quad t\geq 0.
\end{align*}
	According to Lemma \ref{lemma26},
	we have
\begin{equation}\begin{split}
\label{as convergence}
     \sum_{p\in \mathbb{Z}_+^d:|p|= \kappa_f}\langle f,\phi_p\rangle_\varphi H_t^p
     \xrightarrow[t\to \infty]{} \sum_{p\in \mathbb{Z}_+^d:|p|=\kappa_f}\langle f, \phi_p\rangle_{\varphi} H_{\infty}^p,
\end{split}\end{equation}
$\mathbb{P}_{\mu}$-a.s. and in $L^{1+\gamma}(\mathbb{P}_{\mu})$ for any $\gamma\in(0,\beta)$.
	Therefore, it is suffice to show that
\begin{equation}\begin{split}
    J_t
    :=e^{-(\alpha-\kappa_fb)t}\langle \tilde{f},X_t\rangle,
    \quad t\geq 0,
\end{split}\end{equation}
	converges in $L^{1+\gamma}(\mathbb{P}_{\mu})$ for any $\gamma\in(0,\beta)$, and converges a.s. provided $f\in C^2(\mathbb R^d)$ satisfying $D^2f\in \mathcal{P}$.

	Step 1. Let $g\in \mathcal P$.
	Let $\kappa > 0$ be such that $\kappa < \kappa_g$ and $\kappa < \frac{\alpha \beta}{b(1+\beta)}$.
	We will show that for each $\gamma \in (0,\beta)$ there exist constants $C,\delta > 0$ such that
\[
	\|e^{-(\alpha - \kappa b)t} \langle g, X_t\rangle\|_{\mathbb P_\mu;1+\gamma}
	\leq C e^{-\delta t},
	\quad t\geq 0.
\]
	In order to do this,
we choose a $\gamma_0 \in (0,\beta)$ close enough to $\beta$ such that,
for each $\gamma \in [\gamma_0, \beta)$, we have $\kappa < \frac{\alpha\gamma}{b(1+\gamma)}$.
	Then, according to Lemma \ref{lem: control moment}, we have, for each $\gamma \in (0,\beta)$,
\begin{enumerate}
\item
	if $\gamma \in [\gamma_0, \beta)$ and $\alpha\gamma> \kappa_g (1+\gamma)b$, then there exists $C>0$ such that
\[
    \|e^{-(\alpha - \kappa b)t} \langle g, X_t\rangle\|_{\mathbb P_\mu;1+\gamma}
    \leq C e^{-(\alpha-\kappa b)t}e^{(\alpha-\kappa_g b)t}
    \leq C  e^{-(\kappa_g - \kappa )bt},
    \quad t\geq 0;
\]
\item
	if $\gamma \in [\gamma_0, \beta)$ and $\alpha\gamma=\kappa_g(1+\gamma)b$, then there exists $C>0$ such that
\[
     \|e^{-(\alpha - \kappa b)t} \langle g, X_t\rangle\|_{\mathbb P_\mu;1+\gamma}
     \leq C t e^{-(\alpha - \kappa b)t}e^{\frac{\alpha}{1+\gamma}t}
     = C t e^{-(\frac{\alpha \gamma}{1+\gamma} - \kappa b)t},
     \quad t\geq 0;
\]
\item
	if $\gamma \in [\gamma_0, \beta)$ and $\alpha\gamma < \kappa_g (1+\gamma)b$, then there exists $C>0$ such that
\[
    \|e^{-(\alpha - \kappa b)t} \langle g, X_t\rangle\|_{\mathbb{P}_{\mu};1+\gamma}
    \leq C  e^{-(\alpha - \kappa b)t}e^{\frac{\alpha}{1+\gamma}t}
     = C  e^{-(\frac{\alpha \gamma}{1+\gamma} - \kappa b)t},
     \quad t\geq 0;
\]
\item
	if $\gamma \in (0,\gamma_0)$ then, thanks to (1)--(3) above, and the fact that \[\|e^{-(\alpha - \kappa b)t} \langle g, X_t\rangle\|_{\mathbb{P}_{\mu};1+\gamma}
	\leq \|e^{-(\alpha - \kappa b)t} \langle g, X_t\rangle\|_{\mathbb{P}_{\mu};1+\gamma_0},\] there exist $C, \delta >0$ such that
\[
	\|e^{-(\alpha - \kappa b)t} \langle g, X_t\rangle\|_{\mathbb{P}_{\mu};1+\gamma}
	\leq Ce^{-\delta t},
	\quad t\geq 0.
\]
\end{enumerate}
	Thus, the desired conclusion in this step is valid.
	In particular, by taking $g = \tilde f$ and $\kappa = \kappa_f$, we get that $J_t$ converges to $0$ in $L^{1+\gamma}(\mathbb{P}_{\mu})$ for any $\gamma\in(0,\beta)$.

	Step 2.
	Further assume that $f\in C^2(\mathbb R^d)$ and $D^2f \in \mathcal{P}$, we will show that $J_t$ converges to $0$ almost surely.
	For $a \geq 0$ and $ t\geq 0$, and $g\in \mathcal{P}\cap C^2(\mathbb{R}^d)$ satisfying $D^2g\in \mathcal{P}$, define
\begin{equation}\begin{split}
	L_t^{g,a}:=\int_0^t e^{-(\alpha-ab)s}\langle (L+ab)g,X_s\rangle ds,
\end{split}\end{equation}
and
\begin{equation}\begin{split}
    Y_t^{g,a}
    :=\int_0^t e^{-(\alpha-ab)s}|\langle (L+ab)g,X_s\rangle|ds.
\end{split}\end{equation}
	Now choose $a_0 \in (\kappa_{f}, \kappa_f + 1)$ close enough to $\kappa_f$ so that $a_0 < \frac{\alpha \beta}{b(1+\beta)}$.
	According to \eqref{defmartingale}, we have
\begin{align*}
    J_t
    =e^{-(a_0-\kappa_f)bt} (M_t^{\tilde{f}, a_0}+L_t^{\tilde{f}, a_0}),
    \quad t\geq 0.
\end{align*}
	So we only need to show that
\begin{align*}
    e^{-(a_0-\kappa_f)b t}M_t^{\tilde{f},a_0}
    \xrightarrow[t\to \infty]{} 0,
    \quad e^{-(a_0-\kappa_f)b t}L_t^{\tilde{f},a_0}
    \xrightarrow[t\to \infty]{} 0
    \quad \mathbb{P}_{\mu}\text{-a.s.}
\end{align*}
	Notice that $\kappa_{(L+a_0 b)\tilde{f}}=\kappa_{\tilde{f}}\geq \kappa_f+1 > a_0$.
	So according to Step 1, for an arbitrary fixed $\gamma\in (0,\beta)$, there exist $C, \delta>0$ such that for each $t\geq 0$,
\begin{equation}\begin{split}
    \|e^{-(\alpha-a_0 b)t}\langle \tilde{f},X_t\rangle)\|_{\mathbb{P}_{\mu};1+\gamma}
    \leq C e^{-\delta t},
    \quad \|e^{-(\alpha-a_0 b)t}\langle L\tilde{f}+a_0 b\tilde{f},X_t\rangle\|_{\mathbb{P}_{\mu};1+\gamma}
    \leq C e^{-\delta t}.
\end{split}\end{equation}
	Now, by the triangle inequality, for each $t\geq 0$,
\begin{align*}
    &\|L_t^{\tilde{f},a_0}\|_{\mathbb{P}_{\mu};1+\gamma}
    \leq\|Y_t^{\tilde{f},a_0}\|_{\mathbb{P}_{\mu};1+\gamma}
    \\&\leq \int_0^t \|e^{-(\alpha-a_0 b)s}\langle L\tilde{f}+a_0 b\tilde{f},X_s\rangle\|_{\mathbb{P}_{\mu};1+\gamma}ds\leq C \int_0^t e^{-\delta s}ds\leq\frac{C}{\delta}.
\end{align*}
Since $Y_t^{\tilde{f},a_0}$ is increasing in $t$, it converges to some finite random variable $Y_{\infty}^{\tilde{f},a_0}$ almost surely and in $L^{1+\gamma}(\mathbb{P}_{\mu})$.
	As a consequence, almost surely we have
\begin{align*}
    \lim_{t\rightarrow \infty}e^{-(a_0 - \kappa_f)bt}|L_t^{\tilde{f},a_0}|
    \leq  \lim_{t\rightarrow \infty}e^{-(a_0 - \kappa_f)bt}|Y_t^{\tilde{f},a_0}|=0.
\end{align*}
	On the other hand, the martingale $M_t^{\tilde{f},a_0}$ satisfies
\begin{align*}
    \|M_t^{\tilde{f},a_0}\|_{\mathbb{P}_{\mu};1+\gamma}\leq  \|e^{-(\alpha-a_0 b)t}\langle \tilde{f},X_t\rangle)\|_{\mathbb{P}_{\mu};1+\gamma}+\|L_t^{\tilde{f},a_0}\|_{\mathbb{P}_{\mu};1+\gamma}\leq C(e^{-\delta t}+\frac{1}{\delta}),\quad t\geq 0.
\end{align*}
	This implies that the martingale converges almost surely.
	As a consequence,
\[
	\lim_{t\rightarrow\infty} e^{-(a_0-\kappa_f)bt}M_t^{\tilde{f},a_0}
	=0,
	\quad \mathbb P_\mu\text{-a.s.}
\]
	
	The proof is now completed.

\subsection{The critical rate case: $\alpha\beta=\kappa_fb(1+\beta)$}

\begin{lem}\label{lem: mainlemma}
Let $f\in \mathcal{P}$.
Assume that $\alpha\beta\leq \kappa_fb(1+\beta)$.
Then for all $k\geq 0$ and $\mu \in \mathcal M_c(\mathbb R^d)$, under $\mathbb{P}_{\mu}(\cdot | D ^c)$, we have
 \begin{equation}\begin{split}
      \gamma_{t,k}:=\frac{\mathcal I_{t-k-1}^{t-k}\langle f ,X_t\rangle}{(e^{\alpha k}\|X_{t-k-1}\|)^{\frac{1}{1+\beta}}}\xrightarrow{d}\zeta_k, \quad t\rightarrow \infty, \label{limitdistribution1}
 \end{split}\end{equation}
 where $\zeta_k$ is a $(1+\beta)$-stable random variable with characteristic function
 $$\mathbb{E}(e^{i\theta\zeta_k})=\exp(e^{\alpha}m_k[\theta f]),\quad \theta \in \mathbb R.$$
 \end{lem}
 \begin{proof}
    We only need to show that
\begin{equation}\begin{split}
    \mathbb{P}_{\mu}[\exp(i\gamma_{t,k}); D^c]
    \xrightarrow[t\rightarrow \infty]{}\mathbb{P}_{\mu}(D^c)\exp(e^{\alpha}m_k[f]),
    \quad \mu \in \mathcal M_c(\mathbb R^d), f\in \mathcal P, k \geq 0.
\end{split}\end{equation}
    In fact, once we prove this, we can replace $f$ by $\theta f$, with $\theta \in \mathbb R$ being  arbitrary,  to get the desired result.
    In the rest of the proof we fix a $\mu \in \mathcal M_c(\mathbb R^d)$ and an $f\in \mathcal P$.
    Define
\[
    A_t(\epsilon):=\{ \|X_t\| \geq e^{(\alpha - \epsilon)t} \},\quad t\geq 0, \epsilon > 0.
\]

    Step 1. We will show that for all $\epsilon > 0, k\geq 0$ and $t>k+1$, we have
\begin{equation}\begin{split}
    \big|\mathbb{P}_{\mu}\big[e^{i\gamma_{t,k}}-e^{e^{\alpha}m_k[f]}; D^c\big]\big|
    \leq J_1(t,k,\epsilon)+J_2(t,k,\epsilon)+J_3(t,k,\epsilon)
\end{split}\end{equation}
    where
\begin{equation}\begin{split}
\label{eq: Def of Ji}
    J_1(t,k,\epsilon)
    &:= \mathbb{P}_{\mu}\big[|\langle K_1(\theta_{t,k}P^\alpha_k f), X_{t-k-1}\rangle|; A_{t-k-1}(\epsilon) \big],
    \\ J_2(t,k,\epsilon)
    &:= \mathbb{P}_{\mu}\big[|\langle Z_1(\theta_{t,k}P^\alpha_k f),X_{t-k-1}\rangle-e^{\alpha}m_k[f]|; A_{t-k-1}(\epsilon)\big],
    \\ J_3(t,k, \epsilon)
    &:=2\mathbb{P}_{\mu}(A_{t-k-1}(\epsilon)\Delta D^c),
    \\\theta_{t,k}
    &:= (e^{\alpha k}\|X_{t-k-1}\|)^{-\frac{1}{1+\beta}}.
\end{split}\end{equation}
In fact, from the definitions of $K_1$ and $Z_1$, we have that for all
$k\geq 0, t\geq k+1$,
\begin{equation}\begin{split}
\label{eq: need1}
    &\mathbb{P}_{\mu}[e^{i\gamma_{t,k}}|\mathscr{F}_{t-k-1}]
    =\mathbb{P}_{\mu}[e^{i\theta_{t,k}\langle P^\alpha_k f,X_{t-k}\rangle-i\theta_{t,k}\langle P^\alpha_{k+1} f, X_{t-k-1}\rangle}|\mathscr{F}_{t-k-1}]
    \\&=e^{\langle (U_1 - iP^\alpha_1 ) (\theta_{t,k}P^\alpha_k f),X_{t-k-1}\rangle}
    =e^{\langle (Z_1 + K_1) (\theta_{t,k}P^\alpha_k f),X_{t-k-1}\rangle}.
\end{split}\end{equation}
       From Proposition \ref{cor: alpha stable rv}, we have
\begin{equation}
\label{eq: need2}
    \operatorname {Re} m_t[f] < 0.
\end{equation}
    According to \eqref{eq: need1}, \eqref{eq: -v has positive real part}, \eqref{eq: need2} and the fact that
\[
    |e^{-x} - e^{-y}| \leq |x-y|,\quad x,y \in \mathbb C_+,
\]
    we get that for all $k\geq 0$, $t\geq k+1$ and $\epsilon> 0$,
\begin{equation}\begin{split}
\label{eq: inequality that will used later}
    &\big|\mathbb{P}_{\mu}\big[e^{i\gamma_{t,k}}-e^{e^{\alpha}m_k[f]}; D^c\big]\big|
    \\& \leq \mathbb{P}_{\mu}\Big[\big| \mathbb{P}_{\mu}[e^{i\gamma_{t,k}}-e^{e^{\alpha}m_k[f]}; D^c | \mathscr F_{t-k-1}]\big|\Big]
    \\& \leq \mathbb{P}_{\mu}\Big[\big| \mathbb{P}_{\mu}[e^{i\gamma_{t,k}}-e^{e^{\alpha}m_k[f]}; A_{t-k-1}(\epsilon)| \mathscr F_{t-k-1}]\big| + 2\mathbb P_\mu(A_{t-k-1}(\epsilon) \Delta D^c| \mathscr F_{t-k-1})\Big]
    \\& = \mathbb{P}_{\mu}\Big[ \big|\mathbb{P}_{\mu}[e^{i\gamma_{t,k}}| \mathscr F_{t-k-1}]-e^{e^{\alpha}m_k[f]}\big|;A_{t-k-1}(\epsilon)\Big] + J_3(t,k,\epsilon)
    \\& \leq \mathbb{P}_{\mu}\big[|e^{\langle (U_1 - iP^\alpha_1 ) (\theta_{t,k}P^\alpha_k f),X_{t-k-1}\rangle}-e^{e^{\alpha}m_k[f]}|;A_{t-k-1}(\epsilon)\big]+  J_3(t,k,\epsilon)
    \\& \leq \mathbb{P}_{\mu}\big[|\langle (U_1 - i T_1^\alpha)(\theta_{t,k}P^\alpha_k f),X_{t-k-1}\rangle-e^{\alpha}m_k[f]|;A_{t-k-1}(\epsilon)\big]+  J_3(t,k,\epsilon)
    \\&\leq J_1(t,k,\epsilon) + J_2(t,k,\epsilon)+J_3(t,k,\epsilon).
\end{split}\end{equation}

Step 2.  We will show that for each $\epsilon>0$ small enough, there exists constants $C, \delta>0$ such that
    \begin{equation}\begin{split}
    \label{lemma31q}
      J_1(t,k,\epsilon)
      \leq Ce^{-\delta (t-k)},
      \quad k\geq 0, t\geq k+1.
    \end{split}\end{equation}
    In fact, let $\delta_0 >0$ be the constant in Lemma \ref{lem: upper bound for usgx}.(7) and let $R$ be the corresponding $(\theta^{2+\beta}\vee \theta^{1+\beta+\delta_0})$-controller.
    Then, we have for all $k\geq 0$, $t\geq k+1$ and $\epsilon> 0$,
\begin{equation}\begin{split}
   & |K_1(\theta_{t,k}P^\alpha_k f)|\mathbf{1}_{A_{t-k-1}(\epsilon)}
   \leq R(|\theta_{t,k}P^\alpha_k f|)\mathbf{1}_{A_{t-k-1}(\epsilon)}
   \\&\leq R \Big(\frac{e^{(\alpha-\kappa_fb)k} Qf}{(e^{\alpha k}e^{(\alpha-\epsilon)(t-k-1)})^\frac{1}{1+\beta}}\Big)
   \\&\leq \sum_{\rho \in \{\delta_0, 1\}}\Big(\frac{e^{(\alpha-\kappa_fb)k}}{(e^{\alpha k}e^{(\alpha-\epsilon)(t-k-1)})^\frac{1}{1+\beta}}\Big)^{1+\beta+ \rho} RQf
   \\&=\sum_{\rho \in \{\delta_0, 1\}}e^{\frac{1+\beta + \rho}{1+\beta}(\alpha\beta-\kappa_fb(1+\beta))k}e^{-\frac{1+\beta+\rho}{1+\beta} (\alpha-\epsilon)(t-k-1)}RQf
   \\&\leq \sum_{\rho \in \{\delta_0,1\}}e^{-\frac{1+\beta+\rho}{1+\beta}(\alpha-\epsilon)(t-k-1)}RQf,
\end{split}\end{equation}
where $Q$ is defined by \eqref{Q}.
Thus for all $k\geq 0$, $t\geq k+1$ and $\epsilon> 0$,
\begin{equation}\begin{split}
\label{eq: estimate of J1}
     J_1(t,k,\epsilon)&
     \leq \sum_{\rho \in \{\delta_0,1\}}e^{-\frac{1+\beta+\rho}{1+\beta}(\alpha-\epsilon)(t-k-1)}\mathbb{P}_{\mu}[\langle RQf,X_{t-k-1}\rangle]\\
     & \leq \sum_{\rho \in \{\delta_0,1\}} \langle Q_0 RQf, \mu \rangle e^{-(\alpha\frac{\rho}{1+\beta}-\epsilon\frac{1+\beta+\rho}{1+\beta})(t-k-1)},
\end{split}\end{equation}
where $Q_0$ is defined by \eqref{Q_k} with $k=0$.
    By taking $\epsilon>0$ small enough, we get the desired result in this step.

    Step 3.
    We will show that for $\epsilon>0$ small enough, there exists constants $C, \delta>0$ such that
    \begin{equation}\begin{split}
    \label{eq:31step3}
      J_2(t,k,\epsilon)
      \leq Ce^{-\delta (t-k)},
      \quad k\geq 0, t\geq k+1.
    \end{split}\end{equation}
    In fact, according to the definitions of $Z_1$ and $m_t$, we have for all $k\geq 0$, $t\geq k+1$ and $\epsilon> 0$,
    \begin{equation}\begin{split}
          &\langle Z_1(\theta_{t,k}P^\alpha_k f),X_{t-k-1}\rangle-e^{\alpha}m_k[f]
          \\&= \theta_{t,k}^{1+\beta} \langle Z_1P^\alpha_k f,X_{t-k-1}\rangle - e^{-\alpha k}\langle  Z_1P^\alpha_k f,\varphi\rangle
          \\&=e^{-\alpha k}\Big(\frac{\langle Z_1P^\alpha_k f ,X_{t-k-1}\rangle}{\|X_{t-k-1}\|}-\langle  Z_1P^\alpha_k f ,\varphi\rangle\Big).
    \end{split}\end{equation}
Therefore, for all $k\geq 0$, $t\geq k+1$ and $\epsilon> 0$,
\begin{equation}\begin{split}
\label{eq: prevJ2}
J_2(t,k,\epsilon)&
    = \mathbb P_\mu\big[|\langle Z_1(\theta_{t,k}P^\alpha_k f),X_{t-k-1}\rangle-e^{\alpha}m_k[f]|;A_{t-k-1}(\epsilon)\big]
    \\&=e^{-\alpha k}\mathbb{P}_{\mu}\bigg[\Big|\frac{\langle Z_1P^{\alpha}_k f,X_{t-k-1}\rangle}{\|X_{t-k-1}\|}-\langle  Z_1P^{\alpha}_k f,\varphi\rangle\Big|;A_{t-k-1}(\epsilon)\bigg]\nonumber\\
    &\leq e^{-\alpha k}e^{-(\alpha-\epsilon)(t-k-1)}e^{(\alpha-\kappa_f b)(1+\beta)k} \mathbb{P}_{\mu}\left[\left|\langle g_k,X_{t-k-1}\rangle\right|\right],
\end{split}\end{equation}
where
\[
g_k
    = \frac{Z_1 P^{\alpha}_k f-\langle  Z_1P^{\alpha}_k f,\varphi\rangle}{e^{(\alpha-\kappa_f b)(1+\beta)k}},
    \quad k \geq 0.
\]
According to Lemma \ref{lem: control of gk}, we get there exists $h\in \mathcal{P}$ such that
 \[
    Q_1 (\operatorname{Re} g_k) \leq h
    \text{ and } Q_1 (\operatorname{Im} g_k)\leq h,
    \quad k \geq 0,
 \]
    where $Q_1$ is defined by \eqref{Q_k} with $k=1$.

    Chose a $\gamma\in(0,\beta)$ small enough such that $\alpha \gamma < b < (1+\gamma)b$.
    According to Lemma \ref{lem: control moment}.(3) (with $\kappa=1$), there exists $C>0$ such that for all $t\geq 0$ and $k\geq 0$,
\begin{equation}\begin{split}
    &\mathbb{P}_{\mu}\left[\left|\langle g_k,X_{t}\rangle\right|\right]
    \leq \|\langle \operatorname{Re} g_k, X_{t}\rangle\|_{\mathbb{P}_{\mu,1+\gamma}} + \|\langle \operatorname{Im} g_k, X_{t}\rangle\|_{\mathbb{P}_{\mu,1+\gamma}}
    \\& \leq 2\sup_{g\in \mathcal P: Q_1 g\leq h} \|\langle g, X_t\rangle\|_{\mathbb P_\mu; 1+\gamma} \leq C e^{\frac{\alpha t}{1+\gamma}}.
\end{split}\end{equation}
Therefore, according to \eqref{eq: prevJ2}, we have there exists constant $C>0$ such that for all $k\geq 0$, $t\geq k+1$ and $\epsilon> 0$,
\begin{equation}\begin{split}
\label{eq: right bound for J2}
    J_2(t,k, \epsilon)&
\leq  e^{-\alpha k}e^{-(\alpha-\epsilon)(t-k-1)}e^{(\alpha-\kappa_f b)(1+\beta)k} \mathbb{P}_{\mu}\left[\left|\langle g_k,X_{t-k-1}\rangle\right|\right]
    \\&\leq C e^{-\alpha k}e^{-(\alpha-\epsilon)(t-k-1)}e^{(\alpha-\kappa_f b)(1+\beta)k} e^{\frac{\alpha}{1+\gamma}(t-k-1)}
    \\&= C e^{(\alpha \beta - \kappa_f b(1+\beta))k}e^{-(\frac{\alpha\gamma}{1+\gamma}-\epsilon)(t-k-1)}
    \\&\leq C e^{-(\frac{\alpha\gamma}{1+\gamma}-\epsilon)(t-k-1)}.
\end{split}\end{equation}
    By taking $\epsilon >0$ small enough, we get the required result in this step.

    Step 4. We will show that, for each $\epsilon\in (0,  \alpha)$, there exist $C,\delta>0$ such that for all $k\geq0, t\geq k+1$,
\begin{equation}\begin{split}\label{ineq: control of J3}
 J_3(t,k,\epsilon)\leq Ce^{-\delta (t-k)}.
\end{split}\end{equation}
    In fact, we have that for all $t\geq 0, \epsilon >0$,
\begin{equation}\begin{split}
    &\mathbb P_{\mu}(A_{t}(\epsilon), D) = \mathbb P_{\mu}[\mathbb P_{\mu}(D|\mathscr F_t);A_t(\epsilon)]
    \\&= \mathbb P_\mu[e^{-\bar v\|X_t\|};A_t(\epsilon)]
    \leq \exp({-\bar v \|\mu\|e^{(\alpha - \epsilon)t}}).
\end{split}\end{equation}
According to Proposition \ref{lem: control of XT}, for each $\epsilon \in (0, \alpha)$, there are constants $C, \delta>0$ such that
for all $t\geq 0$,
\begin{equation}\begin{split}
    \mathbb P_\mu(A_t(\epsilon)^c,D^c) \leq   \mathbb P_\mu(0 < e^{-\alpha t}\|X_t\|\leq e^{ - \epsilon t}) \leq C (e^{-\epsilon \delta t}+e^{-\delta t}).
\end{split}\end{equation}
    Combining these results, we get the desired result in step 4.

    Finally, combining the results in Steps 1-4, noticing that, if $\epsilon>0$ is chosen small enough then $J_{i}(t,k,\epsilon), i = 1,2,3$ converge to $0$ exponentially fast as $t\rightarrow\infty$,
we immediately get the desired result.
\end{proof}

\begin{cor}\label{cor: used in next corollary}
Suppose that $f\in \mathcal{P}$ such that $\alpha\beta\leq \kappa_fb(1+\beta)$.
Then for any $\Theta >0$ and $\mu\in \mathcal M_c(\mathbb R^d)$,
there exist $C>0$ and $\delta>0$ such that for all $k \geq 0, t\geq k+1$ and $|\theta|\leq \Theta$,
\begin{equation}\begin{split}
    \mathbb{P}_{\mu}\Big[\big|\mathbb{P}_{\mu}[e^{i\theta\gamma_{t,k}}-e^{e^{\alpha}m_k[\theta f]}; D^c | \mathscr F_{t-k-1}]\big|\Big]\leq Ce^{-\delta(t-k)}.
\end{split}\end{equation}
\end{cor}
\begin{proof}
	For each $f\in \mathcal P$, define $J_1^f(t,k,\epsilon), J_2^f(t,k,\epsilon)$ and $J_3(t,k,\epsilon)$ as $J_1, J_2$ and $J_3$ in \eqref{eq: Def of Ji}.
	Now, fix a $\mu \in \mathcal M_c(\mathbb R^d)$ and an $f\in \mathcal P$.
    According to \eqref{eq: inequality that will used later},  we have for all $\theta\in \mathbb R$, $k\geq 0$, $t\geq k+1$ and $\epsilon> 0$,
\begin{equation}\begin{split}
    &\mathbb{P}_{\mu}\Big[\big| \mathbb{P}_{\mu}[e^{i\theta \gamma_{t,k}}-e^{e^{\alpha}m_k[\theta f]}; D^c | \mathscr F_{t-k-1}]\big|\Big]
    \\&\leq J^{\theta f}_1(t,k,\epsilon) + J^{\theta f}_2(t,k,\epsilon)+J_3(t,k,\epsilon).
\end{split}\end{equation}

    Let $\delta_0 >0$ be the constant in Lemma \ref{lem: upper bound for usgx}.(7) and let $R$ be the corresponding $(\theta^{2+\beta}\vee \theta^{1+\beta+\delta_0})$-controller.
	According to \eqref{eq: estimate of J1}, we have for all $\theta\in \mathbb R$, $k\geq 0$, $t\geq k+1$ and $\epsilon> 0$,
\begin{equation}\begin{split}
	&J^{\theta f}_1(k,t,\epsilon)
     \leq \sum_{\rho \in \{\delta_0,1\}} \langle Q_0 RQ(\theta f), \mu \rangle e^{-(\alpha\frac{\rho}{1+\beta}-\epsilon\frac{1+\beta+\rho}{1+\beta})(t-k-1)}
     \\& \leq(|\theta|^{2+\beta}\vee |\theta|^{1+\beta+\delta_0}) \sum_{\rho \in \{\delta_0,1\}} \langle Q_0 RQf, \mu \rangle e^{-(\alpha\frac{\rho}{1+\beta}-\epsilon\frac{1+\beta+\rho}{1+\beta})(t-k-1)}.
\end{split}\end{equation}
	From the definitions of $Z_1$ and $m_t$ we get that for all $g\in \mathcal P, \theta \geq 0, t\geq 0$,
\[
	Z_1( \pm \theta g) = \theta^{1+\beta} Z_1(\pm g), \quad m_t[\pm \theta g] = \theta^{1+\beta} m_t[\pm g].
\]
	Therefore, we have for all $\theta >0, k \geq 0, t\geq k+1, \epsilon > 0$,
\[
J^{\pm \theta f}_2(t,k,\epsilon)
	:= \theta^{1+\beta} J_2^{\pm f}(t,k,\epsilon).
\]
	According to this and \eqref{eq: right bound for J2} in the Step 3 in the proof of the previous lemma, we have that there exists a constant $C > 0$ such that for all $\theta\in \mathbb R$, $k\geq 0$, $t\geq k+1$ and $\epsilon> 0$,
\begin{equation}\begin{split}
\label{eq:31step3b}
    J^{\theta f}_2(t,k,\epsilon)
    \leq C |\theta|^{1+\beta}\exp\Big(-(\frac{\alpha\gamma}{1+\gamma}-\epsilon)(t-k-1)\Big).
\end{split}\end{equation}
	Finally, noticing that $|\theta| < \Theta$, using the estimates of $J^{\theta f}_{i}, i = 1,2$ and the estimate of $J_3$ in Step 4 of the proof of the previous lemma, we get the desired result by choosing $\epsilon$ small enough.
\end{proof}

\begin{prop}\label{corollary31}
Suppose that $f\in \mathcal{P}$ such that $\alpha\beta\leq\kappa_fb(1+\beta)$. Then for any $\Theta >0$ and $\mu\in \mathcal M_c(\mathbb R^d)$,
there exist constants $C,\delta>0$ such that
for all $t\geq 0$,
$n \in \{0, \cdots, \lfloor t \rfloor\}$ and $(\theta_0, \cdots, \theta_n)\in \mathbb R^{n+1}$
satisfying $|\theta_i|\leq \Theta$, we have
\begin{equation}\begin{split}
\label{32corollary}
    \Big|\mathbb{\tilde{P}}_{\mu}\Big[\prod_{k=0}^n\exp\Big(i\theta_k \frac {\mathcal I_{t-k-1}^{t-k}\langle f ,X_t\rangle}{(e^{\alpha k}\|X_{t-k-1}\|)^\frac{1}{1+\beta}}\Big)-\prod_{k=0}^n\exp(e^{\alpha}m_k[\theta_k f])\Big]\Big|\leq C e^{-\delta(t-n)}.
\end{split}\end{equation}
\end{prop}
\begin{proof}
	Let $C,\delta > 0$ be the constants in Corollary \ref{cor: used in next corollary}.
    Recall that \[\gamma_{t,k}:=\frac {\mathcal I_{t-k-1}^{t-k}\langle f ,X_t\rangle}{(e^{\alpha k}\|X_{t-k-1}\|)^\frac{1}{1+\beta}},\quad k \geq 0, t\geq k+1. \]
        Fix $t\geq 0$, $n \in \{0, \cdots, \lfloor t \rfloor\}$ and $(\theta_0, \cdots, \theta_n)\in \mathbb R^{n+1}$
    satisfying $|\theta_i|\leq \Theta$.
    For each $k\in\{-1,...,n\}$, we define
    \[a_k:=\prod_{l=0}^{k}\exp(e^{\alpha}m_l[\theta_lf])\mathbb{\tilde{P}}_{\mu}\Big(\prod_{l=k+1}^{n}\exp\left(i\theta_l\gamma_{t,l}\right)\Big),\]
     where by convention the product is $1$ for $k=-1$. Then we get for each $k > -1$,
    \begin{align*}
        &a_{k-1} - a_k
        \\&=\mathbb{P}_{\mu}(D^c)^{-1}\prod_{l=0}^{k-1}e^{e^{\alpha}m_l[\theta_l f]}\mathbb{P}_{\mu}\Big[(e^{i\theta_{k}\gamma_{t,k}}-e^{e^{\alpha}m_k[\theta_k f]})\prod_{l=k+1}^ne^{i\theta_{l}\gamma_{t,l}};D^c\Big]
        \\&=\mathbb{P}_{\mu}(D^c)^{-1}\prod_{l=0}^{k-1}e^{e^{\alpha}m_l[\theta_l f]}\mathbb{P}_{\mu}\Big[\mathbb P_\mu[e^{i\theta_{k}\gamma_{t,k}}-e^{e^{\alpha}m_k[\theta_k f]}; D^c|\mathscr F_{t-k-1}]\prod_{l=k+1}^ne^{i\theta_{l}\gamma_{t,l}}\Big].
    \end{align*}
  According to Proposition \ref{cor: alpha stable rv} and Corollary \ref{cor: used in next corollary},
    there exist $C,\delta>0$ such that for any
        $k\in\{0, 1, \cdots, n\}$, we have
    \begin{align*}
        &|a_{k-1}- a_k|
        \\&\leq \frac{1}{\mathbb{P}_{\mu}(D^c)}\mathbb{P}_{\mu}\Big[\big|\mathbb P_\mu[e^{i\theta_{k}\gamma_{t,k}}-e^{e^{\alpha}m_k[\theta_k f]}; D^c\big|\mathscr{F}_{t-k-1}]\big|\Big]
        \\& \leq C e^{-\delta(t-k)}.
    \end{align*}
Therefore,
\begin{equation}\begin{split}
    \text{LHS of \eqref{32corollary}}&= \left|a_{-1}-a_n\right|
    \leq\sum_{k=0}^n\left|a_{k-1}-a_k\right|
    \leq \sum_{k=0}^n C e^{-\delta(t-k)}.
\end{split}\end{equation}
	Recall that $C, \delta>0$ are independent of the choice of $t\geq 0$, $n \in \{0,...,\lfloor t \rfloor\}$ and $(\theta_0,...,\theta_n)\in \mathbb R^{n+1}$ with $|\theta_i|\leq \Theta$.
    This implies the desired result.
\end{proof}
We now present the proof of Theorem \ref{Theorem12}.
\bigskip

{\it Proof of Theorem \ref{Theorem12}.}\quad
    According to our assumption, $f\in\mathcal{P}$ is fixed and that  $\alpha\beta=\kappa_fb(1+\beta)$.
    Chose $t_0 > 0$ large enough so that $\lceil t_0-\ln t_0\rceil \leq \lfloor t_0 \rfloor - 1.$
    Write
    \begin{align*}
        &(t\|X_t\|)^{-\frac{1}{1+\beta}}\langle f,X_t\rangle
        \\ &=\sum_{k=0}^{\lfloor t-\ln t \rfloor} \frac{\mathcal I_{t-k-1}^{t-k}\langle f ,X_t\rangle}{(t\|X_t\|)^{\frac{1}{1+\beta}}}+\Big(\sum_{k=\lceil t-\ln t \rceil}^{\lfloor t \rfloor-1} \frac{\mathcal I_{t-k-1}^{t-k}\langle f ,X_t\rangle}{(t\|X_t\|)^{\frac{1}{1+\beta}}}+\frac{\mathcal I_0^{t-\lfloor t \rfloor}\langle f ,X_t\rangle}{(t\|X_t\|)^{\frac{1}{1+\beta}}}\Big) +
         \frac{\langle P^\alpha_tf,X_0\rangle}{(t\|X_t\|)^{\frac{1}{1+\beta}}}.
        \\&=:I_1(t)+I_2(t) + I_3(t),
        \quad t\geq t_0.
    \end{align*}
    Define
\[
    \tilde I_1(t)
    :=\sum_{k=0}^{\lfloor t-\ln t \rfloor}\frac{\mathcal I_{t-k-1}^{t-k}\langle f ,X_t\rangle}{(t e^{\alpha(k+1)}\|X_{t-k-1}\|)^{\frac{1}{1+\beta}}},
    \quad t\geq t_0.
\]
    Fix $\mu \in \mathcal M_c(\mathbb R^d)$ and $\theta\in \mathbb R$. Taking $\theta_k=(t e^{\alpha})^{-\frac{1}{1+\beta}} \theta $ and $n={\lfloor t-\ln t \rfloor}$
    in Proposition \ref{corollary31},
    we get that there exist $C,\delta>0$ such that,
\begin{align*}
        \Big|\mathbb{\tilde{P}}_{\mu} [e^{i\theta\tilde{I}_1(t)}]-\exp\Big(\frac{1}{t}\sum_{k=0}^{\lfloor t-\ln t \rfloor}m_k[\theta f]\Big)\Big|\leq C \frac{1}{t^{\delta}},
        \quad t\geq t_0,
\end{align*}
    Where $\mathcal{\mathbb{P}}_{\mu}=\mathbb{P}_{\mu}(\cdot|D^c)$. According to \eqref{para: critical case},  $\tilde{I}_1(t)\xrightarrow[t\to \infty]{d}\zeta$ under $\tilde {\mathbb P}_\mu$.
    Therefore, we only need to prove
 \begin{equation}\label{toprove-1}|\mathbb{\tilde{P}}_{\mu}[e^{i\theta I_1(t)}]-\mathbb{\tilde{P}}_{\mu}[e^{i\theta\tilde{I}_1(t)}]|\xrightarrow[t\to \infty]{} 0,
 \end{equation}
      and
     \begin{equation}\label{toprove-2}
     I_i(t)\xrightarrow[t\to \infty]{d} 0,\quad i = 2,3,  \mbox{ under } \tilde {\mathbb P}_\mu.\end{equation}


By \cite[Lemma 3.4.3]{Durrett2010Probability},
\begin{equation}\label{ineq: control of I1t}
    |\mathbb{\tilde{P}}_{\mu}[e^{i\theta I_1(t)}] - \mathbb{\tilde{P}}_{\mu} [e^{i\theta\tilde{I}_1(t)}]|
    \leq \sum_{k=0}^{\lfloor t-\ln t \rfloor}\mathbb{\tilde{P}}_{\mu}\big[|Y_{t,k}|\big],
    \quad t\geq t_0,
\end{equation}
    where for all $k \geq 0$ and $t\geq k+1$,
\begin{align*}
    Y_{t,k}
    :=\exp\Big(i\theta\frac{\mathcal I_{t-k-1}^{t-k}\langle f ,X_t\rangle}{(t e^{\alpha(k+1)}\|X_{t-k-1}\|)^{\frac{1}{1+\beta}}}\Big)-\exp\Big(i\theta\frac{\mathcal I_{t-k-1}^{t-k}\langle f ,X_t\rangle}{(t\|X_t\|)^{\frac{1}{1+\beta}}}\Big).
\end{align*}
    Let $\gamma \in (0,\beta)$ be close enough to $\beta$ such that
\[
    \frac{\alpha \gamma}{1+\gamma} > \frac{\alpha}{1+\gamma} - \frac{\alpha}{1+\beta} > 0.
\]
    Fix this $\gamma$, then choose $\eta_0,\eta_1>0$ such that
\[
    \frac{\alpha \gamma}{1+\gamma} >\eta_0 > \eta_0 - 3\eta_1 > \frac{\alpha}{1+\gamma} - \frac{\alpha}{1+\beta} > 0.
\]
    Define for all $k \geq 0$ and $t\geq k+1$,
\begin{equation}\begin{split}
\label{def: Dtk}
    \mathcal{D}_{t,k}&:=\left\{|H_t-H_{t-k-1}|\leq  e^{-\eta_0 (t-k-1)}, H_{t-k-1}> 2e^{-\eta_1(t-k-1)}\right\}.
\end{split}\end{equation}

    Step 1. We will show that there exist $C,\delta >0$ such that for all $k \geq 0$ and $t\geq k+1$,
\begin{equation}\begin{split}
\label{thm121}
    \mathbb{\tilde{P}}_{\mu}\big[|Y_{t,k}|;\mathcal{D}^c_{t,k}\big]
    \leq C e^{-\delta (t-k)}.
\end{split}\end{equation}
    Then according to Proposition \ref{lem: control of XT}, Lemma \ref{lem: control of wt} with $|p|=0$ and Chebyshev's inequality, there exists $C, \delta>0$ such that for all $k \geq 0$ and $t\geq k+1$,
\begin{equation}\begin{split}
\label{eq: prob of Dtkc}
    &\mathbb{\tilde{P}}_{\mu}(\mathcal{D}_{t,k}^c)
    \\&\leq \mathbb{\tilde{P}}_{\mu}(|H_t-H_{t-k-1}| > e^{-\eta_0 (t-k-1)})+\mathbb{\tilde{P}}_{\mu}(H_{t-k-1}\leq 2e^{-\eta_1(t-k-1)}),
    \\&\leq \mathbb{P}_{\mu}(D^c)^{-1}e^{\eta_0(t-k-1)}\mathbb{P}_{\mu}[|H_t-H_{t-k-1}|]
    \\&\quad +\mathbb{P}_{\mu}(D^c)^{-1} \mathbb P(H_{t-k-1}\leq 2e^{-\eta_1(t-k-1)}; D^c)
    \\&\leq \mathbb{P}_{\mu}(D^c)^{-1}  e^{\eta_0(t-k-1)}\|H_t - H_{t-k-1}\|_{\mathbb P_\mu; 1+\gamma}
    \\&\quad + \mathbb{P}_{\mu}(D^c)^{-1} \mathbb P(0<H_{t-k-1}\leq 2e^{-\eta_1(t-k-1)})
    \\&\leq C e^{-(\frac{\alpha \gamma}{1+\gamma} - \eta_0)(t-k-1)}+C e^{-\delta(t-k-1)}.
\end{split}\end{equation}
    This implies the desired result in this step, since $|Y_{t,k}| \leq 2$ a.s..

    Step 2. We will show that there exist constant $C,\delta > 0$ such that for all $k\geq 0$ and $t\geq k+1$,
\begin{equation}\begin{split}
\label{thm122}
     \mathbb{\tilde{P}}_{\mu}\big[|Y_{t,k}|\mathbf{1}_{\mathcal{D}_{t,k}}\big]
     \leq  C e^{-\delta (t-k)}.
\end{split}\end{equation}
    In fact, since $|e^{ix}-e^{iy}|\leq|x-y|$ for all $x,y\in \mathbb R$, we have for all $k \geq 0$ and $t\geq k+1$,
\begin{equation}\begin{split}
\label{eq: control of Ykt}
        &\mathbb{\tilde{P}}_{\mu}\big[|Y_{t,k}|\mathbf{1}_{\mathcal{D}_{t,k}}\big]
        \\&\leq|\theta|t^{-\frac{1}{1+\beta}} \mathbb{\tilde{P}}_{\mu}\bigg[|\mathcal I_{t-k-1}^{t-k}\langle f ,X_t\rangle|\cdot\Big|\frac{1}{(e^{\alpha(k+1)}\|X_{t-k-1}\|)^{\frac{1}{1+\beta}}}-\frac{1}{\|X_t\|^{\frac{1}{1+\beta}}}\Big|\mathbf{1}_{\mathcal{D}_{t,k}}\bigg]
        \\&\leq |\theta| e^{-\frac{\alpha}{1+\beta}t}\mathbb{\tilde{P}}_{\mu}\big[|\mathcal I_{t-k-1}^{t-k}\langle f ,X_t\rangle|\cdot K_{t,k}\big],
\end{split}\end{equation}
    where
\begin{equation}
\label{def: Ktk}
    K_{t,k}
    :=\Big|\frac{H_t^{\frac{1}{1+\beta}}-H_{t-k-1}^{\frac{1}{1+\beta}}}{H_t^{\frac{1}{1+\beta}}H_{t-k-1}^{\frac{1}{1+\beta}}}\Big|\mathbf{1}_{\mathcal{D}_{t,k}}.
\end{equation}
    Note that, since $\eta_1 < \eta_0$, we have
\begin{align*}
    H_t
    &\geq H_{t-k-1}- e^{-\eta_0(t-k-1)}
    \geq 2e^{-\eta_1(t-k-1)}-e^{-\eta_0(t-k-1)}
    \\&\geq e^{-\eta_1(t-k-1)},
    \quad \text{ on } \mathcal D_{t,k}.
\end{align*}
    Therefore, for each $k \geq 0$ and $t\geq k+1$,
\begin{align*}
     &\Big|H_t^{\frac{1}{1+\beta}}-H_{t-k-1}^{\frac{1}{1+\beta}}\Big|
     \leq \frac{1}{1+\beta}\max \Big\{H_t^{-\frac{\beta}{1+\beta}},H_{t-k-1}^{-\frac{\beta}{1+\beta}}\Big\}\left|H_t-H_{t-k-1}\right|,
    \\&\leq \frac{1}{1+\beta} \max\{e^{\eta_1 (t-k-1)}, \frac{1}{2}e^{\eta_1(t-k-1)}\}^{\frac{\beta}{1+\beta}}e^{-\eta_0(t-k-1)}
    \\&\leq \frac{1}{1+\beta} e^{\eta_1 (t-k-1)} e^{-\eta_0(t-k-1)}
    =\frac{1}{1+\beta}  e^{-(\eta_0 - \eta_1)(t-k-1)},
    \quad \text{ on } \mathcal D_{t,k},
\end{align*}
    and
\begin{align*}
    |H_t^{\frac{1}{1+\beta}}H_{t-k-1}^{\frac{1}{1+\beta}}|
    \geq 2^{\frac{1}{1+\beta}} e^{-2\eta_1(t-k-1)},
    \quad \text{ on } \mathcal D_{t,k}.
\end{align*}
    Thus, there is a constant $C\geq 0$ such that,
\begin{equation}\begin{split}
\label{ineq: control of Kkt}
     K_{t,k}
     \leq C e^{-(\eta_0 - 3\eta_1)(t-k-1)},
     \quad k \geq 0, t\geq k+1.
\end{split}\end{equation}
   According to Lemma \ref{lem: control of mgtrs}, \eqref{eq: control of Ykt} and  \eqref{ineq: control of Kkt},
    there exist constants $C,C'>0$ such that  for all $k\geq 0$ and $t\geq k+1$,
\begin{equation}\begin{split}
\label{eq: Y in D}
    &\mathbb{\tilde{P}}_{\mu}\big[|Y_{t,k}|\mathbf{1}_{\mathcal{D}_{t,k}}\big]
    \leq C|\theta|e^{-\frac{\alpha}{1+\beta}t}\mathbb{\tilde{P}}_{\mu}\big[|\mathcal{I}_{t-k-1}^{t-k}\langle f,X_t\rangle|\big]e^{-(\eta_0 - 3\eta_1)(t-k-1)}
    \\&\leq \frac{C}{\mathbb{P}_{\mu}(D^c)}|\theta|e^{-\frac{\alpha}{1+\beta}t}\mathbb{P}_{\mu}\big[|\mathcal{I}_{t-k-1}^{t-k}\langle f,X_t\rangle|\big]e^{-(\eta_0 - 3\eta_1)(t-k-1)}
    \\&\leq \frac{C}{\mathbb{P}_{\mu}(D^c)}|\theta|e^{-\frac{\alpha}{1+\beta}t}\|\mathcal{I}_{t-k-1}^{t-k}\langle f,X_t\rangle\|_{\mathbb P_\mu; 1+\gamma} e^{-(\eta_0 - 3\eta_1)(t-k - 1)}
    \\&\leq C' e^{-\frac{\alpha}{1+\beta}t}e^{\frac{\alpha}{1+\gamma}t}e^{\frac{\gamma \alpha-\kappa_f(1+\gamma)b}{1+\gamma}k}e^{-(\eta_0 - 3\eta_1)(t-k)}\\&= C' e^{(\frac{\alpha}{1+\gamma}-\frac{\alpha}{1+\beta})(t-k)}e^{-(\eta_0 - 3\eta_1)(t-k-1)},
\end{split}\end{equation}
    as desired in this step.
    In the last equality, we used the fact that
\[
    -(\frac{\alpha}{1+\gamma}-\frac{\alpha}{1+\beta})
    = \alpha(1-\frac{1}{1+\gamma}) - \alpha(1-\frac{1}{1+\beta})
    = \frac{\gamma \alpha}{1+\gamma} - k_f b
    =\frac{\alpha \gamma-\kappa_f(1+\gamma)b}{1+\gamma}.
\]

    Step 3. We will show that there exist $C, \delta> 0$ such that for $t$ large enough,
\begin{equation}\label{domi-difference of Is}
    |\mathbb{\tilde{P}}_{\mu}[e^{i\theta I_1(t)}] - \mathbb{\tilde{P}}_{\mu}[e^{i\theta\tilde{I}_1(t)}]|
    \leq C t^{-\delta}.
\end{equation}
    In fact, according to \eqref{thm121} and \eqref{thm122}, there exist $C,\delta > 0$ such that
    for all $k \geq 0$ and $t\geq k+1$ we have,
\[
    \tilde{\mathbb P}_\mu\big[|Y_{t,k}|\big]
    \leq C e^{-\delta(t-k)}.
\]
    Therefore, there exist $C, \delta > 0$ and $C'$ such that for all $t \geq 1$,
\begin{equation}\begin{split}
    &\sum_{k=0}^{\lfloor t-\ln t \rfloor} \tilde {\mathbb P}_\mu\big[|Y_{t,k}|\big]
    \leq C\sum_{k=0}^{\lfloor t-\ln t \rfloor} e^{-\delta(t-k)}
    = C e^{-\delta t}\sum_{k=0}^{\lfloor t-\ln t \rfloor} e^{\delta k}
    \\&\leq C' e^{-\delta t}e^{\delta (t-\ln t)}
    = C'\frac{1}{t^{\delta}}.
\end{split}\end{equation}
    Letting $t\to\infty$ in \eqref{domi-difference of Is}, we get \eqref{toprove-1}.

    Step 4. We will show that $I_2(t) \xrightarrow[t\to \infty]{d} 0$ with respect to $\mathbb{\tilde{P}}_{\mu}$.
    In fact, let $\mathcal{E}_t:=\{\|X_t\|>t^{-1/2}e^{\alpha t}\}$. According to Proposition \ref{lem: control of XT}, there exists $C, \delta>0$ such that
\begin{equation}\begin{split}
    \mathbb{\tilde{P}}_{\mu}(\mathcal{E}^c_t)\leq \frac{1}{\mathbb{P}_{\mu}(D^c)}\mathbb{P}_{\mu}(0<e^{-\alpha t}\|X_t\|<t^{-1/2})\leq C( t^{-\delta}+e^{-\delta t}), \quad t\geq0.
\end{split}\end{equation}
    Therefore, there exists $C, \delta>0$ such that
\begin{equation}\begin{split}\label{Theorem123}
    |\mathbb{\tilde{P}}_{\mu}[(e^{i\theta I_2(t)}-1)\mathbf{1}_{\mathcal{E}^c_t}]|
    \leq 2\mathbb{\tilde{P}}_{\mu}(\mathcal{E}^c_t)\leq C(t^{-\delta}+e^{-\delta t}),
    \quad t\geq t_0.
\end{split}\end{equation}
    Choose a $\gamma\in (0,\beta)$ close enough to $\beta$ such that $\alpha(\frac{1}{1+\gamma}-\frac{1}{1+\beta})\leq \frac{1}{2(1+\beta)}$.
	According to Lemma \ref{lem: control of mgtrs}, there exist $C,C',C''>0$ such that
\begin{align*}
    &|\mathbb{\tilde{P}}_{\mu} [ (e^{i\theta I_2(t)}-1)\mathbf{1}_{\mathcal{E}_t}]|
    \leq |\theta| \mathbb{\tilde{P}}_{\mu} \big[ |I_2(t)|\mathbf{1}_{\mathcal{E}_t}\big]
    \\&\leq|\theta| t^{-\frac{1}{2(1+\beta)}}e^{-\frac{\alpha}{1+\beta}t}\Big(\sum_{k=\lceil t-\ln t \rceil}^{\lfloor t \rfloor - 1}\mathbb{\tilde{P}}_{\mu}\big[| \mathcal{I}_{t-k-1}^{t-k}\langle f,X_t\rangle|\big] + \mathbb{\tilde{P}}_{\mu}\big[| \mathcal{I}_{0}^{t-\lfloor t\rfloor}\langle f,X_t\rangle|\big]\Big)
    \\& \leq C |\theta| t^{-\frac{1}{2(1+\beta)}}e^{-\frac{\alpha}{1+\beta}t}\Big(\sum_{k=\lceil t-\ln t \rceil}^{\lfloor t \rfloor - 1}\|\mathcal{I}_{t-k-1}^{t-k}\langle f,X_t\rangle\|_{\mathbb P_\mu; 1+\gamma} + \|\mathcal I_0^{t-\lfloor t \rfloor} \langle f, X_t\rangle\|_{\mathbb P_\mu;1+\gamma}\Big)
    \\ &\leq C' |\theta| t^{-\frac{1}{2(1+\beta)}}e^{-\frac{\alpha}{1+\beta}t}\sum_{k=\lceil t-\ln t \rceil}^{\lfloor t \rfloor}e^{\frac{\alpha}{1+\gamma}t}e^{\frac{\alpha\gamma-\kappa_f(1+\gamma)b}{1+\gamma}k}\\
    &= C' |\theta| t^{-\frac{1}{2(1+\beta)}}e^{(\frac{\alpha }{1+\gamma}-\frac{\alpha }{1+\beta})t} \sum_{k=\lceil t-\ln t \rceil}^{\lfloor t \rfloor}e^{-(\frac{\alpha}{1+\gamma}-\frac{\alpha}{1+\beta})k}\\
    &\leq C' |\theta| t^{-\frac{1}{2(1+\beta)}}e^{(\frac{\alpha }{1+\gamma}-\frac{\alpha }{1+\beta})(t - \lceil t - \ln t\rceil)} \sum_{j=0}^{\infty}e^{-(\frac{\alpha}{1+\gamma}-\frac{\alpha}{1+\beta})j}\\
    &\leq C''|\theta| t^{-\frac{1}{2(1+\beta)}}t^{(\frac{\alpha}{1+\gamma}- \frac{\alpha}{1+\beta})},
    \quad t\geq t_0.
\end{align*}
	From this and \eqref{Theorem123}, we get the desired result in this step.

	Step 5. We will show that $I_3(t) \xrightarrow[t\to \infty]{\tilde {\mathbb P}_\mu \text{-} a.s.} 0$.
In fact, we have
\begin{equation}\begin{split}
	&|I_3(t)|
\leq \frac{\langle |P^\alpha_tf|,X_0\rangle}{(t\|X_t\|)^{\frac{1}{1+\beta}}}
	\leq \frac{\langle e^{\alpha t - \kappa_f b t}Qf,X_0\rangle}{(te^{\alpha t} H_t)^{\frac{1}{1+\beta}}}
	\\& = t^{-\frac{1}{1+\beta}} e^{\frac{\beta \alpha t}{1+\beta} - k_fbt} H_t^{-\frac{1}{1+\beta}} \langle Qf,X_0\rangle
	\\& = t^{-\frac{1}{1+\beta}} H_t^{-\frac{1}{1+\beta}} \langle Qf,X_0\rangle
	\xrightarrow[t\to \infty]{\tilde {\mathbb P}_\mu \text{-} a.s.} 0.
\end{split}\end{equation}

	Finally, combining Steps 3--5, we complete the proof of Theorem \ref{Theorem12}.

\subsection{The small rate case: $\alpha\beta<\kappa_fb(1+\beta)$}
 In this subsection, we prove the central limit theorem for the small rate case.

{\it Proof of Theorem \ref{Theorem13}.}\quad
According to our assumption, $f\in \mathcal P$  is fixed  and  that $\alpha \beta < \kappa_f b(1+\beta)$.
	Let $t_0 > 1$ be large enough so that
\[
	\lceil t - \ln t\rceil
	\leq \lfloor t \rfloor - 1,
	\quad t\geq t_0.
\]
	Write
\begin{align*}
	&\frac{\langle f,X_t\rangle}{\|X_t\|^{\frac{1}{1+\beta}}}
	\\&=\sum_{k=0}^{\lfloor t-\ln t \rfloor} \frac{\mathcal I_{t-k-1}^{t-k}\langle f ,X_t\rangle}{\|X_t\|^{\frac{1}{1+\beta}}}+ \Big(\sum_{k=\lceil t-\ln t \rceil}^{\lfloor t \rfloor-1} \frac{\mathcal I_{t-k-1}^{t-k}\langle f ,X_t\rangle}{\|X_t\|^{\frac{1}{1+\beta}}}+\frac{\mathcal I_0^{t-\lfloor t \rfloor}\langle f ,X_t\rangle}{\|X_t\|^{\frac{1}{1+\beta}}}\Big) +
\frac{\langle P^\alpha_t f, X_0\rangle}{\|X_t\|^{\frac{1}{1+\beta}}}
	\\&=:I'_1(t)+I'_2(t)+I'_3(t),
	\quad t\geq t_0.
\end{align*}
	Define
 \[
 	\tilde I'_1(t)
 	:=\sum_{k=0}^{\lfloor t-\ln t \rfloor}\frac{\mathcal I_{t-k-1}^{t-k}\langle f ,X_t\rangle}{( e^{\alpha(k+1)}\|X_{t-k-1}\|)^{\frac{1}{1+\beta}}},
 	\quad t > t_0.
 \]
    Let $\theta\in \mathbb R$ and $\mu\in \mathcal M_c(\mathbb R^d)$.
    Taking $\theta_k=(e^{\alpha})^{-\frac{1}{1+\beta}} \theta $ and $n={\lfloor t-\ln t \rfloor}$ in Proposition \ref{corollary31},
    we get that there exist $C,\delta > 0$ such that
\begin{align*}
    \Big|\mathbb{\tilde{P}}_{\mu} [e^{i\theta\tilde I'_1(t)} ]-\exp\Big(\sum_{k=0}^{\lfloor t-\ln t \rfloor}m_k[\theta f]\Big)\Big|
    \leq C e^{-\delta(t - \lfloor t - \ln t\rfloor)}
    \leq C \frac{1}{t^{\delta}},
    \quad t\geq 0.
\end{align*}
Hence, according to \eqref{sum-m}, we have $\tilde I'_1(t)\xrightarrow[t\to \infty]{d} \zeta$ under $\tilde {\mathbb P}_\mu$.
   So we only need to prove that $|\mathbb{\tilde{P}}_{\mu}[e^{i\theta I'_1(t)}]-\mathbb{\tilde{P}}_{\mu}[e^{i\theta\tilde I'_1(t)}]|\xrightarrow[t\to \infty]{} 0$ and $I'_i(t)\xrightarrow[t\to \infty]{d} 0,~i=2,3,$ under $\tilde {\mathbb P}_\mu$.

    Step 1. We will show that $|\mathbb{\tilde{P}}_{\mu}[e^{i\theta I'_1(t)}]-\mathbb{\tilde{P}}_{\mu}[e^{i\theta\tilde I'_1(t)}]|\xrightarrow[t\to \infty]{} 0$.
    Define for $k\geq 0$ and $t\geq k+1$,
\begin{align*}
    Y'_{t,k}
    :=\exp\Big(i\theta\frac{\mathcal I_{t-k-1}^{t-k}\langle f ,X_t\rangle}{( e^{\alpha(k+1)}\|X_{t-k-1}\|)^{\frac{1}{1+\beta}}}\Big)-\exp\Big(i\theta\frac{\mathcal I_{t-k-1}^{t-k}\langle f ,X_t\rangle}{\|X_t\|^{\frac{1}{1+\beta}}}\Big).
\end{align*}
	Then we have
\begin{equation}\begin{split}
\label{ineq: control of I1tb}
    |\mathbb{\tilde{P}}_{\mu}[e^{i\theta I'_1(t)}] - \mathbb{\tilde{P}}_{\mu} [e^{i\theta\tilde{I}'_1(t)}]|
    \leq \sum_{k=0}^{\lfloor t-\ln t \rfloor}\mathbb{\tilde{P}}_{\mu}\big[|Y'_{t,k}|\big],
    \quad t\geq t_0.
\end{split}\end{equation}
    Let $\gamma \in (0,\beta)$ be close enough to $\beta$ such that
\[
    \frac{\alpha \gamma}{1+\gamma} > \frac{\alpha}{1+\gamma} - \frac{\alpha}{1+\beta} > 0.
\]
    Fix this $\gamma$, then chose $\eta_0,\eta_1>0$ such that
\[
    \frac{\alpha \gamma}{1+\gamma} >\eta_0 > \eta_0 - 3\eta_1 > \frac{\alpha}{1+\gamma} - \frac{\alpha}{1+\beta} > 0.
\]
	Define $\mathcal D_{t,k}$ and $K_{t,k}$ as in \eqref{def: Dtk} and \eqref{def: Ktk} respectively.
	Using Lemma \ref{lem: control of mgtrs}, \eqref{eq: prob of Dtkc}, \eqref{ineq: control of Kkt} and an argument similar to that used in proving \eqref{eq: control of Ykt}, we get that there exist $C,C',\delta>0$ such that for $k\geq 0$ and $t\geq k+1$,
\begin{align*}
    &\mathbb{\tilde{P}}_{\mu}\big[|Y'_{t,k}|\big]
    = \mathbb{\tilde{P}}_{\mu}\big[|Y'_{t,k}|; \mathcal D_{t,k}\big] + \mathbb{\tilde{P}}_{\mu}\big[|Y'_{t,k}|; \mathcal D_{t,k}^c \big]
    \\& \leq |\theta| e^{-\frac{\alpha}{1+\beta} t}\mathbb{\tilde{P}}_{\mu}\big[|\mathcal I_{t-k-1}^{t-k}\langle f ,X_t\rangle|\cdot K_{t,k}\big] + 2\mathbb{\tilde{P}}_{\mu}( \mathcal D_{t,k}^c )
    \\& \leq C e^{-\frac{\alpha}{1+\beta} t} \|\mathcal I_{t-k-1}^{t-k}\langle f, X_t\rangle \|_{\mathbb P_\mu; 1+\gamma}e^{-(\eta_0 - 3\eta_1)(t-k)} + Ce^{-\delta(t-k)}
    \\& \leq C'( e^{-\frac{\alpha}{1+\beta}t}e^{\frac{\alpha}{1+\gamma}t}e^{\frac{\gamma \alpha-\kappa_f(1+\gamma)b}{1+\gamma}k}e^{-(\eta_0 - 3\eta_1)(t-k)}+ e^{-\delta(t-k)}).
\end{align*}
Since $\alpha\beta<\kappa_f(1+\beta)b$,  we have
\begin{equation}\begin{split}
\label{eq: condition for supercritical}
	&-(\frac{\alpha}{1+\gamma}-\frac{\alpha}{1+\beta})
    = \alpha(1-\frac{1}{1+\gamma}) - \alpha(1-\frac{1}{1+\beta})
    \\&> \frac{\gamma \alpha}{1+\gamma} - k_f b
    =\frac{\alpha \gamma-\kappa_f(1+\gamma)b}{1+\gamma}.
\end{split}\end{equation}
	Using this, we have that there exist $C,C',
    \delta > 0$ such that for $k\geq 0$ and $t\geq k+1$,
\begin{align*}
    &\mathbb{\tilde{P}}_{\mu}\big[|Y'_{t,k}|\big]
    \\& \leq C( e^{(\frac{\alpha}{1+\gamma} - \frac{\alpha}{1+\beta})(t-k)}e^{-(\eta_0 - 3\eta_1)(t-k)}+ e^{-\delta(t-k)})
    \\& \leq C'e^{-\delta (t-k)}.
\end{align*}
Now we can  use the  same argument used in the proof of Theorem \ref{Theorem12}
to prove $|\mathbb{\tilde{P}}_{\mu}[e^{i\theta I'_1(t)}]-\mathbb{\tilde{P}}_{\mu}[e^{i\theta\tilde I'_1(t)}]|\xrightarrow[t\to \infty]{} 0$.

	Step 2.
	We will show that $I'_2(t)\xrightarrow[t\to \infty]{d} 0$.
	Let $\gamma \in (0,\beta)$.
	According to \eqref{eq: condition for supercritical}, we can choose $\epsilon > 0$ small enough so that
\[
	q:= - \frac{\alpha \gamma-\kappa_f(1+\gamma)b}{1+\gamma}
	> \frac{\alpha}{1+\gamma}-\frac{\alpha}{1+\beta} + \frac{2\epsilon}{1+\beta} > 0.
\]
	Recall that
\[
	 A_t(\epsilon)
	:=\{\|X_t\|>e^{(\alpha-\epsilon )t}\},
	\quad t\geq 0.
\]
	According to \eqref{lem: control of XT}, there exists
	$C,\delta>0$ such that
	Therefore, there exists $C>0$ such that
\begin{equation}\begin{split}
    &|\mathbb{\tilde{P}}_{\mu}[(e^{i\theta I'_2(t)}-1)\mathbf{1}_{A_t(\epsilon)^c}]|
    \leq 2\mathbb{\tilde{P}}_{\mu}(A_t(\epsilon)^c)\leq \frac{2}{\mathbb{P}_{\mu}(D^c)}\mathbb{P}_{\mu}(0<e^{-\alpha t}\|X_t\|<e^{-\epsilon t})
    \\&\leq C(e^{-\epsilon\delta t}+e^{-\delta t}).
    \quad t\geq t_0.
\end{split}\end{equation}
	According to Lemma \ref{lem: control of mgtrs}, there exist $C,C',C''>0$ such that for all
	$t\ge t_0$,
\begin{align*}
    &|\mathbb{\tilde{P}}_{\mu} [ (e^{i\theta I'_2(t)}-1)\mathbf{1}_{A_t(\epsilon)^c}]|
    \leq |\theta| \mathbb{\tilde{P}}_{\mu} \big[ |I'_2(t)|\mathbf{1}_{A_t(\epsilon)^c}\big]
    \\&\leq|\theta| e^{-\frac{(\alpha - \epsilon )t}{1+\beta}} \Big(\sum_{k=\lceil t-\ln t \rceil}^{\lfloor t \rfloor - 1}\mathbb{\tilde{P}}_{\mu}\big[| \mathcal{I}_{t-k-1}^{t-k}\langle f,X_t\rangle|\big] + \mathbb{\tilde{P}}_{\mu}\big[| \mathcal{I}_{0}^{t-\lfloor t\rfloor}\langle f,X_t\rangle|\big]\Big)
    \\& \leq C  e^{-\frac{(\alpha - \epsilon )t}{1+\beta}} \Big(\sum_{k=\lceil t-\ln t \rceil}^{\lfloor t \rfloor - 1}\|\mathcal{I}_{t-k-1}^{t-k}\langle f,X_t\rangle\|_{\mathbb P_\mu; 1+\gamma} + \|\mathcal I_0^{t-\lfloor t \rfloor} \langle f, X_t\rangle\|_{\mathbb P_\mu;1+\gamma}\Big)
    \\ &\leq C'  e^{-\frac{(\alpha - \epsilon )t}{1+\beta}} \sum_{k=\lceil t-\ln t \rceil}^{\lfloor t \rfloor}e^{\frac{\alpha}{1+\gamma}t}e^{\frac{\alpha\gamma-\kappa_f(1+\gamma)b}{1+\gamma}k}\\
    &\leq C'  e^{-\frac{\epsilon}{1+\beta} t}e^{(\frac{\alpha }{1+\gamma}-\frac{\alpha }{1+\beta} + \frac{2\epsilon}{1+\beta})t} \sum_{k=\lceil t-\ln t \rceil}^{\lfloor t \rfloor}e^{\frac{\alpha\gamma-\kappa_f(1+\gamma)b}{1+\gamma}k}\\
    &\leq C'  e^{-\frac{\epsilon}{1+\beta} t} e^{qt} \sum_{k=\lceil t-\ln t \rceil}^{\lfloor t \rfloor}e^{-qk}\\
    &\leq C'  e^{-\frac{\epsilon}{1+\beta} t} e^{q(t - \lceil t - \ln t\rceil)} \sum_{j=0}^{\infty}e^{-qj}\leq C'' e^{-\frac{\epsilon}{1+\beta} t} t^q.
\end{align*}
	Therefore, we get the desired result in this step.

	Step 3. We will show that $I'_3(t) \xrightarrow[t\to \infty]{\tilde {\mathbb P}_\mu \text{-} a.s.} 0$.
In fact, we have
\begin{equation}\begin{split}
	&|I'_3(t)|
     \leq \frac{\langle |P^\alpha_tf|,X_0\rangle}{\|X_t\|^{\frac{1}{1+\beta}}}
	\leq \frac{\langle e^{\alpha t - \kappa_f b t}Qf,X_0\rangle}{(e^{\alpha t} H_t)^{\frac{1}{1+\beta}}}
	\\& = e^{(\frac{\beta \alpha }{1+\beta} - k_fb)t} H_t^{-\frac{1}{1+\beta}} \langle Qf,X_0\rangle
	\xrightarrow[t\to \infty]{\tilde {\mathbb P}_\mu \text{-} a.s.} 0.
\end{split}\end{equation}
	The proof is complete.
\subsection{Proof of Theorem \ref{theorem 1.6}}\label{large rate again}
In this subsection, we are back to the large branching rate case, and aim to prove Theorem \ref{theorem 1.6}.

For each $g\in \mathcal{C}_l$, recall the definition of $I_tg$ in \eqref{definition of Itf}.
By Fubini's theorem, we get
\begin{align}\label{equ: transform of mn}
    &\bar{m}_n[g]=e^{\alpha(n-1)}\int_0^1 e^{\alpha s}\langle \eta(iI_{s+n-1}g)^{1+\beta}, \varphi\rangle ds\\
    &=e^{\alpha(n-1)}\langle \int_0^1 P_s^{\alpha}\langle \eta(iI_{s+n-1}g)^{1+\beta}ds, \varphi\rangle\\
    &=e^{\alpha(n-1)}\langle \int_0^1 P_{1-s}^{\alpha}\langle \eta(iI_{n-s}g)^{1+\beta}ds, \varphi\rangle=e^{\alpha(n-1)}\langle Z_1(-I_ng), \varphi\rangle.
\end{align}

Recall that $\Big\{\{H^p_t, \mathscr{F}_t\}_{t\geq 0};p\in \mathbb{Z}_+^d\Big\}$ is a family of martingale defined in Lemma \ref{lemma26}.

\begin{lem}
    For each $n>0$ and $\mu\in\mathcal{M}_c(\mathbb{R}^d)$, under $\mathbb{P}(\cdot|D^c)$,
    \begin{align}
        \bar{\gamma}_{t,n}:=\sum_{p\in\mathcal{N}}a_p\frac{H^p_{t+n-1}-H^p_{t+n}}{e^{-(\alpha-|p|b)t}(e^{-\alpha(n-1)}\|X_{t+n-1}\|)^{\frac{1}{1+\beta}}}\xrightarrow{d}\bar{\zeta}_n,
    \end{align}
    where $\bar{\zeta}_n$ is an $(1+\beta)$-stable random variable with characteristic function
    \begin{align}
        \mathbb{E}[e^{i\theta \bar{\zeta}_n}]=\exp[\bar{m}_n[\theta g]],\quad \theta\in \mathbb{R}.
    \end{align}
\end{lem}
\begin{proof}
We only need to show that
\begin{align}
    \mathbb{P}_{\mu}[\exp(i\bar{\gamma}_{t,n}); D^c]
    \xrightarrow[t\rightarrow \infty]{}\mathbb{P}_{\mu}(D^c)\exp(\bar{m}_n[g]),
    \quad \mu \in \mathcal M_c(\mathbb R^d), n \geq 0.
\end{align}
Put
$\theta_{t,p,n}=a_p\frac{e^{-(\alpha-|p|b)n}}{(e^{-\alpha(n-1)}\|X_{t+n-1}\|)^{\frac{1}{1+\beta}}}$ and $A_t(\epsilon)=\{\|X_t\|\geq e^{(\alpha-\epsilon)t}\}$.
We get that
\begin{align}
    \bar{\gamma}_{t,n}=\sum_{p\in \mathcal{N}}\theta_{t,p,n}(e^{\alpha-|p|b}\langle \phi_p, X_{t+n-1}\rangle-\langle \phi_p, X_{t+n}\rangle).
\end{align}

	Step 1. We will show that for any $\epsilon > 0, n\geq 0$, and $t\geq 0$, we have
\begin{align}
    \big|\mathbb{P}_{\mu}\big[e^{i\bar{\gamma}_{t,n}}-e^{\bar{m}_n[g]}; D^c\big]\big|
    \leq J'_1(t,n,\epsilon)+J'_2(t,n,\epsilon)+J'_3(t,n,\epsilon),
\end{align}
	where
\begin{align}
\label{eq: Def of JJ1}
	J'_1(t,n,\epsilon)
	&:= \mathbb{P}_{\mu}\big[|\langle K_1(-\sum_{p\in \mathcal{N}}\theta_{t,p,n}\phi_p, X_{t+n-1}\rangle|; A_{t+n-1}(\epsilon) \big],
	\\ J'_2(t,n,\epsilon)
	&:= \mathbb{P}_{\mu}\big[|\langle Z_1(-\sum_{p\in \mathcal{N}}\theta_{t,p,n}\phi_p),X_{t+n-1}\rangle-\bar{m}_n[g]|; A_{t+n-1}(\epsilon)\big],\quad\mbox{and}
	\\ J'_3(t,n, \epsilon)
	&:=2\mathbb{P}_{\mu}(A_{t+n-1}(\epsilon)\Delta D^c).
\end{align}
 From the definitions of $K_1$ and $Z_1$, we have
\begin{align}
\label{eq: need11}
    &\displaystyle\mathbb{P}_{\mu}[e^{i\bar{\gamma}_{t,n}}|\mathscr{F}_{t+n-1}]
    =\mathbb{P}_{\mu}[e^{i\sum_{p\in \mathcal{N}}\theta_{t,p,n}(e^{\alpha-|p|b}\langle \phi_p,X_{t+n-1}\rangle-\langle \phi_p, X_{t+n}\rangle)}|\mathscr{F}_{t+n-1}]
    \\&=\displaystyle e^{\langle (K_1+Z_1)(-\sum_{p\in \mathcal{N}}\theta_{t,p,n}\phi_p), X_{t+n-1} \rangle}.
\end{align}

 According to \eqref{eq: need11} and the fact that
\[
	|e^{-x} - e^{-y}| \leq |x-y|,\quad x,y \in \mathbb C_+,
\]
we have
\begin{align}
\label{eq: inequality that will used later1}
    &\big|\mathbb{P}_{\mu}\big[e^{i\bar{\gamma}_{t,n}}-e^{\bar{m}_n[g]}; D^c\big]\big|
    \\& \leq \mathbb{P}_{\mu}\Big[\big| \mathbb{P}_{\mu}[e^{i\bar{\gamma}_{t,n}}-e^{\bar{m}_n[g]}; D^c | \mathscr F_{t+n-1}]\big|\Big]
    \\& \leq \mathbb{P}_{\mu}\Big[\big| \mathbb{P}_{\mu}[e^{i\bar{\gamma}_{t,n}}-e^{\bar{m}_n[g]}; A_{t+n-1}(\epsilon)| \mathscr F_{t+n-1}]\big| + 2\mathbb P_\mu(A_{t+n-1}(\epsilon) \Delta D^c| \mathscr F_{t+n-1})\Big]
    \\& = \mathbb{P}_{\mu}\Big[ \big|\mathbb{P}_{\mu}[e^{i\bar{\gamma}_{t,n}}| \mathscr F_{t+n-1}]-e^{\bar{m}_n[g]}\big|;A_{t+n-1}(\epsilon)\Big] + J'_3(t,n,\epsilon)
    \\&=\mathbb{P}_{\mu}\Big[\big|e^{\langle (K_1+Z_1)(-\sum_{p\in \mathcal{N}}\theta_{t,p,n}\phi_p), X_{t+n-1}\rangle}-e^{\bar{m}_n[g]}\big|;A_{t+n-1}(\epsilon)\Big]+ J'_3(t,n,\epsilon)
    \\&\leq \mathbb{P}_{\mu}\Big[\big|\langle (k_1+Z_1)(-\sum_{p\in \mathcal{N}}\theta_{t,p,n}\phi_p), X_{t+n-1}\rangle-\bar{m}_n[g]\big|;A_{t+n-1}(\epsilon)\Big]+ J'_3(t,n,\epsilon)
    \\&\leq J'_1(t,n,\epsilon)+J'_2(t,n,\epsilon)+J'_3(t,n,\epsilon)
    \quad n\geq 0, t\geq 0, \epsilon> 0.
\end{align}

Step 2.  We will show that for each $\epsilon$ small enough, there exist constants $C,\delta>0$ such that,
    \begin{align}
    \label{eq:large rate J1}
      J'_1(t,n,\epsilon)\leq C e^{-\delta(t+n)},\quad n\geq 1, \, t\geq 0.
    \end{align}
	In fact, let $\delta_0$ be the constant in Lemma \ref{lem: upper bound for usgx}.(7) and $R$ be the $(\theta^{1+\beta}\vee \theta^{1+\beta+\delta_0})$-controller. Then we have,
\begin{align}
   & |K_1(-\sum_{p\in \mathcal{N}}\theta_{t,p,n}\phi_p)|\mathbf{1}_{A_{t+n-1}(\epsilon)}
   \leq R(|\sum_{p\in\mathcal{N}}\theta_{t,p,n}\phi_p|)\mathbf{1}_{A_{t+n-1}(\epsilon)}
   \\&\leq R \Big(\frac{\sum_{p\in\mathcal{N}}|a_p|e^{-(\alpha-|p|b)n}|\phi_p|}{\big(e^{-\alpha(n-1)}e^{(\alpha-\epsilon)(t+n-1)}\big)^{\frac{1}{1+\beta}}}\Big)
   \\&\leq \sum_{\rho\in\{\delta_0,1\}}\Big(\frac{e^{-(\alpha-Kb)n}}{\big(e^{-\alpha(n-1)}e^{(\alpha-\epsilon)(t+n-1)}\big)^{\frac{1}{1+\beta}}}\Big)^{1+\beta+\rho}R|h|
   \\&=\sum_{\rho\in\{\delta_0,1\}}e^{-\alpha\frac{1+\beta+\rho}{1+\beta}}e^{-\frac{1+\beta+\rho}{1+\beta}(\alpha\beta-K(1+\beta)b)n}e^{-\frac{1+\beta+\rho}{1+\beta}(\alpha-\epsilon)(t+n-1)}S|h|
   \\&\leq \sum_{\rho\in\{\delta_0,1\}}e^{-\frac{1+\beta+\rho}{1+\beta}(\alpha-\epsilon)(t+n-1)}R|h|,
   \quad n\geq 0, t\geq 0, \epsilon > 0,
\end{align}
where $h=\sum_{p\in \mathcal{N}}|a_p\phi_p|$.
Thus for each $n\geq 0,  t\geq 0, \epsilon > 0$,
\begin{align}
\label{eq: estimate of J11}
     J'_1(t,n,\epsilon)&
     \leq  \sum_{\rho\in\{\delta_0,1\}}e^{-\frac{1+\beta+\rho}{1+\beta}(\alpha-\epsilon)(t+n-1)}\mathbb{P}_{\mu}[\langle R|h|,X_{t+n-1}\rangle]\\
     & \leq C\sum_{\rho\in\{\delta_0,1\}}\exp\Big\{-\Big(\alpha\frac{\rho}{1+\beta}-\epsilon\frac{1+\beta+\rho}{1+\beta}\Big)(t+n-1)\Big\}.
     \end{align}

Step 3.
	We will show that for each $\epsilon$ small enough, there exists constant $C,\delta>0$ such that,
\begin{align}
\label{eq:31step31}
    J'_2(t,n,\epsilon)
    \leq C e^{-\delta(t+n)},
    \quad n\geq 1, t\geq 0.
\end{align}

In fact, according to the definitions of $Z_1$ and $\bar{m}_n$, and \eqref{equ: transform of mn}, we have
\begin{align}
    &\langle Z_1(-\sum_{p\in \mathcal{N}}\theta_{t,p,n}\phi_p), X_{t+n-1}\rangle-\bar{m}_n[g]\\
    &=e^{\alpha(n-1)}\Big(\frac{\langle Z_1(-I_ng),X_{t+n-1}\rangle}{\|X_{t+n-1}\|}-\langle Z_1(-I_ng),\varphi\rangle\Big),\\
    & t\geq 0, n\geq1.
\end{align}
Therefore,
\begin{align}
    &J'_2(t,n,\epsilon)
	= \mathbb{P}_{\mu}\big[|\langle Z_1(-\sum_{p\in\mathcal{N}}\theta_{t,p,n}\phi_p),X_{t+n-1}\rangle-\bar{m}_n[g]|; A_{t+n-1}(\epsilon)\big]\\
	&\leq e^{\alpha(n-1)}\mathbb{P}_{\mu}\Big[\big|\frac{\langle Z_1(-I_ng),X_{t+n-1}\rangle}{\|X_{t+n-1}\|}-\langle Z_1(-I_ng),\varphi\rangle\big|;A_{t+n-1}(\epsilon)\Big]\\
	&\leq e^{\alpha(n-1)}e^{-(\alpha-Kb)(1+\beta)n}e^{-(\alpha-\epsilon)(t+n-1)}\mathbb{P}_{\mu}
[\langle \bar {g}_n, X_{t+n-1}\rangle],
	\quad t\geq 0, n\geq 1,
\end{align}
where
\begin{align}
    \bar{g}_n:=\frac{Z_1(-I_ng)-\langle Z_1(-I_ng),\varphi\rangle}{e^{-(\alpha-Kb)(1+\beta)n}}.
\end{align}
Using Lemma \ref{control of gn} and Lemma \ref{lem: control moment} (3) with $\kappa=1$,
 there exists $C>0$ such that
\begin{align}
    \mathbb{P}_{\mu}[\langle \bar{g}_n, X_{t+n-1}\rangle]\leq Ce^{\frac{\alpha \gamma}{1+\gamma}(t+n-1)},\quad t\geq0, n\geq 1.
\end{align}
Therefore,
\begin{align}
    &J'_2(t,n,\epsilon)\leq e^{\alpha(n-1)}e^{-(\alpha-Kb)(1+\beta)n}e^{-(\alpha-\epsilon)(t+n-1)}\mathbb{P}_{\mu}[\langle \bar{g}_n, X_{t+n-1}\rangle]\\
    &\leq C e^{\alpha(n-1)}e^{-(\alpha-Kb)(1+\beta)n}e^{-(\alpha-\epsilon)(t+n-1)}e^{\frac{\alpha }{1+\gamma}(t+n-1)}\\
    &=Ce^{-\alpha}e^{-(\alpha\beta-K(1+\beta)b)n}e^{-(\frac{\alpha\gamma}{1+\gamma}-\epsilon)(t+n-1)}\\&\leq Ce^{-(\frac{\alpha\gamma}{1+\gamma}-\epsilon)(t+n-1)},\quad t\geq 0,n\geq 1.
\end{align}
	Step 4. Using similar arguments in the proof of step 4 of Lemma \ref{lem: mainlemma}, we can show that for each $\epsilon< \alpha$ there exist $C,\delta>0$ such that
\begin{align}\label{ineq: control of J31}
    J'_3(t,n,\epsilon)\leq Ce^{-\delta (t+n)}\quad t\geq0, n\geq 1.
\end{align}
Finally, combining the results in steps 1-4. and noticing that, if $\epsilon>0$ is chosen small enough then $J_{i}, i = 1,2,3$, converge to $0$ exponentially fast while $t\rightarrow\infty$, we get the desired result.
	This gives the desired result.
\end{proof}

\begin{lem}\label{lem:lemma03}
    For each $\theta\in \mathbb{R}$ and $\mu \in \mathcal{M}_c(\mathbb{R}^d)$, there exist constants $C,\delta>0$ such that for all $t\geq 0$, $m\in\mathbb{N}$, we have
    \begin{align}\label{ineq: next we will need}
        \Big|\tilde{\mathbb{P}}_{\mu}\Big[\prod_{n=1}^m\exp(i\theta\bar{\gamma}_{t,n})-\prod_{n=1}^m\exp(\bar{m}_n[\theta g])\Big]\Big|\leq C e^{-\delta t}.
    \end{align}
\end{lem}
\begin{proof}
Step 1. Similar to the proof of Corollary \ref{cor: used in next corollary}, we can show that  for any  $\theta\in \mathbb{R}$ and $\mu\in \mathcal{M}_c(\mathbb{R}^d)$, there exist constants $C,\delta>0$ such that, for any $t\geq 0$, $n\geq 1$, we have
\begin{align}
    \mathbb{P}_{\mu}\Big[\big|\mathbb{P}_{\mu}[e^{i\theta\bar{\gamma}_{t,n}}-e^{\bar{m}_n[\theta g]}; D^c | \mathscr F_{t+n-1}]\big|\Big]\leq Ce^{-\delta(t+n-1)}.
\end{align}
Step 2.   Fix $t\geq 0$, $m\in \mathbb{N}$ and $\theta\in \mathbb{R}$.
    For each $n=1,\cdots,m$, we define
    \[\tilde{a}_n:=\mathbb{\tilde{P}}_{\mu}\Big(\prod_{l=0}^{n}\exp\left(i\theta\bar{\gamma}_{t,n}\right)\prod_{l=n+1}^{m}\exp(\bar{m}_l[\theta g])\Big),\]
     where by convention the product is $1$ for $n=0$. Then we get for each $n > 0$,
    \begin{align*}
        &\tilde{a}_{n-1} - \tilde{a}_n
        \\&=\mathbb{P}_{\mu}(D^c)^{-1}\mathbb{P}_{\mu}\Big[\prod_{l=0}^{n-1}e^{i\theta\bar{\gamma}_{t,n}}\Big(e^{\bar{m}_n[\theta g]}-e^{i\theta \bar{\gamma}_{t,n}}\Big);D^c\Big]\prod_{l=n+1}^{m}e^{\bar{m}_l[\theta g]}
        \\&=\mathbb{P}_{\mu}(D^c)^{-1}\mathbb{P}_{\mu}\Big[\prod_{l=0}^{n-1}e^{i\theta\bar{\gamma}_{t,n}}\mathbb{P}_{\mu}[e^{\bar{m}_n[\theta g]}-e^{i\theta \bar{\gamma}_{t,n}};D^c|\mathscr{F}_{t+n-1}]\Big]\prod_{l=n+1}^{m}e^{\bar{m}_l[\theta g]}.
    \end{align*}
   According to step 1 and Proposition \ref{cor: alpha stable rv 1},
    there exist $C,\delta>0$ such that for any $n=1,\cdots, m$ and $t\geq0$, we have
    \begin{align*}
        &|\tilde{a}_{n-1}- \tilde{a}_n|
        \\&\leq \frac{1}{\mathbb{P}_{\mu}(D^c)}\mathbb{P}_{\mu}\Big[\big|\mathbb P_\mu[e^{i\theta\bar{\gamma}_{t,n}}-e^{\bar{m}_n[\theta g]}; D^c\big|\mathscr{F}_{t+n-1}]\big|\Big]
        \\& \leq C e^{-\delta(t+n-1)}.
    \end{align*}
Therefore,
\begin{align}
    \text{LHS of \eqref{ineq: next we will need}}&= \left|\tilde{a}_{0}-\tilde{a}_m\right|
    \leq\sum_{n=1}^m\left|\tilde{a}_{n-1}-\tilde{a}_n\right|
    \leq \sum_{n=1}^m C e^{-\delta(t+n-1)}.
\end{align}
	Notice that $C, \delta>0$ are independent of the choice of $t\geq 0$, $m\in \mathbb{N}$.
\end{proof}
Now let
\begin{align}
    \tilde{\gamma}_{t,n}:=\sum_{p\in \mathcal{N}}a_p\frac{H^p_{t+n-1}-H^p_{t+n}}{e^{-(\alpha-|p|b)t}\|X_{t}\|^{\frac{1}{1+\beta}}},\quad t\geq 0, n\geq 1.
\end{align}
\begin{lem}\label{lem: lemma04}
     For each $\theta\in \mathbb{R}$ and $\mu\in \mathcal{M}_c(\mathbb{R}^d)$, there exist constants $C,\delta>0$
such that for all $m\in \mathbb{N}$ and $t\geq 0$, we have
\begin{align}
    \Big|\tilde{\mathbb{P}}_{\mu}[\prod_{n=1}^me^{i\theta \tilde{\gamma}_{t,n}}]-\tilde{\mathbb{P}}_{\mu}[\prod_{n=1}^me^{i\theta \bar{\gamma}_{t,n}}]\Big|\leq C m e^{-\delta t}.
\end{align}
\end{lem}
\begin{proof}
According to \cite[Lemma 3.4.3]{Durrett2010Probability},
\begin{align}\label{ineq: used next 3}
    \Big|\tilde{\mathbb{P}}_{\mu}[\prod_{n=1}^me^{i\theta \tilde{\gamma}_{t,n}}]-\tilde{\mathbb{P}}_{\mu}[\prod_{n=1}^me^{i\theta \bar{\gamma}_{t,n}}]\Big|\leq \sum_{n=1}^m\tilde{\mathbb{P}}_{\mu}[|Y''_{t,n}|],
\end{align}
where
\begin{align}
    Y''_{t,n}:=e^{i\theta\tilde{\gamma}_{t,n}}-e^{i\theta\bar{\gamma}_{t,n}},\quad t\geq 0, n\geq 1, \theta\in \mathbb{R}.
\end{align}
    Let $\gamma \in (0,\beta)$ be close enough to $\beta$ such that
\[
    \frac{\alpha \gamma}{1+\gamma} > \frac{\alpha}{1+\gamma} - \frac{\alpha}{1+\beta} > 0,
\]
and
\[
   \alpha\gamma>K(1+\gamma)b.
  \]
  Fix this $\gamma$, and then choose $\eta_0,\eta_1>0$ such that
\[
    \frac{\alpha \gamma}{1+\gamma} >\eta_0 > \eta_0 - 3\eta_1 > \frac{\alpha}{1+\gamma} - \frac{\alpha}{1+\beta} > 0.
\]
    Define for all $n \geq 1$ and $t\geq 0$,
\begin{align}
\label{def: Dtk1}
    \mathcal{D}_{t,n}&:=\left\{|H_t-H_{t+n-1}|\leq  e^{-\eta_0 t}, H_{t}> 2e^{-\eta_1t}\right\}.
\end{align}

Step 1. Similar to step 1 in the proof of Theorem \ref{Theorem12},
we can show that there exist constants $C,\delta >0$ such that for all $n \geq 1$ and $t\geq 0$,
\begin{align}
\label{thm12111}
    \mathbb{\tilde{P}}_{\mu}\big[|Y''_{t,n}|\mathbf{1}_{\mathcal{D}^c_{t,n}}\big]
    \leq C e^{-\delta t}.
\end{align}
\begin{comment}

    In fact, let $C_1$ be the constant in  Lemma 2.9 corresponding to this fixed $\gamma$ and $C_2$ be the constant in Lemma 2.10.
    By Chebyshev's inequality, there exists $C>0$ such that for all $n \geq 1$ and $t\geq 0$,
\begin{align}
\label{eq: prob of Dtkc11}
    &\mathbb{\tilde{P}}_{\mu}(\mathcal{D}_{t,n}^c)
    \\&\leq \mathbb{\tilde{P}}_{\mu}(|H_t-H_{t+n-1}| > e^{-\eta_0 t})+\mathbb{\tilde{P}}_{\mu}(H_{t}\leq 2e^{-\eta_1 t}),
    \\&\leq \mathbb{P}_{\mu}(D^c)^{-1}e^{\eta_0 t}\mathbb{P}_{\mu}[|H_t-H_{t+n-1}|]+\mathbb{\tilde{P}}(H_t\leq 2e^{-\eta_1 t})
    \\&\leq \mathbb{P}_{\mu}(D^c)^{-1}  e^{\eta_0 t}\|H_t - H_{t+n-1}\|_{\mathbb P_\mu; 1+\gamma}+\mathbb{\tilde{P}}(H_{t}\leq 2e^{-\eta_1t})
    \\&\leq C_1  \mathbb{P}_{\mu}(D^c)^{-1}  e^{-(\frac{\alpha \gamma}{1+\gamma} - \eta_0)t}+C_2 e^{-\eta_1\beta t}.
\end{align}
    This implies the desired result in this step, since $|Y_{t,k}| \leq 2$ a.s..
\end{comment}

    Step 2. We will show that for each $\mu \in \mathcal{M}_c(\mathbb{R}^d)$, there exist constants $C,\delta > 0$, such that for all $n\geq 1$ and $t\geq 0$,
\begin{align}
\label{thm12211}
     \mathbb{\tilde{P}}_{\mu}\big[|Y''_{t,n}|\mathbf{1}_{\mathcal{D}_{t,n}}\big]\leq Ce^{-\delta t}.
\end{align}
    In fact, since $|e^{ix}-e^{iy}|\leq|x-y|$ for all $x,y\in \mathbb R$, we have, for all $t \geq 0$ and $n\geq 1$,
    \begin{align}\label{large: used next}
        &\tilde{\mathbb{P}}_{\mu}\big[|Y''_{t,n}|\mathbf{1}_{\mathcal{D}_{t,n}}\big]\\
        &\leq |\theta|\sum_{p\in\mathcal{N}}|a_p|e^{(\alpha-|p|b)t}\tilde{\mathbb{P}}_{\mu}\Big[|H_{t+n-1}^p-H_{t+n}^p|\cdot\Big|\frac{1}{(e^{-\alpha(n-1)}\|X_{t+n-1}\|)^{\frac{1}{1+\beta}}}-\frac{1}{\|X_t\|^{\frac{1}{1+\beta}}}\Big|\mathbf{1}_{\mathcal{D}_{t,n}}\Big]\\
        &\leq |\theta|\sum_{p\in\mathcal{N}}|a_p|e^{(\alpha-|p|b)t}e^{-\frac{\alpha}{1+\beta}t}\tilde{\mathbb{P}}_{\mu}\Big[|H_{t+n-1}^p-H_{t+n}^p|\cdot K'_{t,n}\Big],
    \end{align}
    where
    \begin{align}
        K'_{t,n}
    :=\Big|\frac{H_t^{\frac{1}{1+\beta}}-H_{t+n-1}^{\frac{1}{1+\beta}}}{H_t^{\frac{1}{1+\beta}}H_{t+n-1}^{\frac{1}{1+\beta}}}\Big|\mathbf{1}_{\mathcal{D}_{t,n}}.
    \end{align}
\begin{comment}

     Note that, since $\eta_1 < \eta_0$, we have
\begin{align*}
    H_{t+n-1}
    &\geq H_{t}- e^{-\eta_0 t}
    \geq 2e^{-\eta_1t}-e^{-\eta_0 t}
    \\&\geq e^{-\eta_1 t},
    \quad \text{ on } \mathcal D_{t,n}.
\end{align*}
    Therefore, for each $t \geq 0$ and $n\geq 1$,
\begin{align*}
     &\Big|H_t^{\frac{1}{1+\beta}}-H_{t+n-1}^{\frac{1}{1+\beta}}\Big|
     \leq \frac{1}{1+\beta}\max \Big\{H_t^{-\frac{\beta}{1+\beta}},H_{t+n-1}^{-\frac{\beta}{1+\beta}}\Big\}\left|H_t-H_{t+n-1}\right|,
    \\&\leq \frac{1}{1+\beta} \max\{e^{\eta_1 t}, \frac{1}{2}e^{\eta_1 t}\}^{\frac{\beta}{1+\beta}}e^{-\eta_0 t}
    \\&\leq \frac{1}{1+\beta} e^{\eta_1 t} e^{-\eta_0 t}
    =\frac{1}{1+\beta}  e^{-(\eta_0 - \eta_1) t},
    \quad \text{ on } \mathcal D_{t,n},
\end{align*}
    and
\begin{align*}
    |H_t^{\frac{1}{1+\beta}}H_{t+n-1}^{\frac{1}{1+\beta}}|
    \geq 2^{\frac{1}{1+\beta}} e^{-2\eta_1t},
    \quad \text{ on } \mathcal D_{t,n}.
\end{align*}
\end{comment}

Since $\eta_1 < \eta_0$,  using similar arguments used in  step 2 of the proof of Theorem \ref{Theorem12},
we can show that, there is a constant $C\geq 0$ such that,
\begin{align}
\label{ineq: control of Kkt1}
     K'_{t,n}
     \leq C e^{-(\eta_0 - 3\eta_1) t},
     \quad t \geq 0, n\geq 1.
\end{align}
   According to Lemma \ref{lem: control of wt},
    there exist constants $C,C',C''>0$ such that  for all $t\geq 0$ and $n\geq 1$,
    \begin{align}\label{ineq:used next 2}
        &\tilde{\mathbb{P}}_{\mu}\big[|Y_{t,n}|\mathbf{1}_{\mathcal{D}_{t,n}}\big]\leq \frac{C}{\mathbb{P}_{\mu}(D^c)} |\theta|\sum_{p\in\mathcal{N}}|a_p|e^{(\alpha-|p|b)t}e^{-\frac{\alpha}{1+\beta}t}e^{-(\eta_0-3\eta_1)t}\mathbb{P}_{\mu}\Big[|H_{t+n-1}^p-H_{t+n}^p|\Big]\\
        &\leq \frac{C}{\mathbb{P}_{\mu}(D^c)} |\theta|\sum_{p\in\mathcal{N}}|a_p|e^{(\alpha-|p|b)t}e^{-\frac{\alpha}{1+\beta}t}e^{-(\eta_0-3\eta_1)t}\|H_{t+n-1}^p-H_{t+n}^p\|_{\mathbb{P}_{\mu};1+\gamma}
        \\&\leq C' |\theta|\sum_{p\in\mathcal{N}}|a_p|e^{(\alpha-|p|b)t}e^{-\frac{\alpha}{1+\beta}t}e^{-(\eta_0-3\eta_1)t}e^{-\frac{1}{1+\gamma}(\alpha\gamma-|p|(1+\gamma)b)(t+n-1)}\\
        &=C'|\theta|\sum_{p\in\mathcal{N}}|a_p|\exp\Big\{\Big(\frac{\alpha}{1+\gamma}-\frac{\alpha}{1+\beta}-(\eta_0-3\eta_1)\Big)t\Big\}\cdot e^{-\frac{1}{1+\gamma}(\alpha\gamma-|p|(1+\gamma)b)(n-1)}.\\
        &\leq C''|\theta|\exp\Big\{\Big(\frac{\alpha}{1+\gamma}-\frac{\alpha}{1+\beta}-(\eta_0-3\eta_1)\Big)t\Big\}.
    \end{align}
    We get \eqref{thm12211} since $\frac{\alpha}{1+\gamma}-\frac{\alpha}{1+\beta}<\eta_0-3\eta_1$ and $\alpha\gamma>Kb(1+\gamma)\geq |p|b(1+\gamma)$.
  Combinning \eqref{ineq: used next 3}, Steps 1-2, we get the result immediately. Notice that $\delta$ is indenpendent of $m$ and $t$.
\end{proof}
\begin{lem}\label{lem: lemma05}
For each $\theta\in \mathbb{R}$ and $\mu \in \mathcal{M}_c(\mathbb{R}^d)$, there exist constants $C,\delta_1,\delta_2>0$ such that
\begin{align}
  \Big|\tilde{\mathbb{P}}_{\mu}\Big[\exp(i\theta \sum_{n=m}^{\infty}\tilde{\gamma}_{t,n})-1\Big]\Big|\leq C(e^{-\delta_1 t}+e^{\delta_1 t}e^{-\delta_2 m}),\quad m\in \mathbb{N}, t\geq 0.
\end{align}
\end{lem}
\begin{proof}
Recall that $A_t(\epsilon):=\{\|X_t\|>\|\mu\|e^{(\alpha-\epsilon) t}\}$. According to Proposition \ref{lem: control of XT}, there exists
	$C,\delta>0$ such that for all $t\geq 0$
\begin{align}\label{ineq: control of gamma_t on dc}
     \Big|\tilde{\mathbb{P}}_{\mu}\Big[\big(\exp(i\theta \sum_{n=m}^{\infty}\tilde{\gamma}_{t,n})-1\big)\mathbf{1}_{A_t(\epsilon)^c}\Big]\leq 2\tilde{P}_{\mu}(A_t(\epsilon)^c)\leq C e^{-\delta t}.
\end{align}
	 Fix $\epsilon>0$ and $\gamma\in(0,\beta)$ satisfying $\alpha\gamma>Kb(1+\gamma)$.  According to Lemma \ref{lem: control of wt}, there exist $C,C'>0$ such that
\begin{align}\label{ineq: control of gamma_t on d}
  &\Big|\tilde{\mathbb{P}}_{\mu}\Big[\big(\exp(i\theta \sum_{n=m}^{\infty}\tilde{\gamma}_{t,n})-1\big)\mathbf{1}_{A_t(\epsilon)}\Big]\Big|\\
  &\leq \frac{1}{\mathbb{P}_{\mu}(D^c)}|\theta|\sum_{n=m}^{\infty}\sum_{p\in \mathcal{N}}|a_p|\mathbb{P}_{\mu}\Big[\frac{|H_{t+n-1}^p-H_{t+n}^p|}{e^{-(\alpha-|p|b)t}\|X_t\|^{\frac{1}{1+\beta}}}\mathbf{1}_{A_t(\epsilon)}\Big]\\
  &\leq  \frac{1}{\mathbb{P}_{\mu}(D^c)}|\theta|\sum_{p\in \mathcal{N}}|a_p|e^{(\alpha-|p|b)t}e^{-\frac{\alpha-\epsilon}{1+\beta}t}\sum_{n=m}^{\infty}\|H_{t+n-1}^p-H_{t+n}^p\|_{\mathbb{P}_{\mu};1+\gamma}\\
  &\leq C|\theta|\sum_{p\in \mathcal{N}}|a_p|e^{(\alpha-|p|b)t}e^{-\frac{\alpha-\epsilon}{1+\beta}t}\sum_{n=m}^{\infty}e^{-\frac{1}{1+\gamma}(\alpha\gamma-|p|(1+\gamma)b)(t+n-1)}\\
  &\leq C'|\theta|\exp\Big\{\Big(\frac{\alpha}{1+\gamma}-\frac{\alpha}{1+\beta}+\frac{\epsilon}{1+\beta}\Big)t\Big\}e^{-\frac{1}{1+\gamma}(\alpha\gamma-K(1+\gamma)b)m}.\\
  & \qquad\theta\in\mathbb{R}, m\in \mathbb{N}, \epsilon>0, t\geq 0.
\end{align}
Combining with \eqref{ineq: control of gamma_t on dc} and \eqref{ineq: control of gamma_t on d}, we complete the proof.
\end{proof}
Now we are ready to prove Theorem \ref{theorem 1.6}.
\bigskip

{\it Proof of Theorem \ref{theorem 1.6}.}\quad
Let $\delta_0$ be the control parameter in Lemma \ref{lem: lemma04}.  Take $m=\lfloor e^{\frac{\delta_0}{2}t}\rfloor$. First note that
\begin{align}\label{decompose}
    \text{LHS of \eqref{thm: large rate}}=\sum_{n=1}^{\infty}\sum_{p\in \mathcal{N}}a_p\frac{H^p_{t+n-1}-H^p_{t+n}}{e^{-(\alpha-|p|b)t}\|X_t\|^{\frac{1}{1+\beta}}}=\sum_{n=1}^m\tilde{\gamma}_{t,n}+\sum_{n=m+1}^{\infty}\tilde{\gamma}_{t,n},\quad t\geq 0.
\end{align}
According to Lemma \ref{lem:lemma03}, there exists $C,\delta>0$ such that
\begin{align}\label{ineq: last1}
    \Big|\tilde{\mathbb{P}}_{\mu}\Big[\exp(i\theta\sum_{n=1}^m\gamma_{t,n})-\exp(\sum_{n=1}^m\bar{m}_n[\theta g])\Big]\Big|\leq C e^{-\delta t},\quad t\geq 0,\theta\in\mathbb{R}.
\end{align}
According to Lemma \ref{lem: lemma04}, there exists $C>0$ such that
\begin{align}\label{ineq: last2}
    \Big|\tilde{\mathbb{P}}_{\mu}[e^{i\theta\sum_{n=1}^m \tilde{\gamma}_{t,n}}]-\tilde{\mathbb{P}}_{\mu}[e^{i\theta \sum_{n=1}^m\gamma_{t,n}}]\Big|\leq C  e^{-\frac{\delta_0}{2} t},\quad t\geq 0, \theta\in \mathbb{R}.
 \end{align}
 According to Lemma \ref{lem: lemma05}, there exists $C,\delta_1,\delta_2>0$ such that
 \begin{align}\label{ineq: last3}
  \Big|\tilde{\mathbb{P}}_{\mu}\Big[\exp(i\theta \sum_{n=m}^{\infty}\tilde{\gamma}_{t,n})-1\Big]\Big|\leq C(e^{-\delta_1 t}+e^{\delta_1 t}e^{-\delta_2 e^{\frac{\delta_0}{2}t}}),\quad t\geq 0, \theta\in \mathbb{R}.
\end{align}
Also note that $\sum_{n=1}^\infty\bar{m}_n[\theta g]=\bar{m}[\theta g]$, see \eqref{sum-bar-m}.

Combining \eqref{decompose},\eqref{ineq: last1}, \eqref{ineq: last2} and \eqref{ineq: last3}, and letting $t\to\infty$, we get the desired result of Theorem \ref{theorem 1.6}.
\appendix
\section{}

\subsection{Analytic facts}
\begin{lem}
\label{lem: estimate of exponential remaining}
    Suppose that $z\in \mathbb C_+$. Then
\begin{equation}
\label{eq: estimate of exponential remaining}
    \Big|e^{-z} - \sum_{k=0}^n \frac{(-z)^k}{k!} \Big|
    \leq \frac{|z|^{n+1}}{(n+1)!} \wedge \frac{2|z|^{n}}{n!}, \quad n\in \mathbb Z_+.
\end{equation}
\end{lem}
\begin{proof}
    Notice that $|e^{-z}| = e^{- \operatorname{Re} z} \leq 1$.
    Therefore,
\begin{equation}
    |e^{-z} - 1| = \Big| \int_0^1 e^{-\theta z} z d\theta\Big|
    \leq |z|.
\end{equation}
    Also, notice that $|e^{-z} - 1| \leq |e^{-z}|+1 \leq 2$.
    Thus \eqref{eq: estimate of exponential remaining} is true when $n = 0$.
    Now, suppose that \eqref{eq: estimate of exponential remaining} is true when $n = m$ for some $m \in \mathbb Z_+$.
    We have
\begin{equation}\begin{split}
    &\Big|e^{-z} - \sum_{k=0}^{m+1} \frac{(-z)^k}{k!}\Big|
    = \Big| \int_0^1\Big(e^{-\theta z} - \sum_{k=0}^m \frac{(-\theta z)^k}{k!} \Big) z d\theta \Big|
    \\&\quad \leq  \Big(\int_0^1 \frac{|\theta z|^{m+1}}{(m+1)!} |z| d\theta\Big) \wedge \Big(\int_0^1 \frac{2|\theta z|^{m}}{m!} |z| d\theta\Big)
    = \frac{|z|^{m+2}}{(m+2)!} \wedge \frac{2|z|^{m+1}}{(m+1)!},
\end{split}\end{equation}
    which says that \eqref{eq: estimate of exponential remaining} is true for $n = m + 1$.
    The proof is complete.
\end{proof}

\begin{lem}
\label{lem: extension lemma for branching mechanism}
    Suppose that  $\pi$ is a measure on $(0,\infty)$ with $\int_{(0,\infty)} (y \wedge y^2) \pi(dy)< \infty$.
    Then functions
\begin{equation}
    h (z) = \int_{(0,\infty)} (e^{-zy} - 1 + zy) \pi(dy), \quad z \in \mathbb C_+
\end{equation}
and
\begin{equation}
\label{eq: deriavetive of the Poission partb}
    h'(z) = \int_{(0,\infty)}(1- e^{-zy})y \pi(dy), \quad z \in \mathbb C_+
\end{equation}
    are well defined, continuous on $\mathbb C_+$ and holomorphic on $\mathbb C_+^0$.
    Moreover,
\[
    \frac{h(z)-h(z_0)}{z-z_0} \xrightarrow[\mathbb C_+\ni z \to z_0]{} h'(z_0),\quad z_0 \in \mathbb C_+.
\]
\end{lem}
\begin{proof}
    From Lemma \ref{lem: estimate of exponential remaining}, we know that $h$ and $h'$ are well defined on $\mathbb C_+$.
    According to \cite[Theorem 3.2. \& Theorem 3.5]{SchillingSongVondracek2010Bernstein}, $h'$ is continuous on $\mathbb C_+$ and holomorphic on $\mathbb C_+^0$.

    From Lemma \ref{lem: estimate of exponential remaining}, for each $z_0 \in \mathbb C_+$, we have that there exists $C>0$ such that for $z \in \mathbb C_+$ close enough to $z_0$ and any
    $y>0$,
\begin{equation}\begin{split}
    &\Big| \frac{e^{-zy} - e^{-z_0 y}+(z-z_0) y}{z-z_0} \Big|
    = \frac{1}{|z-z_0|}\Big| \int_0^1 \big(-y e^{-(\theta z+(1-\theta)z_0)y}+y\big)(z-z_0)d\theta\Big|
    \\ &\leq y\int_0^1 |1-e^{-(\theta z +(1-\theta)z_0)y}| d\theta
    \leq (2y) \wedge\Big( y^2\int_0^1|\theta z+(1-\theta)z_0|d\theta\Big)
    \leq C(y\wedge y^2).
\end{split}\end{equation}
    Using this and the dominated convergence theorem, we have
\begin{equation}\begin{split}
    &\frac{h(z)-h(z_0)}{z-z_0} = \int_{(0,\infty)} \frac{e^{-zy}+zy -(e^{-z_0 y}+z_0 y)}{z-z_0}  \pi(dy)
    \\&\xrightarrow[\mathbb C_+\ni z\to z_0]{} \int_{(0,\infty)}(1 - e^{-z_0 y} )y\pi(dy) = h'(z_0),
\end{split}\end{equation}
    which says that $h$ is continuous on $\mathbb C_+$ and holomorphic on $\mathbb C_+^0$.
\end{proof}

	For each $z\in \mathbb C\setminus (-\infty,0]$, define
$
	\log z := \log |z| + i \arg z
$
	where $\arg z \in (-\pi,\pi)$ is uniquely determined by
$
	z = |z|e^{i \arg z}.
$ 	
	For all $z\in \mathbb C\setminus (-\infty,0]$ and $\gamma \in \mathbb C$, define
$
	z^\gamma := e^{\gamma \log z}.
$
	Then it is known, see \cite[Theorem 6.1]{SteinShakarchi2003Complex} for example, $z\mapsto \log z$ is holomorphic on $\mathbb C\setminus (-\infty,0]$.
	Therefore, for each $\gamma \in \mathbb C$, $z\mapsto z^\gamma$ is holomorphic on $\mathbb C\setminus (-\infty,0]$.
(We use the convention tha  $0^\gamma := \mathbf 1_{\gamma = 0}$.)
    From this definition we can show that $(z_1z_0)^\gamma = z_1^\gamma z_0^\gamma$ provided $\arg (z_1z_0)=\arg z_1 + \arg(z_0)$.

 Recall that the Gamma function  $\Gamma$ is defined by
\begin{equation}
    \Gamma (x) := \int_0^\infty t^{x-1} e^{-t}dt,
    \quad x>0.
\end{equation}
	It is known, see \cite[Theorem 6.1.3]{SteinShakarchi2003Complex} and the remark following it for example, the function $\Gamma$ has an unique analytic extension on $\mathbb C\setminus\{0, -1,-2,\dots\}$ and that
\[
	\Gamma(z+1) = z \Gamma(z),\quad z\in \mathbb C\setminus\{0, -1,-2,\dots\}.
\]
	Using this recursively, one gets that
\begin{equation}\begin{split}
\label{eq: definition of Gamma function}
    \Gamma(x)
    := \int_0^\infty t^{x-1} \Big(e^{-t} - \sum_{k=0}^{n-1} \frac{(-t)^k}{k!}\Big) dt,
    \quad -n< x< -n+1, n\in \mathbb N.
\end{split}\end{equation}

    Fix a $\beta \in (0,1)$.
    Using the uniqueness of holomorphic extension and Lemma \ref{lem: extension lemma for branching mechanism}, we get that
\begin{equation}
    z^{\beta}
	= \int_0^\infty (e^{-zy}-1) \frac{dy}{\Gamma(-\beta)y^{1+\beta}},
    \quad z\in \mathbb C_+,
\end{equation}
	by showing that the both sides
\begin{itemize}
\item
    are extension of the real function $x\mapsto x^{\beta}$ defined on $[0,\infty)$;
\item
    are holomorphic on $\mathbb C_+^0$;
\item
    are continuous on $\mathbb C_+$.
\end{itemize}
    Similarly, we get that
\begin{equation}
\label{eq: stable branching on C+}
    z^{1+\beta}
    = \int_0^\infty (e^{-zy}-1+zy)\frac{dy}{\Gamma(-1-\beta)y^{2+\beta}},
    \quad z\in \mathbb C_+.
\end{equation}
    Lemma \ref{lem: extension lemma for branching mechanism} also says that the derivative of $z^{1+\beta}$ is $(1+\beta)z^{\beta}$ on $\mathbb C^0_+$.
\begin{lem}
\label{lem: Lip of power function}
    For each $z_0,z_1 \in \mathbb C_+$, we have
\begin{equation}
\label{eq: Lip of power function}
    |z_0^{1+\beta} - z_1^{1+\beta}|
    \leq (1+\beta)(|z_0|^{\beta}+|z_1|^{\beta})|z_0 - z_1|.
\end{equation}

\end{lem}
\begin{proof}
    Since $z^{1+\beta}$ is continuous on $\mathbb C_+$, we only need to prove the lemma assuming $z_0,z_1 \in \mathbb C^0_+$.
    Notice that
\begin{equation}\begin{split}
\label{eq: upper bound for beta power of z}
	|z^\beta|
	= |e^{\beta \log |z| +i\beta \operatorname {arg}z}| = e^{\beta \log |z|} = |z|^\beta,
	\quad z \in \mathbb C\setminus (-\infty, 0].
\end{split}\end{equation}
    Define a path $\gamma: [0,1] \to \mathbb C^0_+$ such that
\[
    \gamma(\theta)
    = z_0 (1-\theta) + \theta z_1,
    \quad \theta \in [0,1].
\]
    Then, we have
\begin{equation}\begin{split}
    |z_0^{1+\beta} - z_1^{1+\beta}|
    &\leq (1+\beta) \int_0^1 |\gamma(\theta)^{\beta}|\cdot |\gamma'(\theta)|d\theta
    \leq (1+\beta)  \sup_{\theta \in [0,1]} |\gamma(\theta)|^{\beta} \cdot |z_1-z_0|
    \\&\leq (1+\beta)  ( |z_1|^{\beta}+|z_0|^{\beta} ) |z_1-z_0|.
    \qedhere
\end{split}\end{equation}
\end{proof}

	Suppose that $\varphi(\theta)$ is a continuous function from $\mathbb R$ into $\mathbb C$ such that $\varphi(0) = 1$ and $\varphi(\theta) \neq 0$ for all $\theta \in \mathbb R$.
	Then according to \cite[Lemma 7.6]{Sato1999Levy}, there is a unique continuous function $f(\theta)$ from $\mathbb R$ into $\mathbb C$ such that $f(0) = 0$ and $e^{f(\theta)} = \varphi(\theta)$.
	Such a function $f$ is called the distinguished logarithm of the function $\varphi$ and is denoted as $\operatorname{Log} \varphi(\theta)$.
	In particular, let $\varphi$ be the characteristic function of an infinitely divisible random variable $Y$, then $\operatorname{Log} \varphi(\theta)$ is called the L\'evy exponent of $Y$.
 	This distinguished logarithm should not be confused with the $\log$ function defined on $\mathbb C\setminus (-\infty, 0]$.

\subsection{Feynman-Kac formula with complex values}
\label{seq: complex Feynman-Kac transform}
    In this subsection we give a version of the Feynman-Kac formula with complex values.
    Suppose that $\{(\xi_t)_{t \in [r,\infty)}; (\Pi_{r,x})_{r\in [0,\infty), x\in E}\}$ is a (possibly non-homogeneous) Hunt process in a locally compact seperable metric space $E$.
    Write
\begin{equation}
    H^{(h)}_{(s,t)}
    := \exp\Big\{\int_s^t h(u,\xi_u) du\Big\},
    \quad 0 \leq s \leq t, h \in \mathcal B_b([0,t] \times E,\mathbb C).
\end{equation}

\begin{lem}
    Let $t \geq 0$. Suppose that $\alpha,\rho\in \mathcal B_b([0,t] \times E, \mathbb C)$ and $f\in \mathcal B_b(E, \mathbb C)$.
    Then
\begin{equation}
    g(r,x) := \Pi_{r,x}[ H_{(r,t)}^{(\alpha+\rho)} f(\xi_t)],\quad r \in [0,t], x\in E,
\end{equation}
    is the unique solution to the equation
\[
    g(r,x)= \Pi_{r,x} [ H_{(r,t)}^{(\alpha)} f(\xi_t)]+\Pi_{r,x} \Big[ \int_r^tH_{(r,s)}^{(\alpha)}\rho(s,\xi_s) g(s,\xi_s)~ds \Big],\quad r \in [0,t], x\in E.
\]
\end{lem}

\begin{proof}
    The proof is similar to the proof of \cite[Lemma A.1.5]{Dynkin1993Superprocesses}. We include it here for the sake of completeness.
    Notice that
\begin{equation}\begin{split}
    \Pi_{r,x} \Big[ \int_r^t | H_{(r,t)}^{(\alpha)}\rho(s,\xi_s) H_{(s,t)}^{(\rho)} f(\xi_t)| ~ds \Big]
    \leq  \int_r^t e^{(t-r)\|\alpha\|_\infty}e^{(t-s)\|\rho\|_\infty}\|\rho\|_\infty\|f\|_\infty ~ds
    < \infty.
\end{split}\end{equation}
    Also notice that
\begin{equation}
\label{eq: crucial for Feynman-Kac}
    \frac{\partial}{\partial s} H^{(\rho)}_{(s,t)}= -H^{(\rho)}_{(s,t)}\rho(s,\xi_s),
    \quad s\in (0,t).
\end{equation}
    Therefore, from the Markov property of $\xi$ and Fubini's theorem we get that
\begin{equation}\begin{split}
    &\Pi_{r,x} \Big[ \int_r^tH_{(r,s)}^{(\alpha)}~(\rho g)(s,\xi_s)~ds \Big]
    =\Pi_{r,x} \Big[ \int_r^t H_{(r,s)}^{(\alpha)}\rho(s,\xi_s) \Pi_{s,\xi_s}[ H_{(s,t)}^{(\alpha+\rho)} f(\xi_t)]~ds \Big]
    \\&= \Pi_{r,x} \Big[ \int_r^t H_{(r,t)}^{(\alpha)}\rho(s,\xi_s) H_{(s,t)}^{(\rho)} f(\xi_t) ~ds \Big]
    = \Pi_{r,x} [ H_{(r,t)}^{(\alpha)}f(\xi_t)(H_{(r,t)}^{(\rho)} - 1)]
    \\&= g(r,x) - \Pi_{r,x} [ H_{(r,t)}^{(\alpha)} f(\xi_t)].
\end{split}\end{equation}
    For the uniqueness, suppose that $\tilde g$ is another solution. Writting $h(r) = \sup_{x\in E}|g(r,x) - \tilde g(r,x)|$, we have
\[
    h(r) \leq e^{t\|\alpha\|_\infty}\|\rho\|_\infty \int_r^t h(s)ds,
    \quad r\le t.
\]
    According to \cite[Lemma A.1.5]{Dynkin1993Superprocesses}, this says that $h(r) =  0$ for $r\in [0,t]$.
\end{proof}


\subsection{Superprocesses}
\label{sec: definition of superprocess}
    In this subsection, we will give the definition of the superprocesses considered in this Appendix.
Let $E$ be locally compact separable metric space. Denote by $\mathcal M(E)$ the collection of all the finite measures on $E$ equipped with weak topology.
    Recall that $X=\{(X_t)_{t\geq 0}; (\mathbf P_\mu)_{\mu \in \mathcal M(E)}\}$ is said to be a $(\xi,\psi)$-superprocess if
\begin{itemize}
\item
    The spatial motion $\xi=\{(\xi_t)_{t\geq 0};(\Pi_x)_{x\in E}\}$ is an $E$-valued Hunt process with its lifetime denoted by $\zeta$.
\item
    The branching mechanism $\psi: E\times[0,\infty) \to \mathbb R$ is given by
\begin{equation}
\label{eq: branching mechanism}
    \psi(x,z)=
    - \alpha(x) z + \rho (x) z^2 + \int_{(0,\infty)} (e^{-zy} - 1 + zy) \pi(x,dy).
\end{equation}
    where $\alpha \in \mathcal B_b(E)$, $\rho \in \mathcal B_b(E, \mathbb R_+)$ and $\pi(x,dy)$ is a kernel from $E$ to $(0,\infty)$ such that $\sup_{x\in E} \int_{(0,\infty)} (y\wedge y^2) \pi(x,dy) < \infty$.
\item
    $X=\{(X_t)_{t\geq 0}; (\mathbf P_\mu)_{\mu \in \mathcal M(E)}\}$ is an $\mathcal M(E)$-valued Hunt process with transition probability determined by
\begin{equation}\begin{split}
    \mathbf P_\mu [e^{-X_t(f)}] = e^{-\mu(V_tf)},
    \quad t\geq 0, \mu \in \mathcal M(E), f\in \mathcal B^+_b(E),
\end{split}\end{equation}
    where for each $f\in \mathcal B_b(E)$, the function $(t,x)\mapsto V_tf(x)$ on $[0,\infty) \times E$ is the unique locally bounded positive solution to the equation
\begin{equation}\begin{split}\label{eq:FKPP_in_definition}
    V_tf(x) + \Pi_x \Big[  \int_0^{t\wedge \zeta} \psi(\xi_s,V_{t-s}f)ds \Big]
    = \Pi_x [ f(\xi_t)\mathbf 1_{t<\zeta} ],
    \quad t \geq 0, x \in E.
\end{split}\end{equation}
\end{itemize}
    We refer our reader to \cite{Li2011Measure-valued} for more discussion about the definition and the existence of superprocesses.
    To avoid triviality, we assume that $\psi(x,z)\neq -\alpha(x)z$ for some $x \in E$ and $z \geq 0$.

    Notice that, the branching mechanism $\psi$ can be extended into a map from $E \times \mathbb C_+$ to $\mathbb C$ using \eqref{eq: branching mechanism}.
    Define
\begin{equation}
    \psi'(x,z):= - \alpha(x) + 2\rho(x) z + \int_{(0,\infty)} (1-e^{-zy})y\pi(x,dy),
    \quad x\in E, z\in \mathbb C_+.
\end{equation}
    Then according to Lemma \ref{lem: extension lemma for branching mechanism}, for each $x \in E$, $z \mapsto \psi(x,z)$ is a holomorphic function on $\mathbb C_+^0$ with deriavetive $z \mapsto \psi'(x,z)$.
    Define $\psi_0(x,z) := \psi(x,z)+ \alpha(x)z $ and $\psi'_0(x,z) := \psi'(x,z) + \alpha(x)$.

    Denote by $\mathbb W$ the space of $\mathcal M(E)$-valued c\`{a}dl\`{a}g paths with its conanical path denoted by $(W_t)_{t\geq 0}$.
    We say $X$ is \emph{non-persistent} if $\mathbf P_{\delta_x}(\|X_t\|= 0) > 0$ for all $x\in E$ and $t> 0$.
    Suppose that $(X_t)_{t\geq 0}$ is non-persistent, then according to \cite[Section 8.4]{Li2011Measure-valued},
    there is a unique family of measures $(\mathbb N_x)_{x\in E}$ on $\mathbb W$ such that
\begin{itemize}
\item
    $\mathbb N_x (\forall t \geq 0, \|W_t\|=0) =0$;
\item
    $\mathbb N_x(\|W_0 \|\neq 0) = 0$;
\item
    For any $\mu \in \mathcal M(E)$, if $\mathcal N$ is a Poisson random measure defined on some probability space
    with intensity $\mathbb N_\mu(\cdot):= \int_E \mathbb N_x(\cdot )\mu(dx)$,
    then the superprocess $\{X;\mathbf P_\mu\}$ can be realized by $\widetilde X_0 := \mu$ and $\widetilde X_t(\cdot) := \mathcal N[W_t(\cdot)]$ for each $t>0$.
\end{itemize}
    We refer to $(\mathbb N_x)_{x\in E}$ as the \emph{Kuznetsov measures} of $X$.
\subsection{{Semigroups for superprocesses}}
\label{sec: definition of vf}
    Let $X$ be the non-presistent superprocess defined in Subsection \ref{sec: definition of superprocess} with its Kuznestov measure denoted by $(\mathbb N_x)_{x\in E}$.
    Define the mean semigroup
\begin{equation}
       P_t^\alpha f(x)
    := \Pi_{x}[e^{\int_0^t \alpha(\xi_s)ds}f(\xi_t) \mathbf 1_{t< \zeta}],
    \quad t\geq 0, x\in E, f\in \mathcal B_b(E,\mathbb R_+).
\end{equation}
    It is known from \cite[Proposition 2.27]{Li2011Measure-valued} and \cite[Theorem 2.7]{Kyprianou2014Fluctuations} that for all $t > 0$, $\mu \in \mathcal M(E)$ and $f\in \mathcal B_b(E,\mathbb R_+)$,
\begin{equation}
\label{eq: mean formula for superprocesses}
    \mathbb N_{\mu}[W_t(f)]
    =\mathbf P_{\mu}[X_t(f)]=
       \mu(P^\alpha_t f).
\end{equation}

    Define
\begin{equation}\begin{split}
    L_1(\xi)
    &:= \{f\in \mathcal B(E): \forall x\in E, t\geq 0, \quad \Pi_x[|f(\xi_t)|]< \infty\},
    \\L_2(\xi)
    &:= \{f \in \mathcal B(E): |f|^2 \in L_1(\xi)\}.
\end{split}\end{equation}
    Using monotonicity and linearity, we get from \eqref{eq: mean formula for superprocesses}  that
\begin{equation}
    \mathbb N_x[W_t(f)]
    =\mathbf P_{\delta_x}[X_t(f)]=P^\alpha_t f(x) \in \mathbb R,
    \quad f\in L_1(\xi), t > 0,x\in E.
\end{equation}
    This says that random variables $X_t(f)$ are well defined under probability $\mathbf P_{\delta_x}$ provided $f\in L_1(\xi)$.
    According to the branching property of the superprocess, they are infinitely divisible random variables.
    Therefore, we can write
\[
    U_t(\theta f)(x) := \operatorname{Log} \mathbf P_{\delta_x}[e^{i \theta X_t(f)}],
    \quad t\geq 0, f\in L_1(\xi), \theta \in \mathbb R, x\in E,
\]
    as their characteristic exponent.
    According to Campbell's formula, see \cite[Theorem 2.7]{Kyprianou2014Fluctuations} for example, we have
\[
    \mathbf P_{\delta_x} [e^{i\theta X_t(f)}]
    = \exp(\mathbb N_x[ e^{i\theta W_t(f)} - 1]),
    \quad t>0, f\in L_1(\xi), \theta \in \mathbb R, x\in E.
\]
    Noticing that $\theta \mapsto \mathbb N_x[e^{i\theta W_t(f)} - 1]$ is a continuous function on $\mathbb R$ and that $\mathbb N_x[e^{i\theta W_t(f)} - 1] = 0$ if $\theta = 0$, according to \cite[Lemma 7.6]{Sato1999Levy}, we have
\begin{equation}
\label{eq: N and characteristic exponent}
    U_t(\theta f)(x) = \mathbb N_x[e^{i W_t(\theta f)} - 1],
    \quad t>0, f\in L_1(\xi), \theta \in \mathbb R, x\in E.
\end{equation}

\begin{lem}
    There exists constants $C\geq 0$ such that for each $f \in L_1(\xi),x\in E$ and $t\geq 0$, we have
\begin{equation}\begin{split}
\label{eq: upper bound of psi(v)}
    \big|\psi\big(x,-U_tf\big)\big|
    \leq C P^\alpha_t |f|(x)+
         C (P^\alpha_t |f| (x))^2.
\end{split}\end{equation}
\end{lem}
\begin{proof}
     Noticing that
\[
     e^{\operatorname{Re} U_tf(x)}
    = |e^{U_tf(x)}|
    = |\mathbf P_{\delta_x}[e^{i X_t(f)}]|
    \leq 1,
\]
    we have
\begin{equation}
\label{eq: -v has positive real part}
 \operatorname{Re} U_tf(x)
    \leq 0.
\end{equation}
    Therefore, we can talk about $\psi(x,-U_tf)$ since $z\mapsto \psi(x,z)$ is well defined on $\mathbb C_+$.
    According to Lemma \ref{lem: estimate of exponential remaining}, we have that
\begin{equation}
\label{eq: upper bound for vf}
    |U_tf(x)| \leq \mathbb N_x[|e^{i W_t(f)} - 1|]
    \leq \mathbb N_x[|i W_t(f)|]
     \leq (P^\alpha_t |f|)(x).
\end{equation}
    Notice that, for any compact $K \subset \mathbb R$,
\begin{equation}
\label{eq: estimate of deriavetive of v(theta)}
    \mathbb N_x\Big[\sup_{\theta \in K} \Big|\frac{\partial}{\partial \theta} (e^{i\theta W_t(f)} - 1) \Big|\Big]
    \leq \mathbb N_x[|W_t(f)|] \sup_{\theta \in K}|\theta|
        \leq (P^\alpha_t |f|)(x) \sup_{\theta \in K}|\theta| < \infty.
\end{equation}
    Therefore, according to \cite[Theorem A.5.2.]{Durrett2010Probability} and \eqref{eq: N and characteristic exponent},
    $U_t(\theta f)(x)$ is differentiable in $\theta \in \mathbb R$ with
\[
    \frac{\partial}{\partial \theta} U_t(\theta f)(x)
    = i\mathbb N_x[W_t(f)e^{i\theta W_t(f)}],
    \quad \theta \in \mathbb R.
\]
    Moreover, from the above, it is clear that
\begin{equation}
\label{eq: upper bounded for derivative of v(theta)}
    \sup_{\theta \in \mathbb R}\Big| \frac{\partial}{\partial \theta}U_t(\theta f)(x)\Big|
         \leq ( P^\alpha_t |f|)(x).
\end{equation}
    From the dominate convergence theorem, we can verify that $(\partial/\partial \theta)U_tf(x)$ is continuous in $\theta$.
    In other words, $\theta \mapsto -U_t(\theta f)(x)$ is a $C^1$ map from $\mathbb R$ to $\mathbb C_+$.
    According to this, we have
\begin{equation}
\label{eq: path integration representation of psi(v)}
    \psi(x,-U_tf) = -\int_0^1 \psi'\big(x,-U_t(\theta f)\big) \frac{\partial}{\partial \theta} U_t(\theta f)(x)~d\theta.
\end{equation}
    Notice that
\begin{equation}
\label{eq: upper bound of psi'(v)}
\begin{split}
    &|\psi'(x, -U_tf)|
    \\&= \Big| -\alpha(x)- 2\rho(x) U_tf(x)+ \int_{(0,\infty)} y (1- e^{y U_tf(x)} ) \pi(x,dy)\Big|
    \\&= \Big| - \alpha(x)- 2\rho(x)\mathbb N_x[e^{i W_{t}(f)} - 1]  + \int_{(0,\infty)} y \mathbf P_{y \delta_x}[1-e^{i X_{t}(f)}] \pi(x,dy) \Big|
\\ &\leq \|\alpha\|_\infty + 2\rho(x)\mathbb N_x[W_t(|f|)]+ \int_{(0,\infty)} y\mathbf P_{y\delta_x}[2\wedge X_t(|f|)] \pi(x,dy)
\\ &\leq \|\alpha\|_\infty + 2\|\rho\|_\infty
  P^\alpha_t |f|(x) + \Big(\sup_{x\in E}\int_{(0,1]}y^2 \pi(x,dy)\Big)~P^\alpha_t |f|(x)
  + 2\sup_{x\in E}\int_{(1,\infty)} y \pi(x,dy)
 \\ &=: C_1 + C_2(P^\alpha_t |f|)(x),
\end{split}
\end{equation}
    where $C_1, C_2$ are constants independent on $f,x$ and $t$.
    Now, using \eqref{eq: path integration representation of psi(v)}, \eqref{eq: upper bounded for derivative of v(theta)} and \eqref{eq: upper bound of psi'(v)}, we have get the desired result.
\end{proof}



    This Lemma also says that if $f\in L^2(\xi)$ then the following expectations
\[
    \Pi_x\Big[\int_0^t \psi(\xi_s,- U_{t-s}f)ds\Big]
    \in \mathbb C,
    \quad x\in E, t\geq 0.
\]
    are well defined.
    In fact, using Jensen's inequality and the Markov property, we have
\begin{equation}\begin{split}
\label{eq: domination of psi(v)}
    &\Pi_x\Big[\int_0^t \big|\psi \big(\xi_s,-U_{t-s}f\big)\big|ds\Big]
    \\&\leq \Pi_x\Big[\int_0^t \big(C_1 P_{t-s}^\alpha|f|(\xi_s)+C_2 P_{t-s}^\alpha|f|(\xi_s)^2\big)ds\Big]
    \\ &\leq \int_0^t \big(C_1 e^{t\|\alpha\|}\Pi_x \big[ \Pi_{\xi_s}[|f(\xi_{t-s})|] \big]+C_2 e^{2t\|\alpha\|}\Pi_x \big[ \Pi_{\xi_s}[|f (\xi_{t-s})|]^2 \big]\big)~ds
    \\ &\leq \int_0^t (C_1 e^{t\|\alpha\|}\Pi_x [ |f(\xi_{t})|]+C_2e^{2t\|\alpha\|}\Pi_x [ |f (\xi_{t})|^2 ])~ds < \infty.
\end{split}\end{equation}

\subsection{}
    Let $X$ be the non-persistent superprocess discussed in Subsection \ref{sec: definition of vf}.
    In this subsection, we will prove the following:
    \begin{prop}
\label{prop: complex FKPP-equation}
    If $f\in L_2(\xi)$,  then for all $t\geq 0$ and $x\in E$,
\begin{equation}
\label{eq: complex FKPP-equation}
    U_tf(x) - \Pi_x \Big[\int_0^t \psi\big(\xi_s, - U_{t-s}f\big) ds \Big]
    \\= i \Pi_x [f(\xi_t)]
\end{equation}
and
\begin{equation}
\label{eq: complex FKPP-equation with FK-transform}
    U_tf(x) -  \int_0^t P_{t-s}^{\alpha} \psi_0\big(\cdot,-U_sf\big) (x)~ds
      \\= iP_t^\alpha f(x).
\end{equation}
\end{prop}

    To prove this, we will need the generalized spine decomposition theorem from \cite{RenSongSun2017Spine} which we now recall.
    Let $X$ be the non-persistent superprocess discussed in Subsection \ref{sec: definition of vf}.
    Let $f\in \mathcal B_b(E,\mathbb R_+)$, $T >0$ and $x\in E$.
    Suppose that $\mathbf P_{\delta_x}[X_T(f)] = \mathbb N_x[W_T(f)] = P^\alpha_T f(x) \in (0,\infty)$, then we can define the following probability transforms:
\begin{equation}
    d\mathbf P_{\delta_x}^{X_T(f)}
   := \frac{X_T(f)}{P_T^\alpha f(x)} d\mathbf P_{\delta_x};
    \quad d\mathbb N_x^{W_T(f)}
    :=  \frac{W_T(f)}{P_T^\alpha f(x)} d\mathbb N_x.
\end{equation}
    Following the definition in \cite{RenSongSun2017Spine}, we say that $\{\xi, \mathbf n;\mathbf Q_{x}^{(f,T)}\}$ is a spine representation of $\mathbb N_x^{W_T(f)}$ if
\begin{itemize}
\item
    The spine process $\{(\xi_t)_{0\leq t\leq T}; \mathbf Q^{(f,T)}_x\}$ is a copy of $\{(\xi_t)_{0\leq t\leq T}; \Pi^{(f,T)}_{x}\}$,
    where
\begin{equation}
    d\Pi_x^{(f,T)} := \frac{f(\xi_T)e^{\int_0^T \alpha(\xi_s)ds}}{P^\alpha_T f(x)} d \Pi_x;
\end{equation}
\item
    Given $\{(\xi_t)_{0\leq t\leq T}; \mathbf Q^{(f,T)}_x\}$, the immigration measure $\{\mathbf n(\xi,ds,dw); \mathbf Q^{(f,T)}_x[\cdot |(\xi_t)_{0\leq t\leq T}]\}$ is a Poisson random measure on $[0,T] \times \mathbb W$ with intensity
\begin{equation}
\label{eq: conditional intensity}
    \mathbf m(\xi,ds,dw)
    := 2 \rho(\xi_s) ds \cdot \mathbb N_{\xi_s}(dw) + ds \cdot \int_{y\in (0,\infty)} y \mathbf P_{y\delta_{\xi_s}}(X\in dw) \pi(\xi_s,dy);
\end{equation}
\item
    $\{(Y_t)_{0\leq t\leq T}; \mathbf Q^{(f,T)}_x\}$ is an $\mathcal M(E)$-valued process defined by
\begin{equation}\begin{split}
    Y_t
    := \int_{(0,t] \times \mathbb W} w_{t-s} \mathbf n(\xi,ds,dw),
    \quad 0 \leq t\leq T.
\end{split}\end{equation}
\end{itemize}
    According to the spine decomposition theorem in \cite{RenSongSun2017Spine}, we have that
\begin{equation}\begin{split}
\label{eq: Spine decomposition 1}
    \{(X_s)_{s \geq 0};\mathbf P_{\delta_x}^{X_T(f)}\}
    \overset{f.d.d.}{=} \{(X_s + W_s)_{s \geq 0};\mathbf P_{\delta_x} \otimes \mathbb N_x^{W_T(f)} \}
\end{split}\end{equation}
    and
\begin{equation}\begin{split}
\label{eq: Spine decomposition 2}
    \{(W_s)_{0\leq s\leq T};\mathbb N_x^{W_T(f)}\}
    \overset{f.d.d.}{=} \{(Y_s)_{s \geq 0};\mathbf Q_x^{(f,T)}\}.
\end{split}\end{equation}

\begin{proof}[Proof of Proposition \ref{prop: complex FKPP-equation}]
    Assume that $f\in \mathcal B_b(E)$.
    Fix $t>0, r\in [0,t), x\in E$ and $g\in \mathcal B_b(E)^{++}$.
    Denote by $\{\xi, \mathbf n; \mathbf Q_x^{(g,t)}\}$ the spine representation of $\mathbb N_x^{W_t(g)}$.
    Conditioned on $\{\xi; \mathbf Q_x^{(g,t)}\}$, denote by $\mathbf m(\xi, ds,dw)$ the conditional intensity of $\mathbf n$ in \eqref{eq: conditional intensity}.
    Denote by $\Pi_{r,x}$ the probability of Hunt process $\{\xi; \Pi\}$ initiated at time $r$ and position $x$.
    From Lemma \ref{lem: estimate of exponential remaining}, we have $\mathbf Q^{(g,t)}_{x}$-almost surely
\begin{equation}\begin{split}
&\int_{[0,t]\times \mathbb W}|e^{i w_{t-s}(f)} - 1| \mathbf m(\xi, ds,dw)
    \leq \int_{[0,t]\times \mathbb W}\big(| w_{t-s}(f)| \wedge 2\big) \mathbf m(\xi, ds,dw)
    \\&\leq \int_0^t \Big(2\rho(\xi_s)\mathbb N_{\xi_s}\big( W_{t-s}(|f|)\big)  + \int_{(0,1]} y \mathbf P_{y \delta_{\xi_s}}[ X_{t-s}(|f|)] \pi(\xi_s,dy)
    \\&\qquad\qquad+ 2\int_{(1,\infty)}y\pi(\xi_s,dy)\Big) ds
     \\&\leq \int_0^t (P_{t-s}^\alpha |f|)(\xi_s)\Big(2\rho(\xi_s)  + \int_{(0,1]} y^2 \pi(\xi_s,dy)\Big) ds + 2t \sup_{x\in E}\int_{(1,\infty)}y\pi(x,dy)
    \\&\leq \Big(2\|\rho\|_\infty +\sup_{x\in E}\int_{(0,1]} y^2 \pi(x,dy)\Big) t e^{t\|\alpha\|_\infty}\|f\|_\infty + 2t \sup_{x\in E}\int_{(1,\infty)}y\pi(x,dy)
    < \infty.
\end{split}\end{equation}
    Using this, Fubini's theorem, \eqref{eq: N and characteristic exponent} and \eqref{eq: -v has positive real part} we have $\mathbf Q^{(g,t)}_{x}$-almost surely,
\begin{equation}\begin{split}
    &\int_{[0,t]\times \mathbb N}(e^{i w_{t-s}(f)} - 1) \mathbf m(\xi, ds,dw)
    \\&=\int_0^t \Big(2\rho(\xi_s)\mathbb N_{\xi_s}(e^{i W_{t-s}(f)} - 1)  + \int_{(0,\infty)} y \mathbf P_{y \delta_{\xi_s}}[e^{i X_{t-s}(f)} - 1] \pi(\xi_s,dy)\Big) ds
    \\&=\int_0^t \Big( 2\rho(\xi_s) U_{t-s} f(\xi_s) + \int_{(0,\infty)} y (e^{y U_{t-s}f(\xi_s)} - 1) \pi(\xi_s,dy) \Big) ds
    \\&= -\int_0^t \psi'_0 \big(\xi_s, -U_{t-s}f\big)ds.
\end{split}\end{equation}
    Therefore, according to \eqref{eq: Spine decomposition 2}, Campbell's formula and above, we have that
\begin{equation}\begin{split}
\label{eq: N to Pi}
    \mathbb N_x^{W_t(g)}[e^{i W_t(f)}]
    &=\mathbf Q_x^{(g,t)} \Big[\exp\Big\{\int_{[0,t]\times \mathbb N}(e^{i w_{t-s}(f)} - 1) \mathbf m(\xi, ds,dw)\Big\}\Big]
    \\&= \Pi_x^{(g,t)} [e^{-\int_0^t \psi'_0(\xi_s, -U_{t-s}f)ds}]
    \\&= \frac{1}{P_t^\alpha g (x)} \Pi_x[ g(\xi_t) e^{-\int_0^t \psi'(\xi_s, -U_{t-s}f)ds} ].
\end{split}\end{equation}
    Let $\epsilon >0$.
    Define $f^+ = (f \vee 0) + \epsilon$ and $f^- = (-f) \vee 0 + \epsilon$, then $f^\pm \in \mathcal B_b(E)^{++}$ and $f = f^+ - f^-$.
    According to \eqref{eq: Spine decomposition 1}, we have that
\begin{equation}
    \frac{\mathbf P_{\delta_x}[X_t(f^{\pm})e^{i X_t(f)}]}{\mathbf P_{\delta_x}[X_t(f^{\pm})]}
    = \mathbf P_{\delta_x}[e^{i X_t(f)}] \mathbb N_x^{W_t(f^{\pm})}[e^{i X_t(f)}].
\end{equation}
    Using \eqref{eq: N to Pi} and the above, we have
\begin{equation}\begin{split}
    \frac{\mathbf P_{\delta_x}[X_t(f)e^{i X_t(f)}] }{\mathbf P_{\delta_x}[e^{i X_t(f)}]}
    &= \mathbf P_{\delta_x}[X_t(f^+)] \mathbb N_x^{W_t(f^+)} [e^{i X_t(f)}] - \mathbf P_{\delta_x}[X_t(f^-)]\mathbb N_x^{W_t(f^-)}[e^{i X_t(f)}]
    \\& = \Pi_x[ f(\xi_t) e^{- \int_0^t \psi'(\xi_s, -U_{t-s}f) ds}  ].
\end{split}\end{equation}
    Therefore, we have
\begin{equation}\begin{split}
    \frac{\partial}{\partial \theta} {U_t(\theta f)(x)}
    = \frac{\mathbf P_{\delta_x}[iX_t(f)e^{i X_t(f)}] }{\mathbf P_{\delta_x}[e^{i X_t(f)}]}
    =  \Pi_x[ if(\xi_t) e^{ - \int_0^t \psi'(\xi_s, -U_{t-s}(\theta f)) ds} ].
\end{split}\end{equation}
    Since $\{(\xi_{r+t})_{t \geq 0}; \Pi_{r,x}\} \overset{d}{=} \{(\xi_{t})_{t\geq 0}; \Pi_{x}\} $, we have

\begin{equation}\begin{split}
    &\frac{\partial}{\partial \theta} U_{t-r}(\theta f)( x)
    = \Pi_x[ i f(\xi_{t-r}) e^{-\int_0^{t-r} \psi'(\xi_s, -U_{t-r-s}(\theta f)) ds} ]
    \\&= \Pi_{r,x}[i f(\xi_t)e^{-\int_0^{t-r} \psi'(\xi_{r+s}, -U_{t-r-s}(\theta f)) ds} ]
    = \Pi_{r,x}[if(\xi_t)e^{-\int_r^t \psi'(\xi_{s}, -U_{t-s}(\theta f)) ds} ].
\end{split}\end{equation}

    From \eqref{eq: upper bound of psi'(v)}, we know that for each $\theta\in \mathbb R$, $(t,x) \mapsto |\psi'(x,-U_tf(x))|$ is locally bounded (i.e. bounded on $[0,T]\times E$ for each $T \geq 0$).
    Therefore, we can apply the argument in Subsection \ref{seq: complex Feynman-Kac transform} and get that
\[
    \frac{\partial}{\partial \theta} U_{t-r}(\theta f)(x) + \Pi_{r,x} \Big[\int_r^t \psi'\big(\xi_s,- U_{t-s}(\theta f)\big)\frac{\partial}{\partial \theta} U_{t-s}(\theta f)(\xi_s)~ds\Big]
    = \Pi_{r,x} [i f(\xi_t)]
\]
    and
\begin{equation}\begin{split}
    &\frac{\partial}{\partial \theta} U_{t-r}(\theta f)(x) + \Pi_{r,x} \Big[\int_r^t e^{\int_r^s \alpha(\xi_u)du}\psi_0'\big(\xi_s,- U_{t-s}(\theta f)\big)\frac{\partial}{\partial \theta} U_{t-s}(\theta f)(\xi_s)~ds\Big]\\
    &= \Pi_{r,x} [i e^{\int_r^t \alpha(\xi_s)ds}f(\xi_t)].
\end{split}\end{equation}
    Integrating the two displays above with respect to $\theta$  on [0,1], using \eqref{eq: path integration representation of psi(v)}, \eqref{eq: upper bound of psi'(v)}, \eqref{eq: upper bounded for derivative of v(theta)} and Fubini's theorem, we get
\begin{equation}
    U_{t-r}f(x) - \Pi_{r,x} \Big[\int_r^t \psi\big(\xi_s,-U_{t-s}f\big) ~ds\Big]
    = i\theta \Pi_{r,x} [f(\xi_t)]
\end{equation}
    and
\begin{equation}
    U_{t-r}f(x) - \Pi_{r,x} \Big[\int_r^t e^{\int_r^s \alpha(\xi_u)du} \psi_0\big(\xi_s,- U_{t-s}f\big) ~ds\Big]
    = i\Pi_{r,x} [e^{\int_r^t\alpha(\xi_u)du}f(\xi_t)].
\end{equation}
    Taking $r = 0$, we get that \eqref{eq: complex FKPP-equation} and \eqref{eq: complex FKPP-equation with FK-transform} are true if $f\in \mathcal B_b(E)$.

    The rest of the proof is to evaluate \eqref{eq: complex FKPP-equation} and \eqref{eq: complex FKPP-equation with FK-transform} for all $f\in L_2(\xi)$. We only do this for \eqref{eq: complex FKPP-equation} since the argument for \eqref{eq: complex FKPP-equation with FK-transform} is similar.
    Let $n \in \mathbb N$.
    Writing $f_n := (f^+ \wedge n) - (f^- \wedge n)$, then $f_n \xrightarrow[n\to \infty]{} f$ pointwise.
    From what we have proved, we have
\begin{equation}
\label{eq: complex FKPP-equation for fn}
    U_tf_n(x) - \Pi_{x} \Big[\int_0^t \psi\big(\xi_s, - U_{t-s}f_n\big) ~ds\Big]
    = i \Pi_{x} [f_n(\xi_t)].
\end{equation}
    Notice the following:
\begin{itemize}
\item
    It is clear that $\Pi_{x}[f_n(\xi_t)] \xrightarrow[n\to \infty]{} \Pi_{x}[f(\xi_t)]$.
\item
     $U_tf_n(x) \xrightarrow[n\to \infty]{} U_tf(x)$ due to \eqref{eq: N and characteristic exponent}, the dominated convergence theorem and the fact that
\[
    |e^{i W_t(f_n)} - 1| \leq W_t(|f|);
    \quad \mathbb N_x[W_t(|f|)] = (P_t^\alpha |f|)(x) < \infty.
\]
\item
     $\Pi_{x} [\int_0^t \psi(\xi_s,- U_{t-s}f_n)ds] \xrightarrow[n\to \infty]{} \Pi_{x} [\int_0^t \psi(\xi_s,- U_{t-s}f)ds]$ due to the dominated convergence theorem, \eqref{eq: domination of psi(v)} and the fact (see \eqref{eq: upper bound of psi(v)}) that
\begin{equation}\begin{split}
    \big|\psi(\xi_s,- U_{t-s}f_n)\big|
    \leq C_1 P_{t-s}^\alpha|f|(\xi_s)+C_2 P_{t-s}^\alpha|f|(\xi_s)^2.
\end{split}\end{equation}
\end{itemize}
    Using the above arguments, letting $n \to \infty$ in \eqref{eq: complex FKPP-equation for fn}, we get the desired result.
\end{proof}

\begin{thebibliography} {10}

\bibitem{ARMP}
Adamczak, RR. Milo\'s, P.:
\emph{CLT for Ornstein-Uhlenbeck branching particle system.}
Electron. J. Probab. 20 (2015), no.42, 1-35.

\bibitem{ASHH}
Asmussen, S. Hering, H.:
\emph{Branching Processes.}
Birkh auser, Boston, 1983.

\bibitem{BAM}
Berestycki, J., Kyrianou, A.E., Murillo-Salas, A
\newblock{\em The prolific backbone for supercritical superprocesses}. Stoch. Proc. Appl. 121, 1315-1331(2011)

\bibitem{Cuppens1975Decomposition}
Cuppens, R.:
\emph{Decomposition of multivariate probabilities.}
Probability and Mathematical Statistics, Vol. 29. Academic Press [Harcourt Brace Jovanovich, Publishers], New York-London, 1975.
\MR{0517412}

\bibitem{Durrett2010Probability}
Durrett, R.:
\emph{Probability: theory and examples.}
Fourth edition. Cambridge Series in Statistical and Probabilistic Mathematics, 31. Cambridge University Press, Cambridge, 2010.
\MR{2722836}

\bibitem{Dynkin1993Superprocesses}
Dynkin,~E.B.
\newblock {\em Superprocesses and partial differential equations}, Ann. Probab (1993): 1185-1262.

\bibitem{DK}
Dynkin, E. B., Kuznetsov, S. E.:
\newblock {\em $\mathbb{N}$-measure  for branching exit Markov system and their applications to differential equations}, Probab. Theory Related Fields 130(2004) 135-150

\bibitem{Kyprianou2014Fluctuations}
Kyprianou, A. E.:
\emph{Fluctuations of L\'{e}vy processes with applications.}
Introductory lectures. Second edition. Universitext. Springer, Heidelberg, 2014.
\MR{3155252}

\bibitem{KBA1}
Kyprianou, A. E.:
\emph{Limit theorems for multitype continuous time Markov branching processes I: the case of an eigenvector linear functional.}
Z. Wahrs. Verw. Gebiete 12 (1969), 320-332.

\bibitem{KBA2}
Kyprianou, A. E.:
\emph{Limit theorems for multitype continuous time Markov branching processes II: the case of an arbitrary linear functional.}
Z. Wahrs. Verw. Gebiete 12 (1969), 204-214.

\bibitem{KBA3}
Kyprianou, A. E.:
\emph{Some redinements in the theory of supercritical multitype Markov branching processes.}
Z. Wahrs. Verw. Gebiete 12 (1971), 47-57.

\bibitem{HKBPS1}
Kesten, H. Stigum, B. P.:
\emph{A limit theorem for multidimensional Galton-Watson processes.}
Ann. Math. Statist. 37 (1966), 1221-1223.

\bibitem{HKBPS2}
Kesten, H. Stigum, B. P.:
\emph{Additional limit theorems for indecomposable multidimensional Galton-Watson processes.}
Ann. Math. Statist. 37 (1966), 1463-1481.

\bibitem{Li2011Measure-valued}
Li, Z.:
\emph{Measure-valued branching Markov processes.}
Probability and its Applications (New York), Springer, Heidelberg, 2011.
\MR{2060602}

\bibitem{Linde1986Probability}
Linde, W.:
\emph{Probability in Banach spaces-stable and infinitely divisible distributions.}
Second edition. A Wiley-Interscience Publication. John Wiley \& Sons, Ltd., Chichester, 1986.

\bibitem{LiuRenSong2009Llog}
Liu, R., Ren, Y.-X., Song, R.:
\emph{{$L log L$} criterion for a class of Superdiffusions.}
J. Appl. Probab. \textbf{46} (2009), no. 2, 479-496.

\bibitem{MP2012}
Milo\'s, P.:
\emph{Spatial CLT for the supercritical Ornstein-Uhlenbeck superprocess.}
Preprint, 2012. arXiv:1203:6661

\bibitem{MM}
Marks,R., Milo\'s, P.:
\emph{CLT for supercritical branching processes with heavy-tailed branching law.}
\ARXIV{183.05491}

\bibitem{GD}
Metafune,~G., Pallara, D.:
\emph{Specturm of Ornstein-Uhlenbeck operators in $\mathcal{L}^p$ space with respect to invariant measures.}
J. Funct. Anal. 196,40-60(2002)

\bibitem{RSZ}
Ren, Y.-X., Song, R., Zhang, R.:
\emph{Central limit theorems for super Ornstein-Uhlenbeck processes.}
Acta Appl. Math. 130(2014)9-49.

\bibitem{RSZ1}
Ren, Y.-X., Song, R., Zhang, R.:
\emph{Central limit theorems for supercritical branching Markov processes.}
J. Funct. Anal. 266(2014): 1716-1756.

\bibitem{RSZ2}
Ren, Y.-X., Song, R., Zhang, R.:
\emph{Central limit theorems for supercritical superprocesses.}
Stochastic Processes and Their Applications, 125(2015) 428-457.

\bibitem{RSZ3}
Ren, Y.-X., Song, R., Zhang, R.:
\emph{Central limit theorems for supercritical branching nonsymmetric Markov processes.}
Ann. Probab., 45(1)(2017): 564-623.

\bibitem{RSZ4}
Ren, Y.-X., Song, R., Zhang, R.:
\emph{Functional central limit theorems for supercritical superprocesses.}
Acta Applicandae Mathematicae 147(1)(2017): 137-175.

\bibitem{RenSongSun2017Spine}
Ren, Y.-X., Song, R., Sun, Z.:
\emph{ Spine decompositions and limit theorems for a class of critical superpeocesses.}
arXiv preprint arXiv: 1711. 09188(2017).

\bibitem{Rudin1987Real}
Rudin, W.:
\emph{Real and complex analysis.}
Third edition. McGraw-Hill Book Co., New York, 1987.
\MR{0924157}

\bibitem{SchillingSongVondracek2010Bernstein}
Schilling, R., Song, R., Vondra\v{c}ek, Z.:
\emph{Bernstein functions. Theory and applications.}
De Gruyter Studies in Mathematics, 37. Walter de Gruyter \& Co., Berlin, 2010.
\MR{2598208}

\bibitem{Sato1999Levy}
Sato, K.:
\emph{L\'evy proesses and infinitely divisible distributions.}
Cambridge Studies in Advanced Mathematics, 68. Cambridge University Press, Cambridge, 1999.

\bibitem{SteinShakarchi2003Complex}
Stein, E. M and Shakarchi, Rami.:
\emph{Complex analysis.}
Princeton Lectures in Analysis, 2. Princeton University Press, Princeton, NJ, 2003.

\end{thebibliography}

\end{document}
