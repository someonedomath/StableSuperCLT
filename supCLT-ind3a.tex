\documentclass[12pt,a4paper]{amsart}
\setlength{\textwidth}{\paperwidth}
\addtolength{\textwidth}{-2in}
\calclayout
\usepackage[utf8]{inputenc}
\usepackage[T1]{fontenc}
\usepackage{mathtools}
\mathtoolsset{showonlyrefs}
\usepackage{stackrel}
\usepackage{mathrsfs}
\usepackage{hyperref}
\usepackage{comment}
\usepackage{amsthm}
\theoremstyle{plain}
\newtheorem{thm}{Theorem}[section]
\newtheorem{lem}[thm]{Lemma}
\newtheorem{prop}[thm]{Proposition}
\newtheorem{cor}[thm]{Corollary}
\newtheorem{conj}[thm]{Conjecture}
\theoremstyle{definition}
\newtheorem{defi}[thm]{Definition}
\newtheorem{rem}[thm]{Remark}
\newtheorem{exa}[thm]{Example}
\newtheorem{asp}{Assumption}
\numberwithin{equation}{section}
\allowdisplaybreaks
\begin{document}
\title
[stable CLT for super-OU processes]
{Stable Central Limit Theorems for Super Ornstein-Uhlenbeck Processes, II}
\author
[Y.-X. Ren, R. Song, Z. Sun and J. Zhao]
{Yan-Xia Ren, Renming Song, Zhenyao Sun and Jianjie Zhao}
\address{
  Yan-Xia Ren \\
  LMAM School of Mathematical Sciences \& Center for Statistical Science \\
  Peking University \\
  Beijing, P. R. China, 100871}
\email{yxren@math.pku.edu.cn}
\thanks{The research of Yan-Xia Ren is supported in part by NSFC (Grant Nos. 11671017  and 11731009) and LMEQF.}
\address{
  Renming Song \\
  Department of Mathematics \\
  University of Illinois at Urbana-Champaign \\
  Urbana, IL, USA, 61801}
\email{rsong@illinois.edu}
\thanks{The Research of Renming Song is support in part by a grant from the Simons Foundation (\#429343, Renming Song)}
\address{
  Zhenyao Sun \\
  School of Mathematics and Statistics\\
  Wuhan University \\
  Hubei, P. R. China, 100871}
\email{zhenyao.sun@gmail.com}
\address{
  Jianjie Zhao \\
  School of Mathematical Sciences \\
  Peking University \\
  Beijing, P. R. China, 100871}
\email{zhaojianjie@pku.edu.cn}

\begin{abstract}
This paper is a continuation of our recent paper (Elect. J Probab., \textbf{24} (2019), no. 141)
and is devoted to the  asymptotic behavior of a supercritical super Ornstein-Uhlenbeck process $(X_t)_{t\geq 0}$
 whose branching mechanism $\psi$ satisfies some perturbation condition which guarantees that,
 when $z\to 0$, $\psi(z)=-\alpha z + \eta z^{1+\beta} (1+o(1))$ with $\alpha > 0$, $\eta>0$ and $\beta\in (0, 1)$.
In the aforementioned paper, we have proved $(1+\beta)$-stable central limit theorems
for  $X_t(f) $ for {\it some} functions $f$ of polynomial growth.
As a consequence of the main result of this paper, we will have
$(1+\beta)$-stable central limit theorems for  $X_t(f) $ for {\it all} functions $f$ of polynomial growth.
\end{abstract}
\subjclass[2010]{60J68, 60F05}
\keywords{Superprocesses, Ornstein-Uhlenbeck processes, Stable distribution, Central limit theorem, Law of large numbers, Branching rate regime}
\maketitle
\section{Introduction}
\subsection{Motivation}
\label{subsec:M}
Let $d \in \mathbb N:= \{1,2,\dots\}$ and $\mathbb R_+:= [0,\infty)$.
Let $\xi=\{(\xi_t)_{t\geq 0}; (\Pi_x)_{x\in \mathbb R^d}\}$ be an $\mathbb R^d$-valued Ornstein-Uhlenbeck process (OU process) with generator
\begin{align}
  Lf(x)
  = \frac{1}{2}\sigma^2\Delta f(x)-b x \cdot \nabla f(x)
  , \quad  x\in \mathbb R^d, f \in C^2(\mathbb R^d),
\end{align}
where $\sigma > 0$ and $b > 0$ are constants.
Let $\psi$ be a function on $\mathbb R_+$ of the form
\begin{align}
  \label{eq: honogeneou branching mechanism}
  \psi(z)
  =- \alpha z + \rho z^2 + \int_{(0,\infty)} (e^{-zy} - 1 + zy)~\pi(dy)
  , \quad  z \in \mathbb R_+,
\end{align}
where $\alpha > 0 $, $\rho \geq0$ and $\pi$ is a measure on $(0,\infty)$ with $\int_{(0,\infty)}(y\wedge y^2)~\pi(dy)< \infty$.
$\psi$ is referred to as a branching mechanism and $\pi$ is referred to as the L\'evy measure of $\psi$.
Denote by $\mathcal M(\mathbb R^d)$ the space of all finite Borel measures on $\mathbb R^d$.
For $f,g\in \mathcal B(\mathbb R^d, \mathbb R)$ and $\mu \in \mathcal M(\mathbb R^d)$,
 write $\mu(f)= \int f(x)\mu(dx)$
and $\langle f, g\rangle = \int f(x)g(x) dx$ whenever the integrals make sense.
We say a real-valued Borel function $f:(t,x)\mapsto f(t,x)$ on $\mathbb R_+\times \mathbb R^d$ is \emph{locally bounded} if, for each $t\in \mathbb R_+$, we have $ \sup_{s\in [0,t],x\in \mathbb R^d} |f(s,x)|<\infty. $
We say that an $\mathcal M(\mathbb R^d)$-valued Hunt process $X = \{(X_t)_{t\geq 0}; (\mathbb{P}_{\mu})_{\mu \in \mathcal M(\mathbb R^d)}\}$
on  $(\Omega, \mathscr{F})$
is a \emph{super Ornstein-Uhlenbeck process (super-OU process)} with branching mechanism $\psi$, or a $(\xi, \psi)$-superprocess, if for each non-negative bounded Borel function $f$ on $\mathbb R^d$, we have
\begin{align}
  \label{eq: def of V_t}
  \mathbb{P}_{\mu}[e^{-X_t(f)}]
  = e^{-\mu(V_tf)}
  , \quad t\geq 0, \mu \in \mathcal M(\mathbb R^d),
\end{align}
where $(t,x) \mapsto V_tf(x)$ is the unique locally bounded non-negative solution to the equation
\begin{align}
  V_tf(x) + \Pi_x \Big[ \int_0^t\psi (V_{t-s}f(\xi_s) )~ds\Big]
	= \Pi_x [f(\xi_t)]
  , \quad x\in \mathbb R^d, t\geq 0.
\end{align}	
The existence of such super-OU process $X$ is well known, see \cite{Dynkin1993Superprocesses} for instance.

Recently, there have been quite a few papers on laws of large numbers for superdiffusions.
In \cite{Englander2009Law, EnglanderWinter2006Law, EnglanderTuraev2002A-scaling}, some weak laws of large numbers (convergence in law or in probability) were established.
The strong law of large numbers for superprocesses was first studied in \cite{ChenRenWang2008An-almost}, followed by \cite{ChenRenSongZhang2015Strong-law, ChenRenYang2019Skeleton, EckhoffKyprianouWinkel2015Spines, KouritzinRen2014A-strong, LiuRenSong2013Strong, Wang2010An-almost} under different settings.
For a good survey on recent developments in laws of large numbers for branching Markov processes and superprocesses, see \cite{EckhoffKyprianouWinkel2015Spines}.


The strong law of large numbers for the super-OU process $X$ above can be stated as follows:
Under some conditions on $\psi$ (these conditions are satisfied under our Assumptions 1 and 2 below),
there exists an $\Omega_0$ of $\mathbb{P}_\mu$-full probability for every $\mu\in\mathcal M(\mathbb R^d)$ such that on $\Omega_0$, for every Lebesgue-a.\/e. continuous bounded non-negative function $f$ on $\mathbb R^d$, we have
 $\lim_{t\to\infty} e^{-\alpha t} X_t(f) =H_\infty\langle f, \varphi\rangle $,
where $H_\infty$ is the limit of the martingale $e^{-\alpha t}X_t(1)$
and $\varphi$ is the invariant density of the OU process $\xi$ defined in \eqref{invariantdensity} below.
See \cite[Theorem 2.13 \& Example 8.1]{ChenRenYang2019Skeleton} and \cite[Theorem 1.2 \& Example 4.1]{EckhoffKyprianouWinkel2015Spines}.

In this paper, we will establish some spatial central limit theorems (CLTs) for the super-OU process $X$ above.
Our key assumption is that $\psi$ satisfies Grey's condition and some perturbation condition which guarantees that,
when  $z\to 0$, $\psi(z)=-\alpha z + \eta z^{1+\beta} (1+o(1))$ with $\alpha > 0$, $\eta>0$ and $\beta\in (0, 1)$.
Our  goal is to find $(F_t)_{t\geq 0}$ and $(G_t)_{t\geq 0}$ so that
$ (X_t(f) -G_t)/F_t $ converges weakly to some non-degenerate random variable as $t\rightarrow\infty$, for a large class of functions $f$.
Note that, in the setting of this paper, $X_t(f)$ typically has infinite second moment.

There are many papers on CLTs for branching processes, branching diffusions and superprocesses under the second moment condition.
See \cite{Heyde1970A-rate, HeydeBrown1871An-invariance, HeydeLeslie1971Improved} for supercritical Galton-Watson processes (GW processes),
\cite{KestenStigum1966Additional,KestenStigum1966A-limit} for supercritical multi-type GW processes, \cite{Athreya1969Limit,Athreya1969LimitB,Athreya1971Some}
for supercritical multi-type continuous time branching processes and \cite{AsmussenHering1983Branching} for general supercritical branching Markov processes under certain conditions.
Some spatial CLTs for supercritical branching OU processes with binary branching mechanism were proved in \cite{AdamczakMilos2015CLT} and some
spatial CLTs for supercritical super-OU processes with branching mechanisms satisfying a fourth moment condition were proved in \cite{Milos2012Spatial}.
These two papers made connections between CLTs and branching rate regimes.
Some spatial CLTs  for supercritical super-OU  processes with branching mechanisms satisfying only a second moment condition were established in \cite{RenSongZhang2014Central}.
Moreover, compared with the results of \cite{AdamczakMilos2015CLT,Milos2012Spatial}, the limit distributions in \cite{RenSongZhang2014Central} are non-degenerate.
Since then, a series of spatial CLTs for a large class of general supercritical branching Markov processes and superprocesses with spatially dependent branching mechanisms were proved in \cite{RenSongZhang2014CentralB,RenSongZhang2015Central,RenSongZhang2017Central}.
The functional version of the CLTs were established in \cite{Janson2004Functional} for supercritical multitype branching processes, and in \cite{RenSongZhang2017Functional} for supercritical superprocesses.


There are also many limit theorems for supercritical branching processes and branching Markov processes with branching mechanisms of infinite second moment.
Heyde \cite{Heyde1971Some} established some  CLTs for supercritical GW processes when the offspring distribution belongs to the domain of attraction of a stable law of index $\alpha\in (1, 2]$, and proved that the limit laws are stable laws.
Similar results  for supercritical multi-type GW processes and supercritical continuous time branching processes,
under some $p$-th ($p\in(1,2]$) moment condition on the offspring distribution, were given in Asmussen \cite{Asmussen76Convergence}.
 Recently, Marks and Milo\'s \cite{MarksMilos2018CLT} considered the limit behavior of supercritical branching OU processes with a special stable offspring distribution.
They established some spatial CLTs in the small and critical branching rate regimes, but they did not prove any CLT type result in the large branching rate regime.
We also mention here that very recently \cite{IksanovKoleskoMeiners2018Stable-like} considered stable fluctuations of Biggins' martingales in the context of branching random walks and \cite{RenSongSun2018Limit} considered the asymptotic behavior
of a class of critical superprocesses with spatially dependent stable branching mechanism.


As far as we know, this paper is the first to study spatial CLTs for supercritical superprocesses without the second moment condition.

\subsection{Main results}
\label{sec:I:R}
We will always assume that the following assumption holds.
\begin{asp}
  \label{asp: Greys condition}
  The branching mechanism $\psi$ satisfies Grey's condition, i.e., there exists $z' > 0$ such that $\psi(z) > 0$ for all $z>z'$ and  $\int_{z'}^\infty \psi(z)^{-1}dz < \infty$.
\end{asp}
For $\mu \in \mathcal M(\mathbb R^d)$, write $\|\mu\| = \mu(1)$.
It is known (see \cite[Theorems 12.5 \& 12.7]{Kyprianou2014Fluctuations} for example) that, under Assumption \ref{asp: Greys condition}, the \emph{extinction event}
$$D :=\{\exists t\geq 0,~\text{such that}~ \|X_t\| =0 \}$$
has positive probability with respect to $\mathbb P_\mu$ for each  $\mu \in \mathcal M(\mathbb R^d)$.
In fact, $ \mathbb{P}_{\mu} (D) = e^{-\bar v \|\mu\|}$ where $ \bar v := \sup\{\lambda \geq 0: \psi(\lambda) = 0\} \in (0,\infty) $ is the largest root of $\psi$.

Denote by $\Gamma$ the gamma function.
For any $\sigma$-finite signed measure $\mu$, we use $|\mu|$ to denote the total variation measure of $\mu$.
In this paper, we will also assume the following:
\begin{asp}
  \label{asp: branching mechanism}
  There exist constants $\eta > 0$ and $\beta \in (0,1)$ such that
  \begin{align}
    \label{eq: asp of branching mechanism}
    \int_{(1,\infty)}y^{1+\beta +\delta}~\Big|\pi(dy)-\frac{\eta~dy}{\Gamma(-1-\beta)y^{2+\beta}}\Big| <\infty
  \end{align}
	for some $\delta > 0$.
\end{asp}

Roughly speaking, Assumption \ref{asp: branching mechanism} says that $\psi$ is ``not too far away'' from $\widetilde \psi(z) := - \alpha z + \eta z^{1+\beta}$ near $0$, see \cite[Remark 1.3]{RenSongSunZhao2019Stable}.
It follows from \cite[Lemma 2.2]{RenSongSunZhao2019Stable}  that, if Assumption \ref{asp: branching mechanism} holds, then $\eta$ and $\beta$ are uniquely determined by the L\'evy measure $\pi$.
In \cite[Lemma 2.3]{RenSongSunZhao2019Stable}, we have shown  that,
under Assumption \ref{asp: branching mechanism},
 $\psi$ satisfies the $L \log L$ condition, i.e., $ \int_{(1,\infty)} y\log y~\pi(dy) < \infty. $
%This guarantees that $H_\infty$, the limit of the non-negative martingale $(e^{-\alpha t} \|X_t\|)_{t\geq 0}$, is non-degenerate.

In the reminder of the paper, we will always use $\eta$ and $\beta$ to denote the constants in Assumption  \ref{asp: branching mechanism}.
	Note that $\delta$ is not uniquely determined by $\pi$.
	In fact, if $\delta>0$ is a constant such that \eqref{eq: asp of branching mechanism} holds, then replacing $\delta$ by any smaller positive number, \eqref{eq: asp of branching mechanism} still holds.
	Therefore, Assumption \ref{asp: branching mechanism} is equivalent to the following statement:
	There exist constants $\eta > 0$ and $\beta \in (0,1)$ such that, for all small enough $\delta>0$, \eqref{eq: asp of branching mechanism} holds.


Let us introduce some notation in order to give the precise formulation of our main result.
Denote by $\mathcal B(\mathbb R^d, \mathbb R)$ the space of all $\mathbb R$-valued Borel functions on $\mathbb R^d$.
Denote by $\mathcal B(\mathbb R^d, \mathbb R_+)$ the space of all $\mathbb R_+$-valued Borel functions on $\mathbb R^d$.
We use  $(P_t)_{t\geq 0}$ to denote the transition semigroup of $\xi$.	
Define
\(
P^{\alpha}_t f(x)
  := e^{\alpha t} P_t f(x)
  = \Pi_x [e^{\alpha t}f(\xi_t)]
\)
for each $x\in \mathbb R^d$, $t\geq 0$ and $f\in \mathcal B(\mathbb R^d, \mathbb R_+)$.
It is known that, see \cite[Proposition 2.27]{Li2011Measure-valued} for example, $(P^\alpha_t)_{t\geq 0}$ is the \emph{mean semigroup} of $X$ in the sense that
\(
  \mathbb{P}_{\mu}[X_t (f)]  = \mu( P^\alpha_t f)
\)
for all $\mu\in \mathcal M(\mathbb R^d)$, $t\geq 0$ and $f\in \mathcal B(\mathbb R^d, \mathbb R_+)$.


The limit behavior of $X$  is closely related to the spectral property of the OU semigroup $(P_t)_{t\geq 0}$ which we now recall (See \cite{MetafunePallaraPriola2002Spectrum} for more details).
It is known that the OU process $\xi$ has an invariant probability on $\mathbb R^d$
\begin{align}
  \label{invariantdensity}
  \varphi(x)dx
  :=\Big (\frac{b}{\pi \sigma^2}\Big )^{d/2}\exp \Big(-\frac{b}{\sigma^2}|x|^2 \Big)dx
\end{align}
which is a   symmetric multivariate Gaussian distribution.
Let $L^2(\varphi)$ be the Hilbert space with inner product
\begin{align}
  \langle f_1, f_2 \rangle_{\varphi}
  := \int_{\mathbb R^d}f_1(x)f_2(x)\varphi(x) dx, \quad f_1,f_2 \in L^2(\varphi).
\end{align}
Let $\mathbb Z_+ := \mathbb N\cup\{0\}$.
For each $p = (p_k)_{k = 1}^d \in \mathbb{Z}_+^{d}$, write $|p|:=\sum_{k=1}^d p_k$, $p!:= \prod_{k= 1}^d p_k!$ and $\partial_p:= \prod_{k = 1}^d(\partial^{p_k}/\partial x_k^{p_k})$.
The \emph{Hermite polynomials} are defined by
\begin{align}
  H_p(x)
  :=(-1)^{|p|}\exp(|x|^2) \partial_p \exp(-|x|^2)
  , \quad x\in \mathbb R^d, p \in \mathbb{Z}_+^{d}.
\end{align}
It is known that $(P_t)_{t\geq 0}$ is a strongly continuous semigroup in $L^2(\varphi)$ and its generator $L$ has discrete spectrum $\sigma(L)= \{-bk: k \in \mathbb Z_+\}$.
For $k \in \mathbb Z_+$, denote by $\mathcal{A}_k$ the eigenspace corresponding to the eigenvalue $-bk$, then $ \mathcal{A}_k = \operatorname{Span} \{\phi_p : p\in \mathbb Z_+^d, |p|=k\}$ where
\begin{align}
  \label{eigenfunction}
  \phi_p(x)
  := \frac{1}{\sqrt{ p! 2^{|p|} }} H_p \Big(\frac{ \sqrt{b} }{\sigma}x \Big)
  , \quad x\in \mathbb R^d, p\in \mathbb Z_+^d.
\end{align}
In other words,
\(
  P_t\phi_p(x)
  = e^{-b|p|t}\phi_p(x)
\)
for all $t\geq 0$, $x\in \mathbb R^d$ and $p\in \mathbb Z_+^d$.
Moreover, $\{\phi_p: p \in \mathbb Z_+^d\}$ forms a complete orthonormal basis of $L^2(\varphi)$.
Thus for each $f\in L^2(\varphi)$, we have
\begin{align}
  \label{semicomp1}
  f
  = \sum_{k=0}^{\infty}\sum_{p\in \mathbb Z_+^d:|p|=k}\langle f, \phi_p \rangle_{\varphi} \phi_p
  , \quad \text{in~} L^2(\varphi).
\end{align}
For each function $f\in L^2(\varphi)$, define the order of $f$ as
\[
  \kappa_f
  := \inf \left \{k\geq 0: \exists ~ p\in \mathbb Z_+^d , {\rm ~s.t.~} |p|=k {\rm ~and~}  \langle f, \phi_p \rangle_{\varphi}\neq 0\right \}
\]
which is the lowest non-trivial frequency in the eigen-expansion \eqref{semicomp1}.
Note that $ \kappa_f\geq 0$ and that, if $f\in L^2(\varphi)$ is non-trivial, then $\kappa_f<\infty$.
In particular, the order of any constant non-zero function is zero.

Denote by $\mathcal M_c(\mathbb R^d)$ the space of all finite Borel measures of compact support on $\mathbb R^d$.
For $p\in \mathbb{Z}_+^d$, define
\[
  H_t^p
  := e^{-(\alpha-|p|b)t}X_t(\phi_p), \qquad t\geq 0.
\]
We will write $H^0_t$ as $H_t$ for simplicity.
For $u \neq -1$, we write $\tilde u = u/(1+ u)$.
We have shown in \cite[Lemma 3.2]{RenSongSunZhao2019Stable} that, (1) for any
$\mu\in \mathcal M_c(\mathbb R^d)$, $(H_t^p)_{t\geq 0}$ is a $\mathbb{P}_{\mu}$-martingale; (2)
if $\alpha \tilde \beta>|p|b$, then for all $\gamma\in (0, \beta)$ and $\mu\in \mathcal M_c(\mathbb R^d)$,  $(H_t^p)_{t\geq 0}$ is a $\mathbb{P}_{\mu}$-martingale bounded in $L^{1+\gamma}(\mathbb{P}_{\mu})$.
Thus the limit $H^p_{\infty}:=\lim_{t\rightarrow \infty}H_t^p$ exists $\mathbb{P}_{\mu}$-almost surely and in $L^{1+\gamma}(\mathbb{P}_{\mu})$.
Similar as before, we will write $H^0_\infty$ as $H_\infty$.



Denote by $\mathcal P$ the class of functions of polynomial growth on $\mathbb R^d$, i.e.,
\begin{align}
  \label{eq: polynomial growth function}
  \mathcal{P}
  := \{f\in \mathcal B(\mathbb R^d, \mathbb R):\exists C>0, n \in \mathbb Z_+ \text{~s.t.~} \forall x\in \mathbb R^d, |f(x)|\leq C(1+|x|)^n \}.
\end{align}
It is clear that $\mathcal{P} \subset L^2(\varphi)$. The following law of large numbers is
\cite[Theorem 1.5]{RenSongSunZhao2019Stable}.
\begin{thm}
  \label{thm: law of large number}
  If $f \in \mathcal{P}$ satisfies $\alpha\tilde \beta>\kappa_f b$, then for all $\gamma\in (0, \beta)$ and  $\mu\in \mathcal M_c(\mathbb R^d)$,
  \[
    e^{-(\alpha-\kappa_fb)t}X_t(f)
    \xrightarrow[t\to \infty]{}\sum_{p\in \mathbb Z_+^d:|p|=\kappa_f}\langle f, \phi_p\rangle_{\varphi} H_{\infty}^p
    \quad in~ L^{1+\gamma}(\mathbb{P}_{\mu}).
  \]
  Moreover, if $f$ is twice differentiable and all its second order partial derivatives are in $\mathcal{P}$, then we also have almost sure convergence.
\end{thm}
If $f\in \mathcal B(\mathbb R^d, \mathbb R_+)$ is non-trivial and  bounded, then $\kappa_f=0$.
Hence, Theorem \ref{thm: law of large number} says that for any $\gamma\in (0, \beta)$ and  $\mu\in \mathcal M_c(\mathbb R^d)$, as $t\rightarrow \infty$,
\(
  e^{-\alpha t}X_t(f)
  \rightarrow \langle f, \varphi\rangle H_{\infty}
\)
in $L^{1+\gamma}(\mathbb{P}_{\mu})$.
Moreover, if $f$ is twice differentiable and all its second order partial derivatives are in $\mathcal{P}$, then we also have a.s.\ convergence.
However, to get a.s.\ convergence for bounded non-negative
Lebesgue-a.e.\ continuous functions $f$, we do not need $f$ to be twice differentiable.
See \cite[Theorem 2.13 \& Example 8.1]{ChenRenYang2019Skeleton} and \cite[Theorem 1.2 \& Example 4.1]{EckhoffKyprianouWinkel2015Spines}.

For the rest of this subsection, we focus on the CLTs of $X_t(f)$ for a large collection of $f\in \mathcal P\setminus \{0\}$.
Now we recall some notation and the CLT from \cite{RenSongSunZhao2019Stable}.
The CLT,  \cite[Theorem 1.6]{RenSongSunZhao2019Stable}, is stated  in the following three cases:
\begin{enumerate}
\item
  the small branching rate case where
$f$ satisfies $\alpha \tilde \beta < \kappa_f b$;
\item
  the critical branching rate case where
$f$ satisfies $\alpha \tilde \beta = \kappa_f b$; and
\item
  the large branching rate case  where
$f$ satisfies $\alpha \tilde \beta > \kappa_f b$.
\end{enumerate}
Here, the small (resp. large) branching rate case means that the branching rate $\alpha$ is small (resp. large) compared to $\kappa_f$;
 and the critical branching rate means that the branching rate $\alpha$ is at a critical balanced value compared to $\kappa_f$.
Define a family of operators $(T_t)_{t\geq 0}$ on $\mathcal P$ by
\begin{align}
  \label{eq:I:R:1}
  T_t f
  := \sum_{p \in \mathbb Z_+^d} e^{-| |p|b - \alpha \tilde \beta |t} \langle f, \phi_p \rangle_{\varphi} \phi_p
  ,\quad t\geq 0, f\in \mathcal P,
\end{align}
and a family of $\mathbb C$-valued functionals $(m_t)_{0 \leq t < \infty}$ on $\mathcal P$ by
\begin{align}
  \label{eq:I:R:2}
  m_t[f]
  := \eta \int_0^t ~du \int_{\mathbb R^d} (-iT_u f(x))^{1+\beta} \varphi(x) ~dx
  , \quad 0 \leq t< \infty, f\in \mathcal P.
\end{align}
Define $ \mathcal C_s := \mathcal P \cap \overline{\operatorname{Span}} \{ \phi_p: \alpha \tilde \beta < |p| b \}$, $\mathcal C_c   := \mathcal P \cap \operatorname{Span} \{ \phi_p : \alpha \tilde \beta = |p| b \} $
and $ \mathcal C_l   := \mathcal P \cap \operatorname{Span} \{ \phi_p: \alpha \tilde \beta > |p| b \}$.
Note that $\mathcal C_s$ is an infinite dimensional space, $ \mathcal C_l$ and $\mathcal C_c$
are finite dimensional spaces, and $\mathcal C_c$ might be empty.
We have shown in \cite[Lemma 2.6 and Propsoition 2.7]{RenSongSunZhao2019Stable} that,
for $f\in \mathcal P\setminus \{0\}$,
\begin{align}
  \label{eq:I:R:3}
  m[f]
  := \begin{cases}
    \lim_{t\to \infty} m_t[f], &
    f \in \mathcal C_s \oplus \mathcal C_l, \\
    \lim_{t\to \infty} \frac{1}{t} m_t[f], & f\in \mathcal P \setminus \mathcal C_s \oplus \mathcal C_l,
  \end{cases}
\end{align}
is well defined, and moreover, there exists a $(1+\beta)$-stable random variable $\zeta^f$ with characteristic function
\begin{align}\label{eq: charac func11}
 \theta \mapsto e^{m[\theta f]}, \quad \theta \in \mathbb R.
\end{align}
The following result is \cite[Theorem 1.6]{RenSongSunZhao2019Stable}.

\begin{thm}
\label{thm:RSSZ}
	If $\mu\in \mathcal M_c(\mathbb R^d)\setminus \{0\}$, then under $\mathbb{P}_{\mu}(\cdot|D^c)$, the following hold:
\begin{enumerate}
\item \label{thm:M:1}
  	if $f\in \mathcal C_s\setminus\{0\}$, then $\|X_t\|^{- \frac{1}{1+\beta}} X_t(f)  \xrightarrow[t\to \infty]{d} \zeta^f$;
\item \label{thm:M:2}
  	if $f\in \mathcal C_c\setminus\{0\}$, then $ \|t X_t\|^{-\frac{1}{1+\beta}} X_t(f) \xrightarrow[t\to \infty]{d} \zeta^f$;
\item \label{thm:M:3}
  	if $f\in \mathcal C_l\setminus\{0\}$, then
\[
    \|X_t\|^{-\frac{1}{1+\beta}} \Big( X_t(f) - \sum_{p\in \mathbb Z^d_+:\alpha \tilde \beta>|p|b}\langle f,\phi_p\rangle_\varphi e^{(\alpha-|p|b)t}H^p_{\infty}\Big)
    \xrightarrow[t\to \infty]{d}
    \zeta^{-f}.
\]
\end{enumerate}
\end{thm}

The theorem above does not cover all $f\in \mathcal P$.
	Theorem \ref{thm:RSSZ}.(1) can be rephrased as if $f\in \mathcal P\setminus\{0\}$ satisfies   $\alpha \tilde \beta < \kappa_f b$, then  under $\mathbb{P}_{\mu}(\cdot|D^c)$, $\|X_t\|^{- \frac{1}{1+\beta}} X_t(f)  \xrightarrow[t\to \infty]{d} \zeta^f$.
	Combining the first two parts of Theorem \ref{thm:RSSZ}, one can easily get that  if $f\in \mathcal P$ satisfies   $\alpha \tilde \beta = \kappa_f b$, then  under $\mathbb{P}_{\mu}(\cdot|D^c)$, $\|t X_t\|^{-\frac{1}{1+\beta}} X_t(f) \xrightarrow[t\to \infty]{d} \zeta^f$.
	A general  $f \in \mathcal P$ can be decomposed as $f_s + f_c + f_l$ with $f_s \in \mathcal C_s$, $f_c \in \mathcal C_c$ and $f_l \in \mathcal C_l$.
	For $f\in  \mathcal P$ satisfying $\alpha \tilde \beta > \kappa_f b$, $f_s$ and $f_c$
	maybe non-zero. In \cite{RenSongSunZhao2019Stable}, we were not able to
	establish a CLT in this case. We conjectured there that the limit random variables in Theorem \ref{thm:RSSZ} for $ f\in \mathcal C_s$, $f\in \mathcal C_c$ and $ f\in \mathcal C_l$ are independent. Once this asymptotic independence is established, a CLT for $ X_t(f)$ for all $f\in  \mathcal P$ would follow.
	
	The main purpose of this paper is to prove the asymptotic independence conjectured in
	\cite{RenSongSunZhao2019Stable} and prove a CLT for $ X_t(f)$ for all $f\in  \mathcal P$.

Let
\begin{align}
S(t):=\Bigg(e^{-\alpha t}\|X_t\|, \frac{X_t(f_s)}{\|X_t\|^{\frac{1}{1+\beta}}},\frac{X_t(f_c)}{\|tX_t\|^{\frac{1}{1+\beta}}},\frac{ X_t(f_l) - \sum_{p\in \mathbb Z^d_+:\alpha \tilde \beta>|p|b}\langle f_l,\phi_p\rangle_\varphi e^{(\alpha-|p|b)t}H^p_{\infty}}{\|X_t\|^{\frac{1}{1+\beta}}}\Bigg).
\end{align}
The main result of this paper is as follows.

\begin{thm}
  \label{thm:M}
 If $\mu\in \mathcal M_c(\mathbb R^d)\setminus \{0\}$,$f_s\in \mathcal C_s$, $f_c \in \mathcal C_c$ and $f_l \in \mathcal C_l$, then under $\mathbb{P}_{\mu}(\cdot|D^c)$,
% the following hold:
\begin{align}
%&\Bigg(e^{-\alpha t}\|X_t\|, \frac{X_t(f_s)}{\|X_t\|^{\frac{1}{1+\beta}}},\frac{X_t(f_c)}{\|tX_t\|^{\frac{1}{1+\beta}}},\frac{ X_t(f_l) - \sum_{p\in \mathbb Z^d_+:\alpha \tilde \beta>|p|b}\langle f_l,\phi_p\rangle_\varphi e^{(\alpha-|p|b)t}H^p_{\infty}}{\|X_t\|^{\frac{1}{1+\beta}}}\Bigg)\\
%&
S(t)
\xrightarrow[t\rightarrow \infty]{d}(H^*,\zeta^{f_s},\zeta^{f_c},\zeta^{-f_l}),
\end{align}
where $H^*$ has the same distribution as $H_{\infty}$ conditioned on $D^c$, $\zeta^{f_s}$, $\zeta^{f_c}$ and $\zeta^{-f_l}$ are $(1+\beta)$-stable random varibales with characteristic functions described by \eqref{eq: charac func11}. Moreover, $H^*$,  $\zeta^{f_s}$, $\zeta^{f_c}$ and $\zeta^{-f_l}$ are independent.
\end{thm}

As a corollary of this theorem, we can get the following CLTs for all function $f\in \mathcal P$.

\begin{cor} Suppose $f\in \mathcal P\setminus\{0\}$. Let $F=f_s + f_c + f_l$ be the decomposition of $f$ with $f_s \in \mathcal C_s$, $f_c \in \mathcal C_c$ and $f_l \in \mathcal C_l$.
If $\mu\in \mathcal M_c(\mathbb R^d)\setminus \{0\}$, then under $\mathbb{P}_{\mu}(\cdot|D^c)$, the following hold:
\begin{enumerate}
\item if $\alpha \tilde \beta < \kappa_f b$, then
\[
\|X_t\|^{- \frac{1}{1+\beta}} X_t(f)  \xrightarrow[t\to \infty]{d} \zeta^f;
\]
\item if $\alpha \tilde \beta = \kappa_f b$, then
\[
\|t X_t\|^{-\frac{1}{1+\beta}} X_t(f) \xrightarrow[t\to \infty]{d} \zeta^{f_c};
\]
\item  if $\alpha \tilde \beta > \kappa_f b$ and $f_c=0$, then
\[
   \|X_t\|^{-\frac{1}{1+\beta}} \Big( X_t(f) - \sum_{p\in \mathbb Z^d_+:\alpha \tilde \beta>|p|b}\langle f,\phi_p\rangle_\varphi e^{(\alpha-|p|b)t}H^p_{\infty}\Big)
    \xrightarrow[t\to \infty]{d}
     \zeta^{f_s}+\zeta^{-f_l};
\]
\item  if $\alpha \tilde \beta > \kappa_f b$ and $f_c\neq0$, then
\[
 \|t X_t\|^{-\frac{1}{1+\beta}}\Big( X_t(f) - \sum_{p\in \mathbb Z^d_+:\alpha \tilde \beta>|p|b}\langle f,\phi_p\rangle_\varphi e^{(\alpha-|p|b)t}H^p_{\infty}\Big)
    \xrightarrow[t\to \infty]{d}\zeta^{f_c}.
\]
\end{enumerate}
\end{cor}







There are still many open questions on stable central limit theorems for superprocesses.
Ren, Song and Zhang have established some spatial  CLTs in \cite{RenSongZhang2015Central} for a class of superprocesses with general spatial motions under
the assumption that the branching mechanisms satisfy a second moment condition.
We hope to prove spatial CLTs for superprocesses with general motions without the second moment assumption on the branching mechanism in a future project.

Recall that our Assumption \ref{asp: branching mechanism} says that the branching mechanism $\psi$ is $-\alpha z +\eta z^{1+\beta}$ plus a small perturbation
$\psi_1(z)$
which satisfies \eqref{eq: asp of branching mechanism} with some $\delta>0$.
It would be interesting to consider more general branching mechanisms.


The rest of the paper is organized as follows:
In Subsection \ref{sec: main},
we will give the proof of Theorem \ref{thm:M}.


\section{Proofs of main results}
\label{proofs of main results}
In this section, we will prove the main result of this paper.
For simplicity, we will write $\mathbb{\widetilde{P}}_{\mu}=\mathbb{P}_{\mu}(\cdot|D^c)$ in this section.



%\subsection{Central limit theorems for unit time intervals}
%\label{sec:critical}
%In this subsection, we will establish the following CLT.
%\begin{thm}
%  \label{lem:PR:LC}
%If  $\mu \in \mathcal M_c(\mathbb R^d)$ and $f\in \mathcal{P}\setminus \{0\}$,
%then  under $\mathbb{P}_{\mu}(\cdot | D ^c)$, we have
%  \begin{align}
%    \label{eq:PR:LC:1}
%    \Upsilon^f_t
%    := \frac{X_{t+1} (f) - X_t(P_1^\alpha f)}{\| X_t\|^{1-\tilde \beta}}
%    \xrightarrow[t\to \infty]{d}\zeta^f_0,
%  \end{align}
%  where $\zeta^f_0$ is a $(1+\beta)$-stable random variable with characteristic function $\theta\mapsto e^{\langle Z_1(\theta f), \varphi\rangle}$.
%\end{thm}
\subsection{Preliminaries}

The following  elementary analysis result will play an important role in this paper.

\begin{lem}\label{ineq: analysis}
There exists a constant $C>0$, such that for any $x,y \in \mathbb R$,
\[
    |(x+y)^{1+\beta}-x^{1+\beta}-y^{1+\beta}|\leq C(|x||y|^{\beta}+|x|^{\beta}|y|).
\]
\end{lem}
\begin{proof}
   Note that
\[
  \lim_{|y|\rightarrow \infty}\frac{(y+1)^{1+\beta}-y^{1+\beta}-1}{y^{\beta}}=\lim_{|y|\rightarrow \infty}\frac{(y+1)^{1+\beta}-y^{1+\beta}}{y^{\beta}}=\lim_{|y|\rightarrow \infty}\big((1+\frac{1}{y})^{1+\beta}-1\big)y = 1+\beta.
\]
using this and continuity, we get that there exists $C_1>0$ such that for all $|y|>1$,
\[
  |(1+y)^{1+\beta}-y^{1+\beta}-1|\leq C_1 |y|^{\beta}.
\]
Hence, when $|x|>|y|$ and $y\neq 0$, we have
\[
|(x+y)^{1+\beta}-x^{1+\beta}-y^{1+\beta}|\leq |y|^{1+\beta}\Big(|(1+\frac{x}{y})^{1+\beta}-(\frac{x}{y})^{1+\beta}-1|\Big)\leq C_1|y||x|^{\beta};
\]
and when $|x|<|y|$ and $x\neq 0$, we have
\[
|(x+y)^{1+\beta}-x^{1+\beta}-y^{1+\beta}|\leq |x|^{1+\beta}\Big(|(1+\frac{y}{x})^{1+\beta}-(\frac{y}{x})^{1+\beta}-1|\Big)\leq C_1|x||y|^{\beta}.
\]
Combining the above, we immediately get the desired conclusion.
\end{proof}

For $g\in \mathcal P$, define $\mathcal P_g:= \{\theta T_ng:n \in \mathbb Z_+, \theta \in [-1,1]\}$. 	For all $t\geq 0$ and $f\in \mathcal P$, define
	\begin{align}
  	Z_t f
  	&:= \int_0^t P^\alpha_{t-s}\big( \eta (-i P^\alpha_sf)^{1+\beta}\big)ds,\\
  	\Upsilon^f_t
   & := \frac{X_{t+1} (f) - X_t(P_1^\alpha f)}{\| X_t\|^{1-\tilde \beta}}.
  	\end{align}
The following corollary to  \cite[Proposition 3.5]{RenSongSunZhao2019Stable} will be used later in the proof of Theorem \ref{thm:M}.

\begin{cor}
  \label{cor:MI}
For each  $f,g\in \mathcal P$ and $\mu\in \mathcal M_c(\mathbb R^d)$, there exist $C,\delta>0$ such that for
all $n_1,n_2 \in \mathbb Z_+$, $(f_j)_{j=0}^{n_1}\subset \mathcal P_f$ and $(g_j)_{j=0}^{n_2}\subset \mathcal P_g$, we have
\begin{align}
  \label{32corollary}
  \Big|\mathbb{\widetilde{P}}_{\mu}\Big[\prod_{k=0}^{n_1}e^{i \Upsilon^{f_k}_{t-k-1}}\prod_{k=0}^{n_2}e^{i \Upsilon^{g_k}_{t+k} }\Big]-\prod_{k=0}^{n_1} e^{\langle Z_1f_k, \varphi\rangle}\prod_{k=0}^{n_2} e^{\langle Z_1g_k, \varphi\rangle}\Big|\leq C e^{-\delta (t-n_1)}.
\end{align}
\end{cor}
\begin{proof}
For $n_1,n_2 \in \mathbb Z_+$, $(f_j)_{j=0}^{n_1}\subset \mathcal P_f$, $(g_j)_{j=0}^{n_2}\subset \mathcal P_g$, $k_1 \in \{-1,0,\dots,n_1\}$ and $k_2 \in \{-1,0,\dots,n_2\}$,  define
  \[
    a_{k_1,k_2}
    :=  \mathbb{\widetilde{P}}_{\mu}\Big[\prod_{j=k_1+1}^{n_1} e^{i\Upsilon_{t-j-1}^{f_j}}\prod_{j=0}^{k_2}e^{i\Upsilon_{t+j}^{g_j}}\Big] \times \Big(\prod_{j=0}^{k_1}e^{\langle Z_1 f_j, \varphi\rangle}\prod_{j=k_2+1}^{n_2} e^{ \langle Z_1g_j,\varphi \rangle} \Big),
  \]
 where by convention the product is 1 for $k_1=-1$ or $k_2 = -1$.
  Then for all  $k_2 \in \{0,\dots,n_2\}$, we have
  \begin{align}
    & a_{-1,k_2} - a_{-1,k_2-1}\\
     & =\mathbb{P}_{\mu}(D^c)^{-1} \mathbb{P}_{\mu}\Big[\prod_{j=0}^{n_1}e^{i\Upsilon_{t-j-1}^{f_j}}\prod_{j=0}^{k_2-1} e^{i\Upsilon_{t+j}^{g_j}}(e^{i\Upsilon^{g_{k_2}}_{t+k_2}}-e^{\langle Z_1g_{k_2}, \varphi\rangle});D^c\Big] \Big(\prod_{j=k_2+1}^n e^{\langle Z_1g_j, \varphi\rangle}\Big)\\
    & =\mathbb{P}_{\mu}(D^c)^{-1} \mathbb{P}_{\mu}\Big[\prod_{j=0}^{n_1}e^{i\Upsilon_{t-j-1}^{f_j}}\prod_{j=0}^{k_2-1} e^{i\Upsilon_{t+j}^{g_j}}\mathbb P_\mu[e^{i\Upsilon^{g_{k_2}}_{t+k_2}}-e^{\langle Z_1g_{k_2}, \varphi\rangle}; D^c|\mathscr F_{t+k_2}] \Big] \Big(\prod_{j=k_2+1}^{n}e^{\langle Z_1g_j, \varphi\rangle}\Big).
  \end{align}
%  By Lemma \ref{thm:Key},
It follows from \cite[Lemma 2.8 and Proposition 3.5]{RenSongSunZhao2019Stable} and the definition of $Z_t$ that
  there exist $C_0,\delta_0 >0$ such that for all $n_1,n_2 \in \mathbb Z_+$, $(f_j)_{j=0}^{n_1}\subset \mathcal P_f$, $(g_j)_{j=0}^{n_2}\subset \mathcal P_g$ and $k_2 \in \{0,\dots,n_2\}$ we have
\begin{align}
    | a_{-1,k_2} - a_{-1,k_2-1}|
    & \leq \mathbb{P}_{\mu}(D^c)^{-1}\mathbb{P}_{\mu}\Big[\big|\mathbb P_\mu[e^{i\Upsilon^{g_{k_2}}_{t+k_2}}-e^{\langle Z_1g_{k_2}, \varphi\rangle}; D^c|\mathscr F_{t+k_2}]\big|\Big]
    \leq C_0 e^{-\delta_0 (t+k_2)}.
\end{align}
Similarly,  we also get that there exist $C_0',\delta_0' >0$ such that for all $n_1,n_2 \in \mathbb Z_+$, $(f_j)_{j=0}^{n_1}\subset \mathcal P_f$, $(g_j)_{j=0}^{n_2}\subset \mathcal P_g$ and $k_1 \in \{0,\dots,n_1\}$,
\begin{align}
    | a_{k_1-1,-1} - a_{k_1,-1}|
    & \leq \mathbb{P}_{\mu}(D^c)^{-1}\mathbb{P}_{\mu}\Big[\big|\mathbb P_\mu[e^{i\Upsilon^{f_{k_1}}_{t-k_1-1}}-e^{\langle Z_1f_{k_1}, \varphi\rangle}; D^c|\mathscr F_{t-k_1-1}]\big|\Big]
    \leq C_0' e^{-\delta_0' (t-k_1)}.
\end{align}
  Therefore, there exist $C,\delta >0$ such that for all $n_1,n_2 \in \mathbb Z_+$, $(f_j)_{j=0}^{n_1}\subset \mathcal P_f$ and $(g_j)_{j=0}^{n_2}\subset \mathcal P_g$, we have
  \begin{align}
    \text{LHS of \eqref{32corollary}}
 %   & = \left|a_{-1,n_2}-a_{n-1,-1}\right|
    & = \left|a_{-1,n_2}-a_{n_1,-1}\right|
      \leq \sum_{k=0}^{n_1}\left|a_{k-1,-1}-a_{k,-1}\right|+\sum_{k=0}^{n_2}\left|a_{-1,k}-a_{-1,k-1}\right|\\
     & \leq \sum_{k=0}^{n_1} C_0' e^{-\delta_0' (t-k)}+\sum_{k=0}^{n_2} C_0 e^{-\delta_0 (t+k)}
      \leq C e^{- \delta (t-n_1)}.
      \qedhere
  \end{align}
\end{proof}

\subsection{Proof of Theorem \ref{thm:M}}
\label{sec: main}
In this subsection, we always fix $\mu \in \mathcal M_c(\mathbb R^d)$, $f_s\in \mathcal C_s$,  $f_c\in \mathcal C_c$, $f_l\in \mathcal C_l$ and $t_0>1$ large enough, so that $\lceil t-\ln t\rceil \leq \lfloor t \rfloor - 1$ for all $t\geq t_0$.
	For any random variable $Y$ with finite mean, we define $\mathcal I_r^t Y:= \mathcal I_r^t [Y, \mu] := \mathbb P_\mu[Y|\mathscr F_t] - \mathbb P_\mu[Y|\mathscr F_r]$ where $0 \leq r \leq t <\infty.$
	We will use the shorter notation $\mathcal I_r^t Y$ when there is no danger of confusion.
For each $t\geq t_0$, we have the following decomposition.
  \begin{multline}
    \label{eq:PM:CLTS:1}
     \|X_t\|^{\tilde \beta - 1}  X_t(f_s)
     = I^{f_s}_1(t) + I^{f_s}_2(t) + I^{f_s}_3(t)
    := \Big(\sum_{k=0}^{\lfloor t-\ln t \rfloor}  \|X_t\|^{\tilde \beta - 1} \mathcal I_{t-k-1}^{t-k} X_t(f_s) \Big)\\
    + \Big(  \|X_t\|^{\tilde \beta - 1} \mathcal I_0^{t-\lfloor t \rfloor} X_t(f_s)   + \sum_{k=\lfloor t-\ln t \rfloor+1}^{\lfloor t \rfloor-1} \|X_t\|^{\tilde \beta - 1} \mathcal I_{t-k-1}^{t-k} X_t(f_s) \Big) + ( \|X_t\|^{\tilde \beta - 1}X_0(P_t^\alpha f_s) ),
  \end{multline}
  \begin{multline}
    \label{eq:PM:CLTS:2}
     \|tX_t\|^{\tilde \beta - 1}  X_t(f_c)
     = I^{f_c}_1(t) + I^{f_c}_2(t) + I^{f_c}_3(t)
    := \Big(\sum_{k=0}^{\lfloor t-\ln t \rfloor}  \|tX_t\|^{\tilde \beta - 1} \mathcal I_{t-k-1}^{t-k} X_t(f_c) \Big)\\
    + \Big(  \|tX_t\|^{\tilde \beta - 1} \mathcal I_0^{t-\lfloor t \rfloor} X_t(f_c)   + \sum_{k=\lfloor t-\ln t \rfloor+1}^{\lfloor t \rfloor-1} \|tX_t\|^{\tilde \beta - 1} \mathcal I_{t-k-1}^{t-k} X_t(f_c) \Big) + ( \|tX_t\|^{\tilde \beta - 1}X_0(P_t^\alpha f_c) ),
  \end{multline}
and
 \begin{align}
    & \frac{X_t(f_l) - \sum_{p\in \mathbb Z_+^d: \alpha \tilde \beta \geq |p|b} \langle f_l,\phi_p\rangle_\varphi e^{(\alpha - |p|b)t}H_\infty^p}{\|X_t\|^{1- \tilde \beta}}
      = \sum_{p\in \mathcal N}\frac{ \langle f_l,\phi_p\rangle_\varphi [X_t(\phi_p) - e^{(\alpha - |p|b)t}H_\infty^p]}{\|X_t\|^{1- \tilde \beta}}
    \\& = \sum_{p \in \mathcal N} \frac{\langle f_l,\phi_p\rangle_\varphi e^{(\alpha - |p|b)t}(H_t^p - H_\infty^p)}{\|X_t\|^{1- \tilde \beta}}
    = \sum_{k=0}^\infty \sum_{p \in \mathcal N}  \langle f_l,\phi_p\rangle_\varphi e^{(\alpha - |p|b)t}\frac{ H_{t+k}^p - H_{t+k+1}^p}{\|X_t\|^{1- \tilde \beta}}
    \\ &= \Big(\sum_{k = 0}^{\lfloor t^2 \rfloor}  +\sum_{k = \lfloor t^2 \rfloor+1}^\infty \Big)\Big(\sum_{p \in \mathcal N}  \langle f_l,\phi_p\rangle_\varphi e^{(\alpha - |p|b)t}\frac{ H_{t+k}^p - H_{t+k+1}^p}{\|X_t\|^{1- \tilde \beta}}\Big)
         = : I^{f_l}_1(t) + I^{f_l}_2(t),
  \end{align}
where $\mathcal N:= \{p\in \mathbb Z_+^d: \alpha \tilde \beta > |p|b\}$.
%Now for each $t\geq t_0$ and $i=1,2,3$, let
Define $I_3^{f_l}(t)=0$, $t\geq t_0$. For all $t\geq t_0$ and $j=1,2,3$, let
\(
%   U_i(t):=\big(I_i^{f_s}(t),I_i^{f_c}(t),I_i^{f_l}(t)\big),
%we should avoid using  as an index
   R_j(t):=\big(I_j^{f_s}(t),I_j^{f_c}(t),I_j^{f_l}(t)\big)
\)
%where we write $I_3^{f_l}(t)=0$, $t\geq t_0$. and Let
and
\begin{align}
    R_0(t)=(I_0^{f_s}(t),I_0^{f_c}(t),I_0^{f_l}(t)):=\big(\sum_{k=0}^{\lfloor t-\ln t \rfloor} \Upsilon_{t-k-1}^{T_k \tilde f_s},t^{\tilde \beta - 1}\sum_{k=0}^{\lfloor t-\ln t \rfloor} \Upsilon_{t-k-1}^{T_{k} \tilde f_c},\sum_{k = 0}^{\lfloor t^2 \rfloor} \Upsilon_{t+k}^{- T_k \tilde f_l}\big),
    %\quad t\geq t_0,
\end{align}
where $\tilde f_s:=e^{\alpha(\tilde \beta - 1)} f_s$, $\tilde f_c:=e^{\alpha(\tilde \beta - 1)} f_c$ and  $\tilde f_l := \sum_{p\in \mathcal N} e^{-(\alpha - |p|b)}\langle f_l, \phi_p \rangle_\varphi \phi_p$.

Define
\begin{align}
R(t):=\Bigg(e^{-\alpha t}\|X_t\|, \frac{X_t(f_s)}{\|X_t\|^{\frac{1}{1+\beta}}},\frac{X_t(f_c)}{\|tX_t\|^{\frac{1}{1+\beta}}},\frac{ X_t(f_l) - \sum_{p\in \mathbb Z^d_+:\alpha \tilde \beta>|p|b}\langle f_l,\phi_p\rangle_\varphi e^{(\alpha-|p|b)t}H^p_{\infty}}{\|X_t\|^{\frac{1}{1+\beta}}}\Bigg).
\end{align}
The following result is a special case of Theorem \ref{thm:M}.
%Having set this notation, we can establish the following CLT.
\begin{thm}\label{thm: II}
%Under $\mathbb P_{\mu}(\cdot|D^c)$,
Under $\mathbb{\widetilde{P}}_{\mu}(\cdot)$,
%\begin{align}
% U(t)&:=\Big( \|X_t\|^{\tilde{\beta}-1}X_t(f_s), \|tX_t\|^{\tilde{\beta}-1}X_t(f_c),\\
%& \|X_t\|^{\tilde{\beta}-1}(X_t(f_l)-\sum_{p\in \mathbb Z_+^d:\alpha\tilde{\beta}>|p|b}\langle f_l,\phi_p\rangle_{\varphi}e^{(\alpha-|p|b)t}H_{\infty}^p \Big)\xrightarrow[t\rightarrow\infty]{d}(\zeta^{f_s},\zeta^{f_c},\zeta^{-f_l}),
%\end{align}
\begin{align}
R(t) \xrightarrow[t\rightarrow\infty]{d}(\zeta^{f_s},\zeta^{f_c},\zeta^{-f_l}),
\end{align}
where $\zeta^{f_s},\zeta^{f_c}$ and $\zeta^{-f_l}$ are independent $(1+\beta)$-stable random variables with characteristic functions described by \eqref{eq: charac func11}.
\end{thm}
\begin{proof}

Note that for each $t>t_0$,
\[
R(t)=R_0(t)+(R_1(t)-R_0(t))+R_2(t)+R_3(t).
\]
% Then the outline of Theorem's proof follows:
\begin{itemize}
\item[(1)] In Lemma \ref{lem: U0T}, we will show that, under $\mathbb{\widetilde{P}}_{\mu}(\cdot)$, $R_0(t) \xrightarrow[t\to \infty]{d}(\zeta^{f_s},\zeta^{f_c},\zeta^{-f_l})$, where $\zeta^{f_s},\zeta^{f_c}$ and $\zeta^{-f_l}$ are independent $(1+\beta)$-stable random variables with characteristic function described by \eqref{eq: charac func11}.
\item[(2)] In Lemma \ref{lem: U1-U0T}, we will show that, under $\mathbb{\widetilde{P}}_{\mu}(\cdot)$, $R_1(t)-R_0(t)\xrightarrow[t\to \infty]{d}0$.
\item[(3)] In Lemma \ref{lem: U2T}, we will show that, under $\mathbb{\widetilde{P}}_{\mu}(\cdot)$, $R_2(t)\xrightarrow[t\to \infty]{d}0$.
\item[(4)] In Lemma \ref{lem: U3T}, we will show that, under $\mathbb{\widetilde{P}}_{\mu}(\cdot)$, $R_3(t)\xrightarrow[t\to \infty]{d}0$.
\end{itemize}
%which will be given by Lemma \ref{lem: U0T}-Lemma \ref{lem: U3T}. Therefore, we get the desired result.
Combining these, we immediately get the conclusion of the theorem.
\end{proof}


\begin{lem}\label{lem: U0T}
% Under $\mathbb P_{\mu}(\cdot |D^c)$,
Under $\mathbb{\widetilde{P}}_{\mu}(\cdot)$,
 $R_0(t) \xrightarrow[t\to \infty]{d}(\zeta^{f_s},\zeta^{f_c},\zeta^{-f_l})$, where $\zeta^{f_s},\zeta^{f_c}$ and $\zeta^{-f_l}$ are independent $(1+\beta)$-stable random variables with characteristic function described by \eqref{eq: charac func11}.
\end{lem}
\begin{proof}
%In fact, for each $t>0$, $\Upsilon_t^f$ is linear in $f$, this means
Since $\Upsilon_t^f$ is linear in $f$, we have
\begin{align}
&\Upsilon_t^{\theta f} = \theta \Upsilon_t^f, \quad \theta \in \mathbb R,\\
&\Upsilon_t^{f+g} = \Upsilon_t^f+\Upsilon_t^g,\quad f,g \in \mathcal P.
\end{align}
Therefore, for each $t>t_0$,
\[
\widetilde{\mathbb P}_{\mu}\Big[\exp\Big(i
%\sum_{i=s,c,l}I_0^{f_i}(t)\Big)\Big]
\sum_{j=s,c,l}I_0^{f_j}(t)\Big)\Big]
= \widetilde{\mathbb P}_{\mu}\Big[\exp\Big(i\sum_{k=0}^{\lfloor t-\ln t \rfloor}(\Upsilon_{t-k-1}^{(T_k(\tilde{f_s}+t^{\tilde{\beta}-1} \tilde{f}_c))})\Big)\exp\Big(i\sum_{k=0}^{\lfloor t^2 \rfloor}\Upsilon_{t+k}^{-T_k\tilde{f}_l}\Big)\Big].
\]
Using Corollary \ref{cor:MI} with $f=\tilde{f_s}+t^{\tilde{\beta}-1} \tilde{f}_c$ and $g = -\tilde{f}_l$, we get that there exist $C_1,\delta_1 > 0$ such that
  \begin{align}
    &\Big|\widetilde{\mathbb P}_{\mu}\Big[\exp\Big(i
%    \sum_{i=s,c,l}I_0^{f_i}(t)\Big)\Big]
    \sum_{j=s,c,l}I_0^{f_j}(t)\Big)\Big]
    -\exp\Big(\sum_{k=0}^{\lfloor t-\ln t \rfloor} \langle Z_1(T_{k}(\tilde f_s+t^{\tilde{\beta}-1}\tilde{f}_c)), \varphi\rangle \Big)\exp\Big(\sum_{k=0}^{\lfloor t^2 \rfloor}\langle Z_1(-T_k\tilde{f}_l),\varphi\rangle\Big)\Big|\\
    &\leq C_1 e^{-\delta_1(t - \lfloor t - \ln t\rfloor)},
    \quad t\geq t_0.
  \end{align}
Next, we claim that
\begin{align}
\lim_{t\rightarrow\infty}\exp\Big(\sum_{k=0}^{\lfloor t-\ln t \rfloor} \langle Z_1(T_{k}(\tilde f_s+t^{\tilde{\beta}-1}\tilde{f}_c)), \varphi\rangle \Big)\exp\Big(\sum_{k=0}^{\lfloor t^2 \rfloor}\langle Z_1(-T_k\tilde{f}_l),\varphi\rangle\Big) =
%\prod_{i=s,c,l}\exp(m[f_i]).
\prod_{j=s,c,l}\exp(m[f_j]).
\end{align}
 Then we have
% $\mathbb{\widetilde{P}}_{\mu} [e^{i \sum_{i=s,c,l}I^{f^i}_0(t)} ]
 $\mathbb{\widetilde{P}}_{\mu} [e^{i \sum_{j=s,c,l}I^{f^j}_0(t)} ]
% \xrightarrow[t\to \infty]{}  \prod_{i=s,c,l}\exp(m[f_i])$.
 \xrightarrow[t\to \infty]{}  \prod_{j=s,c,l}\exp(m[f_j])$.
% For $i=s,c,l$, since $I_0^{f_i}(t)$ is linear in $f_i$, we can replace $f_i$ with $\theta_i f_i$, $\theta_i \in \mathbb R$, and then we get the desired result.
Since $I_0^{f_j}(t)$, $j=s,c,l$, is linear in $f_j$, replacing $f_j$ with $\theta_j f_j$, we immediately get the desired result.

Now we prove the above claim.
 Recall that  $\tilde f_l := \sum_{p\in \mathcal N} e^{-(\alpha - |p|b)}\langle f_l, \phi_p \rangle_\varphi \phi_p$.
  Using \eqref{eq:I:R:2} and the fact that $\varphi(x)dx$ is the invariant distribution of the semigroup $(P_t)_{t\geq 0}$, we get that for each $n\in \mathbb Z_+$,
  \begin{align}
    \label{eq:PM:CLTS:2a}
    & \sum_{k=0}^n \langle Z_1 T_{k} (-\tilde f_l), \varphi \rangle
      = \sum_{k=0}^n \int_0^1 \langle P_u^\alpha ((iP_{1 - u}^\alpha T_k \tilde f_l)^{1+\beta}), \varphi\rangle ~du
    \\& = \sum_{k=0}^n \int_0^1 e^{\alpha u} \langle  (iP_{1 - u}^\alpha T_{k}\tilde f_l)^{1+\beta}, \varphi \rangle ~du
    \\& = \sum_{k=0}^n \int_0^1 \langle  (iT_{k+ u} f_l)^{1+\beta}, \varphi\rangle~du
    = \int_0^{n+1} \langle  (iT_{u} f_l)^{1+\beta}, \varphi\rangle~du = m_{n+1}[-f_l].
  \end{align}
Therefore, we have
\[
\lim_{t\rightarrow \infty}\exp\Big(\sum_{k=0}^{\lfloor t^2 \rfloor}\langle Z_1(-T_k\tilde{f}_l),\varphi\rangle\Big) = \lim_{t\rightarrow \infty}\exp(m_{\lfloor t^2 \rfloor+1}[-f])=\exp(m[-f]).
\]
Similarly, for each $f\in \mathcal C_s  \oplus \mathcal C_c$ and $n\in \mathbb Z_+$.
  \begin{align}
%    \label{eq:PM:CLTS:2}
    & \sum_{k=0}^n \langle Z_1 T_{k} \tilde f, \varphi \rangle
    = \sum_{k=0}^n \int_0^1 \langle P_u^\alpha ((-iP_{1 - u}^\alpha T_k \tilde f)^{1+\beta}), \varphi\rangle ~du
    \\& = \sum_{k=0}^n \int_0^1 e^{\alpha u} \langle  (-iP_{1 - u}^\alpha T_{k}\tilde f)^{1+\beta}, \varphi \rangle ~du
    \\& = \sum_{k=0}^n \int_0^1 \langle  (-iT_{k+1 - u} f)^{1+\beta}, \varphi\rangle~du
    = \int_0^{n+1} \langle  (-iT_{u} f)^{1+\beta}, \varphi\rangle~du = m_{n+1}[f],
  \end{align}
where $\tilde f=e^{\alpha(\tilde \beta - 1)} f$ .
Therefore, for any $t\geq t_0$,
\[
\exp\Big(\sum_{k=0}^{\lfloor t-\ln t \rfloor} \langle Z_1(T_{k}(\tilde f_s+t^{\tilde{\beta}-1}\tilde{f}_c)), \varphi\rangle \Big)=\int_0^{\lfloor t-\ln t \rfloor+1}\langle (-i(T_u(f_s+t^{\tilde{\beta}-1}f_c)))^{1+\beta},\varphi\rangle du.
\]
Note that for each $u>0$, $T_uf_c=f_c$. According to Step 1 in the proof
of \cite[Lemma 2.6]{RenSongSunZhao2019Stable}, for each $f\in \mathcal P$, there exist a constant $\delta\geq 0$ and $h\in \mathcal P$
 such that
 %for all $u\geq 0$, we have $|T_uf|\leq e^{-\delta u}h$, especially, when $f\in \mathcal C_s$, $\delta$ can be chosen strictly positive.
 $|T_uf|\leq e^{-\delta u}h$ for all $u\geq 0$. When $f\in \mathcal C_s$, $\delta$, we can take
 $\delta$ to be  strictly positive.
%Therefore, according to  Lemma \ref{lem:m} and
Combining this with
Lemma \ref{ineq: analysis}, we get that there exist constant $C_1,\delta>0$ and $h\in \mathcal P$ such that for all $u>0$,
\begin{align}
  &|(-i(T_uf_s+t^{\tilde{\beta}-1}T_uf_c))^{1+\beta}-(-iT_uf_s)^{1+\beta}-(-it^{\tilde{\beta}-1}T_uf_c)^{1+\beta}|\\
&\leq C_1(t^{-\frac{\beta}{1+\beta}}|T_uf_s||T_uf_c|^{\beta}+t^{-\frac{1}{1+\beta}}|T_uf_s|^{\beta}|T_uf_c|)\\
&\leq C_1(t^{-\frac{\beta}{1+\beta}}e^{-\delta u}h|f_c|^{\beta}+t^{-\frac{1}{1+\beta}}e^{-\delta\beta u}h^{\beta}|f_c|),
\end{align}
which means that
\begin{align}
&\Big|\exp\Big(\sum_{k=0}^{\lfloor t-\ln t \rfloor} \langle Z_1(T_{k}(\tilde f_s+t^{\tilde{\beta}-1}\tilde{f}_c)), \varphi\rangle \Big)-m_{\lfloor t-\ln t \rfloor+1}[f_s]-\frac{1}{t}m_{\lfloor t-\ln t \rfloor+1}[f_c]\Big|\\
&\leq C_1\int_0^{\lfloor t-\ln t \rfloor+1}(t^{-\frac{\beta}{1+\beta}}e^{-\delta u}h|f_c|^{\beta}+t^{-\frac{1}{1+\beta}}e^{-\delta\beta u}h^{\beta}|f_c|)du\xrightarrow{t\rightarrow \infty} 0.
\end{align}
Combining this with \eqref{eq:I:R:3}, we get that
\[
\lim_{t\rightarrow \infty}\exp\Big(\sum_{k=0}^{\lfloor t-\ln t \rfloor} \langle Z_1(T_{k}(\tilde f_s+t^{\tilde{\beta}-1}\tilde{f}_c)), \varphi\rangle \Big) = m[f_s]+m[f_c].
\]
%Now we complete the proof of claim.
Thus the claim is valid.
\end{proof}
\begin{lem}\label{lem: U1-U0T}
% Under $\mathbb P_{\mu}(\cdot |D^c)$,
Under $\mathbb{\widetilde{P}}_{\mu}(\cdot)$,
 $R_1(t)-R_0(t)\xrightarrow[t\to \infty]{d}0$.
\end{lem}
\begin{proof}
%In fact, by \cite[Lemma 3.4.3]{Durrett2010Probability} we have, for each $t\geq t_0$, that
It follows from \cite[Lemma 3.4.3]{Durrett2010Probability} that for $t\geq t_0$,
\[
\Big|\widetilde {\mathbb P}_{\mu}\Big[\exp\Big(i
%\sum_{i=s,c,l}(I_{1}^{f_i}(t) - I_0^{f_i}(t))\Big)
\sum_{j=s,c,l}(I_{1}^{f_j}(t) - I_0^{f_j}(t))\Big)
 - 1\Big]\Big| \leq  \sum_{k=0}^{\lfloor t-\ln t \rfloor}(\widetilde {\mathbb {P}}_\mu[|Y^s_{t,k}|] + \widetilde {\mathbb {P}}_\mu[|Y^c_{t,k}|] )+ \sum_{k=0}^{\lfloor t^2 \rfloor} \widetilde {\mathbb {P}}_\mu[|Y^l_{t,k}|]
\]
 where
 \begin{align}
 &Y^s_{t,k} := \exp\Big(i(\|X_t\|^{\tilde{\beta}-1}\mathcal I_{t-k-1}^{t-k}X_t(f_s) - \Upsilon_{t-k-1}^{T_{k}\widetilde {f_s}})\Big) - 1, \\
 &Y^c_{t,k} := \exp\Big(i(\|tX_t\|^{\tilde{\beta}-1}\mathcal I_{t-k-1}^{t-k}X_t(f_c) - \Upsilon_{t-k-1}^{T_{k}\widetilde {f_c}})\Big) - 1,\\
&Y^l_{t,k} := \exp\Big(i(\widetilde{\Upsilon}_{t,k} - \Upsilon_{t+k}^{-T_{k}\widetilde {f_l}})\Big) - 1,\\
&\widetilde{\Upsilon}_{t,k}=\|X_t\|^{\tilde{\beta}-1}\Big( \sum_{p\in \mathcal N}\langle f_l,\phi_p\rangle_{\varphi}e^{(\alpha-|p|b)t}(H_{t+k}^p-H_{t+k+1}^p)\Big).
\end{align}
  We claim that there exist $C_1, \delta_1>0$ such that
\begin{align}
% \widetilde {\mathbb {P}}_\mu[|Y^i_{t,k}|]
 \widetilde {\mathbb {P}}_\mu[|Y^j_{t,k}|]
 \leq C_1e^{-\delta_1 (t-k)}, \quad  t\geq t_0, k \in \mathbb Z_+ \text{~satisfying~} k\leq t-1,
%  i=s,c,
  j=s,c,
\end{align}
and
\[
 \widetilde {\mathbb {P}}_\mu[|Y^l_{t,k}|] \leq C_1e^{-\delta_1 t}, \quad t\geq t_0, k \in \mathbb Z_+.
\]
  Then there exists $C_2>0$ such that for all $t>t_0$, we have
%  $\|\widetilde {\mathbb P}_{\mu}[\exp(i\sum_{i=s,c,l}(I_{1}^{f_i}(t) - I_0^{f_i}(t))) - 1]| \leq C_2(t^{-\delta_1}+(1+t^2)e^{-\delta_1 t})$
$$
\big|\widetilde {\mathbb P}_{\mu}\left[\exp\left(i\sum_{j=s,c,l}(I_{1}^{f_j}(t) - I_0^{f_j}(t))\right) - 1\right]\big| \leq C_2(t^{-\delta_1}+(1+t^2)e^{-\delta_1 t})
$$
which, combined with the fact that
%for $i=s,c,l$, $I^{f_i}_1(t) - I^{f_i}_0(t)$  is linear in $f_i$, completes this Lemma.
$I^{f_j}_1(t) - I^{f_j}_0(t)$, $j=s,c,l$, is linear in $f_j$, completes the proof of this lemma.

  We now show the claim above in the following steps.
  First we choose $\gamma\in(0, \beta)$ close enough to $\beta$ so that $\alpha \tilde \gamma > |p|b$ for each $p\in \mathcal N$,
  and that there exist $\eta,\eta'>0$ satisfying $\alpha \tilde \gamma > \eta>\eta - 3\eta'> \alpha (\tilde \beta - \tilde \gamma)>0$.
  %For all $0<s<t$, we define
  %We should avoid using s in this proof.
  For all $0<r<t$, we define
\[
%\mathcal{D}_{t,s} :=\{|H_t-H_{s}|\leq  e^{-\eta s}, H_{s}> 2e^{-\eta' s}\}.
\mathcal{D}_{t,r} :=\{|H_t-H_{r}|\leq  e^{-\eta r}, H_{r}> 2e^{-\eta' r}\}.
\]
 \emph{Step 1.} We show that there exist $C_{1.1},\delta_{1.1} >0$ such that
% for all  $0<s<t$, $ \mathbb{\widetilde{P}}_{\mu} \big[ \mathcal{D}^c_{t,s} \big] \leq C_{1.1} e^{-\delta_{1.1} s}.$
$\mathbb{\widetilde{P}}_{\mu} \big[ \mathcal{D}^c_{t,r} \big] \leq C_{1.1} e^{-\delta_{1.1} r}$ for all $0<r<t$.


 It follows from \cite[Proposition 2.10 \& Lemma 3.3 with $|p|=0$]{RenSongSunZhao2019Stable} and Chebyshev's inequality that there exist $C_{1.1}', \delta_{1.1}'>0$ such that for all
  $0<r<t$,
  \begin{align}
    & \mathbb{\widetilde{P}}_{\mu}(\mathcal{D}_{t,r}^c)
    \leq \mathbb{\widetilde{P}}_{\mu}(|H_t-H_{r}| > e^{-\eta r})+\mathbb{\widetilde{P}}_{\mu}(H_{r}\leq 2e^{-\eta'r}) \\
    & \leq \mathbb{P}_{\mu}(D^c)^{-1}e^{\eta r}\mathbb{P}_{\mu}[|H_t-H_r|] +  \mathbb{P}_{\mu}(D^c)^{-1} \mathbb P_\mu(H_r\leq 2e^{-\eta'r}; D^c) \\
    & \leq \mathbb{P}_{\mu}(D^c)^{-1}  e^{\eta r}\|H_t - H_r\|_{\mathbb P_\mu; 1+\gamma} + \mathbb{P}_{\mu}(D^c)^{-1} \mathbb P_\mu(0<H_r\leq 2e^{-\eta'r}) \\
    & \leq C'_{1.1} e^{-(\alpha \tilde \gamma - \eta)r}+C'_{1.1} e^{-\delta'_{1.1}r}.
  \end{align}
  \emph{Step 2.} We show that there exist $C_{2.1},\delta_{2.1} > 0$ such that for all $t>t_0$ and $k\in \mathbb Z_+$ satisfying $k\leq t-1$, it holds that $ \mathbb{\widetilde{P}}_{\mu} [|Y^s_{t,k}|] \leq  C_{2.1} e^{-\delta_{2.1} (t-k)}$.

  Note that for $f\in \mathcal C_s$ and $k\in \mathbb Z_+$, we have $T_kf = e^{\alpha (\tilde \beta - 1 )k}P_k^\alpha f $. Therefore for all $t>t_0$ and $k\in \mathbb Z_+$ satisfying $k\leq t-1$,
  \begin{align}
    \label{eq:gammafunction11}
    \Upsilon_{t-k-1}^{T_{k} \tilde f_s}
    = \frac{X_{t-k}(T_{k} \tilde  f_s) - X_{t -k-1}(P_1^\alpha T_{k} \tilde f_s)}{\|X_{t-k-1}\|^{1-\tilde \beta}}
    = \frac{\mathcal I_{t - k - 1}^{t - k} X_t(f_s)}{\|e^{\alpha (k+1)}X_{t-k-1} \|^{1 -\tilde \beta}}.
  \end{align}
  Since $|e^{ix}-e^{iy}|\leq|x-y|$ for all $x,y\in \mathbb R$, we have  for all $t>t_0$ and $k\in \mathbb Z_+$ satisfying $k\leq t-1$,
  \begin{align}
    \label{eq: control of Ykt}
    \mathbb{\widetilde{P}}_{\mu}[|Y^s_{t,k}|;\mathcal{D}_{t,t-k-1}]
    & \leq \mathbb{\widetilde{P}}_{\mu}\Big[|\mathcal I_{t-k-1}^{t-k} X_t(f_s) | \cdot \Big| \| e^{\alpha(k+1)}X_{t-k-1}\| ^{ \tilde \beta - 1} - \|X_t\|^{ \tilde \beta - 1}\Big|; \mathcal D_{t,t-k-1}\Big] \\
    & \leq  e^{\alpha(\tilde \beta - 1)t}\mathbb{\widetilde{P}}_{\mu}\big[|\mathcal I_{t-k-1}^{t-k}X_t(f_s)|\cdot K_{t,k}\big],
  \end{align}
  where
  \[
    K_{t,k}
%    := \Big| \frac {H_t^{1- \tilde \beta} - H_{t-k-1}^{1 - \tilde \beta}} {H_t^{1 - \tilde \beta} %H_{t-k-1}^{ 1- \tilde \beta }} \Big| \mathbf{1}_{\mathcal{D}_{t,t-k-1}}.
   := \Big| \frac {(H_t)^{1- \tilde \beta} - (H_{t-k-1})^{1 - \tilde \beta}} {(H_t)^{1 - \tilde \beta}
   (H_{t-k-1})^{ 1- \tilde \beta }} \Big| \mathbf{1}_{\mathcal{D}_{t,t-k-1}}.
  \]
  Note that, since $\eta' < \eta$, we have almost surely on $\mathcal D_{t,t-k-1}$,
  \begin{align}
    H_t
    & \geq H_{t-k-1}- e^{-\eta (t-k-1)}
      \geq 2e^{-\eta'(t-k-1)}-e^{-\eta(t-k-1)}
      \geq e^{-\eta'(t-k-1)}.
  \end{align}
  Therefore, for all $t>t_0$ and $k\in \mathbb Z_+$ satisfying $k\leq t-1$, almost surely  on $\mathcal D_{t,t-k-1}$,
  \begin{align}
%    & \Big|H_t^{1- \tilde \beta}-H_{t-k-1}^{1- \tilde \beta}\Big|
%      \leq (1- \tilde \beta) \max \{ H_t^{-\tilde \beta }, H_{t-k-1}^{ -\tilde \beta} \} | H_t - H_{t-k-1} | \\
   & \Big|(H_t)^{1- \tilde \beta}-(H_{t-k-1})^{1- \tilde \beta}\Big|
      \leq (1- \tilde \beta) \max \{ (H_t)^{-\tilde \beta }, (H_{t-k-1})^{ -\tilde \beta} \} | H_t - H_{t-k-1} | \\
    & \leq (1- \tilde \beta ) \max\{e^{\eta' (t-k-1)}, \frac{1}{2}e^{\eta'(t-k-1)}\}^{\tilde \beta} e^{-\eta(t-k-1)}  \leq (1- \tilde \beta) e^{-(\eta - \eta') (t-k-1)}
  \end{align}
%  and $ |H_t^{1 - \tilde \beta} H_{t-k-1}^{ 1 - \tilde \beta}| \geq 2^{\frac{1}{1+\beta}} e^{-2\eta'(t-k-1)}$.
  and $ |(H_t)^{1 - \tilde \beta} (H_{t-k-1})^{ 1 - \tilde \beta}| \geq 2^{\frac{1}{1+\beta}} e^{-2\eta'(t-k-1)}$.
  Thus, there exists $C_{2.1}'> 0$ such that for all $t>t_0$ and $k\in \mathbb Z_+$ satisfying $k\leq t-1$, almost surely
  \begin{align}
    K_{t,k}
    \leq C_{2.1}' e^{-(\eta - 3\eta')(t-k-1)}.
  \end{align}
  Now, by \cite[Lemma 2.13]{RenSongSunZhao2019Stable}, there exists $C''_{2.1}>0$ such that for all $t>t_0$ and $k\in \mathbb Z_+$ satisfying $k\leq t-1$,
  \begin{align}
    & \mathbb{\widetilde{P}}_{\mu} [|Y^s_{t,k}| ; \mathcal{D}_{t,t-k-1} ]
    \leq C_{2.1}' e^{\alpha (\tilde \beta - 1)t} \mathbb{\widetilde{P}}_{\mu} [ | \mathcal{I}_{t-k-1}^{t-k}X_t(f_s)| ] e^{-(\eta - 3\eta')(t-k-1)} \\
    & \leq \frac{C_{2.1}' } {\mathbb{P}_{\mu}(D^c)} e^{ \alpha (\tilde \beta - 1)t} \|\mathcal{I}_{t-k-1}^{t-k} X_t(f_s)\|_{\mathbb P_\mu; 1+\gamma} e^{-(\eta - 3\eta')(t-k - 1)} \\
    & \leq C_{2.1}'' e^{\alpha(\tilde \beta - \tilde \gamma)t} e^{ (\alpha \tilde \gamma - \kappa_f b)k} e^{-(\eta - 3\eta')(t-k)}
     \leq C_{2.1}'' e^{\alpha(\tilde \beta - \tilde \gamma)(t-k)} e^{-(\eta - 3\eta')(t-k)}.
  \end{align}
  In the last inequality, we used the fact that $\alpha \tilde \beta < \kappa_{f_s} b$. Combining the fact that for all $t>t_0$ and $k\in \mathbb Z_+$ satisfying $k\leq t-1$, $\mathbb{\widetilde{P}}_{\mu} [|Y^s_{t,k}| ; \mathcal{D}^c_{t,t-k-1} ]\leq 2\mathbb{\widetilde{P}}_{\mu} [ \mathcal{D}^c_{t,t-k-1} ]\leq 2 C_{1,1}e^{-\delta_{1.1}(t-k-1)}$, we get the desired result.

 \emph{Step 3.} We show that there exist $C_{3.1},\delta_{3.1} > 0$ such that for all $t>t_0$ and $k\in \mathbb Z_+$ satisfying $k\leq t-1$, $ \mathbb{\widetilde{P}}_{\mu}[|Y^c_{t,k}|] \leq  C_{3.1} e^{-\delta_{3.1} (t-k)}.$

  Note that for $f\in \mathcal C_c$ and $k\in \mathbb Z_+$, we have $T_kf = e^{\alpha (\tilde \beta - 1 )k}P_k^\alpha f$. Therefore for all $t>t_0$ and $k\in \mathbb Z_+$ satisfying $k\leq t-1$,
  \[
    t^{\tilde \beta - 1} \Upsilon_{t-k-1}^{T_{k} \tilde f_c}
    = \frac{X_{t-k}(T_{k} \tilde f_c) - X_{t -k-1}(P_1^\alpha T_{k} \tilde f_c)}{\|t X_{t-k-1}\|^{1-\tilde \beta}}
    = \frac{\mathcal I_{t - k - 1}^{t - k} X_t(f_c)}{\|te^{\alpha (k+1)}X_{t-k-1} \|^{1 -\tilde \beta}}.
  \]
  The rest is similar to Step 2.
  We omit the details.

 \emph{Step 4.} We show that there exist $C_{4.1}, \delta_{4.1}>0$ such that for all $k \in \mathbb Z_+$ and $t\geq t_0$, we have $\widetilde {\mathbb {P}}_\mu[|Y^l_{t,k}|]\leq C_{4.1}e^{- \delta_{4.1} t}$.

 Note that for all $k \in \mathbb Z_+$ and $t\geq t_ 0$,
  \begin{align}
    & \Upsilon_{t+k}^{-T_k\tilde f_l}
      = \frac{X_{t+k}(P^\alpha_1T_k\tilde f_l) - X_{t+k+1}(T_k \tilde f_l)}{\|X_{t+k}\|^{1 - \tilde \beta}}
    \\& = \sum_{p\in \mathcal N}
    \langle\tilde f_l,\phi_p\rangle_\varphi e^{-(\alpha \tilde \beta - |pb|)k}\frac{ X_{t+k}(P_1^\alpha \phi_p) - X_{t+k+1}(\phi_p)}{\|X_{t+k}\|^{1 - \tilde \beta}}
    \\& = \sum_{p\in \mathcal N}
    \langle f_l,\phi_p\rangle_\varphi  e^{(\alpha  -|p|b)t}\frac{H_{t+k}^p-H_{t+k+1}^p }{\|e^{-\alpha k}X_{t+k}\|^{1 - \tilde \beta}}.
  \end{align}
  Therefore for all $k\in \mathbb Z_+$ and $t\geq t_0$,
  \begin{align}
    &|Y^l_{t,k}| \mathbf 1_{\mathcal D_{t+k,t}}
      \leq \Big( \sum_{p\in \mathcal N}|\langle f_l,\phi_p\rangle_\varphi|  e^{(\alpha  -|p|b)t} | H_{t+k}^p-H_{t+k+1}^p |\Big) \Big( \frac{1}{\|X_t\|^{1 - \tilde \beta}} - \frac{1}{\|e^{-\alpha k}X_{t+k}\|^{1 - \tilde \beta}} \Big)\mathbf 1_{\mathcal D_{t+k,t}}
    \\ &= \Big( \sum_{p\in \mathcal N}|\langle f_l,\phi_p\rangle_\varphi|  e^{(\alpha  -|p|b)t} | H_{t+k}^p-H_{t+k+1}^p |\Big)e^{\alpha (\tilde \beta - 1)t} K'_{t,k}
    \\ &= \Big( \sum_{p\in \mathcal N}|\langle f_l,\phi_p\rangle_\varphi|  e^{(\alpha \tilde \beta  -|p|b)t} | H_{t+k}^p-H_{t+k+1}^p |\Big) K'_{t,k},
  \end{align}
  where
  \[
    K'_{t,k}
%    := \Big| \frac {H_t^{1- \tilde \beta} - H_{t+k}^{1 - \tilde \beta}} {H_t^{1 - \tilde \beta} H_{t+k}^{ 1- \tilde \beta }} \Big| \mathbf{1}_{\mathcal{D}_{t+k,t}}.
        := \Big| \frac {(H_t)^{1- \tilde \beta} - (H_{t+k})^{1 - \tilde \beta}} {(H_t)^{1 - \tilde \beta} (H_{t+k})^{ 1- \tilde \beta }} \Big| \mathbf{1}_{\mathcal{D}_{t+k,t}}.
  \]
  Similar to Step 2, we can get that for all $k\in \mathbb Z_+$ and $t\geq t_0$, almost surely $K'_{t,k} \leq C_{4.1}'' e^{- (\eta - 3\eta')t}$.
  It follows from this and \cite[Lemma 3.3]{RenSongSunZhao2019Stable} that there exists $C'''_{4.1}>0$ such that for all $k\in \mathbb Z_+$ and $t\geq t_0$,
  \begin{align}
    & \widetilde{\mathbb P}_\mu[|Y^l_{t,k}|; \mathcal D_{t+k,t}]
      \leq \mathbb P_\mu(D)^{-1}\mathbb P_\mu[ |Y^l_{t,k}| ;\mathcal D_{t+k,t} ]
    \\ & \leq \mathbb P_{\mu}(D)^{-1} C_{4.1}'' e^{- (\eta - 3\eta') t}\sum_{p\in \mathcal {N}} |\langle f_l,\phi_p\rangle_\varphi|  e^{(\alpha \tilde \beta  -|p|b)t} \mathbb P_\mu[| H_{t+k}^p-H_{t+k+1}^p |]
    \\ & \leq \mathbb P_{\mu}(D)^{-1} C_{4.1}'' e^{- (\eta - 3\eta') t}\sum_{p\in \mathcal {N}} |\langle f_l,\phi_p\rangle_\varphi|  e^{(\alpha \tilde \beta  -|p|b)t} \| H_{t+k}^p-H_{t+k+1}^p \|_{\mathbb P_\mu; 1+\gamma}
    \\&\leq  \mathbb P_{\mu}(D)^{-1} C_{4.1}'' e^{- (\eta - 3\eta') t}\sum_{p\in \mathcal N} |\langle f_l,\phi_p\rangle_\varphi|  e^{(\alpha \tilde \beta  -|p|b)t} e^{-(\alpha \tilde \gamma - |p|b)(t+k)} \\
    &  \leq  C_{4.1}''' e^{- (\eta - 3\eta') t} e^{(\alpha \tilde \beta - \alpha \tilde \gamma)t}.
  \end{align}
 Combining this with the fact that, for all $t>t_0$ and $k\in \mathbb Z_+$,  $\mathbb{\widetilde{P}}_{\mu} [|Y^l_{t,k}| ; \mathcal{D}^c_{t+k,t} ]\leq 2\mathbb{\widetilde{P}}_{\mu} [ \mathcal{D}^c_{t+k,t} ]\leq 2 C_{1,1}e^{-\delta_{1.1}t}$, we get the desired result.
\end{proof}
\begin{lem}\label{lem: U2T}
%Under $\mathbb P_{\mu}(\cdot |D^c)$,
Under $\mathbb{\widetilde{P}}_{\mu}(\cdot)$,
$R_2(t)\xrightarrow[t\to \infty]{d}0$.
\end{lem}
\begin{proof}
 Define $\mathcal{E}_t:=\{\|X_t\|>t^{-1/2}e^{\alpha t}\}$. According to
 \cite[Proposition 2.10]{RenSongSunZhao2019Stable}, there exist $C_1, \delta_1>0$ such that
  \begin{align}
    \mathbb{\widetilde{P}}_{\mu}(\mathcal{E}^c_t)
    \leq \frac{1}{\mathbb{P}_{\mu}(D^c)}\mathbb{P}_{\mu}(0<e^{-\alpha t}\|X_t\|\leq t^{-1/2})\leq C_1(t^{-\delta_1 t}+e^{-\delta_1 t})
    , \quad t\geq t_0.
  \end{align}
  Therefore,
  \begin{align}
    \label{Theorem123}
    |\mathbb{\widetilde{P}}_{\mu}[e^{i
%    \sum_{i=s,c,l}I^{f_i}_2(t)}-1;\mathcal{E}^c_t]|
    \sum_{j=s,c,l}I^{f_j}_2(t)}-1;\mathcal{E}^c_t]|
    \leq 2\mathbb{\widetilde{P}}_{\mu}(\mathcal{E}^c_t)
    \leq 2C_1(t^{-\delta_1}+e^{-\delta_1 t}),
    \quad t\geq t_0.
  \end{align}
On the other hand,
\[
   |\mathbb{\widetilde{P}}_{\mu} [ (e^{i
%   \sum_{i=s,c,l}I^{f_i}_2(t)}-1);\mathcal{E}_t]|
%   \leq \sum_{i=s,c,l}\widetilde{\mathbb P}_{\mu}[|I^{f_i}_2(t)|\mathcal{E}_t],
      \sum_{j=s,c,l}I^{f_j}_2(t)}-1);\mathcal{E}_t]|
    \leq \sum_{j=s,c,l}\widetilde{\mathbb P}_{\mu}[|I^{f_j}_2(t)|\mathcal{E}_t],
\quad t\geq t_0.
\]
%We claim that for $i=s,c,l$,  $\lim_{t\rightarrow \infty}\widetilde{\mathbb P}_{\mu}[|I^{f_i}_2(t)|;\mathcal{E}_t]=0$. Then combining \eqref{Theorem123}, we get $\lim_{t\rightarrow \infty}\mathbb{\widetilde{P}}_{\mu}[e^{i \sum_{i=s,c,l}I^{f_i}_2(t)}-1] = 0$. For $i =s,c,l$, since $I_2^{f_i}(t)$ is linear in $f_i$, we can replace $f_i$ with $\theta_i f_i$, $\theta_i \in \mathbb R$, and then the desired result in this step follows.
We claim  $\lim_{t\rightarrow \infty}\widetilde{\mathbb P}_{\mu}[|I^{f_j}_2(t)|;\mathcal{E}_t]=0$, $j=s,c,l$. Then combining this with \eqref{Theorem123}, we get $\lim_{t\rightarrow \infty}\mathbb{\widetilde{P}}_{\mu}[e^{i \sum_{j=s,c,l}I^{f_j}_2(t)}-1] = 0$.
Since $I_2^{f_j}(t)$, $j=s,c,l$, is linear in $f_j$, replacing $f_j$ with $\theta_j f_j$, $\theta_j \in \mathbb R$, we immediately get the desired result in this step.

   We now show the claim above in the following steps.
Fix a $\gamma \in (0,\beta)$ close enough to $\beta$ so that $\alpha(\tilde{\beta}-\tilde{\gamma})\leq \frac{1}{2}(1-\tilde{\beta})$ and $\alpha\tilde{\gamma}>|p|b$ for all $p\in \mathcal N$.

\emph{Step 1.} We show that $\lim_{t\rightarrow \infty}\widetilde{\mathbb P}_{\mu}[|I^{f_s}_2(t)|;\mathcal{E}_t]=0$.

  Since $\kappa_{f_s} b -\alpha \tilde \gamma > \alpha (\tilde \beta - \tilde \gamma)$, we can choose $\epsilon >0$ small enough so that $q:=\kappa_{f_s}b- \alpha \tilde \gamma  > \alpha (\tilde \beta - \tilde \gamma) + \epsilon $.

	According to \cite[Lemma 2.13]{RenSongSunZhao2019Stable}, there exist $C_2',C_2''>0$ such that for each $t\geq t_0 >1$,
  \begin{align}
    & \mathbb{\widetilde{P}}_{\mu} [ |I^{f_s}_2(t)|;\mathcal{E}_t] \\
    & \leq  ( t^{-1/2}e^{\alpha t} )^{\tilde \beta - 1}\Big(\sum_{k=\lfloor t-\ln t \rfloor+1}^{\lfloor t \rfloor - 1}\mathbb{\widetilde{P}}_{\mu} [| \mathcal{I}_{t-k-1}^{t-k} X_t(f_s) |] + \mathbb{\widetilde{P}}_{\mu}[| \mathcal{I}_{0}^{t-\lfloor t\rfloor} X_t(f_s)|]\Big) \\
    & \leq  ( t^{-1/2}e^{\alpha t} )^{\tilde \beta - 1}\Big(\sum_{k=\lfloor t-\ln t \rfloor+1}^{\lfloor t \rfloor - 1}\|\mathcal{I}_{t-k-1}^{t-k} X_t(f_s) \|_{\mathbb P_\mu; 1+\gamma} + \|\mathcal I_0^{t-\lfloor t \rfloor} X_t(f_s)\|_{\mathbb P_\mu;1+\gamma}\Big) \\
    & \leq C_2't^{\frac{1-\tilde{\beta}}{2}} e^{\alpha (\tilde \beta - \tilde \gamma)t}\sum_{k=\lfloor t-\ln t \rfloor+1}^{\lfloor t \rfloor}  e^{(\alpha\tilde \gamma-\kappa_{f_s} b)k}
      \leq C_2''t^{\frac{1-\tilde{\beta}}{2}} e^{(q-\epsilon) t}\sum_{k=\lfloor t-\ln t \rfloor+1}^{\lfloor t \rfloor}  e^{-q k}
    \\ & \leq C_2'''t^{\frac{1-\tilde{\beta}}{2}} e^{q(t - \lfloor t - \ln t\rfloor-1)}e^{-\epsilon t}
         \leq C_2'''t^{\frac{1-\tilde{\beta}}{2}} t^q e^{- \epsilon t}.
  \end{align}
%  From this , we get that $ \mathbb{\widetilde{P}}_{\mu} [ |I^{f_s}_2(t)|;\mathcal{E}_t]\xrightarrow{t\rightarrow \infty} 0$ and get the desired result in this step.
The desired result in this step now follows immediately.

\emph{Step 2.} We show that $\lim_{t\rightarrow \infty}\widetilde{\mathbb P}_{\mu}[|I^{f_c}_2(t)|;\mathcal{E}_t]=0$.

It follows from \cite[Lemma 2.13]{RenSongSunZhao2019Stable} that there exist $C_3',C_3''>0$ such that for each $t\geq t_0$,
  \begin{align}
     &\mathbb{\widetilde{P}}_{\mu} [ |I^{f_c}_2(t);{\mathcal{E}_t}] \\
    & \leq  (t^{\frac{1}{2}} e^{\alpha t} )^{\tilde \beta - 1}\Big(\sum_{k=\lfloor t-\ln t \rfloor+1}^{\lfloor t \rfloor - 1}\mathbb{\widetilde{P}}_{\mu} [| \mathcal{I}_{t-k-1}^{t-k} X_t(f_c) |] + \mathbb{\widetilde{P}}_{\mu}[| \mathcal{I}_{0}^{t-\lfloor t\rfloor} X_t(f_c)|]\Big) \\
    & \leq C_3' t^{\frac{1}{2}(\tilde \beta - 1)} e^{\alpha(\tilde \beta - 1)t}\Big(\sum_{k=\lfloor t-\ln t \rfloor+1}^{\lfloor t \rfloor - 1}\|\mathcal{I}_{t-k-1}^{t-k} X_t(f_c) \|_{\mathbb P_\mu; 1+\gamma} + \|\mathcal I_0^{t-\lfloor t \rfloor} X_t(f_c)\|_{\mathbb P_\mu;1+\gamma}\Big) \\
    & \leq C_3' t^{\frac{1}{2}(\tilde \beta - 1)} e^{\alpha (\tilde \beta - \tilde \gamma)t}\sum_{k=\lfloor t-\ln t \rfloor+1}^{\lfloor t \rfloor}  e^{(\alpha\tilde \gamma-\kappa_{f_c} b)k}
      = C_3' t^{\frac{1}{2}(\tilde \beta - 1)} e^{\alpha(\tilde \beta - \tilde \gamma) t}\sum_{k=\lfloor t-\ln t \rfloor+1}^{\lfloor t \rfloor}  e^{-\alpha (\tilde \beta -\tilde \gamma) k}
    \\ & \leq C_3'' t^{\frac{1}{2}(\tilde \beta - 1)} e^{\alpha (\tilde \beta - \tilde \gamma)(t - \lfloor t - \ln t\rfloor-1)}
         \leq C_3'' t^{\frac{1}{2}(\tilde \beta - 1)} t^{\alpha (\tilde \beta - \tilde \gamma)}.
  \end{align}
  From this, we immediately get the desired result in this step.

\emph{Step 3.} We will show that $\lim_{t\rightarrow \infty}\widetilde{\mathbb P}_{\mu}[|I^{f_l}_2(t)|;\mathcal{E}_t]=0$.


 It follows from \cite[Lemma 3.3]{RenSongSunZhao2019Stable} that there exist $C_4',C_4''>0$ and $\delta_3'>0$ such that
  \begin{align}
    &\widetilde {\mathbb {P}}_\mu[ | I_{2}^{f_l}(t)|; \mathcal {E}_t]
      \leq \sum_{k = \lfloor t^2\rfloor+1}^\infty \widetilde {\mathbb {P}}_\mu[ |\widetilde {\Upsilon}_{t,k}|; \mathcal {E}_t]
    \\ & \leq \mathbb P_\mu(D^c)^{-1} \sum_{k = \lfloor t^2\rfloor+1}^\infty \sum_{p \in \mathcal N} |\langle f_l,\phi_p\rangle_\varphi| e^{(\alpha - |p|b)t}\mathbb {P}_\mu\Big[\frac{ |H_{t+k}^p - H_{t+k+1}^p|}{\|X_t\|^{1- \tilde \beta}}; \mathcal E_t\Big]
    \\ & \leq \mathbb P_\mu(D^c)^{-1} (t^{-1/2}e^{\alpha t})^{\tilde{\beta}-1} \sum_{k = \lfloor t^2\rfloor+1}^\infty \sum_{p \in \mathcal N} |\langle f_l,\phi_p\rangle_\varphi| e^{(\alpha - |p|b)t}\|H_{t+k}^p - H_{t+k+1}^p\|_{\mathbb P_\mu; 1+\gamma}
    \\ & \leq C_4' (t^{-1/2}e^{\alpha t})^{\tilde{\beta}-1} \sum_{k = \lfloor t^2\rfloor+1}^\infty \sum_{p \in \mathcal N} |\langle f_l,\phi_p\rangle_\varphi| e^{(\alpha - |p|b)t} e^{- (\alpha \tilde \gamma - |p|b)(t+k)}
    \\ & = C_4''t^{\frac{1-\tilde{\beta}}{2}} e^{ \alpha (\tilde \beta - \tilde \gamma) t }e^{- \delta'_3 t^2},
  \end{align}
 which completes the proof of this step.
\end{proof}
\begin{lem}\label{lem: U3T}
 $R_3(t) \xrightarrow[t\to \infty]{\widetilde {\mathbb P}_\mu \text{-} a.s.} 0$..
\end{lem}
\begin{proof}
\emph{Step 1.} We claim that $I^{f_s}_3(t) \xrightarrow[t\to \infty]{\widetilde {\mathbb P}_\mu \text{-} a.s.} 0$.
  In fact, if for all $\kappa \in \mathbb Z_+$ and $f\in \mathcal P$, we define
\begin{equation}
\label{eq:Q}
  	Q_\kappa f
  	:= \sup_{t\geq 0} e^{\kappa b t}|P_t f|,
  	\qquad  Q f
  	:= Q_{\kappa_f}f.
\end{equation}
	Then according to \cite[Fact 1.2]{MarksMilos2018CLT}, $Q$ is an operator from $\mathcal P$ to $\mathcal P$. Thus,  we have
  \begin{align}
    & |I^{f_s}_3(t)|
      \leq \frac{X_0(|P^\alpha_tf_s|)}{\|X_t\|^{1 - \tilde \beta }}
      \leq \frac{e^{\alpha t - \kappa_{f_s} b t}X_0(Qf_s)}{(e^{\alpha t} H_t)^{1 - \tilde \beta}}
      = e^{(\alpha \tilde \beta - k_{f_s}b)t} H_t^{\tilde \beta - 1} X_0(Qf_s)
      \xrightarrow[t\to \infty]{\widetilde {\mathbb P}_\mu \text{-} a.s.} 0.
  \end{align}

  \emph{Step 2.} Similar to Step 1, we can show that $I^{f_c}_3(t) \xrightarrow[t\to \infty]{\widetilde {\mathbb P}_\mu \text{-} a.s.} 0$.
  We omit the details.
Since $I_3^{f_l}(t)=0$, the desired result follows immediately.
\end{proof}

%\subsection{Proof of Theorem \ref{thm:M}}
%\label{proofofthm1.4}
\begin{proof}[Proof of Theorem Theorem \ref{thm:M}]
 We first recall some facts about weak convergence which will be used later. For $f:\mathbb R^d\mapsto \mathbb R$, let
 $$
 \|f\|_L:=\sup_{x\neq y}\frac{|f(x)-f(y)|}{|x-y|}
 $$
 and $\|f\|_{BL}:= \|f\|_{\infty}+\|f\|_L. $
 For any distributions $\mu_1$ and $\mu_2$ on $\mathbb R^d$, define
\[
  d(\mu_1,\mu_2):=\sup\Big\{\Big|\int fd\mu_1-\int f d\mu_2\Big|:\|f\|_{BL}\leq 1\Big\}.
\]
Then $d$ is a metric. It follows from \cite[Theorem 11.3.3]{Dudley2002} that the topology generated by $d$ is equivalent to the weak convergence topology. Using the definition, we can easily see that, if $\mu_1$ and $\mu_2$ are the distributions of two $\mathbb R^d $-valued random variables $X$ and $Y$ respectively, then
\begin{align}\label{ineq: distribution control}
  d(\mu_1,\mu_2) \leq \mathbb E|X-Y|.
\end{align}
Recall that $S(t)$ is defined in the display before Theorem \ref{thm:M}.
Now for all $f_s\in \mathcal C_s $, $f_c \in \mathcal C_c$, $f_l \in \mathcal C_l$ and
%$s,t>0$, let
%We should avoid using s. s is repalced by r without marking
$r,t>0$, let
%\begin{align}
%  U(t):=\Big(e^{-\alpha t}\|X_t\|,& \|X_t\|^{\tilde{\beta}-1}X_t(f_s), \|tX_t\|^{\tilde{\beta}-1}X_t(f_c),\\
%& \|X_t\|^{\tilde{\beta}-1}(X_t(f_l)-\sum_{p\in \mathbb Z_+^d:\alpha\tilde{\beta}>|p|b}\langle f_l,\phi_p\rangle_{\varphi}e^{(\alpha-|p|b)t}H_{\infty}^p \Big);
%\end{align}
\begin{align}
  S(t,r):=\Big(e^{-\alpha t}\|X_t\|,& \|X_{t+r}\|^{\tilde{\beta}-1}X_{t+r}(f_s), \|(t+r)X_{t+r}\|^{\tilde{\beta}-1}X_{t+r}(f_c),\\
& \|X_{t+r}\|^{\tilde{\beta}-1}(X_{t+r}(f_l)-\sum_{p\in \mathbb Z_+^d:\alpha\tilde{\beta}>|p|b}\langle f_l,\phi_p\rangle_{\varphi}e^{(\alpha-|p|b)(t+r)}H_{\infty}^p )\Big),
\end{align}
and
\begin{align}
\widetilde{S}(t,r)= \Big(e^{-\alpha (t+r)}\|X_{t+r}\|-e^{-\alpha t}\|X_t\|,0,0,0\Big).
\end{align}
Then $S(t+r)=S(t,r)+\widetilde{S}(t,r)$.

\emph{Step 1.} For each $\mu\in \mathcal M_c(\mathbb R^d)$, $f_s\in \mathcal C_s$,$f_c\in \mathcal C_c$ and $f_l\in \mathcal C_l$, under $\mathbb P_{\mu}(\cdot|D^c)$, we have
\[
S(t,r)\xrightarrow[r\rightarrow \infty]{d}\widetilde{S}(t):=\Big(e^{-\alpha t}\|X_t\|,\zeta^{f_s},\zeta^{f_c},\zeta^{-f_l}\Big),
\]
where $\zeta^{f_s},\zeta^{f_c}$ and $\zeta^{-f_l}$ are independent $(1+\beta)$-stable random variables with characteristic functions described by \eqref{eq: charac func11}. Moreover,  $\zeta^{f_s},\zeta^{f_c}$ and $\zeta^{-f_l}$ are independent of $e^{-\alpha t}\|X_t\|$.

For each $\theta,\theta_s,\theta_c,\theta_l\in \mathbb R$ and $r,t>0$, denote the characteristic function of  $S(t,r)$ under $\mathbb P_{\mu}(\cdot|D^c)$ by $k(\theta,\theta_s,\theta_c,\theta_l,r,t)$, then by the dominated convergence theorem, we have,
\begin{align}
   &k(\theta,\theta_s,\theta_c,\theta_l,r,t)\
   =\widetilde{\mathbb P}_{\mu}\Big[\exp\Big( i\theta e^{-\alpha t}\|X_t\|+A(\theta_s,\theta_c,\theta_l,r,t,\infty)\Big)\Big]\\
  &=\lim_{u\rightarrow \infty}\frac{1}{\mathbb P_{\mu}(D^c)}\mathbb P_{\mu}\Big[\exp\Big( i\theta e^{-\alpha t}\|X_t\|+A(\theta_s,\theta_c,\theta_l,r,t,u)\Big);D^c\Big],
\end{align}
where
\begin{align}
 &A(\theta_s,\theta_c,\theta_l,r,t,u)=i\theta_s\|X_{t+r}\|^{\tilde{\beta}-1}X_{t+r}(f_s)+ i\theta_c \|(t+r)X_{t+r}\|^{\tilde{\beta}-1}X_{t+r}(f_c)\\
&+i\theta_l \|X_{t+r}\|^{\tilde{\beta}-1}\Big(X_{t+r}(f_l)-\sum_{p\in \mathbb Z_+^d:\alpha\tilde{\beta}>|p|b}\langle f_l,\phi_p\rangle_{\varphi}e^{(\alpha-|p|b)(t+r)}H_{u}^p\Big),\quad r+t<u\leq \infty.
\end{align}
Note that the non-extinction event $D^c=\{\|X_r\|>0, r\geq 0\}=D_{\leq t}\cup D_{\geq t}$, where $D_{\leq t}=\{\|X_r\|>0, 0\leq r\leq t\}$ and $D_{\geq t}=\{\|X_r\|>0, r\geq t\}$. By the Markov property and the dominated convergence theorem, we get
\begin{align}
  & k(\theta,\theta_s,\theta_c,\theta_l,r,t)\\
&=\lim_{u\rightarrow \infty}\frac{1}{\mathbb P_{\mu}(D^c)}\mathbb P_{\mu}\Big[\exp\Big(i\theta e^{-\alpha t}\|X_t\|\Big)\mathbf 1_{D_{\leq t}} \mathbb P_{\mu}\Big[\exp\Big(A(\theta_s,\theta_c,\theta_l,r,t,u)\Big)\mathbf 1_{D_{\geq t}}\Big|\mathscr F_t\Big]\Big]\\
&=\lim_{u\rightarrow\infty}\frac{1}{\mathbb P_{\mu}(D^c)}\mathbb P_{\mu}\Big[\exp\Big( i\theta e^{-\alpha t}\|X_t\|\Big)\mathbf 1_{D_{\leq t}}\mathbb P_{X_t}\Big[\exp\Big(A(\theta_s,\theta_c,\theta_l,r,0,u-t)\Big)\mathbf 1_{D^c}\Big]\Big]\\
&=\frac{1}{\mathbb P_{\mu}(D^c)}\mathbb P_{\mu}\Big[\exp\Big( i\theta e^{-\alpha t}\|X_t\|\Big)\mathbf 1_{D_{\leq t}}\mathbb P_{X_t}(D^c)\widetilde{\mathbb P}_{X_t}\Big[\exp\Big(A(\theta_s,\theta_c,\theta_l,r,0,\infty)\Big)\Big]\Big]
\end{align}
It follows from Theorem \ref{thm: II} and the dominated convergence theorem,
\begin{align}
   k(\theta,\theta_s, \theta_c,\theta_l,r,t)& \xrightarrow{r\to \infty}\frac{1}{\mathbb P_{\mu}(D^c)}\mathbb P_{\mu}\Big[\exp\Big(i\theta e^{-\alpha t}\|X_t\|\Big)\mathbf 1_{D_{\leq t}}\mathbb P_{X_t}(D^c)\Big]
%   \prod_{i=s,c,l}\exp\Big(m[\theta_i f_l]\Big)\\
   \prod_{j=s,c,l}\exp\Big(m[\theta_j f_j]\Big)\\
&=\widetilde{\mathbb P}_{\mu}\Big[\exp\Big(i\theta e^{-\alpha t}\|X_t\|\Big)\Big]
%\prod_{i=s,c,l}\exp\Big(m[\theta_i f_i]\Big).
\prod_{j=s,c,l}\exp\Big(m[\theta_j f_j]\Big).
\end{align}
The desired result of this step now follows immediately.

\emph{Step 2.} For all $r,t>0$, let $\mathcal D(r)$ and $\mathcal D(r,t)$ be the distributions of $S(r)$ and $S(t,r)$ under $\mathbb P_{\mu}(\cdot|D^c)$, let $ \widetilde{\mathcal D}(t)$ and $\mathcal D$ be the distributions of $\widetilde{V}(t)$ and $(H^*,\zeta^{f_s},\zeta^{f_c},\zeta^{-f_l})$ under $\mathbb P_{\mu}(\cdot|D^c)$. Then using \eqref{ineq: distribution control} and \cite[Lemma 3.3]{RenSongSunZhao2019Stable}, we have that for each $\gamma\in (0,\beta)$, there exist constant $C>0$ such that,
\begin{align}
 \limsup_{r\rightarrow \infty}d(\mathcal D(t+r),\mathcal D)&\leq \limsup_{r\rightarrow \infty}[d(\mathcal D(t+r),\mathcal D(t,r))+d(\mathcal D(t,r),\widetilde{\mathcal D}(t))+d(\widetilde{\mathcal D}(t),\mathcal D)]\\
&\leq \limsup_{r\rightarrow \infty}\widetilde{\mathbb P}_{\mu}[|e^{-\alpha t}\|X_t\|-e^{-\alpha (t+r)}\|X_{t+r}\||]+0+\widetilde{\mathbb P}_{\mu}[|H_t-H_{\infty}|]\\
&\leq \limsup_{r\rightarrow \infty}\mathbb P_{\mu}(D^c)^{-1}(\|H_t-H_{t+r}\|_{\mathbb P_{\mu};1+\gamma}+\|H_t-H_{\infty}\|_{\mathbb P_{\mu};1+\gamma})\\
&\leq Ce^{-\frac{\gamma}{1+\gamma}t}.
\end{align}
Therefore, using the definition of $\limsup_{s\rightarrow \infty}$, we know that,
\begin{align}
 &\limsup_{r\rightarrow \infty}d(\mathcal D(r),\mathcal D) =  \limsup_{s\rightarrow \infty}d(\mathcal D(t+r),\mathcal D)
\leq \lim_{t\rightarrow \infty}Ce^{-\frac{\gamma}{1+\gamma}t} = 0.
\end{align}
The desired result now follows immediately.
\end{proof}






\subsection*{Acknowledgment}


\begin{thebibliography}{10}

\bibitem{AdamczakMilos2015CLT}
  R. Adamczak and P. Mi{\l}o\'{s}, \emph{C{LT} for {O}rnstein-{U}hlenbeck branching particle system},
  Electron. J. Probab. \textbf{20} (2015), no. 42, 35 pp.

\bibitem{Asmussen76Convergence}
  S. Asmussen, \emph{Convergence rates for branching processes},
  Ann. Probab.  \textbf{4} (1976), no. 1, 139--146.

\bibitem{AsmussenHering1983Branching}
  S. Asmussen and H. Hering, \emph{Branching processes},
  Progress in Probability and Statistics, 3. Birkh\"{a}user Boston, Inc., Boston, MA, 1983.

\bibitem{Athreya1969Limit}
  K. B. Athreya,
  \emph{Limit theorems for multitype continuous time {M}arkov branching processes. {I}. {T}he case of an eigenvector linear functional},
  Z. Wahrscheinlichkeitstheorie und Verw. Gebiete \textbf{12} (1969), 320--332.

\bibitem{Athreya1969LimitB}
  K. B. Athreya,
  \emph{Limit theorems for multitype continuous time {M}arkov branching processes. {II}. {T}he case of an arbitrary linear functional},
  Z. Wahrscheinlichkeitstheorie und Verw. Gebiete \textbf{13} (1969), 204--214.

\bibitem{Athreya1971Some}
  K. B. Athreya,
  \emph{Some refinements in the theory of supercritical multitype {M}arkov branching processes},
  Z. Wahrscheinlichkeitstheorie und Verw. Gebiete \textbf{20} (1971), 47--57.


\bibitem{Bertoin}
J. Bertoin,
\emph{L\'evy processes}.
Cambridge Tracts in Mathematics, 121. Cambridge
University Press, 1996.

\bibitem{BRY}
J. Bertoin, B. Roynette and M. Yor,
\emph{Some connections between (sub)critical branching mechanisms and Bernstein functions}.
arXiv:0412322.


\bibitem{ChenRenWang2008An-almost}
  Z.-Q. Chen, Y.-X. Ren, and H. Wang,
  \emph{An almost sure scaling limit theorem for {D}awson-{W}atanabe superprocesses},
  J. Funct. Anal. \textbf{254} (2008), no. 7, 1988--2019.

 \bibitem{ChenRenSongZhang2015Strong-law}
   Z.-Q. Chen, Y.-X. Ren, R. Song, and R. Zhang,
   \emph{Strong law of large numbers for supercritical superprocesses under second moment condition},
   Front. Math. China \textbf{10} (2015), no. 4, 807--838.

 \bibitem{ChenRenYang2019Skeleton}
   Z.-Q. Chen, Y.-X. Ren, and T. Yang,
   \emph{Skeleton decomposition and law of large numbers for supercritical superprocesses},
   Acta Appl. Math. \textbf{159} (2019), 225--285.

\bibitem{Dudley2002}
  R. M. Dudley
\emph{Real Analysis and Probability},
  Cambridge University Press, 2002.

 \bibitem{Durrett2010Probability}
   R. Durrett,
   \emph{Probability: theory and examples},
   Fourth edition. Cambridge Series in Statistical and Probabilistic Mathematics, 31. Cambridge University Press, Cambridge, 2010.

\bibitem{Dynkin1993Superprocesses}
  E. B. Dynkin,
  \emph{Superprocesses and partial differential equations},
  Ann. Probab. \textbf{21} (1993), no. 3, 1185--1262.

\bibitem{EckhoffKyprianouWinkel2015Spines}
  M. Eckhoff, A. E. Kyprianou, and M. Winkel,
  \emph{Spines, skeletons and the strong law of large numbers for superdiffusions},
  Ann. Probab. \textbf{43} (2015), no. 5, 2545--2610.

\bibitem{Englander2009Law}
  J. Engl\"{a}nder,
  \emph{Law of large numbers for superdiffusions: the non-ergodic case},
  Ann. Inst. Henri Poincar\'{e} Probab. Stat. \textbf{45} (2009), no. 1, 1--6.

\bibitem{EnglanderWinter2006Law}
  J. Engl\"{a}nder and A. Winter,
  \emph{Law of large numbers for a class of superdiffusions},
  Ann. Inst. H. Poincar\'{e} Probab. Statist. \textbf{42} (2006), no. 2, 171--185.

\bibitem{EnglanderTuraev2002A-scaling}
  J. Engl\"{a}nder and  D. Turaev,
  \emph{A scaling limit theorem for a class of superdiffusions},
  Ann. Probab. \textbf{30} (2002), no. 2, 683--722.

\bibitem{Heyde1970A-rate}
  C. C. Heyde,
  \emph{A rate of convergence result for the super-critical {G}alton-{W}atson process},
  J. Appl. Probability \textbf{7} (1970), 451--454.

\bibitem{Heyde1971Some}
  C. C. Heyde,
    \emph{some central limit analogues for supercritical {G}alton-{W}atson processes},
  J. Appl. Probability \textbf{8} (1971), 52--59.

\bibitem{HeydeBrown1871An-invariance}
  C. C. Heyde and B. M. Brown,
  \emph{An invariance principle and some convergence rate results for branching processes},
  Z. Wahrscheinlichkeitstheorie und Verw. Gebiete, \textbf{20} (1971), 271--278.

\bibitem{HeydeLeslie1971Improved}
  C. C. Heyde and J. R. Leslie,
  \emph{Improved classical limit analogues for {G}alton-{W}atson processes with or without immigration},
  Bull. Austral. Math. Soc. \textbf{5} (1971), 145--155.

\bibitem{IksanovKoleskoMeiners2018Stable-like}
  A. Iksanov, K. Kolesko, and M. Meiners,
  \emph{Stable-like fluctuations of {B}gins' martingales},
  Stochastic Process. Appl. (2018).

\bibitem{Janson2004Functional}
  S. Janson,
  \emph{Functional limit theorems for multitype branching processes and generalized {P}\'{o}lya urns},
  Stochastic Process. Appl. \textbf{110} (2004), no. 2, 177--245.

\bibitem{KestenStigum1966Additional}
  H. Kesten and B. P. Stigum,
  \emph{Additional limit theorems for indecomposable multidimensional {G}alton-{W}atson processes},
  Ann. Math. Statist. \textbf{37} (1966), 1463--1481.

\bibitem{KestenStigum1966A-limit}
  H. Kesten and B. P. Stigum,
  \emph{A limit theorem for multidimensional {G}alton-{W}atson processes},
  Ann. Math. Statist. \textbf{37} (1966), 1211--1223.

\bibitem{KouritzinRen2014A-strong}
  M. A. Kouritzin, and Y.-X. Ren,
  \emph{A strong law of large numbers for super-stable processes},
  Stochastic Process. Appl. \textbf{121} (2014), no. 1, 505--521.

\bibitem{Kyprianou2014Fluctuations}
  A. E. Kyprianou,
  \emph{Fluctuations of {L}\'{e}vy processes with applications},
    Introductory lectures. Second edition. Universitext. Springer, Heidelberg, 2014.

\bibitem{Li2011Measure-valued}
  Z. Li,
  \emph{Measure-valued branching {M}arkov processes},
  Probability and its Applications (New York). Springer, Heidelberg, 2011.

\bibitem{LiuRenSong2009Llog}
  R.-L. Liu, Y.-X. Ren, and R. Song,
  \emph{{$L\log L$} criterion for a class of superdiffusions},
  J. Appl. Probab. \textbf{46} (2009), no. 2, 479--496.

\bibitem{LiuRenSong2013Strong}
  R.-L. Liu, Y.-X. Ren, and R. Song,
  \emph{Strong law of large numbers for a class of superdiffusions},
  Acta Appl. Math. \textbf{123} (2013), 73--97.

\bibitem{MarksMilos2018CLT}
  R. Marks and P. Mi{\l}o{\'s},
  \emph{C{LT} for supercritical branching processes with heavy-tailed branching law},
  arXiv:1803.05491.

\bibitem{MetafunePallaraPriola2002Spectrum}
  G. Metafune, D. Pallara, and E. Priola,
  \emph{Spectrum of {O}rnstein-{U}hlenbeck operators in {$L^p$} spaces with respect to invariant  measures},
  J. Funct. Anal. \textbf{196} (2002), no. 1, 40--60.

\bibitem{Milos2012Spatial}
  P. Mi{\l}o{\'s},
  \emph{Spatial central limit theorem for supercritical superprocesses},
  J. Theoret. Probab. \textbf{31} (2018), no. 1, 1--40.

\bibitem{RenSongSun2017Spine}
  Y.-X. Ren, R. Song, and Z. Sun,
  \emph{Spine decompositions and limit theorems for a class of critical superprocesses},
  Acta Appl. Math. (2019).

\bibitem{RenSongSun2018Limit}
  Y.-X. Ren, R. Song, and Z. Sun,
  \emph{Limit theorems for a class of critical superprocesses with stable branching},
  arXiv:1807.02837.

\bibitem{RenSongSunZhao2019Stable}
Y.-X. Ren, R. Song, Z. Sun and J. Zhao,
\emph{Stable central limit theorems for super Ornstein-Uhlenbeck processes},
Elect. J Probab., \textbf{24} (2019), no. 141, 1--42.

\bibitem{RenSongZhang2014Central}
  Y.-X. Ren, R. Song, and R. Zhang,
  \emph{Central limit theorems for super {O}rnstein-{U}hlenbeck processes},
  Acta Appl. Math. \textbf{130} (2014), 9--49.

\bibitem{RenSongZhang2014CentralB}
  Y.-X. Ren, R. Song, and R. Zhang,
  \emph{Central limit theorems for supercritical branching {M}arkov processes},
  J. Funct. Anal. \textbf{266} (2014), no. 3, 1716--1756.

\bibitem{RenSongZhang2015Central}
  Y.-X. Ren, R. Song, and R. Zhang,
  \emph{Central limit theorems for supercritical superprocesses},
  Stochastic Process. Appl. \textbf{125} (2015), no. 2, 428--457.

\bibitem{RenSongZhang2017Central}
  Y.-X. Ren, R. Song, and R. Zhang,
  \emph{Central limit theorems for supercritical branching nonsymmetric {M}arkov processes},
  Ann. Probab. \textbf{45} (2017), no. 1, 564--623.

\bibitem{RenSongZhang2017Functional}
  Y.-X. Ren, R. Song, and R. Zhang,
  \emph{Functional central limit theorems for supercritical superprocesses},
  Acta Appl. Math. \textbf{147} (2017), 137--175.

\bibitem{Sato2013Levy}
  K. Sato,
  \emph{L{\'e}vy processes and infinitely divisible distributions},
  Translated from the 1990 Japanese original. Revised by the author. Cambridge Studies in Advanced Mathematics, 68. Cambridge University Press, Cambridge, 1999.

\bibitem{SchillingSongVondravcek2010Bernstein}
  R. L. Schilling, R. Song, and Z. Vondra\v{c}ek,
  \emph{Bernstein functions.}
  Theory and applications. Second edition. De Gruyter Studies in Mathematics, 37. Walter de Gruyter \& Co., Berlin, 2012.

\bibitem{SteinShakarchi2003Complex}
  E. M. Stein and R. Shakarchi, \emph{Complex analysis},
  Princeton Lectures in Analysis, 2. Princeton University Press, Princeton, NJ, 2003.

\bibitem{Wang2010An-almost}
  L. Wang, \emph{An almost sure limit theorem for super-{B}rownian motion},
  J. Theoret. Probab. \textbf{23} (2010), no. 2, 401--416.

\end{thebibliography}
\end{document}
