% * Preamble
\documentclass[xcolor=dvipsnames]{beamer}
\synctex=1
%\usepackage[UTF8,scheme=plain]{ctex}
\usepackage{hyperref}
\usepackage{lmodern}
\usepackage{mathrsfs}
\usepackage{graphicx}
\usepackage[export]{adjustbox}
\usepackage{comment}
\usepackage{mathtools}
\mathtoolsset{showonlyrefs}
\usetheme{Madrid}
\usecolortheme{seahorse}
\usecolortheme{rose}
\usefonttheme{serif}
\usefonttheme{structurebold}
\setbeamerfont{title}{shape=\itshape,family=\rmfamily}
\setbeamercolor{title}{fg=red!80!black,bg=red!20!white}
% * Top matter
\title[Stable CLT for superprocesses]{Stable Central Limit Theorems for Super Ornstein-Uhlenbeck Processes}
\author[Zhenyao Sun]{ 
  {\bf \Large Zhenyao Sun  }
}
\institute[WHU]{Wuhan University\\
  Joint work with {\bf Yan-Xia Ren}, {\bf Renming Song} and {\bf Jianjie Zhao}}
\date[]{
	Peking University,
 \\ August, 2019}
\begin{document}

\begin{frame}
  \titlepage
\end{frame}

% * Presentation
% ** Model
\section{Model}
% *** Settints
\subsection{Settings}
\begin{frame}{Model/Settings}
\begin{itemize}
\item
	$E$: a locally compact separable metric space.
\item
	$(\xi_t)_{t\geq 0}$: an $E$-valued Hunt process with semigroup $(P_t)_{t\geq 0}$. 
\item
	$\psi$: a function from $[0,\infty)$ to $\mathbb R$ given by 
  
  \[\psi(z)=
  - {\color{red} \alpha} z + \rho z^2 + \int_{(0,\infty)} (e^{-zy} - 1 + zy) {\color{blue}\pi(dy)},
  \quad  z \geq 0,
  \]

	where ${\color{red}\alpha > 0}$, $\rho \geq0$ and $\pi$ is a measure on $(0,\infty)$ with 

\[
	\int_{(0,\infty)}(y\wedge y^2) {\color{blue}\pi(dy)}
  < \infty.
\]

\item
  We call ${\color{red}\alpha}$ the {\color{red} branching rate}; ${\color{blue} \pi(dy)}$ the {\color{blue} L\'evy measure}.
\end{itemize}
\end{frame}

\begin{frame}{Model/Settings}
\begin{itemize}
\item
	$\mathcal M(E)$: the space of all finite Borel measures on $E$ equipped with the weak topology.
\item
  $\mu(f)$ and $\langle f, \mu\rangle$: the integration of function $f$ and measure $\mu$ whenever it make senses.
\item
	Say a function $f$ on $\mathbb R_+\times E$ is locally bounded if
\[
    \sup_{s\in [0,t],x\in E} |f(s,x)|
    <\infty,
    \quad t\in \mathbb R_+.
\]
\end{itemize}
\end{frame}

% *** Superprocesses
\subsection{Superprocesses}
\begin{frame}{Model/Superprocesses}
\begin{definition}[Superprocess]
	Say an $\mathcal M(E)$-valued Markov process $\{(X_t)_{t\geq 0}; (\mathbb P_\mu)_{\mu\in \mathcal M(E)}\}$ is a $(\xi,\psi)$-superprocess if for each bounded non-negative measurable function $f$ on $E$, we have
\begin{equation}
\label{eq: def of V_t}
    \mathbb{P}_{\mu}[e^{-\langle f,X_t \rangle}]
    = e^{-\langle V_tf, \mu \rangle},
    \quad t\geq 0, \mu \in \mathcal M(E),
\end{equation}
	where $(t,x) \mapsto V_tf(x)$ is the unique locally bounded non-negative solution to the equation
\begin{equation}
	V_tf(x) +  \int_0^t P_{t-s}(\psi \circ V_{s}f)(x)ds
	= P_tf(x),
    \quad x\in E, t\geq 0.
\end{equation}
\end{definition}
\end{frame}

\begin{frame}{Model/Superprocesses}
\begin{example}
	Let $\alpha >0$ and ${\color{red} \beta} \in (0,1)$.
	Consider a branching Ornstein-Uhlenbeck(OU) process with
\begin{itemize}
\item
	{\color{blue} $k$} initial particles at the origin;
\item
	killing rate  $2{\color{blue} k}^{\color{red}\beta}$;
\item
	offspring generating function
\(
	F_{\color{blue} k}(x) =x +\frac{1}{2}(1-x)^{1+{\color{red}\beta}} + \frac{-\alpha (1-x)}{2{\color{blue} k}^{\color{red}\beta}}.
\)
\end{itemize}
	Let $X^{({\color{red} k})}_t(A)$ be number of particles in a Borel set $A$ at time $t$. 
	Then,
\[
	\Big(\frac{1}{\color{blue} k} X^{({\color{blue} k})}_t(\cdot)\Big)_{t\geq 0}
	\xrightarrow[{\color{blue} k}\to \infty]{d} (X_t)_{t\geq 0}
\]
where $(X_t)_{t\geq 0}$ is a super-OU process with branching mechanism \[\psi(z) = -\alpha z + z^{1+{\color{red}\beta}}.\]
\end{example}
\end{frame}

% ** Background
\section{Background}
% *** Motivation
\subsection{Motivation}
\begin{frame}{Background/Motivation}
  \begin{block}{Motivation}
    We are interested in the spatial laws of large numbers (LLNs) and central limit theorems (CLT) for superprocesses.
    More precisely, for a testing function $f$, we want to find
    $(F_t)_{t\geq 0}$ and $(G_t)_{t\geq 0}$ such that
    \[  
      \frac{\langle f, X_t \rangle -G_t}{F_t}
    \]

    \begin{itemize}
    \item
      converges in probability or in $L^p$ to a non-degenerate random variable (weak LLNs);
    \item
      converges almost surely to a non-degenerate random variable (strong LLNs);
    \item
      converges weakly to some non-degenerate random variable (CLTs).
    \end{itemize}
  \end{block}
\end{frame}
% *** LLNs of superprocesses
\subsection{LLNs of superprocesses}
\begin{frame}{Background/LLNs of superprocesses}
\begin{itemize}
\item
  {\color{red}Weak LLNs of supercritical superprocesses} under different settings: 
  
  \mbox{ $\circ$ J. Engl\"{a}nder and D. Turaev 2002 (AP);}
  \mbox{ $\circ$ J. Engl\"{a}nder and A. Winter 2006 (AIHPPS);}
  \mbox{ $\circ$ J. Engl\"{a}nder 2009 (AIHPPS);} 
\end{itemize}
\let\thefootnote\relax\footnotetext{
  \mbox{(AP) = Ann. Probab.;} 
  \mbox{(AIHPPS) = Ann. Inst. H. Poincar\'e Probab. Statist.}
}
\end{frame}

\begin{frame}{Background/LLNs of superprocesses}
\begin{itemize}
\item
  {\color{red} Strong LLNs of supercritical superprocesses} under different settings: 
  
  \mbox{ $\circ$ Z.-Q. Chen, Y.-X. Ren and H. Wang 2008 (JFA);} 
  \mbox{ $\circ$ L. Wang 2010 (JTP);} 
  \mbox{ $\circ$ R.-L. Liu, Y.-X. Ren and R. Song 2013 (AAM);} 
  \mbox{ $\circ$ M. A. Kouritzin and Y.-X. Ren 2014 (SPA);} 
  \mbox{ $\circ$ Z.-Q. Chen, Y.-X. Ren, R. Song and R. Zhang 2015 (FMC);} 
  \mbox{ $\circ$ M. Eckhoff, A. E. Kyprianou and M. Winkel 2015 (AP);} 
  \mbox{ $\circ$ Z.-Q. Chen, Y.-X. Ren and T. Yang 2019 (AAM).}
\end{itemize}
\let\thefootnote\relax\footnotetext{
  \mbox{(JFA) = J. Funct. Anal.;}
  \mbox{(JTP) = J. Theoret. Probab.;}
  \mbox{(AAM) = Acta Appl. Math.;}
  \mbox{(SPA) = Stochastic Process. Appl.;}
  \mbox{(FMC) = Front. Math. China.}
}
\end{frame}

\begin{frame}{Background/LLNs of superprocesses}
\begin{example}
	Suppose that $X$ is a super OU process with branching mechanism $\psi(z)=-\alpha z+ z^{1+\beta}$ with $\alpha>0$ and $\beta \in (0,1)$. Then, for each $\mu \in \mathcal M(\mathbb R^d)$ and each continuous bounded non-negative function $f$ on $\mathbb R^d$, we have
\[
	e^{-\alpha t} X_t(f) \xrightarrow[t\to \infty]{\mathbb P_\mu \text{-a.s.}, L^1(\mathbb P_\mu)} H_\infty \int f(x) \varphi(x) dx,
\]
	where $H_\infty$ is the limit of the non-negative martingale $e^{-\alpha t}\|X_t\|$ and $\varphi$ is the invariant probability density of the OU process $\xi$.
\end{example}
\end{frame}

\subsection{CLTs for branching processes with the 2rd moment condition}
\begin{frame}{Background/CLTs for branching processes with 2rd moment conditions}
\begin{itemize}
\item
	{\color{red} CLT of supercritical Galton-Watson(GW) processes} with finite offspring variance: 
  
  \mbox{ $\circ$ C. C. Heyde 1970 (JAP);}
  \mbox{ $\circ$ C. C. Heyde and B. M. Brown 1971 (ZWVG);}
  \mbox{ $\circ$ C. C. Heyde and J. R. Leslie 1971 (BAMS).}
\item 
	{\color{red} CLT of supercritical multi-type GW processes} with a 2rd moment condition:
  
  \mbox{ $\circ$ K. B. Athreya 1969a (ZWVG);}
  \mbox{ $\circ$ K. B. Athreya 1969b (ZWVG);} 
  \mbox{ $\circ$ K. B. Athreya 1971 (ZWVG).}
\end{itemize}
\let\thefootnote\relax\footnotetext{
  \mbox{(JAP) = J. Appl. Probability;}
  \mbox{(ZWVG) = Z. Wahrsch. Verw. Gebiete;}
  \mbox{(BAMS) = Bull. Austral. Math. Soc.}
}
\end{frame}

\begin{frame}{Background/CLTs for branching processes with 2rd moment conditions}
\begin{itemize} 
\item
  {\color{red} CLT of supercritical branching OU processes} with binary branching: 

  \mbox{ $\circ$ R. Adamczak and P. Mi{\l}o{\'s} 2015 (EJP).}
\item
	{\color{red} CLT of some general supercritical branching Markov processes} with a 2rd moment condition:
  
  \mbox{ $\circ$ S. Asmussen and H. Hering 1983 (book);}
  \mbox{ $\circ$ Y.-X. Ren, R. Song and R. Zhang 2014 (JFA);}
  \mbox{ $\circ$ Y.-X. Ren, R. Song and R. Zhang 2017 (AP).}
\end{itemize}
\let\thefootnote\relax\footnotetext{
  \mbox{(EJP) = Electron. J. Probab.}
}
\end{frame}

\begin{frame}{Background/CLTs for Superprocesses with 2rd moment conditions}
\begin{itemize}
\item
	{\color{red} CLT of supercritical superprocesses} with a fourth moment condition:
  
  \mbox{ $\circ$ P. Mi{\l}o\'{s} 2018 (JTP).}
\item
	{\color{red} CLT of supercritical super OU processes} with a 2rd moment condition:
  
  \mbox{ $\circ$ Y.-X. Ren, R. Song and R. Zhang 2014 (AAM).}
\item
	{\color{red} CLT for a large class of general supercritical superprocesses} with a 2rd moment condition:
  
  \mbox{ $\circ$ Y.-X. Ren, R. Song and R. Zhang 2015 (SPA);}
  \mbox{ $\circ$ Y.-X. Ren, R. Song and R. Zhang 2017 (AAM).}
\end{itemize}
\end{frame}

\subsection{Stable CLTs without the 2rd moment condition}
\begin{frame}{Background/Stable CLTs without the 2rd moment condition}
\begin{itemize}
\item
	{\color{red} Stable CLT for supercritical GW processes:}

  \mbox{ $\circ$ C. C. Heyde 1971 (JAP).}
\item
	{\color{red} Stable CLT for supercritical multi-type GW process:} 
  
  \mbox{ $\circ$ S. Asmussen 1976 (AP).}
\item
	{\color{red} Stable CLTs for supercritical branching OU processes} with offspring generating function

\( \displaystyle
    \qquad F(s)
    = ms -(m-1) +(m-1) (1-s)^{1+\beta},
    \quad |s|< 1
\)

	where $m>1$ and $\beta\in (0, 1)$:

  \mbox{ $\circ$ R. Marks and P. Mi{\l}o{\'s} 2018 (arXiv).}
\item
	{\color{red} Stable fluctuation results of Biggins’ martingales of branching random walks:} 

  \mbox{ $\circ$ A. Iksanov, K. Kolesko and M. Meiners 2018 (SPA).}
\end{itemize}
\end{frame}


\section{Settings}
\begin{frame}{Settings/Assumption 1}
\begin{itemize}
\item
	$E:= \mathbb R^d$.
\item
  $\xi = (\xi_t)_{t\geq 0}$, an $\mathbb R^d$-valued OU process with generator
  
  \( \displaystyle \qquad
	Lf(x)
	=\frac{1}{2} \sigma^2 \Delta f(x)-bx\cdot\nabla f(x),
	\quad x\in \mathbb R^d, f\in C^2(\mathbb R^d),
  \)

	where $\sigma,b>0$.
\end{itemize}
\begin{block}{Assumption 1. (Grey's condition)}
	There exists $z' > 0$ such that $\psi(z) > 0$ for all $z>z'$ and  
  \centerline{ \( \displaystyle
    \int_{z'}^\infty \psi(z)^{-1}dz 
    < \infty.
    \)}
\end{block}
\begin{itemize}
\item
	Under Assumption 1, write $D := \{\exists t\geq 0,~\text{s.t.}~\|X_t\|=0\}$, then

  \centerline{ \( \displaystyle
    \mathbb P_\mu(D)
    > 0, 
    \quad \mu \in \mathcal M(\mathbb R^d).
  \)}
\end{itemize}
\end{frame}

\begin{frame}{Settings/Assumption 2}
\begin{itemize}
\item
	$\Gamma$: the gamma function.
\item
	$|\mu|$: the total variation measure of some measure $\mu$.
\item
  Recall that $\pi$ is the L\'evy measure of the branching mechanism $\psi$.
\end{itemize}
\begin{block}{Assumption 2.}
	There exist (unique) constants ${\color{red}\eta} > 0$ and ${\color{red}\beta} \in (0,1)$ such that for some $\delta \in (0, 1- {\color{red}\beta})$ it holds that
  \centerline{ \( \displaystyle	
    \int_{(1,\infty)}y^{1+{\color{red}\beta} +\delta}~\Big|\pi(dy)-\frac{{\color{red}\eta}~dy}{\Gamma(-1-{\color{red}\beta})y^{2+{\color{red}\beta}}}\Big| 
    <\infty.
    \)}
\end{block}
\begin{itemize}
\item
    In particular, if 
    \( \displaystyle
    \pi(dy)
    =\frac{{\color{red}\eta} dy}{ \Gamma(-1-{\color{red}\beta})y^{2+{\color{red}\beta}}},
    \) 
    then the branching mechanism takes the form: 
    \( 
    \psi(z)
    = -\alpha z + \rho z^2 + {\color{red}\eta} z^{1+{\color{red}\beta}}
    ,\quad z\geq 0.
    \)
\end{itemize}
\end{frame}

\section{Preliminary}
\begin{frame}{Preliminary/Hermite polynomials}
\begin{itemize}
\item
  Recall that $(P_t)_{t\geq 0}$ is the transition semigroup of the OU process $\xi$.
\item
  $(P_t)_{t\geq 0}$ has the invariant probability
  
  \centerline{ \( \displaystyle	
    \varphi(x)~dx
    := \Big(\frac{b}{\pi \sigma^2} \Big)^{\frac{d}{2}} e^{-\frac{b}{\sigma^2}|x|^2}~dx,
    \quad x\in \mathbb R^d.
    \)}
\item
  $L^2(\varphi)$: the Hilbert space with inner product
  
  \centerline{ \( \displaystyle    
    \langle f_1, f_2 \rangle_{\varphi}
    := \int_{\mathbb R^d}f_1(x)f_2(x)\varphi(x)~dx.
    \)}
\item
    For each multi-index $p = (p_k)_{k = 1}^d \in \mathbb{Z}_+^{d}$, write 
  
    \centerline{ \( \displaystyle	
      |p|:=\sum_{k=1}^d p_k;
      \quad p!:= \prod_{k= 1}^d p_k!;
      \quad \frac{\partial^p} {\partial x^p}:= \prod_{k = 1}^d\frac{\partial^{p_k}}{\partial x_k^{p_k}}.
      \)}
\item
    Define the \emph{Hermite polynomials} by
    
    \centerline{ \( \displaystyle
      H_p(x)
      :=(-1)^{|p|}e^{|x|^2} \frac{\partial ^{p}}{\partial x^p} e^{-|x|^2} ,
      \quad x\in \mathbb R^d,
      p \in \mathbb{Z}_+^{d}.
      \)}
\end{itemize}
\end{frame}

\begin{frame}{Preliminary/Spectrum of OU semigroup}
\begin{block}{Fact}
	$(P_t)_{t\geq 0}$ is a strongly continuous semigroup on $L^2(\varphi)$ and its generator $L$ has discrete spectrum $\sigma(L)= \{-bk: k \in \mathbb Z_+\}$.
  For each $k \in \mathbb Z_+$, denote by $\mathcal{A}_k$ the eigenspace corresponding to the eigenvalue $-bk$, then

  \[
    \mathcal{A}_k
    = \operatorname{Span} \{\phi_p : p\in \mathbb Z_+^d, |p|=k\},
  \]

    where

    \[
    \phi_p(x)
    := \frac{1}{\sqrt{ p! 2^{|p|} }} H_p \Big(\frac{ \sqrt{b} }{\sigma}x \Big),
    \quad x\in \mathbb R^d, p\in \mathbb Z_+^d.
    \]

    Moreover, $\{\phi_p:p\in \mathbb Z_+^d\}$ forms a complete orthonormal basis for $L^2(\varphi)$.
\end{block}
\end{frame}

\begin{frame}{Preliminary/Martingale limits}
  \begin{lemma}
    Write $\tilde u = \frac{u}{1+ u}$ for all $u \in (0,1)$.
    For each $p\in \mathbb Z_+^d$ with $\alpha \tilde \beta > |p|b$, $\gamma \in (0,\beta)$ and $\mu \in \mathcal M(\mathbb R^d)$ with compact support, the following martingale
    
    \[  
      H_t^p
      := e^{-(\alpha-|p|b) t} \langle \phi_p, X_t\rangle
      , \quad t\geq 0,
    \]
    is bounded in $L^{1+\gamma}(\mathbb P_\mu)$. 
    Thus the limit $H_\infty^p:= \lim_{t\to \infty} H_t^p$ exists $\mathbb P_\mu$-almost surely and in $L^{1+\gamma}(\mathbb P_\mu)$.
  \end{lemma}
  \begin{itemize}
\item
  Roughly speaking, for all $p \in \mathbb Z_+^d$ and $\gamma \in (0,\beta)$, we can estimate that
  
  \centerline{ \( \displaystyle
    \|H_{t+1}^p - H_t^p\|_{1+\gamma} 
    \sim e^{- (\alpha\tilde\gamma - |p|b)t}.
    \)}
  So if $\alpha \tilde \beta < |p| b$ then there exists a $\gamma \in (0,\beta)$ such that $H_t^p$ is not a Cauchy sequence in $L^{1+\gamma}$. 
\end{itemize}
\end{frame}

\begin{frame}{Preliminary/Polynomial growth functions}
\begin{itemize}
\item
  $\mathcal P$: the collection of all functions of polynomial growth, i.e.
  \[
    \mathcal P 
    := \{f \in \mathcal B(\mathbb R^d, \mathbb R) : \exists C, n > 0 \text{ s.t. } \forall x\in \mathbb R^d, |f(x)|\leq C(1+|x|)^n\}.
  \]
\item
  Recall that $\alpha$ is the branching rate and write $\tilde \beta = \frac{\beta}{1+\beta}$. Define
\begin{align}  
    &\mathcal{C}_s
    :=\mathcal P \cap \overline{Span}\{\phi_p: \alpha \tilde\beta < |p|b \}; \\
    &\mathcal{C}_c
    :=\mathcal P\cap\overline{Span}\{\phi_p: \alpha \tilde\beta=|p|b \}; \\
    &\mathcal{C}_l
    :=\mathcal P\cap \overline{Span}\{\phi_p: \alpha \tilde \beta >|p|b \}. 
\end{align}
\item
  Then $\mathcal P = \mathcal C_s \oplus \mathcal C_c \oplus \mathcal C_l$, i.e. for each $f\in \mathcal P$ there exists unique $f_s\in \mathcal C_s, f_c\in \mathcal C_c, f_l\in \mathcal C_l$ such that 
  \( \displaystyle
  f = f_s+f_c+f_l.
  \) 
\item
  The asymptotic behaviors of $X_t(f_s), X_t(f_c)$ and $X_t(f_l)$ will be different. 
\end{itemize}
\end{frame}

\section{Results}
\begin{frame}{Results/LLNs}
  \begin{theorem}
    Suppose that $f \in \mathcal{P}$ satisfies $\alpha \beta > \kappa_f b ( 1 + \beta)$ where
    \[
      \kappa_f :=\inf \{k \geq 0: \exists p \in \mathbb Z_+^d,\text{ s.t. }|p| = k \text{ and } \langle f, \phi_p\rangle_\varphi \neq 0\}.
    \]
Then for all $\gamma\in (0, \beta)$ and  $\mu \in \mathcal M_c(\mathbb R^d)$,
    \[
      e^{-(\alpha-\kappa_fb)t}\langle f, X_t\rangle
      \xrightarrow[t\to \infty]{}\sum_{p\in \mathbb Z_+^d:|p|=\kappa_f}\langle f, \phi_p\rangle_{\varphi} H_{\infty}^p
      \quad in~ L^{1+\gamma}(\mathbb{P}_{\mu}).
    \]
    Moreover, if $f$ is twice differentiable and all its second order partial derivatives are in $\mathcal{P}$, then we also have almost sure convergence.
  \end{theorem}
\end{frame}
\begin{frame}{Results/A Lemma}
  \begin{lemma}
    Define a family of operators $(T_t)_{t\geq 0}$ on $\mathcal P$ by
    \centerline{ \( \displaystyle      
T_tf 
      := \sum_{p \in \mathbb Z_+^d} e^{-\big| |p|b - \alpha \tilde \beta \big|t} \langle f, \phi_p \rangle_{\varphi} \phi_p,
    \)}
    and a family of $\mathbb C$-valued functionals $(m_t)_{t\geq 0}$ on $\mathcal P$ by
   \centerline{ \( \displaystyle
      m_t[f]
      := \eta \int_t^{t+1} ~ds \int_{\mathbb R_d} \big(-iT_sf(x)\big)^{1+\beta} \varphi(x) ~dx ,
    \)}
    Then it holds that
    \begin{itemize}
    \item 
      $(T_t)_{t\geq 0}$ and $(m_t)_{t\geq 0}$ are both well defined;
    \item
      if $f\in \mathcal C_s \oplus \mathcal C_l$, then
      \( \displaystyle
      m[f]
      := \sum_{n \in \mathbb Z_+} m_n[f]
      \in \mathbb C
      \)
      is well defined;
    \item
      if $f \in \mathcal C_c$, then
      \( \displaystyle
      \bar m[f]
      := \lim_{t\to \infty}\frac{1}{t} \sum_{n=0}^t m_n[f]
      \in \mathbb C
      \)
      is well defined.
    \end{itemize}
  \end{lemma}
\end{frame}

\begin{frame}{Results/Stable CLT}
\begin{theorem}
    Let $\mu\in \mathcal M(\mathbb R^d)\setminus \{0\}$ have compact support. Then under $\mathbb{P}_{\mu}(\cdot|D^c)$, the follows hold:
\begin{itemize}
\item
  if $f_s\in \mathcal C_s\setminus\{0\}$, then 
  \( \displaystyle  
    \frac{X_t(f_s)}{\|X_t\|^{\frac{1}{1+\beta}}}
    \xrightarrow[t\to \infty]{d} \zeta_s,\)
    an $(1+\beta)$-stable law with characteristic function (ch.f.) $\theta \mapsto e^{m[\theta f_s]}$; 
\item
  if $f_c\in \mathcal C_c\setminus\{0\}$, then
  \( \displaystyle
    \frac{X_t(f_c)}{\|t X_t\|^{\frac{1}{1+\beta}}}
    \xrightarrow[t\to \infty]{d} 
    \zeta_c,
    \)
    an $(1+\beta)$-stable law with ch.f. $\theta \mapsto e^{\bar m[\theta f_c]}$;
  \item
    if $f_l\in \mathcal C_l\setminus\{0\}$, then
    \centerline{ \( \displaystyle
    \frac{X_t(f_l) - \sum_{p\in \mathbb Z^d_+:\alpha \tilde \beta>pb}\langle f_l,\phi_p\rangle_\varphi e^{(\alpha-|p|b)t}H^p_{\infty} }{\|X_t\|^{\frac{1}{1+\beta}}}
    \xrightarrow[t\to \infty]{d}
    \zeta_l,
    \)}
    an $(1+\beta)$-stable law with ch.f. $\theta \mapsto e^{m[- \theta f_l]}$.
\end{itemize}
\end{theorem}
\end{frame}

\begin{frame}{Remark/Two effects}
\begin{itemize}
\item
	The phase transition phenomenon is due to an interplay of the {\color{red} coarsening effect} and the {\color{blue}smoothing effect}.
  \mbox{ $\circ$ R. Adamczak and P. Mi{\l}o{\'s} 2015 (EJP).}
\item
	{\color{red}Coarsening effect} (increasing of the spatial inequality): 
  A consequence of branching. 
  An area with more particles will produce more offspring.
\item
	{\color{blue}Smoothing effect} (decreasing of the spatial inequality):  
	A consequence of the mixing property of the OU process. 
  Each OU particles will ``forget'' its initial position exponentially fast.   
\item
  Roughly speaking, for all $p \in \mathbb Z_+^d$ and $\gamma \in (0,\beta)$, we can estimate that
  
  \centerline{ \( \displaystyle
    \|H_{t+1}^p - H_t^p\|_{1+\gamma} 
    \sim e^{- ( {\color{red}\alpha\tilde\gamma} - {\color{blue}|p|b})t}.
    \)} 
\end{itemize}
\end{frame}
\begin{frame} {Methods}
  \begin{enumerate}
    \item
      The characteristic exponent $U_t f(x):= \log \mathbb P_{\delta_x}[e^{i \langle f, X_t \rangle}]$ can be characterized by a complex valued integral equation.
      \mbox{ $\circ$ N-measure of the superprocesses}, E.-B. Dykin and S. E. Kuznetsov 2004 (PTRF).

\mbox{ $\circ$ Generalized spine decomposition theorem for superprocesses,} Y.-X. Ren, R. Song and S. 2019 (AAM). 

\mbox{ $\circ$ Discussions with Z. Li.}
\item Some estimations of $|U_tf - iP_t^\alpha f|$ when $t\leq 1$.
  \mbox{ $\circ$ Thanks to Assumption 2 and discussions with R. Zhang.}
\item Tail probability estimation for random variable 
  \(\mathbb P_\mu [\langle f, X_{T}\rangle | \mathscr F_t] - \mathbb P_\mu [\langle f, X_{T}\rangle | \mathscr F_{t+1}].\) 
\item
  CLT for $\langle f,X_T\rangle$.
  \mbox{ $\circ$ Similar to the arguments in R. Marks and P. Mi{\l}o{\'s} 2018 (arXiv)} for branching OU processes. 
  \mbox{ $\circ$ Ideas trace back to the CLT of martingales.}
    \end{enumerate}
    
\end{frame}
\begin{frame}
  \[ \text{ \it \Large Thanks!}\]
\end{frame}

\end{document}
