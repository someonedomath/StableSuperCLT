% * Preamble
\documentclass[xcolor=dvipsnames]{beamer}
\synctex=1
%\usepackage[UTF8,scheme=plain]{ctex}
\usepackage{hyperref}
\usepackage{lmodern}
\usepackage{mathrsfs}
\usepackage{graphicx}
\usepackage[export]{adjustbox}
\usepackage{comment}
\usepackage{mathtools}
\mathtoolsset{showonlyrefs}
\usetheme{Madrid}
\usecolortheme{seahorse}
\usecolortheme{rose}
\usefonttheme{serif}
\usefonttheme{structurebold}
\setbeamerfont{title}{shape=\itshape,family=\rmfamily}
\setbeamercolor{title}{fg=red!80!black,bg=red!20!white}
% * Top matter
\title[Stable CLT for Super OU]{Stable Central Limit Theorems for Super Ornstein-Uhlenbeck Processes}
\author[Z. Sun]{ 
  {\bf \Large Zhenyao Sun\inst{1}}
\\
{\scriptsize  Joint work with {\bf Yan-Xia Ren\inst{1}}, {\bf Renming Song\inst{2}} and {\bf Jianjie Zhao\inst{1} } }
}
\institute[PKU]{
\and
\inst{1}
	Peking University
\and
\inst{2}
	University of Illinois at Urbana-Champaign
}
\date[SHNU, July, 2019]{
	Shanghai Normal University,\\
	July, 2019}
\begin{document}

\begin{frame}
  \titlepage
\end{frame}

\section{Model}
\subsection{Settings}
\begin{frame}{Model/Settings}
\begin{itemize}
\item
	$E$: a locally compact separable metric space.
\item
	$(\xi_t)_{t\geq 0}$: an $E$-valued Hunt process with semigroup $(P_t)_{t\geq 0}$. 
\item
	$\psi$: a function from $[0,\infty)$ to $\mathbb R$ given by 
\begin{equation} \label{eq: honogeneou branching mechanism}
    \psi(z)=
    - {\color{red} \alpha} z + \rho z^2 + \int_{(0,\infty)} (e^{-zy} - 1 + zy) \pi(dy),
    \quad  z \geq 0,
\end{equation}
	where ${\color{red}\alpha} > 0 $, $\rho \geq0$ and $\pi$ is a measure on $(0,\infty)$ with 
\[
	\int_{(0,\infty)}(y\wedge y^2) \pi(dy)< \infty.
\]
\item
  We call ${\color{red}\alpha}$ the {\color{red} branching rate}.
\end{itemize}
\end{frame}

\begin{frame}{Model/Settings}
\begin{itemize}
\item
	$\mathcal M(E)$: the space of all finite Borel measures on $E$ equipped with the weak topology.
\item
  $\left\langle f,\mu \right\rangle$: the integration of function $f$ and measure $\mu$ whenever it make senses.
\item
	Say a function $f$ on $\mathbb R_+\times E$ is \emph{locally bounded} if
\[
    \sup_{s\in [0,t],x\in E} |f(s,x)|
    <\infty,
    \quad t\in \mathbb R_+.
\]
\end{itemize}
\end{frame}

\subsection{Superprocesses}
\begin{frame}{Model/Superprocesses}
\begin{definition}[Superprocess]
	Say an $\mathcal M(E)$-valued Markov process $\{(X_t)_{t\geq 0}; (\mathbb P_\mu)_{\mu\in \mathcal M(E)}\}$ is a $(\xi,\psi)$-superprocess if for each bounded non-negative measurable function $f$ on $E$, we have
\begin{equation}
\label{eq: def of V_t}
    \mathbb{P}_{\mu}[e^{-\langle f,X_t \rangle}]
    = e^{-\langle V_tf, \mu \rangle},
    \quad t\geq 0, \mu \in \mathcal M(E),
\end{equation}
	where $(t,x) \mapsto V_tf(x)$ is the unique locally bounded non-negative solution to the equation
\begin{equation}
	V_tf(x) +  \int_0^t P_{t-s}(\psi \circ V_{s}f)(x)ds
	= P_tf(x),
    \quad x\in E, t\geq 0.
\end{equation}
\end{definition}
\end{frame}

\begin{frame}{Model/Superprocesses}
\begin{example}
	Let $\alpha >0$ and $\beta \in (0,1)$.
	Consider a Branching OU-process with
\begin{itemize}
\item
	{\color{red} $k$} initial particles;
\item
	killing rate  $2{\color{red}k}^\beta$;
\item
	offspring generating function
\[
	F_{\color{red}k}(x) =x +\frac{1}{2}(1-x)^{1+\beta} + \frac{-\alpha (1-x)}{2{\color{red}k}^\beta}.
\]
\end{itemize}
	Let $X^{({\color{red} k})}_t(A)$ be number of particles in a Borel set $A$ at time $t$. 
	Then,
\[
	\Big(\frac{1}{\color{red} k} X^{({\color{red} k})}_t(\cdot)\Big)_{t\geq 0}
	\xrightarrow[{\color{red}k}\to \infty]{w} (X_t)_{t\geq 0}
\]
where $(X_t)_{t\geq 0}$ is a super OU process with branching mechanism \[\psi(z) = -\alpha z + z^{1+\beta}.\]
\end{example}
\end{frame}

\section{Background}
\subsection{Motivation}

\begin{frame}{Background/Motivation}
  \begin{block}{Motivation}
    We are interested in the spatial laws of large numbers (LLNs) and central limit theorems (CLT) for superprocesses.
    More precisely, we want to find
    $(F_t)_{t\geq 0}$ and $(G_t)_{t\geq 0}$ such that
    \[
      \frac{\langle f, X_t \rangle -G_t}{F_t}
    \]
\begin{itemize}
\item
converges in probability or in $L^p$ to a non-degenerate random variable (weak LLNs);
\item
converges almost surely to a non-degenerate random variable (strong LLNs);
\item
converges weakly to some non-degenerate random variable (CLTs).
\end{itemize}
  \end{block}
\end{frame}

\subsection{LLNs of superprocesses}
\begin{frame}{Background/LLNs of superprocesses}
\begin{itemize}
\item
{\color{red}Weak laws of large numbers} under different settings: Englander 2009, Englander and Winter 2006, Englander  and Turaev 2002.
\item
{\color{red}Strong law of large numbers} under different settings: Chen, Ren and Wang 2008, Wang 2010, Liu, Ren and Song 2013, Kouritzin and  Ren 2014, Chen, Ren, Song and Zhang 2015, Eckhoff, Kyprianou and Winkel 2015, Chen, Ren and Yang 2019.
\end{itemize}
\end{frame}

\begin{frame}{Background/LLNs of superprocesses}
\begin{example}[Chen, Ren and Yang 2019]
	Suppose that $X$ is a super OU process with branching mechanism $\psi(z)=-\alpha z+ z^{1+\beta}$ with $\alpha>0$ and $\beta \in (0,1)$. Then, for each $\mu \in \mathcal M(\mathbb R^d)$ and each continuous bounded non-negative function $f$ on $\mathbb R^d$, we have
\[
	e^{-\alpha t} \langle f, X_t\rangle \xrightarrow[t\to \infty]{\mathbb P_\mu \text{-a.s.}, L^1(\mathbb P_\mu)} H_\infty \langle f, \varphi\rangle,
\]
	where $H_\infty$ is the limit of the non-negative martingale $e^{-\alpha t}\langle 1,X_t\rangle$ and $\varphi$ is the invariant density of the OU process $\xi$.
\end{example}
\end{frame}

\subsection{CLTs for branching processes with 2rd moment conditions}
\begin{frame}{Background/CLTs for branching processes with 2rd moment conditions}
\begin{itemize}
\item
	For CLT of supercritical {\color{red} Galton-Watson processes}, see Heyde 1970, Heyde and Brown 1971, Heyde and Leslie 1971.
\item 
	For CLT of supercritical  {\color{red} multi-type Galton-Watson processes}, see Athreya 1969, Athreya 1969, Athreya 1971.
\item
	For CLT of some general supercritical {\color{red} branching Markov processes}, see Asmussen and Hering 1983.
\item
	For CLT of supercritical {\color{red} branching OU processes} with {\color{red} binary branching} mechanism, see Adamczak and Milos 2015.
\end{itemize}
\end{frame}

\begin{frame}{Background/CLTs for Superprocesses with 2rd moment conditions}
\begin{itemize}
\item
	For CLT of some supercritical {\color{red} superprocesses} with branching mechanisms satisfying a {\color{red} fourth moment condition}, see Milos 2018.
\item
	For CLT of supercritical {\color{red} super OU processes} with branching mechanisms satisfying only a {\color{red} second moment condition}, see Ren, Song and Zhang 2014.
\item
	For a series of CLT for a large class of general supercritical {\color{red} branching Markov processes} and {\color{red} superprocesses} with {\color{red} spatially dependent branching mechanisms} satisfying a {\color{red} second moment condition}, see Ren, Song and Zhang 2014, Ren, Song and Zhang 2015, Ren, Song and Zhang 2017.
\end{itemize}
\end{frame}

\subsection{Stable CLTs without the 2rd moment condition}
\begin{frame}{Background/Stable CLTs without the 2rd moment condition}
\begin{itemize}
\item
	For stable CLT for supercritical {\color{red} Galton-Watson process} whose {\color{red} offspring distribution
    belongs to the domain of attraction of a stable law of index $\alpha\in (1, 2]$}, see Heyde 1971.
\item
	For stable CLT for supercritical {\color{red} multi-type Galton-Watson process} and supercritical {\color{red} continuous time branching processes}, see Asmussen 1976.
\item
	Recently, some stable CLTs for supercritical {\color{red} branching OU processes} with offspring generating function
\[
    F(s)
    = ms -(m-1) +(m-1) (1-s)^{1+\beta},
    \quad |s|< 1
\]
	with $m>1$ and $\beta\in (0, 1)$ are established in Marks and Milos 2018.
\item
	Very recently, some stable fluctuation results of Biggins’ martingales in the context of {\color{red} branching random walks} are established in Iksanov, Kolesko and Meiners 2018.
\end{itemize}
\end{frame}


\section{Settings}
\begin{frame}{Settings}
\begin{itemize}
\item
	Let $E= \mathbb R^d$.
\item
	Let $(\xi_t)_{t\geq 0}$ be an $\mathbb R^d$-valued Ornstein-Uhlenbeck (OU) process with generator
\[
	Lf(x)
	=\frac{1}{2} \sigma^2 \Delta f(x)-bx\cdot\nabla f(x),
	\quad x\in \mathbb R^d, f\in C^2(\mathbb R^d),
\]
	where $\sigma,b>0$.
\end{itemize}
\end{frame}

\begin{frame}{Settings/Assumption 1}

\begin{block}{Assumption 1. (Grey's condition)}
	There exists $z' > 0$ such that $\psi(z) > 0$ for all $z>z'$ and  $\int_{z'}^\infty \psi(z)^{-1}dz < \infty$.
\end{block}
\begin{itemize}
\item
	It is known  that, under Assumption 1, 
	the \emph{extinction event} 
	$D:=\{\exists t\geq 0,~\text{s.t.}~\|X_t\|=0\}$ has positive probability, with respect to $\mathbb P_\mu$ for each  $\mu \in \mathcal M(\mathbb R^d)$.
\end{itemize}
\end{frame}

\begin{frame}{Settings/Assumption 2}
\begin{itemize}
\item
	Denote by $\Gamma$ the gamma function.
	For any $\sigma$-finite signed measure $\mu$, we use $|\mu|$ to denote the total variation measure of $\mu$.
\end{itemize}
\begin{block}{Assumption 2.}
	There exist constants $\eta > 0$ and $\beta \in (0,1)$ such that
\begin{equation}
    \int_{(1,\infty)}y^{1+\beta +\delta}~\Big|\pi(dy)-\frac{\eta~dy}{\Gamma(-1-\beta)y^{2+\beta}}\Big| <\infty,
\end{equation}
	for some $\delta \in (0,1-\beta)$.
\end{block}
\end{frame}

\begin{frame}{Settings/Assumption 2}
\begin{itemize}
\item
	Assumption 2 says that 
 there exist constants $\eta>0$ and $\beta > 0$ 
    such that the L\'evy measure $\pi(dy)$ is not too far away from the measure $\eta \Gamma(-1-\beta)^{-1}y^{-2-\beta} dy$.
\item
    In particular, if \[\pi(dy)=\eta \Gamma(-1-\beta)^{-1}y^{-2-\beta} dy,\] then the branching mechanism takes the form: \[\psi(z)=-\alpha z + \rho z^2 + \eta z^{1+\beta},\quad z\geq 0.\]
\end{itemize}
\end{frame}

\section{Preliminary}
\begin{frame}{Preliminary/Invariant measure of OU semigroup}
\begin{itemize}
\item
  Recall that $(P_t)_{t\geq 0}$ is the transition semigroup of the OU process $\xi$.
\item
  $(P_t)_{t\geq 0}$ has the invariant density
\begin{equation}
    \varphi(x)dx
    :=\Big (\frac{b}{\pi \sigma^2}\Big )^{d/2}\exp \Big(-\frac{b}{\sigma^2}|x|^2 \Big)dx,
    \quad x\in \mathbb R^d.
\end{equation}
\item
    Let $L^2(\varphi)$ be the Hilbert space with inner product
\begin{equation}
    \langle f_1, f_2 \rangle_{\varphi}
    := \int_{\mathbb R^d}f_1(x)f_2(x)\varphi(x).
\end{equation}
\end{itemize}
\end{frame}

\begin{frame}{Preliminary/Hermite polynomials}
\begin{itemize}
\item
    For each multi-index $p = (p_k)_{k = 1}^d \in \mathbb{Z}_+^{d}$, write 
\begin{equation}
	|p|:=\sum_{k=1}^d p_k;
	\quad p!:= \prod_{k= 1}^d p_k!;
	\quad \frac{\partial^p} {\partial x^p}:= \prod_{k = 1}^d\frac{\partial^{p_k}}{\partial x_k^{p_k}}.
\end{equation}
\item
    The \emph{Hermite polynomials} are given by
\begin{equation}
    H_p(x)
    :=(-1)^{|p|}\exp(|x|^2) \frac{\partial ^{p}}{\partial x^p} \exp(-|x|^2) ,
    \quad x\in \mathbb R^d,
    p \in \mathbb{Z}_+^{d}.
\end{equation}
\end{itemize}
\end{frame}

\begin{frame}{Preliminary/Spectrum of OU semigroup}

\begin{lemma}[Metafune, Pallara and Priola 2002]
	$(P_t)_{t\geq 0}$ is a strongly continuous semigroup in $L^2(\varphi)$ and its generator $L$ has discrete spectrum $\sigma(L)= \{-bk: k \in \mathbb Z_+\}$.
    For each $k \in \mathbb Z_+$, denote by $\mathcal{A}_k$ the eigenspace corresponding to the eigenvalue $-bk$, then
\[
    \mathcal{A}_k
    = \operatorname{Span} \{\phi_p : p\in \mathbb Z_+^d, |p|=k\},
\]
    where
\begin{equation}\label{eigenfunction}
    \phi_p(x)
    := \frac{1}{\sqrt{ p! 2^{|p|} }} H_p \Big(\frac{ \sqrt{b} }{\sigma}x \Big),
    \quad x\in \mathbb R^d, p\in \mathbb Z_+^d.
\end{equation}
	Moreover, $\{\phi_p:p\in \mathbb Z_+^d\}$ forms a complete orthonormal basis for $L^2(\varphi)$.
\end{lemma}
\end{frame}

\begin{frame}{Preliminary/Order of a testing function}
\begin{itemize}
\item
    For each function $f\in L^2(\varphi)$, denote by
\begin{equation}
    \kappa_f
    :=\inf \left \{k\geq 0: \exists ~ p\in \mathbb Z_+^d ,{\rm ~s.t.~}|p|=k {\rm ~and~}  \langle f, \phi_p \rangle_{\varphi}\neq 0\right \},
\end{equation}
    the order of the function $f$. 
\item
	We say a function $f\in \mathcal B(\mathbb R^d, \mathbb R)$ is of polynomial growth if there exists constants $C,n>0$ such that 
\[
	|f(x)|\leq C(1+|x|)^n,
	\quad \forall x\in \mathbb R^d.
\]
\item
    $\mathcal P$: the collection of all functions of polynomial growth.
\end{itemize}
\end{frame}

\section{Results}
\begin{frame}{Results/Phase transition}
\begin{itemize}
\item
  Recall that $\alpha$ is the branching rate.
\item
	It turns out that the CLTs of $\left\langle f,X_t \right\rangle$ are different in three different regimes
    depending on the sign of $\alpha-\frac{\kappa_f b (1+\beta)}{\beta}$:
\begin{itemize}
\item
  small branching rate regime: $\alpha < \frac{\kappa_f b (1+\beta)}{\beta}$;
\item
  critical branching rate regime: $\alpha = \frac{\kappa_f b (1+\beta)}{\beta}$;
\item
  large branching rate regime: $\alpha > \frac{\kappa_f b (1+\beta)}{\beta}$.
\end{itemize}
\end{itemize}
\end{frame}
\begin{frame}{Results/Small branching rate regime/ CLT}
\begin{theorem}[Ren, Song, S. and Zhao 2019+]
    Let $f\in \mathcal{P}\setminus\{0\}$ satisfy $\alpha\beta<\kappa_f b(1+\beta)$.
    Let $\mu\in \mathcal M(\mathbb R^d)$ have compact support. Then under $\mathbb{P}_{\mu}(\cdot|D^c)$, it holds that
\[
    \|X_t\|^{-\frac{1}{1+\beta}} \langle f,X_t\rangle\xrightarrow[t\rightarrow \infty]{d} \zeta.
\]
    Here, $\zeta$ is a $(1+\beta)$-stable random variable with characteristic function $\theta \mapsto \exp( m[\theta f])$ with 
    \begin{equation}
      m[\theta f]
      :=\eta \int_0^{\infty} e^{-\alpha s} ~ds\int_{\mathbb R^d} \big(-i \theta P_s^\alpha f(x)\big)^{1+\beta} \varphi(x)~dx.
    \end{equation}
\end{theorem}
\end{frame}

\begin{frame}{Results/Critical branching rate regime/CLT}
\begin{theorem}[Ren, Song, S. and Zhao 2019+]
\label{thm: critical clt}
    Let $f\in \mathcal{P}\setminus\{0\}$ satisfy $\alpha\beta=\kappa_f b(1+\beta)$.
    Let $\mu\in \mathcal M(\mathbb R^d)$ have compact support. Then under $\mathbb{P}_{\mu}(\cdot|D^c)$, it holds that
\[
    (t\|X_t\|)^{-\frac{1}{1+\beta}} \langle f,X_t\rangle
        \xrightarrow[t\to \infty]{d} \widetilde{\zeta}.
\]
    Here, $\widetilde{\zeta}$ is a $(1+\beta)$-stable random variable with
    characteristic function $\theta \mapsto \exp( \widetilde m[\theta f])$ where
    \[
      \widetilde{m}[\theta f]
      := \eta\int_{\mathbb R^d} \left(-i \theta \sum_{p\in \mathbb Z_+^d:|p|=\kappa_f}\langle f,\phi_p\rangle_\varphi \phi_p(x)\right)^{1+\beta} \varphi(x)~dx.
    \]
\end{theorem}
\end{frame}

\begin{frame}{Results/Large branching rate regime/ Martingale limit}
\begin{lemma}[Ren, Song, S. and Zhao 2019]
	For each multi-index $p\in \mathbb Z_+^d$ with $\alpha \beta > |p|b(1+\beta)$, each $\gamma \in (0,\beta)$ and each $\mu \in \mathcal M(\mathbb R^d)$ with compact support, the following martingale
\[
	H_t^p:= e^{-(\alpha-|p|b) t} \langle \phi_p, X_t\rangle, \quad t\geq 0,
\]
	is bounded in $L^{1+\gamma}(\mathbb P_\mu)$. 
	Thus the limit $H_\infty^p:= \lim_{t\to \infty} H_t^p$ exists $\mathbb P_\mu$-almost surely and in $L^{1+\gamma}(\mathbb P_\mu)$.
\end{lemma}
\end{frame}

\begin{frame}{Results/Large branching rate regime/LLNs}
\begin{theorem}[Ren, Song, S. and Zhao 2019+]
	Let $f \in \mathcal{P}\setminus\{0\}$ satisfy $\alpha\beta>\kappa_fb(1+\beta)$. 
	Then for each $\gamma\in (0, \beta)$ and $\mu\in \mathcal M(\mathbb R^d)$ with compact support, it holds that
\[
    e^{-(\alpha-\kappa_fb)t}\langle f, X_t\rangle
       \xrightarrow[t\to \infty]{}\sum_{p\in \mathbb Z_+^d:|p|=\kappa_f}\langle f, \phi_p\rangle_{\varphi} H_{\infty}^p
    \quad in~ L^{1+\gamma}(\mathbb{P}_{\mu}).
\]
    Moreover, if $f$ is twice differentiable and all its second order partial derivatives are in $\mathcal{P}$, then we also have almost sure convergence.
\end{theorem}
\end{frame}

\begin{frame}{Results/Spectrum decomposition}
\begin{itemize}
\item
	Define
\begin{align}
	&\mathcal{C}_l
   :=\overline{Span}\{\phi_p: \alpha\beta>|p|(1+\beta)b \}.
	\\ &\mathcal{C}_c
  :=\overline{Span}\{\phi_p: \alpha\beta=|p|(1+\beta)b \}.
	\\&\mathcal{C}_s
  :=\overline{Span}\{\phi_p: \alpha\beta >|p|(1+\beta)b \}.
\end{align}
\item
	For any $f\in \mathcal P\setminus\{0\}$, there is a unique decomposition: 
\[
	f := f_l + f_c + f_s
\]
	where $f_l \in \mathcal C_l, f_c\in \mathcal C_c$ and $f_s\in \mathcal C_s$.
\item
	CLTs for $f_s$ and $f_c$ are already established.
\end{itemize}
\end{frame}

\begin{frame}{Results/Large branching rate regime/CLTs}
\begin{theorem}[Ren, Song, S. and Zhao 2019+]
\label{thm: large clt}
    Let $f\in \mathcal{C}_l\setminus\{0\}$.
    Let $\mu\in \mathcal M(\mathbb R^d)$ have compact support. 
    Then under $\mathbb{P}_{\mu}(\cdot|D^c)$, it holds that
\begin{align}\label{thm: large rate}
    \frac{\langle f, X_t\rangle-\sum_{p\in\mathcal{N}}\langle f,\phi_p\rangle_\varphi e^{(\alpha-|p|b)t}H^p_{\infty}}{\|X_t\|^\frac{1}{1+\beta}}\xrightarrow[t\to \infty]{d}\bar{\zeta}.
\end{align}
    Here $\bar{\zeta}$ is a $(1+\beta)$-stable random variable with characteristic function $\theta \mapsto \exp( \bar m[\theta f])$ where
\[
	\bar{m}[\theta f]
  := \eta\int_0^\infty e^{\alpha s}~ds \int_{\mathbb R^d} \big(i\theta I_sf(x)\big)^{1+\beta} \varphi(x)~dx,
  \quad f\in \mathcal C_l
\]
and $(I_t)_{t\geq 0}$ are the inverse operator of $(P_t^\alpha)_{t\geq 0}$ on $\mathcal{C}_l$.
\end{theorem}
\end{frame}

\begin{frame}{Remark/Phase transition}
\begin{itemize}
\item
	Similar phase transition phenomenon in the context of branching OU processes is first observed by Adamczak and Milos 2015.
 
\item
	Coarsening effect (increasing of the spatial inequalities): A consequence of branching. Simply an area with more particles will produce more offspring.
\item
	Smoothing effect (decreasing of the spatial inequalities):  
	A consequence of the mixing property of the OU process. Simply each OU particles will ``forget'' its initial position exponentially fast.
\item
	The phase transition phenomenon is due to an interplay of the coarsening effect and the smoothing effect.
\end{itemize}
\end{frame}
\begin{frame}
  \centering \Large
  \emph{Thanks!}
\end{frame}

\end{document}
