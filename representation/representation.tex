\documentclass[9pt]{beamer}
\usepackage{lmodern}
\usepackage{mathrsfs}
\usepackage{mathtools}
	\mathtoolsset{showonlyrefs}
\usepackage{graphicx}
\usepackage[export]{adjustbox}
\usepackage[style=authoryear-icomp,maxbibnames=10,maxcitenames=10,uniquename=false]{biblatex}
\usepackage{comment}
\usetheme{Frankfurt}
\usecolortheme{seahorse}
\usecolortheme{rose}
\usefonttheme{serif}
\usefonttheme{structurebold}
\setbeamerfont{title}{shape=\itshape,family=\rmfamily}
\setbeamercolor{title}{fg=red!80!black,bg=red!20!white}
\title{Stable Central Limit Theorems for Super Ornstein-Uhlenbeck Processes}
\author{\bf Zhenyao Sun\inst{1}}
\institute{
	Based on a joint work with Yan-Xia Ren\inst{1,2}, Renming Song\inst{3} and Jianjie Zhao\inst{1}
\and
\inst{1}
	School of Mathematical Sciences, Peking University
\and
\inst{2}
	Center for Statistical Science, Peking University
\and
\inst{3}
	Department of Mathematics, University of Illinois at Urbana-Champaign
}
\date{
	Beijing Normal University,\\
	April, 2019}
\addbibresource{/Users/zhenyao/Repository/orggtd/bib.bib}
\begin{document}
\begin{frame}
  \titlepage
\end{frame}

\section{Background and Motivation}
\begin{frame}[allowframebreaks]{Model}
\begin{itemize}
\item
	In this talk, we consider a measure-valued Markov process $(X_t)_{t\geq 0}$ known as the superprocess.
\item
	Let $E$ be a locally compact separable metric space.
\item
	Let the spatial motion $(\xi_t)_{t\geq 0}$ be an $E$-valued Hunt process with transition semigroup $(P_t)_{t\geq 0}$. 
\item
	Let the branching mechanism $\psi\in \mathcal B(\mathbb R_+, \mathbb R)$ be given by 
\begin{equation} \label{eq: honogeneou branching mechanism}
    \psi(z)=
    - \alpha z + \rho z^2 + \int_{(0,\infty)} (e^{-zy} - 1 + zy) \pi(dy),
    \quad  z \in \mathbb R_+,
\end{equation}
	where $\alpha > 0 $, $\rho \geq0$ and $\pi$ is a measure on $(0,\infty)$ with 
\[
	\int_{(0,\infty)}(y\wedge y^2) \pi(dy)< \infty.
\]
\item
	Denote by $\mathcal M(E)$ the space of all finite Borel measures on $E$ equipped with the weak convergence.
\item
	For each $f\in \mathcal B(E, \mathbb R)$ and $\mu \in \mathcal M(E)$, write $\langle f,\mu\rangle = \int f(x)\mu(dx)$ whenever the integral make sense.
\item
	We say a function $f\in \mathcal B(\mathbb R_+\times E, \mathbb R)$ is \emph{locally bounded} if
\[
    \sup_{s\in [0,t],x\in E} |f(s,x)|
    <\infty,
    \quad t\in \mathbb R_+.
\]
\end{itemize}
\begin{definition}[Superprocess, \cite{Dynkin1991Branching}]
	Say an $\mathcal M(E)$-valued Markov process $\{(X_t)_{t\geq 0}; (\mathbb P_\mu)_{\mu\in \mathcal M_f(E)}\}$ is a $(\xi,\psi)$-superprocess if for each $f\in \mathcal B_b(E, \mathbb R^+)$, we have
\begin{equation}
\label{eq: def of V_t}
    \mathbb{P}_{\mu}[e^{-\langle f,X_t \rangle}]
    = e^{-\langle V_tf, \mu \rangle},
    \quad t\geq 0, \mu \in \mathcal M(E),
\end{equation}
	where $(t,x) \mapsto V_tf(x)$ is the unique locally bounded non-negative solution to the equation
\begin{equation}
	V_tf(x) +  \int_0^t P_{t-s}(\psi \circ V_{t-s}f)(x)ds
	= P_tf(x),
    \quad x\in E, t\geq 0.
\end{equation}
\end{definition}
\begin{itemize}
\item
	When $E=\mathbb R^d$ and $\xi$ is a Brownian motion on $\mathbb R^d$, $X$ is called a super Brownian motion. 
	When $E=\mathbb R^d$ and $\xi$ is a diffusion process on $\mathbb R^d$, $X$ is called a super diffusion process. 
	When $E=\mathbb R^d$ and $\xi$ is an OU process on $\mathbb R^d$ with generator
\[
	Lf(x)=\frac{1}{2}\sigma^2\Delta f(x)-bx\cdot \nabla f(x)
\]
	with $b, \sigma$ being positive constants, $X$ is called a super OU process.
\item
	Superprocess are high-density limits of 
	{\color{red} branching particle systems} (\cite{Watanabe1968A-limit}, \cite{Dawson1975Stochastic}, \cite{Dynkin1991Branching}), 
	{\color{red} long-range contact process} (\cite{MullerTribe1995Stochastic}, \cite{DurrettPerkins1999Rescaled}), 
	{\color{red} voter model } (\cite{CoxDurrettPerkins2000Rescaled}), and 
	{\color{red} long range percolation} (\cite{LalleyZheng2010Spatial}). 
\end{itemize}
\begin{example}[\cite{Li2011Measure-valued}]
	Let $\alpha >0$ and $\beta \in (0,1)$.
	Consider a Branching OU-process with
\begin{itemize}
\item
	{\color{red} $k$} initial particles;
\item
	killing rate  $2{\color{red}k}^\beta$;
\item
	offspring generating function
\[
	F_k(x) =x +\frac{1}{2}(1-x)^{1+\beta} + \frac{-\alpha (1-x)}{2{\color{red}k}^\beta}.
\]
\end{itemize}
	Let $X^{({\color{red} k})}_t(A)$ be number of particles in a Borel set $A$ at time $t$. 
	Then,
\[
	\Big(\frac{1}{\color{red} k} X^{({\color{red} k})}_t(\cdot)\Big)_{t\geq 0}
	\xrightarrow[{\color{red}k}\to \infty]{w} (\xi,\Psi)\text{-superprocess}
\]
	with $\xi$ being an OU process and $\Psi(z) = -\alpha z + z^{1+\beta}$.
\end{example}
\end{frame}
\begin{frame}[allowframebreaks]{Background}
\begin{itemize}
\item
	There have been quite a few papers on laws of large numbers for superprocesses. 
\item
	\cite{Englander2009Law,EnglanderWinter2006Law,EnglanderTuraev2002A-scaling} established some {\color{red}weak laws of large numbers}.
\item
	\cite{ChenRenWang2008An-almost,Wang2010An-almost,LiuRenSong2013Strong,KouritzinRen2014A-strong,ChenRenSongZhang2015Strong,EckhoffKyprianouWinkel2015Spines,ChenRenYang2019Skeleton} established {\color{red}strong law of large numbers} for superprocesses under different settings.
\end{itemize}
\begin{example}[\cite{ChenRenYang2019Skeleton}]
	Suppose that $X$ is a super OU process with branching mechanism $\psi(z)=-\alpha z+ z^{1+\beta}$ with $\alpha>0$ and $\beta \in (0,1)$. Then, for each $\mu \in \mathcal M(\mathbb R^d)$ and each continuous bounded non-negative function $f$ on $\mathbb R^d$, we have
\[
	e^{-\alpha t} \langle f, X_t\rangle \xrightarrow[t\to \infty]{\mathbb P_\mu \text{-a.s.}, L^1(\mathbb P_\mu)} H_\infty \langle f, \varphi\rangle,
\]
	where $H_\infty$ is the limit of the non-negative martingale $e^{-\alpha t}\langle 1,X_t\rangle$ and $\varphi$ is the invariant density of the OU process $\xi$.
\end{example}
\begin{block}{Motivation}
	We are interested in the spatial central limit theorems (CLT) for superprocesses.
	More precisely, we want to find
    $(F_t)_{t\geq 0}$ and $(G_t)_{t\geq 0}$ such that
\[
    \frac{\langle f, X_t \rangle -G_t}{F_t}
\]
    converges weakly to some non-degenerate random variable as $t\rightarrow\infty$, for a large class of superprocesses $(X_t)_{t\geq 0}$ and testing functions $f$.
\end{block}
\begin{itemize}
\item
	There are many papers studying CLT for branching processes, branching diffusions and superprocesses.
\item
	For CLT of supercritical {\color{red} Galton-Watson processes}, see \cite{Heyde1970A-rate,HeydeBrown1871An-invariance,HeydeLeslie1971Improved}.
\item 
	For CLT of supercritical  {\color{red} multi-type Galton-Watson processes}, see \cite{Athreya1969Limit,Athreya1969LimitB,Athreya1971Some}.
\item
	For CLT of some general supercritical {\color{red} branching Markov processes}, see \cite{AsmussenHering1983Branching}.
\item
	For CLT of supercritical {\color{red} branching OU processes} with {\color{red} binary branching} mechanism, see \cite{AdamczakMilos2015CLT}.
\item
	For CLT of some supercritical {\color{red} superprocesses} with branching mechanisms satisfying a {\color{red} fourth moment condition}, see \cite{Milos2018Spatial}.
\item
	For CLT of supercritical {\color{red} super OU processes} with branching mechanisms satisfying only a  {\color{red} second moment condition}, see \cite{RenSongZhang2014Central}.
\item
	For a series of CLT for a large class of general supercritical {\color{red} branching Markov processes} and {\color{red} superprocesses} with {\color{red} spatially dependent branching mechanisms} satisfying a {\color{red} second moment condition}, see \cite{RenSongZhang2014CentralB,RenSongZhang2015Central,RenSongZhang2017Central}.
\item
    All CLTs described above hold for systems with branching mechanisms with finite variance. 
    It is natural to ask whether there are counterparts of CLTs for the cases when the branching mechanisms are of infinite variance.
\item
	For stable CLT for supercritical {\color{red} Galton-Watson process} whose {\color{red} offspring distribution
    belongs to the domain of attraction of a stable law of index $\alpha\in (1, 2]$}, see \cite{Heyde1971Some}.
\item
	For stable CLT for supercritical {\color{red} multi-type Galton-Watson process} and supercritical {\color{red} continuous time branching processes}, see \cite{Asmussen1976Convergence}.
\item
	Recently, some stable CLTs for supercritical {\color{red} branching OU processes} with offspring generating function
\[
    F(s)
    = ms -(m-1) +(m-1) (1-s)^{1+\beta},
    \quad |s|< 1
\]
	with $m>1$ and $\beta\in (0, 1)$ are established in \cite{MarksMilos2018CLT}.
\item
	Very recently, some stable fluctuation results of Biggins’ martingales in the context of {\color{red} branching random walks} are established in \cite{IksanovKoleskoMeiners2018Fluctuations}.
\end{itemize}
\end{frame}
\section{Results}
\begin{frame}[allowframebreaks]{Settings}
\begin{itemize}
\item
	Let $E= \mathbb R^d$.
\item
	Let $(\xi_t)_{t\geq 0}$ be an $\mathbb R^d$-valued Ornstein-Uhlenbeck (OU) process with generator
\[
	Lf(x)
	=\frac{1}{2} \sigma^2 \Delta f(x)-bx\cdot\nabla f(x),
	\quad x\in \mathbb R^d, f\in C^2(\mathbb R^d),
\]
	where $\sigma,b>0$.
\end{itemize}
\begin{block}{Assumption 1. (Grey's condition)}
	There exists $z' > 0$ such that $\psi(z) > 0$ for all $z>z'$ and  $\int_{z'}^\infty \psi(z)^{-1}dz < \infty$.
\end{block}
\begin{itemize}
\item
	It is known  that, under Assumption 1, 
	the \emph{extinction event} 
	$D:=\{\exists t\geq 0,~\text{s.t.}~\|X_t\|=0\}$ has positive probability, with respect to $\mathbb P_\mu$ for each  $\mu \in \mathcal M(\mathbb R^d)$.
\item
	Denote by $\Gamma$ the gamma function.
	For any $\sigma$-finite signed measure $\mu$, we use $|\mu|$ to denote the total variation measure of $\mu$.
\end{itemize}
\begin{block}{Assumption 2.}
	There exist constants $\eta > 0$ and $\beta \in (0,1)$ such that
\begin{equation}
    \int_{(1,\infty)}y^{1+\beta +\delta}~\Big|\pi(dy)-\frac{\eta~dy}{\Gamma(-1-\beta)y^{2+\beta}}\Big| <\infty,
\end{equation}
	for some $\delta \in (0,1-\beta)$.
\end{block}
\begin{itemize}
\item
	Assumption 2 says that 
 there exist constants $\eta>0$ and $\beta > 0$ 
    such that the L\'evy measure $\pi(dy)$ is not too far away from the measure $\eta \Gamma(-1-\beta)^{-1}y^{-2-\beta} dy$.
\item
    In particular, if \[\pi(dy)=\eta \Gamma(-1-\beta)^{-1}y^{-2-\beta} dy,\] then the branching mechanism takes the form: \[\psi(z)=-\alpha z + \rho z^2 + \eta z^{1+\beta},\quad z\geq 0.\]
\end{itemize}
\begin{lemma}[\cite{RenSongSunZhao2019Stable}]
	Suppose that Assumption  2 holds.
	Suppose that there are constants $\eta'>0, \beta'\in (0,1)$ and $\delta' \in (0, 1-\beta)$ such that
\[
	 \int_{(1,\infty)}y^{1+\beta' +\delta'}~\Big|\pi(dy)-\frac{\eta'~dy}{\Gamma(-1-\beta)y^{2+\beta'}}\Big| <\infty.
\]
	Then $\eta'= \eta$ and $\beta ' = \beta$.
\end{lemma}
\begin{itemize}
\item
	Therefore, $\eta$ and $\beta$ are uniquely determined by the L\'evy measure $\pi$.
\end{itemize}
\begin{lemma}[\cite{RenSongSunZhao2019Stable}]
	If $\psi$ satisfies Assumption  2, then $\psi$ satisfies the $L \log L$ condition, i.e.,
\[
    \int_{(1,\infty)} y \log y~\pi(dy)< \infty.
\]
\end{lemma}
\begin{itemize}
\item
	This guarantees that $H_\infty$, the limit of the non-negative martingale $(e^{-\alpha t} \|X_t\|)_{t\geq 0}$, is non-degenerate.
\end{itemize}
\end{frame}
\begin{frame}[allowframebreaks]{OU semigroup}
\begin{itemize}
\item
	We use $(P_t)_{t\geq 0}$ to denote the transition semigroup of the OU process $\xi$.
\item
	Define
\[
    P^{\alpha}_t f(x)
    :=
    e^{\alpha t} P_t f(x) =
    \Pi_x [e^{\alpha t}f(\xi_t)],
    \quad x\in \mathbb R^d,t\geq 0, f\in \mathcal B(\mathbb R^d, \mathbb R_+).
\]
	Then $(P^\alpha_t)_{t\geq 0}$ is the mean semigroup of the super OU process $X$, in the sense that
\[
    \mathbb{P}_{\mu}[\langle f, X_t \rangle]
    = \langle P^\alpha_t f, \mu \rangle,
    \quad t\geq 0, f\in \mathcal B(\mathbb R^d, \mathbb R_+), \mu \in \mathcal M(\mathbb R^d).
\]
\item
	The limiting behavior of the super OU process is closely related to the asymptotic property of its mean semigroup $(P^\alpha_t)_{t\geq 0}$, and therefore, to the property of the OU semigroup $(P_t)_{t\geq 0}$.
\item
    The OU process $(\xi_t)_{t\geq 0}$ has an invariant density
\begin{equation}
    \varphi(x)dx
    :=\Big (\frac{b}{\pi \sigma^2}\Big )^{d/2}\exp \Big(-\frac{b}{\sigma^2}|x|^2 \Big)dx,
    \quad x\in \mathbb R^d.
\end{equation}
\item
    Let $L^2(\varphi):= \left\{ h  \in \mathcal B(\mathbb R^d, \mathbb R): \int_{\mathbb R^d} |h(x)|^2 \varphi(x) dx < \infty \right\}$.
    Then $L^2(\varphi)$ is a Hilbert space with inner product
\begin{equation}
    \langle f_1, f_2 \rangle_{\varphi}
    := \int_{\mathbb R^d}f_1(x)f_2(x)\varphi(x) dx, \quad f_1,f_2 \in L^2(\varphi).
\end{equation}
\item
    For each multi-index $p = (p_k)_{k = 1}^d \in \mathbb{Z}_+^{d}$, write 
\begin{equation}
	|p|:=\sum_{k=1}^d p_k;
	\quad p!:= \prod_{k= 1}^d p_k!;
	\quad \frac{\partial^p} {\partial x^p}:= \prod_{k = 1}^d\frac{\partial^{p_k}}{\partial x_k^{p_k}}.
\end{equation}
\item
    The \emph{Hermite polynomials} are given by
\begin{equation}
    H_p(x)
    :=(-1)^{|p|}\exp(|x|^2) \frac{\partial ^{|p|}}{\partial x^p} \exp(-|x|^2) ,
    \quad x\in \mathbb R^d,
    p \in \mathbb{Z}_+^{d}.
\end{equation}
\begin{lemma}[\cite{MetafunePallaraPriola2002Spectrum}]
	$(P_t)_{t\geq 0}$ is a strongly continuous semigroup in $L^2(\varphi)$ and its generator $L$ has discrete spectrum $\sigma(L)= \{-bk: k \in \mathbb Z_+\}$.
    For each $k \in \mathbb Z_+$, denote by $\mathcal{A}_k$ the eigenspace corresponding to the eigenvalue $-bk$, then
\[
    \mathcal{A}_k
    = \operatorname{Span} \{\phi_p : p\in \mathbb Z_+^d, |p|=k\},
\]
    where
\begin{equation}\label{eigenfunction}
    \phi_p(x)
    := \frac{1}{\sqrt{ p! 2^{|p|} }} H_p \Big(\frac{ \sqrt{b} }{\sigma}x \Big),
    \quad x\in \mathbb R^d, p\in \mathbb Z_+^d.
\end{equation}
	Moreover, $\{\phi_p:p\in \mathbb Z_+^d\}$ forms a complete orthonormal basis for $L^2(\varphi)$.
\end{lemma}
\item
    For each function $f\in L^2(\varphi)$, denote by
\begin{equation}
    \kappa_f
    :=\inf \left \{k\geq 0: \exists ~ p\in \mathbb Z_+^d ,{\rm ~s.t.~}|p|=k {\rm ~and~}  \langle f, \phi_p \rangle_{\varphi}\neq 0\right \},
\end{equation}
    the order of the function $f$. 
\item
	We say a function $f\in \mathcal B(\mathbb R^d, \mathbb R)$ is of polynomial growth if there exists constants $C,n>0$ such that 
\[
	|f(x)|\leq C(1+|x|)^n,
	\quad \forall x\in \mathbb R^d.
\]
\item
    Let $\mathcal P \subset L^2(\varphi)$ the collection of all functions of polynomial growth.
\end{itemize}
\begin{block}{Goal in specific}
	Let $(X_t)_{t\geq 0}$ be the super OU process which satisfies Assumption 1 and 2.
	Let $f\in \mathcal P$ and $f\not \equiv 0$.
	We want to find
    $(F_t)_{t\geq 0}$ and $(G_t)_{t\geq 0}$ such that
\[
    \frac{\langle f, X_t \rangle -G_t}{F_t}
\]
    converges weakly to some non-degenerate random variable as $t\rightarrow\infty$.
\end{block}
\begin{itemize}
\item
	It turns out that the statements of the results are different in three different regimes
    depending on the sign of $\alpha\beta-\kappa_f b (1+\beta)$.
\end{itemize}
\end{frame}
\begin{frame}{Small branching rate regime: $\alpha\beta < \kappa_f b (1+\beta)$}
\begin{lemma}[\cite{MarksMilos2018CLT}]
	Let $f\in \mathcal{P}\setminus\{0\}$ satisfy $\alpha\beta<\kappa_f b(1+\beta)$.
	Then the following integral is well defined:
\begin{equation}
    m[f]
    :=\eta \int_0^{\infty} e^{-\alpha s} ~ds\int_{\mathbb R^d} \big(-iP_s^\alpha f(x)\big)^{1+\beta} \varphi(x)~dx.
\end{equation}
	Moreover, $\theta \mapsto \exp( m[\theta f])$ is the characteristic function of a $(1+\beta)$-stable random variable.
\end{lemma}
\begin{theorem}[\cite{RenSongSunZhao2019Stable}]
    Let $f\in \mathcal{P}\setminus\{0\}$ satisfy $\alpha\beta<\kappa_f b(1+\beta)$.
    Let $\mu\in \mathcal M(\mathbb R^d)$ have compact support. Then under $\mathbb{P}_{\mu}(\cdot|D^c)$, it holds that
\[
    \|X_t\|^{-\frac{1}{1+\beta}} \langle f,X_t\rangle\xrightarrow[t\rightarrow \infty]{d} \zeta,
\]
    where $\zeta$ is a $(1+\beta)$-stable random variable with characteristic function $\theta \mapsto \exp( m[\theta f])$.
\end{theorem}
\end{frame}
\begin{frame}{Critical branching rate regime: $\alpha\beta = \kappa_f b (1+\beta)$}
\begin{lemma}[\cite{MarksMilos2018CLT}]
	Let $f\in \mathcal{P}\setminus\{0\}$ satisfy $\alpha\beta=\kappa_f b(1+\beta)$, then the following integral is well defined
\[
	\widetilde{m}[f]
    := \eta\int_{\mathbb R^d} \Big(-i\sum_{p\in \mathbb Z_+^d:|p|=\kappa_f}\langle f,\phi_p\rangle_\varphi \phi_p(x)\Big)^{1+\beta} \varphi(x)~dx.
\]
	Moreover, $\theta \mapsto \exp( \widetilde m[\theta f])$ is the characteristic function of a $(1+\beta)$-stable random variable.
\end{lemma}
\begin{theorem}[\cite{RenSongSunZhao2019Stable}]
\label{thm: critical clt}
    Let $f\in \mathcal{P}\setminus\{0\}$ satisfy $\alpha\beta=\kappa_f b(1+\beta)$.
    Let $\mu\in \mathcal M(\mathbb R^d)$ have compact support. Then under $\mathbb{P}_{\mu}(\cdot|D^c)$, it holds that
\[
    (t\|X_t\|)^{-\frac{1}{1+\beta}} \langle f,X_t\rangle
        \xrightarrow[t\to \infty]{d} \widetilde{\zeta},
\]
    where $\widetilde{\zeta}$ is a $(1+\beta)$-stable random variable with
    characteristic function $\theta \mapsto \exp( \widetilde m[\theta f])$.
\end{theorem}
\end{frame}
\begin{frame}[allowframebreaks]{Large branching rate regime: $\alpha\beta > \kappa_f b (1+\beta)$.}
\begin{lemma}[\cite{RenSongSunZhao2019Stable}]
	For each multi-index $p\in \mathbb Z_+^d$ with $\alpha \beta > |p|b(1+\beta)$, each $\gamma \in (0,\beta)$ and each $\mu \in \mathcal M(\mathbb R^d)$ with compact support, the following martingale
\[
	H_t^p:= e^{-(\alpha-|p|b) t} \langle \phi_p, X_t\rangle, \quad t\geq 0,
\]
	is bounded in $L^{1+\gamma}(\mathbb P_\mu)$. 
	Thus the limit $H_\infty^p:= \lim_{t\to \infty} H_t^p$ exists $\mathbb P_\mu$-almost surely and in $L^{1+\gamma}(\mathbb P_\mu)$.
\end{lemma}
\begin{theorem}[\cite{RenSongSunZhao2019Stable}]
	Let $f \in \mathcal{P}\setminus\{0\}$ satisfy $\alpha\beta>\kappa_fb(1+\beta)$. 
	Then for each $\gamma\in (0, \beta)$ and $\mu\in \mathcal M(\mathbb R^d)$ with compact support, it holds that
\[
    e^{-(\alpha-\kappa_fb)t}\langle f, X_t\rangle
       \xrightarrow[t\to \infty]{}\sum_{p\in \mathbb Z_+^d:|p|=\kappa_f}\langle f, \phi_p\rangle_{\varphi} H_{\infty}^p
    \quad in~ L^{1+\gamma}(\mathbb{P}_{\mu}).
\]
    Moreover, if $f$ is twice differentiable and all its second order partial derivatives are in $\mathcal{P}$, then we also have almost sure convergence.
\end{theorem}
\begin{itemize}
\item
	Define
\begin{align}
	\mathcal{N}_l
	&:=\{p\in \mathbb{Z}_+^d:  \alpha\beta>|p|(1+\beta)b\},
	\quad \mathcal{C}_l
	:=Span\{\phi_p:p\in\mathcal N_l\}.
	\\\mathcal{N}_c
	&:=\{p\in \mathbb{Z}_+^d:  \alpha\beta=|p|(1+\beta)b\},
	\quad \mathcal{C}_c
	:=Span\{\phi_p:p\in\mathcal N_c\}.
	\\\mathcal{N}_s
	&:=\{p\in \mathbb{Z}_+^d:  \alpha\beta<|p|(1+\beta)b\},
	\quad \mathcal{C}_s
	:=\overline{Span}\{\phi_p:p\in\mathcal N_s\}.
\end{align}
\item
	For any $f\in \mathcal P\setminus\{0\}$, we have a unique decomposition: 
\[
	f := f_l + f_c + f_s
\]
	where $f_l \in \mathcal C_l, f_c\in \mathcal C_c$ and $f_s\in \mathcal C_s$.
\item
	Not that CLTs for $f_s$ and $f_c$ are already established.
\item
	For each $t\ge 0$, define an operator $I_t$ as the inverse of $P_t^\alpha$ on $\mathcal{C}_l$, i.e.
\begin{align}\label{definition of Itf}
    I_tf(x)
    :=\sum_{p\in \mathcal{N}}\langle f, \phi_p\rangle_{\varphi} e^{-(\alpha-|p|b)t}\phi_p(x),
   \quad x\in \mathbb{R}^d, f\in \mathcal C_l.
\end{align}
\end{itemize}
\begin{lemma}[\cite{RenSongSunZhao2019Stable}]
	Let $f\in \mathcal{C}_l\setminus\{0\}$, then the following integral is well defined
\[
	\bar{m}[f]
    := \eta\int_0^\infty e^{\alpha s}~ds \int_{\mathbb R^d} \big(iI_sf(x)\big)^{1+\beta} \varphi(x)~dx,
    \quad f\in \mathcal C_l.
\]
	Moreover, $\theta \mapsto \exp( \bar m[\theta f])$ is the characteristic function of a $(1+\beta)$-stable random variable.
\end{lemma}
\begin{theorem}[\cite{RenSongSunZhao2019Stable}]
\label{thm: large clt}
    Let $f\in \mathcal{C}_l\setminus\{0\}$.
    Let $\mu\in \mathcal M(\mathbb R^d)$ have compact support. 
    Then under $\mathbb{P}_{\mu}(\cdot|D^c)$, it holds that
\begin{align}\label{thm: large rate}
    \frac{\langle f, X_t\rangle-\sum_{p\in\mathcal{N}}\langle f,\phi_p\rangle_\varphi e^{(\alpha-|p|b)t}H^p_{\infty}}{\|X_t\|^\frac{1}{1+\beta}}\xrightarrow[t\to \infty]{d}\bar{\zeta},
\end{align}
    where $\bar{\zeta}$ is a $(1+\beta)$-stable random variable with
    characteristic function $\theta \mapsto \exp( \bar m[\theta f])$.
\end{theorem}
\end{frame}
\begin{frame}{Interpretation}
\begin{itemize}
\item
	A phase transition phenomenon for CLT of spatial Markovian branching systems is first observed by \cite{AdamczakMilos2015CLT} in the context of branching OU processes and by \cite{Milos2018Spatial} in the context of super OU processes. (Both assuming finite second moment.)
\item
	This phase transition phenomenon is due to an interplay of coarsening and smoothing. 
\item
	Coarsening means the increasing of the spatial inequalities. It is a consequence of branching: Simply an area with more particles will produce more offspring. The coarsening effect is quantified by the branching rate $\alpha$.
\item
	Smoothing means the decreasing of the spatial inequalities. 
	It is a consequence of the mixing property of the OU process. Imaging an offspring with birth position $x$. This OU particle will ``forget'' its initial position exponentially fast. This will reduce the spatial inequalities. 
	This smoothing effect is quantified by $b$ (the spectral gap of the OU semigroup).
\item
	Put $f\in \mathcal P$ with an order $\kappa_f \geq 1$. Consider an OU particle $\xi_t$. The expected value of $f(\xi_t)$ will decrease exponentially fast. This will also reduce the spatial inequality in the integration $\langle f,X_t\rangle$. 
	This smoothing effect is quantified by $\kappa_f$.
\end{itemize}
\end{frame}
\section{Methods}
\begin{frame}[allowframebreaks]{General Ideas}
\begin{itemize}
\item
	For any $\mu\in  \mathcal M(\mathbb R^d)$ and any random variable $Y$ with finite mean, define
\[
    \mathcal I_s^t Y:=\mathcal I_s^t [Y, \mu]
    := \mathbb P_\mu[Y|\mathcal F_t] - \mathbb P_\mu[Y|\mathcal F_s],\quad 0 \leq s \leq t <\infty.
\]
\item
	Then for any $f\in \mathcal{P}$, we have 
\[
    \langle f,X_t\rangle
    :=\sum_{k=0}^{\lfloor t \rfloor-1} \mathcal I_{t-k-1}^{t-k}\langle f ,X_t\rangle+\mathcal I_0^{t-\lfloor t \rfloor}\langle f ,X_t\rangle + \langle P^\alpha_tf,X_0\rangle,
    \quad t\geq 0.
\]
\item
	We can study the fluctuation of $\langle f,X_t\rangle$ by finding the fluctuation of each term $\mathcal I_{t-k-1}^{t-k}\langle f ,X_t\rangle$.
\item
	This idea can be traced back to the CLT of martingales. 
	Note that $(\mathbb P_\mu[\langle f,X_t\rangle|\mathscr F_s])_{0\leq s\leq t}$ is a martingale in $s$, therefore $\mathcal I_{t-k-1}^{t-k}\langle f ,X_t\rangle$ is simply the conditional increments of this martingale.
\item
	Taking $f = 1$ as an example. Note that, conditioned on $X_s = \mu$, the conditional distribution of $\mathcal I_s^{s+1}\langle 1, X_t\rangle$ is the same to the distribution of 
\[
	\langle 1, X_1\rangle - \|\mu\|
\]
	under probability $\mathbb P_{\mu}$.
\item
	Also note that, on the event of non-extinction, $\|X_s\|$ will grow exponentially.
	So from the branching property, if $s$ is very large, after some normalization, $\mathcal I_s^{s+1}\langle 1, X_t\rangle$ should distributed approximately like a stable random variable.
\item
	If the superprocess has finite second moment, then after some normalization $\mathcal I_s^{s+1}\langle 1, X_t\rangle$ will distributed like a Gaussian random variable. So we only need to calculate its variance.
\item
	However, under our settings, the superprocess has infinite second moment. So we need to develop new method to characterize the distribution of $\mathcal I_s^{s+1}\langle 1, X_t\rangle$.
\item
	If $\infty >\kappa_f \geq 1$ then $f$ must take both positive and negative values. 
	In this case, the Laplace transform of $\langle f,X_t\rangle$ may not exist.
\end{itemize}
\end{frame}
\begin{frame}[allowframebreaks]{Method}
\begin{itemize}
\item
	Note that $\langle f,X_t\rangle$ is an infinitely divisible random variable. Define its characteristic exponent as
	\[U_t(\theta f)(x) := \operatorname{Log} \mathbb P_{\delta_x}[e^{i\theta \langle f, X_t\rangle}],
	\quad t\geq 0, \theta \in \mathbb R, x\in \mathbb R^d.\]
	It can be proved that $-U_t(\theta f)(x)$ always takes values in $\mathbb C_+:= \{x+iy:x\geq 0, y\in \mathbb R\}$.
\item
	Define
\[
	\psi_0(z) = \psi(z)+\alpha z = \rho z^2 + \int_{(0,\infty)} (e^{-zy} - 1 + zy) \pi(dy),
    \quad  z \in \mathbb R_+.
\]
	It can be proved that $\psi_0$ can be uniquely extended as a complex-valued continuous functions on $\mathbb C_+$ which are holomorphic on $\mathbb C_+^0:= \{x+iy: x>0, y\in \mathbb R\}$.
\end{itemize}
\begin{lemma}[\cite{RenSongSunZhao2019Stable}]
    If $f\in \mathcal P$, then for all $t\geq 0$ and $x\in E$, we have
\begin{equation}
    U_tf(x) -  \int_0^t P_{t-s}^\alpha [\psi_0 \circ (- U_{s}f)](x)~ds
    = iP_t^{\alpha} f(x).
\end{equation}
\end{lemma}
\begin{itemize}
\item
	An analog of this Lemma for general non-persistent superprocesses with spatially dependent branching mechanisms is also developed in \cite{RenSongSunZhao2019Stable}.
\item
	Using the above Lemma and Assumption 2, we can derive:
\end{itemize}
\begin{lemma}[\cite{RenSongSunZhao2019Stable}]
    For all $g \in \mathcal P$ and $\mu \in \mathcal M(\mathbb R^d)$ with compact support, there exists a constant $C > 0$ such that for all $\lambda > 0$ and $0\leq r\leq s\leq t<\infty$ with $s-r \leq 1$, we have
\[
    \mathbb P_{\mu}(|\mathcal I_r^s\langle g, X_t\rangle|>\lambda)
    \leq C e^{\alpha r} \bigg(\Big( \frac{e^{(t-s)({\color{red}\alpha - \kappa_g b})}}{\lambda}\Big)^{1+\beta} + \Big( \frac{e^{(t-s)({\color{red}\alpha - \kappa_g b})}}{\lambda}\Big)^{2} \bigg).
\]
\end{lemma}
\begin{lemma}[\cite{RenSongSunZhao2019Stable}]
    For all $g \in \mathcal P$, $\mu \in \mathcal M_c(\mathbb R^d)$ and $\gamma\in (0, \beta)$, there exists a constant $C > 0$ such that for all $0\leq r\leq s\leq t<\infty$ with $s-r \leq 1$, we have
\[
    \mathbb P_\mu\big[|\mathcal I_r^s\langle g, X_t\rangle|^{1+\gamma}\big]
    \leq C e^{t\alpha+(t-s) (\gamma{\color{red}\alpha}- (1+\gamma){\color{red}\kappa_g b})}.
\]
\end{lemma}
\begin{lemma}[\cite{RenSongSunZhao2019Stable}]
    For all $g \in \mathcal P$, $\mu \in \mathcal M(\mathbb R^d)$ with compact support and $\gamma\in (0, \beta)$, there exists a constant $C > 0$ such that for each $t\geq 0$, we have
\begin{itemize}
\item
    $\|\langle g,X_t\rangle\|_{\mathbb{P}_{\mu};1+\gamma}\leq C e^{(\alpha-\kappa_g b)t}$ provided {\color{red}$\alpha\gamma >  (1+\gamma)\kappa_gb$};
\item
    $\|\langle g,X_t\rangle\|_{\mathbb{P}_{\mu};1+\gamma}\leq C te^{\frac{\alpha}{1+\gamma}t}$ provided {\color{red}$\alpha\gamma = (1+\gamma)\kappa_gb$};
\item
    $\|\langle g,X_t\rangle\|_{\mathbb{P}_{\mu};1+\gamma}\leq C e^{\frac{\alpha}{1+\gamma}t}$ provided {\color{red}$\alpha\gamma < (1+\gamma)\kappa_g b$}.
\end{itemize}
\end{lemma}
\begin{lemma}[\cite{MarksMilos2018CLT}]
	Let $f\in \mathcal{P}\setminus\{0\}$.
	Then the following integral is well defined:
\begin{equation}
    m_t[f]
    := \eta \int_t^{t+1}e^{-\alpha s}~ds\int_{\mathbb R^d} (-iP_{s}^\alpha f(x))^{1+\beta} \varphi(x)~dx,
    \quad t\geq 0.
\end{equation}
	Moreover, $\theta \mapsto \exp( m_t[\theta f])$ is the characteristic function of a $(1+\beta)$-stable random variable.
\end{lemma}
\begin{theorem}[\cite{RenSongSunZhao2019Stable}]
\label{thm: small clt}
    Let $f\in \mathcal{P}\setminus\{0\}$ satisfy $\alpha\beta\leq \kappa_f b(1+\beta)$.
    Let $\mu\in \mathcal M(\mathbb R^d)$ have compact support. Then under $\mathbb{P}_{\mu}(\cdot|D^c)$, it holds that
\[
    \big(\|X_t\|^{-\frac{1}{1+\beta}}\mathcal I^{t-k}_{t-k-1}\langle f,X_t\rangle\big)_{k=1}^n
    \xrightarrow [t\to \infty]{d} (\zeta_k)_{k=1}^n.
\]
    where $(\zeta_k)_{k \in \mathbb N}$ are independent $(1+\beta)$-stable random variables with characteristic functions
\[
     \mathbf E[e^{i\theta \zeta_k}] = \exp( m_k[\theta f]),\quad k \in \mathbb N.
\]
\end{theorem}
\begin{itemize}
\item
	For the small branching regime $\alpha \beta < \kappa_f b(1+\beta)$. The above Theorem roughly says that
\[
	\|X_t\|^{-\frac{1}{1+\beta}}\langle f,X_t\rangle \xrightarrow[t\to \infty]{d} \sum_{k=0}^\infty \zeta_k \overset{d}{=}: \zeta,
\]
	where $\zeta$ is a $(1+\beta)$-stable random variable with characteristic function
\[
	E[e^{i\theta \zeta}] = \exp(m[\theta f]) := \exp\Big(\sum_{k=0}^\infty m_k[\theta f]\Big).
\]
\item
	For the critical branching regime $\alpha \beta = \kappa_f b(1+\beta)$. The above Theorem roughly says that
\[
	(t\|X_t\|)^{-\frac{1}{1+\beta}}\langle f,X_t\rangle \overset{d}{\approx} t^{-\frac{1}{1+\beta}} \sum_{k=0}^{\lfloor t\rfloor} \zeta_k \xrightarrow[t\to \infty]{d} \widetilde \zeta
\]
	where $\widetilde \zeta$ is a $(1+\beta)$-stable random variable with characteristic function
\[
	E[e^{i\theta \widetilde \zeta}] = \exp(\widetilde m[\theta f]) := \exp\Big(\lim_{t\to \infty} \frac{1}{t} \sum_{k=0}^{\lfloor t\rfloor} m_k[\theta f]\Big),\quad \theta \in \mathbb R.
\]
\item
	For  testing functions $f\in \mathcal C_l$, the general idea is almost the same, except that we need to consider the decomposition over the time interval $[t,\infty)$. 
\item
	Taking $f = \phi_p$ with $\alpha \beta > |p|b(1+\beta)$ as an example, we do the following decomposition:
\[
    H^p_t-H^p_\infty=\sum^{\infty}_{n=1}(H^p_{t+n-1}-H^p_{t+n}).
\]
    We established the fluctuation behaviors of each terms.
\end{itemize}
\end{frame}
\begin{frame}[allowframebreaks]{Remarks}
\begin{itemize}
\item
	For a general $f$, we have a unique decomposition: $f=f_l+f_c+f_s$ where $f_l \in \mathcal C_l$, $f_c \in \mathcal C_c$ and $f_s\in \mathcal C_s$.
	Our main results give CLT for $\langle f_l, X_t\rangle$, $\langle f_c, X_t\rangle$ and $\langle f_s, X_t\rangle$, respectively.
\item
	We conjecture that the limits of these three terms, normalized properly, are independent.
	If this is valid, we can get a CLT for $\langle f, X_t\rangle$ for general $f\in\mathcal{P}$.
\item
	This independence was proved under the second moment condition by \cite{RenSongZhang2015Central}.
\item
	We only considered a special case: Super OU with spatial independent branching mechanism. 
	It is interesting to ask whether stable CLTs are valid for general superprocesses with spatially dependent branching mechanism.
\end{itemize}
\end{frame}
\begin{frame}[allowframebreaks]
	\frametitle{References}
	\printbibliography
\end{frame}
\end{document}



