
\documentclass[a4paper,12pt, reqno]{amsart}
\usepackage{amssymb,amsmath,latexsym}
\usepackage{color}%cyan,red;magenta,green;yellow,blue
\usepackage{comment}
\oddsidemargin 0in
\evensidemargin 0in
\topmargin -0.2in
\textwidth 6.6in
\textheight 9.6in
%\def\doublespace{\baselineskip=16pt}
\def\doublespace{\baselineskip=14pt}
%\usepackage{hyperref}
%\usepackage{showkeys}
\allowdisplaybreaks
%\def\singlespace{\baselineskip=12pt}
\def\singlespace{\baselineskip=10pt}
\newcommand{\red}{\color{red}}
\newcommand{\blue}{\color{blue}}
\newcommand{\magenta}{\color{magenta}}
\newcommand{\normal}{\color{black}}
\newcommand{\bk}{\color{black}}



%\begin{doublespace}

\def\1{{\bf 1}}
\def\ind{{\bf 1}}
\def\nn{\nonumber}
\def\bee{\begin{equation}}
\def\eee{\end{equation}}
\def\sA {{\mathcal A}} \def\sB {{\mathcal B}} \def\sC {{\mathcal C}}
\def\sD {{\mathcal D}} \def\sE {{\mathcal E}} \def\sF {{\mathcal F}}
\def\sG {{\mathcal G}} \def\sH {{\mathcal H}} \def\sI {{\mathcal I}}
\def\sJ {{\mathcal J}} \def\sK {{\mathcal K}} \def\sL {{\mathcal L}}
\def\sM {{\mathcal M}} \def\sN {{\mathcal N}} \def\sO {{\mathcal O}}
\def\sP {{\mathcal P}} \def\sQ {{\mathcal Q}} \def\sR {{\mathcal R}}
\def\sV {{\mathcal V}} \def\sW {{\mathcal W}} \def\sX {{\mathcal X}}
\def\sY {{\mathcal Y}} \def\sZ {{\mathcal Z}}

\def\bA {{\mathbb A}} \def\bB {{\mathbb B}} \def\bC {{\mathbb C}}
\def\bD {{\mathbb D}} \def\bE {{\mathbb E}} \def\bF {{\mathbb F}}
\def\bG {{\mathbb G}} \def\bH {{\mathbb H}} \def\bI {{\mathbb I}}
\def\bJ {{\mathbb J}} \def\bK {{\mathbb K}} \def\bL {{\mathbb L}}
\def\bM {{\mathbb M}} \def\bN {{\mathbb N}} \def\bO {{\mathbb O}}
\def\bP {{\mathbb P}} \def\bQ {{\mathbb Q}} \def\bR {{\mathbb R}}
\def\bS {{\mathbb S}} \def\bT {{\mathbb T}}
\def\bV {{\mathbb V}} \def\bW {{\mathbb W}} \def\bX {{\mathbb X}}
\def\bY {{\mathbb Y}} \def\bZ {{\mathbb Z}}
\def\R {{\mathbb R}} \def\RR {{\mathbb R}} \def\H {{\mathbb H}}
\def\n{{\bf n}} \def\Z {{\mathbb Z}}

\def\supp{{\mathop {{\rm supp\, }}}}
\def\esssup{{\mathop{\rm ess \; sup \, }}}
\def\essinf{{\mathop{\rm ess \; inf \, }}}
\def\essosc{{\mathop{\rm ess \; osc \, }}}



\newcommand{\expr}[1]{\left( #1 \right)}
\newcommand{\cl}[1]{\overline{#1}}
\newtheorem{thm}{Theorem}[section]
\newtheorem{lemma}[thm]{Lemma}
\newtheorem{defn}[thm]{Definition}
\newtheorem{prop}[thm]{Proposition}
\newtheorem{corollary}[thm]{Corollary}
\newtheorem{remark}[thm]{Remark}
\newtheorem{example}[thm]{Example}
\numberwithin{equation}{section}
\def\ee{\varepsilon}
\def\qed{{\hfill $\Box$ \bigskip}}
\def\NN{{\mathcal N}}
\def\AA{{\mathcal A}}
\def\MM{{\mathcal M}}
\def\BB{{\mathcal B}}
\def\CC{{\mathcal C}}
\def\LL{{\mathcal L}}
\def\DD{{\mathcal D}}
\def\FF{{\mathcal F}}
\def\EE{{\mathcal E}}
\def\QQ{{\mathcal Q}}
\def\SS{{\mathcal S}}
\def\RR{{\mathbb R}}
\def\R{{\mathbb R}}
\def\L{{\bf L}}
\def\K{{\bf K}}
\def\S{{\bf S}}
\def\A{{\bf A}}
\def\E{{\mathbb E}}
\def\F{{\bf F}}
\def\P{{\mathbb P}}
\def\N{{\mathbb N}}
\def\eps{\varepsilon}
\def\wh{\widehat}
\def\wt{\widetilde}
\def\pf{\noindent{\bf Proof.} }
\def\pff{\noindent{\bf Proof} }
\def\cp{\mathrm{Cap}}
\usepackage{hyperref}
%\bibliographystyle{plain}
\begin{document}
\title[]{Notes}
\author{}


 \date{}


%\begin{abstract}

%\end{abstract}
\maketitle

%\bigskip
%\noindent {\bf AMS 2010 Mathematics Subject Classification}:


%\bigskip\noindent
%{\bf Keywords and phrases}:


\smallskip

The following statement, corresponding to Step 1 in the proof of Lemma 2.6 of the EJP paper, is correct, by looking at things separately.

If  $f(x)=f_1(x)+t^{\widetilde{\beta}-1}f_2(x)$ with
$f_1, f_2\in\mathcal P$, then there exists $h\in \mathcal P$ such that 
$|T_tf|\le e^{-\delta t}h$ for all $t\ge 1$, where
$$
\delta=(\inf\{|\tilde \beta\alpha-|p|b|: p\in\mathbb Z^d_+, \langle f_1, \phi_p\rangle\neq 0\})
\wedge (\inf\{|\tilde \beta\alpha-|p|b|: p\in\mathbb Z^d_+, \langle f_2, \phi_p\rangle\neq 0\}).
$$


The statement of Lemma 2.9 of the EJP paper can be changed to the following form
with proof basically unchanged.

{\bf Lemma 2.9} Suppose that $g(x)=g_1(x)+t^{\tilde{\beta}-1}g_2(x)$ with
$g_1, g_2\in\mathcal P$. Then there exists $h\in\mathcal P^+$ such that for all
$f\in \mathcal P_g=\{\theta_nT_ng: n\in \mathbb Z_+, \theta\in [-1, 1]\}$ and $t\ge 1$, we have $|P_t(Z_1f-\langle Z_1f, \varphi\rangle)|\le e^{-b t}h$.



I am proposing to change the statement of Proposition 3.5 to the following form:

{\bf Proposition 3.5} Fro all $\mu\in\mathcal M_c(\mathbb R^d)$ and $g(x)=g_1(x)+t^{\tilde{\beta}-1}g_2(x)$ with
$g_1, g_2\in\mathcal P$, there exist $C, \delta>0$ such that for all $t\ge 1$ and $f\in \mathcal P_g:=\{\theta T_ng: n\in \mathbb Z_+, \theta\in [-1, 1]\}$, we have
$$
\mathbb P_\mu\left[|\mathbb P_\mu[e^{i\upsilon^f_t}-e^{\langle Z_1f, \varphi\rangle}; D^c|\mathcal F_t|\right]\le C e^{-\delta t}.
$$

I think that the proof stays the same.

\begin{lem}
	\label{lem:P:R}
	Suppose that $g \in \mathcal P$, then there exists $h \in \mathcal P^+$ such that for all $ f \in \mathcal P_g := \{\theta T_n g: n \in \mathbb Z_+, \theta \in [-1,1]\} $ and $t\geq 0$, we have $ | P_t (Z_1 f - \langle Z_1 f, \varphi \rangle )| \leq e^{-bt} h$.
\end{lem}
\begin{proof}
	Fix $g \in \mathcal P$.
	We write  $g = g_0 + g_1$ with $g_0 \in \mathcal C_s \oplus \mathcal C_c$ and $g_1 \in \mathcal C_l$,  and $q_f:=Z_1f - \langle Z_1f, \varphi \rangle\in \mathcal P^*$ for each $f\in \mathcal P$.
	We need to prove that there exists $h \in \mathcal P^+$ such that for each $f\in \mathcal P_g$, $|P_tq_f| \leq e^{-bt} h$.
	
	\emph{Step 1.} We claim  that we only need to prove the result for all
	$f \in \widetilde{\mathcal P}_g:= \{T_{n+1} g : n \in \mathbb Z_+\}$.
	In fact, both $\operatorname{Re} q_g$ and $\operatorname{Im} q_g$ are functions in $\mathcal P$ of order $\geq 1$.
	The result is valid for $f = T_0 g = g$ according to \cite[Fact 1.2]{MarksMilos2018CLT}.
	Also, note that if the result is valid for some $f \in \mathcal P$, it is also valid for any $\theta f$ with $\theta \in [-1,1]$.
	
	
	\emph{Step 2.} We show that $\{T_s g: s> 0\} \subset C_\infty (\mathbb R^d) \cap \mathcal P$.
	In fact, for each $s > 0$,
	\[
	T_s g
	= T_s (g_0 + g_1)
	= e^{\alpha \tilde \beta s}P_s g_0 + \sum_{p \in \mathbb Z_+^d: \alpha \tilde \beta > |p|b} 
	\langle g_1, \phi_p \rangle_\varphi e^{-(\alpha \tilde \beta - |p|b)s} \phi_p.
	\]
	Notice that the second term is in $C_\infty(\mathbb R^d)\cap \mathcal P$ since it is a finite sum of polynomials, and the first term is also in $C_\infty (\mathbb R^d) \cap \mathcal P$ according to \cite[Fact 1.1]{MarksMilos2018CLT}.
	
	\emph{Step 3.} We show that there exists $h_3 \in \mathcal P^+$ such that for all $j \in \{1,\dots, d\}$ and $f \in \widetilde {\mathcal P}_g$, it holds that $|\partial_j f| \leq h_3$.
	In fact, it is known that  (see \cite{MetafunePallaraPriola2002Spectrum} for example)
	\begin{align}
	\label{eq:P:R:3:-1}
	P_t f(x)
	= \int_{\mathbb R^d} f\big(x e^{-bt} + y \sqrt{1-e^{-2bt}}\big) \varphi(y)~dy,
	\quad t\geq 0, x\in \mathbb R^d, f\in \mathcal P.
	\end{align}
	For $f \in C_\infty(\mathbb R^d)\cap \mathcal P$ it can be verified from above that
	\begin{align}
	\label{eq:P:R:3:1}
	\partial_j P_t f
	= e^{-bt} P_t \partial_j f,
	\quad t \geq 0, j \in \{1,\dots, d\}.
	\end{align}
	Thanks to Step 2, $T_1 g_0 \in C_\infty(\mathbb R^d)\cap \mathcal P$.
	According to \cite[Fact 1.3]{MarksMilos2018CLT} and the fact that $\alpha \tilde \beta \leq \kappa _{g_0} b$, we have for each $j \in \{1,\dots, d\}$,
	\[
	\kappa_{(\partial_j T_1 g_0)}
	\geq \kappa_{(T_1 g_0)} - 1
	= \kappa_{g_0} - 1
	\geq \frac{\alpha \tilde \beta}{b} - 1.
	\]
	Therefore, there exists  $h'_3\in \mathcal P^+$ such that for all $n \in \mathbb Z_+$ and $j\in \{1,\dots,d\}$,
	\begin{align}
	& | \partial_j T_{n+1}g_0 |
	= | \partial_j e^{\alpha \tilde \beta n}P_n T_1g_0 |
	= e^{\alpha \tilde \beta n-bn} |P_n \partial_j T_1 g_0| \\
	& \leq e^{\alpha \tilde \beta n-bn} \exp\{-\kappa_{(\partial_j T_1 g_0)}bn\}Q \partial_j T_1g_0
	\leq Q\partial_j T_1g_0
	\leq h'_3.
	\end{align}
	On the other hand, there exists $h_3''\in \mathcal P^+$ such that for all $n \in \mathbb Z_+$ and $j\in \{1,\dots,d\}$,
	\begin{align}
	& |\partial_j T_{n+1}g_1 |
	= \Big| \sum_{p\in \mathbb Z_+^d: \alpha \tilde \beta > |p|b} e^{- (\alpha \tilde \beta - |p|b)(n+1)} \langle g_1, \phi_p \rangle_\varphi \partial_j \phi_p \Big| \\
	& \leq \sum_{p\in \mathbb Z_+^d: \alpha \tilde \beta > |p|b} |\langle g_1, \phi_p \rangle_\varphi \partial_j \phi_p |
	\leq h_3''.
	\end{align}
	Then the desired result in this step follows.
	
	\emph{Step 4.} We show that there exists $h_{4} \in \mathcal P^+$ such that for all $j \in \{1,\dots, d\}, u \in [0, 1]$ and $f \in \widetilde {\mathcal P}_g$, it holds that $ | \partial_j  P_{1-u}^\alpha (- i P_u^\alpha f)^{1+\beta} | \leq h_4$.
	In fact, thanks to Step 2 and \eqref{eq:P:R:3:1}, for all $j \in \{1,\dots, d\}, u \in [0, 1]$ and $f \in \widetilde{\mathcal P}_g$, we have
	\begin{align}
	& \partial_j  P_{1-u}^\alpha (- i P_u^\alpha f)^{1+\beta}
	= e^{-(1-u)b} P_{1-u}^\alpha \partial_j (- i P_u^\alpha f)^{1+\beta}
	\\ & = (1+\beta) e^{-(1-u)b} P_{1-u}^\alpha [ (- i P_u^\alpha f)^\beta \partial_j (- i P_u^\alpha f) ]
	\\ & = -i(1+\beta) e^{-(1-u)b} P_{1-u}^\alpha[ (- i P_u^\alpha f)^\beta e^{-ub} P_u^\alpha \partial_j f]
	\\ & = -i(1+\beta) e^{-b} e^{(1-u)\alpha} e^{u\alpha (1+\beta)} P_{1-u} [ (- i P_u f)^\beta P_u \partial_j f ].
	\end{align}
	Recall from Step 1 in the proof of Lemma \ref{lem:m} there exists $h'_4\in \mathcal P^+$ such that for each $f \in \{T_sg:s\geq 0\}$ it holds that $|f| \leq h'_4$.
	Therefore, using Step 3, we have for all $j \in \{1,\dots, d\}, u \in [0, 1]$ and $f \in \widetilde {\mathcal P}_g$,
	\begin{align}
	& |\partial_j  P_{1-u}^\alpha (- i P_u^\alpha f)^{1+\beta}|
	\leq (1+\beta) e^{\alpha (1+\beta)} P_{1-u} [  (P_u |f|)^\beta P_u |\partial_j f| ]
	\\ & \leq (1+\beta) e^{\alpha (1+\beta)} P_{1-u} [  (P_u h'_4)^\beta P_u h_3 ]
	\leq (1+\beta) e^{\alpha (1+\beta)} Q_0 [  (Q_0 h_4')^\beta Q_0 h_3 ],
	\end{align}
	where $Q_0$ is defined by \eqref{eq:Q}.
	This implies the desired result in this step.
	
	\emph{Step 5.}
	We show that there exists $h_5 \in \mathcal P^+$ such that for each $f \in \widetilde {\mathcal P}_g$, we have $ |\nabla (Z_1f)| \leq h_5$.
	In fact, according to Step 4, for all $j \in \{1,\dots, d\}$, $f \in \widetilde{\mathcal P}_g$ and compact $A \subset \mathbb R^d$, we have
	\[
	\int_0^1 \sup_{x\in A} | (\partial_j  P_{1-u}^\alpha (-i P_u^\alpha f)^{1+\beta}) (x) |~du
	\leq \sup_{x\in A} h_4(x) < \infty.
	\]
	Using this and \cite[Theorem A.5.2]{Durrett2010Probability}, for all $j \in \{1,\dots, d\}$, $f\in \widetilde {\mathcal P}_g$ and $x\in \mathbb R^d$, it holds that
	\[
	| \partial_j Z_1 f(x)|
	= \Big| \int_0^1  ( \partial_jP_{1-u}^\alpha (-iP_u^\alpha f)^{1+\beta} ) (x) ~du  \Big|
	\leq h_4(x).
	\]
	Now, the desired result for this step is valid.
	
	\emph{Step 6.}
	Let $h_5$ be the function in Step 5.
	There are $c_0, n_0> 0$ such that for all $x\in \mathbb R^d$, $h_5(x) \leq c_0(1+|x|)^{n_0}$.
	Note that for all $x, y \in \mathbb R^d$, $1+|x|+|y|\leq (1+|x|) (1+|y|)$; and that for all $\theta \in [0,1]$, $|\sqrt {1 - \theta} - 1| \leq \theta$.
	Write $D_{x,y} = \{ax+by: a, b \in [0,1]\}$ fo $x, y \in \mathbb R^d$.
	Using \eqref{eq:P:R:3:-1} and Step 5, there exists  $h_6 \in \mathcal P^+$ such that for all $t \geq 0$, $f \in \widetilde {\mathcal P}_g$ and $x \in \mathbb R^d$,
	\begin{align}
	& |P_t q_f(x)|
	= \Big| \int_{\mathbb R^d} ( (Z_1f)(x e^{-bt}+ y \sqrt{1 - e^{-2bt}}) - Z_1 f (y) ) \varphi(y) ~dy \Big| \\
	& \leq \int_{\mathbb R^d} \Big(\sup_{z\in D_{x,y}} |\nabla  (Z_1f) (z) |\Big) | x e^{-bt} + y \sqrt{1 - e^{-2bt}} - y | \varphi(y) ~dy \\
	& \leq e^{-bt} \int_{\mathbb R^d} c_0(1+|x|+|y|)^{n_0} (|x|+|y|) \varphi(y) ~dy \\
	& \leq c_0 e^{-bt}(1+|x|)^{n_0}\Big(|x|\int_{\mathbb R^d} (1+|y|)^{n_0}\varphi(y) ~dy + \int_{\mathbb R^d} (1+ |y|)^{n_0} |y| \varphi(y) ~dy \Big) \\
	& \leq e^{-bt} h_6(x).
	\qedhere
	\end{align}
\end{proof}


%\end{doublespace}

\end{document}
