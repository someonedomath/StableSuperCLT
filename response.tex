%%%----Documentstyles-----------------------
\documentclass[12pt,a4paper]{amsart}
\setlength{\textwidth}{\paperwidth}\addtolength{\textwidth}{-2in}\calclayout
\usepackage{hyperref}
\usepackage{comment}
\usepackage{mathtools}
\DeclarePairedDelimiter\ceil{\lceil}{\rceil}
\DeclarePairedDelimiter\floor{\lfloor}{\rfloor}

%%%----Environments------------------------
%\newtheorem{thm}{Theorem}[section]
%\newtheorem{lem}[thm]{Lemma}
%\newtheorem{prop}[thm]{Proposition}
%\numberwithin{equation}{section}

%%%---------Top matter----------------
%YX \title
%YX [Response]
\title[Response]{\large Response to the referee's comments}
%%%----Main-------------------------------
\begin{document}
\maketitle	
	We are grateful to the referee for the very helpful comments.
We have incorporated all the comments made by the referee.
Here are the specific changes in response to the comments of the referee.

\begin{itemize}
\item
There are many duplications in the current presentation, see the comments below for more details.
With a more general formulation, these duplications can be removed reducing the number of pages
dramatically. For example, it is realistic that the same content can be presented in 45 pages rather
than 60 pages.

\emph{We made a more general formulation. The paper has been rearranged and streamlined.
See the details below. We were not able to reduce the length to 45 pages, but we are pretty close. The paper is now 46 pages.}

\item
Also, the authors should give stronger interpretations of the assumptions and the results. What
do the the parameters $\alpha$, $\beta$, $\kappa_f$, $b$ represent, and why does the limit behavior change with these
parameters? For example, what's the interpretation of the inequality $\alpha\beta > \kappa_f b(1 + \beta)$? Why is
this referred to this as the large branching rate regime?

\emph{We added some explanation about these at the end of Subsection 1.2.}

\end{itemize}
\begin{enumerate}
\item
  P.2.L.-11.
  Please provide an interpretation of the form of $\psi$.
  If you were to approximate this superprocess by a branching particle system, what are the implications on the distribution of the number of off-springs?
  Does the term $\psi_1$ imply that the offspring distribution is heavy tailed?

  \emph{We added an interpretation in Remark 1.1.
   See also the bullet point about $\beta$ on page 7.
    }
\item
  P.2.L.-8.
%   Please provide an interpretation of the expression $\alpha\beta - \kappa_f b(1+\beta)$?
  Please provide an interpretation of the expression $\alpha\beta - \kappa_f b(1+\beta)$.

\emph{We added some discussion about this in the middle of page 7.}

  Perhaps it might be easier to interpret if you wrote the inequality $\alpha \beta > \kappa_f b(1 + \beta)$ as $\alpha \frac{\beta}{1+\beta} > \kappa_f b$.
  This way the parameters on the left are branching related, whereas the parameters on the right are related to the Ornstein-Uhlenbeck process.

  \emph{We now write it as $\alpha \tilde \beta > \kappa b$ where $\tilde \beta := \frac{\beta}{1+\beta}$.}

  Is there an intuitive explanation why $\rho$ doesn't play a role?
  Why does $b$ come into play but $\sigma$ does not?

  \emph{We added some discussion about this at the end of Subsection 1.2}.
\item
  P.4.Equation.(1.5).
  Please provide an interpretation of the form of the invariant density.
  For example, you can mention that it has a mode at the origin, it is symmetric around the origin and $b$ determines the spread etc.

\emph{We mentioned there that it is symmetric Gaussian.}

  Please also give some intuition about the role of the $\sigma^2$ specific form of density in the results.
  If the invariant density had different properties, would the results still hold?

  \emph{We added some discussion about this at the end of Subsection 1.2.
We mentioned that we hope to prove spatial CLTs for superprocesses with general motions without the second moment assumption
on the branching mechanism in a future project (see paragraph 2 of page 8)
}.
\item
  P.5.L.-9.
  Please provide an interpretation of $\kappa_f$.
  What does it mean for $\kappa_f$ to be large or small?

  \emph{We added some discussion about this in the middle of page 7.}

\item
  P.5.L.-3.
  Please explain why the three cases are referred to as large, critical and small branching rate regimes?

  \emph{We added some explanation at the beginning of page 6.}
\item
  P.6.
  Why is Theorem 1.5 restricted to test functions in $\mathcal C_l$, whereas Theorem 1.6 and 1.7 allow $f$ to be in $\mathcal P$?
  Is this a limitation of the methodology used or there is a more substantial reason?
  Please explain.

  \emph{We added some explanation about this after Theorem 1.4.}
\item
  There is substantial repetition in the statements of Theorem 1.5, 1.6 and 1.7.
  These can be combined into a single Theorem, and differences can be managed as different cases in the same proof.

  \emph{They are now unified in the new Theorem 1.4.}
\item
  P.8.
  Definitions of $m(f ), \tilde m(f ), \bar m(f )$ can be combined into a single definition, and instead of using different notation $\tilde m, \bar m$, the definition of $m(f)$ can be extended to f in the critical and the large branching rate regimes by the formula given for $\tilde m$ and $\bar m$.

  \emph{The definitions are now unified in (1.9).}
\item
  P.8.
  Instead of calling this section ``some intuitive explanation'', it would be more appropriate to call it ``an outline of the methodology''.

  \emph{Done as suggested.}
\item
  P.10.L.1.
  Please provide an interpretation of the examples for $\psi$?
  In which contexts do these examples arise?

  \emph{We added some explanation and gave two new references before the example.}

\item
  P.12. Lemma.2.4.
  There is substantial repetition in statements 2 and 3, please combine these into one statement.
  In fact, it would be even better if all the hypothesis on the operators, index sets , etc can be given at the beginning of the lemma, and then all the results can be listed without saying "Suppose that" each time.
  Similarly assumptions don't need to be restated in the proof.
  There are duplications in the proof, these should be removed as well.
  For example you can say, "If $f,g \in \mathcal P^+$ with $f \leq g$, then $R_AR_Bf \leq R_AR_Bg$, and $R_A \times R_B f \leq R_A \times R_B g$ and $R_A +R_B f \leq (R_A +R_B)g$, and ...".
  These should reduce the length of Lemma 2.4 and and its proof by half.

  \emph{The presentation of Lemma 2.4 is now rearranged and its proof is now omitted, since the results are easy to verify.}
\item
  P.16-17.Lemma 2.6., Proposition 2.7.
  If you use a single notation $m_t(f)$, instead of $m_t, \bar m$ and $\tilde m$ , you can simply say that $\theta \mapsto \exp(m_t[\theta f])$ is the characteristic function of ..." without repeating it 5 times.
  No need to provide separate statements for $m_t$ and $m$, since $m$ can be considered as a special case of $m_t$ for a suitable value of $t$, (e.g. $t = \infty$, or $t = \Delta$.)

  \emph{We now use the single notation $m_t[f]$. Now the presentation of Lemma 2.6 and Proposition 2.7 is much shorter.}
\item
  P.19.
  Combine Lemma 2.9 and 2.10 into a single Lemma.

  \emph{We now unify them in the new Lemma 2.9.}
\item
  P.20.L.3.
  Please give a reference after the statement ``It is well known that''.

  \emph{We added a new reference above equation (2.8) as suggested.}
\item
  P.22.L.8-12.
  The argument given here is not clear.
  Please provide more details.

  \emph{We rearranged that sentence. See the sentence right after (2.11).}
\item
  P.22. Equations. 2.18-2.19.
  These equations can be stated in a single line.
  For example, if $0< \lambda \leq v_t(\lambda) < \bar v < v_t(\lambda^*)<\lambda^* < \infty$, then.

  \emph{We rearranged that sentence. 
See the paragraph before (2.12).}

\item
  P.24.
  Please give more details about the argument from [10, Theorem 3.3.6].

  \emph{We added more details there. 
    See the first display of the proof of Theorem 2.11.}

\item
  P.27.Line.-2.
  Typo. ``..is an eigenfunction of $L$, ...''.

  \emph{Corrected as suggested.}
\item
  Page 31. Lemma 3.4 and Lemma 3.7 are nearly duplicates of each other.
  They should be written in a unifying way to cover both cases. Same applies to their proofs.

  \emph{We added a new Subsection 3.2 which deals with the CLTs for unit time intervals in a unified way.}
\item
  Proof of Lemma 3.4. Please indicate clearly where $\alpha \beta \leq \kappa f b(1 + \beta)$ is used.
  Same goes for the case $\alpha \beta > \kappa_f b(1 + \beta)$.

  \emph{It turns out that the proof of Lemma 3.4 doesn't use this condition. In the new version, We added a new subsection 3.2 which deals with the CLTs for one unite time intervals in a unified way.}

\item
  P.40.
  The proof of Theorem 1.7 is very similar to the proof of Theorem 1.6 with the exception of Step 2.
  Please combine these two proofs into one, and deal with differences as separate cases in the same proof (case 1: $\alpha \beta = \kappa f b(1 + \beta)$, case 2: $\alpha \beta < \kappa_f b(1 + \beta)$.)

  \emph{We agree that the structure of these two proofs are very similar. But there are
differences lying in the details which we believe are pretty important:
(i) the scaling coefficient in case 1 is $\|tX_t\|^{\frac{1}{1+\beta}}$ but in case 2 is $\|X_t\|^{\frac{1}{1+\beta}}$;
 (ii) In Step 4, the exceptional event for case 1 is chosen as $\mathcal E_t = \{\|X_t\| > t^{-1/2} e^{\alpha t}\}$ but in case 2 is chosen as $ A_t(\epsilon) = \{\|X_t\|> e^{(\alpha - \epsilon )t}\}$. }

\emph{
    Nonetheless, in the new version, the proof of CLTs for three different cases (Subsection 3.3, 3.4 and 3.5) are rearranged in such a way that (1) their structure can be clearly compared; and (2) trivial duplications are avoided as much as possible.
}

\item
  P.45.
  The proof of Lemma 3.8. is very similar to the proof of Proposition 3.6, please revise Proposition 3.6 to cover the case in Lemma 3.8.

  \emph{They are now unified in the new Corollary 3.6.}

\item
  P.46.Lemma 3.9 and Lemma 3.10. should be stated before the proof of Theorem 1.6 in more general way, that they can be used in the proof of both cases of CLT $\alpha \beta \leq \kappa_f b(1 + \beta)$ and $\alpha \beta > \kappa_f b(1 + \beta).$

  \emph{These two lemmas are now deleted, and their proofs are moved into the new Subsection 3.5.
In the new version, the proof of CLTs for three different cases (Subsection 3.3, 3.4 and 3.5) are rearranged in such a way that (1) their structure can be clearly compared; and (2) trivial duplications are avoided as much as possible.}
\end{enumerate}
\end{document}
