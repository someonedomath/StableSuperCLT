%%%%%%%%%%%%%%%%%%%%%%%%%%%%%%%%%%%%%%%%%%%%%%%%%%%%%%%%%%%%%%%%%%%%%%%%%%%%%%%%%%%%%%%%%%%%%%%%%%%%%%%%%%
%%
%%
%%            NOTE: The content in this template, including title, authors' name and affiliation, and etc.
%%                 is ONLY a sample for the author's guidance.
%%
%%
%%%%%%%%%%%%%%%%%%%%%%%%%%%%%%%%%%%%%%%%%%%%%%%%%%%%%%%%%%%%%%%%%%%%%%%%%%%%%%%%%%%%%%%%%%%%%%%%
\documentclass{amse-new}

\numberwithin{equation}{section} %%% Equations numbered by section. If you don't want it, please delete it.

\begin{document}

 \PageNum{1}
 \Volume{201x}{Sep.}{x}{x}
 \OnlineTime{August 15, 201x}
 \DOI{0000000000000000}
 \EditorNote{Received x x, 201x, accepted x x, 201x}

\abovedisplayskip 6pt plus 2pt minus 2pt \belowdisplayskip 6pt
plus 2pt minus 2pt
%%%%%%%%%%%%%%%%
\def\vsp{\vspace{1mm}}
\def\th#1{\vspace{1mm}\noindent{\bf #1}\quad}
\def\proof{\vspace{1mm}\noindent{\it Proof}\quad}
\def\no{\nonumber}
\newenvironment{prof}[1][Proof]{\noindent\textit{#1}\quad }
{\hfill $\Box$\vspace{0.7mm}}
\def\q{\quad} \def\qq{\qquad}
\allowdisplaybreaks[4]
%%%%%%%%%%%%%%%%%%%%%%%%%%%%%%%%%%%%%%%%%%%%%%%%%%%%%%%%%%%%%%%%%%%%%%%%%%%%%%%%%%%%%%%%%%%%%%%
%%-------------------       Beginning of  Author's Definitions       -------------------%%
%%                     Note: You may add your own definitions here.



%%-------------------         the end of  Author's Definitions           -------------------%%



\AuthorMark{Surname1 F1. and Surname2 F2. }                             %%%  appear on the head of even pages  %%%

\TitleMark{A Template for Journal}  %%% Running Title, appear on the head of odd pages  %%%

\title{A Template for Journal        %%%   Main Title of your paper  %%%
\footnote{Supported by \ldots (Grant No. \ldots)}}                  %%%   the Fund which you are supported by  %%%

\author{Firstname1 \uppercase{Surname1}}             %%%  1st Author's information   %%%
    {Address\\
    E-mail\,$:$ }

\author{Firstname2 SURNAME2}     %%%  2nd Author's info, if exists, or you may delete this part directly  %%%
    {Address\\
    E-mail\,$:$  }

\maketitle%


\Abstract{Please make sure NO reference number in your Abstract since it is misunderstood independent of full text.}      % the abstract

\Keywords{Aaaa, bbbb, cccc}        % the keywords

\MRSubClass{05B05, 05B25, 20B25}      % MR(2000) Subject Classification

%\baselineskip 15pt

\section{Introduction}

\subsection{A Subsection}

Please make sure that your paper contains correct reference
sequence (please resort them according to its {\it alphabetical
order} and make sure that each bibliographical item is labelled
and that these items are recalled using the command
\verb|\cite{...}|, such as \cite{LT}, and \cite{HB,T1,T2,T3})

All equations, theorems, definitions, lemmas, propositions,
corollaries, examples, remarks etc. would be better to be numbered
consecutively and unrepeatedly within each section. For example,
Definition 2.1, Lemma 2.2, Theorem 2.3 \ldots.

Use \verb|\label| and \verb|\ref| or \verb|\eqref| to
automatically cross-reference sections, equations, theorems and
theorem-like environments, tables, figures, etc.

\begin{theorem}[\cite{HB}]\label{th:1.1} %Theorem 1.1 could be recalled by using Theorem \ref{th:1.1}
The statements of theorems, lemmas, definitions, propositions,
corollaries, conjectures, etc. are set in italics, by using
\begin{verbatim}
\begin{theorem/lemma/definition/proposition/corollary/conjecture}
\end{theorem/lemma/definition/proposition/corollary/conjecture}.
\end{verbatim}
\end{theorem}

\begin{prof}  %you can also use the environment \begin{proof}\end{proof}
Observe that
\begin{align}\label{E:1.1}
AAAAAAAAAA &= BBBBBBBBBBB\nonumber \\
           &\quad + CCCCCCCCCC\nonumber \\
           &= DDDDDDDDDDDDD.
\end{align}
Now apply induction on $n$ to \eqref{E:1.1}\ldots
\end{prof}

\begin{remark}\label{re:1.2}
Remarks, examples, problems, etc. are set in roman type.
\end{remark}


\subsection{Table}

\begin{table}
\begin{tabular}{|c|c|c|l|c|}
\hline $P(x)$ & $i$& $(e(1),e(2),e(4))$ & $(e(3),e(6),e(12),e(24))$ & $T(E)$ \\
\hline $P_1$  &    & & &$\emptyset$ \\
\hline $P_2$  & 4  & & $(1,1,1,0)\rightarrow(0,0,0,1)$ &2\\
\hline $P_3$  & 2  & &$(1,1,1,0)\rightarrow(0,0,2,0)$ &1\\
\hline $P_4$  & 2  & $(0,1,1)\rightarrow(1,2,0)$ & &1\\
\hline $P_5$  & 2  & $(0,1,1)\rightarrow(1,2,0)$ &$(1,1,1,0)\rightarrow(0,0,0,1)$ &$1,2$\\
\hline $P_6$  & 6  & $(0,1,1)\rightarrow(1,2,0)$ &$(1,1,1,0)\rightarrow(2,2,0,0)$ &1\\
\hline $P_7$  & 3  & $(0,1,1)\rightarrow(1,0,1)$ &$(1,1,1,0)\rightarrow(2,0,1,0)$ &0\\
\hline $P_8$  & 3  & $(0,1,1)\rightarrow(2,1,0)$ &$(1,1,1,0)\rightarrow(2,0,1,0) \rightarrow(3,1,0,0)$ &$0,1$\\
\hline
\end{tabular}
\caption{Aaa bbb ccc\label{tab}}
\end{table}


\subsection{Figure}

%\centerline{\includegraphics[scale=1.2]{actmark.eps}}
%\centerline{\small Figure 1\quad Journal mark}

\acknowledgements{\rm We thank the referees for their time and
comments. }



\begin{thebibliography}{99}

\bibitem{HB} %% Books %% surname(s), initial(s), title, publisher, place of publication, year
Huppert, B., Blackburn, N.: Finite Groups II, Springer-Verlag, New
York, 1982

\bibitem{LT} %% Journals %% surname(s), initial(s), article, journal, volume, relevant page numbers, year
Lorenzini, D., Tucker, T. J.: The equations and the method of
Chabauty--Coleman. \emph{Invent. Math.}, \textbf{148}, 1--46
(2002)

\bibitem{T1}
Test

\bibitem{T2}
Test

\bibitem{T3}
Test

\end{thebibliography}


\end{document}
