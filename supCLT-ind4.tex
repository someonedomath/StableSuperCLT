%supCLT-ind1.tex ?/?/? by Jianjie
%supCLT-ind2.tex ?/?/? by ?
%supCLT-ind3.tex 03/05/2020 by Renming
%supCLT-ind4.tex 11/05/2020 by Zhenyao 
\documentclass[12pt,a4paper]{amsart}
\setlength{\textwidth}{\paperwidth}
\addtolength{\textwidth}{-2in}
\calclayout
\usepackage[utf8]{inputenc}
\usepackage[T1]{fontenc}
\usepackage{mathtools}
\mathtoolsset{showonlyrefs}
\usepackage{stackrel}
\usepackage{mathrsfs}
\usepackage[backref]{hyperref}
\usepackage{comment}
\usepackage{xcolor}
\usepackage{amsthm}
\theoremstyle{plain}
\newtheorem{thm}{Theorem}[section]
\newtheorem{lem}[thm]{Lemma}
\newtheorem{prop}[thm]{Proposition}
\newtheorem{cor}[thm]{Corollary}
\newtheorem{conj}[thm]{Conjecture}
\theoremstyle{definition}
\newtheorem{defi}[thm]{Definition}
\newtheorem{rem}[thm]{Remark}
\newtheorem{exa}[thm]{Example}
\newtheorem{asp}{Assumption}
\numberwithin{equation}{section}
\allowdisplaybreaks
\begin{document}
\title
[stable CLT for super-OU processes]
{Stable Central Limit Theorems for Super Ornstein-Uhlenbeck Processes, II}
\author
[Y.-X. Ren, R. Song, Z. Sun and J. Zhao]
{Yan-Xia Ren, Renming Song, Zhenyao Sun and Jianjie Zhao}
\address{
  Yan-Xia Ren \\
  LMAM School of Mathematical Sciences \& Center for Statistical Science \\
  Peking University \\
  Beijing, P. R. China, 100871}
\email{yxren@math.pku.edu.cn}
\thanks{The research of Yan-Xia Ren is supported in part by NSFC (Grant Nos. 11671017  and 11731009) and LMEQF.}
\address{
  Renming Song \\
  Department of Mathematics \\
  University of Illinois at Urbana-Champaign \\
  Urbana, IL, USA, 61801}
\email{rsong@illinois.edu}
\thanks{The Research of Renming Song is support in part by a grant from the Simons Foundation (\#429343, Renming Song)}
\address{
  Zhenyao Sun \\
  School of Mathematics and Statistics\\
  Wuhan University \\
  Hubei, P. R. China, 100871}
\email{zhenyao.sun@gmail.com}
\address{
  Jianjie Zhao \\
  School of Mathematical Sciences \\
  Peking University \\
  Beijing, P. R. China, 100871}
\email{zhaojianjie@pku.edu.cn}

\begin{abstract}
This paper is a continuation of our recent paper (Elect. J. Probab. \textbf{24} (2019), no. 141)
and is devoted to the  asymptotic behavior of a supercritical super Ornstein-Uhlenbeck process $(X_t)_{t\geq 0}$
 whose branching mechanism $\psi$ satisfies some perturbation condition which guarantees that,
 when $z\to 0$, $\psi(z)=-\alpha z + \eta z^{1+\beta} (1+o(1))$ with $\alpha > 0$, $\eta>0$ and $\beta\in (0, 1)$.
In the aforementioned paper, we have proved $(1+\beta)$-stable central limit theorems
for  $X_t(f) $ for {\it some} functions $f$ of polynomial growth.
As a consequence of the main result of this paper, we will have
$(1+\beta)$-stable central limit theorems for  $X_t(f) $ for {\it all} functions $f$ of polynomial growth.
\end{abstract}
\subjclass[2010]{60J68, 60F05}
\keywords{Superprocesses, Ornstein-Uhlenbeck processes, Stable distribution, Central limit theorem, Law of large numbers, Branching rate regime}
\maketitle
\section{Introduction}
\subsection{Main result}
\label{subsec:M}
Let $d \in \mathbb N:= \{1,2,\dots\}$ and $\mathbb R_+:= [0,\infty)$.
Let $\xi=\{(\xi_t)_{t\geq 0}; (\Pi_x)_{x\in \mathbb R^d}\}$ be an $\mathbb R^d$-valued Ornstein-Uhlenbeck process (OU process) with generator
\begin{align}
  Lf(x)
  = \frac{1}{2}\sigma^2\Delta f(x)-b x \cdot \nabla f(x)
  , \quad  x\in \mathbb R^d, f \in C^2(\mathbb R^d),
\end{align}
where $\sigma > 0$ and $b > 0$ are constants.
Let $\psi$ be a function on $\mathbb R_+$ of the form
\begin{align}
  \label{eq: honogeneou branching mechanism}
  \psi(z)
  =- \alpha z + \rho z^2 + \int_{(0,\infty)} (e^{-zy} - 1 + zy) {\color{red} \chi}(\mathrm dy)
  , \quad  z \in \mathbb R_+,
\end{align}
where $\alpha > 0 $, $\rho \geq0$ and {\color{red}$\chi$} is a measure on $(0,\infty)$ with $\int_{(0,\infty)}(y\wedge y^2)\chi(\mathrm dy)< \infty$.
$\psi$ is referred to as a branching mechanism and {\color{red}$\chi$} is referred to as the L\'evy measure of $\psi$.
Denote by $\mathcal M(\mathbb R^d)$ the space of all finite Borel measures on $\mathbb R^d$.
Denote by $\mathcal B(\mathbb R^d, \mathbb R)$ the space of all $\mathbb R$-valued Borel functions on $\mathbb R^d$.
For $f,g\in \mathcal B(\mathbb R^d, \mathbb R)$ and $\mu \in \mathcal M(\mathbb R^d)$,
 write $\mu(f)= \int f(x)\mu(\mathrm dx)$
and $\langle f, g\rangle = \int f(x)g(x) \mathrm dx$ whenever the integrals make sense.
We say a real-valued Borel function $f:(t,x)\mapsto f(t,x)$ on $\mathbb R_+\times \mathbb R^d$ is \emph{locally bounded} if, for each $t\in \mathbb R_+$, we have $ \sup_{s\in [0,t],x\in \mathbb R^d} |f(s,x)|<\infty. $
We say that an $\mathcal M(\mathbb R^d)$-valued Hunt process $X = \{(X_t)_{t\geq 0}; (\mathbb{P}_{\mu})_{\mu \in \mathcal M(\mathbb R^d)}\}$
is a \emph{super Ornstein-Uhlenbeck process (super-OU process)} with branching mechanism $\psi$, or a $(\xi, \psi)$-superprocess, if for each non-negative bounded Borel function $f$ on $\mathbb R^d$, we have
\begin{align}
  \label{eq: def of V_t}
  \mathbb{P}_{\mu}[e^{-X_t(f)}]
  = e^{-\mu(V_tf)}
  , \quad t\geq 0, \mu \in \mathcal M(\mathbb R^d),
\end{align}
where $(t,x) \mapsto V_tf(x)$ is the unique locally bounded non-negative solution to the equation
\begin{align}
  V_tf(x) + \Pi_x \Big[ \int_0^t\psi (V_{t-s}f(\xi_s) ) \mathrm ds\Big]
	= \Pi_x [f(\xi_t)]
  , \quad x\in \mathbb R^d, t\geq 0.
\end{align}	
The existence of such super-OU process $X$ is well known, see \cite{Dynkin1993Superprocesses} for instance.

{\color{red}There have been a lot of papers on laws of large numbers and central limit theorems for branching Markov processes under different settings, including several superprocesses. Those papers can be traced via the references given in \cite{RenSongSunZhao2019Stable}. In this paper, we will be focusing on the asymptotic behavior of the super-OU process $X$ whose branching mechanism $\psi$ satisfies the following two additional assumptions.}

{\color{red}
\begin{asp}[Grey's condition]
	\label{asp: Greys condition}
	There exists $z' > 0$ such that $\psi(z) > 0$ for all $z>z'$ and  $\int_{z'}^\infty \psi(z)^{-1} \mathrm dz < \infty$.
\end{asp}
}
Denote by $\Gamma$ the gamma function.
For any $\sigma$-finite signed measure $\mu$, denote by $|\mu|$ the total variation measure of $\mu$.
\begin{asp}
  \label{asp: branching mechanism}
  There exist constants $\eta > 0$ and $\beta \in (0,1)$ such that
  \begin{align}
    \label{eq: asp of branching mechanism}
    \int_{(1,\infty)}y^{1+\beta +\delta}\Big|\chi(\mathrm dy)-\frac{\eta \mathrm dy}{\Gamma(-1-\beta)y^{2+\beta}}\Big| <\infty
  \end{align}
	for some $\delta > 0$.
\end{asp}

Let us give some comments on those Assumptions.
For each $\mu \in \mathcal M(\mathbb R^d)$, write $\|\mu\| = \mu(1)$.
It is known (see \cite[Theorems 12.5 \& 12.7]{Kyprianou2014Fluctuations} for example) that, under Assumption \ref{asp: Greys condition}, the \emph{extinction event}
$$D :=\{\exists t\geq 0,~\text{such that}~ \|X_t\| =0 \}$$
is non-trivial with respect to $\mathbb P_\mu$ for each  $\mu \in \mathcal M(\mathbb R^d)\setminus\{0\}$.
In fact, $ \mathbb{P}_{\mu} (D) = e^{-\bar v \|\mu\|}$ where $ \bar v := \sup\{\lambda \geq 0: \psi(\lambda) = 0\} \in (0,\infty) $ is the largest root of $\psi$. 
For each $\mu \in \mathcal M(\mathbb R^d)\setminus\{0\}$, write $\widetilde {\mathbb P}_\mu(\cdot):= \mathbb P_\mu(\cdot | D^c)$.
Assumption \ref{asp: branching mechanism} says that $\psi$ is ``not too far away'' from $\widetilde \psi(z) := - \alpha z + \eta z^{1+\beta}$ near $0$, see \cite[Remark 1.3]{RenSongSunZhao2019Stable}.
It follows from \cite[Lemma 2.2]{RenSongSunZhao2019Stable}  that, if Assumption \ref{asp: branching mechanism} holds, then $\eta$ and $\beta$ are uniquely determined by the L\'evy measure $\chi$.
In \cite[Lemma 2.3]{RenSongSunZhao2019Stable}, we have shown  that,
under Assumption \ref{asp: branching mechanism},
 $\psi$ satisfies the $L \log L$ condition, i.e., $ \int_{(1,\infty)} y\log y \chi(\mathrm dy) < \infty. $
In the reminder of the paper, we will always use $\eta$ and $\beta$ to denote the constants in Assumption  \ref{asp: branching mechanism}.
	Note that $\delta$ is not uniquely determined by $\chi$.

The limit behavior of $X$  is closely related to the spectral property of the OU semigroup $(P_t)_{t\geq 0}$ which we now recall (See \cite{MetafunePallaraPriola2002Spectrum} for more details).
Denote by $\mathcal B(\mathbb R^d, \mathbb R_+)$ the space of all $\mathbb R_+$-valued Borel functions on $\mathbb R^d$.
We use  $(P_t)_{t\geq 0}$ to denote the transition semigroup of $\xi$.	
Define
\(
P^{\alpha}_t f(x)
  := e^{\alpha t} P_t f(x)
  = \Pi_x [e^{\alpha t}f(\xi_t)]
\)
for each $x\in \mathbb R^d$, $t\geq 0$ and $f\in \mathcal B(\mathbb R^d, \mathbb R_+)$.
It is known that, see \cite[Proposition 2.27]{Li2011Measure-valued} for example, $(P^\alpha_t)_{t\geq 0}$ is the \emph{mean semigroup} of $X$ in the sense that
\(
  \mathbb{P}_{\mu}[X_t (f)]  = \mu( P^\alpha_t f)
\)
for all $\mu\in \mathcal M(\mathbb R^d)$, $t\geq 0$ and $f\in \mathcal B(\mathbb R^d, \mathbb R_+)$.
It is known that the OU process $\xi$ has an invariant probability on $\mathbb R^d$
\begin{align}
  \label{invariantdensity}
  \varphi(x)\mathrm dx
  :=\Big (\frac{b}{\pi \sigma^2}\Big )^{d/2}\exp \Big(-\frac{b}{\sigma^2}|x|^2 \Big)\mathrm dx
\end{align}
which is a   symmetric multivariate Gaussian distribution.
Let $L^2(\varphi)$ be the Hilbert space with inner product
\begin{align}
  \langle f_1, f_2 \rangle_{\varphi}
  := \int_{\mathbb R^d}f_1(x)f_2(x)\varphi(x) \mathrm dx, \quad f_1,f_2 \in L^2(\varphi).
\end{align}
Let $\mathbb Z_+ := \mathbb N\cup\{0\}$.
For each $p = (p_k)_{k = 1}^d \in \mathbb{Z}_+^{d}$, write $|p|:=\sum_{k=1}^d p_k$, $p!:= \prod_{k= 1}^d p_k!$ and $\partial_p:= \prod_{k = 1}^d(\partial^{p_k}/\partial x_k^{p_k})$.
The \emph{Hermite polynomials} are defined by
\begin{align}
  {\color{red}\mathcal H}_p(x)
  :=(-1)^{|p|}e^{|x|^2} \partial_p e^{-|x|^2}
  , \quad x\in \mathbb R^d, p \in \mathbb{Z}_+^{d}.
\end{align}
It is known that $(P_t)_{t\geq 0}$ is a strongly continuous semigroup in $L^2(\varphi)$ and its generator $L$ has discrete spectrum $\sigma(L)= \{-bk: k \in \mathbb Z_+\}$.
For $k \in \mathbb Z_+$, denote by $\mathcal{A}_k$ the eigenspace corresponding to the eigenvalue $-bk$, then $ \mathcal{A}_k = \operatorname{Span} \{\phi_p : p\in \mathbb Z_+^d, |p|=k\}$ where
\begin{align}
  \label{eigenfunction}
  \phi_p(x)
  := \frac{1}{\sqrt{ p! 2^{|p|} }} {\color{red}\mathcal H_p} \Big(\frac{ \sqrt{b} }{\sigma}x \Big)
  , \quad x\in \mathbb R^d, p\in \mathbb Z_+^d.
\end{align}
In other words,
\(
  P_t\phi_p(x)
  = e^{-b|p|t}\phi_p(x)
\)
for all $t\geq 0$, $x\in \mathbb R^d$ and $p\in \mathbb Z_+^d$.
Moreover, $\{\phi_p: p \in \mathbb Z_+^d\}$ forms a complete orthonormal basis of $L^2(\varphi)$.
Thus for each $f\in L^2(\varphi)$, we have
\begin{align}
  \label{semicomp1}
  f
  = \sum_{k=0}^{\infty}\sum_{p\in \mathbb Z_+^d:|p|=k}\langle f, \phi_p \rangle_{\varphi} \phi_p
  , \quad \text{in~} L^2(\varphi).
\end{align}
For each function $f\in L^2(\varphi)$, define the order of $f$ as
\[
  \kappa_f
  := \inf \left \{k\geq 0: \exists ~ p\in \mathbb Z_+^d , {\rm ~s.t.~} |p|=k {\rm ~and~}  \langle f, \phi_p \rangle_{\varphi}\neq 0\right \}
\]
which is the lowest non-trivial frequency in the eigen-expansion \eqref{semicomp1}.
Note that $ \kappa_f\geq 0$ and that, if $f\in L^2(\varphi)$ is non-trivial, then $\kappa_f<\infty$.
In particular, the order of any constant non-zero function is zero.
Denote by $\mathcal M_c(\mathbb R^d)$ the space of all finite Borel measures of compact support on $\mathbb R^d$.
For $p\in \mathbb{Z}_+^d$, define
\[
  H_t^p
  := e^{-(\alpha-|p|b)t}X_t(\phi_p), \qquad t\geq 0.
\]
We will write $H^0_t$ as $H_t$.
For each $u \neq -1$, we write $\tilde u = u/(1+ u)$.
We have shown in \cite[Lemma 3.2]{RenSongSunZhao2019Stable} the following. 
\begin{equation}\label{eq:Hinfty}
\begin{minipage}{0.9\textwidth}
	For any $\mu\in \mathcal M_c(\mathbb R^d)$, $(H_t^p)_{t\geq 0}$ is a $\mathbb{P}_{\mu}$-martingale.
	Futhermore, if $\alpha \tilde \beta>|p|b$, then for all $\gamma\in (0, \beta)$ and $\mu\in \mathcal M_c(\mathbb R^d)$,  $(H_t^p)_{t\geq 0}$ is a $\mathbb{P}_{\mu}$-martingale bounded in $L^{1+\gamma}(\mathbb{P}_{\mu})$;
	thus $H^p_{\infty}:=\lim_{t\rightarrow \infty}H_t^p$ exists $\mathbb{P}_{\mu}$-almost surely and in $L^{1+\gamma}(\mathbb P_\mu)$.
\end{minipage}
\end{equation}
We will write $H^0_\infty$ as $H_\infty$.

{\color{red}Let us also recall some results from \cite{RenSongSunZhao2019Stable} before we formulate our main theorem.}
Denote by $\mathcal P$ the class of functions of polynomial growth on $\mathbb R^d$, i.e.,
\begin{align}
  \label{eq: polynomial growth function}
  \mathcal{P}
  := \{f\in \mathcal B(\mathbb R^d, \mathbb R):\exists C>0, n \in \mathbb Z_+ \text{~s.t.~} \forall x\in \mathbb R^d, |f(x)|\leq C(1+|x|)^n \}.
\end{align}
It is clear that $\mathcal{P} \subset L^2(\varphi)$. 
Define 
\begin{equation}
\begin{minipage}{0.9\textwidth}
	$\mathcal C_\mathrm s := \mathcal P \cap \overline{\operatorname{Span}} \{ \phi_p: \alpha \tilde \beta < |p| b \}$, $\mathcal C_\mathrm c   := \mathcal P \cap \operatorname{Span} \{ \phi_p : \alpha \tilde \beta = |p| b \} $
	\\ and $ \mathcal C_\mathrm l   := \mathcal P \cap \operatorname{Span} \{ \phi_p: \alpha \tilde \beta > |p| b \}$.
\end{minipage}
\end{equation}
Define a semigroup
\begin{align}
\label{eq:I:R:1}
T_t f
:= \sum_{p \in \mathbb Z_+^d} e^{-\big| |p|b - \alpha \tilde \beta \big|t} \langle f, \phi_p \rangle_{\varphi} \phi_p
,\quad t\geq 0, f\in \mathcal P,
\end{align}
and a family of functional
\begin{align}
\label{eq:I:R:2}
m_t[f]
:= \eta \int_0^t \mathrm du \int_{\mathbb R^d} \big(-iT_u f(x)\big)^{1+\beta} \varphi(x) \mathrm dx
, \quad 0 \leq t< \infty, f\in \mathcal P.
\end{align}
We have shown in \cite[Lemma 2.6 and Propsoition 2.7]{RenSongSunZhao2019Stable} that,
\begin{equation} \label{eq:I:R:3}
\begin{minipage}{0.9\textwidth}
for each {\color{red}$f\in \mathcal P$}, there exists a $(1+\beta)$-stable random variable $\zeta^f$ with characteristic function
$
\theta \mapsto e^{m[\theta f]}, \theta \in \mathbb R,
$
where
\begin{equation}
m[f]
:= \begin{cases}
\lim_{t\to \infty} m_t[f], &
f \in \mathcal C_\mathrm s \oplus \mathcal C_\mathrm l, \\
\lim_{t\to \infty} \frac{1}{t} m_t[f], & f\in \mathcal P \setminus \mathcal C_\mathrm s \oplus \mathcal C_\mathrm l.
\end{cases}
\end{equation}
\end{minipage}
\end{equation}
Furthermore, we proved in \cite[Theorem 1.6]{RenSongSunZhao2019Stable} that
\begin{equation}\label{eq:oldResult}
\begin{minipage}{0.9\textwidth}
if $\mu\in \mathcal M_\mathrm c(\mathbb R^d)\setminus \{0\}$, $f_\mathrm s\in \mathcal C_s\setminus\{0\}$, $f_\mathrm c \in \mathcal C_c\setminus\{0
\}$ and $f_\mathrm l \in \mathcal C_l\setminus\{0\}$, then under $\mathbb {\widetilde P}_\mu$,
\begin{equation}
\begin{split}
&e^{-\alpha t}\|X_t\| \xrightarrow[t\to \infty]{\text{a.s.}} H_\infty;
\quad \frac{X_t(f_\mathrm s)}{\|X_t\|^{1-\tilde \beta}} \xrightarrow[t\to \infty]{d} \zeta^{f_\mathrm s};
\quad \frac{X_t(f_\mathrm c)}{\|t X_t\|^{1-\tilde \beta} } \xrightarrow[t\to \infty]{d} \zeta^{f_\mathrm c}; 
\\&\frac{X_t(f_\mathrm l) - \sum_{p\in \mathbb Z^d_+:\alpha \tilde \beta>|p|b}\langle f_\mathrm l,\phi_p\rangle_\varphi e^{(\alpha-|p|b)t}H^p_{\infty}}{\|X_t\|^{1-\tilde \beta}}
\xrightarrow[t\to \infty]{d}
\zeta^{-f_\mathrm l}.
\end{split}
\end{equation}
\end{minipage}
\end{equation}

	The main contribution of this note is to show that the convergences in \eqref{eq:oldResult} are \emph{asymptotic independent} in the following sense.

\begin{thm}
	\label{thm:M}
	If $\mu\in \mathcal M_\mathrm c(\mathbb R^d)\setminus \{0\}$, $f_\mathrm s\in \mathcal C_s\setminus\{0\}$, $f_\mathrm c \in \mathcal C_c\setminus\{0
\}$ and $f_\mathrm l \in \mathcal C_l\setminus\{0\}$, then under $\mathbb {\widetilde P}_\mu$,
	\begin{align}
	&S(t):=\Bigg(e^{-\alpha t}\|X_t\|, \frac{X_t(f_\mathrm s)}{\|X_t\|^{1-\tilde \beta}},\frac{X_t(f_\mathrm c)}{\|tX_t\|^{1-\tilde \beta}},\frac{ X_t(f_\mathrm l) - \sum_{p\in \mathbb Z^d_+:\alpha \tilde \beta>|p|b}\langle f_\mathrm l,\phi_p\rangle_\varphi e^{(\alpha-|p|b)t}H^p_{\infty}}{\|X_t\|^{1-\tilde \beta}}\Bigg)
	\\&\xrightarrow[t\rightarrow \infty]{d}({\color{red}\widetilde H_\infty},\zeta^{f_\mathrm s},\zeta^{f_\mathrm c},\zeta^{-f_\mathrm l}),
	\end{align}
	where ${\color{red}\widetilde H_\infty}$ has the distribution of $\{H_{\infty}; \widetilde {\mathbb P}_\mu\}$; $\zeta^{f_\mathrm s}$, $\zeta^{f_\mathrm c}$ and $\zeta^{-f_\mathrm l}$ are $(1+\beta)$-stable random variables given by \eqref{eq:I:R:3}; $\widetilde H_\infty$,  $\zeta^{f_\mathrm s}$, $\zeta^{f_\mathrm c}$ and $\zeta^{-f_\mathrm l}$ are independent.
\end{thm}

As a corollary of this theorem, we have {\color{red} central limit theorems for $X_t(f)$ for all $f\in \mathcal P$.}

\begin{cor} Let $\mu\in \mathcal M_c(\mathbb R^d)\setminus \{0\}$ and $f\in \mathcal P\setminus\{0\}$. Let {\color{red} $f=f_\mathrm s + f_\mathrm c + f_\mathrm l$ be the unique decomposition of $f$} with $f_\mathrm s \in \mathcal C_s$, $f_\mathrm c \in \mathcal C_c$ and $f_\mathrm l \in \mathcal C_l$.
Then under $\widetilde {\mathbb{P}}_{\mu}$, it holds that
\begin{enumerate}
\item  {\color{red}if $f_\mathrm c=0$}, then
\[
    \frac{ X_t(f) - \sum_{p\in \mathbb Z^d_+:\alpha \tilde \beta>|p|b}\langle f,\phi_p\rangle_\varphi e^{(\alpha-|p|b)t}H^p_{\infty} }{\|X_t\|^{1-\tilde \beta}}
    \xrightarrow[t\to \infty]{d}
     \zeta^{f_\mathrm s}+\zeta^{-f_\mathrm l};
\]
\item  {\color{red}if $f_\mathrm c\neq0$}, then
\[
  \frac{ X_t(f) - \sum_{p\in \mathbb Z^d_+:\alpha \tilde \beta>|p|b}\langle f,\phi_p\rangle_\varphi e^{(\alpha-|p|b)t}H^p_{\infty} }{\|tX_t\|^{1-\tilde \beta}}
    \xrightarrow[t\to \infty]{d}\zeta^{f_\mathrm c}.
\]
\end{enumerate}
{\color{red} Here, $\zeta^{f_\mathrm s}$, $\zeta^{f_\mathrm c}$ and $\zeta^{-f_\mathrm l}$ are independent $(1+\beta)$-stable random variables given by \eqref{eq:I:R:3}.}
\end{cor}


\section{Proofs of main results}
\label{proofs of main results}
In this section, we will prove the main result of this paper.
For simplicity, we will write $\mathbb{\widetilde{P}}_{\mu}=\mathbb{P}_{\mu}(\cdot|D^c)$ in this section.

\subsection{Preliminaries}

The following  elementary analysis result will play an important role in this paper.

\begin{lem}\label{ineq: analysis}
There exists a constant $C>0$, such that for any $x,y \in \mathbb R$,
\[
    |(x+y)^{1+\beta}-x^{1+\beta}-y^{1+\beta}|\leq C(|x||y|^{\beta}+|x|^{\beta}|y|).
\]
\end{lem}
\begin{proof}
   Note that
\[
  \lim_{|y|\rightarrow \infty}\frac{(y+1)^{1+\beta}-y^{1+\beta}-1}{y^{\beta}}=\lim_{|y|\rightarrow \infty}\frac{(y+1)^{1+\beta}-y^{1+\beta}}{y^{\beta}}=\lim_{|y|\rightarrow \infty}\big((1+\frac{1}{y})^{1+\beta}-1\big)y = 1+\beta.
\]
{\color{red}Using} this and continuity, we get that there exists $C_1>0$ such that for all {\color{red}$|y|\geq 1$},
\[
  |(1+y)^{1+\beta}-y^{1+\beta}-1|\leq C_1 |y|^{\beta}.
\]
{\color{red} Note that if $x = 0$ or $y= 0$, then the desired result is trivial. So we only have to consider the case that $x \neq 0$ and $y \neq 0$.}
{\color{red} In this case}, {\color{red}if $|x|\geq |y|$}, we have
\[
|(x+y)^{1+\beta}-x^{1+\beta}-y^{1+\beta}|\leq |y|^{1+\beta}\Big(|(1+\frac{x}{y})^{1+\beta}-(\frac{x}{y})^{1+\beta}-1|\Big)\leq C_1|y||x|^{\beta};
\]
and {\color{red} if $|x|\leq |y|$}, we have
\[
|(x+y)^{1+\beta}-x^{1+\beta}-y^{1+\beta}|\leq |x|^{1+\beta}\Big(|(1+\frac{y}{x})^{1+\beta}-(\frac{y}{x})^{1+\beta}-1|\Big)\leq C_1|x||y|^{\beta}.
\]
Combining the above, we immediately get the desired conclusion.
\end{proof}

For $g\in \mathcal P$, define $\mathcal P_g:= \{\theta T_ng:n \in \mathbb Z_+, \theta \in [-1,1]\}$. 	For all $t\geq 0$ and $f\in \mathcal P$, define
	\begin{align}
  	Z_t f
  	&:= \int_0^t P^\alpha_{t-s}\big( \eta (-i P^\alpha_sf)^{1+\beta}\big)ds,\\
  	\Upsilon^f_t
   & := \frac{X_{t+1} (f) - X_t(P_1^\alpha f)}{\| X_t\|^{1-{\color{blue}\tilde \beta}}}.
  	\end{align}
The following corollary to  \cite[Proposition 3.5]{RenSongSunZhao2019Stable} will be used later in the proof of Theorem \ref{thm:M}.

\begin{cor}
  \label{cor:MI}
For each  $f,g\in \mathcal P$ and $\mu\in \mathcal M_c(\mathbb R^d)$, there exist $C,\delta>0$ such that for
all $n_1,n_2 \in \mathbb Z_+$, {\color{red}$t\geq n_1+1$,} $(f_j)_{j=0}^{n_1}\subset \mathcal P_f$ and $(g_j)_{j=0}^{n_2}\subset \mathcal P_g$, we have 
\begin{equation}
  \label{32corollary}
  \Big|\mathbb{\widetilde{P}}_{\mu}\Big[  \Big(\prod_{k=0}^{n_1}e^{i \Upsilon^{f_k}_{t-k-1}}\Big)  \Big( \prod_{k=0}^{n_2}e^{i \Upsilon^{g_k}_{t+k} } \Big) \Big]  -  \Big(\prod_{k=0}^{n_1} e^{\langle Z_1f_k, \varphi\rangle}\Big) \Big(\prod_{k=0}^{n_2} e^{\langle Z_1g_k, \varphi\rangle}\Big) \Big|\leq C e^{-\delta (t-n_1)}.
\end{equation}
\end{cor}
\begin{proof}
	{\color{red} In this proof, let us fix $f,g\in \mathcal P$, $\mu\in \mathcal M_c(\mathbb R^d)$, $n_1,n_2 \in \mathbb Z_+$, $t\geq n_1 + 1$, $(f_j)_{j=0}^{n_1}\subset \mathcal P_f$ and $(g_j)_{j=0}^{n_2}\subset \mathcal P_g$.}
{\color{red} For any $k_1 \in \{-1,0,\dots,n_1\}$ and $k_2 \in \{-1,0,\dots,n_2\}$},  define
  \[
    a_{k_1,k_2}
    :=  \mathbb{\widetilde{P}}_{\mu}\Big[ \Big(\prod_{j=k_1+1}^{n_1} e^{i\Upsilon_{t-j-1}^{f_j}} \Big)  \Big(\prod_{j=0}^{k_2}e^{i\Upsilon_{t+j}^{g_j}}\Big) \Big] \Big(\prod_{j=0}^{k_1}e^{\langle Z_1 f_j, \varphi\rangle}\Big) \Big(\prod_{j=k_2+1}^{n_2} e^{ \langle Z_1g_j,\varphi \rangle} \Big),
  \]
 where by convention {\color{red} any product with the form $\prod_{j=0}^{-1} x(j)$ takes value $1$.}
  Then for all  $k_2 \in \{0,\dots,n_2\}$, we have
\begin{align}
& \begin{multlined}
	a_{-1,k_2} - a_{-1,k_2-1}
	\color{red} = \mathbb{\widetilde{P}}_{\mu}\Big[ \Big(\prod_{j=0}^{n_1} e^{i\Upsilon_{t-j-1}^{f_j}} \Big)  \Big(\prod_{j=0}^{k_2}e^{i\Upsilon_{t+j}^{g_j}}\Big) \Big] \Big(\prod_{j=k_2+1}^{n_2} e^{ \langle Z_1g_j,\varphi \rangle} \Big) 
	\\ \color{red} - \mathbb{\widetilde{P}}_{\mu}\Big[ \Big(\prod_{j=0}^{n_1} e^{i\Upsilon_{t-j-1}^{f_j}} \Big)  \Big(\prod_{j=0}^{k_2 - 1}e^{i\Upsilon_{t+j}^{g_j}}\Big) \Big] \Big(\prod_{j=k_2}^{n_2} e^{ \langle Z_1g_j,\varphi \rangle} \Big) ,
\end{multlined}
    \\ & =\frac{1}{\mathbb{P}_{\mu}(D^c)} \mathbb{P}_{\mu}\Big[\Big(\prod_{j=0}^{n_1}e^{i\Upsilon_{t-j-1}^{f_j}}\Big) \Big(\prod_{j=0}^{k_2-1} e^{i\Upsilon_{t+j}^{g_j}}\Big) (e^{i\Upsilon^{g_{k_2}}_{t+k_2}}-e^{\langle Z_1g_{k_2}, \varphi\rangle});D^c\Big] \Big(\prod_{j=k_2+1}^{{\color{red} n_2}} e^{\langle Z_1g_j, \varphi\rangle}\Big)\\
    & = \frac{1}{\mathbb{P}_{\mu}(D^c)}  \mathbb{P}_{\mu}\Big[\Big(\prod_{j=0}^{n_1}e^{i\Upsilon_{t-j-1}^{f_j}}\Big)\Big(\prod_{j=0}^{k_2-1} e^{i\Upsilon_{t+j}^{g_j}}\Big) \mathbb P_\mu[e^{i\Upsilon^{g_{k_2}}_{t+k_2}}-e^{\langle Z_1g_{k_2}, \varphi\rangle}; D^c|\mathscr F_{t+k_2}] \Big] \Big(\prod_{j=k_2+1}^{{\color{red} n_2}}e^{\langle Z_1g_j, \varphi\rangle}\Big).
\end{align}

It follows from {\color{red}\cite[Proposition 3.5]{RenSongSunZhao2019Stable}} that
  there exist $C_0,\delta_0 >0$, {\color{red} only depend on $\mu$ and $g$}, such that {\color{red} for each $k \in \{0, \dots, n_2 \}$,}
\begin{equation}
    | a_{-1,k_2} - a_{-1,k_2-1}|
    \leq \mathbb{P}_{\mu}(D^c)^{-1}\mathbb{P}_{\mu}\Big[\big|\mathbb P_\mu[e^{i\Upsilon^{g_{k_2}}_{t+k_2}}-e^{\langle Z_1g_{k_2}, \varphi\rangle}; D^c|\mathscr F_{t+k_2}]\big|\Big]
    \leq C_0 e^{-\delta_0 (t+k_2)}.
\end{equation}
{\color{red} 
	Notice that, for any $k_1 \in \{0, \dots , n_1\}$,
	\begin{align}
	&a_{k_1-1,-1} - a_{k_1,-1}
	\\ & \begin{multlined}
	=  \mathbb{\widetilde{P}}_{\mu}\Big[ \prod_{j=k_1}^{n_1} e^{i\Upsilon_{t-j-1}^{f_j}} \Big] \Big(\prod_{j=0}^{k_1-1}e^{\langle Z_1 f_j, \varphi\rangle}\Big) \Big(\prod_{j=0}^{n_2} e^{ \langle Z_1g_j,\varphi \rangle} \Big) - {}
	\\ \mathbb{\widetilde{P}}_{\mu}\Big[ \prod_{j=k_1+1}^{n_1} e^{i\Upsilon_{t-j-1}^{f_j}} \Big] \Big(\prod_{j=0}^{k_1}e^{\langle Z_1 f_j, \varphi\rangle}\Big) \Big(\prod_{j=0}^{n_2} e^{ \langle Z_1g_j,\varphi \rangle} \Big)
	\end{multlined}
	\\& =  \mathbb{\widetilde{P}}_{\mu}\Big[ \big(e^{i\Upsilon_{t-k_1-1}^{f_{k_1}}} -e^{\langle Z_1 f_{k_1}, \varphi\rangle} \big) \prod_{j=k_1+1}^{n_1} e^{i\Upsilon_{t-j-1}^{f_j}} \Big] \Big(\prod_{j=0}^{k_1-1}e^{\langle Z_1 f_j, \varphi\rangle}\Big) \Big(\prod_{j=0}^{n_2} e^{ \langle Z_1g_j,\varphi \rangle} \Big)
	\\& \begin{multlined}
	=  \frac{1}{\mathbb P_\mu(D^c)} \mathbb{P}_{\mu}\Big[ \mathbb P_\mu\big[\big(e^{i\Upsilon_{t-k_1-1}^{f_{k_1}}} -e^{\langle Z_1 f_{k_1}, \varphi\rangle} \big) ; D^c \big| \mathscr F_{t-k_1 - 1}\big] \prod_{j=k_1+1}^{n_1} e^{i\Upsilon_{t-j-1}^{f_j}} \Big] \times {}
	\\ \Big(\prod_{j=0}^{k_1-1}e^{\langle Z_1 f_j, \varphi\rangle}\Big) \Big(\prod_{j=0}^{n_2} e^{ \langle Z_1g_j,\varphi \rangle} \Big)
	\end{multlined}
	\end{align}
}
	Therefore,  there exist $C_1,\delta_1 >0$, {\color{red} only depend on $\mu$ and $f$, such that for any $k_1 \in \{0,\dots,n_1\}$,}
\begin{align}
    | a_{k_1-1,-1} - a_{k_1,-1}|
    & \leq \frac{1}{\mathbb{P}_{\mu}(D^c)}\mathbb{P}_{\mu}\Big[\big|\mathbb P_\mu[e^{i\Upsilon^{f_{k_1}}_{t-k_1-1}}-e^{\langle Z_1f_{k_1}, \varphi\rangle}; D^c|\mathscr F_{t-k_1-1}]\big|\Big]
    \leq C_1 e^{-\delta_1 (t-k_1)}.
\end{align}

  Therefore, there exist $C,\delta >0$ {\color{red} only depend on $f,g$ and $\mu$, such that }
  \begin{align}
    \text{LHS of \eqref{32corollary}}
    & = \left|a_{-1,n_2}-a_{n_1,-1}\right|
      \leq \sum_{k=0}^{n_1}\left|a_{k-1,-1}-a_{k,-1}\right|+\sum_{k=0}^{n_2}\left|a_{-1,k}-a_{-1,k-1}\right|\\
     & \leq \sum_{k=0}^{n_1} C_1 e^{-\delta_1 (t-k)}+\sum_{k=0}^{n_2} C_0 e^{-\delta_0 (t+k)}
      \leq C e^{- \delta (t-n_1)}.
      \qedhere
  \end{align}
\end{proof}

\subsection{Proof of Theorem \ref{thm:M}}
\label{sec: main}
In this subsection, we always fix {\color{red} $\mu \in \mathcal M_c(\mathbb R^d)\setminus\{0\}$, $f_\mathrm s\in \mathcal C_s\setminus\{0\}$,  $f_\mathrm c\in \mathcal C_c\setminus\{0\}$, $f_\mathrm l\in \mathcal C_l\setminus\{0\}$} and $t_0>1$ large enough, so that $\lceil t-\ln t\rceil \leq \lfloor t \rfloor - 1$ for all $t\geq t_0$.
	For any random variable $Y$ with finite mean, we define $\mathcal I_r^t Y:= \mathcal I_r^t [Y, \mu] := \mathbb P_\mu[Y|\mathscr F_t] - \mathbb P_\mu[Y|\mathscr F_r]$ where $0 \leq r \leq t <\infty.$
	For each $t\geq t_0$, we have the following decomposition.
		{\color{blue} \tt Maybe here we can simplify the notation, by allowing $r<0$, and by using $\lceil \cdot \rceil$.}
  \begin{multline}
    \label{eq:PM:CLTS:1}
     \|X_t\|^{\tilde \beta - 1}  X_t(f_\mathrm s)
     = I^{f_\mathrm s}_1(t) + I^{f_\mathrm s}_2(t) + I^{f_\mathrm s}_3(t)
    := \Big(\sum_{k=0}^{\lfloor t-\ln t \rfloor}  \|X_t\|^{\tilde \beta - 1} \mathcal I_{t-k-1}^{t-k} X_t(f_\mathrm s) \Big)\\
    + \Big(  \|X_t\|^{\tilde \beta - 1} \mathcal I_0^{t-\lfloor t \rfloor} X_t(f_\mathrm s)   + \sum_{k=\lfloor t-\ln t \rfloor+1}^{\lfloor t \rfloor-1} \|X_t\|^{\tilde \beta - 1} \mathcal I_{t-k-1}^{t-k} X_t(f_\mathrm s) \Big) + ( \|X_t\|^{\tilde \beta - 1}X_0(P_t^\alpha f_\mathrm s) ),
  \end{multline}
  \begin{multline}
    \label{eq:PM:CLTS:2}
     \|tX_t\|^{\tilde \beta - 1}  X_t(f_\mathrm c)
     = I^{f_\mathrm c}_1(t) + I^{f_\mathrm c}_2(t) + I^{f_\mathrm c}_3(t)
    := \Big(\sum_{k=0}^{\lfloor t-\ln t \rfloor}  \|tX_t\|^{\tilde \beta - 1} \mathcal I_{t-k-1}^{t-k} X_t(f_\mathrm c) \Big)\\
    + \Big(  \|tX_t\|^{\tilde \beta - 1} \mathcal I_0^{t-\lfloor t \rfloor} X_t(f_\mathrm c)   + \sum_{k=\lfloor t-\ln t \rfloor+1}^{\lfloor t \rfloor-1} \|tX_t\|^{\tilde \beta - 1} \mathcal I_{t-k-1}^{t-k} X_t(f_\mathrm c) \Big) + ( \|tX_t\|^{\tilde \beta - 1}X_0(P_t^\alpha f_\mathrm c) ),
  \end{multline}
and
 \begin{align}
    & \frac{X_t(f_\mathrm l) - \sum_{p\in \mathcal N} \langle f_\mathrm l,\phi_p\rangle_\varphi e^{(\alpha - |p|b)t}H_\infty^p}{\|X_t\|^{1- \tilde \beta}}
      = \sum_{p\in \mathcal N}\frac{ \langle f_\mathrm l,\phi_p\rangle_\varphi [X_t(\phi_p) - e^{(\alpha - |p|b)t}H_\infty^p]}{\|X_t\|^{1- \tilde \beta}}
    \\& = \sum_{p \in \mathcal N} \frac{\langle f_\mathrm l,\phi_p\rangle_\varphi e^{(\alpha - |p|b)t}(H_t^p - H_\infty^p)}{\|X_t\|^{1- \tilde \beta}}
    {\color{blue}=} \sum_{k=0}^\infty \sum_{p \in \mathcal N}  \langle f_\mathrm l,\phi_p\rangle_\varphi e^{(\alpha - |p|b)t}\frac{ H_{t+k}^p - H_{t+k+1}^p}{\|X_t\|^{1- \tilde \beta}}
    \\ &= \Big(\sum_{k = 0}^{\lfloor t^2 \rfloor}  +\sum_{k = \lceil t^2 \rceil}^\infty \Big)\Big(\sum_{p \in \mathcal N}  \langle f_\mathrm l,\phi_p\rangle_\varphi e^{(\alpha - |p|b)t}\frac{ H_{t+k}^p - H_{t+k+1}^p}{\|X_t\|^{1- \tilde \beta}}\Big)
         = : I^{f_\mathrm l}_1(t) + I^{f_\mathrm l}_2(t),
  \end{align}
where $\mathcal N:= \{p\in \mathbb Z_+^d: \alpha \tilde \beta > |p|b\}$.

Define $I_3^{f_\mathrm l}(t)=0$, $t\geq t_0$. For all $t\geq t_0$ and $j=1,2,3$, let
\(
   R_j(t):=\big(I_j^{f_\mathrm s}(t),I_j^{f_\mathrm c}(t),I_j^{f_\mathrm l}(t)\big)
\)
and
\begin{align}
    R_0(t)=(I_0^{f_\mathrm s}(t),I_0^{f_\mathrm c}(t),I_0^{f_\mathrm l}(t))
    :=\Big(\sum_{k=0}^{\lfloor t-\ln t \rfloor} \Upsilon_{t-k-1}^{T_k \tilde f_\mathrm s},t^{\tilde \beta - 1}\sum_{k=0}^{\lfloor t-\ln t \rfloor} \Upsilon_{t-k-1}^{T_{k} \tilde f_\mathrm c},\sum_{k = 0}^{\lfloor t^2 \rfloor} \Upsilon_{t+k}^{- T_k \tilde f_\mathrm l}\Big),
\end{align}
where $\tilde f_\mathrm s:=e^{\alpha(\tilde \beta - 1)} f_\mathrm s$, $\tilde f_\mathrm c:=e^{\alpha(\tilde \beta - 1)} f_\mathrm c$ and  $\tilde f_\mathrm l := \sum_{p\in \mathcal N} e^{-(\alpha - |p|b)}\langle f_\mathrm l, \phi_p \rangle_\varphi \phi_p$.

Define
\begin{align}
R(t):=\Bigg(e^{-\alpha t}\|X_t\|, \frac{X_t(f_\mathrm s)}{\|X_t\|^{\frac{1}{1+\beta}}},\frac{X_t(f_\mathrm c)}{\|tX_t\|^{\frac{1}{1+\beta}}},\frac{ X_t(f_\mathrm l) - \sum_{p\in \mathbb Z^d_+:\alpha \tilde \beta>|p|b}\langle f_\mathrm l,\phi_p\rangle_\varphi e^{(\alpha-|p|b)t}H^p_{\infty}}{\|X_t\|^{\frac{1}{1+\beta}}}\Bigg).
\end{align}
The following result is a special case of Theorem \ref{thm:M}.
\begin{thm}\label{thm: II}
Under $\mathbb{\widetilde{P}}_{\mu}(\cdot)$,
\begin{align}
R(t) \xrightarrow[t\rightarrow\infty]{d}(\zeta^{f_\mathrm s},\zeta^{f_\mathrm c},\zeta^{-f_\mathrm l}),
\end{align}
where $\zeta^{f_\mathrm s},\zeta^{f_\mathrm c}$ and $\zeta^{-f_\mathrm l}$ are independent $(1+\beta)$-stable random variables with characteristic functions described by \eqref{eq: charac func11}.
\end{thm}
\begin{proof}

Note that for each $t\geq t_0$,
\[
R(t)=R_0(t)+(R_1(t)-R_0(t))+R_2(t)+R_3(t).
\]

\begin{itemize}
\item[(1)] In Lemma \ref{lem: U0T}, we will show that, under $\mathbb{\widetilde{P}}_{\mu}$, $R_0(t) \xrightarrow[t\to \infty]{d}(\zeta^{f_\mathrm s},\zeta^{f_\mathrm c},\zeta^{-f_\mathrm l})$.
\item[(2)] In Lemma \ref{lem: U1-U0T}, we will show that, under $\mathbb{\widetilde{P}}_{\mu}(\cdot)$, $R_1(t)-R_0(t)\xrightarrow[t\to \infty]{d} (0,0,0)$.
\item[(3)] In Lemma \ref{lem: U2T}, we will show that, under $\mathbb{\widetilde{P}}_{\mu}(\cdot)$, $R_2(t)\xrightarrow[t\to \infty]{d}(0,0,0)$.
\item[(4)] In Lemma \ref{lem: U3T}, we will show that, under $\mathbb{\widetilde{P}}_{\mu}(\cdot)$, $R_3(t)\xrightarrow[t\to \infty]{d}(0,0,0)$.
\end{itemize}

{\color{blue}\tt Do we have this kind of result?: If $(X_t)_{t\geq 0}$ and $(Y_t)_{t\geq 0}$ are defined in a same probability space such that $X_t\xrightarrow[t\to \infty]{d}0$ and $Y_t\xrightarrow[t\to \infty]{d} 0$. Then $(X_t,Y_t) \xrightarrow[t\to \infty]{d} (0,0)$. If we do, then maybe (2)-(4) above are trivial.}

Combining these, we immediately get the conclusion of the theorem. {\color{blue}\tt Maybe we want to mention Slutsky's theorem here?}
\end{proof}

\begin{lem}\label{lem: U0T}

Under $\mathbb{\widetilde{P}}_{\mu}(\cdot)$,
 $R_0(t) \xrightarrow[t\to \infty]{d}(\zeta^{f_\mathrm s},\zeta^{f_\mathrm c},\zeta^{-f_\mathrm l})$, where $\zeta^{f_\mathrm s},\zeta^{f_\mathrm c}$ and $\zeta^{-f_\mathrm l}$ are independent $(1+\beta)$-stable random variables with characteristic function described by \eqref{eq: charac func11}.
\end{lem}
\begin{proof}
Since $\Upsilon_t^f$ is linear in $f$, for each $t>t_0$,
\[
\widetilde{\mathbb P}_{\mu}\Big[\exp\Big(i
\sum_{j=s,c,l}I_0^{f_j}(t)\Big)\Big]
= \widetilde{\mathbb P}_{\mu}\Big[\exp\Big(i\sum_{k=0}^{\lfloor t-\ln t \rfloor}\Upsilon_{t-k-1}^{T_k(\tilde{f_\mathrm s}+t^{\tilde{\beta}-1} \tilde{f}_c)}\Big)\exp\Big(i\sum_{k=0}^{\lfloor t^2 \rfloor}\Upsilon_{t+k}^{-T_k\tilde{f}_l}\Big)\Big].
\]
Using Corollary \ref{cor:MI} with $f=\tilde{f_\mathrm s}+t^{\tilde{\beta}-1} \tilde{f}_c$ and $g = -\tilde{f}_l$, we get that there exist $C_1,\delta_1 > 0$ such that
  \begin{align}
    &\Big|\widetilde{\mathbb P}_{\mu}\Big[\exp\Big(i
    \sum_{j=s,c,l}I_0^{f_j}(t)\Big)\Big]
    -\exp\Big(\sum_{k=0}^{\lfloor t-\ln t \rfloor} \langle Z_1(T_{k}(\tilde f_\mathrm s+t^{\tilde{\beta}-1}\tilde{f}_c)), \varphi\rangle \Big)\exp\Big(\sum_{k=0}^{\lfloor t^2 \rfloor}\langle Z_1(-T_k\tilde{f}_l),\varphi\rangle\Big)\Big|\\
    &\leq C_1 e^{-\delta_1(t - \lfloor t - \ln t\rfloor)},
    \quad t\geq t_0.
  \end{align}
Next, we claim that
\begin{align}
\lim_{t\rightarrow\infty}\exp\Big(\sum_{k=0}^{\lfloor t-\ln t \rfloor} \langle Z_1(T_{k}(\tilde f_\mathrm s+t^{\tilde{\beta}-1}\tilde{f}_c)), \varphi\rangle \Big)\exp\Big(\sum_{k=0}^{\lfloor t^2 \rfloor}\langle Z_1(-T_k\tilde{f}_l),\varphi\rangle\Big) =
\prod_{j=s,c,l}\exp(m[f_j]).
\end{align}
 Then we have
 $\mathbb{\widetilde{P}}_{\mu} [e^{i \sum_{j=s,c,l}I^{f^j}_0(t)} ]
 \xrightarrow[t\to \infty]{}  \prod_{j=s,c,l}\exp(m[f_j])$.
Since $I_0^{f_j}(t)$, $j=s,c,l$, is linear in $f_j$, replacing $f_j$ with $\theta_j f_j$, we immediately get the desired result.

Now we prove the above claim.

 Recall that  $\tilde f_\mathrm l := \sum_{p\in \mathcal N} e^{-(\alpha - |p|b)}\langle f_\mathrm l, \phi_p \rangle_\varphi \phi_p$.
  Using \eqref{eq:I:R:2} and the fact that $\varphi(x)dx$ is the invariant distribution of the semigroup $(P_t)_{t\geq 0}$, we get that for each $n\in \mathbb Z_+$,
  \begin{align}
    \label{eq:PM:CLTS:2a}
    & \sum_{k=0}^n \langle Z_1 T_{k} (-\tilde f_\mathrm l), \varphi \rangle
      = \sum_{k=0}^n \int_0^1 \langle P_u^\alpha ((iP_{1 - u}^\alpha T_k \tilde f_\mathrm l)^{1+\beta}), \varphi\rangle ~du
    \\& = \sum_{k=0}^n \int_0^1 e^{\alpha u} \langle  (iP_{1 - u}^\alpha T_{k}\tilde f_\mathrm l)^{1+\beta}, \varphi \rangle ~du
    \\& = \sum_{k=0}^n \int_0^1 \langle  (iT_{k+ u} f_\mathrm l)^{1+\beta}, \varphi\rangle~du
    = \int_0^{n+1} \langle  (iT_{u} f_\mathrm l)^{1+\beta}, \varphi\rangle~du = m_{n+1}[-f_\mathrm l].
  \end{align}
Therefore, we have
\[
\lim_{t\rightarrow \infty}\exp\Big(\sum_{k=0}^{\lfloor t^2 \rfloor}\langle Z_1(-T_k\tilde{f}_l),\varphi\rangle\Big) = \lim_{t\rightarrow \infty}\exp(m_{\lfloor t^2 \rfloor+1}[-f])=\exp(m[-f]).
\]

Similarly, for each $f\in \mathcal C_s  \oplus \mathcal C_c$ and $n\in \mathbb Z_+$.
  \begin{align}
    & \sum_{k=0}^n \langle Z_1 T_{k} \tilde f, \varphi \rangle
    = \sum_{k=0}^n \int_0^1 \langle P_u^\alpha ((-iP_{1 - u}^\alpha T_k \tilde f)^{1+\beta}), \varphi\rangle ~du
    \\& = \sum_{k=0}^n \int_0^1 e^{\alpha u} \langle  (-iP_{1 - u}^\alpha T_{k}\tilde f)^{1+\beta}, \varphi \rangle ~du
    \\& = \sum_{k=0}^n \int_0^1 \langle  (-iT_{k+1 - u} f)^{1+\beta}, \varphi\rangle~du
    = \int_0^{n+1} \langle  (-iT_{u} f)^{1+\beta}, \varphi\rangle~du = m_{n+1}[f],
  \end{align}
where $\tilde f=e^{\alpha(\tilde \beta - 1)} f$.
Therefore, for any $t\geq t_0$,
\[
{\color{red}\sum_{k=0}^{\lfloor t-\ln t \rfloor} \langle Z_1T_{k}(\tilde f_\mathrm s+t^{\tilde{\beta}-1}\tilde{f}_c), \varphi\rangle }
=\int_0^{\lceil t-\ln t \rceil}\big\langle \big(-iT_u(f_\mathrm s+t^{\tilde{\beta}-1}f_\mathrm c)\big)^{1+\beta},\varphi \big\rangle du.
\]
Note that for each {\color{red}$u\geq 0$}, $T_uf_\mathrm c=f_\mathrm c$. 
According to Step 1 in the proof of \cite[Lemma 2.6]{RenSongSunZhao2019Stable}, {\color{red}there exist $\delta> 0$ and $h\in \mathcal P$, (only depend on $f_\mathrm s$), such that for each $u\geq 0$,}
 $|T_uf_\mathrm s|\leq e^{-\delta u}h$. 

{\color{red}According to Lemma \ref{ineq: analysis}, there exist $C>0$, (independent of the chose of $f_\mathrm s$ and $f_\mathrm c$), such that for each $u\geq 0$ and $t\geq 0$,}
\begin{align}
  &|(-i(T_uf_\mathrm s+t^{\tilde{\beta}-1}T_uf_\mathrm c))^{1+\beta}-(-iT_uf_\mathrm s)^{1+\beta}-(-it^{\tilde{\beta}-1}T_uf_\mathrm c)^{1+\beta}|
  \\& \color{red} = |-i|^{1+\beta} |(T_u f + t^{\tilde \beta -1} T_u f_\mathrm c)^{1+\beta} - (T_u f_\mathrm s)^{1+\beta} - (t^{\tilde \beta - 1}T_u f_\mathrm c)^{1+\beta}|
  \\&\leq {\color{red} C}(t^{-\frac{\beta}{1+\beta}}|T_uf_\mathrm s||T_uf_\mathrm c|^{\beta}+t^{-\frac{1}{1+\beta}}|T_uf_\mathrm s|^{\beta}|T_uf_\mathrm c|)\\
&\leq {\color{red}C}(t^{-\frac{\beta}{1+\beta}}e^{-\delta u}h|f_\mathrm c|^{\beta}+t^{-\frac{1}{1+\beta}}e^{-\delta\beta u}h^{\beta}|f_\mathrm c|),
\end{align}
which means that
\begin{align}
&\Big|{\color{red} \Big(\sum_{k=0}^{\lfloor t-\ln t \rfloor} \langle Z_1T_{k}(\tilde f_\mathrm s+t^{\tilde{\beta}-1}\tilde{f}_c), \varphi\rangle \Big)}-m_{\lceil t-\ln t \rceil}[f_\mathrm s]-\frac{1}{t}m_{\lceil t-\ln t \rceil}[f_\mathrm c]\Big|\\
&{\color{blue}\leq} C_1\int_0^{\lceil t-\ln t \rceil}(t^{-\frac{\beta}{1+\beta}}e^{-\delta u}h|f_\mathrm c|^{\beta}+t^{-\frac{1}{1+\beta}}e^{-\delta\beta u}h^{\beta}|f_\mathrm c|)du {\color{blue}\xrightarrow{t\rightarrow \infty} 0.}
\end{align}
Combining this with \eqref{eq:I:R:3}, we get that
\[
\lim_{t\rightarrow \infty}\exp\Big(\sum_{k=0}^{\lfloor t-\ln t \rfloor} \langle Z_1T_{k}(\tilde f_\mathrm s+t^{\tilde{\beta}-1}\tilde{f}_c), \varphi\rangle \Big) {\color{red} = \exp\{ m[f_\mathrm s]+m[f_\mathrm c]\}.}
\]
Thus the claim is valid.
\end{proof}

\begin{lem}\label{lem: U1-U0T}
Under $\mathbb{\widetilde{P}}_{\mu}(\cdot)$,
 $R_1(t)-R_0(t)\xrightarrow[t\to \infty]{d}0$.
\end{lem}
\begin{proof}
It follows from \cite[Lemma 3.4.3]{Durrett2010Probability} that for $t\geq t_0$,
\[
\Big|\widetilde {\mathbb P}_{\mu}\Big[\exp\Big(i
\sum_{j=\mathrm s,\mathrm c,\mathrm l}(I_{1}^{f_j}(t) - I_0^{f_j}(t))\Big)
 - 1\Big]\Big| \leq  \sum_{k=0}^{\lfloor t-\ln t \rfloor}(\widetilde {\mathbb {P}}_\mu[|Y^s_{t,k}|] + \widetilde {\mathbb {P}}_\mu[|Y^c_{t,k}|] )+ \sum_{k=0}^{\lfloor t^2 \rfloor} \widetilde {\mathbb {P}}_\mu[|Y^l_{t,k}|]
\]
 where
 \begin{align}
 &Y^s_{t,k} := \exp\Big(i(\|X_t\|^{\tilde{\beta}-1}\mathcal I_{t-k-1}^{t-k}X_t(f_\mathrm s) - \Upsilon_{t-k-1}^{T_{k}\widetilde {f_\mathrm s}})\Big) - 1, \\
 &Y^c_{t,k} := \exp\Big(i(\|tX_t\|^{\tilde{\beta}-1}\mathcal I_{t-k-1}^{t-k}X_t(f_\mathrm c) - \Upsilon_{t-k-1}^{T_{k}\widetilde {f_\mathrm c}})\Big) - 1,\\
&Y^l_{t,k} := \exp\Big(i(\widetilde{\Upsilon}_{t,k} - \Upsilon_{t+k}^{-T_{k}\widetilde {f_\mathrm l}})\Big) - 1,\\
&\widetilde{\Upsilon}_{t,k}=\|X_t\|^{\tilde{\beta}-1}\Big( \sum_{p\in \mathcal N}\langle f_\mathrm l,\phi_p\rangle_{\varphi}e^{(\alpha-|p|b)t}(H_{t+k}^p-H_{t+k+1}^p)\Big).
\end{align}
  We claim that there exist $C_1, \delta_1>0$ such that
\begin{align}
 \widetilde {\mathbb {P}}_\mu[|Y^j_{t,k}|]
 \leq C_1e^{-\delta_1 (t-k)}, \quad  t\geq t_0, k \in \mathbb Z_+ \text{~satisfying~} k\leq t-1,
  j=\mathrm s,\mathrm c,
\end{align}
and
\[
 \widetilde {\mathbb {P}}_\mu[|Y^l_{t,k}|] \leq C_1e^{-\delta_1 t}, \quad t\geq t_0, k \in \mathbb Z_+.
\]
  Then there exists $C_2>0$ such that for all $t>t_0$, we have
$$
\big|\widetilde {\mathbb P}_{\mu}\left[\exp\left(i\sum_{j=s,c,l}(I_{1}^{f_j}(t) - I_0^{f_j}(t))\right) - 1\right]\big| \leq C_2(t^{-\delta_1}+(1+t^2)e^{-\delta_1 t})
$$
which, combined with the fact that
$I^{f_j}_1(t) - I^{f_j}_0(t)$, $j=s,c,l$, is linear in $f_j$, completes the proof of this lemma.

  We now show the claim above in the following steps.
  First we choose $\gamma\in(0, \beta)$ close enough to $\beta$ so that $\alpha \tilde \gamma > |p|b$ for each $p\in \mathcal N$,
  and that there exist $\eta,\eta'>0$ satisfying $\alpha \tilde \gamma > \eta>\eta - 3\eta'> \alpha (\tilde \beta - \tilde \gamma)>0$.
  For all $0<r<t$, we define
\[
\mathcal{D}_{t,r} :=\{|H_t-H_{r}|\leq  e^{-\eta r}, H_{r}> 2e^{-\eta' r}\}.
\]
 \emph{Step 1.} We show that there exist $C_{1.1},\delta_{1.1} >0$ such that
$\mathbb{\widetilde{P}}_{\mu} \big[ \mathcal{D}^c_{t,r} \big] \leq C_{1.1} e^{-\delta_{1.1} r}$ for all $0<r<t$.


 It follows from \cite[Proposition 2.10 \& Lemma 3.3 with $|p|=0$]{RenSongSunZhao2019Stable} and Chebyshev's inequality that there exist $C_{1.1}', \delta_{1.1}'>0$ such that for all
  $0<r<t$,
  \begin{align}
    & \mathbb{\widetilde{P}}_{\mu}(\mathcal{D}_{t,r}^c)
    \leq \mathbb{\widetilde{P}}_{\mu}(|H_t-H_{r}| > e^{-\eta r})+\mathbb{\widetilde{P}}_{\mu}(H_{r}\leq 2e^{-\eta'r}) \\
    & \leq \mathbb{P}_{\mu}(D^c)^{-1}e^{\eta r}\mathbb{P}_{\mu}[|H_t-H_r|] +  \mathbb{P}_{\mu}(D^c)^{-1} \mathbb P_\mu(H_r\leq 2e^{-\eta'r}; D^c) \\
    & \leq \mathbb{P}_{\mu}(D^c)^{-1}  e^{\eta r}\|H_t - H_r\|_{\mathbb P_\mu; 1+\gamma} + \mathbb{P}_{\mu}(D^c)^{-1} \mathbb P_\mu(0<H_r\leq 2e^{-\eta'r}) \\
    & \leq C'_{1.1} e^{-(\alpha \tilde \gamma - \eta)r}+C'_{1.1} e^{-\delta'_{1.1}r}.
  \end{align}
  \emph{Step 2.} We show that there exist $C_{2.1},\delta_{2.1} > 0$ such that for all $t>t_0$ and $k\in \mathbb Z_+$ satisfying $k\leq t-1$, it holds that $ \mathbb{\widetilde{P}}_{\mu} [|Y^s_{t,k}|] \leq  C_{2.1} e^{-\delta_{2.1} (t-k)}$.

  Note that for $f\in \mathcal C_s$ and $k\in \mathbb Z_+$, we have $T_kf = e^{\alpha (\tilde \beta - 1 )k}P_k^\alpha f $. Therefore for all $t>t_0$ and $k\in \mathbb Z_+$ satisfying $k\leq t-1$,
  \begin{align}
    \label{eq:gammafunction11}
    \Upsilon_{t-k-1}^{T_{k} \tilde f_\mathrm s}
    = \frac{X_{t-k}(T_{k} \tilde  f_\mathrm s) - X_{t -k-1}(P_1^\alpha T_{k} \tilde f_\mathrm s)}{\|X_{t-k-1}\|^{1-\tilde \beta}}
    = \frac{\mathcal I_{t - k - 1}^{t - k} X_t(f_\mathrm s)}{\|e^{\alpha (k+1)}X_{t-k-1} \|^{1 -\tilde \beta}}.
  \end{align}
  Since $|e^{ix}-e^{iy}|\leq|x-y|$ for all $x,y\in \mathbb R$, we have  for all $t>t_0$ and $k\in \mathbb Z_+$ satisfying $k\leq t-1$,
  \begin{align}
    \label{eq: control of Ykt}
    \mathbb{\widetilde{P}}_{\mu}[|Y^s_{t,k}|;\mathcal{D}_{t,t-k-1}]
    & \leq \mathbb{\widetilde{P}}_{\mu}\Big[|\mathcal I_{t-k-1}^{t-k} X_t(f_\mathrm s) | \cdot \Big| \| e^{\alpha(k+1)}X_{t-k-1}\| ^{ \tilde \beta - 1} - \|X_t\|^{ \tilde \beta - 1}\Big|; \mathcal D_{t,t-k-1}\Big] \\
    & \leq  e^{\alpha(\tilde \beta - 1)t}\mathbb{\widetilde{P}}_{\mu}\big[|\mathcal I_{t-k-1}^{t-k}X_t(f_\mathrm s)|\cdot K_{t,k}\big],
  \end{align}
  where
  \[
    K_{t,k}
   := \Big| \frac {(H_t)^{1- \tilde \beta} - (H_{t-k-1})^{1 - \tilde \beta}} {(H_t)^{1 - \tilde \beta}
   (H_{t-k-1})^{ 1- \tilde \beta }} \Big| \mathbf{1}_{\mathcal{D}_{t,t-k-1}}.
  \]
  Note that, since $\eta' < \eta$, we have almost surely on $\mathcal D_{t,t-k-1}$,
  \begin{align}
    H_t
    & \geq H_{t-k-1}- e^{-\eta (t-k-1)}
      \geq 2e^{-\eta'(t-k-1)}-e^{-\eta(t-k-1)}
      \geq e^{-\eta'(t-k-1)}.
  \end{align}
  Therefore, for all $t>t_0$ and $k\in \mathbb Z_+$ satisfying $k\leq t-1$, almost surely  on $\mathcal D_{t,t-k-1}$,
  \begin{align}
   & \Big|(H_t)^{1- \tilde \beta}-(H_{t-k-1})^{1- \tilde \beta}\Big|
      \leq (1- \tilde \beta) \max \{ (H_t)^{-\tilde \beta }, (H_{t-k-1})^{ -\tilde \beta} \} | H_t - H_{t-k-1} | \\
    & \leq (1- \tilde \beta ) \max\{e^{\eta' (t-k-1)}, \frac{1}{2}e^{\eta'(t-k-1)}\}^{\tilde \beta} e^{-\eta(t-k-1)}  \leq (1- \tilde \beta) e^{-(\eta - \eta') (t-k-1)}
  \end{align}
  and $ |(H_t)^{1 - \tilde \beta} (H_{t-k-1})^{ 1 - \tilde \beta}| \geq 2^{\frac{1}{1+\beta}} e^{-2\eta'(t-k-1)}$.
  Thus, there exists $C_{2.1}'> 0$ such that for all $t>t_0$ and $k\in \mathbb Z_+$ satisfying $k\leq t-1$, almost surely
  \begin{align}
    K_{t,k}
    \leq C_{2.1}' e^{-(\eta - 3\eta')(t-k-1)}.
  \end{align}
  Now, by \cite[Lemma 2.13]{RenSongSunZhao2019Stable}, there exists $C''_{2.1}>0$ such that for all $t>t_0$ and $k\in \mathbb Z_+$ satisfying $k\leq t-1$,
  \begin{align}
    & \mathbb{\widetilde{P}}_{\mu} [|Y^s_{t,k}| ; \mathcal{D}_{t,t-k-1} ]
    \leq C_{2.1}' e^{\alpha (\tilde \beta - 1)t} \mathbb{\widetilde{P}}_{\mu} [ | \mathcal{I}_{t-k-1}^{t-k}X_t(f_\mathrm s)| ] e^{-(\eta - 3\eta')(t-k-1)} \\
    & \leq \frac{C_{2.1}' } {\mathbb{P}_{\mu}(D^c)} e^{ \alpha (\tilde \beta - 1)t} \|\mathcal{I}_{t-k-1}^{t-k} X_t(f_\mathrm s)\|_{\mathbb P_\mu; 1+\gamma} e^{-(\eta - 3\eta')(t-k - 1)} \\
    & \leq C_{2.1}'' e^{\alpha(\tilde \beta - \tilde \gamma)t} e^{ (\alpha \tilde \gamma - \kappa_f b)k} e^{-(\eta - 3\eta')(t-k)}
     \leq C_{2.1}'' e^{\alpha(\tilde \beta - \tilde \gamma)(t-k)} e^{-(\eta - 3\eta')(t-k)}.
  \end{align}
  In the last inequality, we used the fact that $\alpha \tilde \beta < \kappa_{f_\mathrm s} b$. Combining the fact that for all $t>t_0$ and $k\in \mathbb Z_+$ satisfying $k\leq t-1$, $\mathbb{\widetilde{P}}_{\mu} [|Y^s_{t,k}| ; \mathcal{D}^c_{t,t-k-1} ]\leq 2\mathbb{\widetilde{P}}_{\mu} [ \mathcal{D}^c_{t,t-k-1} ]\leq 2 C_{1,1}e^{-\delta_{1.1}(t-k-1)}$, we get the desired result.

 \emph{Step 3.} We show that there exist $C_{3.1},\delta_{3.1} > 0$ such that for all $t>t_0$ and $k\in \mathbb Z_+$ satisfying $k\leq t-1$, $ \mathbb{\widetilde{P}}_{\mu}[|Y^c_{t,k}|] \leq  C_{3.1} e^{-\delta_{3.1} (t-k)}.$

  Note that for $f\in \mathcal C_c$ and $k\in \mathbb Z_+$, we have $T_kf = e^{\alpha (\tilde \beta - 1 )k}P_k^\alpha f$. Therefore for all $t>t_0$ and $k\in \mathbb Z_+$ satisfying $k\leq t-1$,
  \[
    t^{\tilde \beta - 1} \Upsilon_{t-k-1}^{T_{k} \tilde f_\mathrm c}
    = \frac{X_{t-k}(T_{k} \tilde f_\mathrm c) - X_{t -k-1}(P_1^\alpha T_{k} \tilde f_\mathrm c)}{\|t X_{t-k-1}\|^{1-\tilde \beta}}
    = \frac{\mathcal I_{t - k - 1}^{t - k} X_t(f_\mathrm c)}{\|te^{\alpha (k+1)}X_{t-k-1} \|^{1 -\tilde \beta}}.
  \]
  The rest is similar to Step 2.
  We omit the details.

 \emph{Step 4.} We show that there exist $C_{4.1}, \delta_{4.1}>0$ such that for all $k \in \mathbb Z_+$ and $t\geq t_0$, we have $\widetilde {\mathbb {P}}_\mu[|Y^l_{t,k}|]\leq C_{4.1}e^{- \delta_{4.1} t}$.

 Note that for all $k \in \mathbb Z_+$ and $t\geq t_ 0$,
  \begin{align}
    & \Upsilon_{t+k}^{-T_k\tilde f_\mathrm l}
      = \frac{X_{t+k}(P^\alpha_1T_k\tilde f_\mathrm l) - X_{t+k+1}(T_k \tilde f_\mathrm l)}{\|X_{t+k}\|^{1 - \tilde \beta}}
    \\& = \sum_{p\in \mathcal N}
    \langle\tilde f_\mathrm l,\phi_p\rangle_\varphi e^{-(\alpha \tilde \beta - |pb|)k}\frac{ X_{t+k}(P_1^\alpha \phi_p) - X_{t+k+1}(\phi_p)}{\|X_{t+k}\|^{1 - \tilde \beta}}
    \\& = \sum_{p\in \mathcal N}
    \langle f_\mathrm l,\phi_p\rangle_\varphi  e^{(\alpha  -|p|b)t}\frac{H_{t+k}^p-H_{t+k+1}^p }{\|e^{-\alpha k}X_{t+k}\|^{1 - \tilde \beta}}.
  \end{align}
  Therefore for all $k\in \mathbb Z_+$ and $t\geq t_0$,
  \begin{align}
    &|Y^l_{t,k}| \mathbf 1_{\mathcal D_{t+k,t}}
      \leq \Big( \sum_{p\in \mathcal N}|\langle f_\mathrm l,\phi_p\rangle_\varphi|  e^{(\alpha  -|p|b)t} | H_{t+k}^p-H_{t+k+1}^p |\Big) \Big( \frac{1}{\|X_t\|^{1 - \tilde \beta}} - \frac{1}{\|e^{-\alpha k}X_{t+k}\|^{1 - \tilde \beta}} \Big)\mathbf 1_{\mathcal D_{t+k,t}}
    \\ &= \Big( \sum_{p\in \mathcal N}|\langle f_\mathrm l,\phi_p\rangle_\varphi|  e^{(\alpha  -|p|b)t} | H_{t+k}^p-H_{t+k+1}^p |\Big)e^{\alpha (\tilde \beta - 1)t} K'_{t,k}
    \\ &= \Big( \sum_{p\in \mathcal N}|\langle f_\mathrm l,\phi_p\rangle_\varphi|  e^{(\alpha \tilde \beta  -|p|b)t} | H_{t+k}^p-H_{t+k+1}^p |\Big) K'_{t,k},
  \end{align}
  where
  \[
    K'_{t,k}
        := \Big| \frac {(H_t)^{1- \tilde \beta} - (H_{t+k})^{1 - \tilde \beta}} {(H_t)^{1 - \tilde \beta} (H_{t+k})^{ 1- \tilde \beta }} \Big| \mathbf{1}_{\mathcal{D}_{t+k,t}}.
  \]
  Similar to Step 2, we can get that for all $k\in \mathbb Z_+$ and $t\geq t_0$, almost surely $K'_{t,k} \leq C_{4.1}'' e^{- (\eta - 3\eta')t}$.
  It follows from this and \cite[Lemma 3.3]{RenSongSunZhao2019Stable} that there exists $C'''_{4.1}>0$ such that for all $k\in \mathbb Z_+$ and $t\geq t_0$,
  \begin{align}
    & \widetilde{\mathbb P}_\mu[|Y^l_{t,k}|; \mathcal D_{t+k,t}]
      \leq \mathbb P_\mu(D)^{-1}\mathbb P_\mu[ |Y^l_{t,k}| ;\mathcal D_{t+k,t} ]
    \\ & \leq \mathbb P_{\mu}(D)^{-1} C_{4.1}'' e^{- (\eta - 3\eta') t}\sum_{p\in \mathcal {N}} |\langle f_\mathrm l,\phi_p\rangle_\varphi|  e^{(\alpha \tilde \beta  -|p|b)t} \mathbb P_\mu[| H_{t+k}^p-H_{t+k+1}^p |]
    \\ & \leq \mathbb P_{\mu}(D)^{-1} C_{4.1}'' e^{- (\eta - 3\eta') t}\sum_{p\in \mathcal {N}} |\langle f_\mathrm l,\phi_p\rangle_\varphi|  e^{(\alpha \tilde \beta  -|p|b)t} \| H_{t+k}^p-H_{t+k+1}^p \|_{\mathbb P_\mu; 1+\gamma}
    \\&\leq  \mathbb P_{\mu}(D)^{-1} C_{4.1}'' e^{- (\eta - 3\eta') t}\sum_{p\in \mathcal N} |\langle f_\mathrm l,\phi_p\rangle_\varphi|  e^{(\alpha \tilde \beta  -|p|b)t} e^{-(\alpha \tilde \gamma - |p|b)(t+k)} \\
    &  \leq  C_{4.1}''' e^{- (\eta - 3\eta') t} e^{(\alpha \tilde \beta - \alpha \tilde \gamma)t}.
  \end{align}
 Combining this with the fact that, for all $t>t_0$ and $k\in \mathbb Z_+$,  $\mathbb{\widetilde{P}}_{\mu} [|Y^l_{t,k}| ; \mathcal{D}^c_{t+k,t} ]\leq 2\mathbb{\widetilde{P}}_{\mu} [ \mathcal{D}^c_{t+k,t} ]\leq 2 C_{1,1}e^{-\delta_{1.1}t}$, we get the desired result.
\end{proof}

\begin{lem}\label{lem: U2T}
Under $\mathbb{\widetilde{P}}_{\mu}(\cdot)$,
$R_2(t)\xrightarrow[t\to \infty]{d}0$.
\end{lem}
\begin{proof}
 Define $\mathcal{E}_t:=\{\|X_t\|>t^{-1/2}e^{\alpha t}\}$. According to
 \cite[Proposition 2.10]{RenSongSunZhao2019Stable}, there exist $C_1, \delta_1>0$ such that
  \begin{align}
    \mathbb{\widetilde{P}}_{\mu}(\mathcal{E}^c_t)
    \leq \frac{1}{\mathbb{P}_{\mu}(D^c)}\mathbb{P}_{\mu}(0<e^{-\alpha t}\|X_t\|\leq t^{-1/2})\leq C_1(t^{-\delta_1 t}+e^{-\delta_1 t})
    , \quad t\geq t_0.
  \end{align}
  Therefore,
  \begin{align}
    \label{Theorem123}
    |\mathbb{\widetilde{P}}_{\mu}[e^{i
    \sum_{j=s,c,l}I^{f_j}_2(t)}-1;\mathcal{E}^c_t]|
    \leq 2\mathbb{\widetilde{P}}_{\mu}(\mathcal{E}^c_t)
    \leq 2C_1(t^{-\delta_1}+e^{-\delta_1 t}),
    \quad t\geq t_0.
  \end{align}
On the other hand,
\[
   |\mathbb{\widetilde{P}}_{\mu} [ (e^{i
      \sum_{j=s,c,l}I^{f_j}_2(t)}-1);\mathcal{E}_t]|
    \leq \sum_{j=s,c,l}\widetilde{\mathbb P}_{\mu}[|I^{f_j}_2(t)|\mathcal{E}_t],
\quad t\geq t_0.
\]
We claim  $\lim_{t\rightarrow \infty}\widetilde{\mathbb P}_{\mu}[|I^{f_j}_2(t)|;\mathcal{E}_t]=0$, $j=s,c,l$. Then combining this with \eqref{Theorem123}, we get $\lim_{t\rightarrow \infty}\mathbb{\widetilde{P}}_{\mu}[e^{i \sum_{j=s,c,l}I^{f_j}_2(t)}-1] = 0$.
Since $I_2^{f_j}(t)$, $j=s,c,l$, is linear in $f_j$, replacing $f_j$ with $\theta_j f_j$, $\theta_j \in \mathbb R$, we immediately get the desired result in this step.

   We now show the claim above in the following steps.
Fix a $\gamma \in (0,\beta)$ close enough to $\beta$ so that $\alpha(\tilde{\beta}-\tilde{\gamma})\leq \frac{1}{2}(1-\tilde{\beta})$ and $\alpha\tilde{\gamma}>|p|b$ for all $p\in \mathcal N$.

\emph{Step 1.} We show that $\lim_{t\rightarrow \infty}\widetilde{\mathbb P}_{\mu}[|I^{f_\mathrm s}_2(t)|;\mathcal{E}_t]=0$.

  Since $\kappa_{f_\mathrm s} b -\alpha \tilde \gamma > \alpha (\tilde \beta - \tilde \gamma)$, we can choose $\epsilon >0$ small enough so that $q:=\kappa_{f_\mathrm s}b- \alpha \tilde \gamma  > \alpha (\tilde \beta - \tilde \gamma) + \epsilon $.

	According to \cite[Lemma 2.13]{RenSongSunZhao2019Stable}, there exist $C_2',C_2''>0$ such that for each $t\geq t_0 >1$,
  \begin{align}
    & \mathbb{\widetilde{P}}_{\mu} [ |I^{f_\mathrm s}_2(t)|;\mathcal{E}_t] \\
    & \leq  ( t^{-1/2}e^{\alpha t} )^{\tilde \beta - 1}\Big(\sum_{k=\lfloor t-\ln t \rfloor+1}^{\lfloor t \rfloor - 1}\mathbb{\widetilde{P}}_{\mu} [| \mathcal{I}_{t-k-1}^{t-k} X_t(f_\mathrm s) |] + \mathbb{\widetilde{P}}_{\mu}[| \mathcal{I}_{0}^{t-\lfloor t\rfloor} X_t(f_\mathrm s)|]\Big) \\
    & \leq  ( t^{-1/2}e^{\alpha t} )^{\tilde \beta - 1}\Big(\sum_{k=\lfloor t-\ln t \rfloor+1}^{\lfloor t \rfloor - 1}\|\mathcal{I}_{t-k-1}^{t-k} X_t(f_\mathrm s) \|_{\mathbb P_\mu; 1+\gamma} + \|\mathcal I_0^{t-\lfloor t \rfloor} X_t(f_\mathrm s)\|_{\mathbb P_\mu;1+\gamma}\Big) \\
    & \leq C_2't^{\frac{1-\tilde{\beta}}{2}} e^{\alpha (\tilde \beta - \tilde \gamma)t}\sum_{k=\lfloor t-\ln t \rfloor+1}^{\lfloor t \rfloor}  e^{(\alpha\tilde \gamma-\kappa_{f_\mathrm s} b)k}
      \leq C_2''t^{\frac{1-\tilde{\beta}}{2}} e^{(q-\epsilon) t}\sum_{k=\lfloor t-\ln t \rfloor+1}^{\lfloor t \rfloor}  e^{-q k}
    \\ & \leq C_2'''t^{\frac{1-\tilde{\beta}}{2}} e^{q(t - \lfloor t - \ln t\rfloor-1)}e^{-\epsilon t}
         \leq C_2'''t^{\frac{1-\tilde{\beta}}{2}} t^q e^{- \epsilon t}.
  \end{align}
The desired result in this step now follows immediately.

\emph{Step 2.} We show that $\lim_{t\rightarrow \infty}\widetilde{\mathbb P}_{\mu}[|I^{f_\mathrm c}_2(t)|;\mathcal{E}_t]=0$.

It follows from \cite[Lemma 2.13]{RenSongSunZhao2019Stable} that there exist $C_3',C_3''>0$ such that for each $t\geq t_0$,
  \begin{align}
     &\mathbb{\widetilde{P}}_{\mu} [ |I^{f_\mathrm c}_2(t);{\mathcal{E}_t}] \\
    & \leq  (t^{\frac{1}{2}} e^{\alpha t} )^{\tilde \beta - 1}\Big(\sum_{k=\lfloor t-\ln t \rfloor+1}^{\lfloor t \rfloor - 1}\mathbb{\widetilde{P}}_{\mu} [| \mathcal{I}_{t-k-1}^{t-k} X_t(f_\mathrm c) |] + \mathbb{\widetilde{P}}_{\mu}[| \mathcal{I}_{0}^{t-\lfloor t\rfloor} X_t(f_\mathrm c)|]\Big) \\
    & \leq C_3' t^{\frac{1}{2}(\tilde \beta - 1)} e^{\alpha(\tilde \beta - 1)t}\Big(\sum_{k=\lfloor t-\ln t \rfloor+1}^{\lfloor t \rfloor - 1}\|\mathcal{I}_{t-k-1}^{t-k} X_t(f_\mathrm c) \|_{\mathbb P_\mu; 1+\gamma} + \|\mathcal I_0^{t-\lfloor t \rfloor} X_t(f_\mathrm c)\|_{\mathbb P_\mu;1+\gamma}\Big) \\
    & \leq C_3' t^{\frac{1}{2}(\tilde \beta - 1)} e^{\alpha (\tilde \beta - \tilde \gamma)t}\sum_{k=\lfloor t-\ln t \rfloor+1}^{\lfloor t \rfloor}  e^{(\alpha\tilde \gamma-\kappa_{f_\mathrm c} b)k}
      = C_3' t^{\frac{1}{2}(\tilde \beta - 1)} e^{\alpha(\tilde \beta - \tilde \gamma) t}\sum_{k=\lfloor t-\ln t \rfloor+1}^{\lfloor t \rfloor}  e^{-\alpha (\tilde \beta -\tilde \gamma) k}
    \\ & \leq C_3'' t^{\frac{1}{2}(\tilde \beta - 1)} e^{\alpha (\tilde \beta - \tilde \gamma)(t - \lfloor t - \ln t\rfloor-1)}
         \leq C_3'' t^{\frac{1}{2}(\tilde \beta - 1)} t^{\alpha (\tilde \beta - \tilde \gamma)}.
  \end{align}
  From this, we immediately get the desired result in this step.

\emph{Step 3.} We will show that $\lim_{t\rightarrow \infty}\widetilde{\mathbb P}_{\mu}[|I^{f_\mathrm l}_2(t)|;\mathcal{E}_t]=0$.


 It follows from \cite[Lemma 3.3]{RenSongSunZhao2019Stable} that there exist $C_4',C_4''>0$ and $\delta_3'>0$ such that
  \begin{align}
    &\widetilde {\mathbb {P}}_\mu[ | I_{2}^{f_\mathrm l}(t)|; \mathcal {E}_t]
      \leq \sum_{k = \lfloor t^2\rfloor+1}^\infty \widetilde {\mathbb {P}}_\mu[ |\widetilde {\Upsilon}_{t,k}|; \mathcal {E}_t]
    \\ & \leq \mathbb P_\mu(D^c)^{-1} \sum_{k = \lfloor t^2\rfloor+1}^\infty \sum_{p \in \mathcal N} |\langle f_\mathrm l,\phi_p\rangle_\varphi| e^{(\alpha - |p|b)t}\mathbb {P}_\mu\Big[\frac{ |H_{t+k}^p - H_{t+k+1}^p|}{\|X_t\|^{1- \tilde \beta}}; \mathcal E_t\Big]
    \\ & \leq \mathbb P_\mu(D^c)^{-1} (t^{-1/2}e^{\alpha t})^{\tilde{\beta}-1} \sum_{k = \lfloor t^2\rfloor+1}^\infty \sum_{p \in \mathcal N} |\langle f_\mathrm l,\phi_p\rangle_\varphi| e^{(\alpha - |p|b)t}\|H_{t+k}^p - H_{t+k+1}^p\|_{\mathbb P_\mu; 1+\gamma}
    \\ & \leq C_4' (t^{-1/2}e^{\alpha t})^{\tilde{\beta}-1} \sum_{k = \lfloor t^2\rfloor+1}^\infty \sum_{p \in \mathcal N} |\langle f_\mathrm l,\phi_p\rangle_\varphi| e^{(\alpha - |p|b)t} e^{- (\alpha \tilde \gamma - |p|b)(t+k)}
    \\ & = C_4''t^{\frac{1-\tilde{\beta}}{2}} e^{ \alpha (\tilde \beta - \tilde \gamma) t }e^{- \delta'_3 t^2},
  \end{align}
 which completes the proof of this step.
\end{proof}

\begin{lem}\label{lem: U3T}
 $R_3(t) \xrightarrow[t\to \infty]{\widetilde {\mathbb P}_\mu \text{-} a.s.} 0$..
\end{lem}
\begin{proof}
\emph{Step 1.} We claim that $I^{f_\mathrm s}_3(t) \xrightarrow[t\to \infty]{\widetilde {\mathbb P}_\mu \text{-} a.s.} 0$.
  In fact, if for all $\kappa \in \mathbb Z_+$ and $f\in \mathcal P$, we define
\begin{equation}
\label{eq:Q}
  	Q_\kappa f
  	:= \sup_{t\geq 0} e^{\kappa b t}|P_t f|,
  	\qquad  Q f
  	:= Q_{\kappa_f}f.
\end{equation}
	Then according to \cite[Fact 1.2]{MarksMilos2018CLT}, $Q$ is an operator from $\mathcal P$ to $\mathcal P$. Thus,  we have
  \begin{align}
    & |I^{f_\mathrm s}_3(t)|
      \leq \frac{X_0(|P^\alpha_tf_\mathrm s|)}{\|X_t\|^{1 - \tilde \beta }}
      \leq \frac{e^{\alpha t - \kappa_{f_\mathrm s} b t}X_0(Qf_\mathrm s)}{(e^{\alpha t} H_t)^{1 - \tilde \beta}}
      = e^{(\alpha \tilde \beta - k_{f_\mathrm s}b)t} H_t^{\tilde \beta - 1} X_0(Qf_\mathrm s)
      \xrightarrow[t\to \infty]{\widetilde {\mathbb P}_\mu \text{-} a.s.} 0.
  \end{align}

  \emph{Step 2.} Similar to Step 1, we can show that $I^{f_\mathrm c}_3(t) \xrightarrow[t\to \infty]{\widetilde {\mathbb P}_\mu \text{-} a.s.} 0$.
  We omit the details.
Since $I_3^{f_\mathrm l}(t)=0$, the desired result follows immediately.
\end{proof}

\begin{proof}[Proof of Theorem Theorem \ref{thm:M}]
 We first recall some facts about weak convergence which will be used later. For $f:\mathbb R^d\mapsto \mathbb R$, let
 $$
 \|f\|_L:=\sup_{x\neq y}\frac{|f(x)-f(y)|}{|x-y|}
 $$
 and $\|f\|_{BL}:= \|f\|_{\infty}+\|f\|_L. $
 For any {\color{red}probability distributions} $\mu_1$ and $\mu_2$ on $\mathbb R^d$, define
\[
  d(\mu_1,\mu_2):=\sup\Big\{\Big|\int fd\mu_1-\int f d\mu_2\Big|:\|f\|_{BL}\leq 1\Big\}.
\]
Then $d$ is a metric. It follows from \cite[Theorem 11.3.3]{Dudley2002} that the topology generated by $d$ is equivalent to the weak convergence topology. 

Using the definition, we can easily see that, if $\mu_1$ and $\mu_2$ are the distributions of two $\mathbb R^d $-valued random variables $X$ and $Y$ respectively, {\color{red} defined in same probability space} then
\begin{align}\label{ineq: distribution control}
  d(\mu_1,\mu_2) \leq \mathbb E|X-Y|.
\end{align}
Recall that $S(t)$ is defined in {\color{blue}the display before Theorem \ref{thm:M}.}
Now for all $f_\mathrm s\in \mathcal C_s $, $f_\mathrm c \in \mathcal C_c$, $f_\mathrm l \in \mathcal C_l$ and
$r,t>0$, let

\begin{align}
  S(t,r):=\Big(e^{-\alpha t}\|X_t\|,& \|X_{t+r}\|^{\tilde{\beta}-1}X_{t+r}(f_\mathrm s), \|(t+r)X_{t+r}\|^{\tilde{\beta}-1}X_{t+r}(f_\mathrm c),\\
& \|X_{t+r}\|^{\tilde{\beta}-1}(X_{t+r}(f_\mathrm l)-\sum_{p\in \mathbb Z_+^d:\alpha\tilde{\beta}>|p|b}\langle f_\mathrm l,\phi_p\rangle_{\varphi}e^{(\alpha-|p|b)(t+r)}H_{\infty}^p )\Big),
\end{align}
and
\begin{align}
\widetilde{S}(t,r)= \Big(e^{-\alpha (t+r)}\|X_{t+r}\|-e^{-\alpha t}\|X_t\|,0,0,0\Big).
\end{align}
Then $S(t+r)=S(t,r)+\widetilde{S}(t,r)$.

\emph{Step 1.} For each $\mu\in \mathcal M_c(\mathbb R^d)$, $f_\mathrm s\in \mathcal C_s$,$f_\mathrm c\in \mathcal C_c$ and $f_\mathrm l\in \mathcal C_l$, under $\mathbb P_{\mu}(\cdot|D^c)$, we have
\[
S(t,r)\xrightarrow[r\rightarrow \infty]{d}\widetilde{S}(t):=\Big(e^{-\alpha t}\|X_t\|,\zeta^{f_\mathrm s},\zeta^{f_\mathrm c},\zeta^{-f_\mathrm l}\Big),
\]
where $\zeta^{f_\mathrm s},\zeta^{f_\mathrm c}$ and $\zeta^{-f_\mathrm l}$ are independent $(1+\beta)$-stable random variables with characteristic functions described by \eqref{eq: charac func11}. Moreover,  $\zeta^{f_\mathrm s},\zeta^{f_\mathrm c}$ and $\zeta^{-f_\mathrm l}$ are independent of $e^{-\alpha t}\|X_t\|$.

For each $\theta,\theta_s,\theta_c,\theta_l\in \mathbb R$ and $r,t>0$, denote the characteristic function of  $S(t,r)$ under $\mathbb P_{\mu}(\cdot|D^c)$ by $k(\theta,\theta_s,\theta_c,\theta_l,r,t)$, {\color{red}then by \eqref{eq:Hinfty} and bounded convergence theorem} we have,
\begin{align}
   &k(\theta,\theta_s,\theta_c,\theta_l,r,t)\
   =\widetilde{\mathbb P}_{\mu}\Big[\exp\Big( i\theta e^{-\alpha t}\|X_t\|+A(\theta_s,\theta_c,\theta_l,r,t,\infty)\Big)\Big]\\
  &=\lim_{u\rightarrow \infty}\frac{1}{\mathbb P_{\mu}(D^c)}\mathbb P_{\mu}\Big[\exp\Big( i\theta e^{-\alpha t}\|X_t\|+A(\theta_s,\theta_c,\theta_l,r,t,u)\Big);D^c\Big],
\end{align}
where
\begin{align}
 &A(\theta_s,\theta_c,\theta_l,r,t,u)=i\theta_s\|X_{t+r}\|^{\tilde{\beta}-1}X_{t+r}(f_\mathrm s)+ i\theta_c \|(t+r)X_{t+r}\|^{\tilde{\beta}-1}X_{t+r}(f_\mathrm c)\\
&+i\theta_l \|X_{t+r}\|^{\tilde{\beta}-1}\Big(X_{t+r}(f_\mathrm l)-\sum_{p\in \mathbb Z_+^d:\alpha\tilde{\beta}>|p|b}\langle f_\mathrm l,\phi_p\rangle_{\varphi}e^{(\alpha-|p|b)(t+r)}H_{u}^p\Big),\quad r+t<u\leq \infty.
\end{align}
Note that the non-extinction event $D^c=\{\|X_r\|>0, r\geq 0\}=D_{\leq t}\cup D_{\geq t}$, where $D_{\leq t}=\{\|X_r\|>0, 0\leq r\leq t\}$ and $D_{\geq t}=\{\|X_r\|>0, r\geq t\}$. 
{\color{blue} \tt Maybe we want to mention $\mathbf 1_{D^c} = \mathbf 1_{D\leq t} \mathbf 1_{D\geq t}$ here?}
By the Markov property and the dominated convergence theorem, we get
\begin{align}
  & k(\theta,\theta_s,\theta_c,\theta_l,r,t)\\
&=\lim_{u\rightarrow \infty}\frac{1}{\mathbb P_{\mu}(D^c)}\mathbb P_{\mu}\Big[\exp\Big(i\theta e^{-\alpha t}\|X_t\|\Big)\mathbf 1_{D_{\leq t}} \mathbb P_{\mu}\Big[\exp\Big(A(\theta_s,\theta_c,\theta_l,r,t,u)\Big)\mathbf 1_{D_{\geq t}}\Big|\mathscr F_t\Big]\Big]\\
&=\lim_{u\rightarrow\infty}\frac{1}{\mathbb P_{\mu}(D^c)}\mathbb P_{\mu}\Big[\exp\Big( i\theta e^{-\alpha t}\|X_t\|\Big)\mathbf 1_{D_{\leq t}}\mathbb P_{X_t}\Big[\exp\Big(A(\theta_s,\theta_c,\theta_l,r,0,u-t)\Big)\mathbf 1_{D^c}\Big]\Big]\\
&=\frac{1}{\mathbb P_{\mu}(D^c)}\mathbb P_{\mu}\Big[\exp\Big( i\theta e^{-\alpha t}\|X_t\|\Big)\mathbf 1_{D_{\leq t}}\mathbb P_{X_t}(D^c)\widetilde{\mathbb P}_{X_t}\Big[\exp\Big(A(\theta_s,\theta_c,\theta_l,r,0,\infty)\Big)\Big]\Big]
\end{align}
It follows from Theorem \ref{thm: II} and the {\color{red}bounded convergence theorem,}
\begin{align}
   k(\theta,\theta_s, \theta_c,\theta_l,r,t)& \xrightarrow{r\to \infty}\frac{1}{\mathbb P_{\mu}(D^c)}\mathbb P_{\mu}\Big[\exp\Big(i\theta e^{-\alpha t}\|X_t\|\Big)\mathbf 1_{D_{\leq t}}\mathbb P_{X_t}(D^c)\Big]
   \prod_{j=s,c,l}\exp\Big(m[\theta_j f_j]\Big)\\
&=\widetilde{\mathbb P}_{\mu}\Big[\exp\Big(i\theta e^{-\alpha t}\|X_t\|\Big)\Big]
\prod_{j=s,c,l}\exp\Big(m[\theta_j f_j]\Big).
\end{align}
The desired result of this step now follows immediately. {\color{blue} \tt Maybe we can use linearity to simplify the argument here?}

\emph{Step 2.} For all $r,t>0$, let $\mathcal D(r)$ and $\mathcal D(r,t)$ be the distributions of $S(r)$ and $S(t,r)$ under $\mathbb P_{\mu}(\cdot|D^c)$, let $ \widetilde{\mathcal D}(t)$ and $\mathcal D$ be the distributions of {\color{red}$\widetilde{S}(t)$} and $(H^*,\zeta^{f_\mathrm s},\zeta^{f_\mathrm c},\zeta^{-f_\mathrm l})$ under $\mathbb P_{\mu}(\cdot|D^c)$. Then using \eqref{ineq: distribution control} and \cite[Lemma 3.3]{RenSongSunZhao2019Stable}, we have that for each $\gamma\in (0,\beta)$, there exist constant $C>0$ such that,
\begin{align}
 \varlimsup_{r\rightarrow \infty}d(\mathcal D(t+r),\mathcal D)&\leq \varlimsup_{r\rightarrow \infty}[d(\mathcal D(t+r),\mathcal D(t,r))+d(\mathcal D(t,r),\widetilde{\mathcal D}(t))+d(\widetilde{\mathcal D}(t),\mathcal D)]\\
&\leq \varlimsup_{r\rightarrow \infty}\widetilde{\mathbb P}_{\mu}[|e^{-\alpha t}\|X_t\|-e^{-\alpha (t+r)}\|X_{t+r}\||]+0+\widetilde{\mathbb P}_{\mu}[|H_t-H_{\infty}|]\\
&\leq \varlimsup_{r\rightarrow \infty}\mathbb P_{\mu}(D^c)^{-1}(\|H_t-H_{t+r}\|_{\mathbb P_{\mu};1+\gamma}+\|H_t-H_{\infty}\|_{\mathbb P_{\mu};1+\gamma})\\
&\leq Ce^{-\frac{\gamma}{1+\gamma}t}.
\end{align}
Therefore, using the definition of $\varlimsup_{s\rightarrow \infty}$, we know that,
\begin{align}
 &\varlimsup_{r\rightarrow \infty}d(\mathcal D(r),\mathcal D) 
 = {\color{red} \varlimsup_{t\to \infty}}\varlimsup_{r\rightarrow \infty}d(\mathcal D(t+r),\mathcal D)
\leq \varlimsup_{t\rightarrow \infty}Ce^{-\frac{\gamma}{1+\gamma}t} = 0.
\end{align}
The desired result now follows immediately.
\end{proof}
{\tt\color{red} Maybe we should cleanup the references?}

\begin{thebibliography}{10}

\bibitem{AdamczakMilos2015CLT}
  R. Adamczak and P. Mi{\l}o\'{s}, \emph{C{LT} for {O}rnstein-{U}hlenbeck branching particle system},
  Electron. J. Probab. \textbf{20} (2015), no. 42, 35 pp.

\bibitem{Asmussen76Convergence}
  S. Asmussen, \emph{Convergence rates for branching processes},
  Ann. Probab.  \textbf{4} (1976), no. 1, 139--146.

\bibitem{AsmussenHering1983Branching}
  S. Asmussen and H. Hering, \emph{Branching processes},
  Progress in Probability and Statistics, 3. Birkh\"{a}user Boston, Inc., Boston, MA, 1983.

\bibitem{Athreya1969Limit}
  K. B. Athreya,
  \emph{Limit theorems for multitype continuous time {M}arkov branching processes. {I}. {T}he case of an eigenvector linear functional},
  Z. Wahrscheinlichkeitstheorie und Verw. Gebiete \textbf{12} (1969), 320--332.

\bibitem{Athreya1969LimitB}
  K. B. Athreya,
  \emph{Limit theorems for multitype continuous time {M}arkov branching processes. {II}. {T}he case of an arbitrary linear functional},
  Z. Wahrscheinlichkeitstheorie und Verw. Gebiete \textbf{13} (1969), 204--214.

\bibitem{Athreya1971Some}
  K. B. Athreya,
  \emph{Some refinements in the theory of supercritical multitype {M}arkov branching processes},
  Z. Wahrscheinlichkeitstheorie und Verw. Gebiete \textbf{20} (1971), 47--57.


\bibitem{Bertoin}
J. Bertoin,
\emph{L\'evy processes}.
Cambridge Tracts in Mathematics, 121. Cambridge
University Press, 1996.

\bibitem{BRY}
J. Bertoin, B. Roynette and M. Yor,
\emph{Some connections between (sub)critical branching mechanisms and Bernstein functions}.
arXiv:0412322.


\bibitem{ChenRenWang2008An-almost}
  Z.-Q. Chen, Y.-X. Ren, and H. Wang,
  \emph{An almost sure scaling limit theorem for {D}awson-{W}atanabe superprocesses},
  J. Funct. Anal. \textbf{254} (2008), no. 7, 1988--2019.

 \bibitem{ChenRenSongZhang2015Strong-law}
   Z.-Q. Chen, Y.-X. Ren, R. Song, and R. Zhang,
   \emph{Strong law of large numbers for supercritical superprocesses under second moment condition},
   Front. Math. China \textbf{10} (2015), no. 4, 807--838.

 \bibitem{ChenRenYang2019Skeleton}
   Z.-Q. Chen, Y.-X. Ren, and T. Yang,
   \emph{Skeleton decomposition and law of large numbers for supercritical superprocesses},
   Acta Appl. Math. \textbf{159} (2019), 225--285.

\bibitem{Dudley2002}
  R. M. Dudley
\emph{Real Analysis and Probability},
  Cambridge University Press, 2002.

 \bibitem{Durrett2010Probability}
   R. Durrett,
   \emph{Probability: theory and examples},
   Fourth edition. Cambridge Series in Statistical and Probabilistic Mathematics, 31. Cambridge University Press, Cambridge, 2010.

\bibitem{Dynkin1993Superprocesses}
  E. B. Dynkin,
  \emph{Superprocesses and partial differential equations},
  Ann. Probab. \textbf{21} (1993), no. 3, 1185--1262.

\bibitem{EckhoffKyprianouWinkel2015Spines}
  M. Eckhoff, A. E. Kyprianou, and M. Winkel,
  \emph{Spines, skeletons and the strong law of large numbers for superdiffusions},
  Ann. Probab. \textbf{43} (2015), no. 5, 2545--2610.

\bibitem{Englander2009Law}
  J. Engl\"{a}nder,
  \emph{Law of large numbers for superdiffusions: the non-ergodic case},
  Ann. Inst. Henri Poincar\'{e} Probab. Stat. \textbf{45} (2009), no. 1, 1--6.

\bibitem{EnglanderWinter2006Law}
  J. Engl\"{a}nder and A. Winter,
  \emph{Law of large numbers for a class of superdiffusions},
  Ann. Inst. H. Poincar\'{e} Probab. Statist. \textbf{42} (2006), no. 2, 171--185.

\bibitem{EnglanderTuraev2002A-scaling}
  J. Engl\"{a}nder and  D. Turaev,
  \emph{A scaling limit theorem for a class of superdiffusions},
  Ann. Probab. \textbf{30} (2002), no. 2, 683--722.

\bibitem{Heyde1970A-rate}
  C. C. Heyde,
  \emph{A rate of convergence result for the super-critical {G}alton-{W}atson process},
  J. Appl. Probability \textbf{7} (1970), 451--454.

\bibitem{Heyde1971Some}
  C. C. Heyde,
    \emph{some central limit analogues for supercritical {G}alton-{W}atson processes},
  J. Appl. Probability \textbf{8} (1971), 52--59.

\bibitem{HeydeBrown1871An-invariance}
  C. C. Heyde and B. M. Brown,
  \emph{An invariance principle and some convergence rate results for branching processes},
  Z. Wahrscheinlichkeitstheorie und Verw. Gebiete, \textbf{20} (1971), 271--278.

\bibitem{HeydeLeslie1971Improved}
  C. C. Heyde and J. R. Leslie,
  \emph{Improved classical limit analogues for {G}alton-{W}atson processes with or without immigration},
  Bull. Austral. Math. Soc. \textbf{5} (1971), 145--155.

\bibitem{IksanovKoleskoMeiners2018Stable-like}
  A. Iksanov, K. Kolesko, and M. Meiners,
  \emph{Stable-like fluctuations of {B}gins' martingales},
  Stochastic Process. Appl. (2018).

\bibitem{Janson2004Functional}
  S. Janson,
  \emph{Functional limit theorems for multitype branching processes and generalized {P}\'{o}lya urns},
  Stochastic Process. Appl. \textbf{110} (2004), no. 2, 177--245.

\bibitem{KestenStigum1966Additional}
  H. Kesten and B. P. Stigum,
  \emph{Additional limit theorems for indecomposable multidimensional {G}alton-{W}atson processes},
  Ann. Math. Statist. \textbf{37} (1966), 1463--1481.

\bibitem{KestenStigum1966A-limit}
  H. Kesten and B. P. Stigum,
  \emph{A limit theorem for multidimensional {G}alton-{W}atson processes},
  Ann. Math. Statist. \textbf{37} (1966), 1211--1223.

\bibitem{KouritzinRen2014A-strong}
  M. A. Kouritzin, and Y.-X. Ren,
  \emph{A strong law of large numbers for super-stable processes},
  Stochastic Process. Appl. \textbf{121} (2014), no. 1, 505--521.

\bibitem{Kyprianou2014Fluctuations}
  A. E. Kyprianou,
  \emph{Fluctuations of {L}\'{e}vy processes with applications},
    Introductory lectures. Second edition. Universitext. Springer, Heidelberg, 2014.

\bibitem{Li2011Measure-valued}
  Z. Li,
  \emph{Measure-valued branching {M}arkov processes},
  Probability and its Applications (New York). Springer, Heidelberg, 2011.

\bibitem{LiuRenSong2009Llog}
  R.-L. Liu, Y.-X. Ren, and R. Song,
  \emph{{$L\log L$} criterion for a class of superdiffusions},
  J. Appl. Probab. \textbf{46} (2009), no. 2, 479--496.

\bibitem{LiuRenSong2013Strong}
  R.-L. Liu, Y.-X. Ren, and R. Song,
  \emph{Strong law of large numbers for a class of superdiffusions},
  Acta Appl. Math. \textbf{123} (2013), 73--97.

\bibitem{MarksMilos2018CLT}
  R. Marks and P. Mi{\l}o{\'s},
  \emph{C{LT} for supercritical branching processes with heavy-tailed branching law},
  arXiv:1803.05491.

\bibitem{MetafunePallaraPriola2002Spectrum}
  G. Metafune, D. Pallara, and E. Priola,
  \emph{Spectrum of {O}rnstein-{U}hlenbeck operators in {$L^p$} spaces with respect to invariant  measures},
  J. Funct. Anal. \textbf{196} (2002), no. 1, 40--60.

\bibitem{Milos2012Spatial}
  P. Mi{\l}o{\'s},
  \emph{Spatial central limit theorem for supercritical superprocesses},
  J. Theoret. Probab. \textbf{31} (2018), no. 1, 1--40.

\bibitem{RenSongSun2017Spine}
  Y.-X. Ren, R. Song, and Z. Sun,
  \emph{Spine decompositions and limit theorems for a class of critical superprocesses},
  Acta Appl. Math. (2019).

\bibitem{RenSongSun2018Limit}
  Y.-X. Ren, R. Song, and Z. Sun,
  \emph{Limit theorems for a class of critical superprocesses with stable branching},
  arXiv:1807.02837.

\bibitem{RenSongSunZhao2019Stable}
Y.-X. Ren, R. Song, Z. Sun and J. Zhao,
\emph{Stable central limit theorems for super Ornstein-Uhlenbeck processes},
Elect. J Probab., \textbf{24} (2019), no. 141, 1--42.

\bibitem{RenSongZhang2014Central}
  Y.-X. Ren, R. Song, and R. Zhang,
  \emph{Central limit theorems for super {O}rnstein-{U}hlenbeck processes},
  Acta Appl. Math. \textbf{130} (2014), 9--49.

\bibitem{RenSongZhang2014CentralB}
  Y.-X. Ren, R. Song, and R. Zhang,
  \emph{Central limit theorems for supercritical branching {M}arkov processes},
  J. Funct. Anal. \textbf{266} (2014), no. 3, 1716--1756.

\bibitem{RenSongZhang2015Central}
  Y.-X. Ren, R. Song, and R. Zhang,
  \emph{Central limit theorems for supercritical superprocesses},
  Stochastic Process. Appl. \textbf{125} (2015), no. 2, 428--457.

\bibitem{RenSongZhang2017Central}
  Y.-X. Ren, R. Song, and R. Zhang,
  \emph{Central limit theorems for supercritical branching nonsymmetric {M}arkov processes},
  Ann. Probab. \textbf{45} (2017), no. 1, 564--623.

\bibitem{RenSongZhang2017Functional}
  Y.-X. Ren, R. Song, and R. Zhang,
  \emph{Functional central limit theorems for supercritical superprocesses},
  Acta Appl. Math. \textbf{147} (2017), 137--175.

\bibitem{Sato2013Levy}
  K. Sato,
  \emph{L{\'e}vy processes and infinitely divisible distributions},
  Translated from the 1990 Japanese original. Revised by the author. Cambridge Studies in Advanced Mathematics, 68. Cambridge University Press, Cambridge, 1999.

\bibitem{SchillingSongVondravcek2010Bernstein}
  R. L. Schilling, R. Song, and Z. Vondra\v{c}ek,
  \emph{Bernstein functions.}
  Theory and applications. Second edition. De Gruyter Studies in Mathematics, 37. Walter de Gruyter \& Co., Berlin, 2012.

\bibitem{SteinShakarchi2003Complex}
  E. M. Stein and R. Shakarchi, \emph{Complex analysis},
  Princeton Lectures in Analysis, 2. Princeton University Press, Princeton, NJ, 2003.

\bibitem{Wang2010An-almost}
  L. Wang, \emph{An almost sure limit theorem for super-{B}rownian motion},
  J. Theoret. Probab. \textbf{23} (2010), no. 2, 401--416.

\end{thebibliography}
\end{document}
