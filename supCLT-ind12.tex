%supCLT-ind1.tex ?/?/? by Jianjie
%supCLT-ind2.tex ?/?/? by ?
%supCLT-ind3.tex 03/05/2020 by Renming
%supCLT-ind4.tex 11/05/2020 by Zhenyao
%supCLT-ind5.tex 14/05/2020 by Yanxia
%supCLT-ind6.tex 16/05/2020 by JianJie
%supCLT-ind7.tex 17/05/2020 by Renming
%supCLT-ind8.tex 20/05/2020 by Zhenyao
%supCLT-ind9.tex 21/05/2020 by Yanxia
%%supCLT-ind10.tex 22/05/2020 by Jianjie
%%supCLT-ind11.tex 22/05/2020 by Renming
%%supCLT-ind12.tex 23/05/2020 by Zhenyao
\documentclass[12pt,a4paper]{amsart}
\setlength{\textwidth}{\paperwidth}
\addtolength{\textwidth}{-2in}
\calclayout
\usepackage[utf8]{inputenc}
\usepackage[T1]{fontenc}
\usepackage{mathtools}
\mathtoolsset{showonlyrefs}
\usepackage{mathrsfs}
%\usepackage[backref]{hyperref}
\usepackage{hyperref}
%\usepackage[inline]{showlabels}
\usepackage{comment}
%\usepackage{xcolor}
\usepackage{amsthm}
\theoremstyle{plain}
\newtheorem{thm}{Theorem}[section]
\newtheorem{lem}[thm]{Lemma}
\newtheorem{prop}[thm]{Proposition}
\newtheorem{cor}[thm]{Corollary}
\newtheorem{conj}[thm]{Conjecture}
\theoremstyle{definition}
\newtheorem{defi}[thm]{Definition}
\newtheorem{rem}[thm]{Remark}
\newtheorem{exa}[thm]{Example}
\newtheorem{asp}{Assumption}
\numberwithin{equation}{section}
\allowdisplaybreaks
\begin{document}
\title
[stable CLT for super-OU processes]
{Stable Central Limit Theorems for Super Ornstein-Uhlenbeck Processes, II}
\author
[Y.-X. Ren, R. Song, Z. Sun and J. Zhao]
{Yan-Xia Ren, Renming Song, Zhenyao Sun and Jianjie Zhao}
\address{
  Yan-Xia Ren \\
  LMAM School of Mathematical Sciences \& Center for Statistical Science \\
  Peking University \\
  Beijing 100871, P. R. China}
\email{yxren@math.pku.edu.cn}
\thanks{The research of Yan-Xia Ren is supported in part by NSFC (Grant Nos. 11671017  and 11731009) and LMEQF}
\address{
  Renming Song \\
  Department of Mathematics \\
  University of Illinois at Urbana-Champaign \\
  Urbana, IL 61801, USA}
\email{rsong@illinois.edu}
\thanks{The Research of Renming Song is support in part by a grant from the Simons Foundation (\#429343, Renming Song)}
\address{
  Zhenyao Sun \\
  Faculty of Industrial Engineering and Management\\
  Technion, Israel Institute of Technology \\
  Haifa 3200003, Israel}
\email{zhenyao.sun@gmail.com}
\address{
  Jianjie Zhao \\
  School of Mathematical Sciences \\
  Peking University \\
  Beijing 100871, P. R. China}
\email{zhaojianjie@pku.edu.cn}

\begin{abstract}
This paper is a continuation of our recent paper (Elect. J. Probab. \textbf{24} (2019), no. 141)
and is devoted to the  asymptotic behavior of a class of supercritical super Ornstein-Uhlenbeck processes $(X_t)_{t\geq 0}$ with branching mechanisms of infinite second moment.
In the aforementioned paper, we proved stable central limit theorems for  $X_t(f) $ for {\it some} functions $f$ of polynomial growth
in three different regimes. However, we were not able to prove
central limit theorems for
$X_t(f) $ for {\it all} functions $f$ of polynomial growth.
In this note, we show that the limit stable random variables in the three different regimes are independent, and as a consequence, we get stable central limit theorems for  $X_t(f) $ for {\it all} functions $f$ of polynomial growth.
\end{abstract}
\subjclass[2010]{60J68, 60F05}
\keywords{Superprocesses, Ornstein-Uhlenbeck processes, Stable distribution, Central limit theorem, Law of large numbers, Branching rate regime}
\maketitle

\section{Introduction and main result}
\label{subsec:M}
Let $d \in \mathbb N:= \{1,2,\dots\}$ and $\mathbb R_+:= [0,\infty)$.
Let $\xi=\{(\xi_t)_{t\geq 0}; (\Pi_x)_{x\in \mathbb R^d}\}$ be an $\mathbb R^d$-valued Ornstein-Uhlenbeck process (OU process) with generator
\begin{align}
  Lf(x)
  = \frac{1}{2}\sigma^2\Delta f(x)-b x \cdot \nabla f(x)
  , \quad  x\in \mathbb R^d, f \in C^2(\mathbb R^d),
\end{align}
where $\sigma > 0$ and $b > 0$ are constants.
Let $\psi$ be a function on $\mathbb R_+$ of the form
\begin{align}
  \psi(z)
  =- \alpha z + \rho z^2 + \int_{(0,\infty)} (e^{-zy} - 1 + zy) \pi(\mathrm dy)
  , \quad  z \in \mathbb R_+,
\end{align}
where $\alpha > 0 $, $\rho \geq0$ and $\pi$ is a measure on $(0,\infty)$ with $\int_{(0,\infty)}(y\wedge y^2)\pi(\mathrm dy)< \infty$.
$\psi$ is referred to as a branching mechanism and $\pi$ is referred to as the L\'evy measure of $\psi$.
Denote by $\mathcal M(\mathbb R^d)$
($\mathcal M_c(\mathbb R^d)$)
the space of all finite Borel measures (of compact support) on $\mathbb R^d$.
Denote by $\mathcal B(\mathbb R^d, \mathbb R)$
($\mathcal B(\mathbb R^d, \mathbb R_+)$)
the space of all $\mathbb R$-valued
($\mathbb R_+$-valued)
Borel functions on $\mathbb R^d$.
For $f,g\in \mathcal B(\mathbb R^d, \mathbb R)$ and $\mu \in \mathcal M(\mathbb R^d)$,
 write $\mu(f)= \int f(x)\mu(\mathrm dx)$
and $\langle f, g\rangle = \int f(x)g(x) \mathrm dx$ whenever the integrals make sense.
We say a real-valued Borel function $f$
on $\mathbb R_+\times \mathbb R^d$ is \emph{locally bounded} if, for each $t\in \mathbb R_+$, we have $ \sup_{s\in [0,t],x\in \mathbb R^d} |f(s,x)|<\infty. $
For any $\mu \in \mathcal M(\mathbb R^d)$, we write $\|\mu\| = \mu(1)$.
For any $\sigma$-finite signed measure $\mu$, denote by $|\mu|$ the total variation measure of $\mu$.

We say that an $\mathcal M(\mathbb R^d)$-valued Hunt process $X = \{(X_t)_{t\geq 0}; (\mathbb{P}_{\mu})_{\mu \in \mathcal M(\mathbb R^d)}\}$
is a \emph{super Ornstein-Uhlenbeck process (super-OU process)} with branching mechanism $\psi$, or a \emph{$(\xi, \psi)$-superprocess}, if for each non-negative bounded Borel function $f$ on $\mathbb R^d$, we have
\begin{align}
  \label{eq: def of V_t}
  \mathbb{P}_{\mu}[e^{-X_t(f)}]
  = e^{-\mu(V_tf)},
   \quad t\geq 0, \mu \in \mathcal M(\mathbb R^d),
\end{align}
	where $(t,x) \mapsto V_tf(x)$ is the unique locally bounded non-negative solution to the equation
\begin{align}
  V_tf(x) + \Pi_x \Big[ \int_0^t\psi (V_{t-s}f(\xi_s) ) \mathrm ds\Big]
	= \Pi_x [f(\xi_t)]
  , \quad x\in \mathbb R^d, t\geq 0.
\end{align}	
	The existence of such super-OU process $X$ is well known, see \cite{Dynkin1993Superprocesses} for instance.

	There have been many central limit theorem type results for branching processes, branching diffusions and superprocesses, under the second moment condition.
	See \cite{Heyde1970A-rate, HeydeBrown1871An-invariance, HeydeLeslie1971Improved} for supercritical Galton-Watson processes (GW processes), \cite{KestenStigum1966Additional,KestenStigum1966A-limit} for supercritical multi-type GW processes, \cite{Athreya1969Limit,Athreya1969LimitB,Athreya1971Some} for supercritical multi-type continuous time branching processes and \cite{AsmussenHering1983Branching} for general supercritical branching Markov processes under certain conditions.
	Some spatial central limit theorems for supercritical branching OU processes with binary branching mechanism were proved
	in \cite{AdamczakMilos2015CLT}, and some
	spatial central limit theorems for supercritical super-OU processes with branching mechanisms satisfying a fourth moment condition were proved in \cite{Milos2012Spatial}.
	These two papers made connections between central limit theorems and branching rate regimes.
	The results of \cite{Milos2012Spatial} were extended and refined in \cite{RenSongZhang2014Central}.
		Since then, a series of spatial central limit theorems for a large class of general supercritical branching Markov processes and superprocesses with spatially dependent branching mechanisms were proved in \cite{RenSongZhang2014CentralB,RenSongZhang2015Central,RenSongZhang2017Central}.

There are also  central  limit theorem type results for supercritical branching processes and branching Markov processes with branching mechanisms of infinite second moment. For earlier
papers, \cite{Heyde1971Some, Asmussen76Convergence}. 	Recently, Marks and Milo\'s \cite{MarksMilos2018CLT} established some spatial central limit theorems in the small and critical branching rate regimes, for
some supercritical branching OU processes with a special stable offspring distribution.
In \cite{RenSongSunZhao2019Stable}, we
established stable central limit theorems for
super-OU processes $X$ with branching mechanisms $\psi$ satisfying the following two assumptions.


\begin{asp}[Grey's condition]
	\label{asp: Greys condition}
	There exists $z' > 0$ such that $\psi(z) > 0$ for all $z>z'$ and  $\int_{z'}^\infty \psi(z)^{-1} \mathrm dz < \infty$.
\end{asp}

\begin{asp}
	\label{asp: branching mechanism}
  There exist constants $\eta > 0$ and $\beta \in (0,1)$ such that
  \begin{align}
    \int_{(1,\infty)}y^{1+\beta +\delta}\Big|\pi(\mathrm dy)-\frac{\eta \mathrm dy}{\Gamma(-1-\beta)y^{2+\beta}}\Big| <\infty
  \end{align}
for some $\delta > 0$.
\end{asp}

It is known (see \cite[Theorems 12.5 \& 12.7]{Kyprianou2014Fluctuations} for example) that, under Assumption \ref{asp: Greys condition}, the \emph{extinction event}
$$D :=\{\exists t\geq 0,~\text{such that}~ \|X_t\| =0 \}$$
is non-trivial with respect to $\mathbb P_\mu$ for each  $\mu \in \mathcal M(\mathbb R^d)\setminus\{0\}$.
In fact, $ \mathbb{P}_{\mu} (D) = e^{-\bar v \|\mu\|}$, where $ \bar v := \sup\{\lambda \geq 0: \psi(\lambda) = 0\} \in (0,\infty) $ is the largest root of $\psi$.
Assumption \ref{asp: branching mechanism} says that $\psi$ is ``not too far away'' from $\widetilde \psi(z) := - \alpha z + \eta z^{1+\beta}$ near $0$, see \cite[Remark 1.3]{RenSongSunZhao2019Stable}.
It follows from \cite[Lemma 2.2]{RenSongSunZhao2019Stable}  that, if Assumption \ref{asp: branching mechanism} holds, then $\eta$ and $\beta$ are uniquely determined by the L\'evy measure $\pi$.
In \cite[Lemma 2.3]{RenSongSunZhao2019Stable}, we have shown  that,
under Assumption \ref{asp: branching mechanism},
 $\psi$ satisfies the $L \log L$ condition, i.e., $ \int_{(1,\infty)} y\log y \pi(\mathrm dy) < \infty. $
In the reminder of the paper, we will always use $\eta$ and $\beta$ to denote the constants in Assumption  \ref{asp: branching mechanism}.
	Note that $\delta$ is not uniquely determined by $\pi$.

The limit behavior of $X$  is closely related to the spectral property of the OU semigroup $(P_t)_{t\geq 0}$ which we now recall (see \cite{MetafunePallaraPriola2002Spectrum} for more details).
We use  $(P_t)_{t\geq 0}$ to denote the transition semigroup of $\xi$.	
Define
\(
P^{\alpha}_t f(x)
  := e^{\alpha t} P_t f(x)
  = \Pi_x [e^{\alpha t}f(\xi_t)]
\)
for each $x\in \mathbb R^d$, $t\geq 0$ and $f\in \mathcal B(\mathbb R^d, \mathbb R_+)$.
It is known that, see \cite[Proposition 2.27]{Li2011Measure-valued} for example, $(P^\alpha_t)_{t\geq 0}$ is the \emph{mean semigroup} of $X$ in the sense that
\(
  \mathbb{P}_{\mu}[X_t (f)]  = \mu( P^\alpha_t f)
\)
for all $\mu\in \mathcal M(\mathbb R^d)$, $t\geq 0$ and $f\in \mathcal B(\mathbb R^d, \mathbb R_+)$.
It is known that the OU process $\xi$ has an invariant probability on $\mathbb R^d$
\begin{align}
  \varphi(x)\mathrm dx
  :=\Big (\frac{b}{\pi \sigma^2}\Big )^{d/2}\exp \Big(-\frac{b}{\sigma^2}|x|^2 \Big)\mathrm dx
\end{align}
which is a symmetric multivariate Gaussian distribution.
Let $L^2(\varphi)$ be the Hilbert space with inner product
\begin{align}
  \langle f_1, f_2 \rangle_{\varphi}
  := \int_{\mathbb R^d}f_1(x)f_2(x)\varphi(x) \mathrm dx, \quad f_1,f_2 \in L^2(\varphi).
\end{align}
Let $\mathbb Z_+ := \mathbb N\cup\{0\}$.
For each $p = (p_k)_{k = 1}^d \in \mathbb{Z}_+^{d}$, write $|p|:=\sum_{k=1}^d p_k$, $p!:= \prod_{k= 1}^d p_k!$ and $\partial_p:= \prod_{k = 1}^d(\partial^{p_k}/\partial x_k^{p_k})$.
The \emph{Hermite polynomials} are defined by
\begin{align}
  \mathcal H_p(x)
  :=(-1)^{|p|}e^{|x|^2} \partial_p e^{-|x|^2}
  , \quad x\in \mathbb R^d, p \in \mathbb{Z}_+^{d}.
\end{align}
It is known that $(P_t)_{t\geq 0}$ is a strongly continuous semigroup in $L^2(\varphi)$ and its generator $L$ has discrete spectrum $\sigma(L)= \{-bk: k \in \mathbb Z_+\}$.
For $k \in \mathbb Z_+$, denote by $\mathcal{A}_k$ the eigenspace corresponding to the eigenvalue $-bk$, then $ \mathcal{A}_k = \operatorname{Span} \{\phi_p : p\in \mathbb Z_+^d, |p|=k\}$ where
\begin{align}
  \phi_p(x)
  := \frac{1}{\sqrt{ p! 2^{|p|} }} \mathcal H_p \Big(\frac{ \sqrt{b} }{\sigma}x \Big)
  , \quad x\in \mathbb R^d, p\in \mathbb Z_+^d.
\end{align}
In other words,
\(
  P_t\phi_p(x)
  = e^{-b|p|t}\phi_p(x)
\)
for all $t\geq 0$, $x\in \mathbb R^d$ and $p\in \mathbb Z_+^d$.
Moreover, $\{\phi_p: p \in \mathbb Z_+^d\}$ forms a complete orthonormal basis of $L^2(\varphi)$.
Thus for each $f\in L^2(\varphi)$, we have
\begin{align}
  \label{semicomp1}
  f
  = \sum_{k=0}^{\infty}\sum_{p\in \mathbb Z_+^d:|p|=k}\langle f, \phi_p \rangle_{\varphi} \phi_p
  , \quad \text{in~} L^2(\varphi).
\end{align}
For each function $f\in L^2(\varphi)$, define the order of $f$ as
\[
  \kappa_f
  := \inf \left \{k\geq 0: \exists ~ p\in \mathbb Z_+^d , {\rm ~s.t.~} |p|=k {\rm ~and~}  \langle f, \phi_p \rangle_{\varphi}\neq 0\right \}
\]
which is the lowest non-trivial frequency in the eigen-expansion \eqref{semicomp1}.
Note that $ \kappa_f\geq 0$ and that, if $f\in L^2(\varphi)$ is non-trivial, then $\kappa_f<\infty$.
In particular, the order of any constant non-zero function is zero.
For $p\in \mathbb{Z}_+^d$, define
\[
  H_t^p
  := e^{-(\alpha-|p|b)t}X_t(\phi_p), \qquad t\geq 0.
\]
We will write $H^0_t$ as $H_t$.
For each $u \neq -1$, we write $\tilde u = u/(1+ u)$.
We have shown in \cite[Lemma 3.2]{RenSongSunZhao2019Stable} the following:
\begin{equation}\label{eq:Hinfty}
\begin{minipage}{0.9\textwidth}
	For any $\mu\in \mathcal M_c(\mathbb R^d)$, $(H_t^p)_{t\geq 0}$ is a $\mathbb{P}_{\mu}$-martingale.
	Futhermore, if $\alpha \tilde \beta>|p|b$, then for every $\gamma\in (0, \beta)$ and $\mu\in \mathcal M_c(\mathbb R^d)$,  $(H_t^p)_{t\geq 0}$ is a $\mathbb{P}_{\mu}$-martingale bounded in $L^{1+\gamma}(\mathbb{P}_{\mu})$;
	thus $H^p_{\infty}:=\lim_{t\rightarrow \infty}H_t^p$ exists $\mathbb{P}_{\mu}$-almost surely and in $L^{1+\gamma}(\mathbb P_\mu)$.
\end{minipage}
\end{equation}
We will write $H^0_\infty$ as $H_\infty$.

Let us also recall some results from \cite{RenSongSunZhao2019Stable} before we formulate our main theorem.
Denote by $\mathcal P$ the class of functions of polynomial growth on $\mathbb R^d$, i.e.,
\begin{align}
  \mathcal{P}
  := \{f\in \mathcal B(\mathbb R^d, \mathbb R):\exists C>0, n \in \mathbb Z_+ \text{~s.t.~} \forall x\in \mathbb R^d, |f(x)|\leq C(1+|x|)^n \}.
\end{align}
It is clear that $\mathcal{P} \subset L^2(\varphi)$.
Define
\begin{equation}
\begin{minipage}{0.9\textwidth}
	$\mathcal C_\mathrm s := \mathcal P \cap \overline{\operatorname{Span}} \{ \phi_p: \alpha \tilde \beta < |p| b \}, \quad \mathcal C_\mathrm c   := \mathcal P \cap \operatorname{Span} \{ \phi_p : \alpha \tilde \beta = |p| b \} $, and
	\\ $ \mathcal C_\mathrm l   := \mathcal P \cap \operatorname{Span} \{ \phi_p: \alpha \tilde \beta > |p| b \}$.
\end{minipage}
\end{equation}
Note that $\mathcal C_\mathrm s$ is an infinite dimensional space, $ \mathcal C_\mathrm l$ and $\mathcal C_\mathrm c$
are finite dimensional spaces, and $\mathcal C_c$ might be empty.
Define a semigroup
\begin{align}
T_t f
:= \sum_{p \in \mathbb Z_+^d} e^{-\big| |p|b - \alpha \tilde \beta \big|t} \langle f, \phi_p \rangle_{\varphi} \phi_p
,\quad t\geq 0, f\in \mathcal P,
\end{align}
and a family of functionals
\begin{align}\label{eq:M.13}
m_t[f]
:= \eta \int_0^t \mathrm du \int_{\mathbb R^d} \big(-iT_u f(x)\big)^{1+\beta} \varphi(x) \mathrm dx
, \quad 0 \leq t< \infty, f\in \mathcal P.
\end{align}
For each $\mu \in \mathcal M(\mathbb R^d)\setminus\{0\}$, write
$\widetilde {\mathbb P}_\mu(\cdot):= \mathbb P_\mu(\cdot | D^c).$
We have shown in \cite[Lemma 2.6 and Propsoition 2.7]{RenSongSunZhao2019Stable} that,
\begin{equation} \label{eq:I:R:3}
\begin{minipage}{0.9\textwidth}
for each $f\in \mathcal P$, there exists a $(1+\beta)$-stable random variable $\zeta^f$ with characteristic function
$
\theta \mapsto e^{m[\theta f]}, \theta \in \mathbb R,
$
where
\begin{equation}
m[f]
:= \begin{cases}
\lim_{t\to \infty} m_t[f], &
f \in \mathcal C_\mathrm s \oplus \mathcal C_\mathrm l, \\
\lim_{t\to \infty} \frac{1}{t} m_t[f], & f\in \mathcal P \setminus \mathcal C_\mathrm s \oplus \mathcal C_\mathrm l.
\end{cases}
\end{equation}
\end{minipage}
\end{equation}
Furthermore, we proved in \cite[Theorem 1.6]{RenSongSunZhao2019Stable} that
\begin{equation}\label{eq:oldResult}
\begin{minipage}{0.9\textwidth}
if $\mu\in \mathcal M_\mathrm c(\mathbb R^d)\setminus \{0\}$, $f_\mathrm s\in \mathcal C_\mathrm s\setminus\{0\}$, $f_\mathrm c \in \mathcal C_\mathrm c\setminus\{0
\}$ and $f_\mathrm l \in \mathcal C_\mathrm l\setminus\{0\}$, then under $\mathbb {\widetilde P}_\mu$,
\[
\begin{split}
&e^{-\alpha t}\|X_t\| \xrightarrow[t\to \infty]{\text{a.s.}} \widetilde{H}_\infty;
\quad \frac{X_t(f_\mathrm s)}{\|X_t\|^{1-\tilde \beta}} \xrightarrow[t\to \infty]{d} \zeta^{f_\mathrm s};
\\&\frac{X_t(f_\mathrm c)}{\|t X_t\|^{1-\tilde \beta} } \xrightarrow[t\to \infty]{d} \zeta^{f_\mathrm c};
\quad\frac{X_t(f_\mathrm l) - \mathrm x_t(f_\mathrm l)}{\|X_t\|^{1-\tilde \beta}}
\xrightarrow[t\to \infty]{d}
\zeta^{-f_\mathrm l},
\end{split}
\]
where 
$\widetilde H_\infty$ has the distribution of $\{H_{\infty}; \widetilde {\mathbb P}_\mu\}$;
$\zeta^{f_\mathrm s}$, $\zeta^{f_\mathrm c}$ and $\zeta^{-f_\mathrm l}$
are the $(1+\beta)$-stable random variables described in \eqref{eq:I:R:3}; and
\[
\mathrm x_t(f) : = \sum_{p\in \mathbb Z^d_+:\alpha \tilde \beta>|p|b}\langle f,\phi_p\rangle_\varphi e^{(\alpha-|p|b)t}H^p_{\infty},\quad t\geq 0, f\in \mathcal P.
\]
\end{minipage}
\end{equation}
	The above result gives the central limit theorem for $X_t(f)$ if $f\in \mathcal P\setminus\{0\}$ satisfies $\alpha \tilde \beta \leq \kappa_f b$.
A general  $f \in \mathcal P$ can be decomposed as $f_s + f_c + f_l$ with $f_s \in \mathcal C_\mathrm s$, $f_c \in \mathcal C_\mathrm c$ and $f_l \in \mathcal C_\mathrm l$; and if $f\in  \mathcal P$ satisfies $\alpha \tilde \beta > \kappa_f b$, 
then $f_\mathrm c$ and $f_\mathrm l$ maybe non-zero.
	In \cite{RenSongSunZhao2019Stable}, we were not able to
	establish a central limit theorem in this case. We conjectured there that the limit random variables in \eqref{eq:oldResult}
	for $f_\mathrm s\in \mathcal C_\mathrm s$, $f_\mathrm c\in \mathcal C_\mathrm c$ and $f_\mathrm l\in \mathcal C_\mathrm l$ are independent.
	Once this asymptotic independence is established, a central limit theorem  for $ X_t(f)$ for all $f\in  \mathcal P$ would follow.

	The main purpose of this note is to show that
	the limit  random variables in \eqref{eq:oldResult} are independent.

\begin{thm}
	\label{thm:M}
	If $\mu\in \mathcal M_\mathrm c(\mathbb R^d)\setminus \{0\}$, $f_\mathrm s\in \mathcal C_\mathrm s\setminus\{0\}$, $f_\mathrm c \in \mathcal C_\mathrm c\setminus\{0\}$ and $f_\mathrm l \in \mathcal C_\mathrm l\setminus\{0\}$, then under $\mathbb {\widetilde P}_\mu$,
	\begin{align} \label{eq:M.1}
	&S(t):=
	\Bigg(e^{-\alpha t}\|X_t\|, \frac{X_t(f_\mathrm s)}{\|X_t\|^{1-\tilde \beta}},\frac{X_t(f_\mathrm c)}{\|tX_t\|^{1-\tilde \beta}},
	\frac{ X_t(f_\mathrm l) - \mathrm x_t(f_\mathrm l)}{\|X_t\|^{1-\tilde \beta}}
	\Bigg)
	\\&\xrightarrow[t\rightarrow \infty]{d}(\widetilde H_\infty,\zeta^{f_\mathrm s},\zeta^{f_\mathrm c},\zeta^{-f_\mathrm l}),
	\end{align}
	where 
$\mathrm x_t(f_\mathrm l)$ is defined in \eqref{eq:oldResult} with $f$ replaced with $f_\mathrm l$;
$\widetilde H_\infty$ has the distribution of $\{H_{\infty}; \widetilde {\mathbb P}_\mu\}$; $\zeta^{f_\mathrm s}$, $\zeta^{f_\mathrm c}$ and $\zeta^{-f_\mathrm l}$ are the $(1+\beta)$-stable random variables described in \eqref{eq:I:R:3}; $\widetilde H_\infty$,  $\zeta^{f_\mathrm s}$, $\zeta^{f_\mathrm c}$ and $\zeta^{-f_\mathrm l}$ are independent.
\end{thm}

As a corollary of this theorem, we get  central limit theorems for $X_t(f)$ for all $f\in \mathcal P$.

\begin{cor} Let $\mu\in \mathcal M_c(\mathbb R^d)\setminus \{0\}$ and $f\in \mathcal P\setminus\{0\}$. Let  $f=f_\mathrm s + f_\mathrm c + f_\mathrm l$ be the unique decomposition of $f$ with $f_\mathrm s \in \mathcal C_\mathrm s$, $f_\mathrm c \in \mathcal C_\mathrm c$ and $f_\mathrm l \in \mathcal C_\mathrm l$.
Then under $\widetilde {\mathbb{P}}_{\mu}$, it holds that
\begin{enumerate}
\item  if $f_\mathrm c=0$, then
\[
    \frac{ X_t(f) - \mathrm x_t(f)}{\|X_t\|^{1-\tilde \beta}}
    \xrightarrow[t\to \infty]{d}
       \zeta^{f_\mathrm s}+\zeta^{-f_\mathrm l},
\]
	where $\zeta^{f_\mathrm s}$ and $\zeta^{-f_\mathrm l}$  are the $(1+\beta)$-stable random variables described in \eqref{eq:I:R:3}, $\zeta^{f_\mathrm s}$ and $\zeta^{-f_\mathrm l}$ are independent;
\item if $f_\mathrm c\neq0$, then
\[
    \frac{ X_t(f) - \mathrm x_t(f)}{\|tX_t\|^{1-\tilde \beta}}
    \xrightarrow[t\to \infty]{d}\zeta^{f_\mathrm c}.
\]
	where $\zeta^{f_\mathrm c}$ is the $(1+\beta)$-stable random variables described in \eqref{eq:I:R:3}.
\end{enumerate}
Here $\mathrm x_t(f)$ is defined in \eqref{eq:oldResult}.
\end{cor}


\section{Proof of main result}
\label{proofs of main results}

We first make some preparations before proving Theorem \ref{thm:M}.
For every $t\geq 0$ and $f\in \mathcal P$, define
\[
  	Z_t f
  	%:= \int_0^t P^\alpha_{t-s}\big( \eta (-i P^\alpha_sf)^{1+\beta}\big)ds,\quad
  	:= \int_0^t P^\alpha_{t-s}\big( \eta (-i P^\alpha_sf)^{1+\beta}\big)\mathrm ds,\quad
  	\Upsilon^f_t
   := \frac{X_{t+1} (f) - X_t(P_1^\alpha f)}{\| X_t\|^{1-\tilde \beta}}.
\]
Form \cite[Theorem 3.4]{RenSongSunZhao2019Stable} we know that, for each $f\in \mathcal P$, $\langle Z_1f,\varphi\rangle$ is the characteristic exponent of the limit of $\Upsilon^f_t$.
For $g\in \mathcal P$, define $\mathcal P_g:= \{\theta T_ng:n \in \mathbb Z_+, \theta \in [-1,1]\}$.
The following generalization of
\cite[Proposition 3.5]{RenSongSunZhao2019Stable} will be used later in the proof of  Theorem \ref{thm: II}, a special case of Theorem \ref{thm:M}.

\begin{prop}
  \label{cor:MI}
For each  $f,g\in \mathcal P$ and $\mu\in \mathcal M_c(\mathbb R^d)$, there exist $C,\delta>0$ such that for
all $n_1,n_2 \in \mathbb Z_+$, $(f_j)_{j=0}^{n_1}\subset \mathcal P_f$, $(g_j)_{j=0}^{n_2}\subset \mathcal P_g$ and $t\geq n_1+1$, we have
\begin{equation}
  \label{32corollary}
  \Big|\mathbb{\widetilde{P}}_{\mu}\Big[  \Big(\prod_{k=0}^{n_1}e^{i \Upsilon^{f_k}_{t-k-1}}\Big)  \Big( \prod_{k=0}^{n_2}e^{i \Upsilon^{g_k}_{t+k} } \Big) \Big]  -  \Big(\prod_{k=0}^{n_1} e^{\langle Z_1f_k, \varphi\rangle}\Big) \Big(\prod_{k=0}^{n_2} e^{\langle Z_1g_k, \varphi\rangle}\Big) \Big|\leq C e^{-\delta (t-n_1)}.
\end{equation}
\end{prop}
\begin{proof}
	In this proof, we fix $f,g\in \mathcal P$, $\mu\in \mathcal M_c(\mathbb R^d)$, $n_1,n_2 \in \mathbb Z_+$, $(f_j)_{j=0}^{n_1}\subset \mathcal P_f$, $(g_j)_{j=0}^{n_2}\subset \mathcal P_g$ and $t\geq n_1 + 1$.
 For any $k_1 \in \{-1,0,\dots,n_1\}$ and $k_2 \in \{-1,0,\dots,n_2\}$,  define
  \[
    a_{k_1,k_2}
    :=  \mathbb{\widetilde{P}}_{\mu}\Big[ \Big(\prod_{j=k_1+1}^{n_1} e^{i\Upsilon_{t-j-1}^{f_j}} \Big)  \Big(\prod_{j=0}^{k_2}e^{i\Upsilon_{t+j}^{g_j}}\Big) \Big] \Big(\prod_{j=0}^{k_1}e^{\langle Z_1 f_j, \varphi\rangle}\Big) \Big(\prod_{j=k_2+1}^{n_2} e^{ \langle Z_1g_j,\varphi \rangle} \Big),
  \]
where we used the convention that
$\prod_{j=0}^{-1} =1.$
  Then for all  $k_2 \in \{0,\dots,n_2\}$, we have
\begin{align}
&a_{-1,k_2} - a_{-1,k_2-1}
\\& \begin{multlined}
	=\mathbb{\widetilde{P}}_{\mu}\Big[ \Big(\prod_{j=0}^{n_1} e^{i\Upsilon_{t-j-1}^{f_j}} \Big)  \Big(\prod_{j=0}^{k_2}e^{i\Upsilon_{t+j}^{g_j}}\Big) \Big] \Big(\prod_{j=k_2+1}^{n_2} e^{ \langle Z_1g_j,\varphi \rangle} \Big)
	\\- \mathbb{\widetilde{P}}_{\mu}\Big[ \Big(\prod_{j=0}^{n_1} e^{i\Upsilon_{t-j-1}^{f_j}} \Big)  \Big(\prod_{j=0}^{k_2 - 1}e^{i\Upsilon_{t+j}^{g_j}}\Big) \Big] \Big(\prod_{j=k_2}^{n_2} e^{ \langle Z_1g_j,\varphi \rangle} \Big)
\end{multlined}
	\\& \begin{multlined}
	= \frac{1}{\mathbb{P}_{\mu}(D^c)} \Big(\prod_{j=k_2+1}^{ n_2} e^{\langle Z_1g_j, \varphi\rangle}\Big) \times {}
	\\ \mathbb{P}_{\mu}\Big[\Big(\prod_{j=0}^{n_1}e^{i\Upsilon_{t-j-1}^{f_j}}\Big) \Big(\prod_{j=0}^{k_2-1} e^{i\Upsilon_{t+j}^{g_j}}\Big) (e^{i\Upsilon^{g_{k_2}}_{t+k_2}}-e^{\langle Z_1g_{k_2}, \varphi\rangle});D^c\Big]
	\end{multlined}
	\\ \label{eq:MI.1}& \begin{multlined}
	= \frac{1}{\mathbb{P}_{\mu}(D^c)}  \Big(\prod_{j=k_2+1}^{n_2}e^{\langle Z_1g_j, \varphi\rangle}\Big) \times {}
	\\ \mathbb{P}_{\mu}\Big[\Big(\prod_{j=0}^{n_1}e^{i\Upsilon_{t-j-1}^{f_j}}\Big)\Big(\prod_{j=0}^{k_2-1} e^{i\Upsilon_{t+j}^{g_j}}\Big) \mathbb P_\mu[e^{i\Upsilon^{g_{k_2}}_{t+k_2}}-e^{\langle Z_1g_{k_2}, \varphi\rangle}; D^c|\mathscr F_{t+k_2}] \Big].
	\end{multlined}
\end{align}
	Therefore, there exist $C_0,\delta_0 >0$,  depending only on $\mu$ and $g$,
  such that  for each $k_2 \in \{0, \dots, n_2 \}$,
\begin{align}
    &| a_{-1,k_2} - a_{-1,k_2-1}|
    \overset{\eqref{eq:MI.1}}\leq \mathbb{P}_{\mu}(D^c)^{-1}\mathbb{P}_{\mu}\Big[\big|\mathbb P_\mu[e^{i\Upsilon^{g_{k_2}}_{t+k_2}}-e^{\langle Z_1g_{k_2}, \varphi\rangle}; D^c|\mathscr F_{t+k_2}]\big|\Big]
    \\&\label{eq:MI.15}\overset{\text{\cite[Proposition 3.5]{RenSongSunZhao2019Stable}}}\leq C_0 e^{-\delta_0 (t+k_2)}.
\end{align}
	Notice that, for any $k_1 \in \{0, \dots , n_1\}$,
	\begin{align}
	&a_{k_1-1,-1} - a_{k_1,-1}
	\\ & \begin{multlined}
	=  \mathbb{\widetilde{P}}_{\mu}\Big[ \prod_{j=k_1}^{n_1} e^{i\Upsilon_{t-j-1}^{f_j}} \Big] \Big(\prod_{j=0}^{k_1-1}e^{\langle Z_1 f_j, \varphi\rangle}\Big) \Big(\prod_{j=0}^{n_2} e^{ \langle Z_1g_j,\varphi \rangle} \Big) - {}
	\\ \mathbb{\widetilde{P}}_{\mu}\Big[ \prod_{j=k_1+1}^{n_1} e^{i\Upsilon_{t-j-1}^{f_j}} \Big] \Big(\prod_{j=0}^{k_1}e^{\langle Z_1 f_j, \varphi\rangle}\Big) \Big(\prod_{j=0}^{n_2} e^{ \langle Z_1g_j,\varphi \rangle} \Big)
	\end{multlined}
	\\& =  \mathbb{\widetilde{P}}_{\mu}\Big[ \big(e^{i\Upsilon_{t-k_1-1}^{f_{k_1}}} -e^{\langle Z_1 f_{k_1}, \varphi\rangle} \big) \prod_{j=k_1+1}^{n_1} e^{i\Upsilon_{t-j-1}^{f_j}} \Big] \Big(\prod_{j=0}^{k_1-1}e^{\langle Z_1 f_j, \varphi\rangle}\Big) \Big(\prod_{j=0}^{n_2} e^{ \langle Z_1g_j,\varphi \rangle} \Big)
	\\& \label{eq:MI.2}\begin{multlined}
	=  \frac{1}{\mathbb P_\mu(D^c)} \Big(\prod_{j=0}^{k_1-1}e^{\langle Z_1 f_j, \varphi\rangle}\Big) \Big(\prod_{j=0}^{n_2} e^{ \langle Z_1g_j,\varphi \rangle} \Big) \times {}
	\\ \qquad \mathbb{P}_{\mu}\Big[ \mathbb P_\mu\big[e^{i\Upsilon_{t-k_1-1}^{f_{k_1}}} -e^{\langle Z_1 f_{k_1}, \varphi\rangle}  ; D^c \big| \mathscr F_{t-k_1 - 1}\big] \prod_{j=k_1+1}^{n_1} e^{i\Upsilon_{t-j-1}^{f_j}} \Big] .
	\end{multlined}
	\end{align}
	Therefore,  there exist $C_1,\delta_1 >0$, depending only on $\mu$ and $f$,
	such that for any $k_1 \in \{0,\dots,n_1\}$,
\begin{align}
    &| a_{k_1-1,-1} - a_{k_1,-1}|
    \overset{\eqref{eq:MI.2}}\leq \frac{1}{\mathbb{P}_{\mu}(D^c)}\mathbb{P}_{\mu}\Big[\big|\mathbb P_\mu[e^{i\Upsilon^{f_{k_1}}_{t-k_1-1}}-e^{\langle Z_1f_{k_1}, \varphi\rangle}; D^c|\mathscr F_{t-k_1-1}]\big|\Big]
    \\&\label{eq:MI.25}\overset{\text{\cite[Proposition 3.5]{RenSongSunZhao2019Stable}}}\leq C_1 e^{-\delta_1 (t-k_1)}.
\end{align}
  Therefore, there exist $C,\delta >0$, depending only on $f,g$ and $\mu$, such that
  \begin{align}
    &\text{LHS of \eqref{32corollary}}
    = \left|a_{-1,n_2}-a_{n_1,-1}\right|
      \leq \sum_{k=0}^{n_1}\left|a_{k-1,-1}-a_{k,-1}\right|+\sum_{k=0}^{n_2}\left|a_{-1,k}-a_{-1,k-1}\right|\\
     & \overset{\eqref{eq:MI.15},\eqref{eq:MI.25}}\leq \sum_{k=0}^{n_1} C_1 e^{-\delta_1 (t-k)}+\sum_{k=0}^{n_2} C_0 e^{-\delta_0 (t+k)}
      \leq C e^{- \delta (t-n_1)}.
      \qedhere
  \end{align}
\end{proof}

The following elementary result will also be used in the proof of Theorem \ref{thm: II}.
\begin{lem}\label{ineq: analysis}
There exists a constant $C>0$, such that for any $x,y \in \mathbb R$,
\[
    |(x+y)^{1+\beta}-x^{1+\beta}-y^{1+\beta}|\leq C(|x||y|^{\beta}+|x|^{\beta}|y|).
\]
\end{lem}
\begin{proof}
   Note that
\[
  \lim_{|y|\rightarrow \infty}\frac{(y+1)^{1+\beta}-y^{1+\beta}-1}{y^{\beta}}=\lim_{|y|\rightarrow \infty}\frac{(y+1)^{1+\beta}-y^{1+\beta}}{y^{\beta}}=\lim_{|y|\rightarrow \infty}\big((1+\frac{1}{y})^{1+\beta}-1\big)y = 1+\beta.
\]
% Using this and continuity, we get that there exists {\color{blue}$C_1>0$} such that for all $|y|\geq 1$,
Using this and continuity, we get that there exists $C_1>0$ such that for all $|y|\geq 1$,
\[
  |(1+y)^{1+\beta}-y^{1+\beta}-1|\leq C_1 |y|^{\beta}.
\]
 Note that if $x = 0$ or $y= 0$, then the desired result is trivial.
So we only need to consider
 the case that $x \neq 0$ and $y \neq 0$.
 In this case, if $|x|\geq |y|$, we have
\[
|(x+y)^{1+\beta}-x^{1+\beta}-y^{1+\beta}|\leq |y|^{1+\beta}\bigg(\Big|\Big(1+\frac{x}{y}\Big)^{1+\beta}-\Big(\frac{x}{y}\Big)^{1+\beta}-1\Big|\bigg)\leq C_1|y||x|^{\beta};
\]
and  if $|x|\leq |y|$, we have
\[
	|(x+y)^{1+\beta}-x^{1+\beta}-y^{1+\beta}|
	\leq |x|^{1+\beta}\bigg( \Big| \Big(1+\frac{y}{x}\Big)^{1+\beta}-\Big(\frac{y}{x}\Big)^{1+\beta}-1\Big|\bigg)\leq C_1|x||y|^{\beta}.
\]
Combining the above, we immediately get the desired result.
\end{proof}

In the remainder of this section,
we always fix $\mu \in \mathcal M_c(\mathbb R^d)\setminus\{0\}$, $f_\mathrm s\in \mathcal C_\mathrm s\setminus\{0\}$,  $f_\mathrm c\in \mathcal C_\mathrm c\setminus\{0\}$ and $f_\mathrm l\in \mathcal C_\mathrm l\setminus\{0\}$.
For any random variable $Y$ with finite mean under $\mathbb P_\mu$, we define
\begin{align}
\mathcal I_r^t Y := \mathbb P_\mu[Y|\mathscr F_{t\vee 0}] - \mathbb P_\mu[Y|\mathscr F_{r\vee 0}], \quad -\infty < r, t <\infty.
\end{align}
	For each $t\geq 1$,
	we have the following decomposition.
  \begin{align}
     I^{f_\mathrm s}(t) &:= \frac{X_t(f_\mathrm s)}{\|X_t\|^{1- \tilde \beta }}  = I^{f_\mathrm s}_1(t) + I^{f_\mathrm s}_2(t) + I^{f_\mathrm s}_3(t)
        \\&:=  \Big(\sum_{k\in \mathbb N \cap [0,t-\ln t]}  \frac{\mathcal I_{t-k-1}^{t-k} X_t(f_\mathrm s)}{\|X_t\|^{1-\tilde \beta} } \Big) + \Big(\sum_{k\in\mathbb{N}\cap(t-\ln t, t]} \frac{\mathcal I_{t-k-1}^{t-k} X_t(f_\mathrm s)}{\|X_t\|^{1-\tilde \beta}}  \Big)
    + \Big( \frac{X_0(P_t^\alpha f_\mathrm s)}{\|X_t\|^{1-\tilde \beta}}\Big),
       \\ I^{f_\mathrm c}(t)&:=\frac{X_t(f_\mathrm c)}{\|tX_t\|^{1-\tilde \beta}} = I^{f_\mathrm c}_1(t) + I^{f_\mathrm c}_2(t) + I^{f_\mathrm c}_3(t)
    \\&:=  \Big(\sum_{k\in \mathbb N \cap [0,t-\ln t]}  \frac{\mathcal I_{t-k-1}^{t-k} X_t(f_\mathrm c)}{\|tX_t\|^{1-\tilde \beta} } \Big) + \Big(\sum_{k\in\mathbb{N}\cap(t-\ln t, t]} \frac{\mathcal I_{t-k-1}^{t-k} X_t(f_\mathrm c)}{\|tX_t\|^{1-\tilde \beta}}  \Big)    + \Big( \frac{X_0(P_t^\alpha f_\mathrm c)}{\|tX_t\|^{1-\tilde \beta}}\Big),
   \\ I^{f_\mathrm l}(t) &:= \frac{X_t(f_\mathrm l) - \mathrm x_t(f_\mathrm l)}{\|X_t\|^{1- \tilde \beta}}
  = I^{f_\mathrm l}_1(t) + I^{f_\mathrm l}_2(t) + I^{f_\mathrm l}_3(t)
  \\& := \Big(\sum_{k\in \mathbb N \cap [0,t^2]} \frac{ \mathcal I_{t+k+1}^{t+k}\mathrm x_t(f_\mathrm l)}{\|X_t\|^{1- \tilde \beta}}\Big) + \Big(\sum_{k\in \mathbb N \cap (t^2,\infty)}  \frac{ \mathcal I_{t+k+1}^{t+k}\mathrm x_t(f_\mathrm l)}{\|X_t\|^{1- \tilde \beta}}\Big)+0,
  \end{align}
 where $\mathrm x_t(f_\mathrm l)$ is defined in \eqref{eq:oldResult} with $f$ replaced with $f_\mathrm l$.
	For every $t\geq 1$, define
	\begin{align}
	R_j(t)&:=\big(I_j^{f_\mathrm s}(t),I_j^{f_\mathrm c}(t),I_j^{f_\mathrm l}(t)\big), \quad j = 1,2,3,
	\\ R(t)&:=\Big( \frac{X_t(f_\mathrm s)}{\|X_t\|^{1-\tilde \beta}},\frac{X_t(f_\mathrm c)}{\|tX_t\|^{1-\tilde \beta}},\frac{ X_t(f_\mathrm l) - \mathrm x_t(f_\mathrm l)}{\|X_t\|^{1-\tilde \beta}}\Big),
    \\R_0(t)&=\big(I_0^{f_\mathrm s}(t),I_0^{f_\mathrm c}(t),I_0^{f_\mathrm l}(t)\big)
       \\&:=\Big(\sum_{k=0}^{\lfloor t-\ln t \rfloor} \Upsilon_{t-k-1}^{T_k \tilde f_\mathrm s},t^{\tilde \beta - 1}\sum_{k=0}^{\lfloor t-\ln t \rfloor} \Upsilon_{t-k-1}^{T_{k} \tilde f_\mathrm c},\sum_{k = 0}^{\lfloor t^2 \rfloor} \Upsilon_{t+k}^{- T_k \tilde f_\mathrm l}\Big),
\end{align}
where 
 $\mathrm x_t(f_\mathrm l)$ is defined in \eqref{eq:oldResult} with $f$ replaced with $f_\mathrm l$.
$\tilde f_\mathrm s:=e^{\alpha(\tilde \beta - 1)} f_\mathrm s$, $\tilde f_\mathrm c:=e^{\alpha(\tilde \beta - 1)} f_\mathrm c$ and  $\tilde f_\mathrm l := \sum_{p\in \mathcal N} e^{-(\alpha - |p|b)}\langle f_\mathrm l, \phi_p \rangle_\varphi \phi_p$.
The following result is a special case of Theorem \ref{thm:M}.

\begin{thm}\label{thm: II}
Under $\mathbb{\widetilde{P}}_{\mu}$,
$
R(t) \xrightarrow[t\rightarrow\infty]{d}(\zeta^{f_\mathrm s},\zeta^{f_\mathrm c},\zeta^{-f_\mathrm l}),
$
where $\zeta^{f_\mathrm s},\zeta^{f_\mathrm c}$ and $\zeta^{-f_\mathrm l}$ are
the $(1+\beta)$-stable random variables described in \eqref{eq:I:R:3}, and $\zeta^{f_\mathrm s},\zeta^{f_\mathrm c}$ and $\zeta^{-f_\mathrm l}$ are independent.
\end{thm}
\begin{proof}
%In this proof, we always work under $\mathbb{\widetilde{P}}_{\mu}(\cdot)$.
In this proof, we always work under $\mathbb{\widetilde{P}}_{\mu}$.
Note that for each $t\geq 1$,
\[
R(t)=R_0(t)+\big(R_1(t)-R_0(t)\big)+R_2(t)+R_3(t).
\]
Note that
$$
(R_1(t)-R_0(t))=(I_1^{f_\mathrm s}(t)-I_0^{f_\mathrm s}(t), I_1^{f_\mathrm c}(t)-I_0^{f_\mathrm c}(t), I_1^{f_\mathrm l}(t)-I_0^{f_\mathrm l}(t)).
$$
In the proof of Theorem 1.6(1) in \cite{RenSongSunZhao2019Stable}, we proved that
$I_1^{f_\mathrm s}(t)-I_0^{f_\mathrm s}(t)\xrightarrow[t\to \infty]{d} 0$, $I_2^{f_\mathrm s}(t)\xrightarrow[t\to \infty]{d} 0$ and $I_3^{f_\mathrm s}(t)\xrightarrow[t\to \infty]{\mathbb{\widetilde{P}}_{\mu}-a.s.} 0.$
 In the proof of Theorem 1.6(2) in \cite{RenSongSunZhao2019Stable}, we proved that
 $I_1^{f_\mathrm c}(t)-I_0^{f_\mathrm c}(t)\xrightarrow[t\to \infty]{d} 0$, $I_2^{f_\mathrm c}(t)\xrightarrow[t\to \infty]{d} 0$ and $I_3^{f_\mathrm c}(t)\xrightarrow[t\to \infty]{\mathbb{\widetilde{P}}_{\mu}-a.s.} 0.$
In the proof of Theorem 1.6(3) in \cite{RenSongSunZhao2019Stable}, we proved that
$I_1^{f_\mathrm l}(t)-I_0^{f_\mathrm l}(t)\xrightarrow[t\to \infty]{d} 0$ and  $I_2^{f_\mathrm l}(t)\xrightarrow[t\to \infty]{d} 0$.
Thus we have 
$R_1(t)-R_0(t)\xrightarrow[t\to \infty]{d} (0,0,0)$, $R_2(t)\xrightarrow[t\to \infty]{d}(0,0,0)$ and
$R_3(t)\xrightarrow[t\to \infty]{d}(0,0,0)$.
Combining the above results and using Slutsky's theorem, we only need to  show that, under $\mathbb{\widetilde{P}}_{\mu}$,
\begin{equation}
\label{lem:UOT}R_0(t) \xrightarrow[t\to \infty]{d}(\zeta^{f_\mathrm s},\zeta^{f_\mathrm c},\zeta^{-f_\mathrm l}).
\end{equation}

Now we prove \eqref{lem:UOT}.
Since $\Upsilon_t^f$ is linear in $f$, for each $t\geq 1$,
\[
\widetilde{\mathbb P}_{\mu}\Big[\exp\Big(i
\sum_{j=\mathrm s,\mathrm c,\mathrm l}I_0^{f_j}(t)\Big)\Big]
= \widetilde{\mathbb P}_{\mu}\Big[\exp\Big(i\sum_{k=0}^{\lfloor t-\ln t \rfloor}\Upsilon_{t-k-1}^{T_k(\tilde{f_\mathrm s}+t^{\tilde{\beta}-1} \tilde{f}_{\mathrm c})}\Big)\exp\Big(i\sum_{k=0}^{\lfloor t^2 \rfloor}\Upsilon_{t+k}^{-T_k\tilde{f}_{\mathrm l}}\Big)\Big].
\]
%Using Corollary \ref{cor:MI} 
Using Proposition \ref{cor:MI} 
with $f=\tilde{f_\mathrm s}+t^{\tilde{\beta}-1} \tilde{f}_{\mathrm c}$ and $g = -\tilde{f}_{\mathrm l}$, we get that there exist $C_1,\delta_1 > 0$ such that
for every $t\geq 1$,
  \begin{align}
    &\Big|\widetilde{\mathbb P}_{\mu}\Big[\exp\Big(i \sum_{j=\mathrm s,\mathrm c,\mathrm l}I_0^{f_j}(t)\Big)\Big]
    -\exp\Big(\sum_{k=0}^{\lfloor t-\ln t \rfloor} \langle Z_1(T_{k}(\tilde f_\mathrm s+t^{\tilde{\beta}-1}\tilde{f}_{\mathrm c})), \varphi\rangle \Big)\exp\Big(\sum_{k=0}^{\lfloor t^2 \rfloor}\langle Z_1(-T_k\tilde{f}_{\mathrm l}),\varphi\rangle\Big)\Big|\\
    &\leq C_1 e^{-\delta_1(t - \lfloor t - \ln t\rfloor)}.
  \end{align}
%Next, we claim that
We claim that
\begin{align} \label{eq:UOT.1}
&\lim_{t\rightarrow\infty}\exp\Big(\sum_{k=0}^{\lfloor t-\ln t \rfloor} \langle Z_1(T_{k}(\tilde f_\mathrm s+t^{\tilde{\beta}-1}\tilde{f}_\mathrm c)), \varphi\rangle \Big)\exp\Big(\sum_{k=0}^{\lfloor t^2 \rfloor}\langle Z_1(-T_k\tilde{f}_\mathrm l),\varphi\rangle\Big)\\
& =\exp(m[f_\mathrm s]+m[f_\mathrm c]+m[-f_\mathrm l]).
\end{align}
Given this claim, we have
\[
	\widetilde{\mathbb P}_{\mu}\Big[\exp\Big(i
	\sum_{j=\mathrm s,\mathrm c,\mathrm l}I_0^{f_j}(t)\Big)\Big]
	 \xrightarrow[t\to \infty]{}
	 \exp(m[f_\mathrm s]+m[f_\mathrm c]+m[-f_\mathrm l]).
\]
 Since $I_0^{f_j}(t)$ are linear in 
     $f_j\in \mathcal C_j (j=\mathrm s,\mathrm c,\mathrm l)$,
 replacing $f_j$ with $\theta_j f_j$, we immediately get 
 \eqref{lem:UOT}.

Now we prove the claim \eqref{eq:UOT.1}.
For every $f\in \mathcal C_\mathrm s  \oplus \mathcal C_\mathrm c$ and $n\in \mathbb Z_+$,
 \begin{align}
  & \sum_{k=0}^n \langle Z_1 T_{k} \tilde f, \varphi \rangle
  = \sum_{k=0}^n \int_0^1 \langle P_u^\alpha (\eta(-iP_{1 - u}^\alpha T_k \tilde f)^{1+\beta}), \varphi\rangle \mathrm du
 \\& = \sum_{k=0}^n \int_0^1 e^{\alpha u} \langle \eta (-iP_{1 - u}^\alpha T_{k}\tilde f)^{1+\beta}, \varphi \rangle \mathrm du
 \\& = \sum_{k=0}^n \int_0^1 \langle \eta (-iT_{k+1 - u} f)^{1+\beta}, \varphi\rangle \mathrm du
 = \int_0^{n+1} \langle  \eta (-iT_{u} f)^{1+\beta}, \varphi\rangle \mathrm du = m_{n+1}[f],
 \end{align}
where $\tilde f=e^{\alpha(\tilde \beta - 1)} f$.
Therefore, for any $t\geq 1$,
\begin{equation} \label{eq:UOT.15}
\sum_{k=0}^{\lfloor t-\ln t \rfloor} \langle Z_1T_{k}(\tilde f_\mathrm s+t^{\tilde{\beta}-1}\tilde{f}_\mathrm c), \varphi\rangle
=\eta \int_0^{\lfloor t-\ln t \rfloor+1}\big\langle \big(-iT_u(f_\mathrm s+t^{\tilde{\beta}-1}f_\mathrm c)\big)^{1+\beta},\varphi \big\rangle \mathrm du.
\end{equation}
Note that for each $u\geq 0$, $T_uf_\mathrm c=f_\mathrm c$.
 Also note that according to
 Step 1 in the proof of \cite[Lemma 2.6]{RenSongSunZhao2019Stable}, there exist $\delta> 0$ and
$h\in \mathcal P$ (depending only on $f_\mathrm s$)
such that for each $u\geq 0$,
 $|T_u f_\mathrm s|\leq e^{-\delta u}h$.
It follows from Lemma \ref{ineq: analysis} that there exists $C>0$ such that for all $u\geq 0$ and $t\geq 0$,
\begin{align}
  &|(-i(T_uf_\mathrm s+t^{\tilde{\beta}-1}T_uf_\mathrm c))^{1+\beta}-(-iT_uf_\mathrm s)^{1+\beta}-(-it^{\tilde{\beta}-1}T_uf_\mathrm c)^{1+\beta}|
  \\&  = |-i|^{1+\beta} |(T_u f + t^{\tilde \beta -1} T_u f_\mathrm c)^{1+\beta} - (T_u f_\mathrm s)^{1+\beta} - (t^{\tilde \beta - 1}T_u f_\mathrm c)^{1+\beta}|
  % \\&\overset{Lemma \ \ref{ineq: analysis}}
  \\&\overset{\text{Lemma \ref{ineq: analysis}}}
  \leq  C(t^{-\frac{\beta}{1+\beta}}|T_uf_\mathrm s||T_uf_\mathrm c|^{\beta}+t^{-\frac{1}{1+\beta}}|T_uf_\mathrm s|^{\beta}|T_uf_\mathrm c|)
  \\&\label{eq:UOT.21}\leq C(t^{-\frac{\beta}{1+\beta}}e^{-\delta u}h|f_\mathrm c|^{\beta}+t^{-\frac{1}{1+\beta}}e^{-\delta\beta u}h^{\beta}|f_\mathrm c|).
\end{align}
This means that there exists $C_1 >0$ such that for all $t\geq 1$,
\begin{align}
&\Big|\Big(\sum_{k=0}^{\lfloor t-\ln t \rfloor} \langle Z_1T_{k}(\tilde f_\mathrm s+t^{\tilde{\beta}-1}\tilde{f}_c), \varphi\rangle \Big)-m_{\lfloor t-\ln t \rfloor+1}[f_\mathrm s]-\frac{1}{t}m_{\lfloor t-\ln t \rfloor+1}[f_\mathrm c]\Big|
\\& \begin{multlined}
\overset{\eqref{eq:UOT.15},\eqref{eq:M.13}}\leq\Big| \eta \int_0^{\lfloor t-\ln t \rfloor+1}\big\langle \big(-iT_u(f_\mathrm s+t^{\tilde{\beta}-1}f_\mathrm c)\big)^{1+\beta},\varphi \big\rangle \mathrm du - {}
\\ \eta \int_0^{\lfloor t-\ln t \rfloor+1} \langle (-iT_u f_\mathrm s)^{1+\beta}, \varphi \rangle  \mathrm du - \eta \int_0^{\lfloor t-\ln t \rfloor+1} \langle (-iT_u f_\mathrm c)^{1+\beta}, \varphi \rangle  \mathrm du\Big|
\end{multlined}
\\ & \overset{\eqref{eq:UOT.21}}\leq C_1\int_0^{\lfloor t-\ln t \rfloor+1}\langle t^{-\frac{\beta}{1+\beta}}e^{-\delta u}h|f_\mathrm c|^{\beta}+t^{-\frac{1}{1+\beta}}e^{-\delta\beta u}h^{\beta}|f_\mathrm c|, \varphi \rangle \mathrm du
\\& \leq C_1 t^{-\frac{\beta}{1+\beta}} \langle h|f_\mathrm c|^{\beta},\varphi \rangle \int_0^\infty e^{-\delta u} \mathrm du + C_1 t^{-\frac{1}{1+\beta}} \langle h^\beta|f_\mathrm c|,\varphi \rangle \int_0^\infty e^{-\delta \beta u} \mathrm du
\\& \xrightarrow[t\rightarrow \infty]{} 0.
\end{align}
Combining this with \eqref{eq:I:R:3}, we get that
\begin{equation} \label{eq:UOT.4}
\lim_{t\rightarrow \infty}\exp\Big(\sum_{k=0}^{\lfloor t-\ln t \rfloor} \langle Z_1T_{k}(\tilde f_\mathrm s+t^{\tilde{\beta}-1}\tilde{f}_c), \varphi\rangle \Big)  = \exp( m[f_\mathrm s]+m[f_\mathrm c]).
\end{equation}
Also note that according to the Step 1 in the Proof of Theorem 1.6.(3) in \cite{RenSongSunZhao2019Stable}, we have
\begin{equation}\label{eq:UOT.5}
\lim_{t\rightarrow \infty}\exp\Big(\sum_{k=0}^{\lfloor t^2 \rfloor}\langle Z_1(-T_k\tilde{f}_\mathrm l),\varphi\rangle\Big) =\exp(m[-f_\mathrm l]).
\end{equation}
Thus the desired claim follows from \eqref{eq:UOT.4} and \eqref{eq:UOT.5}.
\end{proof}

\begin{proof}[Proof of Theorem \ref{thm:M}]
 We first recall some facts about weak convergence which will be used later. For $f:\mathbb R^d\mapsto \mathbb R$, let
 $$
 \|f\|_L:=\sup_{x\neq y}\frac{|f(x)-f(y)|}{|x-y|}
 $$
 and $\|f\|_{BL}:= \|f\|_{\infty}+\|f\|_L. $
 For any probability distributions $\mu_1$ and $\mu_2$ on $\mathbb R^d$, define
\[
  d(\mu_1,\mu_2)
  %:=\sup\Big\{\Big|\int fd\mu_1-\int f d\mu_2\Big|:\|f\|_{BL}\leq 1\Big\}.
  :=\sup\Big\{\Big|\int f \mathrm d\mu_1-\int f \mathrm d\mu_2\Big|:\|f\|_{BL}\leq 1\Big\}.
\]
Then $d$ is a metric. It follows from \cite[Theorem 11.3.3]{Dudley2002} that the topology generated by $d$ is equivalent to the weak convergence topology.
Using the definition, we can easily see that, if $\mu_1$ and $\mu_2$ are the distributions of two $\mathbb R^d $-valued random variables $X$ and $Y$ respectively,
defined on same probability space then
\begin{align}\label{ineq: distribution control}
  d(\mu_1,\mu_2) \leq \mathbb E|X-Y|.
\end{align}
In this proof, let us fix $\mu\in \mathcal M_\mathrm c(\mathbb R^d)\setminus \{0\}$, $f_\mathrm s\in \mathcal C_\mathrm s\setminus\{0\}$, $f_\mathrm c \in \mathcal C_\mathrm c\setminus\{0\}$ and $f_\mathrm l \in \mathcal C_\mathrm l\setminus\{0\}$.
Recall that $S(t)~(t\geq 0)$ is given by \eqref{eq:M.1}.
For every $r,t> 0$, let
\begin{align}
 S(t,r):=\Big(e^{-\alpha t}\|X_t\|,& \frac{X_{t+r}(f_\mathrm s)}{\|X_{t+r}\|^{1-\tilde{\beta}}}, \frac{X_{t+r}(f_\mathrm c)}{\|(t+r)X_{t+r}\|^{1-\tilde{\beta}}}, \frac{X_{t+r}(f_\mathrm l)-\mathrm x_{t+r}(f_\mathrm l) }{\|X_{t+r}\|^{1-\tilde{\beta}}}\Big),
\end{align}
and
\begin{align}
\widetilde{S}(t,r)= \Big(e^{-\alpha (t+r)}\|X_{t+r}\|-e^{-\alpha t}\|X_t\|,0,0,0\Big),
\end{align}
where, for any $t>0$, $\mathrm x_t(f_\mathrm l)$ is defined in \eqref{eq:oldResult} with $f$ replaced with $f_\mathrm l$.
Then $S(t+r)=S(t,r)+\widetilde{S}(t,r)$.
%Let us claim that
We claim that
\begin{equation}\label{eq:M.3}
\begin{minipage}{0.9\textwidth}
	for each $t> 0$, under $\widetilde{\mathbb P}_{\mu}$, we have
	\[
	S(t,r)\xrightarrow[r\rightarrow \infty]{d} (\widetilde H_t,\zeta^{f_\mathrm s},\zeta^{f_\mathrm c},\zeta^{-f_\mathrm l}),
	\]
	where $\widetilde H_t$ has the distribution of $\{e^{-\alpha t} \|X_t\|; \widetilde {\mathbb P}_\mu\}$,
	$\zeta^{f_\mathrm s},\zeta^{f_\mathrm c}$ and $\zeta^{-f_\mathrm l}$ are the $(1+\beta)$-stable random variables described in \eqref{eq:I:R:3}, 
	%and $\widetilde H_t$ is independent of $(\zeta^{f_\mathrm s},\zeta^{f_\mathrm c},\zeta^{-f_\mathrm l})$.
	and $\widetilde H_t,\zeta^{f_\mathrm s},\zeta^{f_\mathrm c}$ and $\zeta^{-f_\mathrm l}$ are independent.
\end{minipage}
\end{equation}

For every $r,t\geq 0$, let $\mathcal D(r)$ and $\mathcal D(r,t)$ be the distributions of $S(r)$ and $S(t,r)$ under $\widetilde{\mathbb P}_{\mu}$ respectively;
let $\widetilde{\mathcal D}(t)$ and $\mathcal D$ be the distributions of $(\widetilde H_t,\zeta^{f_\mathrm s},\zeta^{f_\mathrm c},\zeta^{-f_\mathrm l})$ and $(\widetilde H_\infty,\zeta^{f_\mathrm s},\zeta^{f_\mathrm c},\zeta^{-f_\mathrm l})$, respectively.
 Then for each $\gamma\in (0,\beta)$, there exist constant $C>0$
 such that for every $t > 0$,
\begin{align}
 &\varlimsup_{r\rightarrow \infty}d(\mathcal D(t+r),\mathcal D)
 \\& \overset{\rm triangle~inequality}\leq \varlimsup_{r\rightarrow \infty}\Big( d\big(\mathcal D(t+r),\mathcal D(t,r)\big)+d\big(\mathcal D(t,r),\widetilde{\mathcal D}(t)\big)+d\big(\widetilde{\mathcal D}(t),\mathcal D\big)\Big)
  \\&\overset{\eqref{ineq: distribution control}} \leq \varlimsup_{r\rightarrow \infty}\widetilde{\mathbb P}_{\mu}[|S(t+r) - S(t,r)|]+ \varlimsup_{r\rightarrow \infty} d\big(\mathcal D(t,r),\widetilde{\mathcal D}(t)\big) +\widetilde{\mathbb P}_{\mu}[|H_t-H_{\infty}|]
\\&\overset{\eqref{eq:M.3}}\leq \varlimsup_{r\rightarrow \infty}\widetilde{\mathbb P}_{\mu}[|H_t - H_{t+r}|]+\widetilde{\mathbb P}_{\mu}[|H_t-H_{\infty}|]
  \\&\overset{\text{H\"older inequality}}\leq \varlimsup_{r\rightarrow \infty}\mathbb P_{\mu}(D^c)^{-1}(\|H_t-H_{t+r}\|_{L_{1+\gamma} (\mathbb P_\mu)}+\|H_t-H_{\infty}\|_{L_{1+\gamma} (\mathbb P_\mu)})
\\&\label{eq:M.4}\overset{\text{\cite[Lemma 3.3]{RenSongSunZhao2019Stable}}}\leq C e^{-\alpha \tilde \gamma t}.
\end{align}
Therefore,
\begin{align}
 &\varlimsup_{r\rightarrow \infty}d\big(\mathcal D(r),\mathcal D\big)
 =\varlimsup_{t\to \infty}\varlimsup_{r\rightarrow \infty}d\big(\mathcal D(t+r),\mathcal D\big)
\overset{\eqref{eq:M.4}}\leq \varlimsup_{t\rightarrow \infty}Ce^{-\alpha \tilde \gamma t} = 0.
\end{align}
The desired result now follows immediately.

Now we prove the claim \eqref{eq:M.3}.
For every $r,t >0$, let
\[
\theta,\theta_\mathrm s,\theta_\mathrm c,\theta_\mathrm l\in \mathbb R \mapsto k(\theta,\theta_\mathrm s,\theta_\mathrm c,\theta_\mathrm l,r,t)
\]
be the characteristic function of  $S(t,r)$ under $\widetilde{\mathbb P}_{\mu}$.
Then for each $\theta,\theta_\mathrm s,\theta_\mathrm c,\theta_\mathrm l\in \mathbb R$ and $r,t> 0$,
\begin{align}
&k(\theta,\theta_\mathrm s,\theta_\mathrm c,\theta_\mathrm l,r,t)
=\widetilde{\mathbb P}_{\mu}\big[\exp\big( i\theta e^{-\alpha t}\|X_t\|+A(\theta_\mathrm s,\theta_\mathrm c,\theta_\mathrm l,r,t,\infty)\big)\big]\\
&\label{eq:M.5}\overset{\text{bounded convergence}}=\lim_{u\rightarrow \infty}\frac{1}{\mathbb P_{\mu}(D^c)}\mathbb P_{\mu}\big[\exp\big( i\theta e^{-\alpha t}\|X_t\|+A(\theta_\mathrm s,\theta_\mathrm c,\theta_\mathrm l,r,t,u)\big);D^c\big],
\end{align}
where for each $u\in [0,\infty]$,
\begin{align}
&A(\theta_\mathrm s,\theta_\mathrm c,\theta_\mathrm l,r,t,u)
\\&:=i\theta_\mathrm s \frac{X_{t+r}(f_\mathrm s)}{\|X_{t+r}\|^{1-\tilde{\beta}}} + i\theta_\mathrm c \frac{X_{t+r}(f_\mathrm c)}{\|(t+r)X_{t+r}\|^{1-\tilde{\beta}}} + i\theta_\mathrm l  \frac{ X_{t+r}(f_\mathrm l)- \mathbb P_\mu[\mathrm x_{t+r}(f_\mathrm l)|\mathscr F_u]}{\|X_{t+r}\|^{1-\tilde{\beta}}}
\\&\label{eq:M.6}\begin{multlined}
\overset{\eqref{eq:Hinfty}}=i\theta_\mathrm s \frac{X_{t+r}(f_\mathrm s)}{\|X_{t+r}\|^{1-\tilde{\beta}}} + \frac{i\theta_\mathrm c}{(t+r)^{1-\tilde{\beta}}} \frac{X_{t+r}(f_\mathrm c)}{\|X_{t+r}\|^{1-\tilde{\beta}}} + {}
\\\quad  i\theta_\mathrm l  \frac{ X_{t+r}(f_\mathrm l) - \sum_{p\in \mathbb Z_+^d:\alpha\tilde{\beta}>|p|b} e^{(\alpha - |p|b)(t+r)} e^{-(\alpha - |p|b)u}X_u(\phi_p) }{\|X_{t+r}\|^{1-\tilde{\beta}}}.
\end{multlined}
\end{align}
Now for each $t>0$, we get
\begin{align}
& \lim_{r\to \infty}k(\theta,\theta_\mathrm s,\theta_\mathrm c,\theta_\mathrm l,r,t)
\\&\overset{\eqref{eq:M.5}}=\lim_{r\to \infty}\lim_{u\rightarrow \infty}\frac{1}{\mathbb P_{\mu}(D^c)}\mathbb P_{\mu}\big[\exp\{i\theta e^{-\alpha t}\|X_t\|\} \mathbf 1_{\|X_t\|>0} \mathbb P_{\mu}[\exp\{A(\theta_\mathrm s,\theta_\mathrm c,\theta_\mathrm l,r,t,u)\}\mathbf 1_{D^c}|\mathscr F_t]\big]
\\&\begin{multlined}
\overset{\text{\eqref{eq:M.6}, Markov property}}=\lim_{r\to \infty}\lim_{u\rightarrow\infty}\frac{1}{\mathbb P_{\mu}(D^c)} \mathbb P_{\mu}\Bigg[\exp\{ i\theta e^{-\alpha t}\|X_t\|\}\mathbf 1_{\|X_t\|>0} \times {}
\\ \mathbb P_{X_t}\bigg[\exp\bigg\{A\bigg(\theta_\mathrm s,\theta_\mathrm c\Big(\frac{r}{t+r}\Big)^{1-\tilde \beta},\theta_\mathrm l,r,0,u-t\bigg)\bigg\}\mathbf 1_{D^c}\bigg]\Bigg]
\end{multlined}
\\&\begin{multlined}\overset{\text{bounded convergence}}=\lim_{r\to \infty} \mathbb P_{\mu}\Bigg[\exp\{ i\theta e^{-\alpha t}\|X_t\|\} \mathbf 1_{\|X_t\|>0}\frac{\mathbb P_{X_t}(D^c)}{\mathbb P_{\mu}(D^c)} \times {}
\\\widetilde{\mathbb P}_{X_t}\bigg[\exp\bigg\{A\bigg(\theta_\mathrm s,\theta_\mathrm c\Big(\frac{r}{t+r}\Big)^{1-\tilde \beta},\theta_\mathrm l,r,0,\infty\bigg)\bigg\}\bigg]\Bigg].
\end{multlined}
\\& \overset{\text{Theorem \ref{thm: II}}}=  \mathbb P_{\mu}\Big[\exp\{ i\theta e^{-\alpha t}\|X_t\|\} \mathbf 1_{\|X_t\|>0}\frac{\mathbb P_{X_t}(D^c)}{\mathbb P_{\mu}(D^c)} \Big]\Big(\prod_{j=\mathrm s,\mathrm c}\exp\{m[\theta_j f_j]\}\Big)\exp\{m[-\theta_\mathrm l f_\mathrm l]\}
%\\&=\frac{1}{\mathbb P_{\mu}(D^c)}\mathbb P_{\mu}\Big[\exp\{ i\theta e^{-\alpha t}\|X_t\|\} \mathbf 1_{\|X_t\|>0}\mathbb P_{\mu}(D^c|\mathscr F_t) \Big]\Big(\prod_{j=\mathrm s,\mathrm c}\exp\{m[\theta_j f_j]\}\Big)\exp\{m[-\theta_\mathrm l f_\mathrm l]\}
%%
\\&=\widetilde{\mathbb P}_{\mu}[\exp\{i\theta e^{-\alpha t}\|X_t\|\}]
\Big(\prod_{j=\mathrm s,\mathrm c}\exp\{m[\theta_j f_j]\}\Big)\exp\{m[-\theta_\mathrm l f_\mathrm l]\}.
\qedhere
\end{align}
\end{proof}

\begin{thebibliography}{10}

\bibitem{AdamczakMilos2015CLT}
  R. Adamczak and P. Mi{\l}o\'{s}, \emph{C{LT} for {O}rnstein-{U}hlenbeck branching particle system},
  Electron. J. Probab. \textbf{20} (2015), no. 42, 35 pp.

\bibitem{Asmussen76Convergence}
  S. Asmussen, \emph{Convergence rates for branching processes},
  Ann. Probab.  \textbf{4} (1976), no. 1, 139--146.

\bibitem{AsmussenHering1983Branching}
  S. Asmussen and H. Hering, \emph{Branching processes},
  Progress in Probability and Statistics, 3. Birkh\"{a}user Boston, Inc., Boston, MA, 1983.

\bibitem{Athreya1969Limit}
  K. B. Athreya,
  \emph{Limit theorems for multitype continuous time {M}arkov branching processes. {I}. {T}he case of an eigenvector linear functional},
  Z. Wahrscheinlichkeitstheorie und Verw. Gebiete \textbf{12} (1969), 320--332.

\bibitem{Athreya1969LimitB}
  K. B. Athreya,
  \emph{Limit theorems for multitype continuous time {M}arkov branching processes. {II}. {T}he case of an arbitrary linear functional},
  Z. Wahrscheinlichkeitstheorie und Verw. Gebiete \textbf{13} (1969), 204--214.

\bibitem{Athreya1971Some}
  K. B. Athreya,
  \emph{Some refinements in the theory of supercritical multitype {M}arkov branching processes},
  Z. Wahrscheinlichkeitstheorie und Verw. Gebiete \textbf{20} (1971), 47--57.

\bibitem{Dudley2002}
  R. M. Dudley,
\emph{Real Analysis and Probability},
  Cambridge University Press, 2002.

\bibitem{Dynkin1993Superprocesses}
  E. B. Dynkin,
  \emph{Superprocesses and partial differential equations},
  Ann. Probab. \textbf{21} (1993), no. 3, 1185--1262.

\bibitem{Heyde1970A-rate}
C. C. Heyde,
 \emph{A rate of convergence result for the super-critical {G}alton-{W}atson process},
  J. Appl. Probability \textbf{7} (1970), 451--454.

\bibitem{Heyde1971Some}
  C. C. Heyde,
   \emph{Some central limit analogues for supercritical {G}alton-{W}atson processes},
  J. Appl. Probability \textbf{8} (1971), 52--59.

\bibitem{HeydeBrown1871An-invariance}
  C. C. Heyde and B. M. Brown,
  \emph{An invariance principle and some convergence rate results for branching processes},
 Z. Wahrscheinlichkeitstheorie und Verw. Gebiete, \textbf{20} (1971), 271--278.

\bibitem{HeydeLeslie1971Improved}
  C. C. Heyde and J. R. Leslie,
  \emph{Improved classical limit analogues for {G}alton-{W}atson processes with or without immigration},
  Bull. Austral. Math. Soc. \textbf{5} (1971), 145--155.

\bibitem{KestenStigum1966Additional}
  H. Kesten and B. P. Stigum,
  \emph{Additional limit theorems for indecomposable multidimensional {G}alton-{W}atson processes},
  Ann. Math. Statist. \textbf{37} (1966), 1463--1481.

\bibitem{KestenStigum1966A-limit}
  H. Kesten and B. P. Stigum,
  \emph{A limit theorem for multidimensional {G}alton-{W}atson processes},
  Ann. Math. Statist. \textbf{37} (1966), 1211--1223.

\bibitem{Kyprianou2014Fluctuations}
  A. E. Kyprianou,
  \emph{Fluctuations of {L}\'{e}vy processes with applications},
    Introductory lectures. Second edition. Universitext. Springer, Heidelberg, 2014.

\bibitem{Li2011Measure-valued}
  Z. Li,
  \emph{Measure-valued branching {M}arkov processes},
  Probability and its Applications (New York). Springer, Heidelberg, 2011.

\bibitem{MarksMilos2018CLT}
  R. Marks and P. Mi{\l}o{\'s},
  \emph{C{LT} for supercritical branching processes with heavy-tailed branching law},
  arXiv:1803.05491.

\bibitem{MetafunePallaraPriola2002Spectrum}
  G. Metafune, D. Pallara, and E. Priola,
  \emph{Spectrum of {O}rnstein-{U}hlenbeck operators in {$L^p$} spaces with respect to invariant  measures},
  J. Funct. Anal. \textbf{196} (2002), no. 1, 40--60.

\bibitem{Milos2012Spatial}
  P. Mi{\l}o{\'s},
  \emph{Spatial central limit theorem for supercritical superprocesses},
  J. Theoret. Probab. \textbf{31} (2018), no. 1, 1--40.

\bibitem{RenSongSunZhao2019Stable}
Y.-X. Ren, R. Song, Z. Sun and J. Zhao,
\emph{Stable central limit theorems for super Ornstein-Uhlenbeck processes},
Elect. J Probab., \textbf{24} (2019), no. 141, 1--42.

\bibitem{RenSongZhang2014Central}
  Y.-X. Ren, R. Song, and R. Zhang,
  \emph{Central limit theorems for super {O}rnstein-{U}hlenbeck processes},
  Acta Appl. Math. \textbf{130} (2014), 9--49.

\bibitem{RenSongZhang2014CentralB}
  Y.-X. Ren, R. Song, and R. Zhang,
  \emph{Central limit theorems for supercritical branching {M}arkov processes},
  J. Funct. Anal. \textbf{266} (2014), no. 3, 1716--1756.

\bibitem{RenSongZhang2015Central}
  Y.-X. Ren, R. Song, and R. Zhang,
  \emph{Central limit theorems for supercritical superprocesses},
  Stochastic Process. Appl. \textbf{125} (2015), no. 2, 428--457.

\bibitem{RenSongZhang2017Central}
  Y.-X. Ren, R. Song, and R. Zhang,
  \emph{Central limit theorems for supercritical branching nonsymmetric {M}arkov processes},
  Ann. Probab. \textbf{45} (2017), no. 1, 564--623.
  
\end{thebibliography}
\end{document}
